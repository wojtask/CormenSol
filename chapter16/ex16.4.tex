\subchapter{Teoretyczne podstawy strategii zachłannych}

\exercise %16.4-1
Rodzina $\mathcal{I}_k$ jest dziedziczna, bo każdy podzbiór zbioru o~mocy co najwyżej $k$ też jest mocy co najwyżej $k$.
Jeśli teraz wybierzemy $A$, $B\in\mathcal{I}_k$ takie, że $|A|<|B|$, to dla każdego $x\in B\setminus A$, $A\cup\{x\}\in\mathcal{I}_k$, bo $|A\cup\{x\}|\le|B|\le k$.
A~zatem $M=\langle S,\mathcal{I}_k\rangle$ spełnia własność wymiany, co kończy dowód, że $M$ jest matroidem.

\exercise %16.4-2
\note{Macierz może być określona nad dowolnym ciałem (w~szczególności\/ $\mathbb{R}$).}

\noindent Oczywiście zbiór $S$ jest skończony.
Dziedziczność rodziny $\mathcal{I}$ wynika z~faktu, że każdy podzbiór zbioru kolumn liniowo niezależnych także składa się z~kolumn liniowo niezależnych.

W~dowodzie własności wymiany wykorzystamy pojęcie rzędu macierzy, czyli mocy największego zbioru jej kolumn liniowo niezależnych (patrz podrozdział 28.1).
Przy założeniu, że $n\ge1$, rodzina $\mathcal{I}$ zawiera co najmniej 2 elementy -- zbiór pusty i~singleton złożony z~dowolnej kolumny macierzy $T$.
Wybierzmy dowolne $A$, $B\in\mathcal{I}$ takie, że $|A|<|B|$.
Kolumny w~zbiorze $A$ są niezależne, więc jeśli zbudujemy z~nich macierz $T_A$, to $\mathrm{rank}(T_A)=|A|$, i~podobnie dla macierzy $T_B$ złożonej z~kolumn ze zbioru $B$, $\mathrm{rank}(T_B)=|B|$.
Ponieważ $|A|<|B|$, to $\mathrm{rank}(T_A)<\mathrm{rank}(T_B)$.

Załóżmy teraz nie wprost, że każda kolumna w~$B$ jest liniową kombinacją kolumn w~$A$.
Oznacza to, że każdy wektor będący liniową kombinacją kolumn z~$B$ jest także liniową kombinacją kolumn z~$A$.
Zatem przestrzeń liniowa generowana przez wektory będące kolumnami macierzy $T_B$ stanowi podprzestrzeń przestrzeni liniowej generowanej przez kolumny macierzy $T_A$.
Z~własności podprzestrzeni, jej baza (czyli zbiór wektorów liniowo niezależnych generujący tę podprzestrzeń) ma moc nie większą niż baza obejmującej ją przestrzeni; zachodzi zatem $\mathrm{rank}(T_B)\le\mathrm{rank}(T_A)$, co jest sprzeczne z~poprzednią nierównością.
Uzyskana sprzeczność dowodzi istnienia kolumny $x\in B\setminus A$ liniowo niezależnej z~kolumnami z~$A$, dlatego $A\cup\{x\}\in\mathcal{I}$.

\exercise %16.4-3
Wybierzmy $B'\in\mathcal{I}'$ i~niech $A'\subseteq B'$.
Na mocy definicji rodziny $\mathcal{I}'$ mamy, że istnieje taki zbiór maksymalny $B\in\mathcal{I}$, że $B\subseteq S\setminus B'$.
Inkluzja $A'\subseteq B'$ implikuje inkluzję $S\setminus B'\subseteq S\setminus A'$, skąd $B\subseteq S\setminus A'$, a~to oznacza, że $A'\in\mathcal{I}'$.

Pozostaje do wykazania, że zachodzi własność wymiany dla $\langle S,\mathcal{I}'\rangle$.
Niech $A'$, $B'\in\mathcal{I}'$, gdzie $|A'|<|B'|$.
Z~założenia mamy, że istnieje taki maksymalny zbiór $A\in\mathcal{I}$, że $A\subseteq S\setminus A'$, a~także taki maksymalny zbiór $B\in\mathcal{I}$, że $B\subseteq S\setminus B'$.
Rozważmy zbiór $X=B'\setminus(A'\cup A)$ i~zbadajmy dwa przypadki.
W~pierwszym z~nich $X\ne\emptyset$, więc wybierzmy dowolny element $x\in X$.
Zbiory $A'\cup\{x\}$ i~$A$ są rozłączne, więc $A\subseteq S\setminus(A'\cup\{x\})$, a~to oznacza, że $A'\cup\{x\}\in\mathcal{I}'$.
W~przypadku, gdy $X=\emptyset$ zachodzi
\[
	A' \supseteq (A'\cap B')\cup(A'\cap B) \quad\text{i}\quad B' = (A'\cap B')\cup(A\cap B'),
\]
skąd
\[
	|A'| \ge |A'\cap B'|+|A'\cap B| \quad\text{i}\quad |B'| = |A'\cap B'|+|A\cap B'|.
\]
Z~założenia, że $|A'|<|B'|$ mamy $|A'\cap B'|+|A'\cap B|<|A'\cap B'|+|A\cap B'|$, skąd $|A'\cap B|<|A\cap B'|$.

Skonstruujemy teraz pewien maksymalny zbiór niezależny $C$.
Początkowo przyjmijmy $C=B\setminus A'$.
Oczywiście $|C|\le|B|=|A|$ oraz $C\in\mathcal{I}$.
Jeśli $|C|<|A|$, to możemy skorzystać z~własności wymiany matroidu $\langle S,\mathcal{I}\rangle$, by stwierdzić, że istnieje $y\in A\setminus C$ takie, że $C\cup\{y\}\in\mathcal{I}$.
Dodajmy więc $y$ do $C$, a~następnie zastosujmy własność wymiany dla otrzymanego zbioru.
Proces ten powtarzamy, aż $C$ osiągnie moc równą $|A|$.
W~tym momencie $C\in\mathcal{I}$ oraz $|C|=|A|$, czyli $C$ jest maksymalnym zbiorem niezależnym w~$\langle S,\mathcal{I}\rangle$.
Na podstawie pokazanej wcześniej nierówności wnioskujemy, że $C$ nie zawiera wszystkich elementów zbioru $A\cap B'$, czyli że istnieje element $x\in(A\cap B')\setminus C$.
Wynika stąd, że $C\subseteq S\setminus(A'\cup\{x\})$, czyli $A'\cup\{x\}\in\mathcal{I}'$, co kończy dowód, że $\langle S,\mathcal{I}'\rangle$ jest matroidem.

\exercise %16.4-4
Niech $B\in\mathcal{I}$.
Z~definicji rodziny $\mathcal{I}$ mamy, że do zbioru $B$ należy po co najwyżej jednym elemencie z~każdego zbioru $S_1$, $S_2$, \dots, $S_k$.
Oczywiście własność ta cechuje też każdy zbiór $A\subseteq B$, toteż $A\in\mathcal{I}$.

Wybierzmy teraz $A,B\in\mathcal{I}$ takie, że $|A|<|B|$, oraz $x\in B\setminus A$.
Istnieje więc $1\le j\le k$, dla którego $x\in S_j$.
Wynika stąd, że $A\cap S_j=\emptyset$ oraz $|(A\cup\{x\})\cap S_j|=1$.
Ponadto dla każdego $i\ne j$, $(A\cup\{x\})\cap S_i=A\cap S_i$, dlatego $|(A\cup\{x\})\cap S_i|\le1$ na mocy przynależności zbioru $A$ do rodziny $\mathcal{I}$.
To kończy dowód, że para $\langle S,\mathcal{I}\rangle$ jest matroidem.

\exercise %16.4-5
Niech $M=\langle S,\mathcal{I}\rangle$ będzie matroidem ważonym z~funkcją wagi $w$ i~niech $W$ będzie liczbą większą niż waga któregokolwiek elementu zbioru $S$.
Dla każdego $x\in S$ zdefiniujmy nową dodatnią funkcję wagi $w'(x)=W-w(x)$.
Wówczas dla dowolnego $A\in\mathcal{I}$,
\[
	w'(A) = \sum_{x\in A}w'(x) = \sum_{x\in A}(W-w(x)) = W|A|-w(A).
\]
Niech $B\in\mathcal{I}$ będzie zbiorem maksymalnym o~największej wadze $w(B)$.
Jeśli istnieje inny zbiór maksymalny $B'\in\mathcal{I}$, to z~tw.\ 16.6 mamy, że $|B|=|B'|$.
Zachodzi wtedy
\[
	w'(B)-w'(B') = W|B|-w(B)-(W|B'|-w(B')) = w(B')-w(B) \le 0,
\]
a~zatem $w'(B)\le w'(B')$.
Otrzymujemy stąd, że każdy maksymalny zbiór matroidu $M$ maksymalizuje wagę $w$ wtedy i~tylko wtedy, gdy minimalizuje wagę $w'$.
