\subchapter{Kody Huffmana}

\exercise %16.3-1
Drzewo $T$, które nie jest regularne, zawiera co najmniej jeden węzeł $x$ o~stopniu 1.
Jeśli węzeł ten zostanie ,,wycięty'' z~$T$, to głębokość każdego potomka węzła $x$ zmniejszy się o~1 i~powstałe w~ten sposób drzewo będzie mieć koszt niższy od $T$.
Można ten koszt obniżać, pozbywając się w~ten sposób wszystkich węzłów stopnia 1 i~uzyskując drzewo regularne.

\exercise %16.3-2
Jeden z~optymalnych kodów Huffmana dla podanego zbioru liter i~ich częstości ilustruje rys.\ \ref{fig:16.3-2}.
\begin{figure}[!ht]
	\centering \begin{tikzpicture}[
	level/.append style = {level distance=10mm, sibling distance=10mm}
]

\newcommand\leafnode[1]{%
	node[tree node, rectangle, minimum height=4mm, minimum width=7mm, inner sep=0pt] {#1}
}

\node[outer] (pic a) {
\begin{tikzpicture}
	\matrix[array, nodes={draw=none, text depth=0pt, anchor=west}] (arr) {
		\texttt{a} \quad 1111111 \\
		\texttt{b} \quad 1111110 \\
		\texttt{c} \quad 111110 \\
		\texttt{d} \quad 11110 \\
		\texttt{e} \quad 1110 \\
		\texttt{f} \quad 110 \\
		\texttt{g} \quad 10 \\
		\texttt{h} \quad 0 \\
	};
\end{tikzpicture}
};

\node[outer, right=30mm of pic a] (pic b) {
\begin{tikzpicture}[anchor=center]
\node[tree node] (root) {54}
	child {\leafnode {\texttt{h}:21} edge from parent node[index node, left] {0}}
	child {node[tree node] {33}
		child {\leafnode {\texttt{g}:13} edge from parent node[index node, left] {0}}
		child {node[tree node] {20}
			child {\leafnode {\texttt{f}:8} edge from parent node[index node, left] {0}}
			child {node[tree node] {12}
				child {\leafnode {\texttt{e}:5} edge from parent node[index node, left] {0}}
				child {node[tree node] {7}
					child {\leafnode {\texttt{d}:3} edge from parent node[index node, left] {0}}
					child {node[tree node] {4}
						child {\leafnode {\texttt{c}:2} edge from parent node[index node, left] {0}}
						child {node[tree node] {2}
							child {\leafnode {\texttt{b}:1} edge from parent node[index node, left] {0}}
							child {\leafnode {\texttt{a}:1} edge from parent node[index node, right] {1}}
							edge from parent node[index node, right] {1}
						}
						edge from parent node[index node, right] {1}
					}
					edge from parent node[index node, right] {1}
				}
				edge from parent node[index node, right] {1}
			}
			edge from parent node[index node, right] {1}
		}
		edge from parent node[index node, right] {1}
	};
\end{tikzpicture}
};

\node[subpicture label, left=5mm of pic a] {(a)};
\node[subpicture label, left=5mm of pic b] {(b)};

\end{tikzpicture}

	\caption{{\sffamily\bfseries(a)} Optymalny kod Huffmana dla liter \texttt{a}, \texttt{b}, \dots, \texttt{h} o~częstościach będących początkowymi liczbami Fibonacciego.
{\sffamily\bfseries(b)} Drzewo odpowiadające optymalnemu temu kodowi uzyskane w~wyniku działania procedury \proc{Huffman}.} \label{fig:16.3-2}
\end{figure}

Zastanówmy się, jak może wyglądać optymalny kod Huffmana dla $n$ liter o~częstościach będących początkowymi liczbami Fibonacciego.
Przyjrzyjmy się wartości \attrib{z}{f} obliczanej w~algorytmie \proc{Huffman} w~linii 7.
Nietrudno zauważyć, że w~\singledash{$i$}{tej} iteracji pętli \kw{for} jest to suma najmniejszych $i+1$ częstości liter, czyli suma $i+1$ pierwszych liczb Fibonacciego:
\[
	\attrib{z}{f} = \sum_{k=1}^{i+1}F_k = F_{i+3}-1.
\]
Gdy $i>1$, to $F_{i+2}<\attrib{z}{f}<F_{i+3}$, dlatego wartość \attrib{z}{f} w~iteracji $i+1$ jest drugą najmniejszą w~kolejce $Q$ i~węzeł ten zostaje wybrany na prawego syna kolejnego węzła $z$.
W~pierwszej iteracji zachodzi $\attrib{z}{f}=2=F_3$, ale implementacja kolejki priorytetowej także wtedy doprowadzi do opisanego powiązania węzłów.
Zbudowane drzewo będzie mieć wysokość $n-1$, przy czym lewy syn każdego jego węzła wewnętrznego będzie liściem.
W~otrzymanym kodzie Huffmana, literze o~częstości $F_k$, gdzie $3\le k\le n$, przypisane zostanie słowo kodowe postaci
\[
	\underbrace{11\cdots1}_{n-k}0.
\]
Litery o~częstościach $F_1=F_2=1$ są nierozróżnialne z~punktu widzenia tego problemu, więc następujące dwa słowa kodowe:
\[
	\underbrace{11\cdots1}_{n-2}0 \quad\text{oraz}\quad \underbrace{11\cdots1}_{n-1}
\]
mogą być przypisane do tych liter w~dowolnej kolejności.

\exercise %16.3-3
\exercise %16.3-4
\exercise %16.3-5
\exercise %16.3-6
\exercise %16.3-7
\exercise %16.3-8
