\subchapter{Problem szeregowania zadań}

\exercise %16.5-1
Dla tak zmodyfikowanych kar algorytm zachłanny wybierze kolejno zadania $a_7$, $a_6$, $a_5$, $a_4$ i~$a_3$, a~odrzuci $a_2$ i~$a_1$.
W~otrzymanym optymalnym uszeregowaniu $\langle a_5,a_4,a_6,a_3,a_7,a_2,a_1\rangle$ suma kar wynosi $w_2+w_1=30$.

\exercise %16.5-2
Potraktujemy zbiór $A$ jak tablicę przechowującą terminy wykonania zadań z~tego zbioru, a~następnie wykonamy fragment procedury \proc{Counting-Sort} z~wierszy \doubledash{1}{7} z~parametrem $k=n$.
Tablica $C$ będzie wówczas przechowywać ciąg wartości $N_t(A)$.
Na mocy lematu 16.12 zbiór $A$ jest niezależny wtedy i~tylko wtedy, gdy dla każdego $t=1$, 2, \dots, $n$ zachodzi $C[t]\le t$.
Aby sprawdzić, czy nierówności te są spełnione, wystarczy przejrzeć tablicę $C$.
