\subchapter{Problem wyboru zajęć}
\note{W~procedurze \proc{Greedy-Activity-Selector} w~linii 1 zmienna\/ $n$ powinna przyjąć wartość\/ $\attrib{s}{length}-2$.
Wynika to z~tego, że tablica\/ $s$ składa się tak naprawdę z\/~$n+2$ elementów -- oprócz czasów rozpoczęcia \/$n$ rzeczywistych zajęć przechowuje także czasy fikcyjnych zajęć\/ $a_0$ i\/~$a_{n+1}$.}
\bignegskip

\exercise %16.1-1
W~poniższym algorytmie przyjmujemy, że zajęcia zostały uporządkowane niemalejąco według czasów zakończenia.
\begin{codebox}
\Procname{$\proc{Dynamic-Activity-Selector}(s,f)$}
\li	$n\gets\attrib{s}{length}-2$
\li	\For $l\gets2$ \To $n+2$ \label{li:dynamic-activity-selector-main-loop-begin}
\li		\Do \For $i\gets0$ \To $n-l+2$
\li				\Do $j\gets i+l-1$
\li					$c[i,j]\gets0$
\li					$A_{i,j}\gets\emptyset$
\li					\For $k\gets i+1$ \To $j-1$
\li						\Do $q\gets c[i,k]+c[k,j]+1$
\li							\If $f_i\le s_k<f_k\le s_j$ i~$q>c[i,j]$
\li								\Then $c[i,j]\gets q$
\li									$A_{i,j}\gets A_{i,k}\cup\{a_k\}\cup A_{k,j}$
								\End
						\End
				\End
		\End \label{li:dynamic-activity-selector-main-loop-end}
\li	\Return $A_{0,n+1}$
\end{codebox}
Algorytm oblicza wartości $c[i,j]$ oraz konstruuje zbiory $A_{i,j}$ dla $0\le i<j\le n+1$.
W~pierwszej iteracji pętli \kw{for} w~wierszach \doubledash{\ref{li:dynamic-activity-selector-main-loop-begin}}{\ref{li:dynamic-activity-selector-main-loop-end}} dla $i=0$, 1, \dots, $n$ za pomocą równania (16.3) wyznaczane są wartości $c[i,i+1]$, a~na podstawie wzoru (16.2) zbiory $A_{i,i+1}$, czyli rozwiązywane są podproblemy składające się z~dokładnie dwóch zajęć.
W~kolejnych iteracjach wyznaczane są rozwiązania podproblemów z~coraz większą ilością zajęć.
Zwracanym wynikiem jest najliczniejszy zbiór zajęć stanowiący rozwiązanie podproblemu $S_{0,n+1}=S$.

Struktura pętli zaprezentowanego algorytmu przypomina tę z~procedury \proc{Matrix-Chain-Order} z~podrozdziału 15.2.
Z~tego powodu analizę efektywności tej procedury można zaadaptować do obecnego algorytmu i~dojść do oszacowania $\Theta(n^3)$ na czas jego działania -- znacznie wyższego od złożoności czasowej rozwiązania zachłannego.

\exercise %16.1-2
Załóżmy, że zajęcia $a_1$, $a_2$, \dots, $a_n$, wraz z~dwoma fikcyjnymi $a_0$ i~$a_{n+1}$, uporządkowane są według czasów rozpoczęcia:
\[
	s_0 \le s_1 \le s_2 \le \dots \le s_n \le s_{n+1}.
\]
Wówczas prawdziwe jest następujące twierdzenie analogiczne do tw.\ 16.1:

\bigskip
\noindent\textsf{\textbf{Twierdzenie 16.1\/$'$.}} \textit{Rozważmy niepusty podproblem\/ $S_{i,j}$ i~niech\/ $a_m$ będą zajęciami w\/~$S_{i,j}$ rozpoczynającymi się najpóźniej:
\[
	s_m = \max\bigl\{\,s_k:a_k\in S_{i,j}\,\bigr\}.
\]
Wtedy:
\begin{enumerate}
	\item Wśród najliczniejszych podzbiorów parami zgodnych zajęć z\/~$S_{i,j}$ istnieje taki, który zawiera\/ $a_m$.
	\item Podproblem\/ $S_{m,j}$ jest pusty, tak więc po wybraniu\/ $a_m$ jedynym niepustym podproblemem może być tylko\/ $S_{i,m}$.
\end{enumerate}
}

Twierdzenie to mówi nam, że w~optymalnym rozwiązaniu podproblemu $S_{i,j}$, po wyborze zajęcia $a_m\in S_{i,j}$ rozpoczynającego się najpóźniej, do rozwiązania pozostanie tylko jeden podproblem, podczas gdy drugi z~nich jest pusty.
O~wyborze zajęć o~najpóźniejszym czasie rozpoczęcia można więc myśleć jak o~wyborze ,,zachłannym'', gdyż pozostawia on jak najwięcej możliwości wybrania zgodnych zajęć, czyli -- podobnie jak w~oryginalnej strategii -- maksymalizuje on ilość czasu do zagospodarowania.
Algorytm realizujący strategię wyboru zajęć o~najpóźniejszym starcie może rozwiązywać każdy podproblem zstępująco przy pomocy rekurencji lub iteracyjnie, analogicznie do oryginalnych procedur zaprezentowanych w~Podręczniku.
Poniższy pseudokod jest implementacją wariantu iteracyjnego.
\begin{codebox}
\Procname{$\proc{Greedy-Activity-Selector}'(s,f)$}
\li	$n\gets\attrib{s}{length}-2$
\li	$A\gets\{a_n\}$
\li	$i\gets n$
\li	\For $m\gets n-1$ \Downto 1
\li		\Do \If $f_m\le s_i$
\li				\Then $A\gets A\cup\{a_m\}$
\li					$i\gets m$
				\End
		\End
\li	\Return $A$
\end{codebox}

\exercise %16.1-3
Założymy, że liczba dostępnych sal wykładowych jest odpowiednio duża, aby pomieścić wszystkie zajęcia.
Wykorzystamy pomysł z~rozwiązania \refExercise{14.3-7}.
Zdarzenia polegające na rozpoczęciu lub zakończeniu zajęć będziemy przeglądać od najwcześniejszego do najpóźniejszego, przy czym w~danej chwili zakończenia będą rozważane przed rozpoczęciami (co wynika z~prawostronnej otwartości przedziałów czasowych zajmowanych przez zajęcia).
Równocześnie będziemy utrzymywać stos $F$ złożony z~wolnych sal wykładowych oraz zbiór $B$ sal zajmowanych w~danej chwili.
Początkowo zbiór $B$ jest pusty.
Jeśli kolejnym napotkanym zdarzeniem jest rozpoczęcie pewnych zajęć $a_i$, to przypiszemy je do nowej wolnej sali poprzez pobranie jej ze stosu $F$ i~umieszczenie w~zbiorze $B$ wraz z~informacją, że jest ona zajmowana przez zajęcia $a_i$.
Gdy z~kolei napotkamy zakończenie pewnych zajęć $a_j$, to w~zbiorze $B$ należy odszukać jedyną salę zajmowaną przez $a_j$, a~następnie umieścić ją na stosie $F$.

Wykorzystanie stosu do przechowywania wolnych sal sprawia, że użyjemy ich tylko tyle, ile potrzeba.
Załóżmy, że w~trakcie działania algorytmu największym rozmiarem zbioru $B$ jest $m$ i~niech $a_k$ będą pierwszymi zajęciami przypisanymi do \singledash{$m$}{tej} sali.
Powodem, dla którego zajęcia $a_k$ zajęły tę salę, jest to, że pozostałe $m-1$ sal było zajętych w~chwili $s_k$, co oznacza, że w~chwili tej odbywa się jednocześnie $m$ zajęć.
Stąd w~każdym przydziale zajęć do sal musi być użytych co najmniej $m$ sal, więc przydział zwracany przez nasz algorytm jest optymalny.

Czas działania algorytmu jest sumą czasu spędzonego na sortowaniu rozpoczęć i~zakończeń zajęć oraz czasu spędzonego na ich przeglądaniu i~konstruowaniu rozwiązania.
Do efektywnego zaimplementowania zbioru dynamicznego $B$ można zastosować np.\ drzewa czerwono-czarne.
Jesteśmy więc w~stanie podać implementację tego algorytmu działającą w~czasie $O(n\lg n)$, gdzie $n$ jest ilością zajęć na wejściu.

\exercise %16.1-4
\note{Oryginalna treść zadania wymaga też podania kontrprzykładu dla strategii polegającej na wyborze zajęć o~najwcześniejszym czasie rozpoczęcia spośród zajęć zgodnych z~dotychczas wybranymi.}

\noindent Oto zbiory zajęć stanowiących kontrprzykłady dla strategii wyboru zajęć, odpowiednio, o~najkrótszym czasie trwania oraz o~najwcześniejszym czasie rozpoczęcia:
\begin{center}
	\begin{tabular}{cccc}
		$i$ & 1 & 2 & 3 \\ \hline
		$s_i$ & 5 & 1 & 7 \\
		$f_i$ & 8 & 6 & 13
	\end{tabular}
	\hskip3cm
	\begin{tabular}{cccc}
		$i$ & 1 & 2 & 3 \\ \hline
		$s_i$ & 1 & 3 & 5 \\
		$f_i$ & 6 & 4 & 12
	\end{tabular}
\end{center}
W~obu sytuacjach zastosowanie danej strategii skutkuje uzyskaniem zbioru $\{a_1\}$, podczas gdy zbiór $\{a_2,a_3\}$ złożony z~parami zgodnych zajęć jest liczniejszy.

Dla strategii opartej na wyborze zajęcia kolidującego z~najmniejszą liczbą pozostałych zajęć do rozpatrzenia rozważmy następujący zbiór:
\begin{center}
	\begin{tabular}{cccccccccccc}
		$i$ & 1 & 2 & 3 & 4 & 5 & 6 & 7 & 8 & 9 & 10 & 11 \\ \hline
		$s_i$ & 1 & 2 & 2 & 2 & 3 & 4 & 5 & 6 & 6 & 6 & 7 \\
		$f_i$ & 3 & 4 & 4 & 4 & 5 & 6 & 7 & 8 & 8 & 8 & 9
	\end{tabular}
\end{center}
Istnieje tylko jeden zbiór czterech parami zgodnych zajęć, $\{a_1,a_5,a_7,a_{11}\}$, natomiast w~obranej strategii w~pierwszym kroku wybierane są zajęcia $a_6$, co prowadzi do rozwiązania mniej licznego.
