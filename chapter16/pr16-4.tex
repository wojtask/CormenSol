\problem{Inne metody szeregowania} %16-4

\subproblem %16-4(a)
Suma kar za niedotrzymanie terminu w~tym problemie zależy wyłącznie od zbioru $L$ zadań spóźnionych.
Wartość tę można próbować zmniejszyć poprzez usunięcie z~$L$ pewnego zadania, czyniąc go terminowym.
Alternatywnie można próbować wymienić pewne zadanie z~$L$ z~innym zadaniem terminowym.
Dzięki zachłannej naturze problemu, każde zmniejszenie kosztu rozwiązania jedną z~opisanych modyfikacji, zbliża nas do optymalnego zbioru $L_0$, dla którego suma kar za spóźnione zadania jest minimalna.

Rozważmy rozwiązanie zwrócone przez opisany algorytm.
Niech $a_j$ będzie dowolnym zadaniem w~zbiorze $L$ tego rozwiązania zajmującym przedział $k>d_j$.
Wynika stąd, że w~trakcie działania algorytmu $d_j$ zadań o~wyższych karach zdążyło już zająć przedziały 1, 2, \dots, $d_j$ i~wszystkie z~nich są zadaniami terminowymi.
Zamiana miejscami zadania $a_j$ z~innym takim zadaniem terminowym $a_i$ może tylko powiększyć koszt rozwiązania, bowiem $k>d_i$.
Gdyby zachodziło $k\le d_i$, to zadanie $a_i$, które jest przetwarzane przed $a_j$, byłoby umieszczone w~najpóźniejszym pustym przedziale przed terminem $d_i$, a~skoro przedział $k$ był pusty, zanim przetwarzane było zadanie $a_j$, to przedział ten zostałby wybrany dla $a_i$.
A~zatem nie sposób zmniejszyć kosztu tego rozwiązania, skąd wynika, że jest ono optymalne.

\subproblem %16-4(b)
\note{Rozwiązanie zostanie podane w~wersji 0.7.}
