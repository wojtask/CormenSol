\problem{Inne metody szeregowania} %16-4

\subproblem %16-4(a)
Rozważmy rozwiązanie zwrócone przez opisany algorytm.
Niech $a_j$ będzie dowolnym zadaniem spóźnionym zajmującym w~tym rozwiązaniu przedział $s_j>d_j$.
Wynika stąd, że w~trakcie działania algorytmu $d_j$ zadań o~nie mniejszych karach zdążyło już zająć przedziały 1, 2, \dots, $d_j$ i~wszystkie z~nich są zadaniami terminowymi.
Niech $a_i$ będzie jednym z~tych zadań, umieszczonym w~przedziale $s_i$.
Zamiana miejscami zadania $a_j$ z~$a_i$ może tylko powiększyć koszt rozwiązania, bowiem $s_j>d_i$.
Gdyby zachodziło $s_j\le d_i$, to w~momencie przyporządkowania $a_i$ do przedziału $s_i\le d_j$, wszystkie przedziały od $s_i+1$ do $d_i$ byłyby już zajęte i~zadanie $a_j$ musiałoby wówczas zająć przedział na prawo od $d_i$ -- sprzeczność.
A~zatem nie sposób zmniejszyć kosztu tego rozwiązania, skąd wynika, że jest ono optymalne.

\subproblem %16-4(b)
\note{Rozwiązanie zostanie podane w~wersji 0.7.}
