\subchapter{Podstawy strategii zachłannej}

\exercise %16.2-1
Pokażemy, że w~optymalnym plecaku będącym rozwiązaniem ciągłego problemu plecakowego musi wystąpić maksymalna ilość najbardziej wartościowego przedmiotu $a_m$, nieprzekraczająca pojemności plecaka $W$.
Załóżmy, że dysponujemy rozwiązaniem problemu, w~którym przedmiot $a_m$ występuje w~wadze $u<\min(w_m,W)$ kilogramów.
Wyrzućmy teraz z~plecaka łącznie $d=\min(w_m,W)-u$ kilogramów innych przedmiotów i~na to miejsce dobierzmy $d$ kilogramów przedmiotu $a_m$.
Ponieważ wartości jednostkowe usuniętych przedmiotów nie przekraczają wartości na jednostkę masy przedmiotu $a_m$, to ubytek wartości plecaka spowodowany ich usunięciem jest niewiększy niż $dv_m$.
Jednak przyrost wartości po dodaniu $d$ kilogramów przedmiotu $a_m$ jest równy $dv_m$, więc uzyskujemy plecak niegorszy od początkowego.

\exercise %16.2-2
Niech $a_1$, $a_2$, \dots, $a_n$ będą przedmiotami, które złodziej będzie wkładał do plecaka.
Przedmioty te ważą, odpowiednio $w_1$, $w_2$, \dots, $w_n$ i~są warte, odpowiednio, $v_1$, $v_2$, \dots, $v_n$.
Niech $S_{i,j}$, dla $0\le i\le n$, $0\le j\le W$, będzie podproblemem polegającym na wybraniu do plecaka przedmiotów spośród $a_1$, \dots, $a_i$ o~sumarycznej wadze nieprzekraczającej $j$ i~możliwie największej sumarycznej wartości.
Podczas rozwiązywania podproblemu $S_{i,j}$, gdzie $i\ge1$, złodziej musi zdecydować, czy włożyć do plecaka przedmiot $a_i$ (o~ile $w_i\le j$).
Jeśli tak, to do sumarycznej wartości konstruowanego plecaka należy dodać wartość $v_i$ wprowadzaną przez ten przedmiot, a~następnie rozwiązać podproblem $S_{i-1,j-w_i}$.
W~przeciwnym przypadku podproblem, jaki zostaje do rozpatrzenia to $S_{i-1,j}$, a~sumaryczna wartość nie zostaje powiększona.
Zdefiniujmy $K[i,j]$ jako największą sumaryczną wartość przedmiotów wchodzących w~skład rozwiązania podproblemu $S_{i,j}$.
Z~naszego rozumowania wynika zależność
\[
	K[i,j] = \begin{cases}
		0, & \text{jeśli $i=0$}, \\
		K[i-1,j], & \text{jeśli $i\ge1$, $w_i>j$}, \\
		\max(K[i-1,j],K[i-1,j-w_i]+v_i), & \text{jeśli $i\ge1$, $w_i\le j$}.
	\end{cases}
\]

Następujący algorytm oparty na programowaniu dynamicznym wylicza kolejne wartości w~tablicy $K$:
\begin{codebox}
\Procname{$\proc{0-1-Knapsack}(w,v,W)$}
\li	$n\gets\attrib{w}{length}$
\li	\For $j\gets0$ \To $W$
\li		\Do $K[0,j]\gets0$
		\End
\li	\For $i\gets1$ \To $n$
\li		\Do \For $j\gets0$ \To $W$
\li				\Do $K[i,j]\gets K[i-1,j]$
\li					\If $w_i\le j$ i~$K[i-1,j-w_i]+v_i>K[i,j]$
\li						\Then $K[i,j]\gets K[i-1,j-w_i]+v_i$
						\End
				\End
		\End
\li	\Return $K$
\end{codebox}
Nietrudno przekonać się, że wypełnienie całej tablicy $K$ wymaga czasu $\Theta(nW)$ i~pamięci tego samego rzędu.

Do wypisania optymalnego zbioru przedmiotów, jakie należy umieścić w~plecaku, służy poniższa procedura.
W~trakcie przeglądania tablicy $K$, w~zależności od podjętej decyzji w~algorytmie \proc{Knapsack}, wypisywany jest odpowiedni przedmiot.
\begin{codebox}
\Procname{$\proc{Print-Knapsack}(K,w,i,j)$}
\li	\If $i\ge1$
\li		\Then \If $K[i,j]=K[i-1,j]$
\li				\Then $\proc{Print-Knapsack}(K,w,i-1,j)$
\li				\Else $\proc{Print-Knapsack}(K,w,i-1,j-w_i)$
\li					wypisz $a_i$
				\End
		\End
\end{codebox}
W~celu wypisania rozwiązania całego problemu, czyli $S_{n,W}$, procedura powinna zostać wywołana jako $\proc{Print-Knapsack}(K,w,n,W)$, co zajmuje czas $\Theta(n)$.

\exercise %16.2-3
Załóżmy, że przedmioty $a_1$, $a_2$, \dots, $a_n$ posortowane są według niemalejących wag, czyli $w_1\le w_2\le\dots\le w_n$.
Z~założenia mamy też, że wartości tych przedmiotów tworzą ciąg nierosnący: $v_1\ge v_2\ge\dots\ge v_n$.
Przez $A_W$ oznaczmy optymalny podzbiór przedmiotów będący rozwiązaniem dyskretnego problemu plecakowego, gdzie $W$ jest pojemnością plecaka.
Niech $1\le i<j\le n$.
Udowodnimy, że jeśli $a_j\in A_W$, to $a_i\in A_W$, co jest równoważne temu, że istnieje $0\le k\le n$, że $A_W=\{a_1,a_2,\dots,a_k\}$.
Ustalmy $i$, $j$ i~załóżmy, że $a_j\in A_W$, ale $a_i\not\in A_W$.
Rozważmy zbiór $A_W'=A_W\setminus\{a_j\}\cup\{a_i\}$.
Ponieważ $w_i\le w_j$, to elementy $A_W'$ mieszczą się w~plecaku o~pojemności $W$, toteż $A_W'$ może stanowić rozwiązanie problemu.
Jednak suma wartości przedmiotów wchodzących w~skład $A_W'$ jest większa od sumarycznej wartości przedmiotów z~$A_W$ o~$v_i-v_j\ge0$.
Oznacza to, że $A_W'$ jest rozwiązaniem rozważanego problemu plecakowego o~niemniejszej wartości, co stoi w~sprzeczności z~definicją $A_W$.

Udowodniona obserwacja pozwala nam na skonstruowanie rozwiązania za pomocą algorytmu zachłannego.
Wystarczy przeglądać przedmioty od najlżejszych do najcięższych (czyli równoważnie, od najbardziej do najmniej wartościowych) i~włączać aktualny przedmiot do rozwiązania, o~ile tylko wraz z~poprzednio wybranymi przedmiotami mieści się w~plecaku.

\exercise %16.2-4
Załóżmy, że na trasie znajduje się $m$ stacji benzynowych i~że żadne kolejne stacje nie są oddalone od siebie o~więcej niż $n$ kilometrów.
Ponumerujmy je kolejnymi liczbami naturalnymi od 1 do $m$ wzdłuż trasy pokonywanej przez profesora.
Pokażemy, że problem ma własność optymalnej podstruktury.
Rozważmy optymalne rozwiązanie zawierające $s$ stacji z~pierwszym przystankiem znajdującym się na stacji o~numerze $k$.
Wówczas rozwiązanie dla podproblemu z~pozostałymi $m-k$ stacjami musi być optymalne; w~przeciwnym wypadku bowiem moglibyśmy zastąpić je lepszym, tzn.\ takim, w~którym występuje mniej niż $s-1$ stacji, uzyskując tym samym dla głównego problemu rozwiązanie o~mniej niż $s$ stacjach.

Problem cechuje także własność wyboru zachłannego.
Załóżmy, że na trasie w~odległości co najwyżej $n$ kilometrów od startu znajduje się dokładnie $k$ stacji benzynowych.
Profesor nie może wybrać na pierwszy przystanek stacji o~numerze większym niż $k$, gdyż nie jest możliwe dojechanie do niej bez tankowania po drodze.
W~optymalnym rozwiązaniu nie opłaca się też wybierać jako pierwszej stacji o~numerze $j<k$.
Jest tak dlatego, że gdyby wybrana została stacja $j$, to później, w~momencie przejeżdżania obok stacji $k$, w~baku pozostałoby mniej paliwa niż w~momencie opuszczania stacji $k$, gdy pierwsze tankowanie jest właśnie na niej.
Innymi słowy, wybór stacji $k$ pozwala na dojechanie dalej, niż gdyby wybór padł na stację $j$.

Korzystając z~tych własności, możemy podać prosty algorytm zachłanny wyznaczający optymalny ciąg tankowań w~czasie $O(m)$.

\exercise %16.2-5
\note{Przyjmiemy, że punkty ze zbioru wejściowego znajdują się na prostej rzeczywistej, według oryginalnej treści zadania.}

\noindent Punkty wejściowe utożsamimy z~ich współrzędnymi na prostej rzeczywistej.
Rozważmy podproblem $S_i$ tego problemu, w~którym zbiorem punktów wejściowych jest $X_i=\{x_i,x_{i+1},\dots,x_n\}$, przy czym $x_i$ jest punktem położonym na prostej rzeczywistej najbardziej na lewo spośród punktów z~$X_i$.
Udowodnimy, że problem wykazuje własność optymalnej podstruktury.
Niech $C_i$ będzie rozwiązaniem podproblemu $S_i$, czyli najmniejszym zbiorem przedziałów domkniętych o~długości jednostkowej, których suma zawiera wszystkie punkty z~$X_i$.
Rozważmy jeden z~przedziałów $I\in C_i$ zawierający punkt $x_i$.
Jeśli przedział ten pokrywa wszystkie punkty z~$X_i$, to $C_i=\{I\}$.
W~przeciwnym przypadku niech $x_j$ będzie punktem w~$X_i\setminus I$ o~najmniejszej współrzędnej rzeczywistej.
Wówczas $C_i=\{I\}\cup C_j$.
Oczywiście zbiór $C_j$ jest optymalnym rozwiązaniem podproblemu $S_j$.
Gdyby było inaczej, moglibyśmy zastąpić go mniejszym, uzyskując dla $S_i$ rozwiązanie o~mniejszej liczności niż $C_i$, co stoi w~sprzeczności z~definicją zbioru $C_i$.

Problem ma też własność wyboru zachłannego.
Załóżmy, że dysponujemy rozwiązaniem podproblemu $S_i$ zawierającym taki przedział $I=[a,a+1]$, że $a<x_i\le a+1$.
Usuńmy teraz $I$ z~rozwiązania i~zastąpmy go przedziałem $I'=[x_i,x_i+1]$.
Podobnie jak $I$, przedział $I'$ też zawiera punkt $x_i$, a~ponadto może potencjalnie pokryć więcej punktów wejściowych niż $I$, bo jego prawy koniec $x_i+1$ leży na prawo od prawego końca $a+1$ przedziału $I$.
Nowe rozwiązanie jest więc niegorsze od początkowego, zawierającego $I$.

Algorytm rozwiązujący problem będzie sortował punkty wejściowe, a~następnie przeglądał je od lewej do prawej.
W~każdym kroku pętli znajdowany będzie najbardziej na lewo położony punkt $x$, który nie został jeszcze pokryty żadnym przedziałem.
Następnie do rozwiązania dodawany będzie przedział o~lewym końcu pokrywającym się z~punktem $x$.
Po wyczerpaniu wszystkich punktów zwrócony zostanie skonstruowany zbiór przedziałów.
Czas działania algorytmu zdominowany jest przez czas spędzony na sortowaniu punktów.

\exercise %16.2-6
Załóżmy, że przedmioty $a_1$, $a_2$, \dots, $a_n$ zostały posortowane według nierosnących wartości na jednostkę masy, tzn.\ zachodzi $v_1/w_1\ge v_2/w_2\ge\dots\ge v_n/w_n$.
W~optymalnym plecaku stanowiącym rozwiązanie ciągłego problemu plecakowego istnieje takie $1\le k\le n$, że przedmioty $a_{k'}$ dla $1\le k'<k$ znajdują się w~plecaku w~całości, a~przedmiotów $a_{k''}$ dla $k<k''\le n$ nie ma w~plecaku w~ogóle.
A~zatem problem ten sprowadza się do znalezienia odpowiedniego $k$ oraz udziału przedmiotu $a_k$ w~plecaku.

Pokażemy, jak to zrobić w~czasie $O(n)$ w~sytuacji, gdy nie znamy uporządkowania przedmiotów według wartości jednostkowej.
Wykorzystując algorytm \proc{Select} z~podrozdziału 9.3, wyznaczmy medianę $m$ zbioru ilorazów $v_i/w_i$ i~zdefiniujmy zbiory
\[
	G = \{\,a_i:v_i/w_i>m\,\}, \quad E = \{\,a_i:v_i/w_i=m\,\}, \quad L = \{\,a_i:v_i/w_i<m\,\}
\]
oraz sumy
\[
	w_G = \sum_{a_i\in G}w_i, \quad w_E = \sum_{a_i\in E}w_i, \quad w_L = \sum_{a_i\in L}w_i.
\]
Jeśli $w_G>W$, to w~plecaku nie możemy umieścić w~całości wszystkich przedmiotów ze zbioru $G$, więc wywołujemy nasz algorytm rekurencyjnie dla tych przedmiotów i~zwracamy wynik tego wywołania jako wynik oryginalnego problemu.
W~przeciwnym przypadku $w_G\le W$, więc zbiór $G$ można włączyć do rozwiązania, a~następnie dobrać maksymalną ilość przedmiotów z~$E$ (pamiętajmy, że możemy brać części ułamkowe), która zmieści się w~pozostałej pojemności $W-w_G$.
Jeśli w~trakcie dobierania przedmiotów z~$E$ w~plecaku skończyło się miejsce, czyli gdy spełniona jest nierówność $w_G+w_E\ge W$, to można zakończyć algorytm.
W~przeciwnym wypadku wystarczy wywołać go rekurencyjnie dla przedmiotów z~$L$ oraz plecaka o~pojemności $W-w_G-w_E$ i~dołączyć jego wynik do konstruowanego rozwiązania.

Na każdym poziomie rekursji dla zbioru przedmiotów o~rozmiarze $n$ wykonywanych jest $O(n)$ operacji plus co najwyżej jedno wywołanie rekurencyjne.
Wybór mediany na ,,element rozdzielający'' jest optymalne w~tym sensie, że kolejne wywołanie rekurencyjne dostaje na wejściu zbiór przedmiotów o~rozmiarze nieprzekraczającym połowy rozmiaru bieżącego zbioru przedmiotów.
Otrzymujemy stąd zależność opisującą czas wymagany przez algorytm:
\[
	T(n) \le T(n/2)+O(n),
\]
której rozwiązaniem jest $T(n)=O(n)$.

\exercise %16.2-7
Posortujmy zbiór $A$ i~zbiór $B$ niemalejąco.
Niech $1\le i<j\le n$.
Wówczas $a_i\le a_j$, $b_i\le b_j$ oraz
\[
	\frac{a_i^{b_i}a_j^{b_j}}{a_i^{b_j}a_j^{b_i}} = \frac{a_j^{b_j-b_i}}{a_i^{b_j-b_i}} = \biggl(\frac{a_j}{a_i}\biggr)^{b_j-b_i} \ge 1.
\]
Ostatnia nierówność zachodzi, ponieważ $a_j/a_i\ge1$ i~$b_j-b_i\ge0$.
Wynika z~tego, że w~iloczynie będącym zyskiem z~uporządkowania bardziej opłacalne są czynniki postaci $a^b$, w~których podstawa $a$ jest tą samą statystyką pozycyjną w~zbiorze $A$, co wykładnik $b$ w~zbiorze $B$.
Przykładem uporządkowania maksymalizującego zysk jest więc porządek niemalejący obu zbiorów lub porządek nierosnący obu zbiorów.
Algorytm rozwiązujący ten problem może zatem jedynie sortować oba zbiory, co przy efektywnej implementacji zajmuje czas $\Theta(n\lg n)$.
