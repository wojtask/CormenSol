\problem{Podgrafy acykliczne} %16-3

\subproblem %16-3(a)
Twierdzenie to pokrywa się z~tw.\ 16.5 udowodnionym w~Podręczniku.

\subproblem %16-3(b)
\note{Kluczowe jest doprecyzowanie, że\/ $M$ jest macierzą określoną nad ciałem\/ $\mathbb{Z}_2$ liczb całkowitych modulo\/ $2$.}

\noindent Niech $A$ będzie zbiorem liniowo niezależnych kolumn macierzy $M$.
Załóżmy, że odpowiadający mu zbiór krawędzi $E_A\subseteq E$ zawiera cykl $\langle v_0,v_1,v_2,\dots,v_{k-1},v_0\rangle$ o~długości $k>1$, tzn.\ $\{v_i,v_{(i+1)\bmod k}\}\in E_A$ dla $i=0$, 1, \dots, $k-1$.
A~zatem dla każdego $i=0$, 1, \dots, $k-1$, $M_{v_i,e}=1$ wtedy i~tylko wtedy, gdy $e=\bigl\{v_{(i-1)\bmod k},v_i\bigr\}$ lub $e=\bigl\{v_i,v_{(i+1)\bmod k}\bigr\}$.
Stąd suma kolumn ze zbioru $A$ stanowi wektor składający się z~dwójek na każdej pozycji, czyli wektor zerowy w~$\mathbb{Z}_2$.
Kolumny te są więc liniowo zależne, co stoi w~sprzeczności z~założeniem.

Dla dowodu w~drugą stronę przyjmijmy, że $E_A\subseteq E$ jest zbiorem $k>1$ krawędzi, które nie tworzą cyklu i~niech $A=\{a_1,a_2,\dots,a_k\}$ będzie podzbiorem kolumn macierzy $M$ odpowiadających zbiorowi $E_A$.
Gdyby zbiór $A$ zawierał kolumny liniowo zależne, to istniałyby takie współczynniki $c_1$, $c_2$, \dots, $c_k\in\mathbb{Z}_2$, nie wszystkie równe 0, że $\sum_{i=1}^kc_ia_i$ byłoby wektorem zerowym w~$\mathbb{Z}_2$.
Ponieważ działania wykonywane są w~arytmetyce modulo 2, to równoważnie istniałby niepusty podzbiór kolumn $A'\subseteq A$, których suma byłaby wektorem złożonym z~samych liczb parzystych.
Oznacza to, że w~grafie $G_{A'}=\langle V,A'\rangle$ każdy wierzchołek byłby parzystego stopnia.
Każda składowa grafu $G_{A'}$ byłaby więc tzw.\ \textbf{grafem eulerowskim}, czyli miała cykl Eulera, o~czym mówi problem \refProblem{22-3} zmodyfikowany do wersji nieskierowanych grafów.
Otrzymana sprzeczność dowodzi, że zbiór $A$ nie może składać się z~kolumn liniowo zależnych.

Niech $S$ będzie zbiorem kolumn macierzy $M$, a~$\mathcal{I}'$ rodziną podzbiorów zbioru $S$ taką, że $A\in\mathcal{I}'$ wtedy i~tylko wtedy, gdy kolumny należące do $A$ są liniowo niezależne.
Każdej kolumnie macierzy $M$ odpowiada inna krawędź grafu $G=\langle V,E\rangle$.
Jeśli przez $E_A$ oznaczymy zbiór krawędzi grafu $G$ odpowiadających kolumnom z~$A$, to istnieje jednoznaczne odwzorowanie $\mathcal{I}'$ na rodzinę $\mathcal{I}=\{\,E_A\subseteq E:A\in\mathcal{I}'\,\}$.
W~\refExercise{16.4-2} pokazaliśmy, że $\langle S,\mathcal{I}'\rangle$ jest matroidem, a~zatem $\langle E,\mathcal{I}\rangle$ także jest matroidem.

\subproblem %16-3(c)
Dla grafu $G=\langle V,E\rangle$ rozważmy matroid ważony $M=\langle E,\mathcal{I}\rangle$ zdefiniowany jak w~punkcie (a) wraz z~funkcją wagi $w$.
Do znalezienia acyklicznego podzbioru krawędzi grafu $G$ o~największej wadze można użyć algorytmu opartego na \proc{Greedy}, w~którym używana jest efektywna metoda weryfikowania, czy dany zbiór krawędzi $A$ nie zawiera cyklu, czyli czy $A\in\mathcal{I}$.
Wymaganie to spełnia algorytm Kruskala z~podrozdziału 23.2, z~tą różnicą, że ma on na celu minimalizację sumarycznej wagi krawędzi.
Aby wartość ta była maksymalizowana, wystarczy zastosować w~tym algorytmie funkcję wagową zmodyfikowaną metodą z~\refExercise{16.4-5}.

\subproblem %16-3(d)
Na rys.\ \ref{fig:16-3c} zamieszczono przykładowy graf skierowany $G=\langle V,E\rangle$, dla którego zdefiniowana w~treści problemu para $\langle E,\mathcal{I}\rangle$ nie jest matroidem.
Rozważmy zbiory $A=\{e_1,e_4\}$, $B=\{e_1,e_2,e_3\}$.
Oczywiście $A$, $B\in\mathcal{I}$, ale zarówno w~$A\cup\{e_2\}$, jak i~w~$A\cup\{e_3\}$ znajduje się cykl, czyli zbiory te nie należą do $\mathcal{I}$.
Para $\langle E,\mathcal{I}\rangle$ nie spełnia więc własności wymiany, przez co nie stanowi matroidu.
\begin{figure}[!ht]
	\centering \begin{tikzpicture}

\node[tree node] (a) {};
\node[tree node, above right=10mm and 7mm of a] (b) {};
\node[tree node, right=16mm of a] (c) {};
\path (a) edge[arrow] node[index node, auto] {$e_1$} (b) edge[arrow, bend left=25] node[index node, auto] {$e_3$} (c);
\path (b) edge[arrow] node[index node, auto] {$e_2$} (c);
\path (c) edge[arrow, bend left=25] node[index node, auto] {$e_4$} (a);

\end{tikzpicture}

	\caption{Przykładowy graf skierowany, dla którego struktura $\langle E,\mathcal{I}\rangle$ nie jest matroidem.} \label{fig:16-3c}
\end{figure}

\subproblem %16-3(e)
Jeśli pewien podgraf $G'=\langle V',E'\rangle$ grafu $G$ jest cyklem, to sam w~sobie stanowi on cykl Eulera.
Na podstawie punktu (a) problemu \refProblem{22-3} stopień wejściowy każdego wierzchołka $v$ w~grafie $G'$ jest równy stopniowi wyjściowemu $v$.
Oznacza to, że dla każdego wiersza macierzy $M$ odpowiadającego wierzchołkowi z~$V'$ wartości w~kolumnach odpowiadającym krawędziom z~$E'$ sumują się do zera.
Zatem dla każdego grafu skierowanego $G$ zawierającego cykl istnieje podmacierz macierzy $M$ o~wszystkich wierszach sumujących się do zera.
Płynie z~tego wniosek, że kolumny macierzy $M$ są liniowo zależne lub, równoważnie, jeśli kolumny macierzy $M$ są liniowo niezależne, to odpowiadający im graf nie zawiera cyklu.

\subproblem %16-3(f)
W~punkcie (e) udowodniliśmy wynikanie tylko w~jedną stronę.
Implikacja przeciwna jest fałszywa; istnieją bowiem grafy skierowane niezawierające cyklu, których macierz incydencji zawiera kolumny liniowo zależne.
Jednym z~nich jest graf z~rys.\ \ref{fig:16-3c} pozbawiony krawędzi $e_4$.
Struktura zdefiniowana w~punkcie (d) dla takiego grafu jest matroidem.
Z~kolei w~macierzy incydencji tego grafu
\[
	\bordermatrix{~ & e_1 & e_2 & e_3 \cr
		v_1 & -1 & 0 & -1 \cr
		v_2 & 1 & -1 & 0 \cr
		v_3 & 0 & 1 & 1}
\]
ostatnia kolumna jest sumą pozostałych, przez co zbiór kolumn tej macierzy nie tworzy matroidu.
Powodem tego jest brak własności, którą pokazaliśmy w~punkcie (b) dla grafów nieskierowanych, czyli brak dokładnej odpowiedniości między niezależnym zbiorem krawędzi macierzy incydencji grafu skierowanego a~acyklicznym zbiorem krawędzi w~tym grafie.
