\problem{Wydawanie reszty} %16-1

\subproblem %16-1(a)
Optymalna reszta o~wartości $n$ składa się z~monety o~pewnym nominale $d\le n$ plus optymalnej reszty dla podproblemu wydawania reszty o~wartości $n-d$.
Ma tutaj zastosowanie rozumowanie ,,wytnij i~wklej'', co stanowi uzasadnienie właściwości optymalnej podstruktury problemu.

Pokażemy, że zachodzi też własność wyboru zachłannego polegająca na tym, że rozwiązanie optymalne dla reszty o~wartości $n$ zawiera monetę o~największym nominale $d$ nieprzekraczającym $n$.
Jeśli $n=1$, to jedynym rozwiązaniem jest jedna moneta 1~gr.
Przyjmijmy więc, że $n>1$ i~załóżmy, że dysponujemy pewnym rozwiązaniem dla reszty o~wartości $n$, w~którym nie ma monety o~nominale $d$.
Będziemy wprowadzać modyfikacje w~tym rozwiązaniu polegające na zamianach monet o~niskich nominałach na monety o~wyższych nominałach, zmniejszając tym samym liczbę monet w~rozwiązaniu, ale też utrzymując jego wartość.

Jeśli $n\ge2$, to każdą parę jednogroszówek wymieniamy na jedną dwugroszówkę.
Teraz rozwiązanie zawiera co najwyżej jedną niesparowaną jednogroszówkę.

Jeśli $n\ge5$, to rozważamy dwie sytuacje w~zależności od obecności monety 1~gr w~aktualnym rozwiązaniu.
Zamieniamy wówczas albo zestaw 2~gr, 2~gr, 1~gr na jedną pięciogroszówkę, albo zestaw trzech monet 2~gr na dwie inne monety -- pięciogroszówkę i~jednogroszówkę.
Proces powtarzamy, aż do wyczerpania takich zestawów.

Jeśli $n\ge10$, to rozwiązanie na pewno zawiera parę pięciogroszówek, więc każdą taką parę zamieniamy na dziesięciogroszówkę.

Jeśli $n\ge20$, to analogicznie do poprzedniego przypadku -- każda para dziesięciogroszówek zostaje zamieniona na jedną dwudziestogroszówkę.

Jeśli $n\ge50$, to podobnie jak przy dążeniu do uzyskania 5~gr -- w~zależności od tego, czy moneta 10~gr należy do aktualnego rozwiązania, to albo zestaw 20~gr, 20~gr, 10~gr zostaje zamieniony na monetę pięćdziesięciogroszową, albo zestaw trzech monet 20~gr -- na monety pięćdziesięciogroszową i~dziesięciogroszową.

Zachodzenie pokazanych własności pozwala na użycie algorytmu zachłannego.
W~algorytmie tym dla reszty o~wartości $n$ do rozwiązania dodawany będzie największy nominał $d$ nieprzekraczający $n$, po czym algorytm zostanie wywołany dla reszty o~wartości $n-d$, o~ile $n-d>0$.
Łatwo jednak przekształcić ten algorytm w~taki, który działa w~czasie $O(1)$.
Wystarczy zauważyć, że dany nominał będzie wybierany najdłużej jak to tylko możliwe i~do wydania reszty o~wartości $n$ użytych zostanie kolejno:
\begin{itemize}
	\item $\lfloor n/50\rfloor$ pięćdziesięciogroszówek, pozostawiając resztę o~wartości $n_{50}=n\bmod50$,
	\item $\lfloor n_{50}/20\rfloor$ dwudziestogroszówek, pozostawiając resztę o~wartości $n_{20}=n_{50}\bmod20$,
	\item $\lfloor n_{20}/10\rfloor$ dziesięciogroszówek, pozostawiając resztę o~wartości $n_{10}=n_{20}\bmod10$,
	\item $\lfloor n_{10}/5\rfloor$ pięciogroszówek, pozostawiając resztę o~wartości $n_5=n_{10}\bmod5$,
	\item $\lfloor n_5/2\rfloor$ dwugroszówek, pozostawiając resztę o~wartości $n_2=n_5\bmod2$,
	\item $n_2$ jednogroszówek.
\end{itemize}

\subproblem %16-1(b)
Dla $i=0$, 1, \dots, $k$, niech $a_i$ będzie liczbą monet o~nominale $c^i$ użytych w~rozwiązaniu problemu wydawania reszty o~wartości $n$.
Zauważmy, że w~rozwiązaniu optymalnym dla każdego $i=0$, 1, \dots, $k-1$ zachodzi $a_i\le c-1$.
W~przeciwnym przypadku bowiem, jeśli $a_i\ge c$ dla pewnego $0\le i<k$, to moglibyśmy usunąć $c$ monet o~nominale $c^i$ i~zastąpić je jedną monetą o~nominale $c^{i+1}$.
W~nowym rozwiązaniu mielibyśmy o~$c-1>0$ monet mniej, co przeczyłoby optymalności początkowego rozwiązania.

Pokażemy, że każde rozwiązanie, które nie zawiera największego nominału nieprzekraczającego $n$, jest nieoptymalne.
Zdefiniujmy $j=\max\{\,0\le i\le k:c^i\le n\,\}$ i~rozważmy dowolne rozwiązanie problemu dla $n$ bez nominału $c^j$.
Wówczas $\sum_{i=0}^{j-1}a_ic^i=n\ge c^j$.
Jeśli założymy, że takie rozwiązanie jest optymalne, to z~obserwacji z~poprzedniego paragrafu mamy, że $a_i\le c-1$ dla każdego $i=0$, 1, \dots, $j-1$.
Zachodzi
\[
	n = \sum_{i=0}^{j-1}a_ic^i \le \sum_{i=0}^{j-1}(c-1)c^i = (c-1)\sum_{i=0}^{j-1}c^i = (c-1)\cdot\frac{c^j-1}{c-1} = c^j-1 < c^j,
\]
co jest sprzeczne z~poprzednim wynikiem.
Oznacza to, że rozwiązanie generowane przez algorytm zachłanny jest optymalne.

\subproblem %16-1(c)
Jeśli do dyspozycji będą nominały 1~gr, 3~gr i~4~gr, to np.\ reszta o~wartości 6~gr przez opisany algorytm zachłanny zostanie wydana za pomocą 3 monet: $6\ \mathrm{gr}=4\ \mathrm{gr}+1\ \mathrm{gr}+1\ \mathrm{gr}$, podczas gdy wystarczą 2 monety: $6\ \mathrm{gr}=3\ \mathrm{gr}+3\ \mathrm{gr}$.

\subproblem %16-1(d)
Oznaczmy dysponowane nominały przez $d_1$, $d_2$, \dots, $d_k$, wśród których jednym z~nich jest 1~gr.
W~tablicy $c[0\twodots n]$ na pozycji $c[j]$ będziemy przechowywać minimalną liczbę monet konieczną do wydania reszty o~wartości $j$.
Jako przypadek brzegowy przyjmujemy $c[0]=0$.
Na podstawie optymalnej podstruktury mamy, że jeśli w~optymalnym rozwiązaniu dla $j>0$ użyta została moneta o~nominale $d_i$, to $c[j]=1+c[j-d_i]$.
Zachodzi więc zależność rekurencyjna:
\[
	c[j] = \begin{cases}
		0, & \text{jeśli $j=0$,} \\
		\displaystyle\min_{1\le i\le k}(1+c[j-d_i]), & \text{jeśli $j>1$,}
	\end{cases}
\]
przy czym dla uproszczenia zapisu przyjmujemy, że gdy $j<d_i$, to $c[j-d_i]=\infty$.
Wartości tablicy $c[0\twodots n]$ można wyliczać według rosnących indeksów.
Po zakończeniu wypełniania tablicy $c[n]$ będzie równe minimalnej liczbie monet w~optymalnym rozwiązaniu problemu.

Poniższa procedura realizuje opisany plan w~czasie $O(nk)$, zapisując dodatkowo w~tablicy $\id{denom}[1\twodots n]$ na pozycji $j$ nominał $d_i$ minimalizujący $c[j]$.
\begin{codebox}
\Procname{$\proc{Make-Change}(n,d)$}
\li	$c[0]\gets0$
\li	\For $j\gets1$ \To $n$
\li		\Do $c[j]\gets\infty$
\li			\For $i\gets1$ \To $k$
\li				\Do \If $j\ge d_i$ i~$1+c[j-d_i]<c[j]$
\li						\Then $c[j]\gets1+c[j-d_i]$
\li							$\id{denom}[j]\gets d_i$
						\End
				\End
		\End
\li	\Return $c$ i~\id{denom}
\end{codebox}
Dzięki tablicy \id{denom} nominały użyte w~optymalnym rozwiązaniu można wypisać w~czasie $O(n)$, stosując poniższą procedurę.
\begin{codebox}
\Procname{$\proc{Print-Change}(n,\id{denom})$}
\li	\While $n>0$
\li		\Then wypisz $\id{denom}[n]$
\li			$n\gets n-\id{denom}[n]$
		\End
\end{codebox}
