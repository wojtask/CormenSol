\subchapter{Teoretyczne podstawy strategii zachłannych}

\exercise %16.4-1
Rodzina $\mathcal{I}_k$ jest dziedziczna, bo każdy podzbiór zbioru o~mocy co najwyżej $k$ też jest mocy co najwyżej $k$.
Jeśli teraz wybierzemy $A$, $B\in\mathcal{I}_k$ takie, że $|A|<|B|$, to dla każdego $x\in B\setminus A$, $A\cup\{x\}\in\mathcal{I}_k$, bo $|A\cup\{x\}|\le|B|\le k$.
A~zatem $M=\langle S,\mathcal{I}_k\rangle$ spełnia własność wymiany, co kończy dowód, że $M$ jest matroidem.

\exercise %16.4-2
\exercise %16.4-3
\exercise %16.4-4
Niech $B\in\mathcal{I}$.
Z~definicji rodziny $\mathcal{I}$ mamy, że do zbioru $B$ należy po co najwyżej jednym elemencie z~każdego zbioru $S_1$, $S_2$, \dots, $S_k$.
Oczywiście każdy jego podzbiór $A$ też spełnia taką własność, dlatego $A\in\mathcal{I}$.

Wybierzmy teraz $A,B\in\mathcal{I}$ takie, że $|A|<|B|$, a~także $x\in B\setminus A$.
Istnieje więc $1\le j\le k$, dla którego $x\in S_j$.
Wynika stąd, że $A\cap S_j=\emptyset$ oraz $|(A\cup\{x\})\cap S_j|=1$.
Ponadto dla każdego $i\ne j$, $(A\cup\{x\})\cap S_i=A\cap S_i$, dlatego $|(A\cup\{x\})\cap S_i|\le1$ na mocy przynależności zbioru $A$ do rodziny $\mathcal{I}$.
To kończy dowód, że para $\langle S,\mathcal{I}\rangle$ jest matroidem.

\exercise %16.4-5
