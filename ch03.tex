\section*{Rozdział 3: Rzędy wielkości funkcji}

\subsection*{3.1. Notacja asymptotyczna}

\paragraph{3.1-1.}
Zbadajmy, czy dla pewnych stałych $c_1,c_2,n_0>0$, dla każdego $n\ge n_0$ prawdą jest
\[
  0\le c_1(f(n)+g(n))\le\max(f(n),g(n))\le c_2(f(n)+g(n)).
\]
Jeśli w pewnym zbiorze $S$ zachodzi $f(n)\le g(n)$, to $\max(f(n), g(n)) = g(n)$, więc
\[
  c_1f(n)+c_1g(n)\le c_1g(n)+c_1g(n) = 2c_1g(n) = 2c_1\max(f(n), g(n)).
\]
Wybierając dowolne $0<c_1\le 1/2$ i $n_0$ równe minimum zbioru $S$, spełniamy pierwsze dwie nierówności dla wszystkich $n\ge n_0$ należących do $S$. Identyczny rezultat otrzymujemy w przypadku przeciwnym, gdy $\max(f(n), g(n)) = f(n)$. Ostatnia z nierówności jest oczywiście spełniona jeśli przyjmiemy $c_2=1$.

Wszystkie stałe zostały wyznaczone, wnioskujemy zatem, że $\max(f(n), g(n))$ $= \Theta(f(n)+g(n))$ zgodnie z definicją notacji $\Theta$.

\paragraph{3.1-2.}
By pokazać, że $(n+a)^b=\Theta(n^b)$, należy znaleźć stałe $c_1,c_2,n_0>0$ takie, że $0\le c_1n^b\le (n+a)^b\le c_2n^b$ dla wszystkich $n\ge n_0$. Zauważmy, że $n+a\le n+|a|\le 2n$, gdy $|a|\le n$ oraz $n+a\ge n-|a|\ge n/2$, o ile $|a|\le n/2$. Stąd, jeśli $n\ge 2|a|$, to zachodzi

\[
  0\le n/2\le n+a\le 2n.
\]
Ponieważ $b>0$, to powyższa nierówność jest spełniona również wtedy, gdy podniesiemy wszystkie jej składowe do potęgi $b$:
\[
  0\le (n/2)^b\le (n+a)^b\le (2n)^b,
\]
\[
  0\le (1/2)^bn^b\le (n+a)^b\le 2^bn^b.
\]
Widać zatem, że szukanym stałym można nadać wartości $c_1=(1/2)^b$, $c_2=2^b$ oraz $n_0=2|a|$ i~prawdą jest, że $(n+a)^b=\Theta(n^b)$.

\paragraph{3.1-3.}
Niech $T(n)$ będzie czasem działania algorytmu. $T(n)\ge O(n^2)$ oznacza, że $T(n)\ge f(n)$ dla pewnej funkcji $f(n)$ z klasy $O(n^2)$. Stwierdzenie pozostaje prawdziwe dla każdego $T(n)$, wystarczy bowiem wybrać funkcję $f(n)=0$, która oczywiście jest w $O(n^2)$. Widać więc, że takie określenie nic nam nie mówi o~oszacowaniu czasu działania algorytmu.

\paragraph{3.1-4.}
Znajdźmy stałe $c,n_0>0$ takie, że $0\le 2^{n+1}\le c2^n$ dla każdego $n\ge n_0$. Ponieważ $2^{n+1}=2\cdot 2^n$ dla każdego $n\ge 0$, to można przyjąć $c=2$ oraz $n_0=1$. A zatem $2^{n+1} = O(2^n)$.

Wyznaczmy teraz te same stałe, ale spełniające zależność $0\le 2^{2n}\le c2^n$ dla wszystkich $n\ge n_0$. Mamy $2^{2n}=2^n\cdot 2^n\le c2^n$, z czego wynika, że $c\ge 2^n$, co jednak uzależnia $c$ od $n$, a zatem $c$ musiałoby być dowolnie duże i nie może w~takim wypadku być stałą. Stąd otrzymujemy, że $2^{2n}\ne O(2^n)$.

\paragraph{3.1-5.}
Z definicji notacji $\Theta$ mamy, że $f(n)=\Theta(g(n))$ wtedy i tylko wtedy, gdy istnieją takie stałe $c_1,c_2,n_0>0$, że
\[
  0\le c_1g(n)\le f(n)\le c_2g(n)
\]
zachodzi dla wszystkich $n\ge n_0$. Rozdzielając nierówność otrzymujemy
\[
  \left\{\begin{array}{r}
	0\le c_1g(n)\le f(n), \\
	0\le f(n)\le c_2g(n).
  \end{array}\right.
\]
Pierwsze wyrażenie stanowi definicję $f(n)=\Omega(g(n))$, a drugie -- $f(n)=O(g(n))$.

Dowód przeprowadzony w odwrotnej kolejności pozwala wykazać implikację w drugą stronę, gdyż koniunkcja dwóch składowych nierówności doprowadza do nierówności z definicji $f(n)=\Theta(g(n))$.

\paragraph{3.1-6.}
Jeżeli pesymistyczny czas działania algorytmu wynosi $O(g(n))$, to dla dowolnych danych wejściowych oszacowanie czasu działania tego algorytmu jest nie większe niż $c_1g(n)$ dla pewnej stałej $c_1>0$. Z kolei optymistyczny czas $\Omega(g(n))$ oznacza, że dla dowolnych danych wejściowych oszacowanie czasu działania algorytmu jest nie mniejsze niż $c_2g(n)$ dla stałej $c_2>0$. Widać zatem, że dla dowolnych danych mamy $0\le c_2g(n)\le f(n)\le c_1g(n)$, gdzie $f(n)$ jest czasem działania algorytmu, a stąd otrzymujemy, że $f(n) = \Theta(g(n))$.

Przeprowadzenie powyższego rozumowania w odwrotnej kolejności pozwala wykazać przeciwną implikację.

\paragraph{3.1-7.}
Załóżmy, że twierdzenie jest fałszywe i że istnieje pewna funkcja $f(n)$ należąca do $o(g(n))\cap\omega(g(n))$. Zachodzi zatem zarówno $f(n)=o(g(n))$ jak i~$f(n)=\omega(g(n))$, co oznacza, że dla każdych dodatnich stałych $c_1$ i $c_2$ istnieje pewne dodatnie $n_0$, że
\[
  c_1g(n)<f(n)<c_2g(n)
\]
dla wszystkich $n\ge n_0$. Dochodzimy do sprzeczności, bowiem nieprawdą jest, że $c_1<c_2$ dla każdych liczb $c_1$ i $c_2$, skąd wnioskujemy, że zbiór $o(g(n))\cap\omega(g(n))$ jest pusty.

Dzięki pustości tej klasy, nie ma potrzeby definiowania notacji $\theta$ odpowiadającej $\Theta$ i analogicznej do $o$ i~$\omega$.

\paragraph{3.1-8.}
\begin{eqnarray*}
  \Omega(g(n,m)) &=& \left\{f(n,m):\;\mbox{istnieją dodatnie stałe }c,n_0,m_0\mbox{ takie, że}\right. \\
  && \quad 0\le cg(n,m)\le f(n,m)\mbox{ dla wszystkich }n\ge n_0 \\
  && \left.\quad\mbox{oraz }m\ge m_0\right\} \\\\
  \Theta(g(n,m)) &=& \left\{f(n,m):\;\mbox{istnieją dodatnie stałe }c_1,c_2,n_0,m_0\mbox{ takie, że}\right. \\
  && \quad 0\le c_1g(n,m)\le f(n,m)\le c_2g(n,m)\mbox{ dla wszystkich }n\ge n_0 \\
  && \left.\quad\mbox{oraz }m\ge m_0\right\}
\end{eqnarray*}

\subsection*{3.2. Standardowe notacje i typowe funkcje}

\paragraph{3.2-1.}
Z założenia, jeśli $n_1\le n_2$, to $f(n_1)\le f(n_2)$ i $g(n_1)\le g(n_2)$, więc po dodadaniu nierówności stronami otrzymujemy $f(n_1)+g(n_1)\le f(n_2)+g(n_2)$, czyli że $(f+g)(n)$ jest funkcją monotonicznie rosnącą. Ponieważ $g(n_1)\le g(n_2)$, to traktując wartości funkcji $g$ jako argumenty funkcji $f$, otrzymamy $f(g(n_1))\le f(g(n_2))$, zatem $(f\circ g)(n)\equiv f(g(n))$ jest funkcją monotonicznie rosnącą. Jeśli ponadto założymy, że $f$ i $g$ są nieujemne, to nierówności można pomnożyć stronami bez zmiany znaku nierówności, co daje $f(n_1)f(n_2)\le g(n_1)g(n_2)$, a to oznacza, że również funkcja $(f\cdot g)(n)$ jest monotonicznie rosnąca.

\paragraph{3.2-2.}
Wykorzystując podstawowe własności logarytmów otrzymujemy
\[
  \log_ba^{\log_bc} = \log_bc\cdot\log_ba = \log_bc^{\log_ba}.
\]
Na podstawie różnowartościowości funkcji logarytmicznej wynika tożsamość
\[
  a^{\log_bc}=c^{\log_ba}.
\]

%poprawic
\paragraph{3.2-3.}
Dowód wzoru (3.18). Z wzoru Stirlinga dostajemy
\[
  \lg(n!) = \lg\left(\sqrt{2\pi n}\left(\frac{n}{e}\right)^n\left(1+\Theta\left(\frac{1}{n}\right)\right)\right).
\]
Wykorzystując teraz definicję notacji $\Theta$ oraz twierdzenie 3.1 i wybierając pewną stałą $c>0$ ograniczamy $\lg(n!)$ od dołu:
\begin{eqnarray*}
  \lg(n!) &\ge& \lg\left(\sqrt{2\pi n}\left(\frac{n}{e}\right)^n\left(1+\frac{c}{n}\right)\right) \\
  &=& \lg\sqrt{2\pi} + \lg\sqrt{n} + \lg\left(\frac{n}{e}\right)^n + \lg\left(1+\frac{c}{n}\right) \\
  &=& \underbrace{\lg\sqrt{2\pi}}_{c_1} + \underbrace{\frac{\lg n}{2}}_{c_2\lg n} + n\lg n - \underbrace{n\lg e}_{c_3n} + \underbrace{\lg(n+c)}_{c_4\lg n} - \lg n \\
  &=& \Omega(n\lg n).
\end{eqnarray*}
Wybierając inną stałą i ograniczając $\lg(n!)$ od góry w analogiczny sposób otrzymamy $\lg(n!)=O(n\lg n)$, a więc po kolejnym zastosowaniu twierdzenia 3.1 dostaniemy $\lg(n!)=\Theta(n\lg n)$.

By udowodnić tożsamość $n!=\omega(2^n)$, należy wykazać, że
\[
  \lim_{n\to\infty}\frac{n!}{2^n}=\infty.
\]
Rozwijamy $n!$ wykorzystując wzór Stirlinga i dostajemy:
\begin{eqnarray*}
  \lim_{n\to\infty}\frac{\sqrt{2\pi n}\left(\frac{n}{e}\right)^n\left(1+\Theta\left(\frac{1}{n}\right)\right)}{2^n} &=& \lim_{n\to\infty}\frac{n^n\sqrt{2\pi n}\left(1+\Theta\left(\frac{1}{n}\right)\right)}{(2e)^n} \\
  &=& \sqrt{2\pi}\cdot\lim_{n\to\infty}\frac{n^{n+1/2}}{e^n}\cdot\lim_{n\to\infty}\left(1+\Theta\left(\frac{1}{n}\right)\right) \\
  &=& \infty.
\end{eqnarray*}
Ostania równość zachodzi, ponieważ ciąg $\left(\frac{n^{n+1/2}}{e^n}\right)$ jest rosnący, oraz na podstawie obserwacji, że
\begin{equation}\label{eq-3.2-3}
  \lim_{n\to\infty}\left(1+\Theta\left(\frac{1}{n}\right)\right)\le \lim_{n\to\infty}\left(1+\frac{c}{n}\right)=1,
\end{equation}
gdzie $c>0$ jest dowolną stałą.

Korzystając z (3.1) widać, że dowód tożsamości $n!=o(n^n)$ sprowadza się do pokazania, że
\[
  \lim_{n\to\infty}\frac{n!}{n^n}=0.
\]
Znajdując oszacowanie $n!$ przy pomocy wzoru Stirlinga dostajemy
\begin{eqnarray*}
  \lim_{n\to\infty}\frac{\sqrt{2\pi n}\left(\frac{n}{e}\right)^n\left(1+\Theta\left(\frac{1}{n}\right)\right)}{n^n} &=& \lim_{n\to\infty}\frac{\sqrt{2\pi n}\left(1+\Theta\left(\frac{1}{n}\right)\right)}{e^n} \\
  &=& \sqrt{2\pi}\cdot\lim_{n\to\infty}\frac{\sqrt{n}}{e^n}\cdot\lim_{n\to\infty}\left(1+\Theta\left(\frac{1}{n}\right)\right) \\
  &=& 0.
\end{eqnarray*}
W ostatnim kroku korzystamy z (\ref{eq-3.2-3}) oraz z wzoru (3.5) dla $a=e$ i $b=1/2$.

\paragraph{3.2-4.}
Jeśli pewna funkcja $f(n)$ jest ograniczona wielomianowo, to istnieją stałe $c,k,n_0>0$ takie, że dla każdego $n\ge n_0$ zachodzi $f(n)\le cn^k$. Stąd $\lg f(n)\le ck\lg n$, a więc $\lg f(n)=O(\lg n)$. Stwierdzenie, że funkcja $f(n)$ jest ograniczona wielomianowo, jest przez to  równoważne stwierdzeniu, że $\lg f(n)=O(\lg n)$.

Zanim przejdziemy do głównego dowodu, wykażemy, że $\lceil\lg n\rceil=\Theta(\lg n)$. Zachodzi $\lceil\lg n\rceil\ge\lg n$ oraz $\lceil\lg n\rceil<\lg n+1\le 2\lg n$ dla każdego $n\ge 2$, a więc tożsamość jest prawdziwa.

Logarytmując pierwszą badaną funkcję i wykorzystując powyższą tożsamość oraz wzór 3.18 udowodniony w poprzednim zadaniu, dostajemy
\begin{eqnarray*}
  \lg\left(\lceil\lg n\rceil!\right) &=& \Theta\left(\lceil\lg n\rceil\lg\lceil\lg n\rceil\right) \\
  &=& \Theta(\lg n\lg\lg n) \\
  &=& \omega(\lg n),
\end{eqnarray*}
a zatem $\lg\left(\lceil\lg n\rceil!\right)\ne O(\lg n)$ i $\lceil\lg n\rceil$ nie jest ograniczone wielomianowo.

Dla drugiej funkcji mamy
\begin{eqnarray*}
  \lg\left(\lceil\lg\lg n\rceil!\right) &=& \Theta\left(\lceil\lg\lg n\rceil\lg\lceil\lg\lg n\rceil\right) \\
  &=& \Theta(\lg\lg n\lg\lg\lg n) \\
  &=& o\left(\lg^2(\lg n)\right) \\
  &=& o(\lg n).
\end{eqnarray*}
Ostania krok wynika z tożsamości $\lg^bn=o(n^a)$, w której podstawiono $\lg n$ w miejsce $n$ oraz przyjęto $b=2$ i $a=1$. Otrzymany rezultat potwierdza, że $\lg\left(\lceil\lg\lg n\rceil!\right)=O(\lg n)$, a zatem $\lceil\lg\lg n\rceil$ jest ograniczone wielomianowo.

\paragraph{3.2-5.}
Zachodzi równość $\lg^*(\lg n) = \lg^*n-1$. Jej prawa strona jest funkcją liniową $m_n-1$ dla $m_n = \lg^*n$, podczas gdy $\lg(\lg^*n) = \lg(m_n)$ ma po prawej stronie funkcję logarytmiczną względem $m_n$. Ponieważ $\lg(m_n)=o(m_n-1)$, to stąd mamy
\[
  \lg(\lg^*n) = o(\lg^*(\lg n)).
\]

\paragraph{3.2-6.}
Dla $i=0$ twierdzenie zachodzi trywialnie. Dla $i=1$ mamy $F_1=\frac{\phi-\widehat\phi}{\sqrt{5}}=1$. Załóżmy teraz, że zachodzi
\[
  F_i=\frac{\phi^i-\widehat\phi^i}{\sqrt{5}}\quad\mbox{oraz}\quad F_{i+1}=\frac{\phi^{i+1}-\widehat\phi^{i+1}}{\sqrt{5}}
\]
dla pewnego $i\ge 0$. Ponieważ liczby Fibonacciego dla wszystkich $n\ge 0$ spełniają zależność $F_{n+2}=F_{n+1}+F_n$, to stąd na mocy założenia indukcyjnego zachodzi
\[
  F_{i+2} = \frac{\phi^{i+1}-\widehat\phi^{i+1}}{\sqrt{5}}+\frac{\phi^i-\widehat\phi^i}{\sqrt{5}} = \frac{\phi^i\overbrace{(\phi+1)}^{\phi^2}-\widehat\phi^i\overbrace{\left(\widehat\phi+1\right)}^{\widehat\phi^2}}{\sqrt{5}} = \frac{\phi^{i+2}-\widehat\phi^{i+2}}{\sqrt{5}},
\]
a zatem zależność jest prawdziwa dla każdego $i\ge 0$.

\paragraph{3.2-7.}
Korzystając z wyniku z poprzedniego zadania dostajemy
\begin{eqnarray*}
  F_{i+2} &\ge& \phi^i \\
  \frac{\phi^{i+2}-\widehat\phi^{i+2}}{\sqrt{5}} &\ge& \phi^i \\
  \phi^i\left(\phi^2-\sqrt{5}\right) &\ge& \widehat\phi^{i+2} \\
  \phi^i\cdot\frac{3-\sqrt{5}}{2} &\ge& \widehat\phi^i\cdot\frac{3-\sqrt{5}}{2} \\
  \phi^i &\ge& \widehat\phi^i.
\end{eqnarray*}
Ponieważ $|\phi|>|\widehat\phi|$, to otrzymana nierówność jest spełniona dla każdego $i\ge 0$ i~dowód jest zakończony.

\subsection*{Problemy}

\paragraph{3-1. Asymptotyczne zachowanie wielomianów}

%wykazac dokladnie ze p(n)=\Theta(n^d)
\subparagraph{(a)}
Oczywistym jest, że $p(n) = \Theta(n^d)$, zatem na mocy tw. 3.1 zachodzą równości $p(n) = O(n^d)$ oraz $p(n) = \Omega(n^d)$. Pierwsza z nich oznacza, że dla pewnych stałych $c,n_0>0$ prawdą jest $0\le p(n)\le cn^d$ dla wszystkich $n\ge n_0$. Z kolei $k\ge d$ implikuje $cn^k\ge cn^d$, a zatem $0\le p(n)\le cn^k$, skąd natychmiast otrzymujemy $p(n) = O(n^k)$.
 
\subparagraph{(b)}
$k\le d$ implikuje $cn^k\le cn^d$ dla stałej $c$ z poprzedniego punktu, a~korzystając z tego, że $p(n) = \Omega(n^d)$ mamy $0\le cn^k\le p(n)$, skąd wynika $p(n)=\Omega(n^k)$.

\subparagraph{(c)}
Dla $k=d$, tożsamość $p(n) = \Theta(n^d) = \Theta(n^k)$ trywialnie zachodzi.

\subparagraph{(d)}
Jeśli $k>d$, to $cn^k>cn^d$ dla wszystkich $c>0$. Idąc dalej mamy, że $0\le p(n)<cn^k$, a to prowadzi do wyniku $p(n)=o(n^k)$.

\subparagraph{(e)}
Analogicznie jak w poprzednim punkcie, $k<d$ implikuje $cn^k<cn^d$ dla wszystkich $c>0$, a zatem $0\le cn^k<p(n)$ i otrzymujemy $p(n)=\omega(n^k)$.

\paragraph{3-2. Względny czas asymptotyczny}

\subparagraph{(a)}
Zostało wykazane w tekście, że każdy wielomian rośnie szybciej niż każda funkcja polilogarytmiczna, czyli że $\lg^kn=o(n^\epsilon)$, a stąd wynika, że również $\lg^kn=O(n^\epsilon)$.

\subparagraph{(b)}
Podobnie, z wzoru (3.9) mamy, że $n^k=o(c^n)$, co implikuje również $n^k=O(c^n)$.

\subparagraph{(c)}
Funkcja $\sqrt{n}$ nie jest w żadnej relacji z $n^{\sin n}$, gdyż wartość wykładnika tej ostatniej waha się między $-1$ a $1$ przyjmując wszystkie pośrednie wartości, podczas gdy $\sqrt{n}\equiv n^{1/2}$.

\subparagraph{(d)}
Zachodzi $2^n=\omega(2^{n/2})$, bo
\[
  \lim_{n\to\infty}\frac{2^n}{2^{n/2}}=\lim_{n\to\infty}2^{n/2}=\infty,
\]
przez co prawdą jest także $2^n=\Omega(2^{n/2})$.

\subparagraph{(e)}
Funkcje $n^{\lg c}$ i $c^{\lg n}$ są równe na podstawie tożsamości (3.15).

\subparagraph{(f)}
Z wzoru (3.18), $\lg(n!)=\Theta(n\lg n)$, a $\lg(n^n)=n\lg n=\Theta(n\lg n)$, zatem obie funkcje są asymptotycznie równoważne.

\bigskip
\noindent Na podstawie powyższych faktów otrzymujemy tabelę:
\begin{center}
\begin{tabular}{cc|c|c|c|c|c|}
  $A$ & $B$ & $O$ & $o$ & $\Omega$ & $\omega$ & $\Theta$ \\
  \hline
  $\lg^kn$ & $n^\epsilon$ & tak & tak & nie & nie & nie \\
  \hline
  $n^k$ & $c^n$ & tak & tak & nie & nie & nie \\
  \hline
  $\sqrt{n}$ & $n^{\sin n}$ & nie & nie & nie & nie & nie \\
  \hline
  $2^n$ & $2^{n/2}$ & nie & nie & tak & tak & nie \\
  \hline
  $n^{\lg c}$ & $c^{\lg n}$ & tak & nie & tak & nie & tak \\
  \hline
  $\lg(n!)$ & $\lg(n^n)$ & tak & nie & tak & nie & tak \\
  \hline
\end{tabular}
\end{center}

\paragraph{3-3. Porządkowanie ze względu na rząd wielkości funkcji}

\subparagraph{(a)}
Na poniższej liście funkcje występujące w jednym wierszu należą do tej samej klasy równoważności.
\[
  \begin{array}{ll}
	2^{2^{n+1}} \\
	2^{2^n} \\
	(n+1)! \\
	n! \\
	e^n \\
	n\cdot 2^n \\
	2^n \\
	(3/2)^n \\
	(\lg n)^{\lg n},\quad n^{\lg\lg n} \\
	(\lg n)! \\
	n^3 \\
	4^{\lg n},\quad n^2 \\
	\lg (n!),\quad n\lg n \\
	2^{\lg n},\quad n \\
	\left(\sqrt{2}\right)^{\lg n} \\
	2^{\sqrt{2\lg n}} \\
	\lg^2n \\
	\ln n \\
	\sqrt{\lg n} \\
	\ln\ln n \\
	2^{\lg^*n} \\
	\lg^*(\lg n),\quad \lg^*n \\
	\lg(\lg^*n) \\
	n^{1/\lg n},\quad 1.
  \end{array}
\]

\subparagraph{(b)}
Przykładem takiej funkcji jest
\[
  f(n)=\left\{\begin{array}{ll}
	2^{2^{n+2}}, & \mbox{dla }n\mbox{ parzystych,} \\
	0, & \mbox{dla }n\mbox{ nieparzystych}.
  \end{array}\right.
\]

\paragraph{3-4. Własności notacji asymptotycznej}

\subparagraph{(a)}
Fałsz. Niech np. $f(n)=n$ i $g(n)=n^2$. Wtedy $f(n)=O(g(n))$, ale $g(n)\ne O(f(n))$.

\subparagraph{(b)}
Fałsz. Jako kontrprzykład rozważmy $f(n)=n$ i $g(n)=n^2$. Zachodzi wtedy $\min(f(n),g(n))=f(n)$ oraz $f(n)+g(n)=\Theta(g(n))\ne\Theta(f(n))$.

\subparagraph{(c)}
Prawda. Z faktu, że $f(n)=O(g(n))$ wynika $f(n)\le cg(n)$ dla $n\ge n_0$ i~pewnych stałych $c,n_0>0$. Otrzymujemy
\[
  \lg f(n)\le\lg c+\lg g(n)\le \lg g(n)+\lg g(n) = 2\lg g(n) = O(\lg g(n)).
\]
Ponieważ $c$ jest stałą, to przyjmijmy, że dobrano $n_0$ tak, by $c\le g(n)$ dla wszystkich $n\ge n_0$. Wtedy mamy $\lg c\le\lg g(n)$ i stąd wynika druga nierówność.

\subparagraph{(d)}
Fałsz. Dla funkcji $f(n)=2^n$ oraz $g(n)=2^{n+1}$ zachodzi $f(n)=O(g(n))$, ale $2^{f(n)}\ne O\left(2^{g(n)}\right)$ (z punktu (a) problemu 3-3).

\subparagraph{(e)}
Fałsz. Np. dla $f(n)=2^n$ mamy $f^2(n)=(2^n)^2=4^n$ skąd $f(n)\ne O\left(f^2(n)\right)$.

\subparagraph{(f)}
Prawda. Z definicji notacji $O$, jeśli $f(n)=O(g(n))$, to istnieją stałe $c,n_0>0$, że dla każdego $n\ge n_0$ zachodzi $0\le f(n)\le cg(n)$. Dzieląc nierówność przez $c$ otrzymujemy $0\le f(n)/c\le g(n)$ dla wszystkich $n\ge n_0$, ale $1/c>0$, a~więc $g(n)=\Omega(f(n))$.

\subparagraph{(g)}
Fałsz. Niech np. $f(n)=2^n$, wtedy $f(n/2)=2^{n/2}=\left(\sqrt{2}\right)^n$ oraz $f(n)\ne O(f(n/2))$.

\subparagraph{(h)}
Prawda. Niech $h(n)=o(f(n))$. Wtedy, na podstawie definicji notacji $o$ mamy, że dla każdej stałej $c>0$ istnieje stała $n_0>0$, taka że
\[
  0\le h(n)<cf(n)
\]
zachodzi dla wszystkich $n\ge n_0$. To znaczy, że
\[
  f(n)\le f(n)+o(f(n))=f(n)+h(n)<(c+1)f(n).
\]
Ponieważ $c+1>1$, to można $f(n)+o(f(n))$ ograniczyć od góry przez $c_2f(n)$ wybierając np. $c_2=2$. Dolnym ograniczeniem sumy jest $f(n)$, więc ustalamy $c_1=1$. Stałe $c_1,c_2,n_0$ spełniają założenia definicji notacji $\Theta$, skąd wnioskujemy, że $f(n)+o(f(n))=\Theta(f(n))$.

\bigskip
\noindent \emph{Od tego momentu, w rozwiązaniach zadań nie będziemy odwoływać się do definicji notacji asymptotycznych w obliczaniu oszacowań czasów działania algorytmów, ale korzystając z wyżej udowodnionej tożsamości, będziemy opuszczać składniki niższego rzędu w sumach, których oszacowania chcemy otrzymać.}

\paragraph{3-5. Wariacje na temat notacji $O$ i $\Omega$}

%poprawic
\subparagraph{(a)}
Niech $c,n_0>0$ będą pewnymi stałymi. Dla $n\ge n_0$ wykres funkcji $f(n)$ może przecinać wykres $cg(n)$ w skończonej (być może zerowej) lub nieskończonej liczbie punktów. W pierwszym przypadku dobierzmy $n_0$ tak, by było ono największym argumentem, dla którego następuje przecięcie wykresów. Wtedy, w zależności od znaku różnicy $f(n)-cg(n)$ dla $n>n_0$, funkcja $f(n)$ należy do $O(g(n))$ albo $\stackrel{\infty}{\Omega}\!\!(g(n))$.

Jeśli punktów przecięcia jest nieskończenie wiele, to $f(n)$ może być tylko $\stackrel{\infty}{\Omega}\!\!(g(n))$, bo zawsze można dobrać stałą $c$ tak, by dla nieskończonej liczby punktów wykres $f(n)$ znajdował się ponad wykresem $cg(n)$. Nie jest natomiast prawdą podobne twierdzenie, gdyby zastosować notację $\Omega$ zamiast $\stackrel{\infty}{\Omega}$; jeśli np. $f(n)=n$ oraz $g(n)=n^{\sin n+1}$, to $f(n)=\;\stackrel{\infty}{\Omega}\!\!(g(n))$, ale $f(n)\ne\Omega(g(n))$.

%poprawic
\subparagraph{(b)}
\begin{description}
  \item[Zalety:] wiadomo, kiedy czas działania algorytmu na pewno nie jest $O(g(n))$.
  \item[Wady:] dowód, że $f(n)=\;\stackrel{\infty}{\Omega}\!\!(g(n))$ jest nieco trudniejszy do przeprowadzenia niż w przypadku $f(n)=\Omega(g(n))$.
\end{description}

%sprawdzic
\subparagraph{(c)}
Definicja $O'$ dopuszcza badanie funkcji, które nie są asymptotycznie nieujemne, zachodzi np.
\begin{eqnarray*}
  -n^2 &=& O'(n^2), \\
  -\frac{n^3}{2}+\frac{n^2}{6}-3n &=& O'(n^4).
\end{eqnarray*}
Jeśli przyjmiemy taką definicję $O'$, ale pozostawimy bez zmian definicje $\Omega$ i $\Theta$, to twierdzenie 3.1 nie będzie prawdziwe. Gdy $f(n)=O'(g(n))$, to $g(n)$ jest funkcją asymptotycznie nieujemną. Jeśli jednak $f(n)$ nie jest asymptotycznie nieujemna, to w przypadku gdy $f(n)=\Theta(h(n))$, funkcja $h(n)$ też nie może być nieujemna. Twierdzenie pozostanie prawdziwe, jeśli równoważność zastąpimy implikacją w prawą stronę, tzn. z faktu, że $f(n)=\Theta(g(n))$ wynika, że zachodzą także $f(n)=O(g(n))$ i $f(n)=\Omega(g(n))$.

\subparagraph{(d)}
\begin{eqnarray*}
  \stackrel{\sim}{\Omega}\!\!(g(n)) &=& \left\{f(n):\;\mbox{istnieją dodatnie stałe }c,k,n_0\mbox{ takie, że}\right. \\
  && \quad 0\le cg(n)\lg^kn\le f(n)\mbox{ dla wszystkich }n\ge n_0\left.\right\} \\\\
  \stackrel{\sim}{\Theta}\!\!(g(n)) &=& \left\{f(n):\;\mbox{istnieją dodatnie stałe }c_1,c_2,k_1,k_2,n_0\mbox{ takie, że}\right. \\
  && \quad 0\le c_1g(n)\lg^{k_1}n\le f(n)\le c_2g(n)\lg^{k_2}n \\
  && \left.\quad\mbox{dla wszystkich }n\ge n_0\right\}
\end{eqnarray*}

Dowód twierdzenia 3.1 dla notacji $\stackrel{\sim}{O}$, $\stackrel{\sim}{\Omega}$ i $\stackrel{\sim}{\Theta}$ przebiega analogicznie jak dowód jego oryginalnego odpowiednika, przeprowadzony w zad. 3.1-5.

Z definicji notacji $\stackrel{\sim}{\Theta}$ mamy, że $f(n)=\;\stackrel{\sim}{\Theta}\!\!(g(n))$ wtedy i tylko wtedy, gdy istnieją takie stałe $c_1,c_2,k_1,k_2,n_0>0$, że
\[
  0\le c_1g(n)\lg^{k_1}n\le f(n)\le c_2g(n)\lg^{k_2}n
\]
zachodzi dla wszystkich $n\ge n_0$. Rozdzielając nierówność otrzymujemy
\[
  \left\{\begin{array}{r}
	0\le c_1g(n)\lg^{k_1}n\le f(n), \\
	0\le f(n)\le c_2g(n)\lg^{k_2}n.
  \end{array}\right.
\]
Pierwsze wyrażenie stanowi definicję $f(n)=\;\stackrel{\sim}{\Omega}\!\!(g(n))$, a drugie -- $f(n)=\;\stackrel{\sim}{O}\!\!(g(n))$.

Podobnie jak w oryginalnym dowodzie, przeprowadzenie rozumowania w odwrotnej kolejności wykazuje prawdziwość odwrotnej implikacji.

\paragraph{3-6. Funkcje iterowane}

Za dziedziny funkcji $f_c^*(n)$ przyjęto dziedziny $f(n)$.

\subparagraph{(a)}
\[
  f_c^*(n)=\left\{\begin{array}{ll}
	\lg^*n+1, & \mbox{dla }n\ge 1, \\
	0, & \mbox{dla }0<n<1.
  \end{array}\right.
\]

\subparagraph{(b)}
\[
  f_c^*(n)=\left\{\begin{array}{ll}
	\lceil n\rceil-1, & \mbox{dla }n\ge 1, \\
	0, & \mbox{dla }n<1.
  \end{array}\right.
\]

\subparagraph{(c)}
\[
  f_c^*(n)=\left\{\begin{array}{ll}
	\lceil\lg n\rceil, & \mbox{dla }n\ge 1, \\
	0, & \mbox{dla }n<1.
  \end{array}\right.
\]

\subparagraph{(d)}
\[
  f_c^*(n)=\left\{\begin{array}{ll}
	\lceil\lg n\rceil-1, & \mbox{dla }n\ge 2, \\
	0, & \mbox{dla }n<2.
  \end{array}\right.
\]

\subparagraph{(e)}
\[
  f_c^*(n)=\left\{\begin{array}{ll}
	\lceil\log_4 n\rceil, & \mbox{dla }n>2, \\
	0, & \mbox{dla }0\le n\le 2.
  \end{array}\right.
\]

\subparagraph{(f)}
\[
  f_c^*(n)=\left\{\begin{array}{ll}
	\infty, & \mbox{dla }n>1, \\
	0, & \mbox{dla }0\le n\le 1.
  \end{array}\right.
\]

\subparagraph{(g)}
\[
  f_c^*(n)=\left\{\begin{array}{ll}
	\lceil\log_8 n\rceil, & \mbox{dla }n>2, \\
	0, & \mbox{dla }n\le 2.
  \end{array}\right.
\]

\subparagraph{(h)}
