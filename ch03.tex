\chapter{Rzędy wielkości funkcji}

\subchapter{Notacja asymptotyczna}

\exercise %3.1-1
Załóżmy, że dziedzinami funkcji $f(n)$ i~$g(n)$ jest zbiór $\mathbb{N}$.
Niech $A=\{\,n\in\mathbb{N}:f(n)\ge g(n)\,\}$.
Dla odpowiednio dużych $n\in A$, dla których obie funkcje przyjmują wartości nieujemne, mamy
\[
    \max(f(n),g(n)) = f(n) \le f(n)+g(n) = O(f(n)+g(n)).
\]
Z~kolei
\[
    \max(f(n),g(n)) = f(n) \ge \frac{f(n)+g(n)}{2} = \Omega(f(n)+g(n))
\]
i~na mocy tw.\ 3.1 otrzymujemy, że $\max(f(n),g(n))=\Theta(f(n)+g(n))$.
Identyczny rezultat można uzyskać dla $n\in\mathbb{N}\setminus A$, tzn.\ gdy $f(n)<g(n)$.

\exercise %3.1-2
Aby pokazać, że $(n+a)^b=\Theta(n^b)$, należy znaleźć stałe $c_1$, $c_2$, $n_0>0$ takie, że
\[
	0 \le c_1n^b \le (n+a)^b \le c_2n^b
\]
dla wszystkich $n\ge n_0$.
Zauważmy, że $n+a\le n+|a|\le2n$, gdy $|a|\le n$ oraz $n+a\ge n-|a|\ge n/2$, o~ile $|a|\le n/2$.
Stąd, jeśli $n\ge 2|a|$, to zachodzi
\[
	0 \le n/2 \le n+a \le 2n.
\]
Ponieważ $b>0$, to powyższe nierówności możemy podnieść do potęgi $b$:
\begin{gather*}
	0 \le (n/2)^b \le (n+a)^b \le (2n)^b, \\
	0 \le 2^{-b}n^b \le (n+a)^b \le 2^bn^b.
\end{gather*}
Widać zatem, że szukanym stałym można nadać wartości $c_1=2^{-b}$, $c_2=2^b$ oraz $n_0=2|a|$.
Prawdą jest zatem, że $(n+a)^b=\Theta(n^b)$.

\exercise %3.1-3
Niech $T(n)$ będzie czasem działania algorytmu $A$.
Stwierdzenie ,,$T(n)$ wynosi co najmniej $O(n^2)$'' oznacza, że począwszy od pewnego $n$, $T(n)\ge f(n)$ dla pewnej funkcji $f(n)$ z~klasy $O(n^2)$.
Zdanie to pozostaje prawdziwe dla dowolnego $T$, wystarczy bowiem wybrać funkcję $f(n)$ tożsamościowo równą 0, która oczywiście jest w~$O(n^2)$.
Widać więc, że takie określenie nie przekazuje żadnej użytecznej informacji o~czasie działania algorytmu.

\exercise %3.1-4
Znajdziemy stałe $c$, $n_0>0$ takie, że $0\le2^{n+1}\le c2^n$ dla każdego $n\ge n_0$.
Ponieważ $2^{n+1}=2\cdot2^n$ dla każdego $n\ge1$, to można przyjąć $c=2$ oraz $n_0=1$.
A~zatem $2^{n+1}=O(2^n)$.

Spróbujmy teraz wyznaczyć te same stałe, ale spełniające zależność $0\le2^{2n}\le c2^n$ dla wszystkich $n\ge n_0$.
Mamy $2^{2n}=2^n\cdot2^n\le c2^n$, z~czego wynika, że $c\ge2^n$, co jednak uzależnia $c$ od funkcji zmiennej $n$ przyjmującej dowolnie duże wartości, a~więc $c$ nie może być stałą.
Stąd otrzymujemy, że $2^{2n}\ne O(2^n)$.

\exercise %3.1-5
Z~definicji notacji $\Theta$ mamy, że $f(n)=\Theta(g(n))$ wtedy i~tylko wtedy, gdy istnieją takie stałe $c_1$, $c_2$, $n_0>0$, że nierówności
\[
	0 \le c_1g(n) \le f(n) \le c_2g(n)
\]
zachodzą dla wszystkich $n\ge n_0$.
Możemy je zapisać w~formie układu
\[
	\begin{cases}
		0 \le c_1g(n) \le f(n) \\
		0 \le f(n) \le c_2g(n)
	\end{cases}.
\]
Z~pierwszej nierówności układu dostajemy, że $f(n)=\Omega(g(n))$, a~z~drugiej, że $f(n)=O(g(n))$.

\exercise %3.1-6
Niech $c_1$, $c_2$, $n_1$, $n_2$ będą pewnymi stałymi dodatnimi.
Pesymistyczny czas działania algorytmu wynosi $O(g(n))$ wtedy i~tylko wtedy, gdy dla dowolnych danych wejściowych rozmiaru $n\ge n_1$ czas jego działania $f(n)$ nie przekracza $c_1g(n)$.
Z~kolei to, że optymistyczny czas wynosi $\Omega(g(n))$, oznacza, że dla dowolnych danych wejściowych rozmiaru $n\ge n_2$ czas działania algorytmu $f(n)$ jest nie mniejszy niż $c_2g(n)$.
Widać zatem, że dla dowolnych danych rozmiaru $n\ge\max(n_1,n_2)$ mamy $0\le c_2g(n)\le f(n)\le c_1g(n)$, a~to jest równoważne z~tym, że $f(n)=\Theta(g(n))$.

\exercise %3.1-7
Załóżmy niepustość tego zbioru i~rozważmy pewne $f(n)\in o(g(n))\cap\omega(g(n))$.
Zachodzi zatem zarówno $f(n)=o(g(n))$, jak i~$f(n)=\omega(g(n))$, co oznacza, że dla każdych dodatnich stałych $c_1$ i~$c_2$ istnieje pewne dodatnie $n_0$, że
\[
	c_1g(n) < f(n) < c_2g(n)
\]
dla wszystkich $n\ge n_0$.
Dochodzimy do sprzeczności, bowiem nieprawdą jest, że każde liczby $c_1$ i~$c_2$ spełniają $c_1<c_2$.
Stąd $o(g(n))\cap\omega(g(n))=\emptyset$.

Udowodniona własność pokazuje, że nie ma potrzeby definiowania notacji $\theta$ odpowiadającej $\Theta$ i~analogicznej do $o$ i~$\omega$.

\exercise %3.1-8
\note{Notacja\/ $O$ dla funkcji dwóch zmiennych jest błędnie zdefiniowana -- warunek powinien zachodzić dla wszystkich\/ $n\ge n_0$ lub\/ $m\ge m_0$.}

\noindent Definicje notacji $\Omega$ i~$\Theta$ dla funkcji dwóch zmiennych:
\[
	\begin{split}
		\Omega(g(n,m)) &= \bigl\{\,f(n,m):\text{istnieją dodatnie stałe $c$, $n_0$, $m_0$ takie, że} \\
		&\qquad 0 \le cg(n,m) \le f(n,m) \text{ dla wszystkich $n \ge n_0$ lub $m \ge m_0$}\,\bigr\}, \\[2mm]
		\Theta(g(n,m)) &= \bigl\{\,f(n,m):\text{istnieją dodatnie stałe $c_1$, $c_2$, $n_0$, $m_0$ takie, że} \\
		&\qquad 0 \le c_1g(n,m) \le f(n,m) \le c_2g(n,m) \text{ dla wszystkich $n \ge n_0$ lub $m \ge m_0$}\,\bigr\}.
	\end{split}
\]

\subchapter{Standardowe notacje i~typowe funkcje}

\exercise %3.2-1
Z~założenia, jeśli $n_1\le n_2$, to zachodzi $f(n_1)\le f(n_2)$ oraz $g(n_1)\le g(n_2)$, więc po dodaniu tych nierówności stronami otrzymujemy $f(n_1)+g(n_1)\le f(n_2)+g(n_2)$, czyli że $f(n)+g(n)$ jest funkcją monotonicznie rosnącą.
Traktując wartości funkcji $g(n)$ jako argumenty funkcji $f(n)$, otrzymamy $f(g(n_1))\le f(g(n_2))$, zatem $f(g(n))$ także jest funkcją monotonicznie rosnącą.
Jeśli ponadto założymy, że funkcje $f(n)$ i~$g(n)$ są nieujemne, to początkowe nierówności można pomnożyć stronami, co daje $f(n_1)\cdot g(n_1)\le f(n_2)\cdot g(n_2)$, a~to oznacza, że również funkcja $f(n)\cdot g(n)$ jest monotonicznie rosnąca.

\exercise %3.2-2
Wykorzystując podstawowe własności logarytmów, otrzymujemy
\[
	\log_ba^{\log_bc} = \log_bc\cdot\log_ba = \log_bc^{\log_ba},
\]
skąd na podstawie różnowartościowości funkcji logarytmicznej wynika tożsamość
\[
	a^{\log_bc} = c^{\log_ba}.
\]

\exercise %3.2-3
\begin{proof}[Dowód wzoru (3.18)]
	Górne oszacowanie na $\lg(n!)$ dostajemy dzięki wykorzystaniu własności logarytmów:
	\[
	    \lg(n!) = \lg\biggl(\prod_{i=1}^ni\biggr) = \sum_{i=1}^n\lg i \le \sum_{i=1}^n\lg n = n\lg n = O(n\lg n).
	\]
	Wykorzystując wzór Stirlinga i~wybierając pewną stałą $c>0$, ograniczamy $\lg(n!)$ od dołu:
	\begin{align*}
		\lg(n!) &\ge \lg\biggl(\sqrt{2\pi n}\biggl(\frac{n}{e}\biggr)^n\biggl(1+\frac{c}{n}\biggr)\biggr) \\
		&= \lg\sqrt{2\pi n}+\lg\biggl(\frac{n}{e}\biggr)^n+\lg\biggl(1+\frac{c}{n}\biggr) \\
		&> n\lg n-n\lg e \\
		&\ge n\lg n-\frac{n\lg n}{2} \\
		&= \frac{n\lg n}{2}.
	\end{align*}
	Przedostatnia nierówność zachodzi, o~ile $n\ge e^2$.
Otrzymany wynik dowodzi, że $\lg(n!)=\Omega(n\lg n)$ i~po skorzystaniu z~twierdzenia 3.1 dostajemy $\lg(n!)=\Theta(n\lg n)$.
\end{proof}

\begin{proof}[Dowód tożsamości $n!=\omega(2^n)$]
	Równoważnie należy pokazać, że zachodzi
	\[
		\lim_{n\to\infty}\frac{n!}{2^n} = \infty.
	\]
	Zauważmy, że
	\[
	    \frac{n!}{2^n} = \biggl(\frac{1}{2}\biggr)\biggl(\frac{2}{2}\biggr)\dots\biggl(\frac{n}{2}\biggr).
	\]
	Wszystkie czynniki powyższego iloczynu są dodatnie, a~przy coraz większym $n$ ostatnie czynniki rosną nieograniczenie, zatem cały iloczyn dąży do $\infty$.
\end{proof}

\begin{proof}[Dowód tożsamości $n!=o(n^n)$]
	Na podstawie wzoru (3.1) dowód sprowadza się do pokazania, że
	\[
		\lim_{n\to\infty}\frac{n!}{n^n} = 0.
	\]
	Mamy
	\[
	    \frac{n!}{n^n} = \biggl(\frac{1}{n}\biggr)\biggl(\frac{2}{n}\biggr)\dots\biggl(\frac{n}{n}\biggr).
	\]
	Każdy czynnik po prawej stronie znaku równości jest dodatni i~nie przekracza 1.
Ponadto dla $n$ dążącego do $\infty$ początkowe czynniki zmierzają do 0, a~zatem granicą tego iloczynu jest 0.
\end{proof}

\exercise %3.2-4
Funkcja $f(n)$ jest ograniczona wielomianowo, jeżeli istnieją stałe $c$, $k$, $n_0>0$ takie, że dla każdego $n\ge n_0$ zachodzi $f(n)\le cn^k$.
Stąd $\lg f(n)\le k\lg n+\lg c\le(k+1)\lg n$, o~ile $n\ge c$, a~więc $\lg f(n)=O(\lg n)$.
Stwierdzenie, że funkcja $f(n)$ jest ograniczona wielomianowo, jest więc równoważne stwierdzeniu, że $\lg f(n)=O(\lg n)$.

Zanim przejdziemy do głównego dowodu, zauważmy, że $\lceil\lg n\rceil=\Theta(\lg n)$.
Zachodzi bowiem $\lceil\lg n\rceil\ge\lg n$ oraz $\lceil\lg n\rceil<\lg n+1\le2\lg n$ dla każdego $n\ge2$.

Logarytmując pierwszą badaną funkcję przy wykorzystaniu wzoru (3.18), dostajemy
\[
	\lg(\lceil\lg n\rceil!) = \Theta(\lceil\lg n\rceil\lg\lceil\lg n\rceil) = \Theta(\lg n\lg\lg n) = \omega(\lg n),
\]
a~zatem $\lg(\lceil\lg n\rceil!)\ne O(\lg n)$ i~funkcja $\lceil\lg n\rceil!$ nie jest ograniczona wielomianowo.

Dla drugiej funkcji mamy
\[
	\lg(\lceil\lg\lg n\rceil!) = \Theta(\lceil\lg\lg n\rceil\lg\lceil\lg\lg n\rceil) = \Theta(\lg\lg n\lg\lg\lg n) = o((\lg\lg n)^2) = o(\lg n).
\]
Ostatni krok wynika z~tożsamości $\lg^bn=o(n^a)$ prawdziwej dla stałych $a$, $b>0$, w~której podstawiono $\lg n$ w~miejsce $n$ oraz przyjęto $a=1$ i~$b=2$.
Otrzymany rezultat potwierdza, że $\lg(\lceil\lg\lg n\rceil!)=O(\lg n)$, a~zatem funkcja $\lceil\lg\lg n\rceil!$ jest ograniczona wielomianowo.

\exercise %3.2-5
Zdefiniujmy $n$ jako
\[
    2^{2^{\cdot^{\cdot^{\cdot^{2^\epsilon}}}}}\vbox{\hbox{$\Big\}\scriptstyle k$}\kern0pt},
\]
przy czym $0<\epsilon\le1$, a~$k\ge1$ oznacza liczbę dwójek w~powyższym zapisie.
Zachodzi
\[
    \lg^*n=k \quad\text{oraz}\quad \lg n = 2^{2^{\cdot^{\cdot^{\cdot^{2^\epsilon}}}}}\vbox{\hbox{$\Big\}\scriptstyle k-1$}\kern0pt},
\]
a~zatem
\[
    \lg\lg^*n = \lg k \quad\text{oraz}\quad \lg^*\lg n = k-1.
\]
Oczywiście $k-1=\omega(\lg k)$, więc otrzymujemy, że $\lg^*\lg n=\omega(\lg\lg^*n)$.

\exercise %3.2-6
Łatwo sprawdzić, że dla $i=0$ oraz $i=1$ wzór jest prawdziwy.
Załóżmy teraz, że zachodzi
\[
	F_i = \frac{\phi^i-\widehat\phi^i}{\sqrt{5}} \quad\text{oraz}\quad F_{i+1} = \frac{\phi^{i+1}-\widehat\phi^{i+1}}{\sqrt{5}}
\]
dla pewnego $i\ge0$.
Po wykorzystaniu zależności $\phi+1=\phi^2$ i~$\widehat\phi+1=\widehat\phi^2$ otrzymujemy
\[
	F_{i+2} = F_{i+1}+F_i = \frac{\phi^{i+1}-\widehat\phi^{i+1}}{\sqrt{5}}+\frac{\phi^i-\widehat\phi^i}{\sqrt{5}} = \frac{\phi^i(\phi+1)-\widehat\phi^i\bigl(\widehat\phi+1\bigr)}{\sqrt{5}} = \frac{\phi^{i+2}-\widehat\phi^{i+2}}{\sqrt{5}},
\]
a~zatem wzór jest spełniony dla każdego $i\ge0$.

\exercise %3.2-7
Korzystając z~wyniku z~poprzedniego zadania, mamy
\begin{align*}
    F_{i+2}-\phi^i &= \frac{\phi^{i+2}-\widehat\phi^{i+2}}{\sqrt{5}}-\phi^i \\[1mm]
	&= \frac{\phi^i(\phi^2-\sqrt{5})-\widehat\phi^{i+2}}{\sqrt{5}} \\[1mm]
	&= \frac{\phi^i\cdot\frac{3-\sqrt{5}}{2}-\widehat\phi^i\cdot\frac{3-\sqrt{5}}{2}}{\sqrt{5}} \\
	&= \frac{3-\sqrt{5}}{2\sqrt{5}}\,\bigl(\phi^i-\widehat\phi^i\bigr).
\end{align*}
Ponieważ $\phi>|\widehat\phi|$, to otrzymane wyrażenie jest nieujemne dla każdego $i\ge0$ (równość zachodzi tylko wówczas, gdy $i=0$), a~zatem $F_{i+2}\ge\phi^i$ dla dowolnego $i\ge0$.

\problems

\problem{Asymptotyczne zachowanie wielomianów} %3-1
Udowodnimy najpierw fakt, że $p(n)=\Theta(n^d)$.
Należy znaleźć stałe $c_1$, $c_2$, $n_0>0$ takie, że dla $n\ge n_0$ prawdziwe są nierówności:
\[
	0 \le c_1n^d \le a_dn^d+a_{d-1}n^{d-1}+\dots+a_0 \le c_2n^d.
\]
Po podzieleniu ich przez $n^d$ dostajemy
\[
	0 \le c_1 \le a_d+\underbrace{\frac{a_{d-1}}{n}+\dots+\frac{a_0}{n^d}}_\delta \le c_2,
\]
a~ponieważ składnik $\delta$ można dowolnie zbliżyć do zera, zwiększając parametr $n_0$, to stąd obie wartości $c_1$ i~$c_2$ mogą być dowolnie bliskie $a_d$.

\subproblem %3-1(a)
Zachodzi $p(n)=\Theta(n^d)$, zatem w~szczególności $p(n)=O(n^d)$, co oznacza, że istnieją stałe $c$, $n_0>0$, że dla wszystkich $n\ge n_0$ prawdą jest $0\le p(n)\le cn^d$.
Z~kolei $k\ge d$, więc $n^k\ge n^d$ dla $n\ge1$, a~zatem $0\le p(n)\le cn^d\le cn^k$, skąd dostajemy, że $p(n)=O(n^k)$.

\subproblem %3-1(b)
Nierówność $k\le d$ implikuje $n^k\le n^d$ dla $n\ge1$.
Korzystając z~tego, że $p(n)=\Omega(n^d)$, mamy $0\le cn^k\le cn^d\le p(n)$, skąd wynika $p(n)=\Omega(n^k)$.

\subproblem %3-1(c)
Dla $k=d$ tożsamość $p(n)=\Theta(n^d)=\Theta(n^k)$ zachodzi w~oczywisty sposób.

\subproblem %3-1(d)
Zachodzi $p(n)=O(n^d)$, tzn.\ istnieją stałe $c$, $n_0>0$ takie, że dla każdego $n\ge n_0$ spełniona jest nierówność $0\le p(n)\le cn^d$.
Niech $b>0$ będzie dowolną stałą.
Zbadajmy, dla jakich $n$ zachodzi $cn^d<bn^k$.
Ponieważ $k>d$, to nierówność sprowadzamy do postaci $c/b<n^{k-d}$, skąd $n>n_1=(c/b)^{1/(k-d)}$.
A~zatem dla każdego $b>0$ istnieje $n_2=\max(n_0,n_1+1)$, że dla wszystkich $n\ge n_2$ zachodzi $0\le p(n)\le cn^d<bn^k$, co oznacza, że $p(n)=o(n^k)$.

\subproblem %3-1(e)
Rozumowanie jest analogiczne do tego z~poprzedniego punktu.
Wystarczy wykorzystać fakt, że $p(n)=\Omega(n^d)$ i~pokazać, że dla każdej stałej $b>0$ odpowiednio duże $n$ spełniają nierówność $0\le bn^k<cn^d\le p(n)$, gdzie $c>0$ jest stałą ukrytą w~notacji $\Omega$.

\problem{Względny rząd asymptotyczny} %3-2

\subproblem %3-2(a)
Ponieważ każdy wielomian rośnie szybciej niż dowolna funkcja polilogarytmiczna, czyli $\lg^kn=o(n^\epsilon)$, to stąd wynika, że również $\lg^kn=O(n^\epsilon)$.

\subproblem %3-2(b)
Podobnie, ze wzoru (3.9), mamy, że $n^k=o(c^n)$, co implikuje również $n^k=O(c^n)$.

\subproblem %3-2(c)
Wyrażenie $\sqrt{n}$ nie jest w~żadnej rozważanej relacji z~wyrażeniem $n^{\sin n}$, gdyż wartość wykładnika tego ostatniego przyjmuje wszystkie wartości między $-1$ a~1, podczas gdy $\sqrt{n}\equiv n^{1/2}$.

\subproblem %3-2(d)
Zachodzi $2^n=\omega(2^{n/2})$, bo
\[
	\lim_{n\to\infty}\frac{2^n}{2^{n/2}} = \lim_{n\to\infty}2^{n/2} = \infty,
\]
a~stąd wynika też $2^n=\Omega(2^{n/2})$.

\subproblem %3-2(e)
Funkcje $n^{\lg c}$ i~$c^{\lg n}$ dla $n>0$ są równoważne na podstawie tożsamości (3.15).

\subproblem %3-2(f)
Ze wzoru (3.18) mamy $\lg(n!)=\Theta(n\lg n)$, z~kolei $\lg n^n=n\lg n=\Theta(n\lg n)$, a~zatem obie funkcje są asymptotycznie równoważne.

\bigskip
\noindent Na podstawie powyższych uzasadnień dostajemy tabelę \ref{tab:3-2}.
\begin{table}[ht]
	\begin{center}
		\begin{tabular}{cc|c|c|c|c|c}
			$A$ & $B$ & $O$ & $o$ & $\Omega$ & $\omega$ & $\Theta$ \\
			\hline
			$\lg^kn$ & $n^\epsilon$ & tak & tak & nie & nie & nie \\
			\hline
			$n^k$ & $c^n$ & tak & tak & nie & nie & nie \\
			\hline
			$\sqrt{n}$ & $n^{\sin n}$ & nie & nie & nie & nie & nie \\
			\hline
			$2^n$ & $2^{n/2}$ & nie & nie & tak & tak & nie \\
			\hline
			$n^{\lg c}$ & $c^{\lg n}$ & tak & nie & tak & nie & tak \\
			\hline
			$\lg(n!)$ & $\lg n^n$ & tak & nie & tak & nie & tak
		\end{tabular}
		\caption{Porównanie funkcji na podstawie rzędu asymptotycznego.} \label{tab:3-2}
	\end{center}
\end{table}

\problem{Porządkowanie ze względu na rząd wielkości funkcji} %3-3

\subproblem %3-3(a)
Poniższe uzasadnienia stanowią dowody, że $g_i(n)=\Omega(g_{i+1}(n))$ dla $i=1$, 2, \dots, 29, gdzie $g_i(n)$ to rozważane kolejno funkcje.
W~niektórych dowodach korzystamy z~obserwacji, że jeśli $f(n)=g(n)h(n)$ i~$h(n)=\omega(1)$, to $f(n)=\omega(g(n))$.
Ponadto wykorzystujemy fakt, że $\lg f(n)=\omega(\lg g(n))$ implikuje $f(n)=\omega(g(n))$, co można łatwo wykazać.
\begin{itemize}
\item $2^{2^{n+1}}=\Omega\bigl(2^{2^n}\bigr)$
	\[
		2^{2^{n+1}} = 2^{2^n}\cdot2^{2^n} = \omega\bigl(2^{2^n}\bigr), \quad\text{bo $2^{2^n} = \omega(1)$}.
	\]
\item $2^{2^n}=\Omega((n+1)!)$ \\
	Logarytmując obie funkcje i~wykorzystując wzór (3.18), otrzymujemy
	\[
		\lg 2^{2^n} = 2^n \quad\text{oraz}\quad \lg((n+1)!) = \Theta((n+1)\lg(n+1)) = \Theta(n\lg n).
	\]
	Ponieważ $2^n=\omega(n^2)$ i~$n^2=\omega(n\lg n)$, to mamy, że $2^n=\omega(n\lg n)$.
Powracając do początkowych funkcji, dostajemy $2^{2^n}=\omega((n+1)!)$, skąd wynika prawdziwość dowodzonej zależności.
\item $(n+1)!=\Omega(n!)$
	\[
		(n+1)! = (n+1)\cdot n! = \omega(n!), \quad\text{bo $n+1 = \omega(1)$}.
	\]
\item $n!=\Omega(e^n)$ \\
	Zbadajmy logarytmy obu funkcji.
Ze wzoru (3.18) otrzymujemy $\lg(n!)=\Theta(n\lg n)=\omega(n)$, a~$\lg e^n=\Theta(n)$.
Prawdą jest zatem, że $\lg(n!)=\omega(\lg e^n)$ i~stąd wynika zależność $n!=\omega(e^n)$.
\item $e^n=\Omega(n\cdot2^n)$
	\[
		e^n = (e/2)^n2^n = \omega(n\cdot2^n), \quad\text{bo $(e/2)^n = \omega(n)$},
	\]
	ponieważ funkcje wykładnicze rosną szybciej niż wielomiany.
\item $n\cdot2^n=\Omega(2^n)$ \\
	Tożsamość zachodzi, bo $n=\omega(1)$.
\item $2^n=\Omega((3/2)^n)$
	\[
		2^n = (4/3)^n(3/2)^n = \omega((3/2)^n), \quad\text{bo $(4/3)^n = \omega(1)$}.
	\]
\item $(3/2)^n=\Omega\bigl(n^{\lg\lg n}\bigr)$ \\
	Logarytmując obie funkcje, dostajemy
	\[
		\lg(3/2)^n = \Theta(n) \quad\text{oraz}\quad \lg n^{\lg\lg n} = \lg n\lg\lg n = o(\lg^2n).
	\]
	Wystarczy pokazać, że $n=\omega(\lg^2n)$.
Po podstawieniu $n=2^h$ wzór przyjmuje postać $2^h=\omega(h^2)$, co jest prawdą, ponieważ funkcja wykładnicza rośnie szybciej niż wielomian.
\item $n^{\lg\lg n}=\Omega\bigl((\lg n)^{\lg n}\bigr)$ \\
	Na mocy tożsamości (3.15) funkcje są równoważne.
\item $(\lg n)^{\lg n}=\Omega((\lg n)!)$ \\
	Jeśli podstawimy $n=2^h$, to otrzymamy wzór $h^h=\Omega(h!)$, który jest prawdziwy na podstawie zależności $n!=o(n^n)$, zaprezentowanej w~Podręczniku.
\item $(\lg n)!=\Omega(n^3)$ \\
	Korzystając z~logarytmu pierwszej funkcji oszacowanego w~poprzednim uzasadnieniu oraz z~tego, że $\lg n^3=\Theta(\lg n)$, dostajemy żądany wynik, ponieważ $\lg\lg n=\omega(1)$.
\item $n^3=\Omega(n^2)$ \\
	Tożsamość zachodzi wprost z~punktu (b) problemu \refProblem{3-1}.
\item $n^2=\Omega\bigl(4^{\lg n}\bigr)$ \\
	Funkcje są tożsame -- na podstawie wzoru (3.15) mamy $4^{\lg n}=n^{\lg4}=n^2$.
\item $4^{\lg n}=\Omega(n\lg n)$ \\
	Wzór zachodzi, bo $4^{\lg n}=n^2$, a~$n=\omega(\lg n)$.
\item $n\lg n=\Omega(\lg(n!))$ \\
	Obie funkcje są asymptotycznie równoważne na podstawie wzoru (3.18).
\item $\lg(n!)=\Omega(n)$ \\
	Ze wzoru (3.18) mamy, że $\lg(n!)=\Theta(n\lg n)$, więc tożsamość jest prawdziwa, bo $\lg n=\omega(1)$.
\item $n=\Omega\bigl(2^{\lg n}\bigr)$ \\
	Na mocy wzoru (3.15) zachodzi $2^{\lg n}=n^{\lg2}=n$, a~więc funkcje są tożsame.
\item $2^{\lg n}=\Omega\bigl(\!\bigl(\!\sqrt{2}\bigr)^{\lg n}\bigr)$ \\
	Z~poprzedniego uzasadnienia mamy, że $2^{\lg n}=n$, a~$\bigl(\!\sqrt{2}\bigr)^{\lg n}=n^{\lg\sqrt{2}}=\sqrt{n}$ (ze wzoru (3.15)), więc tożsamość zachodzi, ponieważ $\sqrt{n}=\omega(1)$.
\item $\bigl(\!\sqrt{2}\bigr)^{\lg n}=\Omega\bigl(2^{\sqrt{2\lg n}}\bigr)$ \\
	Rozważmy tożsamość $2^{\lg n}=n$ i~podnieśmy ją do potęgi $\sqrt{2/\!\lg n}$.
Otrzymujemy $2^{\sqrt{2\lg n}}=n^{\sqrt{2/\!\lg n}}$, a~zatem $2^{\sqrt{2\lg n}}=\Theta\bigl(n^{\sqrt{2/\!\lg n}}\bigr)$.
Ponieważ $\bigl(\!\sqrt{2}\bigr)^{\lg n}=n^{1/2}$, to wystarczy pokazać, że $1/2=\Omega\bigl(\!\sqrt{2/\!\lg n}\bigr)$.
Wzór oczywiście zachodzi, gdyż funkcja z~prawej strony jest malejąca i~dąży do 0 wraz ze wzrostem $n$.
\item $2^{\sqrt{2\lg n}}=\Omega(\lg^2 n)$ \\
	Biorąc logarytmy obu funkcji, dostajemy
	\[
		\lg2^{\sqrt{2\lg n}} = \sqrt{2\lg n} = \Theta\bigl(\lg^{1/2}n\bigr) \quad\text{oraz}\quad \lg\lg^2n = \Theta(\lg\lg n).
	\]
	Pozostaje zatem zbadać prawdziwość wzoru $\lg^{1/2}n=\Omega(\lg\lg n)$.
Przyjmując $n=2^{4^h}$, sprowadzamy go do postaci $2^h=\Omega(h)$, co oczywiście zachodzi.
\item $\lg^2n=\Omega(\ln n)$ \\
	Zależność jest prawdziwa, ponieważ $\ln n=\Theta(\lg n)$ oraz $\lg n=\omega(1)$.
\item $\ln n=\Omega\bigl(\!\sqrt{\lg n}\bigr)$ \\
	Wystarczy przyjąć $n=e^h$, aby otrzymać tożsamość $h=\Omega\bigl(\!\sqrt{h}\bigr)$, która zachodzi na mocy tego, że $\sqrt{h}=\omega(1)$.
\item $\sqrt{\lg n}=\Omega(\ln\ln n)$ \\
	Prawa strona jest identyczna z~$\Omega(\lg\lg n)$, zatem przyjmując $n=2^{4^h}$, dostajemy tożsamość $2^h=\Omega(h)$.
\item $\ln\ln n=\Omega\bigl(2^{\lg^*n}\bigr)$ \\
	Logarytmując funkcje, dostajemy
	\[
		\lg\ln\ln n = \Theta\bigl(\lg^{(3)}n\bigr) \quad\text{oraz}\quad \lg\bigl(2^{\lg^*n}\bigr) = \lg^*n.
	\]
	By wykazać prawdziwość tożsamości, dokonajmy podstawienia
	\[
		n = 2^{2^{\cdot^{\cdot^{\cdot^{2^\epsilon}}}}}\vbox{\hbox{$\Big\}\scriptstyle k$}\kern0pt},
	\]
	gdzie $0<\epsilon\le1$, a~$k\ge3$ jest liczbą dwójek.
Wyliczając wartości obu funkcji dla takiego argumentu, otrzymujemy
	\[
		\lg^{(3)}n = 2^{2^{\cdot^{\cdot^{\cdot^{2^\epsilon}}}}}\vbox{\hbox{$\Big\}\scriptstyle k-3$}\kern0pt} \quad\text{oraz}\quad \lg^*n=k.
	\]
	Oczywistym jest, że funkcja po lewej stronie jest asymptotycznie większa od funkcji po prawej stronie.
\item $2^{\lg^*n}=\Omega(\lg^*n)$ \\
	Po zlogarytmowaniu obu funkcji i~wykorzystaniu wzoru (3.15), otrzymujemy
	\[
		\lg2^{\lg^*n} = \lg^*n \quad\text{oraz}\quad \lg\lg^*n.
	\]
	Biorąc $h=\lg^*n$, sprowadzamy tożsamość do udowodnionej wcześniej $h=\Omega(\lg h)$, a~zatem dowodzona zależność jest spełniona.
\item $\lg^*n=\Omega(\lg^*\lg n)$ \\
	Funkcje są asymptotycznie równoważne, ponieważ $\lg^*\lg n=\lg^*n-1$ dla $n\ge2$.
\item $\lg^*\lg n=\Omega(\lg\lg^*n)$ \\
	Tożsamość zachodzi na podstawie rozwiązania \refExercise{3.2-5}.
\item $\lg\lg^*n=\Omega\bigl(n^{1/\!\lg n}\bigr)$ \\
	Z~własności logarytmów mamy, że $1/\!\lg n=\log_n2$, a~więc wykorzystując wzór (3.15), dostajemy $n^{1/\!\lg n}=n^{\log_n2}=2^{\log_nn}=2=\Theta(1)$, skąd wynika prawdziwość zależności.
\item $n^{1/\!\lg n}=\Omega(1)$ \\
	Tożsamość zachodzi, gdyż z~poprzedniego uzasadnienia $n^{1/\!\lg n}=\Theta(1)$.
\end{itemize}

Tabela \ref{tab:3-3} przedstawia badane funkcje uporządkowane względem notacji $\Omega$ na podstawie powyższych dowodów.
\begin{table}[ht]
	\begin{center}
		\[
			\begin{array}{|lc|lc|lc|} \hline
				g_1(n)= & 2^{2^{n+1}} & g_{11}(n)= & (\lg n)! & g_{21}(n)= & \lg^2n \\ \hline
				g_2(n)= & 2^{2^n} & g_{12}(n)= & n^3 & g_{22}(n)= & \ln n \\ \hline
				g_3(n)= & (n+1)! & g_{13}(n)= & n^2 & g_{23}(n)= & \sqrt{\lg n} \\ \cline{1-2}\cline{5-6}
				g_4(n)= & n! & g_{14}(n)= & 4^{\lg n} & g_{24}(n)= & \ln\ln n \\ \hline
				g_5(n)= & e^n & g_{15}(n)= & n\lg n & g_{25}(n)= & 2^{\lg^*n} \\ \cline{1-2}\cline{5-6}
				g_6(n)= & n\cdot2^n & g_{16}(n)= & \lg(n!) & g_{26}(n)= & \lg^*n \\ \cline{1-4}
				g_7(n)= & 2^n & g_{17}(n)= & n & g_{27}(n)= & \lg^*\lg n \\ \cline{1-2}\cline{5-6}
				g_8(n)= & (3/2)^n & g_{18}(n)= & 2^{\lg n} & g_{28}(n)= & \lg\lg^*n \\ \hline
				g_9(n)= & n^{\lg\lg n} & g_{19}(n)= & \bigl(\!\sqrt{2}\bigr)^{\lg n} & g_{29}(n)= & n^{1/\!\lg n} \\ \cline{3-4}
				g_{10}(n)= & (\lg n)^{\lg n} & g_{20}(n)= & 2^{\sqrt{2\lg n}} & g_{30}(n)= & 1 \\ \hline
			\end{array}
		\]
	\end{center}
	\caption{Uporządkowanie funkcji względem asymptotycznego tempa wzrostu.
Funkcje znajdujące się w~tej samej komórce są asymptotycznie równoważne.} \label{tab:3-3}
\end{table}

\subproblem %3-3(b)
Oto przykład funkcji, która nie jest ani $O(g_i(n))$, ani $\Omega(g_i(n))$, gdzie $i=1$, 2, \dots, 30:
\[
	f(n) =
	\begin{cases}
		2^{2^{n+2}}, & \text{jeśli $n$ jest parzyste}, \\
		0, & \text{jeśli $n$ jest nieparzyste}.
	\end{cases}
\]
Gdyby rozważać funkcję $f(n)$ w~dziedzinie liczb parzystych, to byłaby ona $\Omega(g_1(n))$, z~kolei po obcięciu dziedziny do liczb nieparzystych, $f(n)$ byłoby na końcu listy funkcji w~uporządkowaniu z~poprzedniego punktu.
Dlatego wraz ze wzrostem $n$, w~zależności od jego parzystości, $f(n)$ jest asymptotycznie większe od wszystkich funkcji $g_i(n)$ albo od nich asymptotycznie mniejsze.

\problem{Własności notacji asymptotycznej} %3-4

\subproblem %3-4(a)
Fałsz.
Niech np.\ $f(n)=n$ i~$g(n)=n^2$.
Wtedy $f(n)=O(g(n))$, ale $g(n)\ne O(f(n))$.

\subproblem %3-4(b)
Fałsz.
Jako kontrprzykład rozważmy $f(n)=n$ i~$g(n)=n^2$.
Dla $n\ge1$ zachodzi wtedy
\[
	\min(f(n),g(n)) = f(n) \quad\text{oraz}\quad f(n)+g(n) = \Theta(g(n)) \ne \Theta(f(n)).
\]

\subproblem %3-4(c)
Prawda.
Z~faktu, że $f(n)=O(g(n))$, wynika $f(n)\le cg(n)$ dla $n\ge n_0$, gdzie $c$, $n_0>0$ są pewnymi stałymi.
Z~założenia, że $\lg g(n)\ge1$ i~$f(n)\ge1$, otrzymujemy
\[
	0 \le \lg f(n) \le \lg c+\lg g(n) \le \lg c\lg g(n)+\lg g(n) = (\lg c+1)\lg g(n) = O(\lg g(n)).
\]

\subproblem %3-4(d)
Fałsz.
Niech $f(n)=2n$ oraz $g(n)=n$.
Zachodzi $f(n)=O(g(n))$, jednak $2^{f(n)}\ne O\bigl(2^{g(n)}\bigr)$ (z~\refExercise{3.1-4}).

\subproblem %3-4(e)
Fałsz.
Dla $f(n)=1/n$ mamy $f^2(n)=1/n^2$.
Ponieważ nie istnieje żadna dodatnia stała $c$ spełniająca nierówność $1/n\le c/n^2$ dla dowolnie dużych $n$, to stąd $f(n)\ne O\bigl(f^2(n)\bigr)$.

\subproblem %3-4(f)
Prawda.
Z~definicji notacji $O$, jeśli $f(n)=O(g(n))$, to istnieją stałe $c$, $n_0>0$, że dla każdego $n\ge n_0$ zachodzi $0\le f(n)\le cg(n)$.
Dzieląc nierówność przez $c$, otrzymujemy $0\le f(n)/c\le g(n)$, przy czym $1/c>0$, a~więc $g(n)=\Omega(f(n))$.

\subproblem %3-4(g)
Fałsz.
Niech np.\ $f(n)=2^n$.
Wtedy $f(n/2)=2^{n/2}=\sqrt{2^n}$.
Gdyby zachodziło $2^n=O\bigl(\!\sqrt{2^n}\bigr)$, to dla pewnej stałej $c>0$ i~odpowiednio dużych $n$ mielibyśmy $2^n\le c\sqrt{2^n}$.
Ale wówczas $c\ge\sqrt{2^n}$ nie mogłoby być stałą.

\subproblem %3-4(h)
Prawda.
Niech $h(n)=o(f(n))$.
Wtedy, na podstawie definicji notacji $o$ mamy, że dla każdej stałej $c>0$ istnieje stała $n_0>0$ taka, że nierówności $0\le h(n)<cf(n)$ zachodzą dla wszystkich $n\ge n_0$.
To znaczy, że
\[
	f(n) \le f(n)+o(f(n)) = f(n)+h(n) < (c+1)f(n).
\]
Ponieważ $c+1>1$, to można wyrażenie $f(n)+o(f(n))$ ograniczyć od góry przez $c_2f(n)$, wybierając np.\ $c_2=2$.
Jego dolnym ograniczeniem jest $f(n)$, więc ustalamy $c_1=1$.
Stałe $c_1$, $c_2$, $n_0$ spełniają założenia definicji notacji $\Theta$, więc wnioskujemy, że $f(n)+o(f(n))=\Theta(f(n))$.

\problem{Wariacje na temat notacji $O$ i~$\Omega$} %3-5

\subproblem %3-5(a)
Niech $c>0$ będzie pewną stałą.
Załóżmy, że $f(n)\ge cg(n)$ dla $n$ z~pewnego skończonego zbioru, w~przeciwnym razie zachodzi bowiem $f(n)=\overset{\infty}{\Omega}(g(n))$.
Ponieważ zbiór ten jest skończony, to wybierzmy największe $n$, dla którego nierówność ta jest spełniona i~oznaczmy je przez $n_0$.
Mamy zatem $0\le f(n)<cg(n)$ dla $n\ge n_0+1$, czyli $f(n)=O(g(n))$.

Natomiast nie jest prawdą podobne twierdzenie, gdyby zastosować notację $\Omega$ zamiast $\overset{\infty}{\Omega}$ -- jeśli np.\ $f(n)=n$ oraz $g(n)=n^{1+\sin n}$, to $f(n)=\overset{\infty}{\Omega}(g(n))$, ale $f(n)\ne\Omega(g(n))$ i~$f(n)\ne O(g(n))$.

\subproblem %3-5(b)
Zaletą nowej notacji jest fakt, że jeśli o~pewnej funkcji $f(n)$ wiemy, że nie jest klasy $O(g(n))$, to jest klasy $\overset{\infty}{\Omega}(g(n))$ (z~poprzedniego punktu) -- nie istnieją zatem funkcje, których nie da się porównać z~innymi za pomocą pary notacji $O$ i~$\overset{\infty}{\Omega}$.

Niestety, jeśli $f(n)=\overset{\infty}{\Omega}(g(n))$, to funkcja $f(n)$ niekoniecznie dominuje nad funkcją $g(n)$, gdyż w~nieskończenie wielu punktach może przyjmować wartości mniejsze od wartości $g(n)$.
Nieskończenie wiele punktów, o~których mowa w~definicji $\overset{\infty}{\Omega}$, może też występować daleko ponad maksymalnym rozmiarem danych wejściowych algorytmu, którego czas działania opisujemy przy użyciu nowej notacji.
Wady te sprawiają, że nowa notacja ma niewielkie zastosowanie praktyczne i~jest użyteczna głównie w~rozważaniach teoretycznych.

\subproblem %3-5(c)
W~przypadku implikacji w~lewą stronę z~warunku $f(n)=\Theta(g(n))$ wynika, że $f(n)\ge0$ od pewnego dodatniego $n_0$, a~więc $f(n)$ jest funkcją asymptotycznie nieujemną i~twierdzenie stosuje się bez zmian, czyli implikacja zachodzi.

Zbadajmy teraz implikację w~prawo.
Wprost z~definicji, jeśli $f(n)=O'(g(n))$, to istnieją takie stałe $c$, $n_0>0$, że dla wszystkich $n\ge n_0$ zachodzi
\[
	\begin{cases}
		0 \le f(n) \le cg(n), & \text{jeśli $f(n)\ge0$}, \\
		0 \le -f(n) \le cg(n), & \text{jeśli $f(n)<0$}.
	\end{cases}
\]
Załóżmy, że istnieje nieskończenie wiele takich $n\ge n_0$, że $f(n)<0$, ponieważ w~przeciwnym przypadku mielibyśmy do czynienia z~funkcją asymptotycznie nieujemną, dla której jest $f(n)=O(g(n))$.
Teraz jednak nierówności $0\le c_1g(n)\le f(n)$ wynikające z~założenia, że $f(n)=\Omega(g(n))$ nie są spełnione dla nieskończenie wielu $n\ge n_0$ i~dowolnej stałej $c_1>0$.
Wynika stąd wniosek, że jeśli spełnione są założenia, to funkcja $f(n)$ jest asymptotycznie nieujemna i~implikacja stosuje się bez zmian.

Widać, że zastosowanie notacji $O'$ w~miejsce $O$ nie pozbawia prawdziwości twierdzenia 3.1.

\subproblem %3-5(d)
Oto definicje notacji $\widetilde{\Omega}$ i~$\widetilde{\Theta}$:
\[
	\begin{split}
		\widetilde{\Omega}(g(n)) &= \bigl\{\,f(n):\text{istnieją dodatnie stałe $c$, $k$, $n_0$ takie, że} \\
		&\qquad 0 \le cg(n)\lg^kn \le f(n) \text{ dla wszystkich $n \ge n_0$}\,\bigr\}, \\
		\widetilde{\Theta}(g(n)) &= \bigl\{\,f(n):\text{istnieją dodatnie stałe $c_1$, $c_2$, $k_1$, $k_2$, $n_0$ takie, że} \\
		&\qquad 0 \le c_1g(n)\lg^{k_1}n \le f(n) \le c_2g(n)\lg^{k_2}n \text{ dla wszystkich $n \ge n_0$}\,\bigr\}.
	\end{split}
\]

Dowód twierdzenia 3.1 dla notacji $\widetilde{O}$, $\widetilde{\Omega}$ i~$\widetilde{\Theta}$ przebiega analogicznie do dowodu jego oryginalnego odpowiednika przeprowadzonego w~\refExercise{3.1-5}.
Z~definicji notacji $\widetilde{\Theta}$ mamy, że $f(n)=\widetilde{\Theta}(g(n))$ wtedy i~tylko wtedy, gdy istnieją takie stałe $c_1$, $c_2$, $k_1$, $k_2$, $n_0>0$, że nierówności
\[
	0 \le c_1g(n)\lg^{k_1}n \le f(n) \le c_2g(n)\lg^{k_2}n
\]
zachodzą dla wszystkich $n\ge n_0$.
Zapiszmy je w~postaci układu
\[
	\begin{cases}
		0 \le c_1g(n)\lg^{k_1}n \le f(n) \\
		0 \le f(n) \le c_2g(n)\lg^{k_2}n
	\end{cases}.
\]
Z~pierwszej nierówności mamy, że $f(n)=\widetilde{\Omega}(g(n))$, a~z~drugiej, że $f(n)=\widetilde{O}(g(n))$.

\problem{Funkcje iterowane} %3-6
W~każdym punkcie tego problemu za dziedzinę funkcji $f_c^*(n)$ przyjęto dziedzinę $f(n)$.
W~celu wyznaczenia $f_c^*(n)$ szukamy najmniejszego $i\ge0$, dla którego zachodzi $f^{(i)}(n)\le c$.

\subproblem %3-6(a)
\note{Rozwiązanie dotyczy przykładu z~tekstu oryginalnego, w~którym\/ $f(n)=n-1$ oraz\/ $c=0$.}

\noindent Ponieważ $(n-1)^{(i)}\equiv n-i$, to otrzymujemy, że $f_0^*(n)=\max(0,\lceil n\rceil)$ i~oszacowaniem dokładnym jest $f_0^*(n)=\Theta(n)$.

\subproblem %3-6(b)
\note{Rozwiązanie dotyczy przykładu z~tekstu oryginalnego, w~którym\/ $f(n)=\lg n$ oraz\/ $c=1$.}

\noindent Z~definicji logarytmu iterowanego mamy $f_1^*(n)=\lg^*n=\Theta(\lg^*n)$.

\subproblem %3-6(c)
Oczywiście $(n/2)^{(i)}\equiv n/2^i$, więc:
\[
	f_1^*(n) =
	\begin{cases}
		0, & \text{jeśli $n\le1$}, \\
		\lceil\lg n\rceil, & \text{jeśli $n>1$}.
	\end{cases}
\]
Oszacowanie dokładne: $f_1^*(n)=\Theta(\lg n)$.

\subproblem %3-6(d)
Korzystając z~poprzedniego punktu, ale dla $c=2$, mamy:
\[
	f_2^*(n) =
	\begin{cases}
		0, & \text{jeśli $n\le2$}, \\
		\lceil\lg n\rceil-1, & \text{jeśli $n>2$}.
	\end{cases}
\]
Oszacowanie dokładne: $f_2^*(n)=\Theta(\lg n)$.

\subproblem %3-6(e)
Ponieważ $\sqrt{n}\equiv n^{1/2}$, to mamy $\bigl(\!\sqrt{n}\bigr)^{(i)}\equiv n^{1/2^i}$\!.
Jeśli $n>2$, to rozwiązaniem nierówności $n^{1/2^i}\le2$ ze względu na $i$ jest $i\ge\lg\lg n$, skąd dostajemy następujący wynik:
\[
	f_2^*(n) =
	\begin{cases}
		0, & \text{jeśli $0\le n\le2$}, \\
		\lceil\lg\lg n\rceil, & \text{jeśli $n>2$}.
	\end{cases}
\]
Oszacowanie dokładne: $f_2^*(n)=\Theta(\lg\lg n)$.

\subproblem %3-6(f)
Bieżący punkt różni się od poprzedniego jedynie stałą $c$, jednak ta zmiana mocno wpływa na postać funkcji $f_c^*(n)$.
Jeśli $0\le n\le1$, to $f_1^*(n)=0$, a~jeśli $n>1$, to wartości tej funkcji są nieokreślone, ponieważ dla każdego całkowitego $i\ge0$, $\bigl(\!\sqrt{n}\bigr)^{(i)}>1$.

\subproblem %3-6(g)
Postępowanie jest analogiczne do tego z~punktu (e).
Mamy $\bigl(n^{1/3}\bigr)^{(i)}\equiv n^{1/3^i}$, a~stąd:
\[
	f_2^*(n) =
	\begin{cases}
		0, & \text{jeśli $n\le2$}, \\
		\lceil\log_3\lg n\rceil, & \text{jeśli $n>2$}.
	\end{cases}
\]
Oszacowanie dokładne: $f_2^*(n)=\Theta(\lg\lg n)$.

\subproblem %3-6(h)
Kolejne iteracje funkcji $f(n)=n/\!\lg n$ mają zbyt skomplikowaną postać, aby można było je badać podobnie jak w~poprzednich punktach.
Dlatego naszą analizę przeprowadzimy dla oszacowań górnego i~dolnego funkcji $f(n)$.
Niech $g(n)$ i~$h(n)$ będą funkcjami monotonicznie rosnącymi.
Dla $n\ge c$, jeśli $f(n)\le g(n)$ oraz $f_c^*(n)$ i~$g_c^*(n)$ są dobrze określone, to $f_c^*(n)\le g_c^*(n)$, gdyż argument w~kolejnych iteracjach funkcji $f(n)$ maleje szybciej niż w~iteracjach funkcji $g(n)$.
Analogiczny wniosek można uzyskać w~przypadku, gdy $f(n)\ge h(n)$.

Jeśli $n\ge4$, to zachodzi $n/\!\lg n\le n/2$.
Niech $g(n)=n/2$.
Wówczas z~punktu (d) mamy, że $g_4^*(n)=g_2^*(n)-1=\Theta(\lg n)$, a~zatem $f_2^*(n)=f_4^*(n)+1\le g_4^*(n)+1=O(\lg n)$.

Jeśli z~kolei $n\ge16$, to prawdziwa jest nierówność $n/\!\lg n\ge\sqrt{n}$.
Biorąc $h(n)=\sqrt{n}$ i~wykorzystując wynik z~punktu (e), otrzymujemy $h_{16}^*(n)=h_2^*(n)-2=\Theta(\lg\lg n)$, a~zatem $f_2^*(n)=f_{16}^*(n)+2\ge h_{16}^*(n)+2=\Omega(\lg\lg n)$.

Tempo wzrostu funkcji $f_2^*(n)$ znajduje się zatem pomiędzy $\Omega(\lg\lg n)$ a~$O(\lg n)$.
Niestety na podstawie przeprowadzonej analizy nie można podać dokładniejszego oszacowania.

\endinput
