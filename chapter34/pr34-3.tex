\problem{Kolorowanie grafu} %34-3

\subproblem %34-3(a)
Przechodzimy wszerz graf $G$, rozpoczynając od dowolnego jego wierzchołka jako źródła.
Wierzchołki kolorujemy zgodnie z~parzystością ich odległości od źródła.
Jeśli sąsiad przetwarzanego wierzchołka ma już przypisany kolor, którym miał być pokolorowany ten wierzchołek, to znaczy, że \singledash{2}{kolorowanie} grafu nie istnieje.
Algorytm działa w~czasie $O(V+E)$.

\subproblem %34-3(b)
\[
	\text{COLOR} = \bigl\{\,\langle G,k\rangle:\text{$k\ge0$ jest liczbą całkowitą oraz graf $G$ jest \singledash{$k$}{kolorowalny}}\,\bigr\}
\]

Pierwsza implikacja jest oczywista -- jeśli pewne \singledash{$k$}{kolorowanie} grafu $G$ jesteśmy w~stanie otrzymać w~czasie wielomianowym, to jednocześnie w~czasie wielomianowym rozwiązujemy problem COLOR.

\subproblem %34-3(c)
Oba problemy są w~\NPclass, dla świadectwa będącego listą kolorów wierzchołków sprawdzamy, czy kolorowanie to jest poprawne, tzn.\ czy każde dwa sąsiednie wierzchołki mają różne kolory.

Na mocy tego, że problem \singledash{3}{COLOR} jest szczególnym przypadkiem problemu COLOR, to z~\singledash{\NPclass}{trudności} ostatniego wynika \singledash{\NPclass}{trudność} pierwszego.

\subproblem %34-3(d)
\subproblem %34-3(e)
\subproblem %34-3(f)
