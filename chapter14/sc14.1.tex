\subchapter{Dynamiczne statystyki pozycyjne}

\exercise %14.1-1
Algorytm rozpoczyna działanie od $i=10$ oraz ze zmienną $x$ pokazującą na korzeń drzewa $T$.
Ranga klucza 26 wyznaczona w~linii 1 wynosi 13, więc algorytm zostaje wywołany rekurencyjnie dla lewego poddrzewa, czyli dla węzła o~kluczu 17.
W~wywołaniu tym obliczona ranga klucza 17 wynosi 8.
Szukany element jest więc $10-8=2$ (drugim) największym elementem w~prawym poddrzewie aktualnego węzła.
Po wywołaniu rekurencyjnym zmienna $x$ wskazuje na jeden z~węzłów o~kluczu 21, ten na głębokości 2, a~$i=2$.
Ranga tego klucza zostaje wyznaczona na 3, dlatego szukany klucz należy do lewego poddrzewa.
W~kolejnym wywołaniu mamy $x$ wskazujące na węzeł o~kluczu 19 oraz $i=2$.
Tym razem jednak lewe poddrzewo jest puste, ale zdefiniowanie \attribb{T}{nil}{size} jako 0 pozwala obliczyć rangę obecnego elementu jako 1.
Algorytm wywoływany jest więc rekurencyjnie jeszcze raz w~celu znalezienia $2-1=1$ (pierwszego) elementu w~prawym poddrzewie.
Po wyznaczeniu rangi i~porównaniu z~wartością zmiennej $i=1$ zwracany jest wskaźnik do węzła o~kluczu 20.

\exercise %14.1-2
\exercise %14.1-3
\exercise %14.1-4
\exercise %14.1-5
\exercise %14.1-6
\exercise %14.1-7
\exercise %14.1-8
