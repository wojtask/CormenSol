\subchapter{Jak wzbogacać strukturę danych}

\exercise %14.2-1
Aby efektywnie wykonywać podane operacje, wzbogacimy każdy węzeł drzewa statystyk pozycyjnych o~4 pola wskaźnikowe:
\begin{itemize}
	\item \id{min} -- wskaźnik na węzeł o~najmniejszym kluczu w~drzewie,
	\item \id{max} -- wskaźnik na węzeł o~największym kluczu w~drzewie,
	\item \id{pred} -- wskaźnik na poprzednika danego węzła oraz
	\item \id{succ} -- wskaźnik na następnika danego węzła.
\end{itemize}
W~szczególności, jeśli węzeł $x$ ma najmniejszy klucz w~drzewie $T$, to $\attrib{x}{min}=x$ i~$\attrib{x}{pred}=\attrib{T}{nil}$.

Wartość każdego nowego pola można wyznaczyć na podstawie jedynie wartości tych pól w~synach danego węzła:
\begin{itemize}
	\item $\attrib{x}{min}=\attribb{x}{left}{min}=\attribb{x}{right}{min}$;
	\item $\attrib{x}{max}=\attribb{x}{left}{max}=\attribb{x}{right}{max}$;
	\item $\attrib{x}{pred}=\attribb{x}{left}{max}$;
	\item $\attrib{x}{succ}=\attribb{x}{right}{min}$.
\end{itemize}
Zatem zgodnie z~tw.\ 14.1 można zachować poprawne wartości tych pól podczas wstawiania i~usuwania, nie zwiększając asymptotycznej złożoności tych operacji.
Operacje $\proc{Minimum}(T)$ i~$\proc{Maximum}(T)$ polegają na zwróceniu, odpowiednio, \attribb{T}{root}{min} i~\attribb{T}{root}{max}, natomiast operacje $\proc{Predecessor}(T,x)$ i~$\proc{Successor}(T,x)$ polegają na zwróceniu, odpowiednio, \attrib{x}{pred} i~\attrib{x}{succ}.

\exercise %14.2-2
Jeśli czarną wysokość wewnętrznego węzła $x$ w~drzewie czerwono-czarnym $T$ przechowamy w~polu \attrib{x}{bh}, a~ponadto zdefiniujemy $\attribb{T}{nil}{bh}=0$, to zachodzi następująca zależność:
\[
	\attrib{x}{bh} = \begin{cases}
		\attribb{x}{left}{bh}, & \text{jeśli $\attribb{x}{left}{color}=\const{red}$}, \\
		\attribb{x}{left}{bh}+1, & \text{jeśli $\attribb{x}{left}{color}=\const{black}$}.
	\end{cases}
\]
A~zatem na mocy tw.\ 14.1 mamy, że złożoność asymptotyczna operacji na tak wzbogaconym drzewie czerwono-czarnym nie ulegnie zmianie.

\exercise %14.2-3
\exercise %14.2-4
\exercise %14.2-5
