\subchapter{Jak wzbogacać strukturę danych}

\exercise %14.2-1
Aby efektywnie wykonywać podane operacje, wzbogacimy każdy węzeł drzewa statystyk pozycyjnych o~4 pola wskaźnikowe:
\begin{itemize}
	\item \id{min} -- wskaźnik na węzeł o~najmniejszym kluczu w~drzewie,
	\item \id{max} -- wskaźnik na węzeł o~największym kluczu w~drzewie,
	\item \id{pred} -- wskaźnik na poprzednika danego węzła oraz
	\item \id{succ} -- wskaźnik na następnika danego węzła.
\end{itemize}
W~szczególności, jeśli węzeł $x$ ma najmniejszy klucz w~drzewie $T$, to $\attrib{x}{min}=x$ i~$\attrib{x}{pred}=\attrib{T}{nil}$.

Wartość każdego nowego pola można wyznaczyć na podstawie jedynie wartości tych pól w~synach danego węzła:
\begin{itemize}
	\item $\attrib{x}{min}=\attribb{x}{left}{min}=\attribb{x}{right}{min}$;
	\item $\attrib{x}{max}=\attribb{x}{left}{max}=\attribb{x}{right}{max}$;
	\item $\attrib{x}{pred}=\attribb{x}{left}{max}$;
	\item $\attrib{x}{succ}=\attribb{x}{right}{min}$.
\end{itemize}
Zatem zgodnie z~tw.\ 14.1 można zachować poprawne wartości tych pól podczas wstawiania i~usuwania, nie zwiększając asymptotycznej złożoności tych operacji.
Operacje $\proc{Minimum}(T)$ i~$\proc{Maximum}(T)$ polegają na zwróceniu, odpowiednio, \attribb{T}{root}{min} i~\attribb{T}{root}{max}, natomiast operacje $\proc{Predecessor}(T,x)$ i~$\proc{Successor}(T,x)$ polegają na zwróceniu, odpowiednio, \attrib{x}{pred} i~\attrib{x}{succ}.

\exercise %14.2-2
Jeśli czarną wysokość wewnętrznego węzła $x$ w~drzewie czerwono-czarnym $T$ przechowamy w~polu \attrib{x}{bh}, a~ponadto zdefiniujemy $\attribb{T}{nil}{bh}=0$, to zachodzi następująca zależność:
\[
	\attrib{x}{bh} = \begin{cases}
		\attribb{x}{left}{bh}, & \text{jeśli $\attribb{x}{left}{color}=\const{red}$}, \\
		\attribb{x}{left}{bh}+1, & \text{jeśli $\attribb{x}{left}{color}=\const{black}$}.
	\end{cases}
\]
A~zatem na mocy tw.\ 14.1 mamy, że złożoność asymptotyczna operacji na tak wzbogaconym drzewie czerwono-czarnym nie ulegnie zmianie.

\exercise %14.2-3
Nie można tego zrobić przy zachowaniu efektywnych czasów działania operacji na drzewie, ponieważ głębokość węzła zależy od głębokości jego ojca.
Gdy zmienia się głębokość węzła $x$, to zmieniają się także głębokości wszystkich potomków $x$.
A~zatem aktualizacja głębokości korzenia drzewa niesie za sobą konieczność aktualizacji pozostałych $n-1$ węzłów drzewa i~operacje wstawiania i~usuwania działają wtedy w~czasie $O(n\lg n)$.

\exercise %14.2-4
Posługując się rys.\ 13.2 z~Podręcznika, oznaczmy przez $r_\alpha$, $r_\beta$, $r_\gamma$ korzenie poddrzew, odpowiednio, $\alpha$, $\beta$, $\gamma$.
Ponieważ operacja $\otimes$ jest łączna, to mamy
\begin{align*}
	\attrib{x}{f} &= \attrib{r_\alpha}{f}\otimes\attrib{x}{a}\otimes\attrib{r_\beta}{f}\otimes\attrib{y}{a}\otimes\attrib{r_\gamma}{f}, \\
	\attrib{y}{f} &= \phantom{\attrib{r_\alpha}{f}\otimes\attrib{x}{a}\otimes{}}\attrib{r_\beta}{f}\otimes\attrib{y}{a}\otimes\attrib{r_\gamma}{f}.
\end{align*}
Rotacje nie zmieniają porządku inorder węzłów w~żadnym z~poddrzew $\alpha$, $\beta$ i~$\gamma$, dlatego po przeprowadzeniu lewej rotacji wartości pola $f$ wynoszą
\begin{align*}
	\attrib{x}{f} &= \attrib{r_\alpha}{f}\otimes\attrib{x}{a}\otimes\attrib{r_\beta}{f}, \\
	\attrib{y}{f} &= \attrib{r_\alpha}{f}\otimes\attrib{x}{a}\otimes\attrib{r_\beta}{f}\otimes\attrib{y}{a}\otimes\attrib{r_\gamma}{f}
\end{align*}
i~mogą zostać obliczone w~czasie $O(1)$.
Rozumowanie w~przypadku prawej rotacji przeprowadza się analogicznie.

W~drzewie czerwono-czarnym $T$ zdefiniujmy teraz dla każdego węzła pole $a$ przyjmujące wartość 0 dla liści drzewa (reprezentowanych przez \attrib{T}{nil}) oraz 1 dla jego węzłów wewnętrznych.
Jeśli operacją $\otimes$ będzie zwykłe dodawanie, to wartość \attrib{x}{f} będzie rozmiarem poddrzewa o~korzeniu w~$x$, czyli pole $f$ będzie identyczne z~polem \id{size} z~drzew statystyk pozycyjnych.
Dzięki powyżej przedstawionej argumentacji pole \id{size} może być aktualizowane w~czasie $O(1)$ po każdym wykonaniu rotacji w~drzewie $T$.

\exercise %14.2-5
