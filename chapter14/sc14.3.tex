\subchapter{Drzewa przedziałowe}

\exercise %14.3-1
\exercise %14.3-2
Pseudokod procedury \proc{Interval-Search} nie zmieni się.
Jedyna różnica będzie ukryta w~linii 2 tej procedury, a~dokładniej w~definicji zachodzenia przedziałów.
Powiemy, że dwa przedziały otwarte $i$ oraz $i'$ zachodzą na siebie, jeśli $i\cap i'\ne\emptyset$, tj.\ jeśli $\attrib{i}{low}<\attrib{i'}{high}$ i~$\attrib{i'}{low}<\attrib{i}{high}$.

\exercise %14.3-3
\exercise %14.3-4
\exercise %14.3-5
\exercise %14.3-6
\exercise %14.3-7
