\subchapter{Drzewa przedziałowe}

\exercise %14.3-1
\exercise %14.3-2
Pseudokod procedury \proc{Interval-Search} nie zmieni się.
Jedyna różnica będzie ukryta w~linii 2 tej procedury w~definicji zachodzenia przedziałów.
Powiemy, że dwa przedziały otwarte $i$ oraz $i'$ zachodzą na siebie, jeśli $i\cap i'\ne\emptyset$, tj.\ jeśli $\attrib{i}{low}<\attrib{i'}{high}$ i~$\attrib{i'}{low}<\attrib{i}{high}$.

\exercise %14.3-3
\exercise %14.3-4
\exercise %14.3-5
\exercise %14.3-6
Strukturę danych $Q$ zaimplementujemy w~postaci odpowiednio wzbogaconego drzewa czerwono-czarnego.
W~każdym węźle $x$ drzewa $Q$ przechowamy dodatkowe pola:
\begin{itemize}
	\item \attrib{x}{min-key} -- równe najmniejszemu kluczowi w~poddrzewie o~korzeniu w~$x$;
	\item \attrib{x}{max-key} -- równe największemu kluczowi w~poddrzewie o~korzeniu w~$x$;
	\item \attrib{x}{min-gap} -- równe najmniejszej odległości między kluczami w~poddrzewie o~korzeniu w~$x$.
\end{itemize}
Wszystkie nowe pola każdego wewnętrznego węzła $x$ są zależne od innych pól węzła $x$ oraz od pól węzłów \attrib{x}{left} i~\attrib{x}{right}:
\begin{align*}
	\attrib{x}{min-key} &= \min(\attribb{x}{left}{min-key},\attrib{x}{key},\attribb{x}{right}{min-key}), \\
	\attrib{x}{max-key} &= \max(\attribb{x}{left}{max-key},\attrib{x}{key},\attribb{x}{right}{max-key}), \\
	\attrib{x}{min-gap} &= \min(\attribb{x}{left}{min-gap},\attribb{x}{right}{min-gap}, \\
		&\phantom{{}=\min(}\attrib{x}{key}-\attribb{x}{left}{max-key},\attribb{x}{right}{min-key}-\attrib{x}{key}).
\end{align*}
Definiujemy ponadto $\attribb{Q}{nil}{min-key}=\attribb{Q}{nil}{min-gap}=\infty$ oraz $\attribb{Q}{nil}{max-key}=-\infty$.

Wywołanie $\proc{Min-Gap}(Q)$ polega na zwróceniu wartości \attribb{Q}{root}{min-gap} i~działa w~czasie $O(1)$.
Oczywiście wzbogacenie drzewa o~nowe pola nie zmienia działania operacji \proc{Search}.
Dzięki zastosowaniu tw.\ 14.1 mamy z~kolei, że wprowadzenie nowych pól nie zwiększa asymptotycznej złożoności czasowej operacji \proc{Insert} i~\proc{Delete}.

\exercise %14.3-7
