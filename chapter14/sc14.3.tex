\subchapter{Drzewa przedziałowe}

\exercise %14.3-1
Procedura \proc{Interval-Left-Rotate} dla drzew przedziałowych działa identycznie jak procedura \proc{Left-Rotate} dla drzew czerwono-czarnych, z~tą różnicą, że tuż przed zakończeniem działania modyfikuje pola \attrib{x}{max} i~\attrib{y}{max}, gdzie $y$ jest nowym ojcem $x$ po przeprowadzeniu rotacji.
Modyfikacje te polegają najpierw na aktualizacji \attrib{x}{max} wprost z~definicji pola \id{max}, a~następnie analogicznym zaktualizowaniu \attrib{y}{max}, które po wykonaniu rotacji zależy od \attrib{x}{max}.
\begin{codebox}
\Procname{$\proc{Interval-Left-Rotate}(T,x)$}
\li	$\proc{Left-Rotate}(T,x)$
\li	$y\gets\attrib{x}{p}$
\li	$\attrib{x}{max}\gets\max(\attribb{x}{int}{high},\attribb{x}{left}{max},\attribb{x}{right}{max})$
\li	$\attrib{y}{max}\gets\max(\attribb{y}{int}{high},\attrib{x}{max},\attribb{y}{right}{max})$
\end{codebox}

\exercise %14.3-2
Pseudokod procedury \proc{Interval-Search} nie zmieni się.
Jedyna różnica będzie ukryta w~linii 2 tej procedury w~definicji zachodzenia przedziałów.
Powiemy, że dwa przedziały otwarte $i$ oraz $i'$ zachodzą na siebie, jeśli $i\cap i'\ne\emptyset$, tj.\ jeśli $\attrib{i}{low}<\attrib{i'}{high}$ i~$\attrib{i'}{low}<\attrib{i}{high}$.

\exercise %14.3-3
Algorytm wyszukuje w~drzewie przedziałowym $T$ dowolny węzeł $x$, taki że przedział \attrib{x}{int} zachodzi na przedział $i$, wywołując w~tym celu procedurę \proc{Interval-Search}.
Jeśli przedział taki istnieje, czyli $x\ne\attrib{T}{nil}$, to począwszy od $x$, algorytm przechodzi w~dół drzewa przedziałowego w~poszukiwaniu przedziału zachodzącego na $i$ o~najmniejszym lewym końcu.
\begin{codebox}
\Procname{$\proc{Min-Interval-Search}(T,i)$}
\li $x\gets\proc{Interval-Search}(T,i)$
\li \If $x\ne\attrib{T}{nil}$
\li 	\Then $y\gets\attrib{x}{left}$
\li			\While $y\ne\attrib{T}{nil}$ \label{li:min-interval-search-while-begin}
\li				\Do \If $i$ zachodzi na \attrib{y}{int} \label{li:min-interval-search-overlap-test}
\li						\Then $x\gets y$
\li							$y\gets\attrib{x}{left}$ \label{li:min-interval-search-y-modification}
\li						\Else \If $\attrib{y}{left}\ne\attrib{T}{nil}$ i~$\attribb{y}{left}{max}\ge\attrib{i}{low}$ \label{li:min-interval-search-no-overlap-begin}
\li							\Then $y\gets\attrib{y}{left}$
\li							\Else $y\gets\attrib{y}{right}$
							\End \label{li:min-interval-search-no-overlap-end}
						\End
				\End \label{li:min-interval-search-while-end}
		\End
\li	\Return $x$
\end{codebox}

Udowodnimy następujący niezmiennik pętli \kw{while}:
\begin{quote}
Przed każdą iteracją pętli \kw{while} w~wierszach \doubledash{\ref{li:min-interval-search-while-begin}}{\ref{li:min-interval-search-while-end}}, jeśli w~drzewie $T$ znajduje się przedział zachodzący na $i$ o~lewym końcu mniejszym niż \attribb{x}{int}{low}, to należy on do poddrzewa o~korzeniu w~$y$.
\end{quote}
\begin{description}
	\item[Inicjowanie:] Przed pierwszą iteracją pętli \kw{while} $y=\attrib{x}{left}$.
Przedziały o~lewych końcach mniejszych niż \attribb{x}{int}{low} znajdują się w~poddrzewie o~korzeniu w~\attrib{x}{left}, dlatego niezmiennik jest spełniony.
	\item[Utrzymanie:] Załóżmy, że niezmiennik zachodzi przed każdą następną iteracją pętli \kw{while} i~zobaczmy, co zmienia wykonanie iteracji.
Zauważmy, że każda aktualizacja zmiennej $x$ w~trakcie działania procedury pociąga za sobą przestawienie $y$ na \attrib{x}{left}.
Jeśli w~linii \ref{li:min-interval-search-overlap-test} przedział $i$ zachodzi na \attrib{y}{int}, to można zaktualizować wskaźnik $x$ -- przechowujący przedział zachodzący na $i$ o~najmniejszym dotychczas napotkanym lewym końcu -- na $y$, ponieważ $y$ jest węzłem z~lewego poddrzewa $x$, więc $\attribb{y}{int}{low}\le\attribb{x}{int}{low}$.
Niezmiennik pozostaje spełniony, bo jeśli w~drzewie $T$ istnieje przedział zachodzący na $i$ o~mniejszym lewym końcu od nowej wartości $x$, to należy on do lewego poddrzewa $x$, a~w~wierszu \ref{li:min-interval-search-y-modification} $y$ jest aktualizowane na \attrib{x}{left}.

Załóżmy teraz, że przedział $i$ nie zachodzi na \attrib{y}{int}.
Zauważmy, że wiersze \doubledash{\ref{li:min-interval-search-no-overlap-begin}}{\ref{li:min-interval-search-no-overlap-end}} stanowią ciało pętli \kw{while} z~procedury \proc{Interval-Search} z~$y$ w~miejscu $x$.
Wykorzystując niezmiennik pętli \kw{while} tamtej procedury do obecnej iteracji pętli w~procedurze \proc{Min-Interval-Search} otrzymujemy, że jeśli przed aktualizacją $y$ poddrzewo o~korzeniu $y$ zawierało przedział zachodzący na $i$, to zawiera go też poddrzewo o~korzeniu w~węźle wskazywanym przez nową wartość wskaźnika $y$.
Niezmiennik jest więc zachowany, gdyż jak zobaczyliśmy wcześniej, $y$ znajduje się zawsze w~lewym poddrzewie $x$.
	\item[Zakończenie:] Pętla kończy działanie, gdy $y=\attrib{T}{nil}$.
Poddrzewo o~korzeniu w~$y$ jest puste, skąd wynika, że nie istnieje w~$T$ przedział zachodzący na $i$ o~lewym końcu mniejszym niż \attribb{x}{int}{low}, czyli $x$ jest szukanym przedziałem.
\end{description}

Algorytm działa w~czasie $O(\lg n)$ dla drzewa o~$n$ węzłach, ponieważ oprócz wywołania operacji \proc{Interval-Search} schodzi jeszcze od znalezionego węzła do liścia drzewa $T$ po ścieżce prostej.

\exercise %14.3-4
Poniższy pseudokod wypisuje szukane przedziały niemalejąco według ich lewych końców.
Aby wypisać wszystkie przedziały zachodzące na $i$ w~drzewie $T$, należy skorzystać z~wywołania $\proc{Interval-Search-All}(T,\attrib{T}{root},i)$.
Drzewo $T$ nie jest modyfikowane w~trakcie działania algorytmu.
\begin{codebox}
\Procname{$\proc{Interval-Search-All}(T,x,i)$}
\li	\If $\attrib{x}{left}\ne\attrib{T}{nil}$ i~$\attribb{x}{left}{max}\ge\attrib{i}{low}$ \label{li:interval-search-all-examine-left}
\li		\Then $\proc{Interval-Search-All}(T,\attrib{x}{left},i)$
		\End
\li \If $x\ne\attrib{T}{nil}$ oraz $i$ zachodzi na \attrib{x}{int} \label{li:interval-search-all-overlap}
\li		\Then wypisz \attrib{x}{int}
		\End
\li	\If $\attrib{x}{right}\ne\attrib{T}{nil}$ i~$\attribb{x}{right}{max}\ge\attrib{i}{low}$ i~$\attribb{x}{int}{low}\le\attrib{i}{high}$ \label{li:interval-search-all-examine-right}
\li		\Then $\proc{Interval-Search-All}(T,\attrib{x}{right},i)$
		\End
\end{codebox}

Zastanówmy się nad testami przeprowadzanymi, zanim wykonane zostaną wywołania rekurencyjne.
W~linii \ref{li:interval-search-all-examine-left}, gdyby było $\attribb{x}{left}{max}<\attrib{i}{low}$, to każdy przedział z~lewego poddrzewa $x$ leżałby na lewo od $i$, więc wywołanie byłoby zbędne.
Podobnie w~wierszu \ref{li:interval-search-all-examine-right} -- gdyby $\attribb{x}{right}{max}<\attrib{i}{low}$, to wszystkie przedziały z~prawego poddrzewa $x$ leżałyby na lewo od $i$.
Gdyby ponadto zachodziło $\attribb{x}{int}{low}>\attrib{i}{high}$, to zarówno \attrib{x}{int}, jak i~każdy przedział z~prawego poddrzewa $x$ znajdowałby się na prawo od $i$, dlatego także wtedy do wywołania rekurencyjnego nie dochodzi.

Załóżmy teraz, że poddrzewo o~korzeniu w~$x$ nie zawiera przedziałów zachodzących na $i$.
Jeśli $i$ leży na lewo od \attrib{x}{int}, to procedura może zostać wywołana rekurencyjnie tylko dla lewego poddrzewa $x$.
Jeśli z~kolei $i$ leży na prawo od \attrib{x}{int}, to procedura nie zostanie wywołana rekurencyjne dla lewego poddrzewa.
Gdyby tak było, to musiałoby zachodzić $\attribb{x}{left}{max}\ge\attrib{i}{low}$, czyli w~lewym poddrzewie $x$ znajdowałby się przedział $i'$, dla którego $\attrib{i'}{high}=\attribb{x}{left}{max}\ge\attrib{i}{low}$, a~zatem przedział $i'$ zachodziłby na $i$, co przeczyłoby założeniu.
Wynika stąd, że jeśli w~poddrzewie o~korzeniu w~$x$ nie ma przedziału zachodzącego na $i$, to procedura na każdym poziomie wykonuje co najwyżej jedno zejście rekurencyjne, schodząc najdalej do potomnego liścia $x$.

Każdy węzeł drzewa $T$ testowany jest w~algorytmie co najwyżej dwa razy -- raz w~linii \ref{li:interval-search-all-examine-left} albo \ref{li:interval-search-all-examine-right} i~raz w~linii \ref{li:interval-search-all-overlap}.
Każdy taki test zajmuje czas stały, dlatego czas działania algorytmu nie przekracza $O(n)$.
Z~drugiej strony procedura dociera do każdego węzła z~przedziałem zachodzącym na $i$, po czym może zejść rekurencyjnie od niego aż do jednego z~jego potomnych liści w~drzewie (na podstawie poprzedniego paragrafu).
Dlatego czas działania nie może też przekroczyć $O(k\lg n)$, gdyż wysokość drzewa $T$ wynosi $O(\lg n)$, a~liczba przedziałów zachodzących na $i$ w~$T$ wynosi $k$.
Łącząc te ograniczenia, dostajemy, że algorytm działa w~czasie $O(\min(n,k\lg n))$.

\exercise %14.3-5
Zmodyfikujemy operację wstawiania do drzewa przedziałowego w~taki sposób, aby w~przypadku powtarzających się kluczy (lewych końców przedziałów) uporządkowanie wyznaczone było na podstawie prawych końców przedziałów.
Podczas wstawiania nowego węzła $z$ jeśli w~trakcie schodzenia w~dół drzewa napotkany zostanie węzeł $x$, w~którym $\attribb{x}{int}{low}=\attribb{z}{int}{low}$, to porównane zostaną prawe końce przedziałów \attrib{x}{int} i~\attrib{z}{int}, i~w~zależności od wyniku węzeł $z$ zostanie umieszczony w~odpowiednim poddrzewie $x$.
W~procedurze \proc{Interval-Insert} test odpowiadający testowi z~wiersza 5 procedury \proc{RB-Insert}, na której ta pierwsza się opiera, należy zamienić na
\begin{codebox}
\zi \If $\attribb{z}{int}{low}<\attribb{x}{int}{low}$ lub ($\attribb{z}{int}{low}=\attribb{x}{int}{low}$ i~$\attribb{z}{int}{high}<\attribb{x}{int}{high}$)
\end{codebox}
oraz analogicznie test z~wiersza 11 z~$y$ w~miejscu $x$.
Dzięki temu w~porządku inorder wyznaczonym na podstawie lewych końców przedziałów, w~obrębie bloków z~równymi lewymi końcami przedziały uporządkowane będą względem swoich prawych końców.
Oczywiście rotacje zachowują porządek inorder w~drzewie.
Opisana modyfikacja nie zwiększa asymptotycznego czasu działania operacji wstawiania.

Procedura \proc{Interval-Search-Exactly} wyznacza szukany węzeł w~identyczny sposób jak opisane powyżej wyszukiwanie miejsca w~drzewie podczas wstawiania do niego nowego węzła.
Stąd też czas działania nowej operacji można ograniczyć od góry przez $O(\lg n)$, gdzie $n$ jest liczbą węzłów w~drzewie wejściowym.
\begin{codebox}
\Procname{$\proc{Interval-Search-Exactly}(T,i)$}
\li	$x\gets\attrib{T}{root}$
\li	\While $x\ne\attrib{T}{nil}$ i~$\attrib{x}{int}\ne i$
\li		\Do \If $\attrib{i}{low}<\attribb{x}{int}{low}$ lub ($\attrib{i}{low}=\attribb{x}{int}{low}$ i~$\attrib{i}{high}<\attribb{x}{int}{high}$)
\li				\Then $x\gets\attrib{x}{left}$
\li				\Else $x\gets\attrib{x}{right}$
				\End
		\End
\li	\Return $x$
\end{codebox}

\exercise %14.3-6
Strukturę danych $Q$ zaimplementujemy w~postaci odpowiednio wzbogaconego drzewa czerwono-czarnego.
W~każdym węźle $x$ drzewa $Q$ przechowamy dodatkowe pola:
\begin{itemize}
	\item \attrib{x}{min-key} -- równe najmniejszemu kluczowi w~poddrzewie o~korzeniu w~$x$;
	\item \attrib{x}{max-key} -- równe największemu kluczowi w~poddrzewie o~korzeniu w~$x$;
	\item \attrib{x}{min-gap} -- równe najmniejszej odległości między kluczami w~poddrzewie o~korzeniu w~$x$.
\end{itemize}
Wszystkie nowe pola każdego wewnętrznego węzła $x$ są zależne od innych pól węzła $x$ oraz od pól węzłów \attrib{x}{left} i~\attrib{x}{right}:
\begin{align*}
	\attrib{x}{min-key} &= \min(\attribb{x}{left}{min-key},\attrib{x}{key},\attribb{x}{right}{min-key}), \\
	\attrib{x}{max-key} &= \max(\attribb{x}{left}{max-key},\attrib{x}{key},\attribb{x}{right}{max-key}), \\
	\attrib{x}{min-gap} &= \min(\attribb{x}{left}{min-gap},\attribb{x}{right}{min-gap}, \\
		&\phantom{{}=\min(}\attrib{x}{key}-\attribb{x}{left}{max-key},\attribb{x}{right}{min-key}-\attrib{x}{key}).
\end{align*}
Definiujemy ponadto $\attribb{Q}{nil}{min-key}=\attribb{Q}{nil}{min-gap}=\infty$ oraz $\attribb{Q}{nil}{max-key}=-\infty$.

Wywołanie $\proc{Min-Gap}(Q)$ polega na zwróceniu wartości \attribb{Q}{root}{min-gap} i~działa w~czasie $O(1)$.
Oczywiście wzbogacenie drzewa o~nowe pola nie zmienia działania operacji \proc{Search}.
Dzięki zastosowaniu tw.\ 14.1 mamy z~kolei, że wprowadzenie nowych pól nie zwiększa asymptotycznej złożoności czasowej operacji \proc{Insert} i~\proc{Delete}.

\exercise %14.3-7
Ogólna idea algorytmu polega na pomyśle ze wskazówki podanej w~treści zadania: wykorzystywana jest tzw.\ ,,miotła'', czyli pionowa prosta zamiatająca zbiór prostokątów od lewej do prawej.
W~danym momencie pamiętane będą prostokąty, których bok poziomy wyznacza przedział zawierający aktualną pozycję ,,miotły''.
Tak naprawdę zamiast przechowywania wszystkich informacji o~prostokątach wystarczy pamiętać przedziały wyznaczone przez boki pionowe tych prostokątów.
Do tego celu zastosujemy drzewo przedziałowe.

Omówmy teraz szczegóły techniczne algorytmu.
Najpierw utworzone zostanie puste drzewo przedziałowe $T$ oraz lista współrzędnych $x$ wszystkich prostokątów wraz z~informacją o~tym, czy jest to lewa współrzędna swojego prostokąta, czy prawa.
Lista ta zostanie następnie posortowana i~będzie przeglądana od lewej do prawej.
Jeśli kolejna napotkana współrzędna na tej liście stanowi lewą współrzędną $x$ pewnego prostokąta $R$, to na drzewie $T$ wywołana zostanie operacja \proc{Interval-Search} dla przedziału wyznaczonego przez współrzędne $y$ prostokąta $R$ (czyli jego bok pionowy).
Wynik działania tego wywołania stanowi informację o~tym, czy w~zbiorze prostokątów istnieje prostokąt różny od $R$ i~zachodzący na $R$.
Następnie pionowy bok prostokąta $R$ zostanie wstawiony do drzewa $T$.
Z~kolei napotkawszy prawą współrzędną $x$ prostokąta $R$, algorytm usunie pionowy bok $R$ z~drzewa $T$.
W~dowolnym momencie działania algorytmu drzewo $T$ reprezentuje zatem zbiór prostokątów, które są aktualnie przecinane przez prostą zamiatającą.

Czas działania algorytmu dla $n$ prostokątów wynosi $O(n\lg n)$.
Tyle czasu zajmuje sortowanie listy współrzędnych $x$ oraz $O(n)$ wywołań operacji na drzewie przedziałowym (wstawianie, usuwanie i~wyszukiwanie przedziałów zachodzących), z~których każda wymaga czasu $O(\lg n)$.
