\subchapter{Dynamiczne statystyki pozycyjne}

\exercise %14.1-1
Działanie algorytmu ilustruje tabela \ref{tab:14.1-1}.
\begin{table}[!ht]
	\centering
	\begin{tabular}{rrrl}
		\attrib{x}{key} & $i$ & $r$ & akcja \\ \hline
		26 & 10 & 13 & szukaj w~lewym poddrzewie \\
		17 & 10 & 8 & szukaj w~prawym poddrzewie \\
		21 & 2 & 3 & szukaj w~lewym poddrzewie \\
		19 & 2 & 1 & szukaj w~prawym poddrzewie \\
		20 & 1 & 1 & zwróć $x$
	\end{tabular}
	\caption{Przebieg działania wywołania $\proc{OS-Select}(\attrib{T}{root},10)$ dla drzewa $T$ z~rys.\ 14.1 z~Podręcznika.
Każdy wiersz pokazuje wartości w~kolejnych wywołaniach rekurencyjnych -- klucz węzła $x$, pozycję $i$ szukanego klucza w~poddrzewie o~korzeniu w~$x$ i~rangę $r$ węzła $x$ -- a~także wykonywaną w~tym wywołaniu akcję (zejście rekurencyjne albo zwrócenie wyniku).} \label{tab:14.1-1}
\end{table}

\exercise %14.1-2
Na początku działania procedury zmienna $r$ zostaje zainicjalizowana na 1, a~$y$ początkowo wskazuje węzeł $x$.
W~tabeli \ref{tab:14.1-2} zilustrowano kolejne iteracje pętli \kw{while}.
Wynikiem zwracanym przez procedurę jest 16.
\begin{table}[!ht]
	\centering
		\begin{tabular}{r|rr}
			iteracja & \attrib{y}{key} & $r$ \\ \hline
			1 & 35 & 1 \\
			2 & 38 & 1 \\
			3 & 30 & 3 \\
			4 & 41 & 3 \\
			5 & 26 & 16
		\end{tabular}
		\caption{Przebieg działania procedury \proc{OS-Rank} wywołanej dla drzewa $T$ z~rys.\ 14.1 z~Podręcznika i~węzła $x$ o~kluczu 35.
Każdy wiersz tabeli zawiera klucz węzła $y$ oraz jego rangę $r$ na początku danej iteracji pętli \kw{while}.} \label{tab:14.1-2}
\end{table}

\exercise %14.1-3
Iteracyjna wersja \proc{OS-Select} wykorzystuje pętlę \kw{while}, w~której przez odpowiednie aktualizowanie zmiennych symulowana jest rekurencja z~oryginalnej procedury.
\begin{codebox}
\Procname{$\proc{Iterative-OS-Select}(x,i)$}
\li	$r\gets\attribb{x}{left}{size}+1$
\li	\While $r\ne i$
\li		\Do \If $i<r$
\li				\Then $x\gets\attrib{x}{left}$
\li				\Else $x\gets\attrib{x}{right}$
\li					$i\gets i-r$
				\End
\li			$r\gets\attribb{x}{left}{size}+1$
		\End
\li	\Return $x$
\end{codebox}

\exercise %14.1-4
W~naszej implementacji procedura przyjmować będzie jako parametr węzeł $x$ drzewa $T$ zamiast samego drzewa.
Pozwoli to wywoływać procedurę rekurencyjnie w~celu znalezienia rangi klucza w~poddrzewach $T$.
\begin{codebox}
\Procname{$\proc{OS-Key-Rank}(x,k)$}
\li	$r=\attribb{x}{left}{size}+1$
\li	\If $k=\attrib{x}{key}$
\li		\Then \Return $r$ \label{li:os-key-rank-key-found}
		\End
\li	\If $k<\attrib{x}{key}$
\li		\Then \Return $\proc{OS-Key-Rank}(\attrib{x}{left},k)$
\li		\Else \Return $\proc{OS-Key-Rank}(\attrib{x}{right},k)+r$
		\End
\end{codebox}
Po wyznaczeniu rangi $r$ klucza \attrib{x}{key}, jeśli jest on elementem o~szukanej randze, to w~wierszu \ref{li:os-key-rank-key-found} nastąpi zwrócenie $r$.
W~przeciwnym przypadku procedura wywołuje się rekurencyjnie dla lewego albo prawego poddrzewa $x$, w~zależności od tego, w~którym z~nich szukany klucz się znajduje na podstawie własności drzewa wyszukiwań binarnych.
Jeśli $k<\attrib{x}{key}$, to ranga klucza $k$ w~poddrzewie o~korzeniu $x$ jest równa jego randze w~poddrzewie o~korzeniu \attrib{x}{left}.
Gdy natomiast $k>\attrib{x}{key}$, to ranga $k$ w~poddrzewie o~korzeniu $x$ jest o~$r$ większa od jego rangi w~poddrzewie o~korzeniu \attrib{x}{right}.

Aby wyznaczyć rangę klucza $k$ w~drzewie statystyk pozycyjnych $T$ zawierającym parami różne klucze, należy wywołać $\proc{OS-Key-Rank}(\attrib{T}{root},k)$.

\exercise %14.1-5
Załóżmy, że ranga klucza \attrib{x}{key} wynosi $r$.
Wystarczy zauważyć, że $i$\nbhyphen ty następnik węzła $x$ jest węzłem z~kluczem o~randze $r+i$.
Najpierw wyznaczamy więc $r$ w~czasie $O(\lg n)$, wywołując $\proc{OS-Rank}(T,x)$, a~następnie, również w~czasie $O(\lg n)$, szukamy w~$T$ elementu o~randze $r+i$, czyli wykonujemy $\proc{OS-Select}(\attrib{T}{root},r+i)$.

\exercise %14.1-6
Przyjmijmy, że ranga klucza dowolnego węzła $x$ w~poddrzewie o~korzeniu $x$ przechowywana jest w~atrybucie \attrib{x}{rank}.

Wpierw opiszemy zmiany rang, jakie zachodzą podczas przeprowadzania rotacji.
Rozważmy węzeł $x$, którego prawy syn $y$ jest węzłem wewnętrznym drzewa, i~wykonajmy na $x$ lewą rotację.
Wartość \attrib{x}{rank} nie zmieni się, ponieważ lewe poddrzewo $x$ nie zmienia swojego rozmiaru w~wyniku przeprowadzenia rotacji.
Węzeł $x$ staje się lewym synem węzła $y$, a~lewe poddrzewo $y$ sprzed rotacji staje się prawym poddrzewem $x$, w~dalszym ciągu jednak pozostając w~lewym poddrzewie $y$.
Wynika stąd, że po wykonaniu rotacji \attrib{y}{rank} zostanie powiększone o~\attrib{x}{rank}.
Symetrycznie, w~trakcie prawej rotacji węzła $x$ o~lewym synu $y$, zmianie ulegnie jedynie \attrib{x}{rank} -- dokładniej, zmniejszy się o~\attrib{y}{rank}.

Podczas wstawiania nowego węzła $z$ do drzewa statystyk pozycyjnych pokonywana jest ścieżka od korzenia do jednego z~liści.
Dla każdego węzła $x$ na tej ścieżce, jeśli węzeł $z$ jest umieszczany w~lewym poddrzewie $x$, to rozmiar tego poddrzewa rośnie o~1, a~co za tym idzie, również \attrib{x}{rank} rośnie o~1.
Jeśli z~kolei $z$ ląduje w~prawym poddrzewie $x$, to \attrib{x}{rank} nie zmienia się.
Podobnie w~usuwaniu, ranga każdego węzła $x$, który w~swoim lewym poddrzewie zawierał efektywnie usunięty węzeł, zostanie zmniejszona o~1.

\exercise %14.1-7
Niech $A[1\twodots n]$ będzie tablicą zawierającą $n$ różnych liczb.
Oznaczmy przez $r_i$ rangę elementu $A[i]$ w~podtablicy $A[1\twodots i]$.
Dla każdego $i=1$, 2, \dots, $n$ element $A[i]$ jest zatem $r_i$\nbhyphen tym najmniejszym elementem w~podtablicy $A[1\twodots i]$, dlatego z~pozostałymi jej elementami tworzy $i-r_i$ inwersji.
Sumaryczna liczba inwersji w~tablicy $A$ jest więc równa $\sum_{i=1}^n(i-r_i)$.

Rangi elementów tablicy $A$ możemy wyznaczyć, wstawiając do drzewa statystyk pozycyjnych kolejno, $A[1]$, $A[2]$, \dots, $A[n]$.
Tuż po wstawieniu elementu $A[i]$ jego ranga w~aktualnym drzewie wynosi $r_i$, którą to wartość otrzymujemy, wywołując algorytm \proc{OS-Rank}.
Wstawienie dowolnego elementu tablicy $A$ do drzewa oraz obliczenie jego rangi wymaga czasu $O(\lg n)$, dlatego inwersje w~$A$ możemy policzyć w~czasie $O(n\lg n)$.

\exercise %14.1-8
Nadajmy cięciwom etykiety w~postaci kolejnych liczb naturalnych od 1 do $n$.
Utwórzmy teraz ciąg o~długości $2n$ uzyskany w~wyniku poruszania się wzdłuż okręgu w~ustalonym kierunku (rozpoczynając od dowolnego z~jego punktów) i~odczytywania etykiety cięciwy w~momencie napotkania jednego z~jej końców.
Ciąg ten przechowamy w~tablicy $C[1\twodots2n]$.
Nietrudno udowodnić fakt, że cięciwa o~etykiecie $i$ przecina się z~cięciwą o~etykiecie $j$ wtedy i~tylko wtedy, gdy dokładnie jeden egzemplarz $i$ znajduje się w~tablicy $C$ pomiędzy dwoma egzemplarzami $j$.

Tablica $C$ będzie stanowić dane wejściowe dla algorytmu \proc{Intersecting-Chords} wyznaczającego liczbę przecinających się cięciw.
Ogólna idea algorytmu będzie polegała na przeglądnięciu tablicy $C$ od lewej do prawej i~policzeniu etykiet, które pojawiają się dokładnie raz między dwoma egzemplarzami innej etykiety.
Algorytm zwróci na końcu sumę takich wystąpień.

Do stwierdzania, czy dana etykieta została już napotkana podczas przeglądania tablicy $C$, wykorzystamy tablicę $P[1\twodots n]$, którą wypełnimy początkowo zerami.
Będziemy też utrzymywać zbiór dynamiczny $T$ przechowujący pozycje w~$C$ końców cięciw, które pojawiły się do tej pory dokładnie raz.
Dla każdej kolejnej pozycji $k=1$, 2, \dots, $2n$, jeśli etykieta $j=C[k]$ pojawia się pierwszy raz, to do $P[j]$ wpiszemy wartość $k$ i~wstawimy $k$ do zbioru $T$.
W~przeciwnym przypadku znajdziemy liczbę elementów zbioru $T$ większych od $P[j]$ i~dodamy ją do liczby aktualnych przecięć, a~następnie usuniemy $P[j]$ z~$T$.

Aby efektywnie wykonywać operacje na zbiorze dynamicznym $T$, zaimplementujemy go jako drzewo statystyk pozycyjnych.
Do wstawiania, usuwania i~wyszukiwania elementów w~tym drzewie wykorzystamy operacje \proc{OS-Insert}, \proc{OS-Delete} i~\proc{OS-Search}, które działają jak odpowiednie operacje na drzewach czerwono-czarnych i~ewentualnie aktualizują wartości atrybutów \id{size}.
Pierwsze dwie z~nich zostały opisane w~Podręczniku, a~\proc{OS-Search} działa identycznie jak wyszukiwanie zadanego klucza w~drzewie czerwono-czarnym.
W~trackie działania algorytmu w~drzewie $T$ nigdy nie będzie powtarzających się kluczy, więc wyszukiwanie danego klucza w~$T$ zwróci jedyny węzeł z~takim kluczem.
Można też zauważyć, że dla dowolnego węzła $x$ rozmiar drzewa $T$ pomniejszony o~rangę $x$ w~$T$, jest liczbą węzłów o~kluczach większych niż \attrib{x}{key}.
Rozmiar drzewa pozycyjnego $T$ wynosi \attribb{T}{root}{size}, a~do znajdowania rangi elementu wykorzystamy operację \proc{OS-Rank}.
\begin{codebox}
\Procname{$\proc{Intersecting-Chords}(C)$}
\li $n\gets\attrib{C}{length}/2$
\li	\For $k=1$ \To $n$
\li		\Do $P[k]=0$
		\End
\li	$\id{intersections}\gets0$
\li	utwórz puste drzewo statystyk pozycyjnych $T$
\li	\For $k=1$ \To $2n$
\li		\Do $j\gets C[k]$
\li			\If $P[j]=0$
\li				\Then $P[j]\gets k$
\li					utwórz węzeł $x$, w~którym $\attrib{x}{key}=k$
\li					$\proc{OS-Insert}(T,x)$
\li				\Else $x\gets\proc{OS-Search}(T,P[j])$ \label{li:intersecting-chords-count-intersections-begin}
\li					$\id{intersections}\gets\id{intersections}+\attribb{T}{root}{size}-\proc{OS-Rank}(T,x)$
\li					$\proc{OS-Delete}(T,x)$
				\End \label{li:intersecting-chords-count-intersections-end}
		\End
\li	\Return \id{intersections}
\end{codebox}

Zauważmy, że każda z~$n$ etykiet jest wstawiana do $T$ dokładnie raz i~dla każdej z~nich dokładnie raz wykonane zostaną operacje w~wierszach \ref{li:intersecting-chords-count-intersections-begin}\nbendash\ref{li:intersecting-chords-count-intersections-end}.
Ponieważ wszystkie operacje słownikowe, jak również \proc{OS-Rank}, na drzewie pozycyjnym o~$n$ węzłach działają w~czasie $O(\lg n)$, to stąd czasem działania algorytmu jest $O(n\lg n)$.
