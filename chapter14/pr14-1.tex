\problem{Punkt o~największej liczbie przecięć} %14-1

\subproblem %14-1(a)
Niech $I$ będzie rozważanym zbiorem przedziałów, a~$p$ -- jednym z~punktów o~największej liczbie $s$ przecięć z~przedziałami z~$I$.
Oznaczmy przez $i_1$, $i_2$, \dots, $i_s$ parami różne przedziały z~$I$ zawierające $p$.
Oczywiście przecięcie $i=\bigcap_{k=1}^si_k$ również jest przedziałem, który zawiera punkt $p$.
Przedział $i$ składa się z~punktów, na które zachodzi dokładnie $s$ przedziałów z~$I$, w~szczególności końce $i$ stanowią punkty o~największej liczbie przecięć z~przedziałami z~$I$, będąc jednocześnie końcami pewnych przedziałów z~$I$.

\subproblem %14-1(b)
Skorzystamy ze wskazówki i~do przechowywania przedziałów wykorzystamy wzbogacone drzewo czerwono-czarne $T$ zawierające nie same przedziały, ale ich końce.
Wstawianie przedziału będzie polegać na wstawieniu do drzewa lewego i~prawego końca tego przedziału, a~usuwanie -- na usunięciu jego końców z~drzewa.
Opiszemy teraz, jak wzbogacić drzewo $T$ o~nowe atrybuty, dzięki czemu możliwe będzie efektywne wykonywanie operacji wstawiania i~usuwania oraz wyznaczania punktu o~największej liczbie przecięć, także w~przypadku, gdy końce przedziałów nie są parami różne.

Kluczem danego węzła będzie punkt stanowiący koniec przedziału.
Przyjmiemy, że ten sam punkt występujący w~roli końca kilku różnych przedziałów będzie reprezentowany przez jeden węzeł, dzięki czemu klucze w~drzewie $T$ będą parami różne.
Dla każdego węzła wewnętrznego $x$, w~\attrib{x}{low} przechowamy liczbę lewych końców równych \attrib{x}{key} znajdujących się w~$T$, a~w~\attrib{x}{high} -- liczbę takich prawych końców.
Niech $\langle x_1,x_2,\dots,x_n\rangle$ będzie ciągiem węzłów wewnętrznych drzewa $T$ w~porządku inorder.
Dla $1\le j\le k\le n$ zdefiniujmy
\[
	s(j,k) = \sum_{i=j}^k(\attrib{x_i}{low}-\attrib{x_i}{high}) \quad\text{oraz}\quad s'(j,k) = s(j,k-1)+\attrib{x_k}{low}
\]
Na podstawie tych definicji można wywnioskować, że na punkt \attrib{x_i}{key} zachodzi $s'(1,i)$ przedziałów.
Dzięki obserwacji z~poprzedniej części problemu, punkt o~największej liczbie przecięć znajduje się w~końcu \attrib{x_i}{key} przedziału, dla którego wartość $s'(1,i)$ jest maksymalna.

Oznaczmy przez $l_x$ i~$r_x$ pozycje węzłów w~porządku inorder o~odpowiednio, minimalnym i~maksymalnym kluczu spośród węzłów z~poddrzewa o~korzeniu w~$x$.
Atrybut \attrib{x}{overlap} zdefiniujemy jako $s(l_x,r_x)$.
W~polu \attrib{x}{max-overlap} przechowamy z~kolei maksymalną wartość wyrażenia $s'(l_x,i)$ dla $l_x\le i\le r_x$.
W~końcu pole \attrib{x}{pom} przechowywać będzie koniec \attrib{x_i}{key}, dla którego $\attrib{x}{max-overlap}=s'(l_x,i)$.
Przyjmiemy ponadto $\attribb{T}{nil}{overlap}=0$ i~$\attribb{T}{nil}{max-overlap}=-\infty$.

Wartości nowych pól w~węźle wewnętrznym $x$ można wyznaczyć na podstawie informacji zawartych w~węzłach $x$, \attrib{x}{left} i~\attrib{x}{right}, co spełnia założenie tw.\ 14.1:
\begin{align*}
	\attrib{x}{overlap} &= \attribb{x}{left}{overlap}+(\attrib{x}{low}-\attrib{x}{high})+\attribb{x}{right}{overlap}, \\
	\attrib{x}{max-overlap} &= \max(\attribb{x}{left}{max-overlap}, \\
	&\phantom{{}=\max(}\attribb{x}{left}{overlap}+\attrib{x}{low}, \\
	&\phantom{{}=\max(}\attribb{x}{left}{overlap}+(\attrib{x}{low}-\attrib{x}{high})+\attribb{x}{right}{max-overlap}).
\end{align*}
Sposób obliczania \attrib{x}{overlap} jest oczywisty, zatem udowodnimy tylko poprawność wzoru na \attrib{x}{max-overlap}.
Niech $x_i$ będzie węzłem, dla którego $i$ maksymalizuje wyrażenie $s'(l_x,i)$.
Jeśli $x_i$ jest w~lewym poddrzewie $x$, to maksymalna wartość powyższego wyrażenia dla $l_{\attrib{x}{left}}\le i\le r_{\attrib{x}{left}}$ znajduje się w~$\attribb{x}{left}{max-overlap}$, więc $\attrib{x}{max-overlap}=\attribb{x}{left}{max-overlap}$.
Gdy $x_i=x$, to $s'(l_x,i)=s(l_x,i-1)+\attrib{x}{low}$ i~rozważmy dwa przypadki.
Jeśli $\attrib{x}{left}=\attrib{T}{nil}$, to $l_x=i$, a~więc $s'(l_x,i)=\attrib{x}{low}$.
Wzór na \attrib{x}{max-overlap} zachodzi dzięki odpowiedniemu zdefiniowaniu wartości pól dla wartownika.
W~przeciwnym przypadku mamy $l_x=l_{\attrib{x}{left}}$ oraz $i-1=r_{\attrib{x}{left}}$, a~stąd
\[
	s'(l_x,i) = s\bigl(l_{\attrib{x}{left}},r_{\attrib{x}{left}}\bigl)+\attrib{x}{low} = \attribb{x}{left}{overlap}+\attrib{x}{low}.
\]
Wreszcie, gdy $x_i$ należy do prawego poddrzewa $x$ oraz $\attrib{x}{left}\ne\attrib{T}{nil}$, to
\begin{align*}
	s'(l_x,i) &= s(l_x,i-1)+\attrib{x_i}{low} \\
	&= s\bigl(l_{\attrib{x}{left}},r_{\attrib{x}{left}}\bigr)+(\attrib{x}{low}-\attrib{x}{high})+s\bigl(l_{\attrib{x}{right}},i-1\bigr)+\attrib{x_i}{low} \\
	&= \attribb{x}{left}{overlap}+(\attrib{x}{low}-\attrib{x}{high})+s'\bigl(l_{\attrib{x}{right}},i\bigl) \\
	&= \attribb{x}{left}{overlap}+(\attrib{x}{low}-\attrib{x}{high})+\attribb{x}{right}{max-overlap}.
\end{align*}
W~sytuacji, gdy $\attrib{x}{left}$ jest wartownikiem, w~powyższym wzorze składnik $s\bigl(l_{\attrib{x}{left}},r_{\attrib{x}{left}}\bigl)$ należy zastąpić zerem, ale wówczas wzór także pozostaje spełniony.

Na podstawie wartości, jaką przyjmuje \attrib{x}{max-overlap}, ustalana jest odpowiednia wartość dla \attrib{x}{pom}, czyli \attribb{x}{left}{pom}, \attrib{x}{key} albo \attribb{x}{right}{pom}, kolejno dla przypadków opisywanych powyżej.

Jak mówiliśmy wcześniej, dodanie przedziału do drzewa polega na wstawieniu do niego obu jego końców.
Jeśli w~drzewie istnieje już węzeł $x$ przechowujący dany koniec $e$ (co można stwierdzić, uruchamiając operację \proc{Search} w~drzewie $T$), to należy zinkrementować \attrib{x}{low}, gdy $e$ jest lewym końcem, albo \attrib{x}{high}, gdy $e$ jest prawym końcem, a~następnie zaktualizować pola \id{overlap}, \id{max-overlap} i~\id{pom} dla wszystkich przodków $x$.
W~przeciwnym przypadku do drzewa $T$ wstawiany jest nowy węzeł $z$, w~którym $\attrib{z}{key}=e$ oraz:
\begin{itemize}
	\item $\attrib{z}{low}=1$, $\attrib{z}{high}=0$, jeśli $e$ jest lewym końcem, albo
	\item $\attrib{z}{low}=0$, $\attrib{z}{high}=1$, jeśli $e$ jest prawym końcem.
\end{itemize}
Początkowe wartości pozostałych nowych pól wynoszą:
\begin{itemize}
	\item $\attrib{z}{overlap}=\attrib{z}{low}-\attrib{z}{high}$,
	\item $\attrib{z}{max-overlap}=\attrib{z}{low}$,
	\item $\attrib{z}{pom}=\attrib{z}{key}$.
\end{itemize}
Procedura wstawiająca nowy przedział zwróci w~wyniku parę węzłów przechowujących końce tego przedziału.

W~celu usunięcia przedziału, z~drzewa $T$ usuwane są oba jego końce.
W~węźle przechowującym lewy koniec dekrementowana jest wartość atrybutu \id{low}, a~w~węźle z~prawym końcem -- wartość atrybutu \id{high}.
Faktyczne usunięcie któregokolwiek z~tych węzłów nastąpi tylko wówczas, gdy oba jego atrybuty \id{low} i~\id{high} osiągną wartość 0.
W~przeciwnym przypadku zaktualizowane zostaną jedynie wartości nowych pól we wszystkich przodkach aktualizowanego węzła.
Ponieważ węzły drzewa są pamiętane poza wywołaniami procedur, to operację usuwania węzła z~drzewa należy zaimplementować w~oparciu o~rozwiązanie \refExercise{12.3-4}.

Stosując teraz tw.\ 14.1 mamy, że operacje wstawiania i~usuwania przedziału do opisanego drzewa $T$ działają w~czasie $O(\lg n)$, gdzie $n$ jest liczbą przedziałów pamiętanych aktualnie w~$T$.
Operacja $\proc{Find-POM}(T)$ działa w~czasie $O(1)$, zwracając po prostu wartość \attribb{T}{root}{pom}.
