\problem{Punkt o~największej liczbie przecięć} %14-1

\subproblem %14-1(a)
Niech $I$ będzie rozważanym zbiorem przedziałów, a~$p$ -- jednym z~punktów o~największej liczbie $s$ przecięć z~przedziałami z~$I$.
Oznaczmy przez $i_1$, $i_2$, \dots, $i_s$ parami różne przedziały z~$I$ zawierające $p$.
Oczywiście przecięcie $i=\bigcap_{k=1}^si_k$ również jest przedziałem, który zawiera punkt $p$.
Przedział $i$ składa się z~punktów, na które zachodzi dokładnie $s$ przedziałów z~$I$, w~szczególności końce $i$ stanowią punkty o~największej liczbie przecięć z~przedziałami z~$I$.

\subproblem %14-1(b)
