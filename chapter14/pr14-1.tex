\problem{Punkt o~największej liczbie przecięć} %14-1

\subproblem %14-1(a)
Niech $I$ będzie rozważanym zbiorem przedziałów, a~$p$ -- jednym z~punktów o~największej liczbie $s$ przecięć z~przedziałami z~$I$.
Oznaczmy przez $i_1$, $i_2$, \dots, $i_s$ parami różne przedziały z~$I$ zawierające $p$.
Oczywiście przecięcie $i=\bigcap_{k=1}^si_k$ również jest przedziałem, który zawiera punkt $p$.
Przedział $i$ składa się z~punktów, na które zachodzi dokładnie $s$ przedziałów z~$I$, w~szczególności końce $i$ stanowią punkty o~największej liczbie przecięć z~przedziałami z~$I$.

\subproblem %14-1(b)
Skorzystamy ze wskazówki i~do przechowywania przedziałów wykorzystamy wzbogacone drzewo czerwono-czarne $T$ zawierające nie same przedziały, ale ich końce.
Wstawianie przedziału będzie polegać na dodaniu do drzewa końców tego przedziału, a~usuwanie -- na usunięciu jego końców z~drzewa.
Opiszemy teraz, jak wzbogacić drzewo $T$ o~nowe atrybuty, dzięki czemu możliwe będzie efektywne wykonywanie operacji wstawiania i~usuwania oraz wyznaczania punktu o~największej liczbie przecięć, także w~przypadku, gdy końce przedziałów nie są parami różne.

Kluczem danego węzła jest reprezentowany przez ten węzeł koniec przedziału.
Poza standardowymi atrybutami węzła drzewa czerwono-czarnego, czyli \id{key}, $p$, \id{left}, \id{right} i~\id{color}, każdy węzeł w~drzewie $T$ będzie zawierał też kilka nowych.
Przyjmiemy, że każde dwa końce przedziałów będące tym samym punktem reprezentowane są przez wspólny węzeł, dzięki czemu w~drzewie $T$ nie będzie dwóch różnych węzłów o~tym samym kluczu.
W~atrybucie \attrib{x}{low} węzła $x$ przechowamy liczbę lewych końców równych \attrib{x}{key} znajdujących się w~drzewie, a~w~atrybucie \attrib{x}{high} -- liczbę takich prawych końców.

Niech $a=\langle x_1,x_2,\dots,x_n\rangle$ będzie ciągiem węzłów wewnętrznych drzewa $T$ w~porządku inorder.
Zdefiniujmy
\[
	\sigma(j,k) = \sum_{i=j}^k(\attrib{x_i}{low}-\attrib{x_i}{high}) \quad\text{oraz}\quad s(j,k) = \sigma(j,k-1)+\attrib{x_k}{low}
\]
dla $1\le j$, $k\le n$.
Na podstawie tych definicji można wywnioskować, że na punkt \attrib{x_i}{key} zachodzi $s(1,i)$ przedziałów.
Dzięki obserwacji z~poprzedniej części problemu, punkt o~największej liczbie przecięć znajduje się w~końcu \attrib{x_i}{key} przedziału, dla którego wartość $s(1,i)$ jest maksymalna.

Z~każdym węzłem $x$ zwiążemy kolejne atrybuty.
Dla dowolnego węzła wewnętrznego $x$, niech $j_x$ i~$k_x$ oznaczają pozycje w~ciągu $a$ węzłów o~odpowiednio, minimalnym i~maksymalnym kluczu spośród węzłów z~poddrzewa o~korzeniu w~$x$.
Atrybut \attrib{x}{sum} zdefiniujemy jako $\sigma(j_x,k_x)$.
W~polu \attrib{x}{max} przechowamy z~kolei maksymalną wartość wyrażenia $s(j_x,i)$ dla $j_x\le i\le k_x$.
W~końcu pole \attrib{x}{pom} przechowywać będzie koniec \attrib{x_i}{key}, dla którego $\attrib{x}{max}=s(j_x,i)$.
Przyjmujemy ponadto $\attribb{T}{nil}{sum}=0$ oraz $\attribb{T}{nil}{max}=-\infty$.

Wartości nowych pól w~węźle wewnętrznym $x$ można wyznaczyć na podstawie informacji zawartych w~węzłach $x$, \attrib{x}{left} i~\attrib{x}{right}, co spełnia założenie tw.\ 14.1:
\begin{align*}
	\attrib{x}{sum} &= \attribb{x}{left}{sum}+(\attrib{x}{low}-\attrib{x}{high})+\attribb{x}{right}{sum}, \\
	\attrib{x}{max} &= \max(\attribb{x}{left}{max}, \\
	&\phantom{{}=\max(}\attribb{x}{left}{sum}+\attrib{x}{low}, \\
	&\phantom{{}=\max(}\attribb{x}{left}{sum}+(\attrib{x}{low}-\attrib{x}{high})+\attribb{x}{right}{max}).
\end{align*}
Sposób obliczania \attrib{x}{sum} jest oczywisty, udowodnimy zatem tylko postać wzoru na \attrib{x}{max}.
Niech $x_i$ będzie węzłem, dla którego $i$ maksymalizuje wyrażenie $s(j_x,i)$.
Jeśli $x_i$ jest w~lewym poddrzewie $x$, to maksymalna wartość powyższego wyrażenia dla $j_{\attrib{x}{left}}\le i\le k_{\attrib{x}{left}}$ znajduje się w~$\attribb{x}{left}{max}$, więc $\attrib{x}{max}=\attribb{x}{left}{max}$.
Gdy $x_i=x$, to $s(j_x,i)=\sigma(j_x,i-1)+\attrib{x}{low}$ i~rozważmy dwa przypadki.
Jeśli $\attrib{x}{left}=\attrib{T}{nil}$, to $j_x=i$, a~więc $s(j_x,i)=\attrib{x}{low}$.
Wzór na \attrib{x}{max} zachodzi dzięki odpowiedniemu zdefiniowaniu wartości pól dla wartownika.
W~przeciwnym przypadku mamy $j_x=j_{\attrib{x}{left}}$ oraz $i-1=k_{\attrib{x}{left}}$, a~stąd
\[
	s(j_x,i) = \sigma\bigl(j_{\attrib{x}{left}},k_{\attrib{x}{left}}\bigl)+\attrib{x}{low} = \attribb{x}{left}{sum}+\attrib{x}{low}.
\]
Wreszcie, gdy $x_i$ należy do prawego poddrzewa $x$ oraz $\attrib{x}{left}\ne\attrib{T}{nil}$, to
\begin{align*}
	s(j_x,i) &= \sigma(j_x,i-1)+\attrib{x_i}{low} \\
	&= \sigma\bigl(j_{\attrib{x}{left}},k_{\attrib{x}{left}}\bigr)+(\attrib{x}{low}-\attrib{x}{high})+\sigma\bigl(j_{\attrib{x}{right}},i-1\bigr)+\attrib{x_i}{low} \\[1mm]
	&= \attribb{x}{left}{sum}+(\attrib{x}{low}-\attrib{x}{high})+s\bigl(j_{\attrib{x}{right}},i\bigl) \\
	&= \attribb{x}{left}{sum}+(\attrib{x}{low}-\attrib{x}{high})+\attribb{x}{right}{max}.
\end{align*}
W~sytuacji, gdy $\attrib{x}{left}$ jest wartownikiem, w~powyższym wzorze składnik $\sigma\bigl(j_{\attrib{x}{left}},k_{\attrib{x}{left}}\bigl)$ należy zastąpić zerem, ale wówczas wzór także pozostaje spełniony.

Na podstawie wartości, jaką przyjmuje \attrib{x}{max}, ustalana jest odpowiednia wartość dla \attrib{x}{pom}, czyli \attribb{x}{left}{pom}, \attrib{x}{key} albo \attribb{x}{right}{pom}, kolejno dla przypadków opisywanych powyżej.

Jak mówiliśmy wcześniej, dodanie przedziału do drzewa polega na wstawieniu do niego obu jego końców.
Jeśli w~drzewie istnieje już węzeł $x$ przechowujący dany koniec (co można stwierdzić, uruchamiając operację \proc{Search} w~drzewie $T$), to należy zinkrementować \attrib{x}{low} dla lewego końca albo \attrib{x}{high} dla prawego końca, a~następnie zaktualizować pola \id{sum}, \id{max} i~\id{pom} dla wszystkich przodków $x$.
W~przeciwnym przypadku do drzewa $T$ wstawiany jest nowy węzeł $z$, w~którym $\attrib{z}{low}=1$ i~$\attrib{z}{high}=0$, jeśli dodawany jest lewy koniec, albo $\attrib{z}{low}=0$ i~$\attrib{z}{high}=1$, jeśli dodawany jest prawy koniec.
Początkowe wartości pozostałych nowych pól wynoszą: $\attrib{z}{sum}=\attrib{z}{low}-\attrib{z}{high}$, $\attrib{z}{max}=\attrib{z}{low}$ oraz $\attrib{z}{pom}=\attrib{z}{key}$.
Procedura wstawiająca nowy przedział zwróci w~wyniku parę węzłów przechowujących końce tego przedziału.

W~celu usunięcia przedziału, z~drzewa $T$ usuwane są oba jego końce.
W~węźle przechowującym lewy koniec dekrementowana jest wartość atrybutu \id{low}, a~w~węźle z~prawym końcem -- wartość atrybutu \id{high}.
Faktyczne usunięcie któregokolwiek z~tych węzłów nastąpi tylko wówczas, gdy oba jego atrybuty \id{low} i~\id{high} osiągną wartość 0.
W~przeciwnym przypadku zaktualizowane zostaną jedynie wartości nowych pól we wszystkich przodkach aktualizowanego węzła.
Ponieważ węzły drzewa są pamiętane poza wywołaniami procedur, to operację usuwania węzła z~drzewa należy zaimplementować w~oparciu o~rozwiązanie \refExercise{12.3-4}.

Stosując teraz tw.\ 14.1 mamy, że operacje wstawiania i~usuwania przedziału do opisanego drzewa $T$, działają w~czasie $O(\lg n)$, gdzie $n$ jest liczbą przedziałów znajdujących się aktualnie w~$T$.
Operacja $\proc{Find-POM}(T)$ działa w~czasie $O(1)$, zwracając po prostu wartość \attribb{T}{root}{pom}.
