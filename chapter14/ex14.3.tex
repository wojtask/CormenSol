\subchapter{Drzewa przedziałowe}

\exercise %14.3-1
Procedura \proc{Interval-Left-Rotate} dla drzew przedziałowych działa identycznie jak procedura \proc{Left-Rotate}, aktualizując jeszcze przed zakończeniem pola \attribb{x}{p}{max} i~\attrib{x}{max}.
Po wykonaniu rotacji największy koniec przedziału w~poddrzewie o~korzeniu w~\attrib{x}{p} jest taki sam jak w~poddrzewie o~korzeniu w~węźle $x$ sprzed rotacji.
Do \attribb{x}{p}{max} zostaje zatem przypisana wartość \attrib{x}{max}, po czym nowa wartość pola \attrib{x}{max} zostaje wyliczona bezpośrednio z~definicji.
\begin{codebox}
\Procname{$\proc{Interval-Left-Rotate}(T,x)$}
\li	$\proc{Left-Rotate}(T,x)$
\li	$\attribb{x}{p}{max}\gets\attrib{x}{max}$
\li	$\attrib{x}{max}\gets\max(\attribb{x}{int}{high},\attribb{x}{left}{max},\attribb{x}{right}{max})$
\end{codebox}

\exercise %14.3-2
Modyfikacją pseudokodu operacji \proc{Interval-Search} wymaganą do wspierania przedziałów otwartych jest zamiana nierówności nieostrej na ostrą w~linii 3.
Wynika to z~faktu, że w~przypadku przedziałów otwartych warunek $\attribb{x}{left}{max}=\attrib{i}{low}$ oznacza, że żaden przedział z~lewego poddrzewa $x$ nie zachodzi na $i$.
Powiemy, że dwa przedziały otwarte $i$ oraz $i'$ zachodzą na siebie, jeśli $i\cap i'\ne\emptyset$, tj.\ jeśli $\attrib{i}{low}<\attrib{i'}{high}$ i~$\attrib{i'}{low}<\attrib{i}{high}$.
Do takiej definicji powinna odwoływać się zmodyfikowana operacja \proc{Interval-Search} w~linii 2.

\exercise %14.3-3
Algorytm \proc{Interval-Search} zwraca pierwszy napotkany przedział w~drzewie przedziałowym nachodzący na $i$.
Preferuje przy tym schodzenie do lewego poddrzewa -- w~przypadku, gdy oba poddrzewa aktualnego węzła $x$ zawierają przedział nachodzący na $i$, kolejnym odwiedzanym węzłem jest \attrib{x}{left}.
Jednakże natychmiastowe kończenie pracy w~razie znalezienia wyniku uniemożliwia przejrzenie innych takich przedziałów i~wybór tego z~najmniejszym lewym końcem.
Wystarczy jednak zmodyfikować ten algorytm tak, aby schodził aż do liścia, pamiętając w~każdym momencie węzeł z~przedziałem o~najmniejszym do tej pory napotkanym lewym końcu spośród przedziałów nachodzących na $i$.
\begin{codebox}
\Procname{$\proc{Interval-Search-Lowest}(T,i)$}
\li	$x\gets\attrib{T}{root}$
\li	$y\gets\attrib{T}{nil}$
\li	\While $x\ne\attrib{T}{nil}$
\li		\Do \If $i$ zachodzi na \attrib{x}{int} i~($y=\attrib{T}{nil}$ lub $\attribb{x}{int}{low}<\attribb{y}{int}{low}$)
\li				\Then $y\gets x$
				\End
\li			\If $\attrib{x}{left}\ne\attrib{T}{nil}$ i~$\attribb{x}{left}{max}\ge\attrib{i}{low}$
\li				\Then $x\gets\attrib{x}{left}$
\li				\Else $x\gets\attrib{x}{right}$
				\End
		\End
\li	\Return $y$
\end{codebox}

Algorytm działa w~czasie $O(\lg n)$ dla drzewa o~$n$ węzłach, ponieważ schodzi od korzenia do liścia drzewa $T$ po ścieżce prostej.

\exercise %14.3-4
\note{W~treści zadania nie jest zdefiniowane\/ $n$.
Przyjmujemy, że jest to rozmiar drzewa\/ $T$.}

\noindent Poniższy pseudokod wypisuje szukane przedziały niemalejąco według ich lewych końców.
Aby wypisać wszystkie przedziały zachodzące na $i$ w~drzewie $T$, należy skorzystać z~wywołania $\proc{Interval-Search-All}(T,\attrib{T}{root},i)$.
Drzewo $T$ nie jest modyfikowane w~trakcie działania algorytmu.
\begin{codebox}
\Procname{$\proc{Interval-Search-All}(T,x,i)$}
\li	\If $\attrib{x}{left}\ne\attrib{T}{nil}$ i~$\attribb{x}{left}{max}\ge\attrib{i}{low}$ \label{li:interval-search-all-examine-left}
\li		\Then $\proc{Interval-Search-All}(T,\attrib{x}{left},i)$
		\End
\li \If $x\ne\attrib{T}{nil}$ oraz $i$ zachodzi na \attrib{x}{int}
\li		\Then wypisz \attrib{x}{int}
		\End
\li	\If $\attrib{x}{right}\ne\attrib{T}{nil}$ i~$\attribb{x}{right}{max}\ge\attrib{i}{low}$ i~$\attribb{x}{int}{low}\le\attrib{i}{high}$ \label{li:interval-search-all-examine-right}
\li		\Then $\proc{Interval-Search-All}(T,\attrib{x}{right},i)$
		\End
\end{codebox}

Zastanówmy się nad testami przeprowadzanymi, zanim wykonane zostaną wywołania rekurencyjne.
W~linii \ref{li:interval-search-all-examine-left}, gdy $\attribb{x}{left}{max}<\attrib{i}{low}$, to każdy przedział z~lewego poddrzewa $x$ leży na lewo od $i$, więc wyszukiwanie w~tym poddrzewie można pominąć.
Podobnie w~wierszu \ref{li:interval-search-all-examine-right} -- jeśli $\attribb{x}{right}{max}<\attrib{i}{low}$, to wszystkie przedziały z~prawego poddrzewa $x$ leżą na lewo od $i$.
Jeśli ponadto zachodzi $\attribb{x}{int}{low}>\attrib{i}{high}$, to zarówno \attrib{x}{int}, jak i~każdy przedział z~prawego poddrzewa $x$ znajduje się na prawo od $i$, dlatego także wtedy wyszukiwanie w~poddrzewie jest zbędne.

Załóżmy teraz, że poddrzewo o~korzeniu w~$x$ nie zawiera przedziałów zachodzących na $i$.
Jeśli $i$ leży na lewo od \attrib{x}{int}, to procedura może zostać wywołana rekurencyjnie tylko dla lewego poddrzewa $x$.
Jeśli z~kolei $i$ leży na prawo od \attrib{x}{int}, to procedura nie zostanie wywołana rekurencyjnie dla lewego poddrzewa.
Gdyby tak było, to musiałoby zachodzić $\attribb{x}{left}{max}\ge\attrib{i}{low}$, czyli w~lewym poddrzewie $x$ znajdowałby się przedział $i'$, dla którego $\attrib{i'}{high}=\attribb{x}{left}{max}\ge\attrib{i}{low}$, a~zatem przedział $i'$ zachodziłby na $i$, co przeczyłoby założeniu.
Wynika stąd, że jeśli w~poddrzewie o~korzeniu w~$x$ nie ma przedziału zachodzącego na $i$, to procedura na każdym poziomie wykonuje co najwyżej jedno zejście rekurencyjne, schodząc najdalej do ojca potomnego liścia.

Algorytm wywołany początkowo dla korzenia odwiedza węzły, poruszając się zawsze w~kierunku liści, dlatego żaden węzeł nie zostanie odwiedzony powtórnie.
Każde wywołanie rekurencyjne zajmuje czas stały, a~więc czas działania algorytmu nie przekroczy $O(n)$.
Z~drugiej strony procedura dociera do każdego węzła z~przedziałem zachodzącym na $i$, po czym może zejść rekurencyjnie od niego aż do ojca jednego z~jego potomnych liści (na podstawie poprzedniego paragrafu).
Dlatego czas działania można też ograniczyć przez $O(k\lg n)$, gdyż wysokość drzewa $T$ wynosi $O(\lg n)$, a~liczba przedziałów zachodzących na $i$ w~$T$ wynosi $k$.
Łącząc te ograniczenia, dostajemy, że algorytm działa w~czasie $O(\min(n,k\lg n))$.

\exercise %14.3-5
Zmodyfikujemy operację wstawiania do drzewa przedziałowego w~taki sposób, aby w~przypadku powtarzających się kluczy (lewych końców przedziałów) porządek inorder wyznaczony był na podstawie prawych końców przedziałów.
Podczas wstawiania nowego węzła $z$, jeśli w~trakcie schodzenia w~dół drzewa napotkany będzie węzeł $x$, w~którym $\attribb{x}{int}{low}=\attribb{z}{int}{low}$, to porównane zostaną prawe końce przedziałów \attrib{x}{int} i~\attrib{z}{int}, i~w~zależności od wyniku węzeł $z$ zostanie umieszczony w~odpowiednim poddrzewie $x$.
W~procedurze \proc{Interval-Insert} test odpowiadający testowi z~wiersza 5 procedury \proc{RB-Insert}, na której ta pierwsza się opiera, należy zamienić na
\begin{codebox}
\zi \If $\attribb{z}{int}{low}<\attribb{x}{int}{low}$ lub ($\attribb{z}{int}{low}=\attribb{x}{int}{low}$ i~$\attribb{z}{int}{high}<\attribb{x}{int}{high}$)
\end{codebox}
oraz analogicznie test z~wiersza 11 z~$y$ w~miejscu $x$.
Oczywiście rotacje zachowują porządek inorder w~drzewie.
Opisana modyfikacja nie zwiększa asymptotycznego czasu działania operacji wstawiania.

Procedura \proc{Interval-Search-Exactly} wyszukuje zadany przedział, pokonując tę samą ścieżkę od korzenia w~dół drzewa, co opisane powyżej wstawianie takiego przedziału podczas wstawiania do niego nowego węzła.
Stąd też czas działania nowej operacji można ograniczyć od góry przez $O(\lg n)$, gdzie $n$ jest liczbą węzłów w~drzewie wejściowym.
\begin{codebox}
\Procname{$\proc{Interval-Search-Exactly}(T,i)$}
\li	$x\gets\attrib{T}{root}$
\li	\While $x\ne\attrib{T}{nil}$ i~$\attrib{x}{int}\ne i$
\li		\Do \If $\attrib{i}{low}<\attribb{x}{int}{low}$ lub ($\attrib{i}{low}=\attribb{x}{int}{low}$ i~$\attrib{i}{high}<\attribb{x}{int}{high}$)
\li				\Then $x\gets\attrib{x}{left}$
\li				\Else $x\gets\attrib{x}{right}$
				\End
		\End
\li	\Return $x$
\end{codebox}

\exercise %14.3-6
Strukturę danych $Q$ zaimplementujemy w~postaci odpowiednio wzbogaconego drzewa czerwono-czarnego.
W~każdym węźle $x$ drzewa $Q$ przechowamy dodatkowe pola:
\begin{itemize}
	\item \attrib{x}{min-key} -- równe najmniejszemu kluczowi w~poddrzewie o~korzeniu $x$,
	\item \attrib{x}{max-key} -- równe największemu kluczowi w~poddrzewie o~korzeniu $x$,
	\item \attrib{x}{min-gap} -- równe najmniejszej odległości między dwoma różnymi kluczami w~poddrzewie o~korzeniu $x$.
\end{itemize}
Wszystkie nowe pola każdego wewnętrznego węzła $x$ zależą od innych pól węzła $x$ oraz od pól węzłów \attrib{x}{left} i~\attrib{x}{right}:
\begin{align*}
	\attrib{x}{min-key} &= \min(\attrib{x}{key},\attribb{x}{left}{min-key}), \\
	\attrib{x}{max-key} &= \max(\attrib{x}{key},\attribb{x}{right}{max-key}), \\
	\attrib{x}{min-gap} &= \min(\attribb{x}{left}{min-gap}, \\
		&\phantom{{}=\min(}\attribb{x}{right}{min-gap}, \\
		&\phantom{{}=\min(}\attrib{x}{key}-\attribb{x}{left}{max-key}, \\
		&\phantom{{}=\min(}\attribb{x}{right}{min-key}-\attrib{x}{key}).
\end{align*}
Definiujemy ponadto $\attribb{Q}{nil}{min-key}=\attribb{Q}{nil}{min-gap}=\infty$ oraz $\attribb{Q}{nil}{max-key}=-\infty$.

Wywołanie $\proc{Min-Gap}(Q)$ polega na zwróceniu wartości \attribb{Q}{root}{min-gap} i~działa w~czasie $O(1)$.
Oczywiście wzbogacenie drzewa o~nowe pola nie zmienia działania operacji \proc{Search}.
Dzięki zastosowaniu tw.\ 14.1 mamy z~kolei, że wprowadzenie nowych pól nie zwiększa asymptotycznej złożoności czasowej operacji \proc{Insert} i~\proc{Delete}.

\exercise %14.3-7
Ogólna idea algorytmu będzie polegać na pomyśle ze wskazówki podanej w~treści zadania: wykorzystamy tzw.\ ,,miotłę'', czyli pionową prostą zamiatającą zbiór prostokątów od lewej do prawej.
W~danym momencie pamiętane będą prostokąty, których bok poziomy wyznacza przedział zawierający aktualną pozycję ,,miotły''.
Tak naprawdę zamiast przechowywania wszystkich informacji o~prostokątach wystarczy pamiętać przedziały wyznaczone przez boki pionowe tych prostokątów.
Do tego celu zastosujemy drzewo przedziałowe.

Omówmy teraz szczegóły techniczne algorytmu.
Najpierw utworzone zostanie puste drzewo przedziałowe $T$ oraz lista współrzędnych $x$ wszystkich prostokątów, przy czym z~każdą współrzędną zapamiętamy wskaźnik do prostokąta, z~którego ona pochodzi.
Lista ta zostanie następnie posortowana niemalejąco, z~tym, że dla równych elementów na wynikowej liście najpierw zostaną umieszczone te, które stanowią współrzędną $x$ lewego boku swojego prostokąta.

Lista będzie następnie przeglądana od lewej do prawej.
Jeśli kolejny napotkany na niej element stanowi współrzędną $x$ lewego boku swojego prostokąta $R$, to na drzewie $T$ wywołana zostanie operacja \proc{Interval-Search} dla przedziału wyznaczonego przez współrzędne $y$ prostokąta $R$ (czyli jego bok pionowy).
Jeśli wywołanie to zwróci węzeł różny od \attrib{T}{nil}, to w~drzewie $T$ znajduje się pionowy bok innego prostokąta $R'$ zachodzący na pionowy bok prostokąta $R$, a~to oznacza, że prostokąty $R$ i~$R'$ zachodzą na siebie.
Można więc zakończyć działanie procedury, zwracając wartość logiczną \const{true}.
Jeśli jednak w~drzewie $T$ nie ma prostokąta zachodzącego na $R$, to jego pionowy bok zostanie wstawiony do $T$.
Z~kolei napotkawszy współrzędną $x$ prawego boku prostokąta $R$, algorytm usunie pionowy bok $R$ z~drzewa $T$.
W~dowolnym momencie działania algorytmu drzewo $T$ reprezentuje zatem zbiór prostokątów, które są aktualnie przecinane przez ,,miotłę''.
Do odszukania węzłów o~zadanym przedziale, które należy usunąć z~drzewa, wykorzystana zostanie operacja \proc{Interval-Search-Exactly} opisana w~\refExercise{14.3-5}.
Aby działała ona poprawnie, wstawianie przedziałów do drzewa musi być odpowiednio zmodyfikowane według opisu z~tamtego zadania.
Po przetworzeniu wszystkich elementów listy zwrócona zostanie wartość logiczna \const{false}.

Czas działania algorytmu dla $n$ prostokątów wynosi $O(n\lg n)$.
Tyle czasu zajmuje sortowanie listy współrzędnych $x$ oraz $O(n)$ wywołań operacji na drzewie przedziałowym (wstawianie, usuwanie i~wyszukiwanie przedziałów zachodzących), z~których każda wymaga czasu $O(\lg n)$.
