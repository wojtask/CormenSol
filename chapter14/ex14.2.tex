\subchapter{Jak wzbogacać strukturę danych}

\exercise %14.2-1
Aby efektywnie wykonywać podane operacje, wzbogacimy każdy węzeł wewnętrzny $x$ drzewa statystyk pozycyjnych $T$ o~2 nowe pola:
\begin{itemize}
	\item \attrib{x}{prev} -- wskaźnik na poprzednika węzła $x$ albo \attrib{T}{nil}, jeśli poprzednik nie istnieje,
	\item \attrib{x}{next} -- wskaźnik na następnika węzła $x$ albo \attrib{T}{nil}, jeśli następnik nie istnieje.
\end{itemize}
Pola te dodamy także do wartownika \attrib{T}{nil}, dla którego \id{prev} będzie pokazywało na węzeł o~największym kluczu w~drzewie, a~\id{next} -- na węzeł o~najmniejszym kluczu w~drzewie.
W~pustym drzewie $T$ będzie $\attribb{T}{nil}{prev}=\attribb{T}{nil}{next}=\attrib{T}{nil}$.
Innymi słowy, węzły wewnętrzne drzewa będą tworzyć uporządkowaną listę cykliczną z~wartownikiem \attrib{T}{nil}.

Tw.\ 14.1 nie stosuje się do nowych pól, bo ich wartości mogą zależeć nie tylko od pól synów danego węzła -- zarówno następnik, jak i~poprzednik węzła może nie znajdować się w~jego poddrzewach.
Pokażemy jednak, że nadal możliwe jest efektywne zaimplementowanie operacji wstawiania i~usuwania, które aktualizują wartości tych pól.

Zauważmy, że rotacje nie modyfikują pól \id{prev} i~\id{next}, ponieważ nie naruszają porządku inorder węzłów drzewa.
Po wstawieniu do drzewa nowego węzła $z$ można ustawić atrybut \attrib{z}{prev} na wartość zwróconą przez wywołanie oryginalnej wersji operacji \proc{Predecessor}, a~atrybut \id{next} -- na wynik oryginalnej wersji operacji \proc{Successor}.
Należy jeszcze uaktualnić \attribb{z}{prev}{next} i~\attribb{z}{next}{prev} na $z$.
Odpowiednie zdefiniowanie nowych pól w~wartowniku sprawia, że aktualizacja ta ma sens również wtedy, gdy \attrib{z}{prev} lub \attrib{z}{next} są \attrib{T}{nil}.
Nowe pola należy zaktualizować także w~operacji \proc{Delete}.
Tuż przed zakończeniem procedury dla faktycznie usuniętego węzła $y$ wystarczy ustawić \attribb{y}{prev}{next} oraz \attribb{y}{next}{prev} na $y$.
Widać więc, że dodatkowe wywołania nie powiększają asymptotycznego czasu działania operacji \proc{Insert} i~\proc{Delete}.

Dzięki wykorzystaniu nowo dodanych atrybutów operacje na zbiorach dynamicznych można zaimplementować w~czasie $O(1)$, aby zwracały następujące wartości:
\begin{itemize}
	\item $\proc{Minimum}(T)$: \attribb{T}{nil}{next},
	\item $\proc{Maximum}(T)$: \attribb{T}{nil}{prev},
	\item $\proc{Predecessor}(T,x)$: \attrib{x}{prev},
	\item $\proc{Successor}(T,x)$: \attrib{x}{next}.
\end{itemize}

\exercise %14.2-2
Jeśli czarną wysokość wewnętrznego węzła $x$ w~drzewie czerwono-czarnym $T$ przechowamy w~polu \attrib{x}{bh}, a~ponadto zdefiniujemy $\attribb{T}{nil}{bh}=0$, to zachodzi następująca zależność:
\[
	\attrib{x}{bh} = \begin{cases}
		\attribb{x}{left}{bh}, & \text{jeśli $\attribb{x}{left}{color}=\const{red}$}, \\
		\attribb{x}{left}{bh}+1, & \text{jeśli $\attribb{x}{left}{color}=\const{black}$}.
	\end{cases}
\]
A~zatem na mocy tw.\ 14.1 mamy, że złożoność asymptotyczna operacji słownikowych na tak wzbogaconym drzewie czerwono-czarnym nie ulegnie zmianie.

Opiszemy teraz modyfikacje, jakie należy wprowadzić do oryginalnych procedur \proc{RB-Insert} i~\proc{RB-Delete}, aby utrzymywać poprawne wartości atrybutu \id{bh} w~każdym węźle drzewa.
Podczas wstawiania nowego węzła $z$ do drzewa czerwono-czarnego \attrib{z}{bh} będzie inicjalizowane na 1.
Rozważmy modyfikacje czarnych wysokości węzłów zachodzące w~każdym przypadku procedury \proc{RB-Insert-Fixup}, posiłkując się rysunkami z~\refExercise{13.3-3}.
Zmianie ulega jedynie czarna wysokość węzła $C$ w~przypadku 1.
Wystarczy więc zinkrementować \attribbb{z}{p}{p}{bh} tuż po linii 7 w~\proc{RB-Insert-Fixup}.

Podobnie, polegając na rys.\ 13.7 z~Podręcznika, możemy przeanalizować każdy z~przypadków procedury \proc{RB-Delete-Fixup} pod kątem zmian w~czarnych wysokościach.
W~przypadku 2 węzeł $D$ zmienia kolor na czerwony, przez co czarna wysokość węzła $B$ maleje o~1.
Fakt ten można odnotować tuż po linii 10 procedury \proc{RB-Delete-Fixup}, ustawiając \attribb{x}{p}{bh} na $\attrib{x}{bh}+1$.
Z~kolei w~przypadku 4 czarna wysokość węzła $B$ maleje o~1, a~czarna wysokość węzła $D$ rośnie o~1.
Tuż po linii 20 wystarczy więc najpierw do \attribb{x}{p}{bh} wpisać $\attrib{x}{bh}+1$, po czym do \attribbb{x}{p}{p}{bh} wpisać $\attribb{x}{p}{bh}+1$.

\exercise %14.2-3
Nie można tego zrobić przy zachowaniu efektywnych czasów działania operacji na drzewie, ponieważ głębokość węzła zależy od głębokości jego ojca.
Gdy zmienia się głębokość węzła $x$, to zmieniają się także głębokości wszystkich potomków $x$.
A~zatem aktualizacja głębokości korzenia drzewa o~$n$ węzłach niesie za sobą konieczność aktualizacji pozostałych $n-1$ węzłów i~operacje wstawiania i~usuwania wymagają wtedy czasu $\Omega(n\lg n)$.

\exercise %14.2-4
Rotacje nie zmieniają porządku inorder węzłów w~żadnym z~poddrzew $\alpha$, $\beta$ i~$\gamma$ według oznaczeń z~rys.\ 13.2 z~Podręcznika.
Po przeprowadzeniu lewej rotacji zachodzi:
\begin{align*}
	\attrib{x}{f} &= \attribb{x}{left}{f}\otimes\attrib{x}{a}\otimes\attribb{x}{right}{f}, \\
	\attrib{y}{f} &= \attribb{x}{left}{f}\otimes\attrib{x}{a}\otimes\attribb{x}{right}{f}\otimes\attrib{y}{a}\otimes\attribb{y}{right}{f} = \attrib{x}{f}\otimes\attrib{y}{a}\otimes\attribb{y}{right}{f}.
\end{align*}
Zatem wartości pól $f$ modyfikowanych w~rotacji węzłów mogą zostać zaktualizowane w~czasie $O(1)$.
Rozumowanie w~przypadku prawej rotacji przeprowadza się analogicznie.

W~drzewie czerwono-czarnym $T$ zdefiniujmy teraz dla każdego węzła pole $a$ przyjmujące wartość 0 dla liści drzewa (reprezentowanych przez \attrib{T}{nil}) oraz 1 dla jego węzłów wewnętrznych.
Jeśli operacją $\otimes$ jest zwykłe dodawanie, to \attrib{x}{f} zdefiniowane jak w~treści zadania, stanowi rozmiar poddrzewa o~korzeniu w~$x$, czyli pole $f$ jest identyczne z~polem \id{size} z~drzew statystyk pozycyjnych.
Na podstawie powyższej argumentacji pole \id{size} może być aktualizowane w~czasie $O(1)$ po każdym wykonaniu rotacji w~drzewie $T$.

\exercise %14.2-5
Podamy wpierw pomocniczą procedurę opartą na zwykłym wyszukiwaniu w~drzewie czerwono-czarnym, ale w~przypadku braku węzła o~zadanym kluczu, zwracającą następnik tego węzła.
Innymi słowy, procedura ta dla danego klucza $k$ znajdzie węzeł o~najmniejszym możliwym kluczu równym co najmniej $k$, albo \attrib{T}{nil} gdy węzeł taki nie istnieje.
\begin{codebox}
\Procname{$\proc{RB-Search-At-Least}(T,x,y,k)$}
\li	\If $x=\attrib{T}{nil}$ \label{li:rb-search-at-least-leaf-reached}
\li		\Then \Return $y$
		\End
\li	\If $k=\attrib{x}{key}$
\li		\Then \Return $x$
		\End
\li	\If $k<\attrib{x}{key}$
\li		\Then \Return $\proc{RB-Search-At-Least}(T,\attrib{x}{left},x,k)$
\li		\Else \Return $\proc{RB-Search-At-Least}(T,\attrib{x}{right},y,k)$
		\End
\end{codebox}
Procedura przyjmuje, oprócz drzewa $T$ i~klucza $k$, także korzeń $x$ aktualnie przeszukiwanego poddrzewa oraz węzeł $y$ do zwrócenia w~przypadku nieodnalezienia węzła o~kluczu $k$.
W~każdym wywołaniu rekurencyjnym $y$ jest najniżej położonym węzłem na pokonywanej ścieżce, którego lewy syn też jest na tej ścieżce, albo \attrib{T}{nil} w~razie nieistnienia takiego węzła.
Jeśli w~drzewie $T$ nie ma węzła o~kluczu $k$, to w~końcu w~linii \ref{li:rb-search-at-least-leaf-reached} osiągany jest liść $x$, który należałoby zastąpić węzłem o~kluczu $k$, gdybyśmy wstawiali go do drzewa.
Prawe poddrzewo tego hipotetycznego węzła byłoby puste, a~więc na podstawie \refExercise{12.2-6} węzeł $y$ stanowiłby wówczas następnik węzła $x$, co dowodzi poprawności algorytmu w~tym przypadku.

Wywołanie $\proc{RB-Search-At-Least}(T,\attrib{T}{root},\attrib{T}{nil},k)$ pozwala znaleźć węzeł o~najmniejszym kluczu w~drzewie $T$ nie mniejszym niż $k$.
Dla drzewa $T$ o~$n$ węzłach wywołanie to potrzebuje czasu $O(\lg n)$.

Następujący algorytm stanowi implementację nowej operacji.
Przyjmuje on dodatkowo jako parametr drzewo $T$ potrzebne do odwołania się do \attrib{T}{nil} i~wypisuje szukane klucze w~porządku niemalejącym.
Przechodzenie po kolejnych węzłach realizuje seria wywołań \proc{RB-Successor}, czyli procedury będącej adaptacją \proc{Tree-Successor} do drzew czerwono-czarnych.
\begin{codebox}
\Procname{$\proc{RB-Enumerate}(T,a,b)$}
\li $x\gets\proc{RB-Search-At-Least}(T,\attrib{T}{root},\attrib{T}{nil},a)$ \label{li:rb-enumerate-starting-node}
\li	\While $x\ne\attrib{T}{nil}$ i~$\attrib{x}{key}\le b$
\li		\Do wypisz \attrib{x}{key} \label{li:rb-enumerate-print}
\li			$x\gets\proc{RB-Successor}(T,x)$
		\End
\end{codebox}

Na podstawie \refExercise{12.2-8} ciąg wyszukiwań kolejnych $m$ następników w~drzewie czerwono-czarnym zajmuje czas $O(m+\lg n)$ i~takie jest też górne oszacowanie na czas działania tego algorytmu.
Dla dolnego oszacowania zauważmy, że pętla \kw{while} wykona zawsze $m$ iteracji.
Jeśli wywołanie z~linii \ref{li:rb-enumerate-starting-node} zwróciło \attrib{T}{nil} albo węzeł $x$ o~pustym prawym poddrzewie, to podczas tego wywołania musiała zostać pokonana ścieżka o~długości $\Omega(\lg n)$.
W~przeciwnym razie zwrócony został węzeł $x$ o~niepustym prawym poddrzewie.
Wówczas pierwsze wywołanie \proc{RB-Successor} przejdzie po ścieżce od $x$ do liścia będącego następnikiem $x$, a~więc oba wywołania sumarycznie pokonają ścieżkę długości $\Omega(\lg n)$.
Dolnym oszacowaniem na czas działania algorytmu jest zatem $\Omega(m+\lg n)$.
