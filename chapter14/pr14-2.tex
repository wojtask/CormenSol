\problem{Permutacja Józefa} %14-2

\subproblem %14-2(a)
Podamy algorytm, który będzie symulował opisany proces eliminowania osób.
Wykorzystamy do tego celu listę cykliczną zawierającą początkowo liczby całkowite, kolejno 1, 2, \dots, $n$.
Symulacja polega na przechodzeniu po liście i~zatrzymywaniu się na co \singledash{$m$}{tym} węźle.
Klucz tego węzła jest wypisywany i~usuwany z~listy, po czym kroki są powtarzane aż do wyczerpania elementów.
Dzięki temu, że lista jest cykliczna, nie musimy troszczyć się o~obsługę przypadku, w~którym osiągniemy ogon listy, zanim dotrzemy do następnego w~kolejności węzła do usunięcia.

Do efektywnego zaimplementowania opisanej symulacji wystarczy jednokierunkowa lista cykliczna.
Podczas przechodzenia po niej, należy utrzymywać wskaźnik $x$ na węzeł poprzedzający aktualny, dzięki czemu usuwanie aktualnego węzła będzie polegać na aktualizacji pola \attrib{x}{next} na \attribb{x}{next}{next} i~ewentualnie wskaźnika na głowę listy.
Usuwanie węzła z~listy odbywa się w~czasie $O(1)$, skąd całkowity czas działania symulacji wynosi
\[
	\sum_{k=1}^n(m+O(1)) = mn+O(n) = O(n),
\]
ponieważ $m$ jest stałą.

\subproblem %14-2(b)
Załóżmy, że w~pewnym momencie odliczanie zatrzymało się na osobie o~\singledash{$j$}{tym} największym numerze spośród pozostałych $k\le n$ osób.
Osoba ta zostaje usunięta, co powoduje dekrementację zmiennej $k$.
Numer kolejnej osoby w~permutacji wyznaczamy przez zwiększenie $j$ o~$m$, ale ponieważ liczba ta może przekroczyć liczbę pozostałych $k$ osób w~okręgu, to należy wziąć resztę z~jej dzielenia przez $k$, a~także uwzględnić fakt, że osoby numerowane są liczbami od 1 do $n$, poprzez odjęcie jedynki przed braniem reszty, a~następnie dodanie jej do wynikowej liczby.
W~rezultacie otrzymujemy, że następna osoba ma numer na pozycji $(j+m-1)\bmod k+1$ na uszeregowanej rosnąco liście numerów pozostałych osób.

Na podstawie powyższego opisu w~algorytmie wykorzystamy drzewo statystyk pozycyjnych.
Będziemy używać operacji \proc{OS-Insert} oraz \proc{OS-Delete}, które stanowią implementacje operacji słownikowych \proc{Insert} i~\proc{Delete} na drzewie statystyk pozycyjnych.
\begin{codebox}
\Procname{$\proc{Josephus}(n,m)$}
\li	utwórz puste drzewo statystyk pozycyjnych $T$
\li	\For $j\gets1$ \To $n$
\li		\Do utwórz węzeł $x$, w~którym $\attrib{x}{key}=j$
\li			$\proc{OS-Insert}(T,x)$
		\End
\li	$j\gets1$
\li	\For $k\gets n$ \Downto 1
\li		\Do $j\gets(j+m-1)\bmod k+1$
\li			$x\gets\proc{OS-Select}(\attrib{T}{root},j)$
\li			wypisz \attrib{x}{key}
\li			$\proc{OS-Delete}(T,x)$
		\End
\end{codebox}

Zbudowanie drzewa statystyk pozycyjnych $T$ zajmuje czas $O(n\lg n)$.
Następnie wykonywanych jest $n$ wywołań procedur działających na tym drzewie, z~których każde potrzebuje czasu $O(\lg n)$.
A~zatem czas działania algorytmu wynosi $O(n\lg n)$.
