\problem{Permutacja Józefa} %14-2

\subproblem %14-2(a)
Algorytm można zaimplementować jako symulację opisanego procesu eliminowania osób.
W~tym celu można wykorzystać dwukierunkową listę cykliczną zawierającą początkowo liczby całkowite, kolejno 1, 2, \dots, $n$.
Symulacja polega na usuwaniu co \singledash{$m$}{tego} elementu z~tej listy poprzez przejście $m$ razy wskaźnikami \id{next} po tej liście, a~następnie usunięcie aktualnego elementu.
Ostatni element, który pozostanie na liście, zostaje wypisany.

Operacja usuwania z~dwukierunkowej listy cyklicznej działa jak \proc{List-Delete}, ale polega na założeniu, że $\attrib{x}{prev}\ne\const{nil}$ oraz $\attrib{x}{next}\ne\const{nil}$.
Operacja składa się zatem jedynie z~wierszy 2 i~5 procedury \proc{List-Delete}, dlatego jej czasem działania jest oczywiście $O(1)$.
Stąd całkowity czas działania symulacji wynosi
\[
	\sum_{k=2}^n(m+O(1)) = m(n-1)+O(n) = O(n),
\]
ponieważ $m$ jest stałą.

\subproblem %14-2(b)
