\subchapter{Minimum i~maksimum}

\exercise %9.1-1
Wyznaczmy najpierw $\mu$ -- najmniejszy spośród $n$ elementów -- w~następujący sposób.
Elementy łączymy w~pary i~odrzucamy te, które są większe w~swoich parach (przyjmujemy, że element bez pary zwyczajnie przechodzi do kolejnego etapu), po czym wykonujemy te operacje rekurencyjnie dla pozostałego zbioru, aż do uzyskania jednego elementu, którym oczywiście będzie $\mu$.
Na każdym etapie spośród $k$ elementów do kolejnego przechodzi $\lceil k/2\rceil$, będzie więc $\lceil\lg n\rceil$ wywołań rekurencyjnych i~co najwyżej tylu testom poddamy element $\mu$.
Zauważmy, że każdy test odrzuca jeden element, wykonamy zatem dokładnie $n-1$ porównań.

Zastanówmy się teraz, który element zbioru wejściowego może być drugim najmniejszym.
Element ten został odrzucony po porównaniu go z~$\mu$, więc problem sprowadza się do wyznaczenia minimum zbioru tych elementów, które były testowane z~$\mu$.
Z~wcześniejszej obserwacji zbiór ten jest mocy $\lceil\lg n\rceil$, więc wystarczy $\lceil\lg n\rceil-1$ porównań do wyznaczenia jego minimum.

Ostatecznie dostajemy, że drugi najmniejszy element zbioru $n$\nbhyphen elementowego można wyznaczyć, wykonując $n+\lceil\lg n\rceil-2$ porównań.

\exercise %9.1-2
Zadanie rozwiążemy prostszą metodą, niż sugeruje nam to wskazówka.

Jeśli $n$ jest parzyste, to zgodnie z~podaną w~Podręczniku informacją, wykonywanych jest $3n/2-2$ porównań.
Ale dla parzystego $n$ zachodzi $3n/2=\lceil3n/2\rceil$, więc wzór na liczbę potrzebnych porównań przyjmuje postać $\lceil3n/2\rceil-2$.
Niech teraz $n$ będzie liczbą nieparzystą, czyli $n=2k+1$ dla pewnego całkowitego $k$.
Chcemy wykazać, że koniecznych jest $\lceil3n/2\rceil-2$ porównań, czyli $\lceil3k+3/2\rceil-2=3k+2-2=3k$.
Ale wynik ten zgadza się z~opisanym w~Podręczniku dolnym oszacowaniem na liczbę porównań dla nieparzystego $n$, bo $3\lfloor n/2\rfloor=3\lfloor k+1/2\rfloor=3k$.
