\subchapter{Minimum i~maksimum}

\exercise %9.1-1
Wyznaczmy najpierw $\mu$ -- najmniejszy spośród $n$ elementów -- w~następujący sposób.
Elementy łączymy w~pary i~odrzucamy te, które są większe w~swoich parach (przyjmujemy, że element bez pary zwyczajnie przechodzi do kolejnego etapu), po czym wykonujemy te operacje rekurencyjnie dla pozostałego zbioru, aż do uzyskania jednego elementu, którym oczywiście będzie $\mu$.
Na każdym etapie spośród $k$ elementów do kolejnego przechodzi $\lceil k/2\rceil$, będzie więc $\lceil\lg n\rceil$ wywołań rekurencyjnych i~co najwyżej tylu testom poddamy element $\mu$.
Zauważmy, że każdy test odrzuca jeden element, wykonamy zatem dokładnie $n-1$ porównań.

Zastanówmy się teraz, który element zbioru wejściowego może być drugim najmniejszym.
Element ten został odrzucony po porównaniu go z~$\mu$, więc problem sprowadza się do wyznaczenia minimum zbioru tych elementów, które były testowane z~$\mu$.
Z~wcześniejszej obserwacji zbiór ten jest mocy $\lceil\lg n\rceil$, więc wystarczy $\lceil\lg n\rceil-1$ porównań do wyznaczenia jego minimum.

Ostatecznie dostajemy, że drugi najmniejszy element zbioru $n$\nbhyphen elementowego można wyznaczyć, wykonując $n+\lceil\lg n\rceil-2$ porównań.

\exercise %9.1-2
Niech $m$ oznacza zbiór potencjalnych minimów, a~$M$ zbiór potencjalnych maksimów.
Początkowo przyjmiemy, że oba te zbiory są wejściowym zbiorem liczb i~naszym celem będzie zredukowanie ich do singletonów przez wykonywanie porównań na elementach.

Gdy z~porównania dwóch różnych elementów $a$, $b\in m\cap M$ okaże się, że $a<b$, to możemy usunąć $a$ ze zbioru $M$, a~$b$ ze zbioru $m$, redukując rozmiary obu zbiorów o~1.
Wykonując porównanie dwóch elementów należących do tego samego zbioru, redukujemy o~1 rozmiar tylko tego zbioru.
Widać zatem, że w~celu zminimalizowania łącznej liczby porównań opłacalne jest preferowanie porównań pierwszego typu.
Można wykonać ich $\lfloor n/2\rfloor$, zanim przecięcie $m\cap M$ stanie się zbiorem co najwyżej jednoelementowym.
Wtedy zbiory $m$ i~$M$ będą rozmiaru $\lceil n/2\rceil$ każdy i~ich redukcję do singletonów dokończymy, przeprowadzając porównania drugiego typu.
Stąd łączna liczba wykonanych porównań wynosi
\[
    \lfloor n/2\rfloor+(\lceil n/2\rceil-1)+(\lceil n/2\rceil-1) = n+\lceil n/2\rceil-2 = \lceil3n/2\rceil-2.
\]
