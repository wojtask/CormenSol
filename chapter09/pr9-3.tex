\problem{Małe statystyki pozycyjne} %9-3

\subproblem %9-3(a)
\note{W~treści problemu występuje błąd w~sformułowaniu rekurencji\/ $U_i(n)$.
Wartość\/ $T(n)$ powinna być zwracana dla\/ $i\ge n/2$, a~część rekurencyjna -- dla pozostałych wartości\/ $i$.}

\noindent Jeśli $i\ge n/2$, to w~celu znalezienia $i$\nbhyphen tej statystyki pozycyjnej tablicy wejściowej wywołujemy algorytm \proc{Select}.
Stąd $U_i(n)=T(n)$.

Załóżmy teraz, że $i<n/2$ i~że tablicą wejściową jest $A[p\twodots r]$, gdzie $n=r-p+1$.
Algorytm będzie wyszukiwał $i$\nbhyphen tą statystykę pozycyjną tablicy wejściowej, przy okazji partycjonując tablicę wokół szukanego elementu.
Niech $m=\lfloor n/2\rfloor$.
Rozważmy podtablice $L=A[p\twodots p+m-1]$ oraz $R=A[p+m\twodots r]$ tablicy wejściowej.
Dla każdego $j=0$, 1, \dots, $m-1$ porównamy elementy $A[p+j]$ oraz $A[p+m+j]$ i~ewentualnie zamienimy je, aby zachodziło $A[p+j]>A[p+m+j]$.
Wywołamy następnie nasz algorytm rekurencyjnie w~celu znalezienia $i$\nbhyphen tego najmniejszego elementu podtablicy $R$, czego efektem ubocznym będzie podział tej podtablicy względem szukanego elementu.
Po powrocie z~wywołania rekurencyjnego podtablica $L$ zostanie podzielona w~analogiczny sposób.
Dokładniej, dla każdej pary elementów $A[j]$ i~$A[k]$ zamienionych w~wywołaniu rekurencyjnym, gdzie $p+m\le j<k\le r$, zamienione zostaną również elementy $A[j-m]$ i~$A[k-m]$.
Dzięki temu, po wyznaczeniu $i$\nbhyphen tej statystyki pozycyjnej podtablicy $R$, fragmenty $A[p\twodots p+i-1]$ oraz $A[p+m\twodots p+m+i-1]$ będą zawierać po $i$ najmniejszych elementów ze swoich podtablic.
Najmniejszy element tablicy wejściowej znajduje się w~którymś z~tych fragmentów, więc wystarczy połączyć je w~tablicę $B[1\twodots2i]$, a~następnie zwrócić $\proc{Select}(B,1,2i,i)$.

Omówimy teraz pokrótce szczegóły implementacyjne.
Aby zrealizować równoczesny podział tablicy i~utrzymywać elementy w~parach, każda zamiana elementów tablicy będzie pociągać za sobą kaskadowo analogiczne zamiany w~innej części tablicy, zgodnie z~opisem z~poprzedniego akapitu.
W~celu zapewnienia, że każdy element podtablicy ma swojego odpowiednika, można wprowadzić sztuczne elementy równe $\infty$.
Za każdym razem, gdy rozmiar tablicy wejściowej $n$ jest nieparzysty, o~taki element rozszerzana byłaby podtablica $L$, a~także kaskadowo te części tablicy wejściowej na lewo od $L$, których to rozszerzenie doprowadzi do sytuacji, w~której niektóre elementy będą bez pary.
Zauważmy także, że tablica $B$, którą tworzymy pod koniec działania algorytmu, może w~rzeczywistości stanowić część tablicy $A$ -- wystarczy bowiem, aby był to spójny jej fragment, co da się uzyskać poprzez zamianę elementów z~podtablicy $A[p+m\twodots p+m+i-1]$ z~tymi należącymi do podtablicy $A[p\twodots p+i-1]$.

Sprawdzimy teraz, ile wynosi $U_i(n)$ w~przypadku, gdy $i<n/2$.
W~pierwszej fazie algorytm wykonuje $\lfloor n/2\rfloor$ porównań elementów pochodzących z~dwóch części tablicy wejściowej.
Wywołanie rekurencyjne wprowadza składnik $U_i(\lceil n/2\rceil)$, zaś procedura \proc{Select} wywołana na tablicy $B$ wykonuje $T(2i)$ porównań.
Stąd otrzymujemy
\[
    U_i(n) = \lfloor n/2\rfloor+U_i(\lceil n/2\rceil)+T(2i).
\]

\subproblem %9-3(b)
Stosując metodę podstawiania, wykażemy, że jeśli $i<n/2$, to
\[
	U_i(n) \le n+cT(2i)\lg(n/i),
\]
gdzie $c>0$ jest pewną stałą.

Jako przypadek bazowy indukcji będziemy rozważać wyrazy $U_i(n)$, w~których $2i<n\le4i$, skąd $n/4\le i<n/2$.
Wykażemy, że spełniają one podane oszacowanie.
Mamy:
\[
    U_i(n) = \lfloor n/2\rfloor+U_i(\lceil n/2\rceil)+T(2i) < n+T(\lceil n/2\rceil)+T(2i).
\]
Jeśli teraz ograniczymy powyższe wyrażenie od góry przez $n+cT(2i)\lg(n/i)$ i~rozwiążemy ze względu na $c$, to otrzymamy
\[
    c \ge \frac{1}{\lg(n/i)}\biggl(\frac{T(\lceil n/2\rceil)}{T(2i)}+1\biggr).
\]
Pierwszy czynnik po prawej stronie powyższej nierówności można ograniczyć od góry przez $1/\lg2=1$.
W~celu oszacowania drugiego czynnika zauważmy, że rekurencja $T(n)$ jest funkcją dodatnią klasy $\Theta(n)$, zatem istnieją stałe $d_1$, $d_2>0$, że dla każdego $n\ge1$ spełnione są nierówności $d_1n\le T(n)\le d_2n$.
Mamy zatem
\[
    \frac{T(\lceil n/2\rceil)}{T(2i)} \le \frac{d_2\lceil n/2\rceil}{2d_1i} < \frac{d_2n}{d_1n/2} = \frac{2d_2}{d_1},
\]
a~stąd wynika, że $c\ge2d_2/d_1+1$.
Pokazaliśmy, że $c$ istotnie może być stałą, a~zatem podstawa indukcji zachodzi.

W~drugim kroku indukcyjnym mamy $n>4i$ i~przyjmujemy założenie
\[
    U_i(\lceil n/2\rceil) \le \lceil n/2\rceil+cT(2i)\lg(\lceil n/2\rceil/i).
\]
Wówczas:
\begin{align*}
    U_i(n) &= \lfloor n/2\rfloor+U_i(\lceil n/2\rceil)+T(2i) \\
	&\le \lfloor n/2\rfloor+\lceil n/2\rceil+cT(2i)\lg(\lceil n/2\rceil/i)+T(2i) \\
	&= n+cT(2i)\lg\lceil n/2\rceil-cT(2i)\lg i+T(2i) \\
	&\le n+cT(2i)\lg n-cT(2i)\lg i.
\end{align*}
Ostatnia nierówność jest spełniona, jeżeli $cT(2i)\lg\lceil n/2\rceil+T(2i)\le cT(2i)\lg n$, co po uproszczeniu sprowadza się do warunku $c\lg\frac{n}{\lceil n/2\rceil}\ge1$.
Łatwo zauważyć, że aby go spełnić dla $n\ge5$, wystarczy przyjąć $c\ge\log_{5/3}2$, co kończy dowód oszacowania na $U_i(n)$.

\subproblem %9-3(c)
Ponieważ $i$ jest stałe, to $T(2i)$ również można potraktować jako wartość stałą.
Na mocy poprzedniego punktu mamy
\[
	U_i(n) = n+O(T(2i)\lg(n/i)) = n+O(\lg n-\lg i) = n+O(\lg n).
\]

\subproblem %9-3(d)
Dla $k>2$ oszacowanie wynika natychmiast z~części (b) po podstawieniu $i=n/k$, bo wtedy oczywiście $i<n/2$.

Jeśli $k=2$, to $i=n/2$ i~rozwiązaniem rekurencji $U_i(n)$ jest $U_i(n)=T(n)=O(n)$.
Teza w~tym przypadku przyjmuje postać $U_i(n)=n+O(T(n))$, co oczywiście jest spełnione, ponieważ $n+O(T(n))=O(n)$.
