\problem{Mediana ważona} %9-2
\note{Rozwiązanie korzysta z~definicji mediany ważonej (dolnej) z~tekstu oryginalnego.
Jedyną różnicą w~porównaniu z~definicją z~tłumaczenia jest to, że w~oryginale suma wag elementów mniejszych niż mediana ważona\/ $x_k$ jest ostro mniejsza niż\/ $1/2$.}

\subproblem %9-2(a)
Dla tak przyjętych wag elementów mamy
\[
	\sum_{x_i<x_k}w_i = \frac{k-1}{n} \quad\text{oraz}\quad \sum_{x_i>x_k}w_i = \frac{n-k}{n}.
\]
Jeśli ograniczymy obie sumy od góry przez $1/2$ -- pierwszą ostro, drugą nieostro -- to dostaniemy, że medianą ważoną elementów $x_1$, $x_2$, \dots, $x_n$ jest $x_k$, gdzie $n/2\le k<n/2+1$.
Ponieważ $k$ jest całkowite, to musi być $k=\lfloor(n+1)/2\rfloor$.
A~zatem $x_k$ jest również zwykłą medianą (dolną) elementów $x_1$, $x_2$, \dots, $x_n$.

\subproblem %9-2(b)
Po posortowaniu elementów wyznaczenie mediany ważonej odbywa się w~prosty sposób -- należy przeglądać elementy w~kolejności rosnącej i~sumować ich wagi aż do momentu, kiedy suma wag osiągnie lub przekroczy $1/2$.
Wówczas wystarczy zwrócić ostatnio przeglądany element.

Złożoność tej procedury zależy od efektywności sortowania, ale używając odpowiedniego algorytmu, jesteśmy w~stanie osiągnąć czas $O(n\lg n)$ w~pesymistycznym przypadku.

\subproblem %9-2(c)
Rozwiążemy problem metodą ,,dziel i~zwyciężaj''.
Przyjmijmy, że elementy przechowywane są w~tablicy $A[1\twodots n]$, a~ich wagi w~tablicy $w[1\twodots n]$, przy czym $w[i]$ jest wagą elementu $A[i]$ dla $i=1$, 2, \dots, $n$.
Najpierw dzielimy elementy względem ich mediany algorytmem \proc{Select}, w~którym dodatkowo przy każdej zmianie pozycji elementu $A[i]$ na indeks $j$ w~tablicy $A$, będziemy także zmieniać pozycję wagi $w[i]$ na indeks $j$ w~tablicy $w$.
Następnie wyznaczamy $W_L$ i~$W_R$ -- sumy wag elementów, odpowiednio, mniejszych i~większych od znalezionej mediany.
Jeśli $W_L<1/2$ i~$W_R<1/2$, to medianą ważoną jest zwykła mediana.
W~przeciwnym przypadku obszar tablicy wejściowej, który zawiera medianę wraz z~elementami o~większej sumie wag, przekazujemy do następnego wywołania rekurencyjnego.
\begin{codebox}
\Procname{$\proc{Weighted-Median}(A,w,p,r)$}
\li	\If $r-p+1\le2$ \label{li:weighted-median-boundary-case-begin}
\li		\Then
			\If $w[p]\ge w[r]$
\li				\Then \Return $A[p]$
\li				\Else \Return $A[r]$
				\End \label{li:weighted-median-boundary-case-end}
		\End
\li	podziel elementy w~$A[p\twodots r]$ i~ich wagi względem mediany tablicy $A[p\twodots r]$ \label{li:weighted-median-partition}
\li	$q\gets\lfloor(p+r)/2\rfloor$
\li	$W_L\gets0$
\li	\For $i\gets p$ \To $q-1$
\li		\Do $W_L\gets W_L+w[i]$
		\End
\li	$W_R\gets1-W_L-w[q]$
\li	\If $W_L<1/2$ i~$W_R<1/2$
\li		\Then \Return $A[q]$
		\End
\li	\If $W_L\ge1/2$
\li		\Then
			$w[q]\gets w[q]+W_R$
\li			\Return $\proc{Weighted-Median}(A,w,p,q)$
\li		\Else
			$w[q]\gets w[q]+W_L$
\li			\Return $\proc{Weighted-Median}(A,w,q,r)$
		\End
\end{codebox}

W~powyższym pseudokodzie rozważana jest tablica $A[p\twodots r]$ o~$n=r-p+1$ elementach.
Pierwszym krokiem jest sprawdzenie przypadków brzegowych -- jeśli $n=1$, to wystarczy zwrócić jedyny element wejściowy, a~jeśli $n=2$, to zwracany jest element o~większej wadze.
Działanie to implementujemy za pomocą pojedynczej instrukcji \kw{if} w~wierszach \doubledash{\ref{li:weighted-median-boundary-case-begin}}{\ref{li:weighted-median-boundary-case-end}}.
Następnie elementy wraz z~ich wagami są dzielone względem mediany (która po wykonaniu wiersza \ref{li:weighted-median-partition} znajduje się w~$A$ na pozycji $q=\lfloor(p+r)/2\rfloor$) i~obliczane są sumy wag obu części tego podziału.
W~zależności od tego, czy jedna z~tych wartości wynosi co najmniej $1/2$, zwracana jest wyznaczona wcześniej mediana, bądź algorytm zostaje wywoływany rekurencyjnie dla odpowiedniej części podziału.
Aby zachować warunek mówiący o~tym, że suma wag wszystkich elementów tablicy wejściowej wynosi 1, przed kolejnym wywołaniem algorytmu zwiększamy wagę mediany do odpowiedniej wartości -- $w[q]$ ,,kumuluje'' więc wagi tej części tablicy, którą odrzucamy.

Przy założeniu, że wiersz \ref{li:weighted-median-partition} w~najgorszym przypadku zajmuje czas $\Theta(n)$, pesymistyczny czas działania opisanego algorytmu spełnia równanie rekurencyjne $T(n)=T(\lfloor n/2\rfloor+1)+\Theta(n)$, którego rozwiązaniem jest $T(n)=\Theta(n)$.

\subproblem %9-2(d)
Niech $p_k$ będzie medianą ważoną danych punktów $p_1$, $p_2$, \dots, $p_n$ (które są liczbami rzeczywistymi) i~ich wag $w_1$, $w_2$, \dots, $w_n$.
Dla dowolnego punktu $p$ niech $f(p)=\sum_{i=1}^nw_i|p-p_i|$.
Szukamy takiego punktu $p$ spośród danych, dla którego $f(p)$ przyjmuje możliwie najmniejszą wartość.
Pokażemy, że szukanym punktem jest $p_k$.

Niech $p_l$ będzie dowolnym punktem różnym od $p_k$.
Zbadamy znak różnicy
\[
    f(p_l)-f(p_k) = \sum_{i=1}^nw_i|p_l-p_i|-\sum_{i=1}^nw_i|p_k-p_i| = \sum_{i=1}^nw_i(|p_l-p_i|-|p_k-p_i|).
\]

Niech $p_l<p_k$.
Zauważmy, że w~przypadku, gdy $p_l\le p_i<p_k$, zachodzi $|p_l-p_i|-|p_k-p_i|=p_i-p_l-p_k+p_i\ge2p_l-p_l-p_k=p_l-p_k$.
Mamy zatem
\begin{align*}
    f(p_l)-f(p_k) &= \sum_{p_i<p_l}w_i(p_l-p_i-p_k+p_i) \\
	&\quad {}+\!\!\!\sum_{p_l\le p_i<p_k}\!\!\!\!w_i(p_i-p_l-p_k+p_i) \\
	&\quad {}+\,\sum_{p_k\le p_i}w_i(p_i-p_l-p_i+p_k) \\
	&\ge (p_k-p_l)\biggl(\sum_{p_k\le p_i}w_i-\sum_{p_i<p_k}w_i\biggr).
\end{align*}
Pierwszy czynnik ostatniego iloczynu jest dodatni.
Na mocy faktu, że wszystkie wagi sumują się do 1 oraz z~definicji mediany ważonej, mamy
\[
    \sum_{p_k\le p_i}w_i-\sum_{p_i<p_k}w_i = 1-2\sum_{p_i<p_k}w_i > 1-2\cdot1/2 = 0,
\]
czyli drugi z~czynników także jest dodatni.
Pokazaliśmy tym samym, że dla dowolnego $p_l<p_k$ zachodzi $f(p_l)>f(p_k)$, co oznacza, że mediana ważona $p_k$ minimalizuje wartość funkcji $f$.

Podobnie wnioskujemy, gdy $p_l>p_k$, otrzymując
\[
    f(p_l)-f(p_k) \ge (p_k-p_l)\biggl(\sum_{p_k<p_i}w_i-\sum_{p_i\le p_k}w_i\biggr) = (p_k-p_l)\biggl(2\sum_{p_k<p_i}w_i-1\biggr).
\]
W~powyższym iloczynie pierwszy z~czynników jest ujemny, a~drugi można ograniczyć od góry (tym razem nieostro) przez $2\cdot1/2-1=0$.
A~zatem także w~tym przypadku mamy $f(p_l)>f(p_k)$, co kończy dowód twierdzenia.

\subproblem %9-2(e)
Niech $p_i=\langle x_i,y_i\rangle$, gdzie $i=1$, 2, \dots, $n$, będą punktami wejściowymi w~\singledash{2}{wymiarowym} problemie lokalizacji urzędu pocztowego z~metryką miejską.
Jego rozwiązaniem jest taki punkt $p=\langle x_p,y_p\rangle$, dla którego suma $\sum_{i=1}^nw_i(|x_p-x_i|+|y_p-y_i|)$ przyjmuje najmniejszą możliwą wartość.
Zauważmy, że
\[
    \min_{\langle x_p,y_p\rangle\in\mathbb{R}\times\mathbb{R}}\biggl(\sum_{i=1}^nw_i(|x_p-x_i|+|y_p-y_i|)\biggr) = \min_{x_p\in\mathbb{R}}\biggl(\sum_{i=1}^nw_i|x_p-x_i|\biggr)+\min_{y_p\in\mathbb{R}}\biggl(\sum_{i=1}^nw_i|y_p-y_i|\biggr).
\]
A~zatem w~celu wyznaczenia optymalnego punktu $p$ wystarczy znaleźć jego współrzędne niezależnie jako rozwiązania \singledash{1}{wymiarowej} wersji problemu.
Na podstawie poprzedniego punktu zadanie to sprowadza się do wyznaczenia mediany ważonej $x_p$ spośród elementów $x_1$, $x_2$, \dots, $x_n$ o~wagach $w_1$, $w_2$, \dots, $w_n$ i~analogicznie mediany ważonej $y_p$ spośród elementów $y_1$, $y_2$, \dots, $y_n$ o~tych samych wagach, a~następnie zwróceniu pary $\langle x_p,y_p\rangle$.

Zauważmy, że oba ciągi współrzędnych, które przeszukujemy celem znalezienia ich median ważonych, mogą zawierać powtarzające się elementy -- w~szczególności wszystkie punkty mogą mieć równe pierwsze (lub drugie) współrzędne.
Jedynym miejscem w~procedurze \proc{Weighted-Median} z~części (c), gdzie mają znaczenie wartości elementów, jest linia \ref{li:weighted-median-partition}, w~której dokonywany jest podział elementów algorytmem opartym o~\proc{Select}.
Okazuje się jednak, o~czym łatwo się przekonać, że algorytm ten działa poprawnie nawet wówczas, gdy elementy w~tablicy wejściowej powtarzają się -- zmienić się może jedynie jego czas działania, który w~przypadku pesymistycznym wzrasta do kwadratowego.
A~zatem procedura z~punktu (c) może posłużyć jako narzędzie do szukania median ważonych współrzędnych punktów w~algorytmie rozwiązującym rozważany wariant problemu lokalizacji urzędu pocztowego.
