\problem{Mediana ważona} %9-2
\note{Rozwiązanie korzysta z~definicji mediany ważonej (dolnej) z~tekstu oryginalnego.
Jedyną różnicą w~porównaniu z~definicją z~tłumaczenia jest to, że w~oryginale suma wag elementów mniejszych niż mediana ważona\/ $x_k$ jest ostro mniejsza niż\/ $1/2$.}
\bignegskip

\subproblem %9-2(a)
Dla tak przyjętych wag elementów mamy
\[
	\sum_{x_i<x_k}w_i = \frac{k-1}{n} \quad\text{oraz}\quad \sum_{x_i>x_k}w_i = \frac{n-k}{n}.
\]
Jeśli ograniczymy obie sumy od góry przez $1/2$ -- pierwszą ostro, drugą nieostro -- to dostaniemy, że medianą ważoną elementów $x_1$, $x_2$, \dots, $x_n$ jest $x_k$, gdzie $n/2\le k<n/2+1$.
Ponieważ $k$ jest całkowite, to musi być $k=\lceil n/2\rceil=\lfloor(n+1)/2\rfloor$.
A~zatem $x_k$ jest również zwykłą medianą (dolną) elementów $x_1$, $x_2$, \dots, $x_n$.

\subproblem %9-2(b)
Po posortowaniu elementów wyznaczenie mediany ważonej odbywa się w~prosty sposób -- należy przeglądać elementy w~kolejności rosnącej i~sumować ich wagi aż do momentu, kiedy suma wag osiągnie lub przekroczy $1/2$.
Wówczas wystarczy zwrócić ostatnio przeglądany element.

Złożoność tej procedury zależy od efektywności sortowania, ale używając odpowiedniego algorytmu, jesteśmy w~stanie osiągnąć czas $O(n\lg n)$ w~pesymistycznym przypadku.

\subproblem %9-2(c)
Rozważmy następujący algorytm rekurencyjny rozwiązujący zadany problem:
\begin{codebox}
\Procname{$\proc{Weighted-Median}(A,w,p,r)$}
\li	\If $p=r$
\li		\Then \Return $A[p]$
		\End
\li	podziel elementy w~$A[p\twodots r]$ i~ich wagi w~$w[p\twodots r]$ względem mediany tablicy $A[p\twodots r]$ \label{li:weighted-median-partition}
\li	$q\gets\lfloor(p+r)/2\rfloor$
\li	$W_L\gets0$
\li	\For $i\gets p$ \To $q-1$
\li		\Do $W_L\gets W_L+w[i]$
		\End
\li	$W_H\gets1-W_L-w[q]$
\li	\If $W_L<1/2$ i~$W_H\le1/2$
\li		\Then \Return $A[q]$
		\End
\li	\If $W_L\ge1/2$
\li		\Then $w[q]\gets w[q]+W_H$
\li			\Return $\proc{Weighted-Median}(A,w,p,q)$
\li		\Else $w[q]\gets w[q]+W_L$
\li			\Return $\proc{Weighted-Median}(A,w,q,r)$
		\End
\end{codebox}
Algorytm działa na elementach znajdujących się w~podtablicy $A[p\twodots r]$ oraz ich wagach w~podtablicy $w[p\twodots r]$, gdzie $w[i]$ jest wagą elementu $A[i]$ dla $i=p$, $p+1$, \dots, $r$.
W~przypadku brzegowym, gdy $p=r$, zwracany jest jedyny element tablicy.
Jeśli $p<r$, to elementy wraz z~ich wagami są dzielone względem ich mediany, do realizacji czego używany jest zmodyfikowany algorytm \proc{Select}, w~którym podczas podziału elementów jednocześnie wykonywany jest analogiczny podział ich wag.
Dokładniej, wraz z~zamianą elementów $A[i]$, $A[j]$, zamieniane są także wagi $w[i]$, $w[j]$.
W~wyniku wykonania podziału mediana podtablicy $A[p\twodots r]$ znajduje się na pozycji $q=\lfloor(p+r)/2\rfloor$.
Na podstawie definicji mediany ważonej, jeżeli suma wag elementów z~dolnej części podziału $W_L$ jest mniejsza niż $1/2$ i~suma wag elementów z~górnej części podziału $W_H$ jest mniejsza bądź równa $1/2$, to medianą ważoną jest mediana $A[q]$.
W~przeciwnym przypadku część podziału o~elementach z~większą sumą wag, razem z~medianą, jest przekazywana do kolejnego wywołania rekurencyjnego.
Algorytm utrzymuje stałą sumę wag przetwarzanych elementów wynoszącą 1, dlatego tuż przed kolejnym zejściem rekurencyjnym waga mediany zostaje zwiększona do odpowiedniej wartości -- $w[q]$ ,,kumuluje'' więc wagi odrzucanej części podziału.

Aby znaleźć medianę ważoną elementów $x_1$, $x_2$, \dots, $x_n$ zapisanych w~tablicy $A[1\twodots n]$ o~wagach $w_1$, $w_2$, \dots, $w_n$ zapisanych w~tablicy $w[1\twodots n]$, należy wywołać $\proc{Weighted-Median}(A,w,1,n)$.
W~wyniku uruchomienia algorytmu kolejność elementów w~tablicach $A$ i~$w$ może zostać zmieniona, a~suma wag w~tablicy $w$ może nie sumować się do 1.

Przy założeniu, że wykonanie podziału w~wierszu \ref{li:weighted-median-partition} w~najgorszym przypadku zajmuje czas $\Theta(n)$, pesymistyczny czas działania opisanego algorytmu spełnia równanie rekurencyjne $T(n)=T(\lfloor n/2\rfloor+1)+\Theta(n)$, którego rozwiązaniem jest $T(n)=\Theta(n)$.

\subproblem %9-2(d)
W~wariancie 1\nbhyphen wymiarowym minimalizowana jest suma $\sigma(p)=\sum_{i=1}^nw_i|p-p_i|$.
Pokażemy, że szukanym minimum funkcji $\sigma$ jest tak naprawdę mediana ważona $p_k$ punktów $p_1$, $p_2$, \dots, $p_n$.

Niech $p_l\ne p_k$.
Zbadamy znak różnicy
\[
    \sigma(p_k)-\sigma(p_l) = \sum_{i=1}^nw_i|p_k-p_i|-\sum_{i=1}^nw_i|p_l-p_i| = \sum_{i=1}^nw_i(|p_k-p_i|-|p_l-p_i|).
\]
Jeżeli $p_l<p_k$, to dla każdego $p_i$ spełniającego $p_l\le p_i<p_k$ zachodzi
\[
    |p_k-p_i|-|p_l-p_i|=p_k-p_i-p_i+p_l\le p_k+p_l-2p_l=p_k-p_l.
\]
Mamy zatem
\begin{align*}
    \sigma(p_k)-\sigma(p_l) &= \sum_{p_i<p_l}w_i(p_k-p_i-p_l+p_i) \\
	&\quad {}+\!\!\!\sum_{p_l\le p_i<p_k}\!\!\!\!w_i(|p_k-p_i|-|p_l-p_i|) \\
	&\quad {}+\,\sum_{p_k\le p_i}w_i(p_i-p_k-p_i+p_l) \\
	&\le (p_k-p_l)\biggl(\sum_{p_i<p_k}w_i-\sum_{p_k\le p_i}w_i\biggr).
\end{align*}
Oczywiście czynnik $(p_k-p_l)$ jest dodatni.
Ponieważ suma wszystkich wag wynosi 1, to z~definicji mediany ważonej mamy
\[
    \sum_{p_i<p_k}w_i-\sum_{p_k\le p_i}w_i = 2\sum_{p_i<p_k}w_i-1 < 2\cdot1/2-1 = 0,
\]
czyli drugi z~czynników, a w rezultacie także cały iloczyn, jest ujemny.
Pokazaliśmy tym samym, że dla dowolnego $p_l<p_k$ zachodzi $\sigma(p_k)<\sigma(p_l)$, co oznacza, że mediana ważona $p_k$ minimalizuje wartość funkcji $\sigma$.

Podobnie wnioskujemy, gdy $p_l>p_k$:
\[
    \sigma(p_k)-\sigma(p_l) \le (p_l-p_k)\biggl(\sum_{p_i>p_k}w_i-\sum_{p_i\le p_k}w_i\biggr) = (p_l-p_k)\biggl(2\sum_{p_i>p_k}w_i-1\biggr).
\]
W~ostatnim iloczynie pierwszy z~czynników jest dodatni, a~drugi można ograniczyć od góry (tym razem nieostro) przez $2\cdot1/2-1=0$.
A~zatem także w~tym przypadku mamy $\sigma(p_k)<\sigma(p_l)$.

\subproblem %9-2(e)
\note{Oryginalna treść nie wymaga zaprojektowania algorytmu, tylko znalezienia optymalnego rozwiązania problemu.}

\noindent Niech $p_i=\langle x_i,y_i\rangle$, gdzie $i=1$, 2, \dots, $n$, będą punktami wejściowymi w~2\nbhyphen wymiarowym problemie lokalizacji urzędu pocztowego z~metryką miejską.
Jego rozwiązaniem jest taki punkt $p=\langle x_p,y_p\rangle$, dla którego suma $\sum_{i=1}^nw_i(|x_p-x_i|+|y_p-y_i|)$ przyjmuje najmniejszą możliwą wartość.
Zauważmy, że
\[
    \min_{\langle x_p,y_p\rangle\in\mathbb{R}\times\mathbb{R}}\biggl(\sum_{i=1}^nw_i(|x_p-x_i|+|y_p-y_i|)\biggr) = \min_{x_p\in\mathbb{R}}\biggl(\sum_{i=1}^nw_i|x_p-x_i|\biggr)+\min_{y_p\in\mathbb{R}}\biggl(\sum_{i=1}^nw_i|y_p-y_i|\biggr).
\]
A~zatem każda ze współrzędnych optymalnego punktu $p$ stanowi rozwiązanie 1\nbhyphen wymiarowej wersji problemu.
Innymi słowy, rozwiązaniem niniejszego wariantu problemu jest punkt $p=\langle x_p,y_p\rangle$, gdzie $x_p$ to mediana ważona elementów $x_1$, $x_2$, \dots, $x_n$ o~wagach $w_1$, $w_2$, \dots, $w_n$, a~$y_p$ to mediana ważona elementów $y_1$, $y_2$, \dots, $y_n$ o~tych samych wagach.
