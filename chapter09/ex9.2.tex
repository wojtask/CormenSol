\subchapter{Wybór w~oczekiwanym czasie liniowym}

\exercise %9.2-1
Zakładamy, że parametr $i$ jest liczbą całkowitą spełniającą nierówności $1\le i\le r-p+1$.
Wywołanie procedury \proc{Randomized-Partition} w~wierszu 3 zwraca liczbę całkowitą $q$ taką, że $p\le q\le r$.
Dla liczby $k$ wyznaczonej w~kolejnym wierszu zachodzi więc $1\le k\le r-p+1$.
Wywołanie rekurencyjne w~wierszu 8 nastąpi dla tablicy długości 0, jeśli $i<k$ i~$q=p$, ale wówczas $k=1$, co jest sprzeczne z~założeniem o~parametrze $i$.
Podobnie w~wierszu 9 do wywołania rekurencyjnego zostanie przekazana pusta tablica, o~ile $i>k$ i~$q=r$, lecz wtedy $k=r-p+1$ i~również w~tym przypadku dochodzimy do sprzeczności.

\exercise %9.2-2
Czas działania procedury \proc{Randomized-Partition} dla tablicy o~mniej niż $n$ elementach jest niezależny od tego, jak została podzielona tablica o~$n$ elementach w~poprzednim wywołaniu rekurencyjnym.
Jest tak między innymi dlatego, że żaden poziom rekursji nie przekazuje do następnego poziomu informacji o~tym, na jakim fragmencie tablicy działa.

\exercise %9.2-3
Po dokonaniu oczywistych zmian w~oryginalnej procedurze, otrzymujemy następujący pseudokod:
\begin{codebox}
\Procname{$\proc{Iterative-Randomized-Select}(A,p,r,i)$}
\li	\While $p<r$
\li		\Do $q\gets\proc{Randomized-Partition}(A,p,r)$
\li			$k\gets q-p+1$
\li			\If $i=k$
\li				\Then \Return $A[q]$
				\End
\li			\If $i<k$
\li				\Then $r\gets q-1$
\li				\Else $p\gets q+1$
\li					$i\gets i-k$
				\End
		\End
\li	\Return $A[p]$
\end{codebox}

\exercise %9.2-4
W~przypadku szukania elementu najmniejszego pesymistyczny przypadek dzielenia podtablicy występuje, gdy na element rozdzielający wybierany jest za każdym razem jej największy element.
Kolejne wywołania rekurencyjne zmniejszają wówczas obszar poszukiwań o~1, jednocześnie umieszczając na końcu tablicy elementy w~kolejności rosnącej, w~wyniku czego, jako efekt uboczny, tablica zostaje posortowana.
