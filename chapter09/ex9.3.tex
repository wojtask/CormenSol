\subchapter{Wybór w~pesymistycznym czasie liniowym}
\note{Ponieważ nie jest to sprecyzowane w~Podręczniku, w~rozwiązaniach przyjmiemy, że parametry procedury \proc{Select} odpowiadają parametrom procedury \proc{Randomized-Select}, czyli są nimi kolejno: tablica reprezentująca zbiór wejściowy, początkowy i~końcowy indeks przeszukiwanego fragmentu tej tablicy oraz numer szukanej statystyki pozycyjnej.}

\noindent Algorytm \proc{Select} działa poprawnie także dla zbioru z~powtórzeniami, ale wtedy w~przypadku pesymistycznym czas działania algorytmu rośnie do $O(n^2)$.
Przypadek taki ma miejsce, gdy w~zbiorze wejściowym wielokrotnie powtórzona jest mediana median $x$, co skutkuje tworzeniem niezrównoważonych podziałów tablicy wejściowej w~kroku 4 algorytmu i~przekazywaniem niewiele mniejszych tablic do kolejnych wywołań rekurencyjnych.

Jednym ze sposobów utrzymania liniowej złożoności pesymistycznej jest użycie w~kroku 4 zmodyfikowanej metody \proc{Partition}, która dzieli tablicę wejściową względem $x$ na trzy spójne fragmenty: dolny z~elementami mniejszymi od $x$, środkowy z~elementami równymi $x$ oraz górny z~elementami większymi od $x$.
Niech $k_1$ będzie o~jeden większe niż liczba elementów w~dolnej części podziału i~niech $k_2$ będzie o~jeden większe niż suma elementów w~dolnej i~środkowej części podziału, czyli w~górnej części jest $n-k_2$ elementów.
Krok 5 zastępujemy poniższym:
\begin{quote}
	Jeśli $k_1\le i\le k_2$, to zwróć $x$.
	W~przeciwnym razie wywołaj rekurencyjnie procedurę \proc{Select}, by wyznaczyć $i$\nbhyphen ty najmniejszy element w~dolnej części podziału, jeśli $i<k_1$, lub $(i-k_2)$\nbhyphen ty najmniejszy element w~górnej części, jeśli $i>k_2$.
\end{quote}

Usprawnienie to będzie nam potrzebne w~rozwiązaniach niektórych zadań, gdzie nie ma założenia o~parami różnych elementach w~zbiorze wejściowym dla algorytmu \proc{Select}.

\exercise %9.3-1
Dokonajmy analizy algorytmu \proc{Select} w~przypadku, gdy elementy są dzielone na grupy 7\nbhyphen elementowe.
Wówczas w~co najmniej połowie spośród $\lceil n/7\rceil$ grup są po 4 elementy większe od $x$, oprócz jednej grupy o~mniej niż 7 elementach, jeśli $n$ nie jest wielokrotnością 7, i~jednej grupy zawierającej sam element $x$.
Odliczając te dwie grupy, dostajemy, że liczba elementów większych od $x$ wynosi co najmniej
\[
	4\biggl(\biggl\lceil\frac{1}{2}\Bigl\lceil\frac{n}{7}\Bigr\rceil\biggr\rceil-2\biggr) \ge \frac{2n}{7}-8.
\]
Podobnie wykazuje się, że liczba elementów mniejszych od $x$ wynosi co najmniej $2n/7-8$.
Stąd w~kroku 5 algorytm wywoła się rekurencyjnie dla co najwyżej $5n/7+8$ elementów.

Znajdziemy zależność rekurencyjną $T(n)$ opisującą czas działania tego algorytmu w~przypadku pesymistycznym.
Tak jak w~oryginalnej wersji kroki 1, 2 i~4 zajmują czas $O(n)$.
Krok 3 zajmuje czas $T(\lceil n/7\rceil)$, a~krok 5 -- co najwyżej czas $T(5n/7+8)$, przy założeniu, że $T$ jest funkcją niemalejącą.
Rekurencja przyjmuje więc postać
\[
	T(n) = \begin{cases}
		O(1), & \text{jeśli $n<d$}, \\
		T(\lceil n/7\rceil)+T(5n/7+8)+O(n), & \text{jeśli $n\ge d$},
	\end{cases}
\]
gdzie $d>0$ jest pewną stałą, którą wyznaczymy później.

Wykażemy metodą podstawiania, że $T(n)=O(n)$.
Zachodzi oczywiście $T(n)\le cn$ dla pewnej stałej $c>0$ oraz wszystkich $n<d$.
Załóżmy teraz, że $n\ge d$ i~że $T(k)\le ck$ dla pewnej stałej $c>0$ i~wszystkich $k<n$.
Dla pewnej stałej $a>0$ zachodzi wówczas
\begin{align*}
	T(n) &\le c\lceil n/7\rceil+c(5n/7+8)+an \\
	&\le cn/7+c+5cn/7+8c+an \\
	&= 6cn/7+9c+an \\
	&= cn+(-cn/7+9c+an) \\
	&\le cn,
\end{align*}
o~ile składnik $-cn/7+9c+an$ jest niedodatni.
Dla $n>63$ warunek ten jest spełniony, kiedy $c\ge7a(n/(n-63))$.
Jeśli z~kolei $n\ge126$, to $n/(n-63)\le2$ i~wtedy musi być $c\ge14a$.
Można zatem przyjąć $d=126$, co kończy dowód.

Pokażemy teraz, że jeśli podział będzie dokonywany na grupy 3\nbhyphen elementowe, to czas działania tak zmodyfikowanego algorytmu \proc{Select} będzie wyższy od liniowego.
W~takim podziale w~co najmniej połowie z~$\lceil n/3\rceil$ grup są po 2 elementy większe (mniejsze) od $x$, oprócz ewentualnie jednej niepełnej grupy i~jednej grupy zawierającej sam element $x$.
Liczba elementów większych (mniejszych) od $x$ wynosi zatem co najmniej
\[
	2\biggl(\biggl\lceil\frac{1}{2}\Bigl\lceil\frac{n}{3}\Bigr\rceil\biggr\rceil-2\biggr) \ge \frac{n}{3}-4.
\]
Prowadzi to do następującej rekurencji na czas działania algorytmu w~najgorszym przypadku:
\[
	T(n) = \begin{cases}
		O(1), & \text{jeśli $n<d$}, \\
		T(\lceil n/3\rceil)+T(2n/3+4)+O(n), & \text{jeśli $n\ge d$},
	\end{cases}
\]
gdzie $d>0$ jest pewną stałą.

Użyjemy metody podstawiania do pokazania, że $T(n)=\omega(n)$, a~dokładniej, że dla dowolnej stałej $c>0$ i~odpowiednio dużych $n$ zachodzi $T(n)>cn$.
Ustalmy $c$.
Załóżmy, że istnieje stała $n_0\ge d$ taka, że dla wszystkich $n\ge n_0$ prawdziwe są nierówności
\[
	T(\lceil n/3\rceil) > c\lceil n/3\rceil \quad\text{oraz}\quad T(2n/3+4) > c(2n/3+4).
\]
Dla $n\ge n_0$ mamy:
\[
	T(n) > c\lceil n/3\rceil+c(2n/3+4) \ge cn/3+2cn/3+4c = cn+4c > cn.
\]

\exercise %9.3-2
Z~analizy algorytmu \proc{Select} wynika, że zarówno liczba elementów większych od $x$, jak i~liczba elementów mniejszych od $x$, wynosi co najmniej $3n/10-6$.
Gdy $n\ge140$, to
\[
	\frac{3n}{10}-6 = \frac{n}{4}+\frac{n}{20}-6 \ge \frac{n}{4}+\frac{140}{20}-6 = \frac{n}{4}+1 \ge \Bigl\lceil\frac{n}{4}\Bigr\rceil.
\]

\exercise %9.3-3
\note{W~rozwiązaniu przyjmujemy założenie, że elementy tablicy wejściowej są parami różne.}

\noindent W~algorytmie quicksort zamiast użycia procedury \proc{Partition} do podziału tablicy wejściowej wykorzystamy algorytm \proc{Select} do odszukania mediany elementów wejściowych.
Dzięki temu wykonywane będą najbardziej zrównoważone podziały.
Pesymistyczny czas algorytmu quicksort po takiej modyfikacji sprowadzi się do rekurencji $T(n)=2T(\lfloor n/2\rfloor)+O(n)$, której rozwiązaniem jest $T(n)=O(n\lg n)$.

\exercise %9.3-4
Elementy zbioru wejściowego utożsamimy z~wierzchołkami grafu skierowanego $G$, a~wyniki porównań na tych elementach uzyskane przez algorytm -- z~krawędziami $G$.
Dokładniej, jeśli algorytm porównuje elementy $x$, $y$ i~$x<y$, to w~grafie $G$ istnieje krawędź $\langle x,y\rangle$.
Dla wierzchołka $x$ grafu $G$ przez $\mathrm{pred}(x)$ oznaczymy liczbę wierzchołków, z~których osiągalny jest $x$, a~przez $\mathrm{succ}(x)$ -- liczbę wierzchołków osiągalnych z~$x$.

Z~założenia algorytm poprawnie odnajduje $i$\nbhyphen ty najmniejszy element $x_i$ zbioru tylko na podstawie porównań elementów.
Z~definicji $i$\nbhyphen tej statystyki pozycyjnej wynika, że $\mathrm{pred}(x_i)\le i-1$ oraz $\mathrm{succ}(x_i)\le n-i$.
Jeśli $\mathrm{pred}(x_i)<i-1$ lub $\mathrm{succ}(x_i)<n-i$, to istnieje wierzchołek $z$, który jednocześnie nie jest osiągalny z~$x_i$ i~z~którego $x_i$ nie jest osiągalny.
Oznacza to, że algorytm nie rozstrzygnął, czy $x_i<z$, czy $z<x_i$, bezpośrednio je porównując lub pośrednio z~przechodniości relacji $<$, dlatego wskazana przez niego $i$\nbhyphen ta statystyka pozycyjna może być niewłaściwa.
Musi więc zachodzić $\mathrm{pred}(x_i)=i-1$ i~$\mathrm{succ}(x_i)=n-i$.
A~zatem po wyznaczeniu $i$\nbhyphen tego najmniejszego elementu zbioru algorytm dysponuje już wszystkimi informacjami, aby jednoznacznie wskazać $i-1$ najmniejszych oraz $n-i$ największych elementów zbioru bez konieczności wykonywania dodatkowych porównań.

\exercise %9.3-5
Zmodyfikujemy procedurę \proc{Randomized-Select}, dokonując zmiany w~wierszu 3.
Zamiast wywoływać \proc{Randomized-Partition}, znajdziemy medianę $x$ elementów z~tablicy wejściowej za pomocą danej ,,czarnej skrzynki'', po czym podzielimy tablicę względem $x$.
W~każdym wywołaniu rekurencyjnym będzie wówczas dokonywany najbardziej zrównoważony podział, więc czas działania algorytmu wyboru w~przypadku pesymistycznym przyjmie postać $T(n)=T(\lfloor n/2\rfloor)+O(n)$.
Rozwiązaniem tej rekurencji jest $T(n)=O(n)$.

\exercise %9.3-6
Oznaczmy przez $S$ badany zbiór, a~przez $Q_k(S)$ -- zbiór jego kwantyli rzędu $k$, gdzie $1\le k\le|S|+1$.
Zbiór kwantyli nie jest jednoznacznie zdefiniowany, opiszemy jednak jeden z~poprawnych sposobów jego konstrukcji.
Oczywiście zbiór kwantyli rzędu $k=1$ jest pusty.
Jeśli $k$ jest parzyste, to w~skład zbioru kwantyli rzędu $k$ wchodzi mediana $m$ zbioru $S$ oraz kwantyle rzędu $k/2$ zbiorów $\{\,x\in S:x<m\,\}$ i~$\{\,x\in S:x>m\,\}$.
W~przypadku, gdy $k$ jest liczbą nieparzystą większą od 1, a~$m_1$, $m_2$ kwantylami rzędu $k$ najbliższymi mediany $m$, to zbiór kwantyli rzędu $k$, oprócz $m_1$ i~$m_2$, zawiera także kwantyle rzędu $(k-1)/2$ zbiorów $\{\,x\in S:x<m_1\,\}$ i~$\{\,x\in S:x>m_2\,\}$.

Zbiór $S$ reprezentujemy jako tablicę $A$ parami różnych elementów.
Na podstawie powyższego opisu otrzymujemy następujący algorytm oparty o~metodę ,,dziel i~zwyciężaj'':
\begin{codebox}
\Procname{$\proc{Quantiles}(A,p,r,k)$}
\li	\If $k=1$
\li		\Then \Return $\emptyset$
		\End
\li	$n\gets r-p+1$
\li	$q_1\gets p+\lfloor\lfloor k/2\rfloor(n/k)\rfloor$ \>\>\>\>\>\>\Comment pozycje zajmowane przez $\lfloor k/2\rfloor$\nbhyphen ty i~$\lceil k/2\rceil$\nbhyphen ty kwantyl rzędu $k$
\li	$q_2\gets p+\lfloor\lceil k/2\rceil(n/k)\rfloor$ \>\>\>\>\>\>\>podtablicy $A[p\twodots r]$, gdyby została ona uporządkowana
\li	$\proc{Select}(A,p,r,q_1-p+1)$
\li	\If $q_1\ne q_2$
\li		\Then $\proc{Select}(A,q_1+1,r,q_2-q_1)$ \label{li:quantiles-second-select}
		\End
\li	$L\gets\proc{Quantiles}(A,p,q_1-1,\lfloor k/2\rfloor)$
\li	$R\gets\proc{Quantiles}(A,q_2+1,r,\lfloor k/2\rfloor)$
\li	\Return $L\cup\{A[q_1],A[q_2]\}\cup R$
\end{codebox}
Aby wyznaczyć $Q_k(S)$, należy wywołać $\proc{Quantiles}(A,1,n,k)$, gdzie $n=|S|$.

W~pseudokodzie nie rozdzielamy przypadków explicite ze względu na parzystość $k$.
Po wyznaczeniu długości przetwarzanego fragmentu $A[p\twodots r]$ obliczane są pozycje $q_1$ i~$q_2$ zajmowane przez kwantyle $m_1$ i~$m_2$ (jeśli $k$ jest nieparzyste) lub medianę $m$ (jeśli $k$ jest parzyste -- wówczas $q_1=q_2$), gdyby uporządkować podtablicę $A[p\twodots r]$.
Następnie korzystamy z~algorytmu \proc{Select}, aby podzielić tę podtablicę względem elementu $m_1$ (lub $m$ dla parzystego $k$).
Jeśli $k$ jest nieparzyste (czyli $q_1\ne q_2$), to w~wierszu \ref{li:quantiles-second-select} fragment tablicy $A$ zawierający elementy większe niż $m_1$ dzielimy względem $m_2$.
W~tym momencie fragment $A[p\twodots q_1-1]$ składa się z~elementów mniejszych niż $A[q_1]=m_1$, a~fragment $A[q_2+1\twodots r]$ z~elementów większych niż $A[q_2]=m_2$.
Teraz wystarczy znaleźć kwantyle rzędu $\lfloor k/2\rfloor$ w~obu tych fragmentach i~zwrócić wynikowy zbiór.

Nierekurencyjna część algorytmu zajmuje czas $O(n)$, zatem rekursja opisująca pesymistyczny czas jego działania przyjmuje postać $T(n,k)=2T(\lfloor n/2\rfloor,\lfloor k/2\rfloor)+O(n)$.
Jej rozwiązaniem jest $T(n,k)=O(n\lg k)$, o~czym można się przekonać, stosując na przykład metodę drzewa rekursji.

\exercise %9.3-7
Pojęcie ,,bliskości'' można rozumieć jako możliwie najmniejszą odległość w~sensie pozycji w~uporządkowanym zbiorze albo w~sensie wartości bezwzględnej.
Podamy algorytmy wyznaczające liczby najbliższe medianie zbioru przy użyciu zarówno pierwszej, jak i~drugiej definicji.
Jak się okaże, w~obu przypadkach istnieją algorytmy o~złożoności $O(n)$.

W~pierwszej definicji $k$ liczbami najbliższymi medianie zbioru $S$ są elementy oddalone od mediany o~nie więcej niż około $(k-1)/2$ pozycji po ustawieniu elementów $S$ w~kolejności rosnącej.
Zbiór $S$ będzie reprezentowany w~algorytmie jako $n$\nbhyphen elementowa tablica $A$.
W~uporządkowanej tablicy wejściowej pozycją mediany (dolnej) jest $m=\lfloor(n+1)/2\rfloor$, a~szukani ,,sąsiedzi'' mediany zajmują pozycje od $l=m-\lfloor(k-1)/2\rfloor$ do $l+k-1$.
Tablicy $A$ nie trzeba jednak sortować, ale wykorzystać algorytm \proc{Select} do podzielenia jej w~taki sposób, aby fragment $A[l\twodots l+k-1]$ zawierał elementy od $l$\nbhyphen tego najmniejszego do $(l+k-1)$\nbhyphen tego najmniejszego.
Wpierw wyszukany zostanie $l$\nbhyphen ty najmniejszy element tablicy $A[1\twodots n]$, dzięki czemu po tym wywołaniu podtablica $A[l\twodots n]$ będzie zawierać $n-l+1$ największych elementów zbioru wejściowego.
Teraz wystarczy wywołać algorytm \proc{Select} ponownie na ostatniej podtablicy w~celu wyszukania jej $k$\nbhyphen tego najmniejszego elementu i~jednocześnie wykonania jej docelowego podziału.
\begin{codebox}
\Procname{$\proc{Median-Neighbors}(A,k)$}
\li	$n\gets\attrib{A}{length}$
\li $l\gets\lfloor(n+1)/2\rfloor-\lfloor(k-1)/2\rfloor$
\li	$\proc{Select}(A,1,n,l)$
\li	$\proc{Select}(A,l,n,k)$
\li	\Return $A[l\twodots l+k-1]$
\end{codebox}

Zajmijmy się teraz opracowaniem algorytmu rozwiązującego problem po przyjęciu drugiej definicji liczb ,,najbliższych'' medianie.
Algorytm ten wyznaczy medianę $x$ zbioru $S$, a~następnie obliczy odległość każdego elementu z~$S$ od $x$, czyli wartość bezwzględną ich różnicy.
Po znalezieniu $k$\nbhyphen tej najmniejszej odległości $d_k$ zwrócony zostanie zbiór $C$ elementów odległych od $x$ o~nie więcej niż $d_k$.

O~ile elementy wejściowe z~założenia są parami różne, to odległości od mediany mogą się powtarzać -- dokładniej, każda odległość może występować w~co najwyżej dwóch egzemplarzach.
Jednak powtarzające się elementy w~zbiorze wejściowym nie wpływają na poprawność działania algorytmu \proc{Select}, jak również w~tym przypadku na jego efektywność.
Może się też zdarzyć, że skonstruowany zbiór $C$ będzie rozmiaru $k+1$.
Jest tak wówczas, gdy wśród wyznaczonych odległości są dwie kopie $d_k$, a~więc do $C$ trafił zarówno element $x-d_k$, jak i~$x+d_k$ -- wtedy z~$C$ wystarczy usunąć jeden z~nich.

Zbiór dynamiczny $C$ można zaimplementować, opierając się o~listę dwukierunkową (patrz podrozdział 10.2).
Wówczas operacje inicjalizacji zbioru oraz dodawania nowego elementu zajmują czas stały (nie trzeba sprawdzać unikalności wstawianego elementu, bo działamy na elementach parami różnych), natomiast zarówno odczytywanie rozmiaru zbioru, jak i~usuwanie z~niego elementu, odbywa się w~czasie co najwyżej proporcjonalnym do rozmiaru zbioru.
Dzięki temu algorytm działa w czasie $O(n)$.
\begin{codebox}
\Procname{$\proc{Closest-To-Median}(A,k)$}
\li	$n\gets\attrib{A}{length}$
\li	$x\gets\proc{Select}(A,1,n,\lfloor(n+1)/2\rfloor)$
\li utwórz tablicę $D[1\twodots n]$
\li	\For $i\gets1$ \To $n$
\li		\Do $D[i]\gets|A[i]-x|$
		\End
\li	$d_k\gets\proc{Select}(D,1,n,k)$
\li	$C\gets\emptyset$
\li	\For $i\gets1$ \To $n$
\li		\Do \If $|A[i]-x|\le d_k$
\li				\Then $C\gets C\cup\{A[i]\}$
				\End
		\End
\li	\If $|C|=k+1$
\li		\Then $C\gets C\setminus\{x+d_k\}$
		\End
\li	\Return $C$
\end{codebox}

\exercise %9.3-8
Niech $m_X$ będzie medianą (dolną) elementów z~tablicy $X$ i~niech $m_Y$ będzie medianą (dolną) elementów z~tablicy $Y$.
Jeśli $m_X=m_Y$, to wśród wszystkich $2n$ elementów obu tablic co najmniej $n$ jest większych (lub równych) od obu median i~co najmniej $n$ jest od nich mniejszych (lub równych).
Wartość $m_X$ jest więc szukaną medianą wszystkich $2n$ liczb.
Załóżmy teraz, że $n>1$ i, bez straty ogólności, że $m_X<m_Y$.
Odrzucając teraz około $n/2$ elementów $X$ na lewo od $m_X$ i~około $n/2$ elementów $Y$ na prawo od $m_Y$, sprowadzamy problem do podproblemu mniejszego o~około połowę, ponieważ wiadomo, że poszukiwana mediana znajduje się wśród pozostawionych elementów.
Ponadto, jeśli $n$ jest parzyste, to element $m_X$ także może zostać odrzucony, bo na pewno nie jest szukaną medianą.
Do podproblemu trafią zatem pozostawione fragmenty tablic $X$ i~$Y$ o~rozmiarach $\lceil n/2\rceil$ każdy.

Podamy rekurencyjny algorytm rozwiązujący zadany problem.
Gdy $n=1$, to algorytm zwróci mniejszy z~dwóch wejściowych elementów (wynikiem jest mediana dolna).
W~przeciwnym przypadku wyznaczane są indeksy median dolnych i~górnych obu tablic, które dzięki uporządkowaniu obu tablic, jesteśmy w~stanie wyznaczyć natychmiast, po czym algorytm wywołuje się rekurencyjnie zgodnie z~powyższym opisem.
Aby odnaleźć wspólną medianę $2n$ elementów z~tablic $X[1\twodots n]$ i~$Y[1\twodots n]$, należy wywołać $\proc{Joint-Median}(X,1,n,Y,1,n)$.
\begin{codebox}
\Procname{$\proc{Joint-Median}(X,p_X,r_X,Y,p_Y,r_Y)$}
\li	\If $p_X=r_X$
\li		\Then \Return $\min(X[p_X],Y[p_Y])$
		\End
\li	$q_X\gets\lfloor(p_X+r_X)/2\rfloor$ \>\>\>\>\>\>\Comment indeks mediany dolnej podtablicy $X[p_X\twodots r_X]$
\li	$q_X'\gets\lceil(p_X+r_X)/2\rceil$ \>\>\>\>\>\>\Comment indeks mediany górnej podtablicy $X[p_X\twodots r_X]$
\li	$q_Y\gets\lfloor(p_Y+r_Y)/2\rfloor$ \>\>\>\>\>\>\Comment indeks mediany dolnej podtablicy $Y[p_Y\twodots r_Y]$
\li	$q_Y'\gets\lceil(p_Y+r_Y)/2\rceil$ \>\>\>\>\>\>\Comment indeks mediany górnej podtablicy $Y[p_Y\twodots r_Y]$
\li	\If $X[q_X]=Y[q_Y]$
\li		\Then \Return $X[q_X]$
		\End
\li	\If $X[q_X]<Y[q_Y]$
\li		\Then \Return $\proc{Joint-Median}(X,q_X',r_X,Y,p_Y,q_Y)$
\li		\Else \Return $\proc{Joint-Median}(X,p_X,q_X,Y,q_Y',r_Y)$
		\End
\end{codebox}

Pesymistyczny czas działania algorytmu opisuje rekurencja $T(n)=T(\lceil n/2\rceil)+\Theta(1)$, gdzie $n=r_X-p_X+1$.
Jej rozwiązaniem jest oczywiście $T(n)=O(\lg n)$.

\exercise %9.3-9
Oznaczmy przez $y_1$ i~$y_2$, odpowiednio, medianę dolną i~górną współrzędnych $y$ wszystkich wież.
Umieścimy główny rurociąg na współrzędnej $y$ równej $y_r$ takiej, że $y_1\le y_r\le y_2$ i~zbadamy, jak będzie zmieniać się suma długości odnóg północ\nbhyphen południe $s$, gdy będziemy zmieniać jego położenie.

Wybierzmy $y_r'$, spełniające nierówności $y_1\le y_r'\le y_2$, na współrzędną $y$ nowego położenia głównego rurociągu.
Niech $d=|y_r-y_r'|$.
Jeśli $y_1=y_2$, to $y_r'=y_r$, rozważmy więc przypadek przeciwny.
Wtedy zarówno na północ, jak i~na południe od głównego rurociągu znajduje się $n/2$ wież wiertniczych.
Do jednych zbliżyliśmy się o~odległość $d$, ale od pozostałych o~tyle samo się oddaliliśmy.
Widać stąd, że wybór dowolnego $y_r'$ znajdującego się pomiędzy medianami nie zmienia sumy długości odnóg północ\nbhyphen południe.

Niech teraz $y_r'$ będzie współrzędną $y$ głównego rurociągu taką, że $y_r'<y_1$ lub $y_r'>y_2$.
Podobnie jak poprzednio, niech $d=|y_r-y_r'|$.
Gdy $n$ jest parzyste, to po jednej stronie rurociągu mamy co najmniej $n/2+1$ wież, a~po drugiej -- co najwyżej $n/2-1$.
Stąd otrzymujemy oszacowanie na nową sumę długości odnóg:
\[
    s' \ge s+d(n/2+1)-d(n/2-1) = s+2d > s.
\]
W~przypadku, gdy $n$ jest liczbą nieparzystą, rurociąg w~nowym położeniu po jednej stronie ma co najmniej $(n+1)/2$ wież wiertniczych, a~po drugiej stronie co najwyżej $(n-1)/2$.
Nową sumę długości odnóg można więc ograniczyć od dołu:
\[
    s' \ge s+d(n+1)/2-d(n-1)/2 = s+d > s.
\]

A~zatem, niezależnie od parzystości $n$, przesunięcie rurociągu poza obszar wyznaczony przez mediany $y_1$ i~$y_2$, wiąże się z~powiększeniem sumy długości odnóg północ\nbhyphen południe -- dowolna wartość $y_r$ pomiędzy tymi medianami, łącznie z~nimi, jest optymalna.
Problem sprowadza się zatem do wyznaczenia mediany współrzędnych $y$ wież wiertniczych i~można go rozwiązać w~czasie liniowym względem $n$ mimo braku założenia o~tym, że elementy, spośród których wyznaczamy medianę, są parami różne (co zauważyliśmy w~\refExercise{9.3-7}).
