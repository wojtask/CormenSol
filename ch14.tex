\chapter{Wzbogacanie struktur danych}

\subchapter{Dynamiczne statystyki pozycyjne}

\exercise %14.1-1
\exercise %14.1-2
\exercise %14.1-3
\exercise %14.1-4
\exercise %14.1-5
\exercise %14.1-6
\exercise %14.1-7
\exercise %14.1-8

\subchapter{Jak wzbogacać strukturę danych}

\exercise %14.2-1
\exercise %14.2-2
\exercise %14.2-3
\exercise %14.2-4
\exercise %14.2-5

\subchapter{Drzewa przedziałowe}

\exercise %14.3-1
\exercise %14.3-2
\exercise %14.3-3
\exercise %14.3-4
\exercise %14.3-5
\exercise %14.3-6
\exercise %14.3-7

\problems

\problem{Punkt o~największej liczbie przecięć} %14-1

\subproblem %14-1(a)
\subproblem %14-1(b)

\problem{Permutacja Józefa} %14-2

\subproblem %14-2(a)
\subproblem %14-2(b)

\endinput
