\chapter{Mediany i~statystyki pozycyjne}

\subchapter{Minimum i~maksimum}

\exercise %9.1-1
Wyznaczmy najpierw najmniejszą spośród $n$ liczb w~następujący sposób. W~każdym etapie testujemy liczby parami, odrzucając te, które były większe w~swoich parach. Procedurę wykonujemy na pozostawionych liczbach, aż do uzyskania jednej liczby -- będącej oczywiście minimum początkowego zbioru. Przyjmujemy, że w~razie nieparzystej liczby elementów na danym etapie, element bez pary nie jest testowany z~żadnym i~przechodzi do kolejnego etapu. Ponieważ po każdym etapie z~$k$ liczb zostaje $\lceil k/2\rceil$, to będzie $\lceil\lg n\rceil$ etapów i~tyle razy testowi będzie poddawany element najmniejszy. Zauważmy, że każdy test odrzuca jedną liczbę, wykonamy zatem dokładnie $n-1$ testów.

Zastanówmy się teraz która z~odrzuconych liczb może być drugą najmniejszą w~zbiorze. Ponieważ zostanie ona odrzucona tylko w~teście z~elementem najmniejszym, to problem sprowadza się do wyznaczenia minimum zbioru tych liczb, które były testowane z~elementem najmniejszym. Na mocy wcześniejszej obserwacji mamy, że zbiór ten składa się z~$\lceil\lg n\rceil$ elementów, więc wystarczy $\lceil\lg n\rceil-1$ porównań do wyznaczenia jego minimum.

Ostatecznie dostajemy, że drugą najmniejszą spośród $n$ liczb można wyznaczyć, wykonując $n+\lceil\lg n\rceil-2$ porównań.

\exercise %9.1-2
Zadanie rozwiążemy prostszą metodą niż sugeruje nam to wskazówka.

Jeśli $n$ jest parzyste, to zgodnie z~podaną w~podręczniku informacją, wykonywanych jest $3n/2-2$ porównań. Ale dla parzystego $n$ prawdą jest, że $3n/2=\lceil3n/2\rceil$, więc wzór na liczbę koniecznych porównań przyjmuje postać $\lceil3n/2\rceil-2$. Niech teraz $n$ będzie liczbą nieparzystą, czyli $n=2k+1$ dla pewnego całkowitego $k$. Chcemy wykazać, że potrzebnych jest $\lceil3n/2\rceil-2$ porównań, czyli $\lceil3k+3/2\rceil-2=3k+2-2=3k$. Ale wynik ten zgadza się z~opisanym w~podręczniku dolnym oszacowaniem na liczbę porównań dla nieparzystego $n$, bo $3\lfloor n/2\rfloor=3\lfloor k+1/2\rfloor=3k$.

\subchapter{Wybór w~oczekiwanym czasie liniowym}

\exercise %9.2-1
Zakładamy, że parametr $i$ jest liczbą całkowitą i~spełnia warunek $1\le i\le r-p+1$. Wywołanie z~wiersza~3 procedury \proc{Randomized-Partition} zwraca liczbę całkowitą $q$ taką, że $p\le q\le r$. Dla liczby całkowitej $k$ wyznaczonej w~kolejnym wierszu zachodzi więc $1\le k\le r-p+1$. Wywołanie rekurencyjne w~wierszu~8 nastąpi dla tablicy długości~0, jeśli $i<k$ i~$q=p$, ale wtedy $k=1$ i~taki przypadek nie ma prawa zajść. Podobnie w~wierszu~9 funkcja zostanie wywołana rekurencynie dla pustej tablicy, o~ile $i>k$ i~$q=r$, lecz wtedy $k=r-p+1$.

\exercise %9.2-2

\exercise %9.2-3
\begin{codebox}
\Procname{$\proc{Iterative-Randomized-Select}(A,p,r,i)$}
\li	\While $p<r$
\li		\Do
			$q\gets\proc{Randomized-Partition}(A,p,r)$
\li			$k\gets q-p+1$
\li			\If $i=k$
\li				\Then \Return $A[q]$
				\End
\li			\If $i<k$
\li				\Then $r\gets q-1$
\li				\Else
					$p\gets q+1$
\li					$i\gets i-k$
				\End
		\End
\li	\Return $A[p]$
\end{codebox}

\exercise %9.2-4
W~przypadku szukania elementu najmniejszego pesymistyczny przypadek dzielenia podtablicy występuje, gdy na element rozdzielający wybierany jest za każdym razem jej największy element. Kolejne wywołania rekurencyjne zmniejszają obszar poszukiwań o~1, w~wyniku czego efektem ubocznym jest posortowanie tablicy w~czasie kwadratowym. Ciąg tak wyznaczonych podziałów dla tablicy $A$ przedstawia rys.~\ref{fig:9.2-4}.

\begin{figure}[ht]
	\begin{center}
		\includegraphics{fig09.1}
	\end{center}
	\caption{Ciąg pesymistycznych podziałów tablicy $A$ podczas szukania elementu najmniejszego.} \label{fig:9.2-4}
\end{figure}

\subchapter{Wybór w~pesymistycznym czasie liniowym}

\exercise %9.3-1
Dokonajmy analogicznej analizy algorytmu \proc{Select} w~przypadku, gdy podział następuje na grupy \compound{7}{elementowe}. W~co najmniej $\lceil n/7\rceil$ grupach są po 4 elementy większe od $x$, oprócz jednej grupy o~mniej niż 7 elementach, jeśli $n$ nie jest podzielne przez~7, i~jednej grupy zawierającej sam element $x$. Odliczając te dwie grupy, wnioskujemy, że liczba elementów większych od $x$ wynosi co najmniej
\[
	4\left(\left\lceil\frac{1}{2}\left\lceil\frac{n}{7}\right\rceil\right\rceil-2\right) \ge \frac{2n}{7}-8.
\]
Podobnie wykazuje się, że liczba elementów mniejszych od $x$ wynosi co najmniej $2n/7-8$. Stąd wywołanie rekurencyjne nastąpi dla zbioru \compound{$(5n/7+8)$}{elementowego}. Rekurencja przyjmuje więc postać
\[
	T(n) \le \begin{cases}
		\Theta(1), & \text{jeśli $n\le d$}, \\
		T(\lceil n/7\rceil)+T(5n/7+8)+O(n), & \text{jeśli $n>d$},
	\end{cases}
\]
przy czym $d$ jest pewną stałą, którą wyznaczymy później.

Wykażemy metodą przez podstawienie, że $T(n)=O(n)$. Zachodzi oczywiście $T(n)\le cn$ dla pewnej stałej $c>0$ oraz wszystkich $n\le d$. Niech teraz $n>d$ i~załóżmy, że $T(k)\le ck$ dla pewnej stałej $c>0$ i~wszystkich $k<n$. Dla pewnej stałej $a>0$ zachodzi
\begin{align*}
	T(n) &\le c\lceil n/7\rceil+c(5n/7+8)+an \\
	&\le cn/7+c+5cn/7+8c+an \\
	&= 6cn/7+9c+an \\
	&= cn+(-cn/7+9c+an) \\
	&\le cn,
\end{align*}
o~ile składnik $-cn/7+9c+an$ jest niedodatni. Dla $n>63$ warunek ten zachodzi, gdy $c\ge7a(n/(n-63))$. Jeśli z~kolei $n\ge126$, to $n/(n-63)\le2$ i~dostajemy, że $c\ge14a$. Można zatem przyjąć za $d$ wartość 126.

Pokażemy teraz, że jeśli podział będzie dokonywany na grupy \compound{3}{elementowe}, to czas działania tak zmodyfikowanego algorytmu \proc{Select} jest wyższy od liniowego. Rozważmy przypadek, w~którym jest dokładnie $\left\lceil\frac{1}{2}\left\lceil\frac{n}{3}\right\rceil\right\rceil$ grup o~medianach większych lub równych $x$, a~ostatnia, niepełna grupa, zawiera dokładnie 2 elementy większe niż $x$. Stąd liczba elementów większych od $x$ wynosi
\[
	2\left(\left\lceil\frac{1}{2}\left\lceil\frac{n}{3}\right\rceil\right\rceil-1\right)+1 = 2\left\lceil\frac{n}{6}\right\rceil-1.
\]
Procedura może zostać zatem wywołana rekurencyjne dla co najmniej $n-(2\left\lceil n/6\right\rceil-1)\ge n-(2(n/6+1)-1)=2n/3-1$ elementów nieprzekraczających $x$. Zauważmy też, że sortowanie elementów w~kroku 2 algorytmu \proc{Select} zabiera czas nie tylko $O(n)$, ale dokładnie $\Theta(n)$. Rekurencja opisująca czas działania algorytmu w~tym przypadku ma postać
\[
	T(n) \ge \begin{cases}
		\Theta(1), & \text{jeśli $n\le d$}, \\
		T(\lceil n/3\rceil)+T(2n/3-1)+\Theta(n), & \text{jeśli $n>d$},
	\end{cases}
\]
dla pewnej stałej $d>0$. Można wykazać, stosując metodę przez podstawienie, że rozwiązaniem powyższej rekurencji jest $T(n)=\Omega(n\lg n)$, w~związku z~czym algorytm po takiej modyfikacji nie działa w~czasie liniowym.

\exercise %9.3-2
Z~analizy algorytmu \proc{Select} wynika, że liczba elementów większych od $x$ wynosi co najmniej $3n/10-6$ i~podobnie dla liczby elementów mniejszych od $x$. Mamy teraz
\[
	\frac{3n}{10}-6-\left\lceil\frac{n}{4}\right\rceil \ge \frac{3n}{10}-6-\left(\frac{n}{4}+1\right) = \frac{n-140}{20} \ge 0,
\]
ponieważ mieliśmy założenie, że $n\ge140$. A~zatem obie te wartości są nie mniejsze niż $\lceil n/4\rceil$.

\exercise %9.3-3
Można wymusić dokonywanie najbardziej zrównoważonych podziałów przez procedurę \proc{Partition} dzięki zastosowaniu algorytmu \proc{Select} do wyboru mediany sortowanej podtablicy na element rozdzielający. Ponieważ pesymistyczny czas algorytmu \proc{Select} wynosi $O(n)$, to pesymistyczny czas algorytmu quicksort sprowadza się po takiej modyfikacji do rekurencji $T(n)\le2T(n/2)+\Theta(n)$, której rozwiązaniem jest $O(n\lg n)$.

\exercise %9.3-4

\exercise %9.3-5
Problem rozwiązujemy analogicznie jak w~\refExercise{9.3-3} ale dla algorytmu wyboru. Algorytm jest podobny do \proc{Randomized-Select}, w~wierszu~3 wywołujemy jednak procedurę \proc{Partition}, wewnątrz której elementem rozdzielającym czynimy medianę przetwarzanej podtablicy wyznaczoną za pomocą ``czarnej skrzynki''. W~każdym wywołaniu rekurencyjnym będzie dokonywany najbardziej zrównoważony podział podtablicy, więc czas działania algorytmu w~przypadku pesymistycznym jest postaci
\[
	T(n) \le
	\begin{cases}
		O(1), & \text{jeśli $n=1$}, \\
		T(\lfloor n/2\rfloor)+O(n), & \text{jeśli $n>1$}.
	\end{cases}
\]
Rozwiązaniem powyższej rekurencji jest $T(n)=O(n)$.

\exercise %9.3-6
Załóżmy, że badany zbiór przechowujemy w~tablicy \compound{$n$}{elementowej}. Podzielmy ją na $\lfloor n/k\rfloor$ podtablic i~posortujmy każdą z~nich. Wszystkie te podtablice składają się z~$k$ elementów (z~dokładnością do~1), więc ich sortowanie algorytmem quicksort zajmuje łącznie czas $\lfloor n/k\rfloor\cdot O(k\lg k)=O(n\lg k)$. %Zauważmy, że najmniejszy kwantyl rzędu $k$ początkowej tablicy jest największym elemente

\exercise %9.3-7
Znajdujemy elementy $i_1=\lfloor n/2\rfloor-\lfloor k/2\rfloor$ oraz $i_2=\lfloor n/2\rfloor+\lfloor k/2\rfloor$ wejściowej tablicy w~porządku rosnącym za pomocą dwóch wywołań algorytmu \proc{Select} w~czasie $O(n)$. Szukane elementy są większe od \compound{$i_1$}{tego} i~mniejsze od \compound{$i_2$}{tego} -- wyznaczamy je przeglądając liniowo tablicę i~testując czy znajdują się w~wyznaczonym przedziale.

\exercise %9.3-8
Porównajmy mediany każdej z~tablic. Jeśli są one równe, to wśród wszystkich $2n$ elementów $n$ jest większych od tych median -- $n/2$ z~tablicy $X$ i~$n/2$ z~tablicy $Y$. Załóżmy teraz, że mediany są różne i~bez utraty ogólności, niech mniejsza będzie mediana tablicy $X$. Pomijając teraz elementy $X$ mniejsze od jej mediany i~elementy $Y$ większe od jej mediany, sprowadzamy problem do identycznego ale o~połowę mniejszego, ponieważ wiadomo, że poszukiwana mediana znajduje się wśród $n$ pozostawionych elementów.
\begin{codebox}
\Procname{$\proc{Two-Array-Median}(X,p_X,r_X,Y,p_Y,r_Y)$}
\li	\If $p_X=r_X$
\li		\Then \Return $\min(X[p_X],Y[p_Y])$
		\End
\li	$q_X\gets\lfloor(p_X+r_X)/2\rfloor$
\li	$q_Y\gets\lfloor(p_Y+r_Y)/2\rfloor$
\li	\If $X[q_X]=Y[q_Y]$
\li		\Then \Return $X[q_X]$
		\End
\li	\If $X[q_X]>Y[q_Y]$
\li		\Then \Return $\proc{Two-Array-Median}(X,p_X,q_X-1,Y,q_Y+1,r_Y)$
\li		\Else \Return $\proc{Two-Array-Median}(X,q_X+1,r_X,Y,p_Y,q_Y-1)$
		\End
\end{codebox}

Czas działania algorytmu jest opisany przez rekurencję z~\refExercise{2.3-5}, ponieważ jego działanie jest analogiczne do wyszukiwania binarnego -- na każdym poziomie rekursji po wykonaniu jednego porównania odrzucana jest połowa problemu.

\exercise %9.3-9

\problems

\problem{Sortowanie $i$ największych elementów} %9-1

\subproblem %9-1(a)
Liczby można posortować algorytmem sortowania przez scalanie, który w~najgorszym przypadku potrzebuje czasu $\Theta(n\lg n)$. Następnie zwracamy $i$ największych liczb poprzez proste przejście po prawym końcu tablicy, co zajmuje czas $\Theta(i)$, więc całkowity czas algorytmu w~najgorszym przypadku wynosi $\Theta(n\lg n+i)$.

\subproblem %9-1(b)
Zbudowanie kopca typu max dla kolejki priorytetowej wymaga czasu $\Theta(n)$. Wykonanie $i$ operacji \proc{Extract-Max} zajmuje czas
\[
	\sum_{k=1}^i\Theta(\lg(n-k)) = \sum_{k=n-i}^n\Theta(\lg k) = \Theta(n\lg n-(n-i-1)\lg n) = \Theta(i\lg n)
\]
na podstawie punktu~(b) problemu~\refProblem{A-1}, a~zatem całkowitym czasem algorytmu w~przypadku pesymistycznym jest $\Theta(n+i\lg n)$.

\subproblem %9-1(c)
Aby osiągnąć najlepszy czas w~przypadku pesymistycznym, użyjemy procedury \proc{Select} do znalezienia \compound{$i$}{tej} statystyki pozycyjnej. Sortowanie $i$ największych liczb można wykonać algorytmem sortowania przez scalanie, które w~najgorszym przypadku zajmuje czas $\Theta(i\lg i)$, skąd całkowity czas działania algorytmu wynosi $\Theta(n+i\lg i)$, co czyni go najbardziej efektywnym spośród wszystkich rozważanych algorytmów w~obecnym problemie.

\problem{Mediana ważona} %9-2

\subproblem %9-2(a)
Dla tak przyjętych wag elementów mamy
\[
	\sum_{x_i<x_k}w_i = \frac{k-1}{n} \quad\text{oraz}\quad \sum_{x_i>x_k}w_i = \frac{n-k}{n}.
\]
Jeśli ograniczymy obie sumy od góry przez $1/2$, to dostaniemy, że medianą ważoną jest element $x_k$, gdzie $n/2\le k\le n/2+1$, czyli $k=\lfloor(n+1)/2\rfloor$ lub $k=\lceil(n+1)/2\rceil$, a~to jest warunek na to, aby $x_k$ było zwykłą medianą.

\subproblem %9-2(b)
Po posortowaniu elementów wyznaczenie mediany ważonej odbywa się w~prosty sposób -- wystarczy sumować wagi elementów, poruszając się w~ich porządku rosnącym i~zwrócić pierwszy element, dla którego suma wag osiągnie lub przekroczy $1/2$. Złożoność tej procedury zależy od efektywności sortowania, które w~pesymistycznym czasie zajmuje $O(n\lg n)$ dla $n$ elementów.

\subproblem %9-2(c)
\subproblem %9-2(d)
\subproblem %9-2(e)

\problem{Małe statystyki pozycyjne} %9-3

\subproblem %9-3(a)
\note{W~tłumaczeniu występuje błąd w~sformułowaniu rekurencji\/ $U_i(n)$. Wartość\/ $T(n)$ powinna być zwracana nie dla\/ $i\le n/2$, lecz dla\/ $i\ge n/2$.}

\noindent Postępując za podaną wskazówką, grupujemy najpierw elementy w~pary i~wyznaczamy mniejsze w~każdej parze. Zauważmy, że \compound{$i$}{ty} w~kolejności element całego zbioru, gdzie $i<n/2$, znajduje się wśród elementów mniejszych w~parach lub wśród 

\subproblem %9-3(b)
Załóżmy, że $n/2^k\le i<n/2^{k-1}$ i~rozwińmy rekurencję $U_i(n)$ w~przypadku, gdy $i<n/2$:
\[
	U_i(n) = \left\lfloor\frac{n}{2}\right\rfloor+T(2i)+\left\lfloor\frac{1}{2}\left\lceil\frac{n}{2}\right\rceil\right\rfloor+T(2i)+\dots+\left\lfloor\frac{1}{2}\left\lceil\frac{n}{2^{k-1}}\right\rceil\right\rfloor+T(2i)+T\Bigl(\left\lceil\frac{n}{2^k}\right\rceil\Bigr).
\]
Na mocy tego, że $k\ge\lg(n/i)$ oraz $T(\lceil n/2^k\rceil)\le T(n/2^k+1)\le T(i+1)\le T(2i)$, mamy
\begin{align*}
	U_i(n) &\le \frac{n}{2}+\frac{n}{4}+\dots+\frac{n}{2^k}+(k-1)T(2i)+T\Bigl(\left\lceil\frac{n}{2^k}\right\rceil\Bigr) \\
	&< n\sum_{j=1}^\infty\frac{1}{2^j}+kT(2i) \\
	&= n+O(T(2i)\lg(n/i)).
\end{align*}

\subproblem %9-3(c)
Ponieważ $i$ jest stałą, to $T(2i)$ również można potraktować jako wartość stałą. Wykorzystując poprzedni punkt, mamy
\[
	U_i(n) = n+O(T(2i)\lg(n/i)) = n+O(\lg n-\lg i) = n+O(\lg n),
\]
co należało dowieść.

\subproblem %9-3(d)
Dla $k>2$ oszacowanie wynika natychmiast z~części~(b) po podstawieniu $i=n/k$, bo wtedy oczywiście $i<n/2$.

Jeśli $k=2$, to rozwiązaniem rekurencji $U_i(n)$ jest $T(n)$, czyli $O(n)$. Zauważmy, że teza w~tym przypadku sprowadza się do $U_i(n)=n+O(T(n))$ i~rzeczywiście zachodzi, ponieważ $n+O(T(n))=O(n)$.

\endinput
