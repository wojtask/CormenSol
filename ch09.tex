\chapter{Mediany i~statystyki pozycyjne}

\subchapter{Minimum i~maksimum}

\exercise %9.1-1
Wyznaczmy najpierw $\mu$ -- najmniejszą spośród $n$ liczb -- w~następujący sposób. Łączymy liczby w~pary i~odrzucamy te, które są większe w~swoich parach, po czym wykonujemy te operacje rekurencyjnie dla zbioru pozostawionych liczb, aż do uzyskania jednej liczby, którą oczywiście będzie~$\mu$. Przyjmujemy, że w~razie nieparzystej liczby elementów na danym poziomie rekurencji, element bez pary nie jest porównywany z~żadnym innym i~zwyczajnie przechodzi do kolejnego etapu. Ponieważ na każdym poziomie z~$k$ liczb zostaje $\lceil k/2\rceil$, to będzie $\lceil\lg n\rceil$ wywołań rekurencyjnych tej procedury i~co najwyżej tylu testom będzie poddawany element $\mu$. Zauważmy, że każdy test odrzuca jedną liczbę, wykonamy zatem dokładnie $n-1$ porównań.

Zastanówmy się teraz, która z~pozostałych liczb może być drugą najmniejszą w~zbiorze. Liczba ta została odrzucona po porównaniu jej z~elementem $\mu$, więc problem sprowadza się do wyznaczenia minimum zbioru tych liczb, które były testowane z~$\mu$. Na mocy wcześniejszej obserwacji mamy, że zbiór ten składa się z~$\lceil\lg n\rceil$ elementów, więc wystarczy $\lceil\lg n\rceil-1$ porównań do wyznaczenia jego minimum.

Ostatecznie dostajemy, że drugą najmniejszą spośród $n$ liczb można wyznaczyć, wykonując $n+\lceil\lg n\rceil-2$ porównań.

\exercise %9.1-2
Zadanie rozwiążemy prostszą metodą, niż sugeruje nam to wskazówka.

Jeśli $n$ jest parzyste, to zgodnie z~podaną w~Podręczniku informacją, wykonywanych jest $3n/2-2$ porównań. Ale dla parzystego $n$ zachodzi $3n/2=\lceil3n/2\rceil$, więc wzór na liczbę potrzebnych porównań przyjmuje postać $\lceil3n/2\rceil-2$. Niech teraz $n$ będzie liczbą nieparzystą, czyli $n=2k+1$ dla pewnego całkowitego $k$. Chcemy wykazać, że koniecznych jest $\lceil3n/2\rceil-2$ porównań, czyli $\lceil3k+3/2\rceil-2=3k+2-2=3k$. Ale wynik ten zgadza się z~opisanym w~Podręczniku dolnym oszacowaniem na liczbę porównań dla nieparzystego $n$, bo $3\lfloor n/2\rfloor=3\lfloor k+1/2\rfloor=3k$.

\subchapter{Wybór w~oczekiwanym czasie liniowym}

\exercise %9.2-1
Zakładamy, że parametr $i$ jest liczbą całkowitą spełniającą nierówności $1\le i\le r-p+1$. Wywołanie procedury \proc{Randomized-Partition} w~wierszu~3 zwraca liczbę całkowitą $q$ taką, że $p\le q\le r$. Dla liczby $k$ wyznaczonej w~kolejnym wierszu zachodzi więc $1\le k\le r-p+1$. Wywołanie rekurencyjne w~wierszu~8 nastąpi dla tablicy długości~0, jeśli $i<k$ i~$q=p$, ale wówczas $k=1$, co jest sprzeczne z~założeniem o~parametrze $i$. Podobnie w~wierszu~9 funkcja zostanie wywołana rekurencyjnie dla pustej tablicy, o~ile $i>k$ i~$q=r$, lecz wtedy $k=r-p+1$ i~również w~tym przypadku dochodzimy do sprzeczności.

\exercise %9.2-2
Czas działania procedury \proc{Randomized-Partition} dla tablicy o~mniej niż $n$ elementach jest niezależny od tego, jak została podzielona tablica o~$n$ elementach w~poprzednim wywołaniu rekurencyjnym. Jest tak między innymi dlatego, że żaden poziom rekursji nie przekazuje do następnego poziomu informacji o~tym, na jakim fragmencie tablicy działa.

\exercise %9.2-3
Po dokonaniu oczywistych zmian w~oryginalnej procedurze, otrzymujemy następujący pseudokod:
\begin{codebox}
\Procname{$\proc{Iterative-Randomized-Select}(A,p,r,i)$}
\li	\While $p<r$
\li		\Do
			$q\gets\proc{Randomized-Partition}(A,p,r)$
\li			$k\gets q-p+1$
\li			\If $i=k$
\li				\Then \Return $A[q]$
				\End
\li			\If $i<k$
\li				\Then $r\gets q-1$
\li				\Else
					$p\gets q+1$
\li					$i\gets i-k$
				\End
		\End
\li	\Return $A[p]$
\end{codebox}

\exercise %9.2-4
W~przypadku szukania elementu najmniejszego pesymistyczny przypadek dzielenia podtablicy występuje, gdy na element rozdzielający wybierany jest za każdym razem jej największy element. Kolejne wywołania rekurencyjne zmniejszają wówczas obszar poszukiwań o~1, jednocześnie umieszczając na końcu tablicy elementy w~kolejności rosnącej, w~wyniku czego, jako efekt uboczny, tablica zostaje posortowana.

\subchapter{Wybór w~pesymistycznym czasie liniowym}

\exercise %9.3-1
Dokonajmy analizy algorytmu \proc{Select} w~przypadku, gdy elementy będą dzielone na grupy \onedash{7}{elementowe}. Wówczas w~co najmniej połowie spośród $\lceil n/7\rceil$ grup są po 4 elementy większe od $x$, oprócz jednej grupy o~mniej niż 7 elementach, jeśli $n$ nie jest podzielne przez~7, i~jednej grupy zawierającej sam element $x$. Odliczając te dwie grupy, wnioskujemy, że liczba elementów większych od $x$ wynosi co najmniej
\[
	4\biggl(\biggl\lceil\frac{1}{2}\Bigl\lceil\frac{n}{7}\Bigr\rceil\biggr\rceil-2\biggr) \ge \frac{2n}{7}-8.
\]
Podobnie wykazuje się, że liczba elementów mniejszych od $x$ wynosi co najmniej $2n/7-8$. Stąd procedura wywoła się rekurencyjnie dla zbioru co najwyżej \onedash{$(5n/7+8)$}{elementowego}. Rekurencja przyjmuje więc postać
\[
	T(n) \le \begin{cases}
		\Theta(1), & \text{jeśli $n<d$}, \\
		T(\lceil n/7\rceil)+T(5n/7+8)+O(n), & \text{jeśli $n\ge d$},
	\end{cases}
\]
przy czym $d>0$ jest pewną stałą, którą wyznaczymy później.

Wykażemy metodą przez podstawianie, że $T(n)=O(n)$. Zachodzi oczywiście $T(n)\le cn$ dla pewnej stałej $c>0$ oraz wszystkich $n<d$. Załóżmy teraz, że $n\ge d$ i~że $T(k)\le ck$ dla pewnej stałej $c>0$ i~wszystkich $k<n$. Dla pewnej stałej $a>0$ zachodzi wówczas
\begin{align*}
	T(n) &\le c\lceil n/7\rceil+c(5n/7+8)+an \\
	&\le cn/7+c+5cn/7+8c+an \\
	&= 6cn/7+9c+an \\
	&= cn+(-cn/7+9c+an) \\
	&\le cn,
\end{align*}
o~ile składnik $-cn/7+9c+an$ jest niedodatni. Dla $n>63$ warunek ten jest spełniony, kiedy $c\ge7a(n/(n-63))$. Jeśli z~kolei $n\ge126$, to $n/(n-63)\le2$ i~wtedy musi być $c\ge14a$. Można zatem przyjąć za $d$ wartość 126, co kończy dowód.

Pokażemy teraz, że jeśli podział będzie dokonywany na grupy \onedash{3}{elementowe}, to czas działania tak zmodyfikowanego algorytmu \proc{Select} będzie wyższy od liniowego. Rozważmy w~szczególności przypadek, kiedy jest dokładnie $\bigl\lceil\frac{1}{2}\bigl\lceil\frac{n}{3}\bigr\rceil\bigr\rceil$ grup o~medianach większych lub równych $x$, w~tym ostatnia, niepełna grupa, która zawiera 2 elementy większe niż $x$. Stąd całkowita liczba elementów większych od $x$ wynosi
\[
	2\biggl(\biggl\lceil\frac{1}{2}\Bigl\lceil\frac{n}{3}\Bigr\rceil\biggr\rceil-1\biggr)+1 = 2\Bigl\lceil\frac{n}{6}\Bigr\rceil-1.
\]
Procedura jest wywoływana rekurencyjne dla co najmniej $n-(2\lceil n/6\rceil-1)\ge n-(2(n/6+1)-1)=2n/3-1$ elementów nieprzekraczających $x$. Na mocy faktu, że sortowanie elementów w~kroku 2 algorytmu \proc{Select} zabiera czas dokładnie $\Theta(n)$, otrzymujemy rekurencję opisującą czas działania algorytmu w~tym przypadku:
\[
	T(n) \ge \begin{cases}
		\Theta(1), & \text{jeśli $n<d$}, \\
		T(\lceil n/3\rceil)+T(2n/3-1)+\Theta(n), & \text{jeśli $n\ge d$},
	\end{cases}
\]
gdzie $d>0$ jest pewną stałą. Można wykazać, stosując metodę przez podstawianie, że rozwiązaniem powyższej rekurencji jest $T(n)=\Omega(n\lg n)$, co oznacza, że algorytm w~tym wariancie nie działa w~czasie liniowym.

\exercise %9.3-2
Z~analizy algorytmu \proc{Select} wynika, że zarówno liczba elementów większych od $x$, jak i~liczba elementów mniejszych od $x$, wynoszą co najmniej $3n/10-6$. Na mocy założenia, że $n\ge140$, mamy
\[
	\frac{3n}{10}-6-\Bigl\lceil\frac{n}{4}\Bigr\rceil \ge \frac{3n}{10}-6-\Bigl(\frac{n}{4}+1\Bigr) = \frac{n-140}{20} \ge 0,
\]
skąd wnioskujemy, że obie badane liczby są nie mniejsze niż $\lceil n/4\rceil$.

\exercise %9.3-3
\note{W~rozwiązaniu przyjmujemy założenie, że elementy tablicy wejściowej są parami różne. Założenia tego zabrakło w~polskim tłumaczeniu.}

\noindent Wykorzystamy algorytm \proc{Select} do znalezienia mediany $x$ elementów z~tablicy wejściowej. Przy okazji tablica wejściowa zostanie podzielona względem $x$. Można więc w~algorytmie quicksort zastąpić wywołanie procedury \proc{Partition} wywołaniem \proc{Select} szukającym mediany. W~rezultacie wykonywane będą najbardziej zrównoważone podziały i~pesymistyczny czas algorytmu quicksort po takiej modyfikacji sprowadzi się do rekurencji $T(n)=2T(\lfloor n/2\rfloor)+O(n)$, której rozwiązaniem jest $T(n)=O(n\lg n)$.

\exercise %9.3-4
Jedynym źródłem informacji o~elementach we wspomnianym algorytmie wyznaczającym \onedash{$i$}{tą} najmniejszą wartość są porównania. Ponadto tablica wejściowa zawierająca te elementy nie jest modyfikowana -- po zakończeniu działania algorytmu ma ona identyczną zawartość jak na początku. Zakładamy, że wynik każdego porównania liczb (tzn.\ ustalenie relacji $\le$ albo $>$) jest zapamiętywany, więc przy następnej próbie sprawdzenia tych samych liczb wystarczy odczytać wynik z~pamięci. W~związku z~tym pozostaje nam udowodnić, że porównania, które przeprowadzono i~zapamiętano podczas działania algorytmu, aby jednoznacznie stwierdzić, który element jest \onedash{$i$}{tym} najmniejszym, są wystarczające, by jednoznacznie wskazać $i-1$ najmniejszych oraz $n-i$ największych elementów.

Oznaczmy przez $u$ element zwrócony przez algorytm. W~naszej procedurze będziemy korzystać z~przechodniości relacji porządku liczb. Jeśli algorytm stwierdził, że $x\le y$ oraz $y\le z$, to wnioskujemy, że $x\le z$. Wykażemy, że dzięki tej obserwacji możemy uzyskać informacje o~wzajemnym porządku w~każdej parze elementów ze zbioru wejściowego. Załóżmy nie-wprost, że istnieje para $\langle x,y\rangle$, przy czym nie potrafimy stwierdzić, czy $x\le y$, czy $x>y$. Wówczas nie potrafimy także wskazać relacji między $x$ i~każdym elementem, z~którym jesteśmy w~stanie porównać $y$, i~na odwrót. W~szczególności istnieje element $v$, którego relacja z~$u$ nie jest nam znana. Ale to leży w~sprzeczności z~faktem, że $u$ zostało jednoznacznie wskazane jako \onedash{$i$}{ty} najmniejszy ze zbioru wejściowego -- $v$ może bowiem leżeć po jednej albo po drugiej stronie $u$, a~to prowadzi do niejednoznaczności przy określaniu pozycji $u$ w~zbiorze wejściowym.

Mając informacje na temat każdej pary, możemy przejrzeć wyniki porównań elementu $u$ z~każdym innym i~na tej podstawie klasyfikować elementy do jednego z~dwóch zbiorów wynikowych.

\exercise %9.3-5
Zmodyfikujemy procedurę \proc{Randomized-Select}, dokonując zmiany w~wierszu~3. Zamiast wywoływać \proc{Randomized-Partition}, znajdziemy medianę $x$ elementów z~tablicy wejściowej za pomocą danej ,,czarnej skrzynki'', po czym podzielimy tablicę względem $x$. W~każdym wywołaniu rekurencyjnym będzie wówczas dokonywany najbardziej zrównoważony podział, więc czas działania algorytmu wyboru w~przypadku pesymistycznym przyjmie postać $T(n)=T(\lfloor n/2\rfloor)+O(n)$. Rozwiązaniem tej rekurencji jest $T(n)=O(n)$.

\exercise %9.3-6
Oznaczmy przez $S$ badany zbiór, a~przez $Q_k(S)$ -- zbiór jego kwantyli rzędu $k$. Dopuszczalnymi wartościami parametru $k$ są liczby całkowite od~1 do $n+1$, gdzie $n=|S|$. Zbiór kwantyli nie jest jednoznacznie zdefiniowany, opiszemy jednak jeden z~poprawnych sposobów jego konstrukcji. Oczywiście $Q_1(S)=\emptyset$. Jeśli $k$ jest parzyste, to $Q_k(S)$ składa się z~mediany $m$ zbioru $S$ oraz kwantyli rzędu $k/2$ zbiorów $\{\,x\in S:x<m\,\}$ i~$\{\,x\in S:x>m\,\}$. W~przypadku, gdy $k$ jest liczbą nieparzystą i~większą od~1, musimy wyznaczyć te kwantyle rzędu $k$, które znajdują się najbliżej mediany $m$ i~są od niej różne. Oznaczmy je przez $m_1$ i~$m_2$. Wówczas
\[
	Q_k(S) = Q_{(k-1)/2}(\{\,x\in S:x<m_1\,\})\cup\{m_1,m_2\}\cup Q_{(k-1)/2}(\{\,x\in S:x>m_2\,\}).
\]

Zbiór $S$ reprezentujemy jako tablicę $A$ o~parami różnych elementach. Na podstawie powyższego opisu otrzymujemy następujący algorytm oparty o~metodę ,,dziel i~zwyciężaj'':
\begin{codebox}
\Procname{$\proc{Quantiles}(A,p,r,k)$}
\li	\If $k=1$
\li		\Then \Return $\emptyset$
		\End
\li	$n\gets r-p+1$
\li	$q_1\gets p+\lfloor\lfloor k/2\rfloor(n/k)\rfloor$ \>\>\>\>\>\>\Comment pozycje zajmowane przez \onedash{$\lfloor k/2\rfloor$}{ty} i~\onedash{$\lceil k/2\rceil$}{ty} kwantyl rzędu $k$
\li	$q_2\gets p+\lfloor\lceil k/2\rceil(n/k)\rfloor$ \>\>\>\>\>\>\>w~uporządkowanej podtablicy $A[p\twodots r]$
\li	$\proc{Select}(A,p,r,q_1-p+1)$
\li	\If $q_1\ne q_2$
\li		\Then $\proc{Select}(A,q_1+1,r,q_2-q_1)$ \label{li:quantiles-second-select}
		\End
\li	$L\gets\proc{Quantiles}(A,p,q_1-1,\lfloor k/2\rfloor)$
\li	$R\gets\proc{Quantiles}(A,q_2+1,r,\lfloor k/2\rfloor)$
\li	\Return $L\cup\{A[q_1],A[q_2]\}\cup R$
\end{codebox}
Aby wyznaczyć $Q_k(S)$, należy wywołać $\proc{Quantiles}(A,1,n,k)$, gdzie $n=|S|$.

W~pseudokodzie nie rozdzielamy przypadków explicite ze względu na parzystość $k$. Po wyznaczeniu długości przetwarzanego fragmentu $A[p\twodots r]$ obliczane są pozycje $q_1$ i~$q_2$ zajmowane przez kwantyle $m_1$ i~$m_2$ (jeśli $k$ jest nieparzyste) lub medianę $m$ (jeśli $k$ jest parzyste -- wówczas $q_1=q_2$), gdyby uporządkować podtablicę $A[p\twodots r]$. Następnie korzystamy z~algorytmu \proc{Select}, aby podzielić tę podtablicę względem elementu $m_1$ lub $m$ (w~zależności od parzystości $k$). Przyjmujemy, że parametrami procedury \proc{Select} są kolejno: tablica wejściowa, indeks początku przetwarzanego fragmentu tej tablicy, indeks końca tego fragmentu oraz numer statystyki pozycyjnej, którą zamierzamy odnaleźć w~tym fragmencie. Jeśli $k$ jest nieparzyste (czyli $q_1\ne q_2$), to w~wierszu~\ref{li:quantiles-second-select} fragment tablicy $A$ zawierający elementy większe niż $m_1$ dzielimy względem $m_2$. W~tym momencie fragment $A[p\twodots q_1-1]$ składa się z~elementów mniejszych niż $A[q_1]=m_1$, fragment $A[q_2+1\twodots r]$ -- z~elementów większych niż $A[q_2]=m_2$, natomiast elementy z~$A[q_1+1\twodots q_2-1]$ (o~ile istnieją) są pomiędzy $m_1$ a~$m_2$. Teraz wystarczy znaleźć kwantyle rzędu $\lfloor k/2\rfloor$ w~pierwszym i~drugim fragmencie, wykorzystując wywołania rekurencyjne, po czym zwrócić wynikowy zbiór.

Nierekurencyjna część algorytmu zajmuje czas $O(n)$, zatem rekursja opisująca całkowity czas działania przyjmuje postać $T(n,k)\le2T(\lfloor n/2\rfloor,\lfloor k/2\rfloor)+O(n)$. Jej rozwiązaniem jest $T(n,k)=O(n\lg k)$, o~czym można się przekonać, przeprowadzając analizę z~wykorzystaniem metody drzewa rekursji.

\exercise %9.3-7
Wyznaczmy medianę $x$ zbioru $S$, a~następnie dla każdego elementu z~tego zbioru obliczmy jego odległość od $x$, czyli wartość bezwzględną z~ich różnicy. Problem sprowadza się w~tym momencie do znalezienia $k$ najmniejszych odległości, co można zrealizować, szukając wśród nich \onedash{$k$}{tej} statystyki pozycyjnej $y$, a~następnie zwracając elementy odległe od $x$ o~nie więcej niż $y$.

Pojawia się jednak pewna subtelność. O~ile elementy wejściowe z~założenia są parami różne, to odległości od mediany mogą się powtarzać -- dokładniej, każda odległość może występować w~co najwyżej dwóch egzemplarzach. Okazuje się, że algorytm \proc{Select} wykorzystywany do znalezienia $y$ działa w~takim przypadku poprawnie przy tej samej złożoności czasowej. Może się natomiast zdarzyć, że zbiór elementów o~odległościach nieprzekraczających $y$ będzie mieć $k+1$ elementów. Jest tak wówczas, gdy wśród wyznaczonych odległości są dwie kopie $y$. Przed zwróceniem wynikowego zbioru wystarczy więc usunąć z~niego element $x+y$ albo $x-y$.

Poniższy algorytm implementuje opisane podejście:
\begin{codebox}
\Procname{$\proc{Median-Vicinity}(S,k)$}
\li	$n\gets\id{length}[S]$
\li	$x\gets\proc{Select}(S,1,n,\lfloor(n+1)/2\rfloor)$
\li	\For $i\gets1$ \To $n$
\li		\Do $\id{dist}[i]\gets|S[i]-x|$
		\End
\li	$y\gets\proc{Select}(\id{dist},1,n,k)$
\li	$L\gets\emptyset$
\li	\For $i\gets1$ \To $n$
\li		\Do
			\If $|S[i]-x|\le y$
\li				\Then $L\gets L\cup\{S[i]\}$
				\End
		\End
\li	\If $|L|=k+1$
\li		\Then $L\gets L\setminus\{x+y\}$
		\End
\li	\Return $L$
\end{codebox}
Zbiór wejściowy $S$ traktujemy jak tablicę o~$n$ elementach. W~czasie $O(n)$ wyznaczana jest mediana $x$ tego zbioru, jak również tablica $\id{dist}$ zawierająca odległości poszczególnych elementów $S$ od $x$ oraz \onedash{$k$}{ta} najmniejsza wartość $y$ tej tablicy. W~pozostałej części algorytmu budowany jest zbiór $L$ złożony z~$k$ elementów najbliższych $x$. Zakładamy, że operacje inicjalizacji zbioru, pobierania jego rozmiaru i~usuwania z~niego pojedynczego elementu działają w~czasie co najwyżej proporcjonalnym do rozmiaru tego zbioru oraz że dodanie do niego pojedynczego elementu zajmuje czas stały. Założenia te można łatwo spełnić, implementując zbiór $L$ np.\ jako listę dwukierunkową (patrz podrozdział~10.2 w~Podręczniku). Wówczas czasem działania algorytmu jest $O(n)$.

\exercise %9.3-8
Niech $m_X$ będzie medianą elementów z~tablicy $X$ i~niech $m_Y$ będzie medianą elementów z~tablicy $Y$. Jeśli $m_X=m_Y$, to wśród wszystkich $2n$ elementów obu tablic co najmniej $n$ jest większych (bądź równych) od obu median i~co najmniej $n$ jest od nich mniejszych (lub równych). Wartość $m_X$ jest więc szukaną medianą wszystkich $2n$ liczb. Załóżmy teraz, że $m_X\ne m_Y$ i~bez utraty ogólności, niech $m_X<m_Y$. Pomijając teraz około $n/2$ elementów $X$ mniejszych lub równych $m_X$ i~około $n/2$ elementów $Y$ większych lub równych $m_Y$, sprowadzamy problem do identycznego, ale o~około połowę mniejszego, ponieważ wiadomo, że poszukiwana mediana znajduje się wśród pozostawionych elementów. Dokładniej, podproblem będzie operował na tablicach o~rozmiarach $\lfloor n/2\rfloor+1$.

W~celu wyznaczenia mediany $2n$ liczb nasz algorytm będzie wykorzystywał rekurencję o~przypadku brzegowym, gdy $n\le2$. Jeśli $n=1$, to jako wynik algorytmu wystarczy zwrócić mniejszy z~dwóch wejściowych elementów (zgodnie z~konwencją, że interesuje nas mediana dolna). W~przypadku zaś, gdy $n=2$, wyznaczamy większy z~elementów znajdujących się w~pierwszych komórkach tablic oraz mniejszy z~elementów zajmujących drugie komórki. Łatwo sprawdzić, że jeden z~nich to mediana dolna, a~drugi to mediana górna czterech liczb wejściowych. Jako wynik podajemy zatem minimum z~obu tych liczb.

Poniższy pseudokod implementuje opisany algorytm. Po sprawdzeniu warunku brzegowego obliczane są indeksy median dolnych elementów z~każdej tablicy, jak również indeksy ich median górnych. Te ostatnie przekazywane są w~wywołaniach rekurencyjnych w~celu zapewnienia jednakowych rozmiarów obu tablic na kolejnym poziomie rekursji. Aby odnaleźć medianę $2n$ elementów z~tablic $X[1\twodots n]$ i~$Y[1\twodots n]$, wywołujemy $\proc{Two-Array-Median}(X,1,n,Y,1,n)$.

\begin{codebox}
\Procname{$\proc{Two-Array-Median}(X,p_X,r_X,Y,p_Y,r_Y)$}
\li	\If $r_X-p_X\le1$
\li		\Then \Return $\min(\max(X[p_X],Y[p_Y]),\min(X[r_X],Y[r_Y]))$
		\End
\li	$q_X\gets\lfloor(p_X+r_X)/2\rfloor$
\li	$q_X'\gets\lceil(p_X+r_X)/2\rceil$
\li	$q_Y\gets\lfloor(p_Y+r_Y)/2\rfloor$
\li	$q_Y'\gets\lceil(p_Y+r_Y)/2\rceil$
\li	\If $X[q_X]=Y[q_Y]$
\li		\Then \Return $X[q_X]$
		\End
\li	\If $X[q_X]<Y[q_Y]$
\li		\Then \Return $\proc{Two-Array-Median}(X,q_X,r_X,Y,p_Y,q_Y')$
\li		\Else \Return $\proc{Two-Array-Median}(X,p_X,q_X',Y,q_Y,r_Y)$
		\End
\end{codebox}

Czas działania podanego algorytmu w~przypadku pesymistycznym jest opisany przez rekurencję $T(n)=T(\lfloor n/2\rfloor+1)+\Theta(1)$, gdzie $n=r_X-p_X+1$. Jej rozwiązaniem jest oczywiście $T(n)=O(\lg n)$.

\exercise %9.3-9
Oznaczmy przez $y_r$ współrzędną $y$ głównego rurociągu, a~przez $y_1$ i~$y_2$ -- odpowiednio, medianę dolną i~górną współrzędnych $y$ wież wiertniczych. Ponadto dla ustalonego $y_r$, niech $s$ będzie sumą długości odnóg \onedash{północ}{południe}.

Niech $y_1\le y_r\le y_2$. Jeśli teraz zmienimy wartość $y_r$ o~$d$, jednocześnie nadal utrzymując ją pomiędzy wartościami obu median (w~przypadku, gdy $n$ jest nieparzyste, może być tylko $d=0$), to nową sumą długości odnóg będzie $s'=s+d\lfloor n/2\rfloor-d\lfloor n/2\rfloor=s$. Widać stąd, że wykonując taką zmianę, nie pogarszamy ani nie polepszamy wyniku -- każde $y_r$ pomiędzy medianami jest równie dobre.

Zmodyfikujmy teraz wartość $y_r$ o~$d>0$, ale w~taki sposób, aby $y_r<y_1$ albo $y_r>y_2$. Gdy $n$ jest parzyste, to po takiej zmianie po jednej stronie rurociągu jest co najmniej $n/2+1$ wież, a~po drugiej -- co najwyżej $n/2-1$. Stąd otrzymujemy oszacowanie na nową sumę długości odnóg:
\[
    s' \ge s+d(n/2+1)-d(n/2-1) = s+2d > s.
\]
W~przypadku, gdy $n$ jest liczbą nieparzystą, rurociąg w~nowym położeniu po jednej stronie ma co najmniej $(n+1)/2$ wież wiertniczych, a~po drugiej stronie co najwyżej $(n-1)/2$. Nowa suma długości odnóg wynosi więc
\[
    s' \ge s+d(n+1)/2-d(n-1)/2 = s+d > s.
\]

A~zatem, niezależnie od parzystości $n$, przesunięcie rurociągu poza obszar wyznaczony przez mediany $y_1$ i~$y_2$, wiąże się z~powiększeniem sumy długości odnóg -- dowolna wartość $y_r$ pomiędzy tymi medianami jest optymalna. Problem sprowadza się zatem do wyznaczenia mediany współrzędnych $y$ wież wiertniczych, można go więc rozwiązać w~czasie liniowym względem $n$.

\problems

\problem{Sortowanie największych $i$ elementów} %9-1

\subproblem %9-1(a)
Liczby można posortować algorytmem sortowania przez scalanie, który w~najgorszym przypadku potrzebuje czasu $\Theta(n\lg n)$. Wypisanie $i$ największych liczb otrzymanej tablicy, poprzez zwyczajne przeglądnięcie jej $i$ ostatnich elementów, zajmuje czas $\Theta(i)$. Całkowity czas algorytmu w~przypadku pesymistycznym wynosi zatem $\Theta(n\lg n+i)$.

\subproblem %9-1(b)
Zbudowanie kopca typu max dla kolejki priorytetowej algorytmem \proc{Build-Max-Heap} wymaga czasu $\Theta(n)$. Wykonanie kolejno $i$ operacji \proc{Extract-Max} na kopcu o~co najwyżej $n$ elementach wymaga czasu $i\cdot O(\lg n)=O(i\lg n)$. Z~kolei połowa tych operacji będzie wykonywana na kopcu o~co najmniej $n/2$ elementach. Zajmą one czas $(i/2)\cdot\Omega(\lg(n/2))=\Omega(i\lg n)$ w~najgorszym przypadku. Stąd mamy, że wszystkie operacje ekstrakcji zostaną wykonane w~czasie $\Theta(i\lg n)$ w~przypadku pesymistycznym, a~zatem całkowitym czasem algorytmu jest $\Theta(n+i\lg n)$.

\subproblem %9-1(c)
Aby osiągnąć najlepszy czas w~przypadku pesymistycznym, użyjemy procedury \proc{Select} do znalezienia \onedash{$i$}{tej} statystyki pozycyjnej. Jednocześnie tablica elementów wejściowych zostanie podzielona względem tejże statystyki. Sortowanie $i$ największych liczb można wykonać algorytmem sortowania przez scalanie, które w~najgorszym przypadku zajmuje czas $\Theta(i\lg i)$. Stąd całkowity czas działania algorytmu wynosi $\Theta(n+i\lg i)$, co czyni go najbardziej efektywnym spośród wszystkich rozważanych algorytmów w~niniejszym problemie.

\problem{Mediana ważona} %9-2

\subproblem %9-2(a)
Dla tak przyjętych wag elementów mamy
\[
	\sum_{x_i<x_k}w_i = \frac{k-1}{n} \quad\text{oraz}\quad \sum_{x_i>x_k}w_i = \frac{n-k}{n}.
\]
Jeśli ograniczymy obie sumy od góry przez $1/2$, to dostaniemy, że medianą ważoną elementów $x_1$, $x_2$,~\dots,~$x_n$ jest $x_k$, gdzie $n/2\le k\le n/2+1$, czyli $k=\lfloor(n+1)/2\rfloor$ lub $k=\lceil(n+1)/2\rceil$. A~zatem $x_k$ jest również zwykłą medianą elementów $x_1$, $x_2$,~\dots,~$x_n$.

\subproblem %9-2(b)
Po posortowaniu elementów wyznaczenie mediany ważonej odbywa się w~prosty sposób -- należy przeglądać elementy w~kolejności rosnącej i~sumować ich wagi aż do momentu, kiedy suma wag osiągnie lub przekroczy $1/2$. Wówczas wystarczy zwrócić ostatnio przeglądany element.

Złożoność tej procedury zależy od efektywności sortowania, ale używając odpowiedniego algorytmu, jesteśmy w~stanie osiągnąć czas $O(n\lg n)$ w~pesymistycznym przypadku.

\subproblem %9-2(c)
Rozwiążemy problem metodą ,,dziel i~zwyciężaj''. Przyjmujemy, że elementy oraz ich wagi są przechowywane w~tablicach o~rozmiarze $n$.

Najpierw dzielimy elementy względem ich mediany algorytmem zbliżonym do \proc{Select}, pamiętając przy tym, aby wraz z~elementami przenosić na nowe pozycje także ich wagi. Wyznaczamy wartości $W_L$ i~$W_R$ -- sumy wag elementów, odpowiednio, mniejszych i~większych od mediany. Jeśli $W_L<1/2$ oraz $W_R<1/2$, to medianą ważoną jest zwykła mediana. W~przeciwnym przypadku rozważamy część tablicy wejściowej zawierającą elementy o~większej sumie wag, przekazując ją do następnego wywołania rekurencyjnego.

\begin{codebox}
\Procname{$\proc{Weighted-Median}(A,w,p,r)$}
\li	\If $r-p+1\le2$
\li		\Then
			\If $w_p\ge w_r$ \label{li:weighted-median-boundary-case-begin}
\li				\Then \Return $A[p]$
\li				\Else \Return $A[r]$
				\End \label{li:weighted-median-boundary-case-end}
		\End
\li	podziel elementy w~$A[p\twodots r]$ i~ich wagi względem mediany tablicy $A[p\twodots r]$ \label{li:weighted-median-partition}
\li	$q\gets\lfloor(p+r)/2\rfloor$
\li	$W_L\gets0$
\li	\For $i\gets p$ \To $q-1$
\li		\Do $W_L\gets W_L+w_i$
		\End
\li	$W_R\gets1-W_L-w_q$
\li	\If $W_L<1/2$ i~$W_R<1/2$
\li		\Then \Return $A[q]$
		\End
\li	\If $W_L>1/2$
\li		\Then
			$w_q\gets w_q+W_R$
\li			\Return $\proc{Weighted-Median}(A,w,p,q)$
\li		\Else
			$w_q\gets w_q+W_L$
\li			\Return $\proc{Weighted-Median}(A,w,q,r)$
		\End
\end{codebox}

W~powyższym pseudokodzie rozważana jest tablica $A[p\twodots r]$ o~$n=r-p+1$ elementach. Pierwszym krokiem jest sprawdzenie przypadku brzegowego -- jeśli $n=1$, to wystarczy zwrócić jedyny element wejściowy, a~jeśli $n=2$, to zwracany jest element o~większej wadze. Operację tę implementujemy za pomocą pojedynczej instrukcji \kw{if} w~wierszach \twodashes{\ref{li:weighted-median-boundary-case-begin}}{\ref{li:weighted-median-boundary-case-end}}. Następnie elementy wraz z~ich wagami są dzielone względem mediany (o~indeksie $q=\lfloor(p+r)/2\rfloor$) i~obliczane są sumy wag obu części tego podziału. W~zależności od tego, czy wartości te są mniejsze czy większe niż $1/2$, zwracana jest wyznaczona wcześniej mediana, bądź algorytm jest wywoływany rekurencyjnie dla odpowiedniej części podziału. Aby zachować warunek mówiący o~tym, że suma wag wszystkich elementów tablicy wejściowej wynosi 1, przed kolejnym wywołaniem algorytmu zwiększamy wagę mediany do odpowiedniej wartości -- $w_q$ ,,kumuluje'' więc wagi tej części tablicy, którą odrzucamy.

Przy założeniu, że wiersz \ref{li:weighted-median-partition} w~najgorszym przypadku zajmuje czas $\Theta(n)$, pesymistyczny czas działania opisanego algorytmu spełnia równanie rekurencyjne $T(n)=T(\lfloor n/2\rfloor+1)+\Theta(n)$, którego rozwiązaniem jest $T(n)=\Theta(n)$.

\subproblem %9-2(d)
Niech $p_k$ będzie medianą ważoną danych punktów $p_1$, $p_2$,~\dots,~$p_n$ (które są liczbami rzeczywistymi) i~ich wag $w_1$, $w_2$,~\dots,~$w_n$. Dla dowolnego punktu $p$ niech $f(p)=\sum_{i=1}^nw_i|p-p_i|$. Szukamy takiego punktu $p$ spośród danych, dla którego $f(p)$ przyjmuje możliwie najmniejszą wartość. Pokażemy, że szukanym punktem jest $p_k$.

Niech $p_l$ będzie dowolnym punktem różnym od $p_k$. Zbadamy znak różnicy
\[
    f(p_l)-f(p_k) = \sum_{i=1}^nw_i|p_l-p_i|-\sum_{i=1}^nw_i|p_k-p_i| = \sum_{i=1}^nw_i(|p_l-p_i|-|p_k-p_i|).
\]

Niech $p_l<p_k$. Zauważmy, że w~przypadku, gdy $p_l\le p_i<p_k$, zachodzi $|p_l-p_i|-|p_k-p_i|=p_i-p_l-p_k+p_i\ge2p_l-p_l-p_k=p_l-p_k$. Mamy zatem
\begin{align*}
    f(p_l)-f(p_k) &= \sum_{p_i<p_l}w_i(p_l-p_i-p_k+p_i) \\
	&\quad {}+\!\!\!\sum_{p_l\le p_i<p_k}\!\!\!\!w_i(p_i-p_l-p_k+p_i) \\
	&\quad {}+\,\sum_{p_k\le p_i}w_i(p_i-p_l-p_i+p_k) \\
	&\ge (p_k-p_l)\biggl(\sum_{p_k\le p_i}w_i-\sum_{p_i<p_k}w_i\biggr).
\end{align*}
Pierwszy czynnik ostatniego iloczynu jest dodatni. Na mocy faktu, że wszystkie wagi sumują się do~1 oraz z~definicji mediany ważonej, mamy
\[
    \sum_{p_k\le p_i}w_i-\sum_{p_i<p_k}w_i = 1-2\sum_{p_i<p_k}w_i \ge 1-2\cdot1/2 = 0,
\]
czyli drugi z~czynników jest nieujemny. Pokazaliśmy tym samym, że dla dowolnego $p_l<p_k$ zachodzi $f(p_l)>f(p_k)$, co oznacza, że mediana ważona $p_k$ minimalizuje wartość funkcji $f$.

Gdy $p_l>p_k$, wnioskujemy podobnie, otrzymując
\[
    f(p_l)-f(p_k) \ge (p_k-p_l)\biggl(\sum_{p_k<p_i}w_i-\sum_{p_i\le p_k}w_i\biggr) = (p_k-p_l)\biggl(2\sum_{p_k<p_i}w_i-1\biggr).
\]
W~powyższym iloczynie pierwszy z~czynników jest ujemny, a~drugi można ograniczyć od góry przez $2\cdot1/2-1=0$. A~zatem także w~tym przypadku mamy $f(p_l)>f(p_k)$, co kończy dowód twierdzenia.

\subproblem %9-2(e)
Niech $p_i=\langle x_i,y_i\rangle$, gdzie $i=1$, 2,~\dots,~$n$, będą punktami wejściowymi w~\onedash{2}{wymiarowym} problemie lokalizacji urzędu pocztowego z~metryką miejską. Jego rozwiązaniem jest taki punkt $p=\langle x_p,y_p\rangle$, dla którego suma $\sum_{i=1}^nw_i(|x_p-x_i|+|y_p-y_i|)$ przyjmuje najmniejszą możliwą wartość. Zauważmy, że
\[
    \min_{\langle x_p,y_p\rangle\in\mathbb{R}\times\mathbb{R}}\biggl(\sum_{i=1}^nw_i(|x_p-x_i|+|y_p-y_i|)\biggr) = \min_{x_p\in\mathbb{R}}\biggl(\sum_{i=1}^nw_i|x_p-x_i|\biggr)+\min_{y_p\in\mathbb{R}}\biggl(\sum_{i=1}^nw_i|y_p-y_i|\biggr).
\]
A~zatem w~celu wyznaczenia optymalnego punktu $p$ wystarczy znaleźć jego współrzędne niezależnie jako rozwiązania \onedash{1}{wymiarowej} wersji problemu. Na podstawie poprzedniego punktu zadanie to sprowadza się do wyznaczenia mediany ważonej $x_p$ spośród elementów $x_1$, $x_2$,~\dots,~$x_n$ o~wagach $w_1$, $w_2$,~\dots,~$w_n$ i~analogicznie mediany ważonej $y_p$ spośród elementów $y_1$, $y_2$,~\dots,~$y_n$ o~tych samych wagach oraz zwróceniu pary $\langle x_p,y_p\rangle$.

Zauważmy, że oba ciągi współrzędnych, które przeszukujemy celem znalezienia ich median ważonych, mogą zawierać powtarzające się elementy -- w~szczególności wszystkie punkty mogą mieć równe pierwsze (lub drugie) współrzędne. Jedynym miejscem w~procedurze \proc{Weighted-Median} z~części~(c), gdzie mają znaczenie wartości elementów, jest linia~\ref{li:weighted-median-partition}, w~której dokonywany jest podział elementów algorytmem opartym o~\proc{Select}. Okazuje się jednak, o~czym łatwo się przekonać, że algorytm ten działa poprawnie nawet wówczas, gdy elementy w~tablicy wejściowej powtarzają się -- zmienić się może jedynie jego czas działania, który w~przypadku pesymistycznym wzrasta do kwadratowego. A~zatem procedura z~punktu~(c) może posłużyć jako narzędzie do szukania median ważonych współrzędnych punktów w~algorytmie rozwiązującym rozważany wariant problemu lokalizacji urzędu pocztowego.

\problem{Małe statystyki pozycyjne} %9-3

\subproblem %9-3(a)
\note{W~tłumaczeniu występuje błąd w~sformułowaniu rekurencji\/ $U_i(n)$. Wartość\/ $T(n)$ powinna być zwracana dla\/ $i\ge n/2$, a~część rekurencyjna -- dla pozostałych wartości\/ $i$.}

\noindent Jeśli $i\ge n/2$, to w~celu znalezienia \onedash{$i$}{tej} statystyki pozycyjnej tablicy wejściowej wywołujemy algorytm \proc{Select}. Stąd $U_i(n)=T(n)$.

Załóżmy teraz, że $i<n/2$ i~że tablicą wejściową jest $A[p\twodots r]$. Niech $m=\lfloor n/2\rfloor$. Dzielimy tablicę wejściową na dwie podtablice $A[p\twodots p+m-1]$ oraz $A[p+m\twodots r]$. Dla każdego $j=0$, 1,~\dots,~$m-1$ porównamy teraz elementy $A[p+j]$ oraz $A[p+m+j]$ i~ewentualnie zamienimy je, aby zachodziło $A[p+j]>A[p+m+j]$. Wywołujemy następnie nasz algorytm rekurencyjnie w~celu znalezienia \onedash{$i$}{tej} statystyki pozycyjnej podtablicy $A[p+m\twodots r]$. W~wywołaniu rekurencyjnym stosujemy jednak zmodyfikowaną wersję procedury partycjonującej wykorzystywanej przez algorytm \proc{Select}. Będzie ona dokonywać równoczesnego podziału obu części tablicy wejściowej. Dokładniej, wraz z~każdą zamianą elementów $A[j]$ i~$A[k]$, gdzie $p+m\le j$, $k\le p+2m-1$, zamieniane będą również elementy $A[j-m]$ i~$A[k-m]$. Dzięki temu po wyznaczeniu \onedash{$i$}{tego} najmniejszego elementu podtablicy $A[p+m\twodots r]$, fragmenty $A[p\twodots p+i-1]$ oraz $A[p+m\twodots p+m+i-1]$ będą zawierać po $i$ najmniejszych elementów ze swoich podtablic. A~zatem \onedash{$i$}{ty} najmniejszy element tablicy wejściowej znajduje się w~jednym z~tych dwóch fragmentów. Wystarczy więc teraz połączyć je w~tablicę $B[1\twodots2i]$, a~następnie zwrócić $\proc{Select}(B,1,2i,i)$.

Sprawdzimy teraz, ile wynosi $U_i(n)$ w~przypadku, gdy $i<n/2$. W~pierwszej fazie algorytm wykonuje $\lfloor n/2\rfloor$ porównań elementów pochodzących z~dwóch fragmentów tablicy wejściowej. Wywołanie rekurencyjne wprowadza składnik $U_i(\lceil n/2\rceil)$, zaś algorytm \proc{Select} wywołany na tablicy $B$ wykonuje $T(2i)$ porównań. Stąd otrzymujemy
\[
    U_i(n) = \lfloor n/2\rfloor+U_i(\lceil n/2\rceil)+T(2i).
\]

\subproblem %9-3(b)
Stosując metodę podstawiania, wykażemy, że jeśli $i<n/2$, to
\[
	U_i(n) \le n+cT(2i)\lg(n/i),
\]
gdzie $c>0$ jest pewną stałą.

Jako przypadek bazowy indukcji będziemy rozważać wyrazy $U_i(n)$, w~których $2i<n\le4i$, skąd $n/4\le i<n/2$. Wykażemy, że spełniają one podane oszacowanie. Mamy:
\[
    U_i(n) = \lfloor n/2\rfloor+U_i(\lceil n/2\rceil)+T(2i) < n+T(\lceil n/2\rceil)+T(2i).
\]
Jeśli teraz ograniczymy powyższe wyrażenie od góry przez $n+cT(2i)\lg(n/i)$ i~rozwiążemy ze względu na $c$, to otrzymamy
\[
    c \ge \frac{1}{\lg(n/i)}\biggl(\frac{T(\lceil n/2\rceil)}{T(2i)}+1\biggr).
\]
Pierwszy czynnik po prawej stronie powyższej nierówności można ograniczyć od góry przez $1/\lg2=1$. W~celu oszacowania drugiego czynnika zauważmy, że rekurencja $T(n)$ jest funkcją dodatnią klasy $\Theta(n)$, zatem istnieją stałe $d_1$, $d_2>0$, że dla każdego $n\ge1$ spełnione są nierówności $d_1n\le T(n)\le d_2n$. Mamy zatem
\[
    \frac{T(\lceil n/2\rceil)}{T(2i)} \le \frac{d_2\lceil n/2\rceil}{2d_1i} < \frac{d_2n}{d_1n/2} = \frac{2d_2}{d_1},
\]
a~stąd wynika, że $c\ge2d_2/d_1+1$. Pokazaliśmy, że $c$ istotnie jest stałą, a~zatem podstawa indukcji zachodzi.

W~drugim kroku indukcyjnym mamy $n>4i$ i~przyjmujemy założenie
\[
    U_i(\lceil n/2\rceil) \le \lceil n/2\rceil+cT(2i)\lg(\lceil n/2\rceil/i).
\]
Wówczas:
\begin{align*}
    U_i(n) &= \lfloor n/2\rfloor+U_i(\lceil n/2\rceil)+T(2i) \\
	&\le \lfloor n/2\rfloor+\lceil n/2\rceil+cT(2i)\lg(\lceil n/2\rceil/i)+T(2i) \\
	&= n+cT(2i)\lg\lceil n/2\rceil-cT(2i)\lg i+T(2i) \\
	&\le n+cT(2i)\lg n-cT(2i)\lg i.
\end{align*}
Ostatnia nierówność jest spełniona, jeżeli $cT(2i)\lg\lceil n/2\rceil+T(2i)\le cT(2i)\lg n$, co po uproszczeniu sprowadza się do warunku $c\lg\frac{n}{\lceil n/2\rceil}\ge1$. Łatwo zauważyć, że aby go spełnić dla $n\ge5$, wystarczy przyjąć $c\ge\log_{5/3}2$, co kończy dowód oszacowania na $U_i(n)$.

\subproblem %9-3(c)
Ponieważ $i$ jest stałą, to $T(2i)$ również można potraktować jako wartość stałą. Na mocy poprzedniego punktu mamy
\[
	U_i(n) = n+O(T(2i)\lg(n/i)) = n+O(\lg n-\lg i) = n+O(\lg n).
\]

\subproblem %9-3(d)
Dla $k>2$ oszacowanie wynika natychmiast z~części~(b) po podstawieniu $i=n/k$, bo wtedy oczywiście $i<n/2$.

Jeśli $k=2$, to $i=n/2$ i~rozwiązaniem rekurencji $U_i(n)$ jest $U_i(n)=T(n)=O(n)$. Teza w~tym przypadku przyjmuje postać $U_i(n)=n+O(T(n))$, co oczywiście jest spełnione, ponieważ $n+O(T(n))=O(n)$.

\endinput
