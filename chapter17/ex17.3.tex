\subchapter{Metoda potencjału}

\exercise %17.3-1
Dla każdego $i\ge0$ definiujemy potencjał $\Phi'(D_i)=\Phi(D_i)-\Phi(D_0)$.
Wtedy $\Phi'(D_i)\ge0$, $\Phi'(D_0)=\Phi(D_0)-\Phi(D_0)=0$, a~koszt zamortyzowany względem $\Phi'$ wynosi:
\begin{align*}
	\widehat{c_i}' &= c_i+\Phi'(D_i)-\Phi'(D_{i-1}) \\
	&= c_i+(\Phi(D_i)-\Phi(D_0))-(\Phi(D_{i-1})-\Phi(D_0)) \\
	&= c_i+\Phi(D_i)-\Phi(D_{i-1}) \\
	&= \widehat{c_i}.
\end{align*}

\exercise %17.3-2
W~naszej analizie skorzystamy z~następującej funkcji potencjału:
\[
	\Phi(D_i) = \begin{cases}
		0, & \text{jeśli $i=0$}, \\
		k+2, & \text{jeśli $i=2^k$, $k=0$, 1, \dots, $\lfloor\lg n\rfloor$}, \\
		\Phi(D_{i-1})+2, & \text{w~pozostałych przypadkach}.
	\end{cases}
\]
Gdy $i=1$ lub $i$ nie jest potęgą 2, to $c_i=1$, więc $\widehat{c_i}=c_i+\Phi(D_i)-\Phi(D_{i-1})=1+2=3$.
Przyjmijmy teraz, że $i=2^k$ dla pewnego $k=1$, 2, \dots, $\lfloor\lg n\rfloor$.
Mamy:
\begin{align*}
	\widehat{c_{2^k}} &= c_{2^k}+\Phi(D_{2^k})-\Phi(D_{2^k-1}) \\
	&= 2^k+k+2-\bigl(\Phi(D_{2^{k-1}})+2(2^k-1-2^{k-1})\bigr) \\
	&= 2^k+k+2-(k+1+2^k-2) \\
	&= 3.
\end{align*}
Dzięki wybraniu szczególnej funkcji $\Phi$, jako koszt zamortyzowany każdej operacji mogliśmy przyjąć 3, podobnie jak w~rozwiązaniu \refExercise{17.2-2}.
Oczywiście $\Phi(D_i)\ge\Phi(D_0)$ dla każdego $i\ge1$, zatem łączny koszt zamortyzowany ciągu $n$ operacji stanowi górne ograniczenie ich kosztu faktycznego, co daje pesymistyczny koszt tego ciągu równy $O(n)$.

\exercise %17.3-3
Na początkowo pustym kopcu $D_0$ wykonujemy ciąg operacji \proc{Insert} i~\proc{Extract-Min}.
Przez $D_i$ oznaczmy kopiec powstały tuż po wykonaniu $i$\nbhyphen tej operacji, a~przez $n_i$ -- liczbę jego węzłów.
Niech $d_i(x)$ oznacza głębokość węzła $x$ w~kopcu $D_i$ i~niech $k>0$ będzie taką stałą, że zarówno koszt $c_I$ operacji \proc{Insert}, jak i~koszt $c_X$ operacji \proc{Extract-Min} na $n$\nbhyphen elementowym kopcu ograniczony jest z~góry przez $k\lg n$.
Rozważymy funkcję potencjału
\[
	\Phi(D_i) = \sum_{x\in D_i}k(d_i(x)+1).
\]
Dla pustego kopca mamy $\Phi(D_0)=0$, a~dla każdego $i\ge1$, $\Phi(D_i)\ge0$.

Po wstawieniu do kopca $D_{i-1}$ nowego węzła, w~wynikowym kopcu $D_i$ nowy węzeł znajduje się na głębokości $\lfloor\lg(n_{i-1}+1)\rfloor$.
Mamy więc $n_i=n_{i-1}+1$ oraz
\[
	\Phi(D_i)-\Phi(D_{i-1}) = k(\lfloor\lg(n_{i-1}+1)\rfloor+1) = k(\lfloor\lg n_i\rfloor+1),
\]
skąd zamortyzowanym kosztem operacji \proc{Insert} jest
\[
	\widehat{c_I} = c_I+\Phi(D_i)-\Phi(D_{i-1}) \le k\lg n_{i-1}+k(\lfloor\lg n_i\rfloor+1) < k\lg n_i+k(\lg n_i+1) = O(\lg n_i).
\]

Po wykonaniu operacji \proc{Extract-Min} kopiec $D_{i-1}$ zostaje pozbawiony węzła znajdującego się na najniższym jego poziomie.
Potencjał kopca maleje więc o~$k(\lfloor\lg n_{i-1}\rfloor+1)$ i~zamortyzowany koszt tej operacji można oszacować przez
\[
	\widehat{c_X} = c_X+\Phi(D_i)-\Phi(D_{i-1}) \le k\lg n_{i-1}-k(\lfloor\lg n_{i-1}\rfloor+1) < k(\lg n_{i-1}-\lfloor\lg n_{i-1}\rfloor) < k = O(1).
\]

\exercise %17.3-4
Podobnie jak w~analizie w~Podręczniku, definiujemy funkcję $\Phi$ dla stosu jako jego wysokość.
Mamy tu $\Phi(D_0)=s_0$ i~$\Phi(D_n)=s_n$.
Na podstawie równania (17.3) oraz oszacowania na łączny koszt $n$ operacji na początkowo pustym stosie dostajemy, że sumaryczny koszt zamortyzowany ciągu $n$ operacji na stosie o~początkowo $s_0$ elementach wynosi
\[
	\sum_{i=1}^n\widehat{c_i} = \sum_{i=1}^nc_i+\Phi(D_n)-\Phi(D_0) = O(n)+s_n-s_0.
\]
Przy założeniu, że $s_n\ge s_0$, powyższe wyrażenie stanowi górne ograniczenie łącznego kosztu faktycznego.

\exercise %17.3-5
Jeśli $n=\Omega(b)$, to $b=O(n)$.
Zgodnie z~wynikiem przedstawionym w~Podręczniku, łączny koszt $n$ operacji \proc{Increment} można ograniczyć od góry przez $2n+b=O(n)$.

\exercise %17.3-6
Podamy bardziej efektywne rozwiązanie \refExercise{10.1-6}, które pozwoli nam na osiągnięcie zamierzonych czasów zamortyzowanych.
Elementy kolejki będziemy przechowywać na dwóch stosach $S_1$ i~$S_2$.
Operacja \proc{Enqueue} w~tej implementacji będzie wywołaniem \proc{Push} na stosie $S_1$, a~operacja \proc{Dequeue} będzie realizowana przez poniższy pseudokod:
\begin{codebox}
\Procname{$\proc{Effective-Stack-Dequeue}(S_1,S_2)$}
\li	\If $\proc{Stack-Empty}(S_2)$
\li		\Then \While $\proc{Stack-Empty}(S_1)=\const{false}$
\li				\Do $\proc{Push}(S_2,\proc{Pop}(S_1))$
				\End
		\End
\li	\Return $\proc{Pop}(S_2)$
\end{codebox}
Rzeczywisty koszt operacji \proc{Enqueue} wynosi $c_E=1$, zaś rzeczywisty koszt operacji \proc{Dequeue} $c_D$ zależy od pustości stosu $S_2$.
Jeśli stos ten nie jest pusty, to $c_D=1$.
W~przeciwnym przypadku wykonywanych jest $k$ operacji \proc{Push} oraz $k+1$ operacji \proc{Pop}, gdzie $k$ to rozmiar stosu $S_1$ sprzed wykonania operacji.
Wówczas $c_D=2k+1$.

Jako potencjał naszej struktury danych przyjmiemy dwukrotność rozmiaru stosu $S_1$.
W~trakcie wstawiania do kolejki pierwszy stos zwiększa swój rozmiar o~1, dlatego zamortyzowany koszt tej operacji jest równy $\widehat{c_E}=c_E+2\cdot1=3$.
Gdy stos $S_2$ jest niepusty, to koszt zamortyzowany operacji \proc{Dequeue} $\widehat{c_D}$ wynosi $c_D$.
W~przeciwnym razie podczas usuwania elementu z~kolejki stos $S_1$ zmniejsza swój rozmiar o~$k$, więc $\widehat{c_D}=c_D-2k=1$.
Koszty zamortyzowane obu operacji w~tej implementacji wynoszą zatem $O(1)$.

\exercise %17.3-7
\note{Zgodnie z~oryginalnym tekstem w\/~$S$ dopuszczalne są powtarzające się elementy, dlatego\/ $S$ jest multizbiorem dynamicznym, a~nie zbiorem dynamicznym.}

\noindent Do zaimplementowania opisanej struktury danych użyjemy listy dwukierunkowej.
Operacja \proc{Insert} będzie zwyczajnym wstawieniem elementu na tę listę.
Z~kolei w~operacji \proc{Delete-Larger-Half} znaleziona zostanie mediana $x$ multizbioru $S$, następnie usunięte zostaną wszystkie elementy większe od $x$ i~wreszcie wykasowana zostanie odpowiednia ilość elementów równych $x$ tak, aby łączną liczbą usuniętych elementów było $\lceil|S|/2\rceil$.
Do efektywnego odszukania mediany zostanie użyty algorytm \proc{Select} z~podrozdziału 9.3.

Dla analizy kosztu zamortyzowanego przyjmijmy funkcję potencjału $\Phi(S)=2|S|$.
Rzeczywisty koszt $c_I$ operacji \proc{Insert} wynosi 1, co daje koszt zamortyzowany
\[
	\widehat{c_I} = c_I+(2(|S|+1)-2|S|) = 3.
\]
Rzeczywistym kosztem $c_D$ wywołania $\proc{Delete-Larger-Half}(S)$ jest $|S|$, a~kosztem zamortyzowanym
\[
	\widehat{c_D} \le c_D+(|S|-2|S|) = 0.
\]
A~zatem obie operacje potrzebują zamortyzowanego kosztu rzędu $O(1)$ lub, równoważnie, czas działania dowolnego ciągu $m$ tych operacji wynosi $O(m)$.
