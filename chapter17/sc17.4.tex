\subchapter{Tablice dynamiczne}

\exercise %17.4-1
Z~twierdzeń 11.6 i~11.8 oraz z~wniosku 11.7 wynika, że koszty operacji wstawiania i~wyszukiwania elementu w~tablicy z~haszowaniem z~adresowaniem otwartym dążą do nieskończoności, gdy tylko współczynnik zapełnienia zbliża się do 1.
Wybranie pewnej stałej mniejszej niż 1 jako maksymalnego dopuszczalnego współczynnika zapełnienia takiej tablicy z~haszowaniem pozwala na ograniczenie tych kosztów przez stałą.
Gdy tablica wypełni się, wszystkie jej elementy będą wymagały przeniesienia do tablicy o~większym rozmiarze, co powoduje, że operacja wstawiania w~takiej sytuacji zabiera więcej czasu niż $O(1)$.
Przeprowadzając analogiczną analizę do tej z~Podręcznika dla \proc{Table-Insert}, można wykazać, że oczekiwany czas zamortyzowany operacji wstawiania do dynamicznej tablicy z~haszowaniem wynosi $O(1)$.

\exercise %17.4-2
Gdy $\alpha_{i-1}\ge1/2$, to jedyną sytuacją, kiedy współczynnik zapełnienia zostaje obniżony poniżej $1/4$, jest wykonanie operacji \proc{Table-Delete} na tablicy dynamicznej, w~której $\id{size}_{i-1}\le2$ oraz $\id{num}_{i-1}=1$.
Oczywiście zamortyzowany koszt operacji usuwania w~takim przypadku jest ograniczony przez stałą.

Przy założeniu, że $\id{size}_{i-1}>2$, faktyczny koszt operacji wynosi $c_i=1$.
Dla $\alpha_i\ge1/2$ mamy
\begin{align*}
	\widehat{c_i} &= c_i+\Phi_i-\Phi_{i-1} \\
	&= 1+(2\cdot\id{num}_i-\id{size}_i)-(2\cdot\id{num}_{i-1}-\id{size}_{i-1}) \\
	&= 1+2\cdot\id{num}_i-\id{size}_i-2\cdot\id{num}_i-2+\id{size}_i \\
	&= -1.
\end{align*}
Jeśli natomiast $\alpha_i<1/2$, to
\begin{align*}
	\widehat{c_i} &= c_i+\Phi_i-\Phi_{i-1} \\
	&= 1+(\id{size}_i/2-\id{num}_i)-(2\cdot\id{num}_{i-1}-\id{size}_{i-1}) \\
	&= 1+\id{size}_i/2-\id{num}_i-2\cdot\id{num}_i-2+\id{size}_i \\
	&= (3/2)\id{size}_i-3\cdot\id{num}_i-1 \\[1mm]
	&= (3/2)\id{size}_i-3\alpha_i\id{size}_i-1 \\[1mm]
	&< (3/2)\id{size}_i-(3/2)\id{size}_i-1 \\
	&= -1.
\end{align*}
A~zatem we wszystkich przypadkach $\widehat{c_i}$ jest ograniczone od góry przez stałą.

\exercise %17.4-3
W~naszej analizie skorzystamy z~oznaczeń identycznych do tych z~Podręcznika.
Potrzebna nam będzie następująca obserwacja:

\medskip
\noindent\textsf{\textbf{Lemat.}} \textit{Dla dowolnej liczby rzeczywistej\/ $x$ oraz dowolnej nieujemnej liczby rzeczywistej\/ $a$ zachodzi
\[
	|x+a|\ge |x|-a.
\]}
\begin{proof}
Jeśli $x\ge0$, to $|x+a|=x+a\ge x-a=|x|-a$.

Jeśli $-a\le x<0$, to $|x+a|=x+a\ge-x-a=|x|-a$.

Jeśli $x<-a$, to $|x+a|=-x-a=|x|-a$.
\end{proof}

Jeśli wywołanie operacji \proc{Delete} nie powoduje zmniejszenia rozmiaru tablicy, to faktyczny koszt operacji wynosi $c_i=1$.
Wówczas $\id{num}_i=\id{num}_{i-1}-1$, a~$\id{size}_i=\id{size}_{i-1}$ i~na podstawie powyższego lematu mamy
\begin{align*}
	\widehat{c_i} &= c_i+\Phi_i-\Phi_{i-1} \\
	&= 1+|2\cdot\id{num}_i-\id{size}_i|-|2\cdot\id{num}_{i-1}-\id{size}_{i-1}| \\
	&= 1+|2\cdot\id{num}_i-\id{size}_i|-|2\cdot\id{num}_i+2-\id{size}_i| \\
	&\le 1+|2\cdot\id{num}_i-\id{size}_i|-\bigl(|2\cdot\id{num}_i-\id{size}_i|-2\bigr) \\
	&= 3.
\end{align*}
W~sytuacji, w~której tablica zostaje zmniejszona, faktyczny koszt wynosi $c_i=\id{num}_i+1$ oraz zachodzą równości $\id{size}_i/2=\id{size}_{i-1}/3=\id{num}_{i-1}=\id{num}_i+1$.
Koszt zamortyzowany w~tym przypadku wynosi więc
\begin{align*}
	\widehat{c_i} &= c_i+\Phi_i-\Phi_{i-1} \\
	&= (\id{num}_i+1)+|2\cdot\id{num}_i-\id{size}_i|-|2\cdot\id{num}_{i-1}-\id{size}_{i-1}| \\
	&= (\id{num}_i+1)+|2\cdot\id{num}_i-2\cdot\id{num}_i-2|-|2\cdot\id{num}_i+2-3\cdot\id{num}_i-3| \\
	&= (\id{num}_i+1)+|{-}2|-|{-}\id{num}_i-1| \\
	&= 2.
\end{align*}
