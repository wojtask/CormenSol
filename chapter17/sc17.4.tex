\subchapter{Tablice dynamiczne}

\exercise %17.4-1
\exercise %17.4-2
Gdy $\alpha_{i-1}\ge1/2$, to jedynymi sytuacjami, kiedy współczynnik zapełnienia zostaje obniżony poniżej $1/4$, jest wykonanie operacji \proc{Table-Delete} na tablicy dynamicznej, w~której $\id{size}_{i-1}\le2$ oraz $\id{num}_{i-1}=1$.
Oczywiście zamortyzowany koszt operacji usuwania w~takich przypadkach jest ograniczony przez stałą.

Przy założeniu, że $\id{size}_{i-1}>2$, faktyczny koszt operacji wynosi $c_i=1$.
Jeśli $\alpha_i\ge1/2$, to
\begin{align*}
	\widehat{c_i} &= c_i+\Phi_i-\Phi_{i-1} \\
	&= 1+(2\cdot\id{num}_i-\id{size}_i)-(2\cdot\id{num}_{i-1}-\id{size}_{i-1}) \\
	&= 1+2\cdot\id{num}_i-\id{size}_i-2\cdot\id{num}_i-2+\id{size}_i \\
	&= -1.
\end{align*}
Jeśli natomiast $\alpha_i<1/2$, to
\begin{align*}
	\widehat{c_i} &= c_i+\Phi_i-\Phi_{i-1} \\
	&= 1+(\id{size}_i/2-\id{num}_i)-(2\cdot\id{num}_{i-1}-\id{size}_{i-1}) \\
	&= 1+\id{size}_i/2-\id{num}_i-2\cdot\id{num}_i-2+\id{size}_i \\
	&= (3/2)\id{size}_i-3\cdot\id{num}_i-1 \\[1mm]
	&= (3/2)\id{size}_i-3\alpha_i\id{size}_i-1 \\[1mm]
	&< (3/2)\id{size}_i-(3/2)\id{size}_i-1 \\
	&= -1.
\end{align*}
A~zatem we wszystkich przypadkach $\widehat{c_i}$ jest ograniczone od góry przez stałą.

\exercise %17.4-3
W~naszej analizie skorzystamy z~oznaczeń identycznych do tych z~Podręcznika.
Potrzebna nam będzie następująca obserwacja:

\medskip
\noindent\textsf{\textbf{Lemat.}} \textit{Dla dowolnych nieujemnych liczb rzeczywistych\/ $x$, $y$ zachodzi
\[
	|x-y|\le x+y.
\]}
\begin{proof}
Jeśli $x\ge y$, to $|x-y|=x-y\le x+y$, a~jeśli $x<y$, to $|x-y|=-x+y\le x+y$.
\end{proof}

Jeśli wywołanie operacji \proc{Delete} nie powoduje zmniejszenia rozmiaru tablicy, to faktyczny koszt operacji wynosi $c_i=1$.
Wówczas $\id{num}_i=\id{num}_{i-1}-1$, $\id{size}_i=\id{size}_{i-1}$ i~na podstawie powyższego lematu mamy
\begin{align*}
	\widehat{c_i} &= c_i+\Phi_i-\Phi_{i-1} \\
	&= 1+|2\cdot\id{num}_i-\id{size}_i|-|2\cdot\id{num}_{i-1}-\id{size}_{i-1}| \\
	&= 1+|2\cdot\id{num}_i-\id{size}_i|-|2\cdot\id{num}_i-2-\id{size}_i| \\
	&\le 1+(2\cdot\id{num}_i+\id{size}_i)-(2\cdot\id{num}_i+2+\id{size}_i) \\
	&= -1.
\end{align*}
W~sytuacji, w~której tablica zostaje zmniejszona faktyczny koszt wynosi $c_i=\id{num}_i+1$, a~także zachodzą równości $\id{size}_i/2=\id{size}_{i-1}/3=\id{num}_{i-1}=\id{num}_i+1$.
Koszt zamortyzowany w~tym przypadku wynosi więc
\begin{align*}
	\widehat{c_i} &= c_i+\Phi_i-\Phi_{i-1} \\
	&= (\id{num}_i+1)+|2\cdot\id{num}_i-\id{size}_i|-|2\cdot\id{num}_{i-1}-\id{size}_{i-1}| \\
	&= (\id{num}_i+1)+|2\cdot\id{num}_i-2\cdot\id{num}_i-2|-|2\cdot\id{num}_i+2-3\cdot\id{num}_i-3| \\
	&= (\id{num}_i+1)+|{-}2|-|{-}\id{num}_i-1| \\
	&= 2.
\end{align*}
