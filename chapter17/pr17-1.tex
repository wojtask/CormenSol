\problem{Licznik binarny z~odwróconymi bitami} %17-1

\subproblem %17-1(a)
Poniższa procedura oblicza kolejne wartości funkcji $\mathrm{rev}_k$ i~zamienia miejscami elementy wejściowej tablicy, o~ile nie zostały one zamienione już wcześniej.
\begin{codebox}
\Procname{$\proc{Bit-Reversal}(A)$}
\li	$n\gets\attrib{A}{length}$
\li	$k\gets\lg n$
\li	\For $i\gets1$ \To $n-2$
\li		\Do $j\gets\mathrm{rev}_k(i)$
\li			\If $i<j$
\li				\Then zamień $A[i]\leftrightarrow A[j]$
				\End
		\End
\end{codebox}
Ponieważ $\mathrm{rev}_k(0)=0$ oraz $\mathrm{rev}_k(2^k-1)=2^k-1$, to elementy $A[0]$ oraz $A[n-1]$ pozostaną na swoich miejscach, dlatego pętla \kw{for} iteruje od $i=1$ do $i=n-2$.
Nietrudno zauważyć, że czas działania tej procedury ograniczony jest przez $O(nk)$.

\subproblem %17-1(b)
Kolejna wartość licznika powstaje w~wyniku zinkrementowania bieżącej wartości, ale na odwróconych bitach, czyli przy potraktowaniu skrajnie lewego bitu jako najmniej znaczącego, a~skrajnie prawego -- jako najbardziej znaczącego.
Zaimplementujemy procedurę \proc{Increment} z~podrozdziału 17.1 przy pomocy operacji bitowych, a~następnie zmodyfikujemy ją, aby działała dla odwróconej kolejności bitów.

Pętla \kw{while} w~tej procedurze wyszukuje w~liczniku zero umiejscowione najbardziej na prawo oraz zeruje wszystkie mniej znaczące jedynki.
Można zasymulować działanie tej pętli przez utrzymywanie tzw.\ \textbf{maski bitowej} $m$ -- wartości początkowo równej 1, która, przesuwana bitowo w~lewo w~kolejnych iteracjach, posłuży do szybkiego odczytu danego bitu licznika i~jego modyfikacji.
W~\singledash{$i$}{tej} iteracji, gdzie $i=0$, 1, \dots, $k-1$, maska będzie mieć wartość $m=2^i$ otrzymaną przez przesunięcie jedynki w~lewo o~$i$.
Aby pobrać \singledash{$i$}{ty} najmniej znaczący bit licznika, należy wykonać operację bitową AND między licznikiem i~aktualną maską.
Podobnie, aby wyzerować ten bit, należy wykonać bitowe XOR.
Po odnalezieniu szukanego zera, jego zamiana na jedynkę realizowana jest za pomocą bitowego OR.
Przystosowanie opisanych kroków do działania na odwróconych bitach wymaga zastosowania maski początkowo równej $2^{k-1}$, a~następnie przesuwania jej bitowo w~prawo.

W~pseudokodzie poniżej \func{SHL} oznacza operację przesunięcia bitowego w~lewo, a~\func{SHR} -- operację przesunięcia bitowego w~prawo.
\begin{codebox}
\Procname{$\proc{Bit-Reversed-Increment}(a,k)$}
\li	$m\gets1\func{SHL}{}(k-1)$
\li	\While $a\func{AND}m\ne0$
\li		\Do $a\gets a\func{XOR}m$
\li			$m\gets m\func{SHR}1$
		\End
\li	\Return $a\func{OR}m$
\end{codebox}
W~celu wykonania permutacji odwracającej bity utrzymujemy dwa \singledash{$k$}{bitowe} liczniki binarne -- zwykły oraz z~odwróconymi bitami -- i~zamieniamy elementy w~tablicy $A$ znajdujące się na indeksach równych tym licznikom, upewniając się, że zamiany te nie powtarzają się.
\begin{codebox}
\Procname{$\proc{Bit-Reversal}'(A)$}
\li	$n\gets\attrib{A}{length}$
\li	$k\gets\lg n$
\li	$j\gets0$
\li	\For $i\gets1$ \To $n-2$
\li		\Do $j\gets\proc{Bit-Reversed-Increment}(j,k)$
\li			\If $i<j$
\li				\Then zamień $A[i]\leftrightarrow A[j]$
				\End
		\End
\end{codebox}

Procedura \proc{Bit-Reversed-Increment} działa w~tym samym czasie co \proc{Increment}, na której jest oparta, zatem $n$ kolejnych jej wywołań wymaga czasu $O(n)$.
Jest to też czas, jaki potrzebuje procedura $\proc{Bit-Reversal}'$ wywołana na tablicy $A[0\twodots n-1]$.

\subproblem %17-1(c)
Przy takim założeniu inicjalizacja maski w~procedurze \proc{Bit-Reversed-Increment} wymaga czasu $\Theta(k)$.
Sprawia to, że $n$ kolejnych wywołań tej procedury zajmuje czas $O(nk)$.
Bez przygotowania maski nie jesteśmy jednak w~stanie przeglądać bitów w~kolejności od lewej do prawej i~wykonanie permutacji odwracającej bity w~czasie $O(n)$ jest w~tym przypadku niemożliwe.
