\problem{Licznik binarny z~odwróconymi bitami} %17-1

\subproblem %17-1(a)
Poniższa procedura oblicza kolejne wartości funkcji $\mathrm{rev}_k$ i~zamienia miejscami elementy wejściowej tablicy, o~ile nie zostały one zamienione już wcześniej.
\begin{codebox}
\Procname{$\proc{Bit-Reversal-Permutation}(A)$}
\li	$n\gets\attrib{A}{length}$
\li	$k\gets\lg n$
\li	\For $i\gets1$ \To $n-2$
\li		\Do $j\gets\mathrm{rev}_k(i)$
\li			\If $i<j$
\li				\Then zamień $A[i]\leftrightarrow A[j]$
				\End
		\End
\end{codebox}
Ponieważ $\mathrm{rev}_k(0)=0$ oraz $\mathrm{rev}_k(2^k-1)=2^k-1$, to elementy $A[0]$ oraz $A[n-1]$ pozostaną na swoich miejscach, dlatego pętla \kw{for} iteruje od $i=1$ do $i=n-2$.
Nietrudno zauważyć, że czas działania tej procedury ograniczony jest przez $O(nk)$.

\subproblem %17-1(b)
Kolejna wartość licznika powstaje w~wyniku zinkrementowania bieżącej wartości, ale na odwróconych bitach, czyli przy potraktowaniu skrajnie lewego bitu jako najmniej znaczącego, a~skrajnie prawego -- jako najbardziej znaczącego.
Opiszemy implementację procedury \proc{Increment} z~podrozdziału 17.1 przy pomocy operacji bitowych, a~następnie dostosujemy ją do działania na liczniku z~odwróconymi bitami.

W~pętli \kw{while} procedury \proc{Increment} wszystkie mniej znaczące jedynki od najmniej znaczącego zera są zamieniane na zera.
Można zasymulować te czynności przez utrzymywanie tzw.\ \textbf{maski bitowej} służącej do szybkiego odczytu danego bitu licznika i~jego modyfikacji.
W~$i$\nbhyphen tej iteracji, gdzie $i=0$, 1, \dots, $k-1$, maska będzie mieć wartość $2^i$ otrzymaną przez wykonywanie operacji przesunięcia bitowego w~lewo na początkowej wartości $2^0=1$.
Aby odczytać $i$\nbhyphen ty najmniej znaczący bit licznika, należy wykonać operację bitową AND na liczniku i~aktualnej masce.
Podobnie, aby wyzerować ten bit, należy wykonać bitowe XOR\@.
Zamiana znalezionego zera na jedynkę realizowana jest za pomocą bitowego OR\@.
Przystosowanie opisanych kroków do działania na odwrotnej kolejności bitów wymaga zastosowania maski zainicjalizowanej na $2^{k-1}$ i~przesuwania jej bitowo w~prawo.

W~pseudokodzie poniżej \func{SHL} oznacza operację przesunięcia bitowego w~lewo, a~\func{SHR} -- operację przesunięcia bitowego w~prawo.
\begin{codebox}
\Procname{$\proc{Bit-Reversed-Increment}(a,k)$}
\li	$m\gets1\func{SHL}{}(k-1)$
\li	\While $a\func{AND}m\ne0$
\li		\Do $a\gets a\func{XOR}m$
\li			$m\gets m\func{SHR}1$
		\End
\li	\Return $a\func{OR}m$
\end{codebox}
W~celu wyznaczenia permutacji odwracającej bity utrzymujemy dwa $k$\nbhyphen bitowe liczniki binarne -- zwykły oraz z~odwróconymi bitami -- i~zamieniamy elementy w~tablicy $A$ znajdujące się na indeksach równych tym licznikom, upewniając się, że zamiany te nie powtarzają się.
\begin{codebox}
\Procname{$\proc{Bit-Reversal-Permutation}'(A)$}
\li	$n\gets\attrib{A}{length}$
\li	$k\gets\lg n$
\li	$j\gets0$
\li	\For $i\gets1$ \To $n-2$
\li		\Do $j\gets\proc{Bit-Reversed-Increment}(j,k)$
\li			\If $i<j$
\li				\Then zamień $A[i]\leftrightarrow A[j]$
				\End
		\End
\end{codebox}

Procedura \proc{Bit-Reversed-Increment} działa w~tym samym czasie co \proc{Increment}, na której jest oparta, zatem $n$ kolejnych jej wywołań wymaga czasu $O(n)$.
Jest to też czas, jaki potrzebuje procedura $\proc{Bit-Reversal-Permutation}'$ wywołana na tablicy $A[0\twodots n-1]$.

\subproblem %17-1(c)
Przy takim założeniu inicjalizacja maski w~procedurze \proc{Bit-Reversed-Increment} wymaga czasu $\Theta(k)$.
Sprawia to, że $n$ kolejnych wywołań tej procedury zajmuje czas $O(nk)$.
Bez przygotowania maski nie jesteśmy jednak w~stanie przeglądać bitów w~kolejności od lewej do prawej i~wyznaczenie permutacji odwracającej bity w~czasie $O(n)$ jest w~tym przypadku niemożliwe.
