\problem{Koszt przebudowy drzew czerwono-czarnych} %17-4

\subproblem %17-4(a)
Rozważmy drzewa czerwono-czarne opisane w~rozwiązaniu \refExercise{13.1-7}.
Gdy do pierwszego z~nich wstawiony zostanie nowy węzeł, to w~wywołaniu procedury \proc{RB-Insert-Fixup} przypadek 1 wystąpi $\Omega(\lg n)$ razy, a~w~każdym z~nich kolor zmienią dokładnie 3 węzły.

Drugie z~tych drzew potraktujemy jako przykład dla procedury \proc{RB-Delete}.
Podczas usuwania najniższego węzła wewnętrznego nadmiarowa ,,czarna jednostka'' jest przesuwana w~górę drzewa aż do jego korzenia, a~na każdym poziomie swój kolor zmienia dokładnie jeden węzeł (przypadek 2 z~procedury \proc{RB-Delete-Fixup}).
Łącznie wykonanych zostanie więc $\Omega(\lg n)$ takich zmian.

\subproblem %17-4(b)
W~procedurze \proc{RB-Insert-Fixup} kończące są przypadki 2 i~3, a~w~procedurze \proc{RB-Delete-Fixup} -- przypadki 1, 3 i~4.

\subproblem %17-4(c)
W~przypadku 1 procedury \proc{RB-Insert-Fixup} ojciec i~stryj węzła $z$ zmieniają kolor z~czerwonego na czarny, natomiast dziadek węzła $z$ -- z~czarnego na czerwony.
Liczba czerwonych węzłów w~drzewie zmniejsza się zatem o~1, czyli $\Phi(T')=\Phi(T)-1$.

\subproblem %17-4(d)
W~liniach 1\nbendash16 procedury \proc{RB-Insert} następuje umieszczenie nowego węzła w~drzewie i~pokolorowanie go na czerwono -- rośnie więc o~1 potencjał drzewa.
Niekończący przypadek 1 procedury \proc{RB-Insert-Fixup} wykonuje 3 zmiany koloru, zmniejszając przy tym potencjał o~1.
W~każdym z~przypadków kończących (2 i~3) wykonywana jest jedna rotacja, a~potencjał drzewa nie zmienia się.
Ponadto w~przypadku 3 kolor zmieniają 2 węzły.

\subproblem %17-4(e)
Zarówno liczba wykonywanych modyfikacji strukturalnych, jak i~różnica potencjału odpowiadająca liniom 1\nbendash16 procedury \proc{RB-Insert} oraz kończącym przypadkom \proc{RB-Insert-Fixup}, są rzędu $O(1)$.
Niekończący przypadek \proc{RB-Insert-Fixup} może powtórzyć się w~jednym wywołaniu $O(\lg n)$ razy, ale zamortyzowana liczba modyfikacji strukturalnych wynosi 0, gdyż obniżeniem potencjału drzewa jesteśmy w~stanie zapłacić za wykonywane wtedy modyfikacje strukturalne.
W~związku z~tym zamortyzowana liczba modyfikacji strukturalnych przeprowadzanych w~wywołaniu \proc{RB-Insert} wynosi $O(1)$.

\subproblem %17-4(f)
W~przypadku 1 procedury \proc{RB-Insert-Fixup} zachodzą następujące modyfikacje kolorów:
\begin{itemize}
	\item czarny węzeł o~dwóch czerwonych synach staje się czerwony -- potencjał maleje o~2;
	\item czerwony węzeł o~jednym czerwonym synu staje się czarny -- potencjał nie zmienia się;
	\item czerwony węzeł nie mający czerwonych synów staje się czarny -- potencjał rośnie o~1.
\end{itemize}
Zatem w~przypadku niekończącym tej procedury potencjał maleje łącznie o~1.
Zgodnie z~poczynionym założeniem, ta jednostka potencjału pozwala opłacić wykonane w~tym przypadku modyfikacje strukturalne, dlatego zamortyzowaną ich liczbą jest tu 0.
W~przypadkach kończących rzeczywista liczba modyfikacji strukturalnych jest rzędu $O(1)$, skąd zamortyzowana liczba modyfikacji strukturalnych wykonywanych w~wywołaniu \proc{RB-Insert-Fixup} wynosi $O(1)$.

\subproblem %17-4(g)
Przypadek 2 procedury \proc{RB-Delete-Fixup} prowadzi do następujących zmian w~kolorach:
\begin{itemize}
	\item czarny węzeł nie mający czerwonych synów staje się czerwony -- potencjał maleje o~1;
	\item inny węzeł nie mający czerwonych synów zyskuje jednego czerwonego syna -- jeśli węzeł ten jest czerwony, to potencjał nie zmienia się, a~jeśli jest czarny, to potencjał maleje o~1.
\end{itemize}
Łącznie potencjał maleje więc o~1 lub o~2.
Jedna jednostka potencjału wystarcza, by zapłacić za wykonane modyfikacje strukturalne, a~więc w~przypadku 2 zamortyzowaną ich liczbą jest co najwyżej 0.
Przypadki kończące prowadzą do $O(1)$ modyfikacji strukturalnych, dlatego całkowitą ich zamortyzowaną liczbą w~procedurze \proc{RB-Delete-Fixup} jest $O(1)$.

\subproblem %17-4(h)
Ponieważ zamortyzowane liczby modyfikacji strukturalnych w~obu operacjach są rzędu $O(1)$, to ich rzeczywista liczba w~dowolnym ciągu $m$ operacji \proc{RB-Insert} i~\proc{RB-Delete} wykonanych na początkowo pustym drzewie czerwono-czarnym, jest rzędu $O(m)$ w~najgorszym przypadku.
