\subchapter{Metoda potencjału}

\exercise %17.3-1
By zachować takie same koszty zamortyzowane operacji, musimy zapewnić, że przyrosty potencjału względem funkcji $\Phi'$ są identyczne jak przyrosty potencjału względem $\Phi$.
A~zatem dla każdego $i=1$, 2, \dots, $n$ musi zachodzić
\[
	\Phi'(D_i)-\Phi'(D_{i-1}) = \Phi(D_i)-\Phi(D_{i-1}).
\]
Mamy $\Phi'(D_1)-\Phi'(D_0)=\Phi(D_1)-\Phi(D_0)$, skąd $\Phi'(D_1)=\Phi(D_1)-\Phi(D_0)$.
Następnie mamy $\Phi'(D_2)=\Phi(D_2)-\Phi(D_1)+\Phi'(D_1)$ i~wstawiając uprzednio wyznaczone $\Phi'(D_1)$, dostajemy $\Phi'(D_2)=\Phi(D_2)-\Phi(D_0)$.
Postępując w~ten sposób dla kolejnych $i$, skonstruujemy funkcję $\Phi'(D_i)=\Phi(D_i)-\Phi(D_0)$ dla każdego $i=1$, 2, \dots, $n$.

\exercise %17.3-2
W~naszej analizie skorzystamy z~następującej funkcji potencjału:
\[
	\Phi(D_i) = \begin{cases}
		0, & \text{jeśli $i=0$}, \\
		k+2, & \text{jeśli $i=2^k$, $k=1$, 2, \dots, $\lfloor\lg n\rfloor$}, \\
		\Phi(D_{i-1})+2, & \text{w~przeciwnym przypadku}.
	\end{cases}
\]
Gdy $i=1$ lub gdy $i$ nie jest potęgą 2, to $\widehat{c_i}=c_i+\Phi(D_i)-\Phi(D_{i-1})=1+2=3$.
Przyjmijmy teraz, że $i=2^k$ dla pewnego $k=1$, 2, \dots, $\lfloor\lg n\rfloor$.
Mamy
\begin{align*}
	\widehat{c_{2^k}} &= c_{2^k}+\Phi(D_{2^k})-\Phi(D_{2^k-1}) \\
	&= 2^k+k+2-\bigl(\Phi(D_{2^{k-1}})+2(2^k-1-2^{k-1})\bigr) \\
	&= 2^k+k+2-(k+1+2^k-2) \\
	&= 3.
\end{align*}
Dzięki wybraniu szczególnej funkcji $\Phi$, jako koszt zamortyzowany każdej operacji mogliśmy przyjąć 3, podobnie jak w~rozwiązaniu \refExercise{17.2-2}.
Oczywiście $\Phi(D_i)\ge\Phi(D_0)$ dla każdego $i\ge1$, zatem łączny koszt zamortyzowany $n$ operacji stanowi górne ograniczenie ich faktycznego kosztu.
Stąd pesymistyczny koszt ciągu $n$ operacji wynosi $O(n)$.

\exercise %17.3-3
\exercise %17.3-4
Jeśli skorzystamy z~funkcji $\Phi$ przyjmującej wysokość stosu, to $\Phi(D_0)=s_0$ i~$\Phi(D_n)=s_n$.
Zgodnie z~równaniem (17.3) mamy więc, że łączny koszt zamortyzowany ciągu $n$ operacji na stosie wynosi \[
	\sum_{i=1}^n\widehat{c_i} = \sum_{i=1}^nc_i+\Phi(D_n)-\Phi(D_0) = O(n)+s_n-s_0.
\]
Jeśli $s_n\ge s_0$, to wyrażenie powyższe stanowi górne ograniczenie łącznego kosztu faktycznego.

\exercise %17.3-5
Jeśli $n=\Omega(b)$, to $b=O(n)$.
Zgodnie z~wynikiem przedstawionym w~Podręczniku, łączny koszt $n$ operacji \proc{Increment} można ograniczyć od góry przez $2n+b=O(n)$.

\exercise %17.3-6
\exercise %17.3-7
