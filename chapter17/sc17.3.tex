\subchapter{Metoda potencjału}

\exercise %17.3-1
By zachować takie same koszty zamortyzowane operacji, musimy zapewnić, że przyrosty potencjału względem funkcji $\Phi'$ są identyczne jak przyrosty potencjału względem $\Phi$.
A~zatem dla każdego $i=1$, 2, \dots, $n$ musi zachodzić
\[
	\Phi'(D_i)-\Phi'(D_{i-1}) = \Phi(D_i)-\Phi(D_{i-1}).
\]
Mamy $\Phi'(D_1)-\Phi'(D_0)=\Phi(D_1)-\Phi(D_0)$, skąd $\Phi'(D_1)=\Phi(D_1)-\Phi(D_0)$.
Następnie mamy $\Phi'(D_2)=\Phi(D_2)-\Phi(D_1)+\Phi'(D_1)$ i~wstawiając uprzednio wyznaczone $\Phi'(D_1)$, dostajemy $\Phi'(D_2)=\Phi(D_2)-\Phi(D_0)$.
Postępując w~ten sposób dla kolejnych $i$, skonstruujemy funkcję $\Phi'(D_i)=\Phi(D_i)-\Phi(D_0)$ dla każdego $i=1$, 2, \dots, $n$.

\exercise %17.3-2
\exercise %17.3-3
\exercise %17.3-4
\exercise %17.3-5
\exercise %17.3-6
\exercise %17.3-7
