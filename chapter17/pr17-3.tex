\problem{Zamortyzowane drzewa zrównoważone} %17-3

\subproblem %17-3(a)
Węzeł $x$ jest $1/2$\nbhyphen zrównoważony, gdy klucz w~tym węźle stanowi medianę zbioru kluczy znajdujących się w~poddrzewie o~korzeniu w~$x$.
W~lewym poddrzewie $x$ znajdują się wtedy klucze na lewo od mediany, a~w~prawym poddrzewie -- klucze na prawo od mediany.
Aby przebudować całe poddrzewo o~korzeniu w~$x$, należy wpierw rekursywnie przebudować lewe i~prawe poddrzewo węzła $x$.
Przechodząc po poddrzewie o~korzeniu w~$x$ w~porządku inorder, możemy wszystkie jego klucze zebrać w~uporządkowaną tablicę rozmiaru $\attrib{x}{size}$.
Kolejne mediany można wyznaczać wtedy w~czasie $O(1)$ każda i~zbudowanie $1/2$\nbhyphen zrównoważonego poddrzewa o~korzeniu w~$x$ zajmuje czas $\Theta(\attrib{x}{size})$.

\subproblem %17-3(b)
Czas wykonania operacji \proc{Search} w~drzewie wyszukiwań binarnych w~najgorszym przypadku jest rzędu $O(h)$, gdzie $h$ jest wysokością drzewa.
Pokażemy, że w~drzewie $\alpha$\nbhyphen zrównoważonym $h=O(\lg n)$.
Dla dowolnego węzła $x$ w~tym drzewie rozmiar któregokolwiek poddrzewa $x$ (lewego lub prawego) jest równy co najwyżej $\alpha\cdot\attrib{x}{size}$.
Ogólnie, jeśli $y$ jest potomkiem $x$ znajdującym się $d$ poziomów głębiej od $x$, to rozmiar poddrzewa o~korzeniu w~$y$ wynosi co najwyżej $\alpha^d\cdot\attrib{x}{size}$.
A~więc gdy $x$ stanowi korzeń drzewa, to $\attrib{x}{size}=n$ i~$h$ poziomów niżej od $x$ znajdują się poddrzewa składające się z~co najwyżej jednego węzła.
Zatem $\alpha^hn\le 1$, skąd
\[
	h\le\log_\alpha(1/n) = \log_{1/\alpha}n = \frac{\lg n}{\lg(1/\alpha)} = O(\lg n).
\]

\subproblem %17-3(c)
Funkcja potencjału dowolnego drzewa $T$ jest oczywiście nieujemna jako suma wartości bezwzględnych.
Jeśli drzewo $T$ jest $1/2$\nbhyphen zrównoważone, to dla każdego $x\in T$, $\Delta(x)\le1$.
Wówczas suma w~definicji funkcji potencjału składa się z~zerowej liczby składników, a~zatem jest równa 0.

\subproblem %17-3(d)
Niech $T$ będzie drzewem wyszukiwań binarnych o~$m$ węzłach, dla którego własność $\alpha$\nbhyphen zrównoważenia jest naruszona w~korzeniu $r=\attrib{T}{root}$ i~bez utraty ogólności przyjmijmy, że to lewe poddrzewo ma zbyt duży rozmiar:
\[
	\attribb{r}{left}{size} > \alpha\cdot\attrib{r}{size} = \alpha m.
\]
Wówczas rozmiar prawego poddrzewa wynosi
\[
	\attribb{r}{right}{size} = \attrib{r}{size}-\attribb{r}{left}{size}-1 < m-\alpha m-1 < (1-\alpha)m.
\]
Stąd
\[
	\Delta(r) = \attribb{r}{left}{size}-\attribb{r}{right}{size} > \alpha m-(1-\alpha)m = m(2\alpha-1).
\]
Z~założenia, że $\alpha>1/2$ wynika, że lewe poddrzewo ma rozmiar większy o~co najmniej 2 od rozmiaru prawego poddrzewa.
Zatem $\Delta(r)\ge2$ i~$c\cdot\Delta(r)$ jest składnikiem potencjału drzewa $T$, skąd $\Phi(T)\ge c\cdot\Delta(r)>cm(2\alpha-1)$.
By za pomocą takiego potencjału przebudować drzewo $T$, musi być spełniony warunek $cm(2\alpha-1)\ge m$, skąd otrzymujemy $c\ge\frac{1}{2\alpha-1}$.

\subproblem %17-3(e)
By wstawić do drzewa nowy węzeł $z$, należy wyznaczyć odpowiedni liść tego drzewa, który stanie się ojcem $z$.
Z~punktu (b) wiadomo, że wyszukiwanie w~$\alpha$\nbhyphen zrównoważonym drzewie zabiera czas $O(\lg n)$.
W~wyniku działania operacji wstawiania zmianie ulegają wartości $\Delta(x)$ dla wszystkich przodków $x$ węzła $z$, co może doprowadzić do stanu, w~którym drzewo nie jest już $\alpha$\nbhyphen zrównoważone.
Identyczna sytuacja jest skutkiem wykonania operacji usuwania, gdzie $z$ jest faktycznie usuniętym węzłem.

$\alpha$\nbhyphen zrównoważenie drzewa można przywrócić, poruszając się po ścieżce od $z$ do korzenia i~przebudowując napotykane poddrzewa, o~ile nie są $\alpha$\nbhyphen zrównoważone.
Z~poprzedniego punktu wiadomo, że zamortyzowany koszt każdej takiej operacji wynosi $O(1)$.
Poddrzew wymagających przebudowy może być co najwyżej $O(\lg n)$, a~więc całkowity zamortyzowany koszt wstawiania lub usuwania wynosi $O(\lg n)$.
