\problem{Zamortyzowane drzewa zrównoważone} %17-3

\subproblem %17-3(a)
Węzeł $x$ jest \singledash{$1/2$}{zrównoważony}, gdy klucz w~tym węźle stanowi medianę zbioru kluczy znajdujących się w~poddrzewie o~korzeniu w~$x$.
W~lewym poddrzewie $x$ znajdują się wtedy klucze na lewo od mediany, a~w~prawym poddrzewie -- klucze na prawo od mediany.
Aby zrównoważyć całe poddrzewo o~korzeniu w~$x$, należy rekursywnie \singledash{$1/2$}{zrównoważyć} lewe i~prawe poddrzewo węzła $x$.
Dzięki wykorzystaniu dodatkowej pamięci rzędu $O(\attrib{x}{size})$ można posortować klucze z~poddrzewa o~korzeniu w~$x$, przechodząc je w~porządku inorder.
Kolejne mediany można wyznaczać wtedy w~czasie $O(1)$ każda i~zbudowanie \singledash{$1/2$}{zrównoważonego} poddrzewa o~korzeniu w~$x$ zajmuje czas $\Theta(\attrib{x}{size})$.

\subproblem %17-3(b)
Czas wykonania operacji \proc{Search} w~drzewie wyszukiwań binarnych w~najgorszym przypadku jest rzędu $O(h)$, gdzie $h$ jest wysokością drzewa.
Pokażemy, że w~drzewie \singledash{$\alpha$}{zrównoważonym} $h=O(\lg n)$.
Dla dowolnego węzła $x$ w~tym drzewie rozmiar któregokolwiek poddrzewa $x$ jest równy co najwyżej $\alpha\cdot\attrib{x}{size}$.
Rozmiar pewnego poddrzewa $x$ znajdującego się $d$ poziomów niżej od $x$ wynosi zatem co najwyżej $\alpha^d\cdot\attrib{x}{size}$.
Jeśli więc jako $x$ przyjmiemy korzeń drzewa, to $\attrib{x}{size}=n$ i~$h$ poziomów niżej od $x$ znajdują się poddrzewa składające się z~co najwyżej jednego węzła.
Zatem $\alpha^hn\le 1$, skąd
\[
	h\le\log_\alpha(1/n) = \log_{1/\alpha}n = \frac{\lg n}{\lg(1/\alpha)} = O(\lg n).
\]

\subproblem %17-3(c)
Funkcja potencjału dowolnego drzewa $T$ jest oczywiście nieujemna jako suma wartości bezwzględnych.
Jeśli drzewo $T$ jest \singledash{$1/2$}{zrównoważone}, to dla każdego jego węzła $x$,
\[
	\Delta(x) = \bigl|\attribb{x}{left}{size}-\attribb{x}{right}{size}\bigr| = 1.
\]
Wówczas suma w~definicji funkcji potencjału składa się z~zerowej liczby składników, a~zatem jest równa 0.

\subproblem %17-3(d)
Niech $T$ będzie drzewem wyszukiwań binarnych o~$m$ węzłach, dla którego własność \singledash{$\alpha$}{zrównoważenia} jest naruszona w~korzeniu $r=\attrib{T}{root}$ i~bez utraty ogólności przyjmijmy, że to lewe poddrzewo ma zbyt duży rozmiar:
\[
	\attribb{r}{left}{size} > \alpha\cdot\attrib{r}{size} = \alpha m.
\]
Wówczas rozmiar prawego poddrzewa wynosi
\[
	\attribb{r}{right}{size} = \attrib{r}{size}-\attribb{r}{left}{size}-1 < m-\alpha m-1 < (1-\alpha)m.
\]
Stąd
\[
	\Delta(r) = \attribb{r}{left}{size}-\attribb{r}{right}{size} > \alpha m-(1-\alpha)m = (2\alpha-1)m.
\]
Z~założenia, że $\alpha>1/2$ wynika, że lewe poddrzewo ma rozmiar większy o~co najmniej 2 od rozmiaru prawego poddrzewa.
Zatem $\Delta(r)\ge2$ i~$c\cdot\Delta(r)$ jest składnikiem potencjału drzewa $T$, skąd $\Phi(T)\ge c\cdot\Delta(r)>cm(2\alpha-1)$.
By za pomocą takiego potencjału przebudować drzewo $T$, musi być spełniony warunek $cm(2\alpha-1)\ge m$, skąd otrzymujemy $c\ge\frac{1}{2\alpha-1}$.

\subproblem %17-3(e)
W~celu wstawienia do drzewa nowego węzła $z$ należy wyszukać odpowiedni liść tego drzewa, który stanie się ojcem $z$.
Z~punktu (b) wiadomo, że wyszukiwanie w~\singledash{$\alpha$}{zrównoważonym} drzewie wyszukiwań binarnych zabiera czas $O(\lg n)$.
W~wyniku działania operacji wstawiania zmianie ulegają wartości $\Delta(x)$ dla wszystkich przodków $x$ węzła $z$, co może doprowadzić do stanu, w~którym drzewo nie jest już \singledash{$\alpha$}{zrównoważone}.
Identyczna sytuacja jest skutkiem wykonania operacji usuwania, gdzie $z$ jest faktycznie usuniętym węzłem.
Zrównoważenie drzewa można przywrócić, równoważąc kolejne niezrównoważone węzły na ścieżce od $z$ aż do korzenia drzewa.
Z~poprzedniego punktu wiadomo, że zamortyzowany koszt każdej takiej operacji wynosi $O(1)$.
Ponieważ niezrównoważonych węzłów jest co najwyżej $O(\lg n)$, to całkowity zamortyzowany koszt wstawienia lub usunięcia elementu wynosi $O(\lg n)$.
