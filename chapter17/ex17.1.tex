\subchapter{Metoda kosztu sumarycznego}

\exercise %17.1-1
Przyjmijmy, że operacja \proc{Multipush} może jednocześnie umieścić na stosie co najwyżej $k$ elementów.
Jeśli $k$ jest stałą, to z~punktu widzenia analizy kosztu zamortyzowanego operacja ta nie różni się od działającej także w~czasie $O(1)$ zwykłej operacji \proc{Push}.
Gdy jednak $k=\omega(1)$, to nie jesteśmy w~stanie osiągnąć zamortyzowanego kosztu $O(1)$.
Ciąg $n$ operacji mógłby się bowiem składać z~występujących na przemian \proc{Multipush} umieszczającej na stosie $k$ elementów i~\proc{Multipop}, która wszystkie te elementy usuwa ze stosu.
Koszt takiego ciągu operacji wynosi $\Theta(nk)$ i~oszacowaniem na zamortyzowany koszt operacji jest $\Theta(nk)/n=\Theta(k)=\omega(1)$.

\exercise %17.1-2
Rozważmy ciąg $n$ operacji wykonywanych na początkowo wyzerowanym liczniku, gdzie pierwszą operacją jest \proc{Decrement}, a~następnie naprzemiennie występują \proc{Increment} oraz \proc{Decrement}.
Zauważmy, że każda operacja w~tym ciągu odwraca wszystkie bity licznika, działając w~czasie $\Theta(k)$ -- \proc{Decrement} ustawia każdy bit na jedynkę, a~\proc{Increment} na zero.
A~zatem ciąg ten wymaga w~sumie czasu $\Theta(nk)$.

\exercise %17.1-3
Oznaczmy przez $c_i$ koszt $i$\nbhyphen tej operacji.
Mamy $c_i=i$ dla $i=2^k$, gdzie $k=0$, 1, \dots, $\lfloor\lg n\rfloor$.
Koszt pozostałych $n-\lfloor\lg n\rfloor-1$ operacji wynosi 1.
Sumaryczny czas zajmowany przez wszystkie operacje w~tym ciągu jest więc równy
\[
	T(n) = \sum_{i=1}^nc_i = \sum_{k=0}^{\lfloor\lg n\rfloor}2^k+n-\lfloor\lg n\rfloor-1 = 2^{\lfloor\lg n\rfloor+1}+n-\lfloor\lg n\rfloor-1 < 2^{\lg n+1}+n = 2n+n = 3n.
\]
Zatem $T(n)=O(n)$ i~kosztem zamortyzowanym każdej operacji w~tym ciągu jest $T(n)/n=O(1)$.
