\subchapter{Metoda kosztu sumarycznego}

\exercise %17.1-1
Nawet jeśli ograniczymy ilość wkładanych elementów na stos w~operacji \proc{Multipush} przez $n$, to nie jesteśmy w~stanie osiągnąć zamortyzowanego kosztu $O(1)$.
Ciąg $n$ operacji mógłby się bowiem składać z~występujących na przemian \proc{Multipush} umieszczającej na stosie $n$ elementów i~\proc{Multipop}, która wszystkie te elementy usuwa ze stosu.
Koszt takiego ciągu operacji wynosi $O(n^2)$ i~oszacowaniem na zamortyzowany koszt operacji jest $O(n^2)/n=O(n)$.

\exercise %17.1-2
\exercise %17.1-3
