\subchapter{Metoda kosztu sumarycznego}

\exercise %17.1-1
Nawet jeśli ograniczymy ilość wkładanych elementów na stos w~operacji \proc{Multipush} przez $n$, to nie jesteśmy w~stanie osiągnąć zamortyzowanego kosztu $O(1)$.
Ciąg $n$ operacji mógłby się bowiem składać z~występujących na przemian \proc{Multipush} umieszczającej na stosie $n$ elementów i~\proc{Multipop}, która wszystkie te elementy usuwa ze stosu.
Koszt takiego ciągu operacji wynosi $O(n^2)$ i~oszacowaniem na zamortyzowany koszt operacji jest $O(n^2)/n=O(n)$.

\exercise %17.1-2
Rozważmy ciąg $n$ operacji wykonywanych na początkowo wyzerowanym liczniku, gdzie pierwszą operacją jest \proc{Decrement}, a~następnie naprzemiennie występują \proc{Increment} oraz \proc{Decrement}.
Zauważmy, że każda operacja w~tym ciągu odwraca wszystkie bity licznika -- \proc{Decrement} ustawia każdy bit na jedynkę, a~\proc{Increment} na zero -- działając w~czasie $\Theta(k)$.
A~zatem ciąg ten wymaga w~sumie czasu $\Theta(nk)$.

\exercise %17.1-3
Oznaczmy przez $c_i$ koszt \singledash{$i$}{tej} operacji.
Mamy $c_i=i$ dla $i$ postaci $2^k$, gdzie $k=1$, 2, \dots, $\lfloor\lg n\rfloor$.
Koszt pozostałych $n-\lfloor\lg n\rfloor$ operacji wynosi 1.
Sumaryczny czas wymagany przez wszystkie operacje w~tym ciągu jest więc równy
\[
	T(n) = \sum_{i=1}^nc_i = \sum_{k=1}^{\lfloor\lg n\rfloor}2^k+n-\lfloor\lg n\rfloor = 2^{\lfloor\lg n\rfloor+1}-2+n-\lfloor\lg n\rfloor.
\]
Zachodzi $n>\lfloor\lg n\rfloor$, więc powyższe wyrażenie można oszacować przez
\[
	T(n) > 2^{\lg n}-2+n-n = n-2 \quad\text{oraz}\quad T(n) < 2^{\lg n+1}+n = 2n+n = 3n,
\]
skąd $T(n)=\Theta(n)$ i~kosztem zamortyzowanym każdej operacji w~tym ciągu jest $T(n)/n=\Theta(1)$.
