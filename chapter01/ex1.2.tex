\subchapter{Algorytmy jako technologia}

\exercise %1.2-1
Przykładem aplikacji, w~której wykorzystywane są różnorodne algorytmy, jest współczesna gra komputerowa.
Jej silnik grafiki trójwymiarowej jest zaawansowanym środowiskiem, w~którym stosowane są algorytmy geometrii obliczeniowej i~renderowania grafiki 3D, jak również wiele algorytmów numerycznych do wyznaczania interpolacji oraz algorytmy grafowe.
Także w~dziedzinie sztucznej inteligencji opracowano wiele zaawansowanych algorytmów.
Ponadto powszechne problemy, takie jak wyszukiwanie elementu w~tablicy czy sortowanie, są rozwiązywane w~niemal każdej aplikacji.

\exercise %1.2-2
Jeśli $n$ jest liczbą naturalną, to nierówność $8n^2<64n\lg n$ jest spełniona dla $2\le n\le43$.
Tablice o~takich rozmiarach porządkowane są przez sortowanie przez wstawianie szybciej niż przez sortowanie przez scalanie.

\exercise %1.2-3
Najmniejszą dodatnią liczbą całkowitą $n$ spełniającą nierówność $100n^2<2^n$ jest $n=15$.
