\subchapter{Algorytmy}

\exercise %1.1-1
Przykłady występowania poszczególnych problemów:
\begin{description}
	\item[Sortowanie:] Jest to problem bardzo powszechny -- znajduje zastosowanie w~wielu zagadnieniach obliczeniowych.
Najczęstszym powodem sortowania danych jest przygotowanie ich do dalszego przetwarzania, które wówczas jest na ogół bardziej efektywne.
	\item[Najlepsza kolejność mnożenia macierzy:] Problem występuje podczas wyznaczania transformacji graficznych (np.\ skalowań, obrotów); przekształcenia te opisane są za pomocą macierzy, a~ich składanie oznacza obliczanie iloczynu tych macierzy.
	\item[Otoczka wypukła:] Mając zbiór wbitych w~ziemię słupków, chcemy otoczyć pewien obszar siatką ogrodzeniową opierając ją na niektórych słupkach tak, by obszar ten zmaksymalizować.
\end{description}

\exercise %1.1-2
Miary efektywności algorytmu inne niż jego szybkość:
\begin{itemize}
	\item wykorzystanie pamięci (Czy wymaga dużych zasobów pamięciowych?);
	\item wykorzystanie zasobów systemu operacyjnego (Czy otwiera wiele plików, portów sieciowych? Czy tworzy wiele procesów, wątków?);
	\item efektywność dostępu do bazy danych (Czy wykonuje wiele zapytań? Czy są one odpowiednio zoptymalizowane dla konkretnego systemu bazodanowego?);
	\item stopień wykorzystania połączeń sieciowych (Czy potrzebuje przesyłać dużą ilość danych? Z~ilu źródeł? Do ilu miejsc docelowych?);
	\item dostosowanie do konkretnej architektury sprzętowo-programowej (Czy efektywnie wykorzystuje zestaw instrukcji danego procesora? Możliwości danej platformy?);
	\item efektywność działania w~architekturze równoległej lub rozproszonej (Czy efektywnie zrównolegla lub rozprasza wykonywane zadania?).
\end{itemize}

\exercise %1.1-3
Poniżej zestawiono zalety i~wady listy dwukierunkowej w~porównaniu ze zwykłą tablicą.

\medskip
\noindent\textbf{Zalety:}
\begin{itemize}
	\item przy definiowaniu listy nie trzeba z~góry znać jej maksymalnej pojemności, jak to jest w~przypadku definiowania tablicy;
	\item lista jest elastyczna -- może się dowolnie powiększać i~kurczyć w~trakcie wykonywania na niej operacji wstawiania i~usuwania;
	\item wstawianie i~usuwanie elementów z~dowolnej pozycji listy odbywa się w~czasie stałym.
\end{itemize}
\medskip
\noindent\textbf{Wady:}
\begin{itemize}
	\item nie można uzyskać dostępu do dowolnego elementu listy w~stałym czasie;
	\item lista potrzebuje nieco więcej pamięci niż tablica -- oprócz danych pamiętane są wskaźniki na poprzedni i~następny element listy;
	\item w~przeciwieństwie do tablicy lista na ogół nie zajmuje spójnego obszaru pamięci, co może prowadzić do fragmentacji pamięci.
\end{itemize}

\exercise %1.1-4
Obydwa problemy są problemami grafowymi mającymi na celu minimalizację pewnej ścieżki w~grafie.
W~problemie najkrótszej ścieżki poszukuje się minimalnej ścieżki między dwoma wierzchołkami, a~w~problemie komiwojażera -- minimalnego cyklu uwzględniającego wszystkie wierzchołki grafu (minimalny cykl Hamiltona);
Problem najkrótszej ścieżki jest wielomianowy (istnieje dla niego szybki algorytm), podczas gdy problem komiwojażera jest NP-zupełny (prawdopodobnie nie istnieje efektywny algorytm rozwiązujący ten problem).

\exercise %1.1-5
Rozwiązanie dokładne jest jedynym dopuszczalnym np.\ w~problemie wyznaczenia trajektorii sztucznej satelity.
Głównym powodem jest to, że ewentualne zniszczenie lub uszkodzenie satelity wiązałoby się z~ogromnymi stratami finansowymi.

Natomiast przybliżone rozwiązanie jest wystarczające np.\ podczas modelowania pogody.
W~problemie tym występuje bardzo wiele parametrów i~często są one dane tylko jako pewne przybliżenia rzeczywistych wielkości, przez co wyznaczenie dokładnego rozwiązania jest zazwyczaj niemożliwe.
Poza tym pewna tolerancja rozwiązania jest całkowicie dopuszczalna.
