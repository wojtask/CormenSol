\chapter{Zliczanie i~prawdopodobieństwo}

\subchapter{Zliczanie}

\exercise %C.1-1
Załóżmy, że nie rozważamy słowa pustego i~że $1\le k\le n$. Pierwsze \compound{$k$}{podsłowo} zajmuje w~\compound{$n$}{słowie} pozycje 1, 2,~\dots,~$k$, drugie -- 2, 3,~\dots,~$k+1$, itd. Ostatnie \compound{$k$}{podsłowo} leży na pozycjach $n-k+1$, $n-k+2$,~\dots,~$n$. Istnieje zatem $n-k+1$ wszystkich \compound{$k$}{podsłów} \compound{$n$}{słowa}.

By obliczyć łączną ilość podsłów \compound{$n$}{słowa}, należy zsumować liczby \compound{$k$}{podsłów} po wszystkich $1\le k\le n$, co daje
\[
	\sum_{k=1}^n(n-k+1) = \sum_{k=1}^nk = \frac{n(n+1)}{2}.
\]

\exercise %C.1-2
Niech $X=\{0,1,\dots,2^n-1\}$ i~$Y=\{0,1,\dots,2^m-1\}$ będą zbiorami liczb, odpowiednio, \compound{$n$}{bitowych} i~\compound{$m$}{bitowych}. Zauważmy, że funkcji logicznych o~$n$ wejściach i~$m$ wyjściach jest tyle samo, co funkcji $f\colon X\to Y$. Zagadnienie sprowadza się zatem do pytania o~liczbę wszystkich ciągów $\langle y_1,\dots,y_{2^n}\!\rangle$ o~wyrazach z~\compound{$2^m$}{elementowego} zbioru $Y$. Każdy wyraz $y_i$ możemy wybrać na $2^m$ sposobów, co daje $(2^m)^{2^n}=2^{m2^n}\!$ możliwości wyboru ciągu $\langle y_1,\dots,y_{2^n}\!\rangle$. Jest zatem $2^{m2^n}\!$ funkcji logicznych o~$n$ wejściach i~$m$ wyjściach, a~stąd $2^{2^n}\!$ funkcji logicznych o~$n$ wejściach i~1 wyjściu.

\exercise %C.1-3
Niech $S_n$ oznacza szukaną liczbę sposobów ustawienia $n$ osób przy stole. Jeden ze sposobów jest nierozróżnialny z~$n-1$ innymi dzięki temu, że stół jest okrągły, a~osoby mogą przesuwać się miejscami, nie zmieniając kolejności wzajemnego ustawienia. Istnieje $n!$ możliwych permutacji $n$ osób, zatem $nS_n$ jest równe $n!$. Mamy zatem
\[
	S_n = \frac{n!}{n} = (n-1)!.
\]

\exercise %C.1-4
Aby wybrać trzy liczby ze zbioru $\{1,2,\dots,100\}$, które sumują się do liczby parzystej, można postąpić według jednej z~dwóch strategii -- wybrać trzy liczby parzyste albo wybrać dwie liczby nieparzyste i~jedną liczbę parzystą. W~pierwszym przypadku możemy dokonać wyboru na $\binom{50}{3}$ sposobów, a~w drugim -- na $\binom{50}{2}\binom{50}{1}$ sposobów. Łączna liczba możliwości wyboru takich liczb wynosi zatem
\[
	\binom{50}{3}+\binom{50}{2}\binom{50}{1} = 80850.
\]

\exercise %C.1-5
\[
	\binom{n}{k} = \frac{n!}{k!\,(n-k)!} = \frac{n}{k}\cdot\frac{(n-1)!}{(k-1)!\,(n-k)!} = \frac{n}{k}\binom{n-1}{k-1}
\]

\exercise %C.1-6
\[
	\binom{n}{k} = \frac{n!}{k!\,(n-k)!} = \frac{n}{n-k}\cdot\frac{(n-1)!}{k!\,(n-k-1)!} = \frac{n}{n-k}\binom{n-1}{k}
\]

\exercise %C.1-7
Załóżmy, że wybieramy pewien \compound{$k$}{podzbiór} z~\compound{$n$}{elementowego} zbioru $S$, co można zrobić na $\binom{n}{k}$ sposobów. Wyróżnijmy pewien element z~$S$. Jeśli nie został on wybrany w~\compound{$k$}{podzbiorze}, to istnieje $\binom{n-1}{k}$ możliwości wyboru $k$ elementów spośród $n-1$ pozostałych ze zbioru $S$. Jeżeli jednak wyróżniony element należy do wybranego podzbioru, to z~$n-1$ pozostałych elementów należy wybrać jeszcze $k-1$, co można wykonać na $\binom{n-1}{k-1}$ sposobów. Otrzymujemy zatem
\[
	\binom{n}{k} = \binom{n-1}{k}+\binom{n-1}{k-1}.
\]

\exercise %C.1-8
Kilka początkowych wierszy trójkąta Pascala:
\[
	\begin{array}{ccccccccccccc}
		&&&&&& 1 \\
		&&&&& 1 && 1 \\
		&&&& 1 && 2 && 1 \\
		&&& 1 && 3 && 3 && 1 \\
		&& 1 && 4 && 6 && 4 && 1 \\
		& 1 && 5 && 10 && 10 && 5 && 1 \\
		1 && 6 && 15 && 20 && 15 && 6 && 1
	\end{array}
\]
W~pierwszym wierszu mamy tylko jeden element, $\binom{0}{0}=1$. Drugi wiersz zawiera $\binom{1}{0}=1$ i~$\binom{1}{1}=1$. Krańce kolejnych wierszy stanowią jedynki, podczas gdy elementy wewnętrzne obliczane są z~równania z~poprzedniego zadania, czyli poprzez zsumowanie dwóch elementów występujących bezpośrednio nad danym elementem.

\exercise %C.1-9
Z~definicji współczynnika dwumianowego i~z~tożsamości~(A.1) mamy
\[
	\binom{n+1}{2} = \frac{(n+1)!}{2!\,(n-1)!} = \frac{n(n+1)}{2} = \sum_{i=1}^ni.
\]

\exercise %C.1-10
Potraktujmy współczynnik dwumianowy jako funkcję $b_n(k)=\binom{n}{k}$ dla $0\le k\le n$ i~sprawdźmy, dla jakich $k$ wartość $b_n(k)$ jest największa. Badamy stosunek
\[
	\frac{b_n(k+1)}{b_n(k)} = \frac{\binom{n}{k+1}}{\binom{n}{k}} = \frac{n!}{(k+1)!\,(n-k-1)!}\cdot\frac{k!\,(n-k)!}{n!} = \frac{n-k}{k+1}.
\]
Jeśli $n-k\ge k+1$, czyli $k\le(n-1)/2$, to funkcja $b_n$ jest niemalejąca. Z~kolei gdy $k\ge(n-1)/2$, to $b_n$ jest nierosnące. A~zatem gdy $n$ jest nieparzyste, to $b_n$ przyjmuje największą wartość dla $k=(n-1)/2=\lfloor n/2\rfloor$. Ponadto z~wzoru~(C.3) mamy
\[
    b_n((n-1)/2) = \binom{n}{(n-1)/2} = \binom{n}{n-(n-1)/2} = \binom{n}{(n+1)/2} = b_n((n+1)/2),
\]
a~więc wartość największa jest przyjmowana również dla $k=(n+1)/2=\lceil n/2\rceil$.

W~przypadku gdy $n$ jest liczbą parzystą, maksymalna wartość funkcji $b_n$ jest osiągana dla $k=n/2$ lub $k=n/2-1$. Sprawdźmy, która wartość jest większa:
\[
    \frac{b_n(n/2)}{b_n(n/2-1)} = \frac{\binom{n}{n/2}}{\binom{n}{n/2-1}} = \frac{n!}{(n/2)!\,(n/2)!}\cdot\frac{(n/2-1)!\,(n/2+1)!}{n!} = \frac{n/2+1}{n/2} > 1.
\]
Stąd $b_n$ przyjmuje maksimum dla $k=n/2=\lfloor n/2\rfloor=\lceil n/2\rceil$.

\exercise %C.1-11
Dla $n$, $j$,~$k\ge0$ takich, że $j+k\le n$, sprawdzamy zachowanie stosunku
\[
	\frac{\binom{n}{j+k}}{\binom{n}{j}\binom{n-j}{k}} = \frac{n!}{(j+k)!\,(n-j-k)!}\cdot\frac{j!\,(n-j)!}{n!}\cdot\frac{k!\,(n-j-k)!}{(n-j)!} = \frac{j!\,k!}{(j+k)!} = \frac{\prod_{i=1}^ki}{\prod_{i=1}^k(j+i)}.
\]
Oczywistym jest, że iloczyn $k$ liczb całkowitych od 1 do $k$ nie przekracza iloczynu $k$ liczb całkowitych od $j+1$ do $j+k$, zatem otrzymany stosunek nie przekracza 1. Równość zachodzi tylko w~przypadku, gdy $j=0$ lub $k=0$.

Lewą stronę nierówności można zinterpretować jako liczbę możliwych wyborów $j+k$ przedmiotów spośród zbioru \compound{$n$}{elementowego}, prawą zaś jako liczbę możliwości wyboru najpierw $j$ przedmiotów spośród $n$, a~następnie $k$ przedmiotów spośród $n-j$ pozostawionych po pierwszym wyborze. Załóżmy, że $A=\{a_1,a_2,\dots,a_{j+k}\}$ jest zbiorem wybranych elementów. Jest tylko 1 sposób wyboru zadanego zbioru $A$ przy pierwszej strategii i~o~wiele więcej, jeśli zastosuje się drugie podejście. Można mianowicie dowolnie podzielić elementy z~$A$ na $j$ takich, które będą wybierane w~pierwszym kroku i~$k$ takich, które wybierzemy w~drugim kroku.

\exercise %C.1-12
Przypadek dla $k=0$ sprawdzamy w~pierwszym kroku indukcyjnym i~stwierdzamy, że zachodzi. Przyjmijmy zatem, że $k\ge1$ i~załóżmy, że
\[
	\binom{n}{k-1} \le \frac{n^n}{(k-1)^{k-1}(n-k+1)^{n-k+1}}.
\]
Mamy teraz z~\refExercise{C.1-5} i~\refExercise{C.1-6} oraz z~założenia indukcyjnego, że
\[
	\binom{n}{k} = \frac{n}{k}\binom{n-1}{k-1} = \frac{n-k+1}{k}\binom{n}{k-1} \le \frac{n^n}{k(k-1)^{k-1}(n-k+1)^{n-k}}.
\]
Wystarczy wykazać nierówność
\[
	\frac{n^n}{k(k-1)^{k-1}(n-k+1)^{n-k}} \le \frac{n^n}{k^k(n-k)^{n-k}},
\]
co zrobimy poprzez sprawdzenie ilorazu
\[
	\frac{\frac{n^n}{k(k-1)^{k-1}(n-k+1)^{n-k}}}{\frac{n^n}{k^k(n-k)^{n-k}}} = \frac{k^{k-1}(n-k)^{n-k}}{(k-1)^{k-1}(n-k+1)^{n-k}} = \frac{\bigl(\frac{k}{k-1}\bigr)^{k-1}}{\bigl(\frac{n-k+1}{n-k}\bigr)^{n-k}} = \frac{\bigl(1+\frac{1}{k-1}\bigr)^{k-1}}{\bigl(1+\frac{1}{n-k}\bigr)^{n-k}}.
\]
Ciąg $e_n={(1+1/n)}^n$ jest rosnący, skąd dostajemy, że ostatni iloraz nie przekracza 1, o~ile $k-1\le n-k$, czyli $k\le(n+1)/2$, a~więc tym bardziej $k\le n/2$ i~twierdzenie dla takich $k$ jest spełnione.

Z~wzoru~(C.3) mamy $\binom{n}{k}=\binom{n}{n-k}$ i~gdy $n/2<k\le n$, to dowód sprowadza się do pokazania, że
\[
	\binom{n}{n-k} \le \frac{n^n}{k^k(n-k)^{n-k}},
\]
ponieważ wtedy $0\le n-k<n/2$ i~symetrycznie udowodnimy wzór dla $k>n/2$.

\exercise %C.1-13
Wykorzystując wzór Stirlinga, mamy
\[
	\binom{2n}{n} = \frac{(2n)!}{(n!)^2} = \frac{\sqrt{4\pi n}\,\bigl(\frac{2n}{e}\bigr)^{2n}(1+\Theta(1/n))}{2\pi n\,\bigl(\frac{n}{e}\bigr)^{2n}(1+\Theta(1/n))^2} = \frac{2^{2n}}{\sqrt{\pi n}}\cdot\frac{1+\Theta(1/n)}{(1+\Theta(1/n))^2}.
\]

Zajmiemy się teraz ostatnim ułamkiem, którego celowo nie skracaliśmy, ponieważ funkcja w~liczniku nie musi być równa kwadratowi funkcji z~mianownika. Niech $c\ge d>0$ będą stałymi takimi, że funkcja $1+c/n$ ogranicza licznik ułamka od góry, a~funkcja $1+d/n$ -- mianownik ułamka od dołu. Mamy
\[
    \frac{1+\Theta(1/n)}{(1+\Theta(1/n))^2} \le \frac{1+c/n}{(1+d/n)^2} \le \frac{1+c/n}{1+d/n} = \frac{n+c}{n+d} = \frac{n+d}{n+d}+\frac{c-d}{n+d} = 1+O(1/n).
\]
Otrzymujemy ostatecznie
\[
    \binom{2n}{n} = \frac{2^{2n}}{\sqrt{\pi n}}(1+O(1/n)).
\]

\exercise %C.1-14
Niech $g$ i~$h$ będą funkcjami określonymi w~$(0,1)$ i~niech będą różniczkowalne w~tym przedziale. Dla $h(x)=-x\lg x$ mamy
\[
	\frac{dh}{dx}(g(x)) = -\frac{dg}{dx}(x)\cdot\bigl(\lg g(x)+\lg e\bigr).
\]
Liczymy pierwszą pochodną funkcji entropii $H(\lambda)=h(\lambda)+h(1-\lambda)$:
\[
	\frac{dH}{d\lambda}(\lambda) = \frac{dh}{d\lambda}(\lambda)+\frac{dh}{d\lambda}(1-\lambda) = -\lg\lambda-\lg e+\lg(1-\lambda)+\lg e = \lg(1-\lambda)-\lg\lambda = \lg\frac{1-\lambda}{\lambda}.
\]
Pierwsza pochodna $H$ przyjmuje wartość 0, gdy $\lambda=1/2$. Należy jeszcze zbadać drugą pochodną funkcji $H$:
\[
	\frac{d^2\!H}{d\lambda^2}(\lambda) = -\frac{\lg e}{\lambda(1-\lambda)}.
\]
W~punkcie $\lambda=1/2$ pochodna ta jest ujemna, czyli binarna funkcja entropii $H$ osiąga maksimum wynoszące $H(1/2)=1$.

\exercise %C.1-15
Dla $n=0$ równość jest trywialna, przyjmijmy więc, że $n\ge1$. Wówczas
\[
	\sum_{k=1}^n\binom{n}{k}k = \sum_{k=1}^n\binom{n-1}{k-1}n = n\sum_{k=0}^{n-1}\binom{n-1}{k} = n2^{n-1}.
\]
Ostatnia równość zachodzi z~wzoru~(C.4) dla $x=y=1$.

\subchapter{Prawdopodobieństwo}

\exercise %C.2-1
Utwórzmy skończoną lub przeliczalną rodzinę zdarzeń $\{C_1,C_2,\dots\}$, gdzie
\[
	C_k = A_k\setminus\bigcup_{i=1}^{k-1}A_i
\]
dla każdego $k=1$, 2,~\dots. Korzystając z~faktu, że $\bigcup_iA_i = \bigcup_iC_i$ oraz z~tego, że zdarzenia $C_1$,~$C_2$,~\dots{} wzajemnie się wykluczają, otrzymujemy
\[
	\Pr\biggl(\bigcup_iA_i\biggr) = \Pr\biggl(\bigcup_iC_i\biggr) = \sum_i\Pr(C_i).
\]
Ponieważ $\Pr(C_i)\le\Pr(A_i)$ dla każdego $i=1$, 2,~\dots, to prawdą jest, że
\[
	\Pr\biggl(\bigcup_iA_i\biggr) \le \sum_i\Pr(A_i).
\]

\exercise %C.2-2
Zdefiniujmy \compound{3}{słowo} nad alfabetem $\{{\scriptstyle\rm O},{\scriptstyle\rm R}\}$ w~następujący sposób. Pierwszy symbol tego słowa oznacza wynik rzutu monetą profesora Rosencrantza, drugi symbol to wynik rzutu pierwszą monetą profesora Guildensterna, a~trzeci to wynik rzutu jego drugą monetą, przy czym $\scriptstyle\rm O$ oznacza uzyskanie orła, a~$\scriptstyle\rm R$ -- uzyskanie reszki. Tworzymy przestrzeń zdarzeń elementarnych:
\[
	S = \{{\scriptstyle\rm OOO},{\scriptstyle\rm OOR},{\scriptstyle\rm ORO},{\scriptstyle\rm ORR},{\scriptstyle\rm ROO},{\scriptstyle\rm ROR},{\scriptstyle\rm RRO},{\scriptstyle\rm RRR}\}.
\]
Każde ze zdarzeń z~$S$ zachodzi z~prawdopodobieństwem równym $1/8$, w~szczególności zdarzenie $\scriptstyle\rm ORR$ oznaczające wyrzucenie przez profesora Rosencrantza większej ilości orłów od przeciwnika.

\exercise %C.2-3
Oznaczmy zdarzenia:
\begin{itemize}
	\item[$A$] -- numer drugiej karty jest większy od numeru pierwszej karty,
	\item[$B$] -- numer trzeciej karty jest większy od numeru drugiej karty.
\end{itemize}
Jeśli numer drugiej karty wynosi $k$, to liczba zdarzeń sprzyjających $A$ wynosi $k-1$, a~liczba zdarzeń sprzyjających $B$ wynosi $10-k$. Mamy obliczyć $\Pr(A\cap B)$. Korzystając z~reguły iloczynu, dostajemy, że liczba zdarzeń sprzyjających $A\cap B$ wynosi
\[
    \sum_{k=1}^{10}(k-1)(10-k) = 120.
\]
Liczbą możliwych sposobów wyboru trzech kart spośród dziesięciu jest $10!/7!=720$ (liczba wszystkich \compound{3}{permutacji} zbioru \compound{10}{elementowego}), dostajemy zatem wynik $\Pr(A\cap B)=120/720=1/6$.

\exercise %C.2-4
Będziemy rzucać monetą i~tworzyć nowy ciąg zer i~jedynek, w~zależności od wyniku rzutu, dla orła przyjmując~1, a~dla reszki~0. Ponieważ $a<b$, to $a/b<1$, więc część ułamkowa binarnego rozwinięcia ilorazu $a/b$ jest nieskończonym ciągiem zer i~jedynek. Rzucamy monetą, dopóki tworzony przez nas ciąg zgadza się z~początkowymi cyframi rozwinięcia binarnego części ułamkowej $a/b$. W~momencie gdy natrafimy na pierwszą różnicę, wyznaczony dotychczas ciąg traktujemy jako część ułamkową rozwinięcia binarnego pewnej liczby. Jeśli liczba ta jest mniejsza od $a/b$, to zwracamy orła, w~przeciwnym przypadku -- reszkę.

Oczekiwana liczba rzutów monetą potrzebnych do wyznaczenia wyniku jest oczekiwaną liczbą rzutów aż do pierwszej różnicy w~porównaniu z~rozwinięciem binarnym części ułamkowej $a/b$. Jeśli przyjmiemy, że sukcesem jest wynik rzutu monetą niepasujący do bieżącego elementu rozwinięcia, to liczba rzutów $n$ aż do pierwszego sukcesu jest zmienną losową $X$ o~rozkładzie geometrycznym. Ponieważ prawdopodobieństwo sukcesu wynosi $p=1/2$, to z~wzoru~(C.31) otrzymujemy $\E(X)=1/p=2$. Widać zatem, że oczekiwana liczba rzutów monetą w~opisanej procedurze jest stała.

\exercise %C.2-5
Korzystając z~obserwacji, że $(A\cap B)\cup(\overline{A}\cap B)=B$ oraz z~tego, że zdarzenia $A\cap B$ i~$\overline{A}\cap B$ wykluczają się, otrzymujemy
\[
	\Pr(A\mid B)+\Pr(\overline{A}\mid B) = \frac{\Pr(A\cap B)}{\Pr(B)}+\frac{\Pr(\overline{A}\cap B)}{\Pr(B)} = \frac{\Pr(B)}{\Pr(B)} = 1.
\]

\exercise %C.2-6
Dowód przez indukcję względem liczby zdarzeń. Dla $n=1$ dowód jest trywialny, załóżmy zatem, że $n\ge2$. Otrzymujemy
\begin{align*}
	\Pr\biggl(\bigcap_{i=1}^nA_i\biggr) &= \Pr\biggl(A_n\cap\bigcap_{i=1}^{n-1}A_i\biggr) \\
	&= \Pr\biggl(\bigcap_{i=1}^{n-1}A_i\biggr)\Pr\biggl(A_n\biggm|\bigcap_{i=1}^{n-1}A_i\biggr) \\
	&= \Pr(A_1)\Pr(A_2\mid A_1)\Pr(A_3\mid A_1\cap A_2)\dots\Pr\biggl(A_n\biggm|\bigcap_{i=1}^{n-1}A_i\biggr).
\end{align*}
Druga równość wynika z~definicji prawdopodobieństwa warunkowego, a~trzecia -- z~założenia indukcyjnego.

\exercise %C.2-7
Niech $S=\{s_1,s_2,\dots,s_{n^2}\}$ będzie przestrzenią zdarzeń oraz $\Pr(s_i)=1/n^2$ dla każdego $i=1$, 2,~\dots,~$n^2$. Pokażemy teraz jak skonstruować zdarzenia $A_1$, $A_2$,~\dots,~$A_n\subseteq S$, które spełniają warunek z~treści zadania.

Będziemy dążyć do tego, aby dla każdych, parami różnych, $1\le i_1$, $i_2$,~$i_3\le n$, zachodziło
\[
	|A_{i_1}\cap A_{i_2}| = 1 \quad\text{oraz}\quad |A_{i_1}\cap A_{i_2}\cap A_{i_3}| = 0.
\]
Ponieważ dysponujemy $n^2$ zdarzeniami elementarnymi, to wybierzmy spośród nich $n(n-1)/2$ różnych i~umieśćmy po jednym w~każdym możliwym przecięciu $A_i\cap A_j$, gdzie $i<j$. Każde zdarzenie $A_i$ składa się teraz z~$n-1$ zdarzeń elementarnych, uzupełnijmy je zatem tak, aby były zbiorami \compound{$n$}{elementowymi}, zachowując jednocześnie powyższe warunki. Jest to możliwe, ponieważ mamy jeszcze $n(n+1)/2$ niewykorzystanych zdarzeń ze zbioru $S$.

Z~opisanej konstrukcji zdarzeń $A_1$, $A_2$,~\dots,~$A_n$ wynika, że $\Pr(A_i)=1/n$ dla $i=1$, 2,~\dots,~$n$, a~zatem
\[
	\Pr(A_{i_1}\cap A_{i_2}) = \frac{1}{n^2} = \Pr(A_{i_1})\Pr(A_{i_2})
\]
dla wszystkich $1\le i_1<i_2\le n$ oraz
\[
	\Pr\biggl(\bigcap_{j=1}^kA_{i_j}\biggr) = 0 \ne \prod_{j=1}^k\Pr(A_{i_j})
\]
dla każdego $k>2$ i~dowolnych, parami różnych, indeksów $i_1$,~\dots, $i_k$.

Skonstruowany zbiór zawiera $n$ zdarzeń, które są parami niezależne, ale żaden ich \compound{$k$}{podzbiór}, gdzie $k>2$, nie jest wzajemnie niezależny.

\exercise %C.2-8
Rozważmy pewną grupę 400 osób, z~których 100 osiągnęło już wiek 50 lat. W~młodszej grupie jest 20 programistów i~6 matematyków, przy czym tylko 3 osoby jednocześnie programują i~zajmują się matematyką. W~skład grupy seniorów wchodzi 20 programistów oraz 10 matematyków, a~5 jednocześnie jest programistami i~matematykami. Rysunek~\ref{fig:C.2-8} stanowi ilustrację przedstawionego opisu.
\begin{figure}[ht]
	\begin{center}
		\includegraphics{figc.1}
	\end{center}
	\caption{Diagramy Venna ilustrujące rozważaną grupę ludzi. Liczby w~poszczególnych obszarach oznaczają liczby osób należących do odpowiednich grup.} \label{fig:C.2-8}
\end{figure}

Spośród osób z~tej grupy losowo wybieramy jedną. Oznaczmy następujące zdarzenia:
\begin{itemize}
	\item[$A$] -- wybrano osobę w~wieku powyżej 50 lat,
	\item[$B$] -- wybrano programistę,
	\item[$C$] -- wybrano matematyka.
\end{itemize}
Obliczamy ich prawdopodobieństwa:
\[
	\Pr(A) = \frac{1}{4}, \quad \Pr(B) = \frac{1}{10}, \quad \Pr(C) = \frac{1}{25}.
\]
Zauważmy, że zdarzenia $A$ i~$B$ nie są niezależne, ponieważ
\[
	\Pr(A\cap B) = \frac{1}{20} \ne \frac{1}{40} = \Pr(A)\Pr(B).
\]
Są jednak warunkowo zależne od zdarzenia $C$:
\[
	\Pr(A\cap B\mid C) = \frac{5}{16} = \frac{5}{8}\cdot\frac{1}{2} = \Pr(A\mid C)\Pr(B\mid C).
\]

\exercise %C.2-9
Rozważmy zdarzenie $A$ oznaczające, że wybraliśmy zasłonę, za którą znajduje się nagroda. Oczywiście $\Pr(A)=1/3$ i~z~wzoru Bayesa wynika, że prawdopodobieństwo wygranej $W$ wynosi
\[
	\Pr(W) = \Pr(W\mid A)\Pr(A)+\Pr(W\mid\overline{A})\Pr(\overline{A}).
\]
Obliczmy wartość powyższego prawdopodobieństwa w~zależności od podjętej decyzji po podniesieniu przez prowadzącego jednej z~zasłon. W~pierwszej strategii decydujemy się na pozostanie przy aktualnym wyborze, zatem ponieważ nie zmieniamy wybranej zasłony, mamy $\Pr(W\mid A)=1$ i~$\Pr(W\mid\overline{A})=0$, a~więc wygramy z~prawdopodobieństwem $\Pr(W)=1/3$. Jeśli teraz rozważymy drugą strategię, to będziemy mieć $\Pr(W\mid A)=0$ i~$\Pr(W\mid\overline{A})=1$, ponieważ dokonaliśmy zmiany zasłony początkowo wybranej na inną, dotychczas nieodkrytą. W~tym przypadku prawdopodobieństwo wygranej wynosi $\Pr(W)=2/3$. Widać zatem, że zmiana zasłony jest opłacalna, ponieważ podwaja szanse na wygraną.

\exercise %C.2-10
Niech $A_X$, $A_Y$ i~$A_Z$ będą prawdopodobieństwami wyjścia na wolność, odpowiednio, więźnia $X$, $Y$ i~$Z$. Przed rozmową ze strażnikiem prawdopodobieństwo, że $X$ będzie wolny, wynosi $\Pr(A_X)=1/3$. Jeśli więzień $X$ dostał informację, że więzień $Y$ zostanie ścięty, to aby zobaczyć, czy zmienia to szanse $X$ na wolność, obliczmy $\Pr(A_X\mid\overline{A_Y})$. Z~wzoru Bayesa~(C.21), mamy
\[
	\Pr(A_X\mid\overline{A_Y}) = \frac{\Pr(A_X)\Pr(\overline{A_Y}\mid A_X)}{\Pr(\overline{A_Y})}.
\]
Z~kolei wiadomo, że szanse ścięcia $Y$ wynoszą
\begin{align*}
	\Pr(\overline{A_Y}) &= \Pr(\overline{A_Y}\mid A_X)\Pr(A_X)+\Pr(\overline{A_Y}\mid A_Y)\Pr(A_Y)+\Pr(\overline{A_Y}\mid A_Z)\Pr(A_Z) \\
	&= \Pr(\overline{A_Y}\cap A_X)+\Pr(\overline{A_Y}\cap A_Y)+\Pr(\overline{A_Y}\cap A_Z) \\
	&= 1/3+0+1/3 = 2/3,
\end{align*}
a~zatem
\[
	\Pr(A_X\mid\overline{A_Y}) = \frac{(1/3)\cdot1}{2/3} = \frac{1}{2}>\frac{1}{3}.
\]
Wynika stąd, że szanse więźnia $X$ na wyjście na wolność zwiększyły się po rozmowie ze strażnikiem i~teraz wynoszą~1/2.

\subchapter{Dyskretne zmienne losowe}

\exercise %C.3-1
Niech $X$ będzie zmienną losową oznaczającą sumę oczek na obu kostkach. Mamy
\begin{align*}
	\E(X) &= \sum_{x=2}^{12}x\Pr(X=x) \\
	&= 2\cdot\frac{1}{36}+3\cdot\frac{2}{36}+4\cdot\frac{3}{36}+5\cdot\frac{4}{36}+6\cdot\frac{5}{36}+7\cdot\frac{6}{36} \\
	&\quad +8\cdot\frac{5}{36}+9\cdot\frac{4}{36}+10\cdot\frac{3}{36}+11\cdot\frac{2}{36}+12\cdot\frac{1}{36} \\[1mm]
	&= 7.
\end{align*}
Niech teraz $Y$ będzie zmienną losową oznaczającą większą z~liczb oczek na obu kostkach. Zachodzi
\begin{align*}
	\E(Y) &= \sum_{y=1}^{6}y\Pr(Y=y) \\
	&= 1\cdot\frac{1}{36}+2\cdot\frac{3}{36}+3\cdot\frac{5}{36}+4\cdot\frac{7}{36}+5\cdot\frac{9}{36}+6\cdot\frac{11}{36} \\[2mm]
	&\approx 4{,}47.
\end{align*}

\exercise %C.3-2
Niech $X$ będzie zmienną losową przyjmującą wartość indeksu największego elementu tablicy $A$. Zauważmy, że jeśli tablica zawiera losową permutację $n$ liczb, to $\Pr(X=i)=1/n$ dla każdego $i=1$, 2,~\dots,~$n$, a~zatem
\[
	\E(X) = \sum_{i=1}^ni\Pr(X=i) = \frac{1}{n}\sum_{i=1}^ni = \frac{1}{n}\cdot\frac{n(n+1)}{2} = \frac{n+1}{2}.
\]
Wynik jest identyczny dla każdego elementu tablicy, w~szczególności także dla elementu najmniejszego.

\exercise %C.3-3
Zdefiniujmy zmienną losową $X$ przyjmującą wielkość wygranej w~opisanej grze. Mamy obliczyć
\[
	\E(X) = -\Pr(A_0)+\Pr(A_1)+2\Pr(A_2)+3\Pr(A_3),
\]
przy czym $A_i$, dla $i=0$, 1,~2,~3, oznacza zdarzenie, że obstawiona przez gracza liczba oczek pojawiła się na dokładnie $i$ kostkach. Prawdopodobieństwa tych zdarzeń wynoszą:
\[
	\begin{matrix}
	\Pr(A_0) &=& \dfrac{5^3}{6^3} &=& \dfrac{125}{216}, \\[3mm]
	\Pr(A_1) &=& \dfrac{3\cdot 5^2}{6^3} &=& \dfrac{75}{216}, \\[3mm]
	\Pr(A_2) &=& \dfrac{3\cdot 5^1}{6^3} &=& \dfrac{15}{216}, \\[3mm]
	\Pr(A_3) &=& \dfrac{1}{6^3} &=& \dfrac{1}{216}.
	\end{matrix}
\]
Dostajemy zatem
\[
	\E(X) = -\dfrac{17}{216} \approx -0{,}0787,
\]
a~więc gracz straci w~tej grze średnio prawie 8 gr.

\exercise %C.3-4
Załóżmy, że zmienne losowe $X$, $Y$ są określone w~przestrzeni zdarzeń $S$. Niech $X\ge Y$ w~pewnym podzbiorze przestrzeni $S$. Ponieważ $Y\ge0$, więc również $\E(Y)\ge0$, skąd zachodzi
\[
    \E(\max(X,Y)) = \E(X) \le \E(X)+\E(Y).
\]
Przypadek, gdy $Y\ge X$ dowodzi się analogicznie.

\exercise %C.3-5
Zgodnie z~definicją niezależnych zmiennych losowych $X$,~$Y$\!, mamy
\[
	\Pr(X=x\;\;\text{i}\;\;Y=y) = \Pr(X=x)\Pr(Y=y).
\]
Niech $f$ i~$g$ będą dowolnymi funkcjami rzeczywistymi. Jeśli $X$ przyjmuje pewną wartość $x$, to zmienna losowa $f(X)$ przyjmuje wartość $f(x)$. Analogicznie dla $Y$: jeśli $Y=y$, to $g(Y)=g(y)$. Powyższe równanie przyjmuje zatem postać
\[
	\Pr\bigl(f(X)=f(x)\;\;\text{i}\;\;g(Y)=g(y)\bigr) = \Pr\bigl(f(X)=f(x)\bigr)\Pr\bigl(g(Y)=g(y)\bigr),
\]
skąd wnioskujemy, że $f(X)$ i~$g(Y)$ są niezależnymi zmiennymi losowymi.

\exercise %C.3-6
Dowód sprowadza się do wykazania nierówności
\[
	\E(X) \ge t\Pr(X\ge t).
\]
Wartość oczekiwaną $\E(X)$ zapisujemy w~następujący sposób:
\[
    \E(X) = \sum_xx\Pr(X=x) = \sum_{x<t}x\Pr(X=x)+\sum_{x\ge t}x\Pr(X=x).
\]
Ponieważ zmienna losowa $X$ jest nieujemna, to można ograniczyć $E(X)$ od dołu, opuszczając pierwszą sumę w~ostatnim wyrażeniu powyższego wzoru. Zachodzi więc
\[
    \E(X) \ge \sum_{x\ge t}x\Pr(X=x) \ge t\sum_{x\ge t}\Pr(X=x) = t\Pr(X\ge t).
\]

\exercise %C.3-7
Niech $A=\{\,s\in S:X(s)\ge t\,\}$ oraz $A'=\{\,s\in S:X'(s)\ge t\,\}$. Wprost z~definicji zmiennej losowej mamy
\[
	\Pr(X\ge t) = \sum_{s\in A}\Pr(s) \quad\text{oraz}\quad \Pr(X'\ge t) = \sum_{s\in A'}\Pr(s).
\]
Dowodzimy, że
\[
	\sum_{s\in A}\Pr(s)\quad \ge \quad\sum_{s\in A'}\Pr(s). \tag{$*$}\label{eq:C.3-5_1}
\]
Z~założenia wynika, że dla dowolnej liczby $t$, jeśli $X'(s)\ge t$, to $X(s)\ge t$. Niech $S'\subseteq S$ będzie takim zbiorem, że dla każdego $s'\in S'$, $X'(s')<t$ i~$X(s')\ge t$. Wtedy suma po lewej stronie nierówności~(\ref{eq:C.3-5_1}) zawiera o~$|S'|$ więcej składników niż suma po prawej stronie i~z~tego, że $\Pr(s)\ge0$ dla dowolnego $s\in S$, mamy, że nierówność (\ref{eq:C.3-5_1}) zachodzi.

\exercise %C.3-8
Wariancja jest liczbą nieujemną, a~więc wprost z~jej definicji otrzymujemy
\[
	0 \le \Var(X) = \E(X^2)-\E^2(X),
\]
skąd $\E(X^2)\ge\E^2(X)$.

\exercise %C.3-9
Niech $\Pr(X=0)=p$ oraz $\Pr(X=1)=1-p$, skąd
\[
    \E(X)=0\cdot p+1\cdot(1-p)=1-p.
\]
Zauważmy ponadto, że $\E(X^2)=\E(X)$ i~obliczmy wariancję zmiennej losowej $X$:
\begin{align*}
	\Var(X) &= \E(X^2)-\E^2(X) = (1-p)-(1-p)^2 \\[1mm]
	&= p(1-p) = \E(X)(1-\E(X)) = \E(X)\E(1-X),
\end{align*}
przy czym ostatnia równość zachodzi dzięki liniowości wartości oczekiwanej.

\exercise %C.3-10
Wprost z~definicji wariancji oraz z~wzoru~(C.21):
\[
	\Var(aX) = \E(a^2X^2)-\E^2(aX) = a^2\E(X^2)-(a\E(X))^2 = a^2\bigl(\E(X^2)-\E(X)\bigr) = a^2\Var(X).
\]

\subchapter{Rozkłady: geometryczny i~dwumianowy}

\exercise %C.4-1
Rodzina zdarzeń elementarnych $S$ składa się ze zdarzeń, które wymagają $k$ prób, zanim nastąpi pierwszy sukces, gdzie $k=1$, 2,~\dots. Mamy zatem
\[
	\Pr(S) = \sum_{k=1}^\infty\Pr(X=k) = \sum_{k=1}^\infty q^{k-1}p = p\sum_{k=0}^\infty (1-p)^k = \frac{p}{1-(1-p)} \\[1mm] = 1.
\]
Wykorzystano wzór~(A.6) przy założeniu, że $p>0$.

\exercise %C.4-2
Niech sukces oznacza uzyskanie w~rzucie sześcioma monetami trzech orłów i~trzech reszek, a~porażka -- każdy inny wynik. Możemy wybrać dowolne trzy monety spośród sześciu, na których będzie orzeł, mamy zatem $\binom{6}{3}=20$ sposobów osiągnięcia sukcesu. Jest $2^6=64$ wszystkich możliwych wyników, zatem prawdopodobieństwo sukcesu wynosi $p=20/64=5/16$. Z~wzoru~(C.31) otrzymujemy, że musimy wykonać średnio $1/p=3{,}2$ rzutów.

\exercise %C.4-3
Z~definicji rodziny rozkładów dwumianowych mamy
\[
	b(k;n,p) = \binom{n}{k}p^k(1-p)^{n-k} \quad\text{oraz}\quad b(n-k;n,q) = \binom{n}{n-k}q^{n-k}(1-q)^k.
\]
Ponieważ $\binom{n}{n-k}=\binom{n}{k}$ (z~wzoru~(C.3)) oraz $q=1-p$, to otrzymujemy $b(k;n,p)=b(n-k;n,q)$.

\exercise %C.4-4
Ponieważ rozkład dwumianowy przyjmuje maksimum dla pewnego $k$ całkowitego, gdzie $np-q\le k\le(n+1)p$, to dobrym przybliżeniem wartości maksymalnej jest wartość przyjmowana dla $k=np$, które oczywiście należy do tego przedziału. Wyznaczmy tę wartość, nie zważając na to, że argumenty silni mogą być niecałkowite:
\[
	b(np;n,p) = \binom{n}{np}p^{np}(1-p)^{n-np} = \frac{n!}{(np)!\,(n-np)!}\cdot p^{np}(1-p)^{n-np} = \frac{n!}{(np)!\,(nq)!}\cdot p^{np}q^{nq}.
\]
Wykorzystując wzór Stirlinga do przybliżenia silni, możemy uprościć pierwszy czynnik ostatniego wyrażenia:
\[
	\frac{n!}{(np)!\,(nq)!} \approx \frac{\sqrt{2\pi n}\,\bigl(\frac{n}{e}\bigr)^n}{\sqrt{2\pi np}\,\bigl(\frac{np}{e}\bigr)^{np}\sqrt{2\pi nq}\,\bigl(\frac{nq}{e}\bigr)^{nq}} = \frac{\bigl(\frac{n}{e}\bigr)^n\bigl(\frac{e}{np}\bigr)^{np}\bigl(\frac{e}{nq}\bigr)^{nq}}{\sqrt{2\pi npq}} = \frac{1}{p^{np}q^{nq}\sqrt{2\pi npq}}.
\]
Stąd dostajemy przybliżenie
\[
	b(np;n,p) \approx \frac{1}{\sqrt{2\pi npq}}
\]
na maksymalną wartość rozkładu dwumianowego $b(k;n,p)$.

\exercise %C.4-5
Niech $X$ będzie zmienną losową przyjmującą liczbę sukcesów w~tym doświadczeniu losowym. Wtedy
\[
	\Pr(X=0) = b(0;n,1/n) = \biggl(1-\frac{1}{n}\biggr)^n,
\]
co dąży do $1/e$ wraz ze wzrostem $n$. Wynika to stąd, że ciąg $e_n={(1+k/n)}^n$ ma dla dowolnej stałej $k$ granicę równą $e^k$ przy $n$ dążącym do $\infty$.

Analogicznie,
\[
	\Pr(X=1) = b(1;n,1/n) = \frac{\bigl(1-\frac{1}{n}\bigr)^n}{1-\frac{1}{n}}
\]
również dąży do $1/e$, ponieważ mianownik zbliża się do~1 wraz ze wzrostem $n$.

\exercise %C.4-6
Obliczymy dwoma sposobami prawdopodobieństwo uzyskania przez profesorów równej liczby orłów. W~pierwszym z~nich, niech $X$ i~$Y$ będą zmiennymi losowymi oznaczającymi liczby orłów uzyskane kolejno przez obu profesorów. Prawdopodobieństwo uzyskania $k$ orłów ($0\le k\le n$) przez każdego z~nich jest rozkładem dwumianowym, $\Pr(X=k)=\Pr(Y=k)=b(k;n,1/2)$. Zdarzenia $X=k$ i~$Y=k$ są niezależne, zatem prawdopodobieństwo uzyskania przez obu profesorów równej ilości orłów wynosi
\begin{align*}
	\sum_{k=0}^n\Pr(X=k\;\;\text{i}\;\;Y=k) &= \sum_{k=0}^n\Pr(X=k)\Pr(Y=k) \\
	&= \sum_{k=0}^n\binom{n}{k}\biggl(\frac{1}{2}\biggr)^n\binom{n}{k}\biggl(\frac{1}{2}\biggr)^n \\
	&= \frac{\sum_{k=0}^n\binom{n}{k}^2}{4^n}.
\end{align*}

Drugim sposobem uzyskania tego prawdopodobieństwa jest potraktowanie wyniku każdego doświadczenia jako \compound{$2n$}{elementowego} ciągu orłów i~reszek takiego, że początkowych $n$ wyrazów oznacza wyniki uzyskane przez profesora Rosencrantza, a~$n$ końcowych -- wyniki profesora Guildensterna. Wszystkich takich ciągów jest $2^{2n}=4^n$. Niech sukcesem dla profesora Rosencrantza będzie uzyskanie orła, a~dla profesora Guildensterna -- uzyskanie reszki. Zauważmy, że wyrzucenie równej liczby orłów przez obu profesorów jest równoważne z~osiągnięciem przez nich w~sumie $n$ sukcesów. Liczba sposobów, na jakie można to zrobić, jest liczbą możliwości wyboru spośród $2n$ pozycji ciągu $n$ odpowiedzialnych za sukces. Wartość ta wynosi $\binom{2n}{n}$, a~zatem szukane prawdopodobieństwo jest równe
\[
	\frac{\binom{2n}{n}}{4^n}.
\]

Przyrównując do siebie wyniki z~obu sposobów, dostajemy tożsamość
\[
	\sum_{k=0}^n\binom{n}{k}^2 = \binom{2n}{n}.
\]

\exercise %C.4-7
Wykorzystując nierówność
\[
	\binom{n}{\lambda n} \le 2^{nH(\lambda)},
\]
wynikającą z~wzoru~(C.6), otrzymujemy
\[
	b(k;n,1/2) = \binom{n}{k}\biggl(\frac{1}{2}\biggr)^n \le \frac{2^{nH(k/n)}}{2^n} = 2^{nH(k/n)-n}.
\]

\exercise %C.4-8
\note{Znak nierówności, której prawdziwość należy udowodnić w~tym zadaniu, powinien być skierowany przeciwnie. Błąd występuje również w~tekście oryginalnym.}

\noindent Przekształćmy dowodzoną zależność do alternatywnej postaci, rozważając prawdopodobieństwo zdarzenia przeciwnego. Mnożąc obie strony nierówności przez $-1$ i~dodając jedynki, mamy
\[
    1-\Pr(X<k) \le 1-\sum_{i=0}^{k-1}b(i;n,p),
\]
skąd dostajemy (przy wykorzystaniu wzoru~C.35)
\[
    \Pr(X\ge k) \le \sum_{i=k}^nb(i;n,p).
\]
Niech $X'$ będzie zmienną losową oznaczającą liczbę sukcesów w~serii prób Bernoulliego, każda o~prawdopodobieństwie sukcesu równym $p$. Powyższa nierówność przyjmuje zatem postać
\[
    \Pr(X'\ge k) \ge \Pr(X\ge k),
\]
którą dowodzimy, stosując \refExercise{C.4-9} po przyjęciu $p_i'=p$ dla każdego $i=1$, 2,~\dots,~$n$.

\exercise %C.4-9
Niech $S$ będzie przestrzenią zdarzeń będących ciągami \compound{$n$}{elementowymi} o~wyrazach ze zbioru $\{{\scriptstyle\rm S},{\scriptstyle\rm P}\}$, gdzie ${\scriptstyle\rm S}$ oznacza sukces, a~${\scriptstyle\rm P}$ -- porażkę. Ciągowi prób $A$ odpowiada zatem pewien ciąg $s\in S$, dla którego $X(s)$ oznacza liczbę wystąpień ${\scriptstyle\rm S}$ w~$s$. Utwórzmy teraz nowy ciąg prób $A'$ poprzez doświadczenie na próbach z~ciągu $A$. Jeśli \compound{$i$}{ta} próba w~ciągu $A$ zakończyła się sukcesem, co oznaczymy przez $A_i$, to przyjmiemy, że sukces pada również w~\compound{$i$}{tej} próbie ciągu $A'$, co z~kolei oznaczymy przez $A_i'$. W~przeciwnym przypadku będziemy generować sukces (zdarzenie $B_i$) z~prawdopodobieństwem $r_i$, które teraz wyznaczymy. Ponieważ szanse otrzymania sukcesu w~\compound{$i$}{tej} próbie w~ciągu $A'$ wynoszą $p_i'$, a~zdarzenia $B_i$ oraz $\overline{A_i}$ są niezależne, to stąd mamy
\[
    p_i' = \Pr(A_i') = \Pr(A_i)+\Pr(B_i\mid\overline{A_i}) = p_i+r_i,
\]
czyli zdarzenie $B_i$ powinno zachodzić z~prawdopodobieństwem $r_i=p_i'-p_i$.

Operując na tej samej przestrzeni $S$, przyjmujemy, że $X'(s)$ przyjmuje liczbę sukcesów w~serii $n$ prób otrzymanych powyższą procedurą z~ciągu $A$ na podstawie przyjmowanych sukcesów opisanych przez $s$. Dla dowolnego zdarzenia $s\in S$ oczywistym jest, że przy takiej konstrukcji ciągu $A'$ nie zdarzy się, aby w~jego próbach było sumarycznie mniej sukcesów niż w~początkowym ciągu prób $A$, to znaczy $X'(s)\ge X(s)$. Korzystając teraz z~wyniku \refExercise{C.3-7}, otrzymujemy żądany wynik.

\subchapter{Krańce rozkładu dwumianowego}

\exercise %C.5-1
Zanim przejdziemy do porównywania prawdopodobieństw, udowodnimy pomocniczą nierówność
\[
    \binom{4n}{n-1} > 8^n
\]
dla $n\ge20$, stosując indukcję matematyczną. Przypadek bazowy indukcji można zweryfikować, obliczając wartości obu stron nierówności. Załóżmy teraz, że $n>20$ i~skorzystajmy z~\refExercise{C.1-5}, a~następnie trzykrotnie z~\refExercise{C.1-6}:
\[
    \binom{4n}{n-1} = \frac{4n}{n-1}\cdot\binom{4n-1}{n-2} = \frac{4n}{n-1}\cdot\frac{4n-1}{3n+1}\cdot\frac{4n-2}{3n}\cdot\frac{4n-3}{3n-1}\cdot\binom{4(n-1)}{n-2}.
\]
Na mocy założenia indukcyjnego zachodzi
\[
    \binom{4n}{n-1} > \frac{(4n)(4n-1)(4n-2)(4n-3)}{(n-1)(3n+1)(3n)(3n-1)}\cdot8^{n-1} = \frac{(4n-1)(2n-1)(4n-3)}{3(n-1)(3n+1)(3n-1)}\cdot8^n.
\]
Aby dowieść tezy, musimy jeszcze pokazać, że
\[
    \frac{(4n-1)(2n-1)(4n-3)}{3(n-1)(3n+1)(3n-1)} \ge 1.
\]
Okazuje się, że nierówność ta spełniona jest dla każdego $n\ge2$, więc w~szczególności dla $n\ge20$.

Zdefiniujmy teraz $X$ i~$Y$ jako zmienne losowe przyjmujące liczby uzyskanych orłów, odpowiednio, w~pierwszym i~w~drugim doświadczeniu. Mamy zatem
\begin{align*}
	\Pr(X=0) &= b(0;n,1/2) = \binom{n}{0}\biggl(\frac{1}{2}\biggr)^0\biggl(\frac{1}{2}\biggr)^n = \biggl(\frac{1}{2}\biggr)^n, \\
	\Pr(Y<n) &= \sum_{i=0}^{n-1}b(i;4n,1/2) = \sum_{i=0}^{n-1}\binom{4n}{i}\biggl(\frac{1}{2}\biggr)^i\biggl(\frac{1}{2}\biggr)^{4n-i} = \biggl(\frac{1}{2}\biggr)^{4n}\sum_{i=0}^{n-1}\binom{4n}{i}.
\end{align*}
Dzięki uprzednio udowodnionej nierówności pokazujemy, że jeśli $n\ge20$, to
\[
    \frac{\Pr(Y<n)}{\Pr(X=0)} = \frac{\bigl(\frac{1}{2}\bigr)^{4n}\sum_{i=0}^{n-1}\binom{4n}{i}}{\bigl(\frac{1}{2}\bigr)^n} = \frac{\sum_{i=0}^{n-1}\binom{4n}{i}}{8^n} > \frac{\binom{4n}{n-1}}{8^n} > 1.
\]
Wartości obu prawdopodobieństw w~przypadku, gdy $n<20$, obliczamy bezpośrednio. Otrzymujemy ostatecznie, że $\Pr(X=0)>\Pr(Y<n)$, gdy $n\le17$, oraz $\Pr(X=0)<\Pr(Y<n)$ w~przeciwnym przypadku.

\exercise %C.5-2
\begin{proof}[Dowód wniosku~C.6]
	Ponieważ uzyskanie więcej niż $k$ sukcesów w~$n$ próbach jest równoważne uzyskaniu mniej niż $n-k$ porażek, to zachodzi $\Pr(X>k)=\Pr(Y<n-k)$, gdzie $Y$ jest zmienną losową oznaczającą liczbę porażek uzyskanych w~tym doświadczeniu. Mamy $np<k<n$, skąd $0<n-k<nq$, a~zatem możemy zastosować tw.~C.4 dla zmiennej $Y$, zamieniając z~sobą role sukcesu i~porażki, dzięki czemu uzyskujemy żądaną nierówność.
\end{proof}

\begin{proof}[Dowód wniosku~C.7]
	Analogicznie jak w~poprzednim dowodzie, potraktujmy prawdopodobieństwo uzyskania więcej niż $k$ sukcesów jako prawdopodobieństwo uzyskania mniej niż $n-k$ porażek. Ponieważ $(np+n)/2<k<n$, to $0<n-k<nq/2$ i~po zamianie sukcesu z~porażką stosujemy wniosek~C.5.
\end{proof}

\exercise %C.5-3
\note{W~rzeczywistości nierówność nie zachodzi dla podanego warunku. Powinno być\/ $0<k<\frac{a}{a+1}n$, bo tylko wtedy można zastosować twierdzenie~C.4. Błąd występuje także w~tekście oryginalnym.}

\noindent Niech $p=\frac{a}{a+1}$, skąd mamy, że $q=\frac{1}{a+1}$ oraz $a=\frac{p}{1-p}$. Zachodzi
\[
	\sum_{i=0}^{k-1}\binom{n}{i}a^i = \sum_{i=0}^{k-1}\binom{n}{i}\biggl(\frac{p}{1-p}\biggr)^i = \frac{\sum_{i=0}^{k-1}\binom{n}{i}p^i(1-p)^{n-i}}{(1-p)^n} = \frac{\sum_{i=0}^{k-1}b(i;n,p)}{q^n},
\]
zatem z~tw.~C.4 otrzymujemy
\[
	\sum_{i=0}^{k-1}\binom{n}{i}a^i < \frac{\frac{k}{a+1}}{\bigl(\frac{na}{a+1}-k\bigr)\bigl(\frac{1}{a+1}\bigr)^n}\,b(k;n,a/(a+1)) = (a+1)^n\frac{k}{na-k(a+1)}\,b(k;n,a/(a+1)).
\]

\exercise %C.5-4
Wykorzystując obserwację, że $\binom{n}{i}\ge1$ dla $0\le i\le n$ oraz tw.~C.4, mamy
\[
	\sum_{i=0}^{k-1}p^iq^{n-i} \le \sum_{i=0}^{k-1}\binom{n}{i}p^iq^{n-i} = \sum_{i=0}^{k-1}b(i;n,p) < \frac{kq}{np-k}\,b(k;n,p).
\]
Z~kolei na mocy lematu~C.1 dostajemy
\[
	\frac{kq}{np-k}\,b(k;n,p) \le \frac{kq}{np-k}\biggl(\frac{np}{k}\biggr)^k\biggl(\frac{nq}{n-k}\biggr)^{n-k}.
\]

\exercise %C.5-5
Dowód nierówności
\[
    \Pr(\mu-X\ge r) \le \biggl(\frac{(n-\mu)e}{r}\biggr)^r
\]
przeprowadzimy analogicznie, jak przebiegał dowód tw.~C.8. Dla pewnego $\alpha>0$ mamy
\[
	\Pr(\mu-X\ge r) = \Pr\bigl(e^{\alpha(\mu-X)}\ge e^{\alpha r}\bigr)
\]
a~z~nierówności Markowa dostajemy
\[
	\Pr\bigl(e^{\alpha(\mu-X)}\ge e^{\alpha r}\bigr) \le \E\bigl(e^{\alpha(\mu-X)}\bigr)e^{-\alpha r}.
\]
Niech $X_i$, dla $i=1$, 2,~\dots,~$n$, będzie zmienną losową przyjmującą~1, jeśli wynikiem \compound{$i$}{tej} próby Bernoulliego jest sukces i~0 w~przeciwnym przypadku. Wtedy
\[
	X = \sum_{i=1}^nX_i \quad\text{oraz}\quad \mu-X = \sum_{i=1}^n(p_i-X_i),
\]
otrzymujemy zatem
\[
	\E\bigl(e^{\alpha(\mu-X)}\bigr) = \E\biggl(\prod_{i=1}^ne^{\alpha(p_i-X_i)}\biggr) = \prod_{i=1}^n\E\bigl(e^{\alpha(p_i-X_i)}\bigr),
\]
co wynika na mocy wzoru~(C.23), gdyż z~wzajemnej niezależności zmiennych losowych $X_i$ wynika wzajemna niezależność zmiennych losowych $e^{\alpha(p_i-X_i)}$ (\refExercise{C.3-5}). Z~definicji wartości oczekiwanej mamy
\[
	\E\bigl(e^{\alpha(p_i-X_i)}\bigr) = e^{\alpha(p_i-1)}p_i+e^{\alpha(p_i-0)}q_i = p_ie^{-\alpha q_i}+q_ie^{\alpha p_i} \le q_ie^{\alpha}+1 \le \exp(q_ie^\alpha),
\]
skąd zachodzi
\[
	\E\bigl(e^{\alpha(\mu-X)}\bigr) \le \prod_{i=1}^n\exp(q_ie^\alpha) = \exp\biggl(\sum_{i=1}^nq_ie^\alpha\biggr) = \exp((n-\mu)e^\alpha),
\]
bo $\sum_{i=1}^nq_i=\sum_{i=1}^n1-\sum_{i=1}^np_i=n-\mu$. Wracając do oszacowania prawdopodobieństwa, dostajemy
\[
	\Pr(\mu-X\ge r) \le \exp((n-\mu)e^\alpha-\alpha r).
\]
Prawą stronę nierówności~(C.45) minimalizuje $\alpha=\ln(r/\mu)$ (\refExercise{C.5-7}). W~naszym przypadku po prawej stronie znaku nierówności zamiast $\mu$ mamy $n-\mu$, a~więc powyższą nierówność będzie minimalizować $\alpha=\ln(r/(n-\mu))$. Dostajemy ostatecznie
\begin{align*}
	\Pr(\mu-X\ge r) &\le \exp\biggl((n-\mu)e^{\ln\frac{r}{n-\mu}}-r\ln\frac{r}{n-\mu}\biggr) \\
	&= \exp\biggl(r-r\ln\frac{r}{n-\mu}\biggr) \\
	&= \frac{e^r}{\bigl(\frac{r}{n-\mu}\bigr)^r} \\
	&= \biggl(\frac{(n-\mu)e}{r}\biggr)^r.
\end{align*}

W~dowodzie nierówności z~drugiej części zadania zauważmy, że $\mu=\E(X)=np$ i~na mocy nierówności udowodnionej w~pierwszej części, dostajemy
\[
    \Pr(np-X\ge r) \le \biggl(\frac{(n-np)e}{r}\biggr)^r = \biggl(\frac{nqe}{r}\biggr)^r.
\]

\exercise %C.5-6
\note{W treści zadania w~polskim tłumaczeniu występuje błąd. Wskazówka podpowiada, aby udowodnić pewną nierówność, jednak jej prawa strona powinna mieć postać\/ $e^{\alpha^2\!/2}$.}

\noindent\textsf{\textbf{Lemat.}} \textit{Dla dowolnych $\alpha>0$,~\/$p$,~\/$q\ge0$ spełniających\/ $p+q=1$ prawdziwa jest nierówność}
	\[
		pe^{\alpha q}+qe^{-\alpha p} \le e^{\alpha^2\!/2}.
	\]
\begin{proof}
	Przekształćmy nierówność do alternatywnej postaci:
	\begin{align*}
		pe^{\alpha q}+qe^{-\alpha p} &\le e^{\alpha^2\!/2} \\
		pe^{\alpha(1-p)}+(1-p)e^{-\alpha p} &\le e^{\alpha^2\!/2} \\
		pe^\alpha-p+1 &\le e^{\alpha^2\!/2+\alpha p} \\
		\ln(pe^\alpha-p+1) &\le \alpha^2\!/2+\alpha p \\
		\ln(pe^\alpha-p+1)-\alpha^2\!/2-\alpha p &\le 0.
	\end{align*}
	Ustalmy teraz $p$ z~przedziału $[0,1]$ i~oznaczmy wyrażenie po lewej stronie znaku ostatniej nierówności jako funkcję $f_p$ zmiennej $\alpha$. Zauważmy, że granicą prawostronną tej funkcji w~punkcie~0 dla dowolnego $0\le p\le1$ jest~0. Udowodnimy, że wraz ze wzrostem $\alpha$ funkcja $f_p$ maleje, co będzie oznaczać, że przyjmuje ona wartości niedodatnie i~uzasadni nierówność. Obliczmy w~tym celu pochodną $f_p$:
	\[
	    \frac{df_p}{d\alpha}(\alpha) = \frac{pe^\alpha}{pe^\alpha-p+1}-\alpha-p = \frac{pe^\alpha(1-\alpha-p)+(\alpha+p)(p-1)}{pe^\alpha-p+1}.
	\]
	Mianownik ostatniego ułamka jest dodatni. Wystarczy wykazać, że licznik jest ujemny dla każdych $\alpha>0$, $0\le p\le1$, czyli, równoważnie,
	\[
	    (\alpha+p)(p-1) < pe^\alpha(\alpha+p-1).
	\]

	Rozważmy dwa przypadki w~zależności od znaku wyrażenia $\alpha+p-1$. Jeśli $\alpha+p-1\ge0$, to $\alpha\ge1-p$. Zachodzi
	\[
	    (\alpha+p)(p-1)-p(1+\alpha)(\alpha+p-1) = -\alpha(\alpha p+p^2-p+1) < 0.
	\]
	Korzystając z~wzoru~(3.12) i~z~tego, że $\alpha>0$, mamy $1+\alpha<e^\alpha$, zatem
	\[
	    (\alpha+p)(p-1) < p(1+\alpha)(\alpha+p-1) < pe^\alpha(\alpha+p-1).
	\]
	W~drugim przypadku, czyli gdy $\alpha+p-1<0$, mamy $0<\alpha<1-p$ oraz
	\[
	    (\alpha+p)(p-1)-p(1+\alpha+\alpha^2)(\alpha+p-1) = -\alpha(\alpha^2p+\alpha p^2+p^2-p+1) < 0.
	\]
	Wykorzystując ponownie wzór~(3.12), dostajemy $e^\alpha<1+\alpha+\alpha^2$, więc
	\[
		(\alpha+p)(p-1) < p(1+\alpha+\alpha^2)(\alpha+p-1) < pe^\alpha(\alpha+p-1).
	\]

	Otrzymaliśmy ostatecznie, że pochodna funkcji $f_p$, gdzie $0\le p\le1$, jest ujemna dla każdego $\alpha>0$. A~zatem funkcja $f_p$ jest malejąca w~całej dziedzinie.
\end{proof}

Początek głównego rozumowania prowadzimy identycznie jak w~dowodzie twierdzenia~C.8:
\[
	\Pr(X-\mu\ge r) = \Pr\bigl(e^{\alpha(X-\mu)}\ge e^{\alpha r}\bigr) \le \E\bigl(e^{\alpha(X-\mu)}\bigr)e^{-\alpha r}.
\]
Następnie przy tych samych oznaczeniach z~oryginalnego dowodu, zachodzi
\[
	\E\bigl(e^{\alpha(X-\mu)}\bigr) = \prod_{i=1}^n\E\bigl(e^{\alpha(X_i-p_i)}\bigr).
\]
Wykorzystując udowodnioną w~lemacie nierówność dla $p=p_i$ oraz $q=q_i$, otrzymujemy
\[
	\E\bigl(e^{\alpha(X_i-p_i)}\bigr) = e^{\alpha(1-p_i)}p_i+e^{\alpha(0-p_i)}q_i = p_ie^{\alpha q_i}+q_ie^{-\alpha p_i} \le e^{\alpha^2\!/2}
\]
i~dalej mamy
\[
	\E\bigl(e^{\alpha(X-\mu)}\bigr) = \prod_{i=1}^n\E\bigl(e^{\alpha(X_i-p_i)}\bigr) \le \prod_{i=1}^ne^{\alpha^2\!/2} = e^{\alpha^2n/2}.
\]
Wobec tego
\[
	\Pr(X-\mu\ge r) \le \E\bigl(e^{\alpha(X-\mu)}\bigr)e^{-\alpha r} \le \exp(\alpha^2n/2-\alpha r).
\]
Należy teraz wybrać taką wartość $\alpha$, która minimalizuje prawą stronę powyższej nierówności. Argumentem funkcji wykładniczej jest funkcja kwadratowa zmiennej $\alpha$, w~prosty sposób można więc sprawdzić, że osiąga ona minimum dla argumentu $\alpha=r/n$. Dostajemy ostatecznie
\[
	\Pr(X-\mu\ge r) \le \exp\bigl((r/n)^2n/2-(r/n)r\bigr) = e^{-r^2\!/{2n}}.
\]

\exercise %C.5-7
Potraktujmy wyrażenie jako funkcję $f$ zmiennej $\alpha$,
\[
	f(\alpha) = \exp(\mu e^\alpha-\alpha r).
\]
W~celu wyznaczenia jej minimum obliczmy pierwszą i~drugą pochodną:
\begin{align*}
	\frac{df}{d\alpha}(\alpha) &= (\mu e^\alpha-r)\exp(\mu e^\alpha-\alpha r), \\
	\frac{d^2\!f}{d\alpha^2}(\alpha) &= \bigl(\mu e^\alpha+(\mu e^\alpha-r)^2\bigr)\exp(\mu e^\alpha-\alpha r).
\end{align*}
Pierwsza pochodna zeruje się dla $\alpha=\ln(r/\mu)$. Po obliczeniu wartości drugiej pochodnej w~tym punkcie dostajemy
\[
	\frac{d^2\!f}{d\alpha^2}(\alpha) = r\exp(r-r\ln(r/\mu)) > 0,
\]
ponieważ $\exp(x)$ jest dodatnie w~swojej dziedzinie oraz $r>\mu\ge0$, a~zatem w~punkcie $\alpha=\ln(r/\mu)$ istnieje minimum funkcji~$f$.

\problems

\problem{Kule i~urny} %C-1

\subproblem %C-1(a)
Każda kula trafia do jednej z~$b$ urn. Jest $b$ sposobów umieszczenia pierwszej kuli, a~na każdy z~nich przypada $b$ sposobów umieszczenia drugiej kuli, itd. Jest zatem $b^n$ sposobów rozmieszczenia $n$ różnych kul w~$b$ różnych urnach.

\subproblem %C-1(b)
Ponieważ dysponujemy $n$ rozróżnialnymi kulami oraz $b$ nierozróżnialnymi urnami, to nasz problem jest równoważny policzeniu wszystkich możliwych ciągów $n$ różnych kul i~$b-1$ identycznych patyków. Patyki dzielą ciąg kul na spójne podciągi, z~których każdy odpowiada ciągowi kul w~kolejnej urnie, reprezentując pewne uporządkowanie kul względem siebie wewnątrz urny.

Wszystkich takich ciągów jest $(b+n-1)!$, ale ponieważ nie rozróżniamy urn, to musimy podzielić ich liczbę przez liczbę możliwych rozmieszczeń urn między sobą, czyli $(b-1)!$. Istnieje zatem $\frac{(b+n-1)!}{(b-1)!}$ różnych rozmieszczeń kul w~urnach.

\subproblem %C-1(c)
Sytuacja jest podobna jak w~punkcie~(b) z~tą różnicą, że nie rozróżniamy kul między sobą, a~więc również każda permutacja $n$ kul między sobą opisuje ten sam sposób rozmieszczenia kul w~urnach. Mamy zatem $\frac{(b+n-1)!}{n!\,(b-1)!}=\binom{b+n-1}{n}$ możliwości rozmieszczenia kul.

\subproblem %C-1(d)
Zakładając, że $n\le b$, wybieramy spośród $b$ urn $n$ takich, które będą zawierać po jednej kuli. Jest $\binom{b}{n}$ sposobów ich wyboru.

\subproblem %C-1(e)
Zakładamy, że $n\ge b$. Najpierw umieszczamy $b$ kul po jednej w~każdej urnie tak, aby żadna urna nie była pusta. Na mocy punktu~(c) pozostałe $n-b$ kul możemy umieścić w~$b$ urnach na $\binom{b+(n-b)-1}{n-b}=\binom{n-1}{n-b}=\binom{n-1}{b-1}$ sposobów.

\endinput
