\chapter{Zliczanie i prawdopodobieństwo}

\subchapter{Zliczanie}

\exercise{} %C.1-1
Załóżmy, że nie rozważamy słowa pustego i że $1\le k\le n$. Pierwsze \twoparts{$k$}{podsłowo} zajmuje w~\twoparts{$n$}{słowie} pozycje 1, 2,~\dots,~$k$, drugie -- 2, 3,~\dots,~$k+1$ itd. Ostatnie \twoparts{$k$}{podsłowo} leży na pozycjach $n-k+1$, $n-k+2$,~\dots,~$n$. Jest zatem
\[
	n-k+1
\]
\twoparts{$k$}{podsłów} \twoparts{$n$}{słowa}.

By obliczyć łączną ilość podsłów \twoparts{$n$}{słowa}, należy zsumować liczby \twoparts{$k$}{podsłów} po wszystkich $1\le k\le n$, co daje
\[
	\sum_{k=1}^n(n-k+1) = \sum_{i=1}^ni = \frac{n(n+1)}{2}.
\]

\exercise{} %C.1-2
Niech $X=\{0,1,\dots,2^n-1\}$ i $Y=\{0,1,\dots,2^m-1\}$ będą zbiorami liczb, odpowiednio, \twoparts{$n$}{bitowych} i \twoparts{$m$}{bitowych}. Zauważmy, że funkcji logicznych o $n$ wejściach i $m$ wyjściach będzie tyle samo, co funkcji $f\colon X\to Y$.

Zagadnienie sprowadza się zatem do pytania o liczbę wszystkich ciągów $y_1,\dots,y_{2^n}$ o wyrazach ze zbioru \twoparts{$2^m$}{elementowego} $Y$. Każdy wyraz $y_i$ możemy wybrać na $2^m$ sposobów, co daje $(2^m)^{2^n} = 2^{m2^n}$ możliwości wyboru ciągu $y_1,\dots,y_{2^n}$. Jest zatem $2^{m2^n}$ funkcji logicznych o $n$ wejściach i $m$ wyjściach, a~stąd $2^{2^n}$ funkcji logicznych o $n$ wejściach i~1 wyjściu.

\exercise{} %C.1-3
Niech $S_n$ oznacza szukaną liczbę sposobów ustawienia $n$ osób przy stole. Jeden ze sposobów jest nierozróżnialny z $n-1$ innymi, dzięki temu, że stół jest okrągły, a osoby mogą przesuwać się miejscami nie zmieniając kolejności wzajemnego ustawienia. Ponadto, jest $n!$ możliwych permutacji osób, zatem $nS_n$ jest równe $n!$. Mamy zatem
\[
	S_n = \frac{n!}{n} = (n-1)!.
\]

\exercise{} %C.1-4
By wybrać trzy liczby ze zbioru $\{1,2,\dots,100\}$, które w sumie dadzą liczbę parzystą, można postąpić na dwa sposoby:
\begin{itemize}
	\item wybrać 3 liczby parzyste,
	\item wybrać 2 liczby nieparzyste i 1 liczbę parzystą.
\end{itemize}
W pierwszym przypadku możemy to zrobić na $\binom{50}{3}$ sposobów, a w drugim na $\binom{50}{2}\binom{50}{1}$ sposobów. Łączna liczba możliwości wyboru takich liczb wynosi zatem
\[
	\binom{50}{3}+\binom{50}{2}\binom{50}{1} = 80850.
\]

\exercise{} %C.1-5
\[
	\binom{n}{k} = \frac{n!}{k!\,(n-k)!} = \frac{n}{k}\cdot\frac{(n-1)!}{(k-1)!\,(n-k)!} = \frac{n}{k}\binom{n-1}{k-1}.
\]

\exercise{} %C.1-6
\[
	\binom{n}{k} = \frac{n!}{k!\,(n-k)!} = \frac{n}{n-k}\cdot\frac{(n-1)!}{k!\,(n-k-1)!} = \frac{n}{n-k}\binom{n-1}{k}.
\]

\exercise{} %C.1-7
Załóżmy, że wybieramy pewien \twoparts{$k$}{podzbiór} z \twoparts{$n$}{elementowego} zbioru $S$, co można zrobić na $\binom{n}{k}$ sposobów. Wyróżnijmy pewien element z $S$. Jeśli nie został on wybrany w \twoparts{$k$}{podzbiorze}, to istnieje $\binom{n-1}{k}$ możliwości wyboru $k$ elementów spośród $n-1$ pozostałych ze zbioru $S$. Jeżeli jednak wyróżniony element należy do wybranego podzbioru, to z $n-1$ pozostałych elementów należy wybrać jeszcze $k-1$, co można wykonać na $\binom{n-1}{k-1}$ sposobów. Otrzymujemy zatem
\[
	\binom{n}{k} = \binom{n-1}{k}+\binom{n-1}{k-1}.
\]

\exercise{} %C.1-8
Kilka początkowych wierszy trójkąta Pascala:
\[
	\begin{array}{ccccccccccccc}
		&&&&&& 1 \\
		&&&&& 1 && 1 \\
		&&&& 1 && 2 && 1 \\
		&&& 1 && 3 && 3 && 1 \\
		&& 1 && 4 && 6 && 4 && 1 \\
		& 1 && 5 && 10 && 10 && 5 && 1 \\
		1 && 6 && 15 && 20 && 15 && 6 && 1
	\end{array}
\]
W pierwszym wierszu mamy tylko jeden element, $\binom{0}{0}=1$. Drugi wiersz zawiera $\binom{1}{0}=1$ i $\binom{1}{1}=1$. Kolejne wiersze mają jedynki na końcach, lewym i prawym, zaś elementy wewnętrzne powstają przez zsumowanie dwóch liczb z poprzedniego wiersza znajdujących się bezpośrednio nad wyliczanym elementem.

\exercise{} %C.1-9
Z tożsamości~(A.1) mamy
\[
	\sum_{i=1}^ni = \frac{n(n+1)}{2},
\]
a z definicji współczynnika dwumianowego
\[
	\binom{n+1}{2} = \frac{(n+1)!}{2!\,(n-1)!} = \frac{n(n+1)}{2}.
\]
Prawe strony powyższych równań są identyczne, czego należało dowieść.

\exercise{} %C.1-10
Potraktujmy współczynniki dwumianowe jako funkcję $b_n(k)=\binom{n}{k}$ dla $0\le k\le n$ i sprawdźmy, dla jakich $k$ wartość $b_n(k)$ jest największa.
\begin{align*}
	b_n(k+1) &> b_n(k) \\
	\binom{n}{k+1} &> \binom{n}{k} \\
	\frac{n!}{(k+1)!\,(n-k-1)!} &> \frac{n!}{k!\,(n-k)!} \\
	\frac{(n-k)!}{(n-k-1)!} &> \frac{(k+1)!}{k!} \\
	n-k &> k-1 \\
	k &< \frac{n-1}{2},
\end{align*}
a zatem $b_n(k)$ jest funkcją rosnącą o ile $k<(n-1)/2$, z największą wartością osiąganą dla $k=\lfloor(n+1)/2\rfloor$. W zależności od parzystości $n$, liczba ta jest równa $\lfloor n/2\rfloor$ lub $\lceil n/2\rceil$.

\exercise{} %C.1-11
Dla $n$, $j$,~$k\ge0$ takich, że $j+k\le n$ dowodzimy
\begin{align*}
	\binom{n}{j+k} &\le \binom{n}{j}\binom{n-j}{k} \\
	\frac{n!}{(j+k)!\,(n-j-k)!} &\le \frac{n!}{j!\,(n-j)!}\cdot\frac{(n-j)!}{k!\,(n-j-k)!} \\
	\frac{1}{(j+k)!} &\le \frac{1}{j!\,k!} \\
	j!\,k! &\le (j+k)! \\
	j!\,k! &\le j!\cdot\prod_{i=1}^k(j+i) \\
	\prod_{i=1}^ki &\le \prod_{i=1}^k(j+i).
\end{align*}
Iloczyn $k$ liczb całkowitych od 1 do $k$ jest oczywiście niewiększy od iloczynu $k$ liczb całkowitych od $j+1$ do $j+k$, zatem ostatnia nierówność jest prawdziwa. Równość zachodzi dla przypadków, gdy $j=0$ lub $j=1$, $k=0$.

Lewą stronę nierówności można zinterpretować jako liczbę możliwych wyborów $j+k$ przedmiotów spośród zbioru \twoparts{$n$}{elementowego}, prawą zaś jako liczbę możliwych sposobów wyboru najpierw $j$ przedmiotów spośród $n$, a następnie $k$ przedmiotów spośród $n-j$ pozostawionych po pierwszym wyborze.

Załóżmy, że $A=\{a_1,a_2,\dots,a_{j+k}\}$ jest zbiorem wybranych elementów. Jest tylko 1 sposób wyboru zadanego zbioru $A$ przy pierwszej strategii i o wiele więcej, jeśli zastosuje się drugie podejście. Można mianowicie dowolnie podzielić elementy z $A$ na $j$ takich, które będą wybierane w pierwszym kroku i $k$ takich, które wybierzemy w drugim kroku.

\exercise{} %C.1-12
Przypadek dla $k=0$ sprawdzamy w pierwszym kroku indukcyjnym i stwierdzamy, że zachodzi. Przyjmijmy, że $k\ge0$ i załóżmy, że prawdą jest
\[
	\binom{n}{k} \le \frac{n^n}{k^k(n-k)^{n-k}}.
\]
Mamy teraz z założenia indukcyjnego
\[
	\binom{n}{k+1} = \frac{n-k}{k+1}\binom{n}{k} \le \frac{n^n}{(k+1)k^k(n-k)^{n-k-1}}.
\]
Zbadajmy następującą nierówność:
\begin{align}
	\frac{n^n}{(k+1)k^k(n-k)^{n-k-1}} &\le \frac{n^n}{(k+1)^{k+1}(n-k-1)^{n-k-1}} \nonumber \\[1mm]
	(k+1)^k(n-k-1)^{n-k-1} &\le k^k(n-k)^{n-k-1} \nonumber \\[1mm]
	\left(\frac{k+1}{k}\right)^k &\le \left(\frac{n-k}{n-k-1}\right)^{n-k-1} \nonumber \\
	\left(1+\frac{1}{k}\right)^k &\le \left(1+\frac{1}{n-k-1}\right)^{n-k-1}. \label{eq:C.1-13}
\end{align}
Ciąg $e_n={(1+1/n)}^n$ jest rosnący, a stąd dostajemy
\begin{align*}
	k &\le n-k-1 \\
	k &\le (n-1)/2.
\end{align*}
Nierówność (\ref{eq:C.1-13}) pozostaje zatem prawdziwa i twierdzenie zachodzi dla wszystkich $k<n/2$. Z wzoru~(C.3) mamy $\binom{n}{k}=\binom{n}{n-k}$ i gdy $k>n/2$ sprowadzamy dowód do pokazania, że
\[
	\binom{n}{n-k} \le \frac{n^n}{k^k(n-k)^{n-k}},
\]
ponieważ wtedy $0\le n-k<n/2$ i symetrycznie udowodnimy wzór dla $k>n/2$.

Pozostaje jeszcze przypadek, gdy $k=n/2$. Wtedy prawa strona dowodzonej tożsamości przyjmuje postać
\[
	\frac{n^n}{(n/2)^{n/2}(n-n/2)^{n-n/2}} = \left(\frac{n}{n/2}\right)^n = 2^n,
\]
korzystając więc z oszacowania górnego na $\binom{n}{k}$ wykorzystującego funkcję entropii $H(\lambda)$, mamy
\[
	\binom{n}{\lambda n} = \binom{n}{n/2} \le 2^{nH(1/2)} = 2^n,
\]
a więc także w tym przypadku tożsamość jest spełniona.

\exercise{} %C.1-13
Wykorzystując wzór Stirlinga mamy
\begin{align*}
	\binom{2n}{n} &= \frac{(2n)!}{(n!)^2} \\
	&= \frac{\sqrt{4\pi n}\left(\frac{2n}{e}\right)^{2n}\bigl(1+O(1/n)\bigr)}{2\pi n\left(\frac{n}{e}\right)^{2n}\bigl(1+O(1/n)\bigr)^2} \\[1mm]
	&= \frac{2^{2n}\sqrt{\pi n}}{\pi n\bigl(1+O(1/n)\bigr)} \\[1mm]
	&= \frac{2^{2n}}{\sqrt{\pi n}}\bigl(1+O(1/n)\bigr).
\end{align*}
W wyprowadzeniu oszacowania skorzystano z tożsamości
\[
	\frac{1}{\bigl(1+O(1/n)\bigr)} \equiv \bigl(1+O(1/n)\bigr).
\]

\exercise{} %C.1-14
Niech $g$ i $h$ będą surjekcjami w $(0,1)$ i niech będą różniczkowalne w~tym przedziale. Dla $h(x)=-x\lg x$ mamy
\[
	\frac{dh(g(x))}{dx} = -\frac{dg(x)}{dx}\bigl(\lg g(x)+\lg e\bigr).
\]
Liczymy pierwszą pochodną funkcji entropii $H(\lambda)=h(\lambda)+h(1-\lambda)$,
\begin{align*}
	\frac{dH(\lambda)}{d\lambda} &= \frac{dh(\lambda)}{d\lambda}+\frac{dh(1-\lambda)}{d\lambda} \\
	&= -\lg\lambda-\lg e+\lg(1-\lambda)+\lg e \\
	&= \lg(1-\lambda)-\lg\lambda.
\end{align*}
Zbadajmy gdzie $H$ posiada ekstremum przyrównując pierwszą pochodną do 0:
\begin{align*}
	\frac{dH(\lambda)}{d\lambda} &= 0 \\
	\lg(1-\lambda) &= \lg\lambda \\
	1-\lambda &= \lambda \\
	\lambda &= 1/2.
\end{align*}
Należy jeszcze zbadać znak drugiej pochodnej w punkcie $\lambda=1/2$.
\[
	\frac{d^2H(\lambda)}{d\lambda^2} = -\frac{\lg e}{\lambda(1-\lambda)} < 0,
\]
a więc w punkcie $\lambda=1/2$ binarna funkcja entropii $H$ osiąga maksimum wynoszące $H(1/2)=1$.

\exercise{} %C.1-15
Dla $n=0$ równość jest prawdziwa, więc rozważmy sumę od~1 do $n$,
\begin{align*}
	\sum_{k=1}^n\binom{n}{k}k &= \sum_{k=1}^n\binom{n-1}{k-1}n \\
	&= n\sum_{k=0}^{n-1}\binom{n-1}{k} \\
	&= n2^{n-1}.
\end{align*}
Ostatnia równość zachodzi z wzoru~(C.4) dla $x=y=1$.

\subchapter{Prawdopodobieństwo}

\exercise{} %C.2-1
Utwórzmy skończoną lub przeliczalną rodzinę zdarzeń,
\begin{align*}
	C_1 &= A_1 \\
	C_2 &= A_2\setminus A_1 \\
	C_3 &= A_3\setminus(A_1\cup A_2) \\
	C_4 &= A_4\setminus(A_1\cup A_2\cup A_3) \\
	& \,\,\vdots \\
	C_k &= A_k\setminus\bigcup_{i=1}^{k-1}A_i \\
	& \,\,\vdots
\end{align*}
Korzystając z tożsamości
\[
	\bigcup_iA_i = \bigcup_iC_i
\]
oraz z tego, że zdarzenia $C_1$,~$C_2$,~\dots wzajemnie się wykluczają, otrzymujemy
\[
	\Pr\biggl(\bigcup_iA_i\biggr) = \Pr\biggl(\bigcup_iC_i\biggr) = \sum_i\Pr(C_i).
\]
Ponieważ $\Pr(C_i)\le\Pr(A_i)$ dla każdego $i$, to prawdą jest, że
\[
	\Pr\biggl(\bigcup_iA_i\biggr) \le \sum_i\Pr(A_i).
\]

\exercise{} %C.2-2
Zdefiniujmy \twoparts{3}{słowo} nad alfabetem $\{{\scriptstyle\rm O},{\scriptstyle\rm R}\}$ w następujący sposób. Pierwszy symbol tego słowa oznacza wynik rzutu monetą profesora Rosencrantza, drugi symbol to wynik rzutu pierwszą monetą profesora Guildensterna, a~trzeci to wynik rzutu jego drugą monetą, przy czym $\scriptstyle\rm O$ oznacza wyrzucenie orła, a $\scriptstyle\rm R$ -- wyrzucenie reszki. Tworzymy przestrzeń zdarzeń elementarnych
\[
	S = \{{\scriptstyle\rm OOO},{\scriptstyle\rm OOR},{\scriptstyle\rm ORO},{\scriptstyle\rm ORR},{\scriptstyle\rm ROO},{\scriptstyle\rm ROR},{\scriptstyle\rm RRO},{\scriptstyle\rm RRR}\}.
\]
Każde z tych zdarzeń zachodzi z prawdopodobieństwem równym $1/8$, w szczególności zdarzenie $\scriptstyle\rm ORR$, oznaczające wyrzucenie przez profesora Rosencrantza większej ilości orłów od rywala.

\exercise{} %C.2-3
Jeśli kolejno wyciągane karty mają mieć rosnące numery, to pierwsza z nich musi mieć numer od~1 do~8, druga -- od~2 do~9, a trzecia -- od~3 do~10. Oznaczmy zdarzenia: \\
\hspace*\parindent$A$ -- numer drugiej karty jest większy od numeru pierwszej karty, \\
\hspace*\parindent$B$ -- numer trzeciej karty jest większy od numeru drugiej karty. \\
Jeśli numer drugiej karty wynosi $k$, to liczba zdarzeń sprzyjających $A$ wynosi $k-1$, a sprzyjających $B$ jest równa $10-k$.

Mamy obliczyć $\Pr(A\cap B)$. Korzystając z reguły iloczynu dostajemy, że liczba zdarzeń sprzyjających $A\cap B$ wynosi $\sum_{k=2}^9(k-1)(10-k)=120$. Liczba możliwych sposobów wyboru trzech kart spośród dziesięciu wynosi $10!/7!=720$ (liczba wszystkich \twoparts{3}{permutacji} zbioru \twoparts{10}{elementowego}), dostajemy zatem wynik $\Pr(A\cap B)=120/720=1/6$.

\exercise{} %C.2-4
Ponieważ $a<b$, to $a/b<1$. Rozważmy część ułamkową binarnego rozwinięcia ilorazu $a/b$, które jest nieskończonym ciągiem zer i jedynek.

Będziemy rzucać monetą i tworzyć nowy ciąg zer i jedynek, w zależności od wyniku rzutu, dla orła przyjmując~1, a dla reszki~0. Rzucamy monetą dopóki tworzony przez nas ciąg jest równy pewnemu prefiksowi rozwinięcia binarnego $a/b$. W momencie gdy natrafimy na pierwszą różnicę, otrzymany ciąg traktujemy jako część ułamkową rozwinięcia binarnego pewnej liczby. Jeśli liczba ta jest mniejsza od $a/b$, to zwracamy orła, w przeciwnym przypadku -- reszkę.

Oczekiwana liczba rzutów monetą potrzebnych do wyznaczenia wyniku jest oczekiwaną liczbą rzutów aż do pierwszej różnicy w porównaniu z rozwinięciem binarnym $a/b$. Jeśli przyjmiemy, że sukcesem jest wynik rzutu monetą niepasujący do bieżącego elementu rozwinięcia $a/b$, to jego prawdopodobieństwo wynosi $p=1/2$. Liczba rzutów $n$ aż do pierwszego sukcesu jest zmienną losową $X$ o~rozkładzie geometrycznym, dla której
\[
	\Pr(X=n) = (1-p)^{n-1}p.
\]
Z wzoru (C.31) otrzymujemy
\[
	\E(X) = 1/p = 2.
\]
Widać zatem, że oczekiwana liczba rzutów monetą w opisanej procedurze jest $O(1)$.

\exercise{} %C.2-5
Korzystając z tożsamości $(A\cap B)\cup(\overline{A}\cap B)=B$ oraz z tego, że zdarzenia $A\cap B$ i $\overline{A}\cap B$ wykluczają się, otrzymujemy
\[
	\Pr(A\mid B)+\Pr(\overline{A}\mid B) = \frac{\Pr(A\cap B)}{\Pr(B)}+\frac{\Pr(\overline{A}\cap B)}{\Pr(B)} = \frac{\Pr(B)}{\Pr(B)} = 1.
\]

\exercise{} %C.2-6
Dowód przez indukcję względem liczby zdarzeń.

Dla $n=1$ dowód jest trywialny, załóżmy zatem, że $n\ge1$. Otrzymujemy
\begin{align*}
	\Pr\biggl(\bigcap_{i=1}^{n+1}A_i\biggr) &= \Pr\biggl(A_{n+1}\cap\bigcap_{i=1}^nA_i\biggr) \\
	&= \Pr\biggl(\bigcap_{i=1}^nA_i\biggr)\Pr\biggl(A_{n+1}\biggm|\bigcap_{i=1}^nA_i\biggr) \\
	&= \Pr(A_1)\Pr(A_2\mid A_1)\Pr(A_3\mid A_1\cap A_2)\dots\Pr\biggl(A_{n+1}\biggm|\bigcap_{i=1}^nA_i\biggr).
\end{align*}
Druga równość wynika z definicji prawdopodobieństwa warunkowego, a trzecia -- z założenia indukcyjnego.

\exercise{} %C.2-7
Niech $(\Omega,\mathcal{S},\Pr)$ będzie przestrzenią probabilistyczną, w której $\Omega=\{\omega_1,\omega_2,\dots,\omega_{n^2}\}$ stanowi zbiór zdarzeń elementarnych oraz $\Pr(\omega_i)=1/n^2$ dla każdego $i=1$, 2,~\dots,~$n^2$, jest funkcją prawdopodobieństwa. Pokażemy teraz jak skonstruować zdarzenia $A_1$, $A_2$,~\dots,~$A_n\in\mathcal{S}$, które spełniają warunek z treści zadania.

Utwórzmy je tak, aby dla każdych, parami różnych $1\le i_1,i_2,i_3\le n$, zachodziło
\begin{equation}
	|A_{i_1}\cap A_{i_2}| = 1 \quad\text{oraz}\quad |A_{i_1}\cap A_{i_2}\cap A_{i_3}| = 0. \label{eq:C.2-7}
\end{equation}
Ponieważ dysponujemy $n^2$ zdarzeniami elementarnymi, to wybierzmy spośród nich $n(n-1)/2$ różnych i umieśćmy kolejno w każdym możliwym iloczynie $A_{i_1}\cap A_{i_2}$ dla $i_1\ne i_2$. Każde zdarzenie z $\mathcal{S}$ będzie teraz zawierać po $n-1$ zdarzeń elementarnych, uzupełnijmy je zatem tak, aby były zbiorami \twoparts{$n$}{elementowymi}, zachowując jednocześnie warunek~(\ref{eq:C.2-7}). Jest to możliwe, ponieważ mamy $n(n+1)/2$ niewykorzystanych elementów zbioru $\Omega$.

Z opisanej konstrukcji zdarzeń $A_1$, $A_2$,~\dots,~$A_n$ wynika, że $\Pr(A_i)=1/n$ dla $i=1$, 2,~\dots,~$n$, a zatem
\[
	\Pr(A_{i_1}\cap A_{i_2}) = \frac{1}{n^2} = \Pr(A_{i_1})\Pr(A_{i_2}),
\]
dla wszystkich $1\le i_1<i_2\le n$ oraz
\[
	\Pr\biggl(\bigcap_{s=1}^kA_{i_s}\biggr) = 0 \ne \prod_{s=1}^k\Pr(A_{i_s}),
\]
dla każdego $k>2$.

Skonstruowaliśmy zatem zbiór $n$ zdarzeń, które są parami niezależne, ale żaden ich \twoparts{$k$}{pozdbiór} ($k>2$) nie jest wzajemnie niezależny.

\exercise{} %C.2-8
Rozważmy pewną grupę 400 osób, z~których 100 osiągnęło już wiek 50 lat. W młodszej grupie jest 20 programistów i 6 matematyków, przy czym tylko 3 osoby jednocześnie programują i zajmują się matematyką. W skład grupy osób starszych wchodzi 15 programistów oraz 10 matematyków, a 5 jednocześnie jest programistami i matematykami. Rysunek~\ref{fig:C.2-8} stanowi ilustrację przedstawionego opisu.
\begin{figure}[ht]
	\begin{center}
		\includegraphics{figc.1}
	\end{center}
	\caption{Programiści i matematycy} \label{fig:C.2-8}
\end{figure}

Spośród osób z tej grupy losowo wybieramy jedną. Oznaczmy następujące zdarzenia:
\begin{itemize}
	\item $A$ -- wybrano osobę w wieku powyżej 50 lat,
	\item $B$ -- wybrano programistę,
	\item $C$ -- wybrano matematyka.
\end{itemize}
Obliczamy ich prawdopodobieństwa:
\[
	\Pr(A) = \frac{1}{4}, \qquad \Pr(B) = \frac{1}{10}, \qquad \Pr(C) = \frac{1}{25}.
\]
Zauważmy, że zdarzenia $A$ i $B$ nie są niezależne, ponieważ
\[
	\Pr(A\cap B) = \frac{1}{20} \ne \frac{1}{40} = \Pr(A)\Pr(B).
\]
Są jednak warunkowo zależne od zdarzenia $C$:
\[
	\Pr(A\cap B\mid C) = \frac{5}{16} = \frac{5}{8}\cdot\frac{1}{2} = \Pr(A\mid C)\Pr(B\mid C).
\]

\exercise{} %C.2-9
Rozważmy zdarzenia, $A$ -- wybraliśmy zasłonę, za którą jest nagroda i zdarzenie do niego przeciwne $B$. Mamy $\Pr(A)=1/3$ i $\Pr(B)=2/3$. Obliczmy prawdopodobieństwo wygranej $W$ w zależności od podjętej decyzji po podniesieniu przez prowadzącego jednej z zasłon,
\[
	\Pr(W) = \Pr(W\mid A)\Pr(A)+\Pr(W\mid B)\Pr(B).
\]
W pierwszej strategii decydujemy się na pozostanie przy aktualnym wyborze, zatem ponieważ nie zmieniamy wybranej zasłony, mamy $\Pr(W\mid A)=1$ i~$\Pr(W\mid B)=0$, a więc wygramy z prawdopodobieństwem $\Pr(W)=1/3$. Jeśli teraz rozważymy drugą strategię, w której zmienimy zasłonę po ujawnieniu jednej z przegrywających, to będziemy mieć $\Pr(W\mid A)=0$ i $\Pr(W\mid B)=1$, czyli prawdopodobieństwo wygranej wynosi $\Pr(W)=2/3$. Widać zatem, że powinniśmy zdecydować się na zmianę zasłony.

\exercise{} %C.2-10
Niech $A_X$, $A_Y$ i $A_Z$ będą prawdopodobieństwami wyjścia na wolność, odpowiednio, więźnia $X$, $Y$ i $Z$. Przed rozmową ze strażnikiem, prawdopodobieństwo, że $X$ będzie wolny, wynosi $\Pr(A_X)=1/3$. Jeśli $X$ dostał informację, że $Y$ zostanie ścięty, to aby zobaczyć, czy zmienia to szanse $X$ na wolność, obliczmy $\Pr(A_X\mid\overline{A_Y})$. Z wzoru Bayesa~(C.21), mamy
\[
	\Pr(A_X\mid\overline{A_Y}) = \frac{\Pr(\overline{A_Y}\mid A_X)\Pr(A_X)}{\Pr(\overline{A_Y})}.
\]
Z kolei wiadomo, że
\begin{align*}
	\Pr(\overline{A_Y}) &= \Pr(\overline{A_Y}\mid A_X)\Pr(A_X)+\Pr(\overline{A_Y}\mid A_Y)\Pr(A_Y)+\Pr(\overline{A_Y}\mid A_Z)\Pr(A_Z) \\
	&= \Pr(\overline{A_Y}\cap A_X)+\Pr(\overline{A_Y}\cap A_Y)+\Pr(\overline{A_Y}\cap A_Z) \\
	&= 1/3+0+1/3 = 2/3,
\end{align*}
a zatem
\[
	\Pr(A_X\mid\overline{A_Y}) = \frac{1\cdot (1/3)}{2/3} = \frac{1}{2}>\frac{1}{3}.
\]
Wynika stąd, że szanse więźnia $X$ na wyjście na wolność zwiększyły się po rozmowie ze strażnikiem i teraz wynoszą~1/2.

\subchapter{Dyskretne zmienne losowe}

\exercise{} %C.3-1
Niech $X$ będzie zmienną losową oznaczającą sumę oczek na obu kostkach. Mamy
\begin{align*}
	\E(X) &= \sum_{x=2}^{12}x\Pr(X=x) \\
	&= 2\cdot\frac{1}{36}+3\cdot\frac{2}{36}+4\cdot\frac{3}{36}+5\cdot\frac{4}{36}+6\cdot\frac{5}{36}+7\cdot\frac{6}{36} \\
	&\quad +8\cdot\frac{5}{36}+9\cdot\frac{4}{36}+10\cdot\frac{3}{36}+11\cdot\frac{2}{36}+12\cdot\frac{1}{36} \\[1mm]
	&= 7.
\end{align*}
Niech teraz $Y$ będzie zmienną losową oznaczającą większą z liczb oczek na obu kostkach. Zachodzi
\begin{align*}
	\E(Y) &= \sum_{y=1}^{6}y\Pr(Y=y) \\
	&= 1\cdot\frac{1}{36}+2\cdot\frac{3}{36}+3\cdot\frac{5}{36}+4\cdot\frac{7}{36}+5\cdot\frac{9}{36}+6\cdot\frac{11}{36} \\[2mm]
	&\approx 4{,}47.
\end{align*}

\exercise{} %C.3-2
Niech $X$ będzie zmienną losową przyjmującą wartość indeksu największego elementu tablicy $A$. Zauważmy, że jeśli tablica zawiera losową permutację $n$ liczb, to $\Pr(X=i)=1/n$ dla każdego $i=1$, 2,~\dots,~$n$, a zatem
\[
	\E(X) = \sum_{i=1}^ni\Pr(X=i) = \frac{1}{n}\sum_{i=1}^ni = \frac{1}{n}\cdot\frac{n(n+1)}{2} = \frac{n+1}{2}.
\]
Wynik jest identyczny dla każdego elementu tablicy, w szczególności także dla elementu najmniejszego.

\exercise{} %C.3-3
Zdefiniujmy zmienną losową $X$ przyjmującą wielkość wygranej w opisanej grze. Mamy obliczyć
\[
	\E(X) = -\Pr(A_0)+\Pr(A_1)+2\Pr(A_2)+3\Pr(A_3),
\]
przy czym $A_i$ dla $i=0$, 1,~2,~3 oznacza zdarzenie, że obstawiona przez gracza liczba oczek pojawiła się na dokładnie $i$ kostkach. Prawdopodobieństwa tych zdarzeń wynoszą
\[
	\begin{matrix}
	\Pr(A_0) &=& \dfrac{5^3}{6^3} &=& \dfrac{125}{216}, \\[3mm]
	\Pr(A_1) &=& \dfrac{3\cdot 5^2}{6^3} &=& \dfrac{75}{216}, \\[3mm]
	\Pr(A_2) &=& \dfrac{3\cdot 5^1}{6^3} &=& \dfrac{15}{216}, \\[3mm]
	\Pr(A_3) &=& \dfrac{1}{6^3} &=& \dfrac{1}{216}.
	\end{matrix}
\]
Dostajemy zatem
\[
	\E(X) = -\dfrac{17}{216} \approx -0{,}0787,
\]
a więc gracz straci w tej grze średnio prawie 8 gr.

\exercise{} %C.3-4
Załóżmy, że $\max(X,Y)=X$. Z tego, że $Y\ge0$ wynika $X+Y\ge X$, a stąd $\E(X+Y)\ge\E(X)=\E(\max(X,Y))$. Z liniowości wartości oczekiwanej mamy $\E(X+Y)=\E(X)+\E(Y)$, a zatem $\E(X)+\E(Y)\ge\E(\max(X,Y))$. Analogicznie dowodzi się przypadek dla $\max(X,Y)=Y$.

\exercise{} %C.3-5
Zgodnie z definicją niezależnych zmiennych losowych $X$,~$Y$, mamy
\[
	\Pr(X=x\;\;\text{i}\;\;Y=y) = \Pr(X=x)\Pr(Y=y).
\]
Jeśli $X$ przyjmuje pewną wartość $x$, to zmienna losowa $f(X)$ przyjmuje wartość $f(x)$. Analogicznie dla $Y$, jeśli $Y=y$, to $g(Y)=g(y)$. Powyższe równanie przyjmuje zatem postać
\[
	\Pr\bigl(f(X)=f(x)\;\;\text{i}\;\;g(Y)=g(y)\bigr) = \Pr\bigl(f(X)=f(x)\bigr)\Pr\bigl(g(Y)=g(y)\bigr),
\]
a stąd wnioskujemy, że $f(X)$ i $g(Y)$ są zmiennymi losowymi niezależnymi.

\exercise{} %C.3-6
Niech $A$ będzie pewnym zdarzeniem, a $I(A)$ -- zmienną losową wskaźnikową zdarzenia $A$ zdefiniowaną następująco:
\[
	I(A) =
	\begin{cases}
		0, & \text{jeśli $A$ zachodzi,} \\
		1, & \text{jeśli $A$ nie zachodzi.}
	\end{cases}
\]
Dla każdego $t>0$ prawdziwe są nierówności
\[
	X\ge X\cdot I(X\ge t)\ge t\cdot I(X\ge t).
\]
Pierwsza z nich zachodzi w oczywisty sposób, ponieważ $X\ge0$ oraz $I(A)\le1$ dla każdego zdarzenia $A$. Druga nierówność przyjmuje postać
\[
	X\cdot I(X\ge t)\ge t\cdot I(X\ge t)\;\Leftrightarrow\;
	\begin{cases}
		0\ge0, & \text{dla $X<t$}, \\
		X\ge t, & \text{dla $X\ge t$},
	\end{cases}
\]
a więc również zachodzi. Biorąc wartości oczekiwane powyższych zmiennych losowych i korzystając z elementarnych własności wartości oczekiwanej, otrzymujemy
\[
	\E(X) \ge \E(X\cdot I(X\ge t)) \ge \E(t\cdot I(X\ge t)) = t\,\E(I(X\ge t)) = t\,\Pr(X\ge t),
\]
a stąd
\[
	\Pr(X\ge t) \le \E(X)/t.
\]

\exercise{} %C.3-7
Zauważmy, że
\begin{align*}
	\Pr(X\ge t) &= \sum_{\{\,s\in S:X(s)\ge t\,\}}\Pr(s), \\
	\Pr(X'\ge t) &= \sum_{\{\,s\in S:X'(s)\ge t\,\}}\Pr(s).
\end{align*}
Dowodzimy, że
\[
	\sum_{\{\,s\in S:X(s)\ge t\,\}}\Pr(s)\quad \ge \quad\sum_{\{\,s\in S:X'(s)\ge t\,\}}\Pr(s).
\]
Z założenia wynika, że jeśli $X'(s)\ge t$, to $X(s)\ge t$. Niech $S'\subseteq S$ będzie takim zbiorem (być może pustym), że $X'(s')<t$ i $X(s')\ge t$ dla każdego $s'\in S'$. Wtedy suma po lewej stronie powyższej nierówności zawiera o~$|S'|$ więcej składników niż suma po prawej stronie, a z tego, że $\Pr(s)\ge0$ dla dowolnego $s\in S$ mamy, że nierówność zachodzi.

\exercise{} %C.3-8
Zauważmy, że
\[
	\Var(X) = \E\bigl((X-\E(X))^2\bigr) \ge 0,
\]
ponieważ liczymy wartość oczekiwaną nieujemnej zmiennej losowej. Z wzoru~(C.26) otrzymujemy
\[
	0 \le \Var(X) = \E(X^2)-\E^2(X),
\]
a więc $\E(X^2)\ge\E^2(X)$.

\exercise{} %C.3-9
Ponieważ zmienna losowa $X$ przyjmuje wartości ze zbioru $\{0,1\}$, to dla pewnego $0\le p\le1$ zachodzi
\begin{align*}
	\Pr(X=0) &= p, \\
	\Pr(X=1) &= 1-p.
\end{align*}
Wartością oczekiwaną $X$ jest $\E(X)=0\cdot p+1\cdot(1-p)=1-p$. Zauważmy ponadto, że $\E(X^2)=\E(X)$ i obliczmy wariancję zmiennej losowej $X$,
\begin{align*}
	\Var(X) &= \E(X^2)-\E^2(X) \\
	&= (1-p)-(1-p)^2 \\
	&= p(1-p) \\
	&= \E(X)(1-\E(X)) \\
	&= \E(X)\E(1-X),
\end{align*}
przy czym ostatnia równość zachodzi dzięki liniowości wartości oczekiwanej.

\exercise{} %C.3-10
\begin{align*}
	\Var(aX) &= \E(a^2X^2)-\E^2(aX) \\
	&= a^2\E(X^2)-(a\E(X))^2 \\
	&= a^2\bigl(\E(X^2)-\E(X)\bigr) \\
	&= a^2\Var(X).
\end{align*}

\subchapter{Rozkłady: geometryczny i dwumianowy}
\exercise{} %C.4-1
Rodzina zdarzeń elementarnych $S$ zawiera zdarzenia, które wymagają $k$ prób zanim nastąpi pierwszy sukces dla każdego $k=1$, 2,~\dots, mamy zatem
\begin{align*}
	\Pr(S) &= \sum_{k=1}^\infty q^{k-1}p \\
	&= p\cdot\sum_{k=0}^\infty (1-p)^k \\
	&= \frac{p}{1-(1-p)} \\[1mm]
	&= 1.
\end{align*}
Wykorzystano wzór (A.6) przy założeniu, że $p>0$.

\exercise{} %C.4-2
Niech sukces oznacza uzyskanie w rzucie sześcioma monetami trzech orłów i trzech reszek, a porażka -- każdy inny wynik. Możemy wybrać dowolne trzy monety spośród sześciu, na których będzie orzeł, mamy zatem $\binom{6}{3}=20$ sposobów osiągnięcia sukcesu. Jest $2^6=64$ wszystkich możliwych wyników, zatem prawdopodobieństwo sukcesu wynosi $p=20/64=5/16$. Z wzoru~(C.31) otrzymujemy, że musimy wykonać średnio $1/p=3{,}2$ rzutów.

\exercise{} %C.4-3
Z definicji rodziny rozkładów dwumianowych,
\begin{align*}
	b(k;n,p) &= \binom{n}{k}p^k(1-p)^{n-k}, \\
	b(n-k;n,q) &= \binom{n}{n-k}q^{n-k}(1-q)^k.
\end{align*}
Ponieważ $\binom{n}{n-k}=\binom{n}{k}$ (z wzoru~(C.3)) oraz $q=1-p$, otrzymujemy $b(k;n,p)=b(n-k;n,q)$.

\exercise{} %C.4-4
Ponieważ rozkład dwumianowy przyjmuje maksimum dla pewnego $k$ całkowitego z przedziału $np-q\le k\le(n+1)p$, to dobrym przybliżeniem wartości maksymalnej jest wartość dla $k=np$, które oczywiście leży w tym przedziale.
\begin{align*}
	b(np;n,p) &= \binom{n}{np}p^{np}(1-p)^{n-np} \\
	&= \frac{n!}{(np)!\,(n-np)!}\,p^{np}(1-p)^{n-np} \\
	&= \frac{n!}{(np)!\,(nq)!}\,p^{np}q^{nq}.
\end{align*}
Wykorzystując wzór Stirlinga do przybliżenia silni, możemy uprościć pierwszy czynnik następująco:
\begin{align*}
	\frac{n!}{(np)!\,(nq)!} &\approx \frac{\sqrt{2\pi n}\left(\frac{n}{e}\right)^n}{\sqrt{2\pi np}\left(\frac{np}{e}\right)^{np}\sqrt{2\pi nq}\left(\frac{nq}{e}\right)^{nq}} \\[1mm]
	&= \frac{\left(\frac{n}{e}\right)^n\left(\frac{e}{np}\right)^{np}\left(\frac{e}{nq}\right)^{nq}}{\sqrt{2\pi npq}} \\[1mm]
	&= \frac{1}{p^{np}q^{nq}\sqrt{2\pi npq}}.
\end{align*}
Stąd dostajemy przybliżenie
\[
	b(np;p,n) \approx \frac{1}{\sqrt{2\pi npq}}
\]
na maksymalną wartość rozkładu dwumianowego $b(k;n,p)$.

\exercise{} %C.4-5
Niech $X$ będzie zmienną losową przyjmującą liczbę sukcesów w tym doświadczeniu losowym. Wtedy
\[
	\Pr(X=0) = b\Bigl(0;n,\frac{1}{n}\Bigr) = \left(1-\frac{1}{n}\right)^n,
\]
co dąży do $1/e$ wraz ze wzrostem $n$. Wynika to stąd, że ciąg $e_n={(1+k/n)}^n$ ma dla dowolnej stałej $k$ granicę równą $e^k$, przy $n$ dążącym do $\infty$.
Analogicznie,
\[
	\Pr(X=1) = b\Bigl(1;n,\frac{1}{n}\Bigr) = \frac{\left(1-\frac{1}{n}\right)^n}{1-\frac{1}{n}}
\]
dąży do $1/e$, ponieważ mianownik zbliża się do 1 wraz ze wzrostem $n$.

\exercise{} %C.4-6
Obliczymy prawdopodobieństwo uzyskania przez profesorów równej liczby orłów na dwa sposoby. W pierwszym z nich, niech $X$ i $Y$ będą zmiennymi losowymi oznaczającymi liczby orłów uzyskane kolejno przez obu profesorów. Prawdopodobieństwo uzyskania $k$ orłów ($0\le k\le n$) przez każdego z nich jest rozkładem dwumianowym, $\Pr(X=k)=\Pr(Y=k)=b(k;n,1/2)$. Zdarzenia $X=k$ i $Y=k$ są niezależne, zatem prawdopodobieństwo uzyskania przez obu profesorów równej ilości orłów wynosi
\begin{align*}
	\sum_{k=0}^n\Pr(X=k\;\;\text{i}\;\;Y=k) &= \sum_{k=0}^n\Pr(X=k)\Pr(Y=k) \\
	&= \sum_{k=0}^n\binom{n}{k}\left(\frac{1}{2}\right)^n\binom{n}{k}\left(\frac{1}{2}\right)^n \\
	&= \frac{\sum_{k=0}^n\binom{n}{k}^2}{4^n}.
\end{align*}

W drugim sposobie potraktujmy wynik każdego doświadczenia jako \twoparts{$2n$}{elementowy} ciąg wyników taki, że początkowych $n$ wyrazów oznacza wyniki uzyskane przez profesora Rosencrantza, a $n$ końcowych -- wyniki profesora Guildensterna. Na każdej pozycji znajduje się jedna z dwóch wartości, orzeł lub reszka, zatem wszystkich możliwych ciągów jest $2^{2n}=4^n$. Niech sukcesem dla profesora Rosencrantza będzie uzyskanie orła, a dla profesora Guildensterna -- uzyskanie reszki. Zauważmy, że wyrzucenie równej liczby orłów przez obu profesorów jest równoważne z osiągnięciem przez nich w sumie $n$ sukcesów. Liczba sposobów, na jakie można to zrobić, jest liczbą możliwości wyboru spośród $2n$ pozycji ciągu, $n$ odpowiedzialnych za sukces, która to wynosi $\binom{2n}{n}$, a zatem szukane prawdopodobieństwo jest równe
\[
	\frac{\binom{2n}{n}}{4^n}.
\]

Przyrównując do siebie wyniki z obu sposobów, dostajemy tożsamość
\[
	\sum_{k=0}^n\binom{n}{k}^2 = \binom{2n}{n}.
\]

\exercise{} %C.4-7
Wykorzystując nierówność
\[
	\binom{n}{\lambda n} \le 2^{nH(\lambda)},
\]
otrzymujemy
\[
	b\Bigl(k;n,\frac{1}{2}\Bigr) = \binom{n}{k}\left(\frac{1}{2}\right)^n = \frac{\binom{n}{k}}{2^n} \le \frac{2^{nH(n/k)}}{2^n} = 2^{nH(n/k)-n}.
\]

\exercise{} %C.4-8
Na wstępie zauważmy, że $p\ge p_i$ dla każdego $i=1$, 2,~\dots,~$n$ implikuje
\[
	1-p_i \le 1-p \quad\text{oraz}\quad \frac{p_i}{1-p_i}\le\frac{p}{1-p}.
\]
Ponieważ przy potęgowaniu obu stron nierówności oraz przy mnożeniu ich stronami (obie strony są nieujemne) nie zmienia się znak nierówności, to możemy napisać
\[
	(1-p_i)^n\left(\frac{p_i}{1-p_i}\right)^i \le (1-p)^n\left(\frac{p}{1-p}\right)^i,
\]
a stąd
\[
	\binom{n}{i}p_i^i(1-p_i)^{n-i} \le \binom{n}{i}p^i(1-p)^{n-i}.
\]
Sumując powyższe nierówności po $i$, mamy, że dla dowolnego $1\le k\le n$
\[
	\sum_{i=0}^{k-1}\binom{n}{i}p_i^i(1-p_i)^{n-i} \le \sum_{i=0}^{k-1}\binom{n}{i}p^i(1-p)^{n-i} = \sum_{i=0}^{k-1}b(i;n,p),
\]
co kończy dowód.

\exercise{} %C.4-9
Niech $S$ będzie przestrzenią zdarzeń (serii prób Bernoulliego). Rozważmy doświadczenie wykonane dla serii prób z ciągu $s\in S$, którego wynik potraktujmy jako ciąg \twoparts{$n$}{bitowy}. Bit o wartości~0 na \twoparts{$i$}{tej} pozycji w takim ciągu oznacza, że \twoparts{$i$}{ta} próba w tym doświadczeniu zakończyła się sukcesem, a~1 -- że zakończyła się porażką.

Doświadczenie z \twoparts{$i$}{tą} z kolei próbą w pierwszym przypadku zakończy się sukcesem z prawdopodobieństwem równym $p_i$, generując pewien ciąg $A$. Utwórzmy teraz nowy ciąg bitów $A'$ będący reprezentacją doświadczenia wykonanego na $s$ ale dla drugiego przypadku. Sukces w \twoparts{$i$}{tej} próbie z $A'$ wystąpi z niemniejszym prawdopodobieństwem od wystąpienia sukcesu w \twoparts{$i$}{tej} próbie w ciągu $A$. Można zatem utworzyć taki ciąg poprzez przepisanie wszystkich jedynek starego ciągu do odpowiadających pozycji w nowym ciągu, zaś dla każdej pozycji z zerem, wybrać losowo wartość~0 lub 1 na pozycję nowego ciągu.

Taka operacja zapewnia, że ciągi $A'$ mają nie mniej jedynek niż odpowiadające ciągi $A$, a co za tym idzie, doświadczenia wykonane w drugim przypadku obfitują w więcej sukcesów niż te z pierwszego. Zatem dla dowolnego zdarzenia $s\in S$, zachodzi
\[
	X'(s) \ge X(s).
\]
Korzystając teraz z wyniku z \zad{C.3-7}, otrzymujemy tezę.

\subchapter{Krańce rozkładu dwumianowego}

\exercise{} %C.5-1
Niech $X$ i $Y$ będą zmiennymi losowymi przyjmującymi liczby uzyskanych orłów, kolejno w obu zdarzeniach. Mamy zatem
\begin{align*}
	\Pr(X=0) &= b(0;n,1/2) = \binom{n}{0}\left(\frac{1}{2}\right)^0\left(\frac{1}{2}\right)^n = \left(\frac{1}{2}\right)^n, \\
	\Pr(Y<n) &= b(n;4n,1/2) \\
	&< \frac{\frac{n}{2}}{\frac{4n}{2}-n}\,b(n;4n,1/2) \\
	&< \binom{4n}{n}\left(\frac{1}{2}\right)^n\left(\frac{1}{2}\right)^{3n} \\
	&= \frac{(4n)!}{(3n)!\cdot n!\cdot 2^{4n}} \\
	&= \frac{(3n+1)(3n+2)\dots(4n-1)(4n)}{n!}\cdot\left(\frac{1}{2}\right)^{4n}.
\end{align*}
Przy obliczeniu ostatniego prawdopodobieństwa wykorzystano twierdzenie~C.4. Zauważmy, że w powyższym ułamku licznik ma $n$ czynników, zatem można potraktować ułamek jako
\[
	\left(\frac{3n+1}{1}\right)\left(\frac{3n+2}{2}\right)\dots\left(\frac{4n-1}{n-1}\right)\left(\frac{4n}{n}\right).
\]
Wszystkie czynniki w powyższym iloczynie są ograniczone od góry przez~4, dostajemy więc ograniczenie
\[
	\Pr(B) < 4^n\left(\frac{1}{2}\right)^{4n} = \left(\frac{1}{4}\right)^n.
\]
Stąd mamy, że $\Pr(A)>\Pr(B)$, a zatem uzyskanie mniej niż $n$ orłów w $4n$ rzutach monetą jest mniej prawdopodobne od nieuzyskania żadnego orła w $n$ rzutach monetą.

\exercise{} %C.5-2
\begin{proof}[Dowód wniosku C.6]
	Ponieważ uzyskanie co najwyżej $k$ sukcesów w $n$ próbach jest równoważne usyskaniu co najmniej $n-k$ porażek, to zachodzi $\Pr(X\le k)=\Pr(Y\ge n-k)$, gdzie $Y$ jest zmienną losową oznaczającą liczbę porażek. Z tw.~C.2 mamy
	\begin{align*}
		\Pr(X\le k) = \Pr(Y\ge n-k) &= \sum_{i=n-k}^nb(i;n,q) \\
		&< \binom{n}{n-k}q^{n-k} \\
		&= \binom{n}{k}(1-p)^{n-k}
	\end{align*}
	na podstawie tego, że $q=1-p$ i tożsamości~(C.3).
\end{proof}

\begin{proof}[Dowód wniosku C.7]
	Podobnie, przy oznaczeniach w poprzedniego dowodu, mamy $\Pr(X>k)=\Pr(Y<n-k)$, a zatem z tw.~C.4:
	\begin{align*}
		\Pr(X>k) = \Pr(Y<n-k) &= \sum_{i=0}^{n-k-1}b(i;n,q) \\
		&< \frac{(n-k)p}{nq-(n-k)}\,b(n-k;n,q) \\[1mm]
		&= \frac{(n-k)p}{k-np}\binom{n}{n-k}q^{n-k}(1-q)^k \\[1mm]
		&= \frac{(n-k)p}{k-np}\binom{n}{k}(1-p)^{n-k}p^k \\[1mm]
		&= \frac{(n-k)p}{k-np}\,b(k;n,p).
	\end{align*}
\end{proof}

\exercise{} %C.5-3
Niech $p=a/(a+1)$, skąd mamy, że $a=p/(1-p)$. Zachodzi
\[
	\sum_{i=0}^{k-1}\binom{n}{i}a^i = \sum_{i=0}^{k-1}\binom{n}{i}\left(\frac{p}{1-p}\right)^i = \frac{\sum_{i=0}^{k-1}b(i;n,p)}{(1-p)^n},
\]
zatem z tw.~C.4 otrzymujemy:
\begin{align*}
	\sum_{i=0}^{k-1}\binom{n}{i}a^i &< \frac{\frac{k}{a+1}}{\left(\frac{1}{a+1}\right)^n\left(\frac{na}{a+1}-k\right)}\,b\Bigl(k;n,\frac{a}{a+1}\Bigr) \\[1mm]
	&= (a+1)^n\frac{k}{na-k(a+1)}\,b\Bigl(k;n,\frac{a}{a+1}\Bigr).
\end{align*}

\exercise{} %C.5-4
Wykorzystując obserwację, że $\binom{n}{i}\ge1$ dla $0\le i\le n$ oraz twierdzenie~C.4, mamy
\[
	\sum_{i=0}^{k-1}p^iq^{n-i} \le \sum_{i=0}^{k-1}\binom{n}{i}p^iq^{n-i} = \sum_{i=0}^{k-1}b(i;n,p) < \frac{kq}{np-k}\,b(k;n,p).
\]
Z kolei, na mocy wniosku~C.1 dostajemy
\[
	\frac{kq}{np-k}\,b(k;n,p) \le \frac{kq}{np-k}\left(\frac{np}{k}\right)^k\left(\frac{nq}{n-k}\right)^{n-k},
\]
skąd bezpośrednio otrzymujemy tezę.

\exercise{} %C.5-5
Przeprowadzimy dowód analogicznie, jak przebiegał dowód tw.~C.6. Dla dowolnego $\alpha>0$ mamy
\[
	\Pr(\mu-X\ge r) = \Pr\bigl(e^{\alpha(\mu-X)}\ge e^{\alpha r}\bigr)
\]
i z nierówności Markowa dostajemy
\[
	\Pr(\mu-X\ge r) = \E\bigl(e^{\alpha(\mu-X)}\bigr)e^{-\alpha r}.
\]
Niech $X_i$ dla $i=1$, 2,~\dots,~$n$ będzie zmienną losową przyjmującą~1, jeśli wynikiem \twoparts{$i$}{tej} próby Bernoulliego jest sukces i~0 w przeciwnym przypadku. Wtedy
\[
	X = \sum_{i=1}^nX_i
\]
oraz
\[
	\mu-X = \sum_{i=1}^n(p_i-X_i),
\]
otrzymujemy zatem
\[
	\E\bigl(e^{\alpha(\mu-X)}\bigr) = \E\biggl(\prod_{i=1}^ne^{\alpha(p_i-X_i)}\biggr) = \prod_{i=1}^n\E\bigl(e^{\alpha(p_i-X_i)}\bigr),
\]
co wynika z wzajemnej niezależności zmiennych losowych $X_i$, a co za tym idzie także zmiennych losowych $e^{\alpha(p_i-X_i)}$. Z definicji wartości oczekiwanej mamy
\begin{align*}
	\E\bigl(e^{\alpha(p_i-X_i)}\bigr) &= e^{\alpha(p_i-1)}p_i+e^{\alpha(p_i-0)}q_i \\
	&= p_ie^{-\alpha q_i}+q_ie^{\alpha p_i} \\
	&\le q_ie^{\alpha}+1 \\
	&\le \exp(q_ie^\alpha),
\end{align*}
skąd zachodzi
\[
	\E\bigl(e^{\alpha(\mu-X)}\bigr) \le \exp\biggl(\prod_{i=1}^nq_ie^\alpha\biggr) = \exp((n-\mu)e^\alpha),
\]
bo $\mu=\sum_{i=1}^np_i$, a więc $\sum_{i=1}^nq_i=n-\mu$. Wracając do oszacowania prawdopodobieństwa dostajemy
\[
	\Pr(\mu-X\ge r) \le \exp((n-\mu)e^\alpha-\alpha r).
\]
Ponieważ we wzorze~(C.45) wybranie $\alpha=\ln(r/\mu)$ minimalizuje prawą stronę tego wzoru, to w naszym przypadku po prawej stronie zamiast $\mu$ mamy $n-\mu$, a więc przyjmujemy $\alpha=\ln(r/(n-\mu))$. Dostajemy ostatecznie
\begin{align*}
	\Pr(\mu-X\ge r) &\le \exp\left((n-\mu)e^{\ln\frac{r}{n-\mu}}-r\ln\frac{r}{n-\mu}\right) \\
	&= \exp\left(r-r\ln\frac{r}{n-\mu}\right) \\
	&= \frac{e^r}{\left(\frac{r}{n-\mu}\right)^r} \\
	&= \left(\frac{(n-\mu)e}{r}\right)^r.
\end{align*}

\exercise{} %C.5-6
\note{W treści zadania w polskim tłumaczeniu występuje błąd. Wskazówka podpowiada, aby udowodnić pewną nierówność, jednak wyrażenie po prawej stronie znaku nierówności powinno mieć postać\/ $e^{\alpha^2\!/2}$.}

\noindent Zanim przeprowadzimy główny dowód, wykażemy tożsamość
\begin{equation}
	p_ie^{\alpha q_i}+q_ie^{-\alpha p_i} \le e^{\alpha^2\!/2}, \label{eq:C.5-7_1}
\end{equation}
gdzie $\alpha>0$,~$p_i$,~$q_i\ge0$ oraz $p_i+q_i=1$.

\begin{proof}
	Przekształćmy nierówność do alternatywnej postaci:
	\begin{align*}
		p_ie^{\alpha q_i}+q_ie^{-\alpha p_i} &\le e^{\alpha^2\!/2} \\[1mm]
		p_ie^{\alpha(1-p_i)}+(1-p_i)e^{-\alpha p_i} &\le e^{\alpha^2\!/2} \\[2mm]
		\frac{p_ie^\alpha+(1-p_i)}{e^{\alpha p_i}} &\le e^{\alpha^2\!/2} \\[1mm]
		p_ie^\alpha-p_i+1 &\le e^{\alpha^2\!/2+\alpha p_i} \\[1mm]
		\ln(p_ie^\alpha-p_i+1) &\le \alpha^2\!/2+\alpha p_i \\[1mm]
		\ln(p_ie^\alpha-p_i+1)-\alpha^2\!/2-\alpha p_i &\le 0
	\end{align*}
	Wystarczy teraz wykazać, że funkcja $f(\alpha,p_i)$ po lewej stronie znaku ostatniej nierówności przyjmuje wartości niedodatnie.

	Zauważmy, że granicą funkcji $f$ w punkcie $(0,p_i)$ jest 0. Udowodnimy, że $f$ maleje wraz ze wzrostem $\alpha$. Dla $p_i=0$ oraz $p_i=1$ mamy
	\begin{align*}
	    f_0(\alpha) &\equiv f(\alpha,0) = -\alpha^2\!/2, \\
		f_1(\alpha) &\equiv f(\alpha,1) = -\alpha^2\!/2,
	\end{align*}
	które są funkcjami malejącymi ze względu na $\alpha>0$. Przyjmijmy teraz, że $0<p_i<1$ i potraktujmy $f_{p_i}(\alpha)$ jako $f(\alpha,p_i)$ dla ustalonego $p_i$. Obliczmy pochodną funkcji $f_{p_i}$ i zbadajmy jej znak dla dowolnej wartości argumentu $\alpha>0$:
	\begin{align*}
	    \frac{df_{p_i}}{d\alpha}(\alpha) &= \frac{p_ie^\alpha}{p_ie^\alpha-p_i+1}-\alpha-p_i \\[1mm]
		&= \frac{p_ie^\alpha-\alpha p_ie^\alpha+\alpha p_i-\alpha-p_i^2e^\alpha+p_i^2-p_i}{p_ie^\alpha-p_i+1} \\[1mm]
		&= \frac{\overbrace{p_ie^\alpha(1-\alpha-p_i)}^{<0}+\overbrace{\alpha(p_i-1)}^{<0}+\overbrace{p_i(p_i-1)}^{<0}}{\underbrace{p_ie^\alpha-p_i+1}_{>0}} \\
		&< 0.
	\end{align*}
	Otrzymany wynik potwierdza, że dla każdego $0\le p_i\le1$, $f_{p_i}(\alpha)$ jest malejące w zbiorze liczb dodatnich, wartości funkcji $f$ są zatem ujemne, a więc nierówność (\ref{eq:C.5-7_1}) jest prawdziwa.

% 	Spróbujmy znaleźć ekstrema lokalne tej funkcji w jej dziedzinie $D_f=(0,\infty)\times[\,0,1\,]$. Obliczamy pochodne pierwszego rzędu funkcji $f$:
% 	\begin{align*}
% 	    \frac{\partial\!f}{\partial\alpha}(\alpha,p_i) &= \frac{p_ie^\alpha}{p_ie^\alpha-p_i+1}-\alpha-p_i, \\
% 		\frac{\partial\!f}{\partial p_i}(\alpha,p_i) &= \frac{e^\alpha-1}{p_ie^\alpha-p_i+1}-\alpha
% 	\end{align*}
% 	i przyrównujemy je do zera, otrzymując układ równań:
% 	\[
% 	    \begin{cases}
% 	        p_ie^\alpha-\alpha p_ie^\alpha+\alpha p_i-\alpha-p_i^2e^\alpha+p_i^2-p_i = 0 \\
% 			e^\alpha-1-\alpha p_ie^\alpha+\alpha p_i-\alpha = 0.
% 	    \end{cases}
% 	\]
% 	Po odjęciu drugiej równości od pierwszej mamy
% 	\begin{align*}
% 	    p_ie^\alpha-p_i^2e^\alpha+p_i^2-p_i-e^\alpha+1 &= 0 \\
% 		(e^\alpha-1)(p_i-p_i^2-1) &= 0,
% 	\end{align*}
% 	skąd dostajemy, że ekstremum, o ile istnieje, znajduje się w punktach $(0,p_i)$. Obliczając wartość funkcji $f$ w tych punktach, otrzymujemy $f(0,p_i)=0$. Analiza na podstawie pochodnych $f$ drugiego rzędu nie rozstrzyga jednak, czy jest to ekstremum funkcji $f$. Możemy jednak wykazać, że $f$ jest malejące wraz ze wzrostem $\alpha$. Biorąc pewne $0<\alpha_1\le\alpha_2$ mamy
% 	\[
% 	    f(\alpha_1,p_i)-f(\alpha_2,p_i) = \ln\biggl(\frac{p_ie^{\alpha_1}-p_i+1}{p_ie^{\alpha_2}-p_i+1}\biggr)+(\alpha_2^2/2-\alpha_1^2/2)+(\alpha_2p_i-\alpha_1p_i),
% 	\]

	
\end{proof}

Początek głównego rozumowania prowadzimy identycznie jak w dowodzie twierdzenia~C.8:
\begin{equation}
	\Pr(X-\mu\ge r) = \Pr\bigl(e^{\alpha(X-\mu)}\ge e^{\alpha r}\bigr) \le \E\bigl(e^{\alpha(X-\mu)}\bigr)e^{-\alpha r}. \label{eq:C.5-7_2}
\end{equation}
Następnie, przy tych samych oznaczeniach z oryginalnego dowodu, zachodzi
\[
	\E\bigl(e^{\alpha(X-\mu)}\bigr) = \prod_{i=1}^n\E\bigl(e^{\alpha(X_i-p_i)}\bigr).
\]
Wykorzystując wcześniej udowodnioną nierówność~(\ref{eq:C.5-7_1}), otrzymujemy
\begin{align*}
	\E\bigl(e^{\alpha(X_i-p_i)}\bigr) &= e^{\alpha(1-p_i)}p_i+e^{\alpha(0-p_i)}q_i \\
	&= p_ie^{\alpha q_i}+q_ie^{-\alpha p_i} \\
	&\le e^{\alpha^2\!/2}
\end{align*}
i dalej mamy
\begin{equation}
	\E\bigl(e^{\alpha(X-\mu)}\bigr) = \prod_{i=1}^n\E\bigl(e^{\alpha(X_i-p_i)}\bigr) \le \prod_{i=1}^ne^{\alpha^2\!/2} = e^{\alpha^2n/2}. \label{eq:C.5-7_3}
\end{equation}
Z nierówności~(\ref{eq:C.5-7_2}) i (\ref{eq:C.5-7_3}) wynika, że
\[
	\Pr(X-\mu\ge r) \le \exp(\alpha^2n/2-\alpha r).
\]
Należy teraz wybrać taką wartość $\alpha$, która minimalizuje prawą stronę powyższej nierówności. Argumentem funkcji wykładniczej jest funkcja kwadratowa zmiennej $\alpha$, w prosty sposób można więc sprawdzić, że osiąga ona minimum dla argumentu $\alpha=r/n$. Dostajemy zatem
\[
	\Pr(X-\mu\ge r) \le \exp\bigl((r/n)^2n/2-(r/n)r\bigr) = e^{-r^2\!/n}
\]
co kończy dowód tożsamości.

\exercise{} %C.5-7
Potraktujmy wyrażenie jako funkcję $f$ zmiennej $\alpha$:
\[
	f(\alpha) = \exp(\mu e^\alpha-\alpha r).
\]
W celu wyznaczenia jej minimum, obliczmy pierwszą i drugą pochodną:
\begin{align*}
	\frac{df}{d\alpha}(\alpha) &= (\mu e^\alpha-r)\exp(\mu e^\alpha-\alpha r), \\
	\frac{d^2\!f}{d\alpha^2}(\alpha) &= \left(\mu e^\alpha+(\mu e^\alpha-r)^2\right)\exp(\mu e^\alpha-\alpha r).
\end{align*}
Przyrównując pierwszą pochodną do~0, otrzymujemy, że w punkcie $\alpha_0=\ln(r/\mu)$ może istnieć ekstremum $f$. Po obliczeniu wartości drugiej pochodnej w tym punkcie dostajemy
\[
	\frac{d^2\!f}{d\alpha^2}(\alpha_0) = r\exp(r-r\ln(r/\mu)) > 0,
\]
ponieważ $\exp(x)$ jest rosnące oraz $r>\mu\ge0$, a zatem w punkcie $\ln(r/\mu)$ istnieje minimum funkcji~$f$.

\problems

\problem{Kule i urny} %C-1

\subproblem %C-1(a)
Każda kula trafia do jednej z $b$ urn. Jest $b$ sposobów umieszczenia pierwszej kuli, na każdy z nich przypada $b$ sposobów umieszczenia drugiej kuli, itd. Jest zatem $b^n$ sposobów rozmieszczenia $n$ różnych kul w $b$ różnych urnach.

\subproblem %C-1(b)
Ponieważ dysponujemy $n$ rozróżnialnymi kulami oraz $b$ nierozróżnialnymi urnami, to nasz problem jest równoważny policzeniu wszystkich możliwych ciągów $n$ różnych kul i $b-1$ identycznych patyków. Patyki dzielą ciąg kul na części, z których każda odpowiada zbiorowi kul w kolejnej urnie, a każdy spójny podciąg kul reprezentuje pewne ich uporządkowanie względem siebie wewnątrz pewnej urny.

Wszystkich takich ciągów jest $(b+n-1)!$, ale ponieważ nie rozróżniamy urn, to musimy podzielić ich liczbę przez liczbę możliwych rozmieszczeń urn między sobą, czyli $(b-1)!$. Istnieje zatem $\frac{(b+n-1)!}{(b-1)!}$ różnych rozmieszczeń kul w urnach.

\subproblem %C-1(c)
Sytuacja jest podobna jak w punkcie~(b) z tą różnicą, że nie rozróżniamy kul między sobą, a więc również każda permutacja $n$ kul między sobą opisuje ten sam sposób rozmieszczenia kul w urnach. Mamy zatem $\frac{(b+n-1)!}{n!\,(b-1)!}=\binom{b+n-1}{n}$ możliwości rozmieszczenia kul.

\subproblem %C-1(d)
Zakładając, że $n\le b$, wybieramy spośród $b$ urn $n$ takich, które będą zawierać po jednej kuli. Jest $\binom{b}{n}$ sposobów ich wyboru.

\subproblem %C-1(e)
Zakładamy, że $n\ge b$. Najpierw umieszczamy $b$ kul, po jednej w każdej urnie tak, aby żadna urna nie była pusta. Pozostałe $n-b$ kul możemy umieścić w $b$ urnach, zgodnie z punktem~(c), na $\binom{b+(n-b)-1}{n-b}=\binom{n-1}{n-b}=\binom{n-1}{b-1}$ sposobów.

\endinput
