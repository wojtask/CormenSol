\subchapter{Planowanie czynności na liniach montażowych}

\exercise %15.1-1
Poniższa rekurencyjna procedura wypisuje wszystkie stanowiska o~numerach od 1 do $j$ w~kolejności rosnącej, przy założeniu, że linia użyta na stanowisku $j$ ma numer $i$.
Do wywołania rekurencyjnego przekazywany jest numer wykorzystanej linii na stanowisku $j-1$ odczytany z~$l_i[j]$.
Imitowane jest dzięki temu zachowanie pętli z~oryginalnej procedury.
\begin{codebox}
\Procname{$\proc{Print-Stations}'(l,i,j)$}
\li	\If $j\ge1$
\li	\Then $\proc{Print-Stations}'(l,l_i[j],j-1)$
\li		wypisz ,,linia '' $i$ ,,{}, stanowisko '' $j$
	\End
\end{codebox}
Aby wypisać cały ciąg stanowisk, należy wywołać $\proc{Print-Stations}'(l,l^*\!,n)$.

\exercise %15.1-2
W~pierwszym kroku indukcyjnym wzór oczywiście zachodzi:
\[
	r_1(n) = r_2(n) = 1 = 2^{n-n}.
\]
Niech teraz $j=1$, 2, \dots, $n-1$ i~przyjmijmy, że $r_1(j+1)=r_2(j+1)=2^{n-(j+1)}$.
Stąd
\[
	r_1(j) = r_2(j) = r_1(j+1)+r_2(j+1) = 2^{n-(j+1)}+2^{n-(j+1)} = 2^{n-(j+1)+1} = 2^{n-j},
\]
a~więc wzór jest prawdziwy dla dowolnego $j$.

\exercise %15.1-3
Na podstawie wyniku poprzedniego zadania oraz wzoru (A.5), łączna liczba odwołań do wartości $f_i[j]$ wynosi:
\[
	\sum_{i=1}^2\sum_{j=1}^nr_i(j) = 2\sum_{j=1}^n2^{n-j} = 2\sum_{k=0}^{n-1}2^k = 2\cdot\frac{2^n-1}{2-1} = 2^{n+1}-2.
\]

\exercise %15.1-4
Możemy zrezygnować z~większości komórek tablic $f_i$.
Wystarczy zauważyć, że jedynymi wartościami z~tablic $f_i$ potrzebnymi do obliczenia $f_i[j]$ są $f_i[j-1]$.
Tablice $f_i$ nie są też wykorzystywane podczas wypisywania optymalnego rozwiązania w~procedurze \proc{Print-Stations}.
Dokładniej mówiąc, w~procedurze \proc{Fastest-Way} wystarczy zaalokować jedynie $f_1[1\twodots2]$ oraz $f_2[1\twodots2]$, czyli w~sumie 4 komórki.
Na pozycji $f_i[1]$ zapisane zostaną wartości dla poprzedniego stanowiska i~wykorzystane następnie do obliczenia wartości dla obecnego stanowiska, które zostaną zapamiętane w~$f_i[2]$.
Na początku każdej iteracji pętli \kw{for} $f_i[1]$ będzie nadpisywane przez $f_i[2]$.
Dzięki tej modyfikacji tablice $f_i$ i~$l_i$ będą zawierać łącznie $4+(2n-2)=2n+2$ pozycji.

\exercise %15.1-5
Przypisanie $l_1[j]\gets2$ wykonywane jest w~procedurze \proc{Fastest-Way} tylko w~linii 8 i~odbywa się jedynie wtedy, gdy warunek z~wiersza 4 jest fałszywy.
Podobnie, przypisanie $l_2[j]\gets1$ odbywa się jedynie wtedy, gdy warunek z~wiersza 9 jest fałszywy.
Oba te przypisania zostaną więc wykonane dla tego samego $j$, o~ile spełnione zostaną nierówności
\[
	f_1[j-1]+a_{1,j}>f_2[j-1]+t_{2,j-1}+a_{1,j} \quad\text{oraz}\quad f_2[j-1]+a_{2,j}>f_1[j-1]+t_{1,j-1}+a_{2,j}.
\]
Jednakże po dodaniu ich stronami i~zredukowaniu powtarzających się wyrazów, otrzymujemy $t_{1,j-1}+t_{2,j-1}<0$, co jest sprzeczne z~założeniem.
