\subchapter{Planowanie czynności na liniach montażowych}

\exercise %15.1-1
Poniższa procedura wypisuje wszystkie stanowiska w~kolejności rosnącej -- stanowisko o~numerze $n$ wypisywane jest po stanowiskach o~niższych numerach, co realizowane jest poprzez zastosowanie rekursji.
\begin{codebox}
\Procname{$\proc{Print-Stations}'(l,l^*\!,n)$}
\li	\If $n\ge1$
\li	\Then $\proc{Print-Stations}'(l,l_{l^*\!}[n],n-1)$
\li		wypisz ,,linia '' $l^*\!$ ,,{}, stanowisko '' $n$
	\End
\end{codebox}
Do wywołania rekurencyjnego przekazywany jest numer wykorzystanej linii na stanowisku $n-1$, który odczytywany jest z~$l_{l^*\!}[n]$.
Imitujemy dzięki temu zachowanie pętli z~oryginalnej procedury.

\exercise %15.1-2
\exercise %15.1-3
\exercise %15.1-4
\exercise %15.1-5
