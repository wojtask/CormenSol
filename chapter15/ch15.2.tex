\subchapter{Mnożenie ciągu macierzy}

\exercise %15.2-1
Posługując się algorytmami \proc{Matrix-Chain-Order} i~\proc{Print-Optimal-Parens}, dostajemy, że dla ciągu macierzy $\langle A_1,\dots,A_6\rangle$ o~zadanych rozmiarach, optymalnym nawiasowaniem jest $((A_1A_2)((A_3A_4)(A_5A_6)))$.
Mnożąc macierze zgodnie z~tym nawiasowaniem, wykonamy 2010 mnożeń skalarnych.

\exercise %15.2-2
Nasz algorytm oprzemy na \proc{Print-Optimal-Parens}, ale zamiast wypisywać nawiasowanie, będziemy przekazywać macierze zwrócone przez wywołania rekurencyjne do procedury odpowiedzialnej za rzeczywiste ich pomnożenie.
\begin{codebox}
\Procname{$\proc{Matrix-Chain-Multiply}(A,s,i,j)$}
\li	\If $i=j$
\li		\Then \Return $A_i$
		\End
\li	\Return $\proc{Matrix-Multiply}($
\zi	\>\>\> $\proc{Matrix-Chain-Multiply}(A,s,i,s[i,j]),$
\zi	\>\>\> $\proc{Matrix-Chain-Multiply}(A,s,s[i,j]+1,j))$
\end{codebox}

\exercise %15.2-3
Udowodnimy przez indukcję, że $P(n)\ge2^{n-2}$ dla każdego $n\ge1$, co pozwoli nam wywnioskować, że $P(n)=\Omega(2^n)$.

W~przypadkach, gdy $n=1$, 2, 3, 4, nierówność jest spełniona:
\begin{align*}
	P(1) &= 1 \ge 2^{1-2}, \\
	P(2) &= P(1)P(1) = 1 \ge 2^{2-2}, \\
	P(3) &= P(1)P(2)+P(2)P(1) = 1+1 = 2 \ge 2^{3-2}, \\
	P(4) &= P(1)P(3)+P(2)P(2)+P(3)P(1) = 2+1+2 = 5 \ge 2^{4-2}.
\end{align*}
Niech teraz $n\ge5$.
Dla każdego $i=1$, \dots, $n-1$ przyjmiemy założenie indukcyjne $P(i)\ge2^{i-2}$.
Mamy wówczas:
\[
	P(n) = \sum_{k=1}^{n-1}P(k)P(n-k) \ge \sum_{k=1}^{n-1}2^{k-2}2^{n-k-2} = \sum_{k=1}^{n-1}2^{n-4} = \frac{n-1}{4}\cdot2^{n-2}.
\]
Gdy $n\ge5$, to $(n-1)/4\ge1$, dlatego otrzymujemy $P(n)\ge2^{n-2}$.

\exercise %15.2-4
W~celu obliczenia $m[i,j]$, w~każdej iteracji pętli \kw{for} w~wierszach \doubledash{8}{12} wykorzystywane są dwie inne wartości z~tej tablicy.
Łączna liczba odwołań do tablicy $m$ jest zatem dwukrotnie większa od sumarycznej liczby iteracji najbardziej wewnętrznej pętli.
Mamy więc:
\begin{align*}
	\sum_{i=1}^n\sum_{j=i}^nR(i,j) &= \sum_{l=2}^n\sum_{i=1}^{n-l+1}\sum_{k=i}^{i+l-2}2 \\
	&= \sum_{l=2}^n2(l-1)(n-l+1) \\
	&= \sum_{l=1}^{n-1}2l(n-l) \\
	&= 2n\sum_{l=1}^{n-1}l-2\sum_{l=1}^{n-1}l^2 \\
	&= 2n\cdot\frac{n(n-1)}{2}-2\cdot\frac{n(n-1)(2n-1)}{6} \\
	&= n^3-n^2-\frac{2n^3-3n^2+n}{3} \\
	&= \frac{n^3-n}{3}.
\end{align*}

\exercise %15.2-5
Obserwację udowodnimy przez indukcję po długości $n$ wyrażenia $\Phi$.
Jeśli $n=1$, to pojedynczy element wyrażenia $\Phi$ sam w~sobie ma ustalone pełne nawiasowanie -- występuje tu zatem 0 par nawiasów i~podstawa indukcji zachodzi.

Niech teraz $n\ge2$.
Załóżmy, że dla każdego wyrażenia o~$k<n$ elementach w~ich pełnym nawiasowaniu występuje dokładnie $k-1$ par nawiasów.
Rozważmy dowolne pełne nawiasowanie wyrażenia $\Phi$ o~$n$ elementach.
Z~definicji mamy, że istnieje $1<k<n$ takie, że wyrażenie $\Phi$ stanowi iloczyn podwyrażenia o~$k$ elementach i~podwyrażenia o~$n-k$ elementach, oba o~ustalonych pełnych nawiasowaniach.
Z~założenia mamy, że pełne nawiasowania tych podwyrażeń mają, odpowiednio, $k-1$ i~$n-k-1$ par nawiasów.
Dodając do tych liczb dodatkową parę nawiasów, która obejmuje całe wyrażenie $\Phi$, mamy, że łącznie w~pełnym nawiasowaniu $\Phi$ występuje $(k-1)+(n-k-1)+1=n-1$ par nawiasów.
