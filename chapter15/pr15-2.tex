\problem{Estetyczny wydruk} %15-2
Danymi wejściowymi w~naszym rozwiązaniu będzie ciąg $l$ długości słów wejściowych oraz pojemność wiersza $M$.
Przedstawiony algorytm można nietrudno przekształcić tak, aby przyjmował rzeczywisty akapit tekstu i~wypisywał jego słowa podzielone na linie.
Zakładamy ponadto, że każde słowo z~tego akapitu mieści się w~wierszu, tzn.\ $l_i\le M$ dla każdego $i=1$, 2, \dots, $n$.

Zdefiniujmy $\id{extras}[i,j]$, gdzie $i\le j$, jako liczbę zbędnych znaków odstępu na końcu wiersza zawierającego słowa o~numerach od $i$ do $j$, czyli $\id{extras}[i,j]=M-j+i-\sum_{k=i}^jl_k$.
Z~kolei $\id{lc}[i,j]$ niech będzie kosztem wprowadzanym przez wiersz ze słowami od $i$ do $j$.
Przyjmiemy, że $\id{lc}[i,j]=\infty$, jeśli słowa te nie mieszczą się w~wierszu -- taki zabieg sprawi, że podczas minimalizacji tych wartości przepełnione wiersze nie będą wybierane.
Ostatni wiersz, jeśli nie jest przepełniony, wprowadza zerowy koszt, zgodnie ze specyfikacją problemu.
Mamy więc:
\[
	\id{lc}[i,j] = \begin{cases}
		\infty, & \text{jeśli $\id{extras}[i,j]<0$}, \\
		0, & \text{jeśli $j=n$ i~$\id{extras}[i,j]\ge0$}, \\
		(\id{extras[i,j]})^3, & \text{w~pozostałych przypadkach}.
	\end{cases}
\]

Naszym zadaniem jest minimalizacja sumy wartości \id{lc} po wszystkich wierszach, na jakie można podzielić tekst wejściowy.
Rozważmy optymalne rozmieszczenie słów o~numerach od 1 do $j$, gdzie $1\le j\le n$.
Niech $i$ będzie numerem słowa, które rozpoczyna ostatni wiersz w~takim podziale.
Poprzednie wiersze, o~ile istnieją, składają się zatem ze słów od 1 do $i-1$ i~rozmieszczenie ich w~wierszach musi być optymalne, co można uzasadnić za pomocą metody ,,wytnij i~wklej''.
Niech $c[j]$ będzie kosztem optymalnego układu słów od 1 do $j$.
Jeśli ostatni wiersz rozpoczyna słowo o~numerze $i$, to $c[j]=c[i-1]+\id{lc}[i,j]$.
Odpowiednią wartość $i$ minimalizującą $c[j]$ można wyznaczyć, sprawdzając każdy numer od 1 do $j$ i~wybierając optymalny z~nich.
Możemy ponadto przyjąć $c[0]=0$ jako brzegowy przypadek tej rekurencji:
\[
	c[j] = \begin{cases}
		0, & \text{jeśli $j=0$}, \\
		\displaystyle\min_{1\le i\le j}(c[i-1]+\id{lc}[i,j]), & \text{jeśli $j>0$}.
	\end{cases}
\]
Jej wartości możemy przechowywać w~tablicy i~obliczać od lewej do prawej.

Zaobserwujmy jednak, że wiersz może zawierać co najwyżej $\lceil M/2\rceil$ słów.
W~wierszu składającym się z~$j-i+1$ słów o~numerach od $i$ do $j$, jeśli $j-i+1>\lceil M/2\rceil$, to wiadomo, że $\id{lc}[i,j]=\infty$.
Stąd podczas obliczania $c[j]$ można tak naprawdę ograniczyć zakres sprawdzanych numerów $i$ od dołu przez $\max(1,j-\lceil M/2\rceil+1)$.

Zauważmy też, że możemy zrezygnować z~tablic \id{extras} oraz \id{lc} i~pamiętać jedynie potrzebne w~danej chwili wartości.
Posłużymy się w~tym celu ciągiem sum prefiksowych długości słów, który przechowamy w~tablicy $L[0\twodots n]$.
Dokładniej, dla każdego $j=0$, 1, \dots, $n$, $L[j]=\sum_{k=1}^jl_k$.
Zdefiniowane wcześniej $\id{extras}[i,j]$ wynosi wówczas $M-j+i-(L[j]-L[i-1])$ i~wartość tę można wyznaczyć w~czasie stałym.
Podobnie wartości z~tablicy \id{lc} można wyliczać na bieżąco, ponieważ w~danym momencie wykorzystywana jest tylko jedna z~nich.

Do zapamiętywania, na~których pozycjach wejściowego ciągu następuje podział na wiersze, wykorzystamy dodatkową tablicę $p$.
W~trakcie obliczania $c[j]$, jeśli $c[j]=c[i-1]+\id{lc}[i,j]$, to do $p[j]$ wpiszemy $i$.
Dzięki temu będziemy wiedzieć, że ostatni wiersz w~znalezionym rozwiązaniu zawiera słowa o~numerach od $p[n]$ do $n$, przedostatni -- słowa o~numerach od $p[p[n]-1]$ do $p[n]-1$ itd.

\begin{codebox}
\Procname{$\proc{Break-Lines}(l,M)$}
\li	$n\gets\attrib{l}{length}$
\li	$L[0]\gets0$
\li	\For $i\gets1$ \To $n$
\li		\Do $L[i]\gets L[i-1]+l_i$
		\End
\li	$c[0]\gets0$
\li	\For $j\gets1$ \To $n$
\li		\Do $c[j]\gets\infty$
\li			$j_0\gets\max(1,j-\lceil M/2\rceil+1)$
\li			\For $i\gets j_0$ \To $j$
\li				\Do $\id{extras}\gets M-j+i-(L[j]-L[i-1])$
\li					\If $\id{extras}<0$
\li						\Then $\id{lc}\gets\infty$
\li					\ElseIf $j=n$
\li						\Then $\id{lc}\gets0$
\li					\ElseNoIf $\id{lc}\gets\id{extras}^3$
						\End
\li					\If $c[i-1]+\id{lc}<c[j]$
\li						\Then $c[j]\gets c[i-1]+\id{lc}$
\li							$p[j]\gets i$
						\End
				\End
		\End
\li	\Return $c$ i~$p$
\end{codebox}

Analizując liczbę wykonywanych iteracji poszczególnych pętli, można przekonać się o~tym, że czasowa złożoność algorytmu wynosi $O(nM)$, a~pamięciowa -- $\Theta(n)$.
Wypisanie znalezionego rozwiązania w~postaci ciągu numerów słów rozpoczynających kolejne wiersze realizuje wywołanie $\proc{Print-Lines}(p,n)$ poniższej procedury rekurencyjnej.
\begin{codebox}
\Procname{$\proc{Print-Lines}(p,j)$}
\li	\If $p[j]>1$
\li		\Then $\proc{Print-Lines}(p,p[j]-1)$
		\End
\li	wypisz $p[j]$
\end{codebox}
Ponieważ drugi argument jest zmniejszany w~kolejnych wywołaniach rekurencyjnych, to czasem działania tej procedury jest $O(n)$.
