\problem{Bitoniczny problem komiwojażera} %15-1
Podążając za wskazówką z~treści problemu, będziemy przeglądać punkty wejściowe od lewej do prawej, po uprzednim posortowaniu ich po współrzędnych $x$.
Tak uporządkowane punkty oznaczymy przez $p_1$, $p_2$, \dots, $p_n$, a~więc $p_1$ jest punktem wysuniętym najbardziej na lewo, a~$p_n$ punktem wysuniętym najbardziej na prawo.

Dla $1\le i\le j\le n$ rozważmy zbiory $B_{i,j}$ ścieżek zawierających każdy punkt $p_1$, $p_2$, \dots, $p_j$ dokładnie raz, z~wyjątkiem przypadku $i=j$, w~którym $p_j$ występuje dokładnie 2 razy, i~mających postać $\langle p_i,\dots,p_1,\dots,p_j\rangle$, przy czym podścieżka $\langle p_i,\dots,p_1\rangle$ (złożona z~tylko jednego punktu, gdy $i=1$) biegnie po punktach o~malejących indeksach, a~podścieżka $\langle p_1,\dots,p_j\rangle$ (złożona z~tylko jednego punktu, gdy $j=1$) -- po punktach o~rosnących indeksach.
Przez $|pp'|$ oznaczymy odległość euklidesową między punktami $p$ i~$p'$, a~przez $b[i,j]$ -- długość mierzoną odległością euklidesową najkrótszej ścieżki ze zbioru $B_{i,j}$.
Zauważmy, że zbiory $B_{j,j}$ składają się ze ścieżek bitonicznych między punktami $p_1$, $p_2$, \dots, $p_j$.
Rozwiązanie problemu stanowi więc wartość $b[n,n]$.

Niech $\beta_{i,j}$ będzie najkrótszą ścieżką w~$B_{i,j}$.
Oczywiście $b[1,1]=0$, przyjmijmy więc, że $j>1$.
Gdy $i=1$ lub $i<j-1$, bezpośrednio przed $p_j$ na $\beta_{i,j}$ znajduje się punkt $p_{j-1}$.
Podścieżka $\langle p_i,\dots,p_{j-1}\rangle$ musi być najkrótszą ścieżką z~$B_{i,j-1}$; inaczej moglibyśmy ją ,,wyciąć'' i~,,wkleić'' w~jej miejsce podścieżkę z~tego zbioru o~mniejszej długości, uzyskując ścieżkę krótszą niż $\beta_{i,j}$.
Stąd długość $b[i,j]$ ścieżki $\beta_{i,j}$ wynosi $b[i,j-1]+|p_{j-1}p_j|$.
Jeśli z~kolei $i>1$ oraz zachodzi $i=j-1$ lub $i=j$, to punkt $p_j$ jest bezpośrednio poprzedzony pewnym punktem $p_k$, gdzie $k<i$.
Tutaj także ma miejsce optymalna podstruktura -- podścieżka $\langle p_i,\dots,p_k\rangle$ jest odwróconą najkrótszą ścieżką z~$B_{k,i}$, o~czym można się przekonać, stosując metodę ,,wytnij i~wklej''.
W~tym przypadku ścieżka $\beta_{i,j}$ ma długość $b[i,j]=\min_{1\le k<i}(b[k,i]+|p_kp_j|)$.
Na podstawie tej analizy dostajemy następującą zależność rekurencyjną:
\[
	b[i,j] = \begin{cases}
		0, & \text{jeśli $i=j=1$}, \\
		b[i,j-1]+|p_{j-1}p_j|, & \text{jeśli $i=1<j$ lub $i<j-1$}, \\
		\displaystyle\min_{1\le k<i}(b[k,i]+|p_kp_j|), & \text{w~pozostałych przypadkach}.
	\end{cases}
\]

Aby zrekonstruować rozwiązanie, dla każdych $1\le i<j\le n$ obliczymy $r[i,j]$, czyli indeks punktu bezpośrednio poprzedzającego $p_j$ na ścieżce $\beta_{i,j}$.
Poniższy pseudokod wyznacza wartości w~tablicach $b$ i~$r$, wykorzystując programowanie dynamiczne.
\begin{codebox}
\Procname{$\proc{Bitonic-TSP}(p)$}
\li	$n\gets\attrib{p}{length}$
\li	posortuj ciąg punktów wejściowych $p$ rosnąco względem ich współrzędnych $x$ \label{li:bitonic-tsp-sorting}
\li	$b[1,1]\gets0$
\li	\For $j\gets2$ \To $n$
\li		\Do \For $i\gets1$ \To $j$
\li				\Do \If $i=1$ lub $i<j-1$
\li						\Then $b[i,j]\gets b[i,j-1]+|p_{j-1}p_j|$
\li							$r[i,j]\gets j-1$
\li						\Else $b[i,j]\gets\infty$ \label{li:bitonic-tsp-min-begin}
\li							\For $k\gets1$ \To $i-1$
\li								\Do $q\gets b[k,i]+|p_kp_j|$
\li									\If $q<b[i,j]$
\li										\Then $b[i,j]\gets q$
\li											$r[i,j]\gets k$
										\End
								\End
						\End \label{li:bitonic-tsp-min-end}
				\End
		\End
\li	\Return $b$ i~$r$
\end{codebox}

Dla posortowanego ciągu punktów wejściowych $p$ znalezione rozwiązanie wypiszemy, zaczynając od $p_n$, następnie podamy punkty na podścieżce zawierającej $p_{n-1}$, aż do $p_1$, po czym pozostałe punkty z~podścieżki biegnącej w~prawo aż do $p_n$.
\begin{codebox}
\Procname{$\proc{Print-Bitonic-Path}(p,r)$}
\li	$n\gets\attrib{p}{length}$
\li	wypisz $p_n$
\li	wypisz $p_{n-1}$
\li	$\proc{Print-Path}(p,r,n-1,n)$
\end{codebox}
\begin{codebox}
\Procname{$\proc{Print-Path}(p,r,i,j)$}
\li	\If $i<j$ i~($i>1$ lub $r[i,j]>1$)
\li		\Then $\proc{Print-Path}(p,r,i,r[i,j])$
\li			wypisz $p_{r[i,j]}$
		\End
\li	\If $i>j$ i~($j>1$ lub $r[j,i]>1$)
\li		\Then wypisz $p_{r[j,i]}$
\li			$\proc{Print-Path}(p,r,r[j,i],j)$
		\End
\end{codebox}
Wywołanie $\proc{Print-Path}(p,r,i,j)$ wypisuje ścieżkę o~minimalnej długości między punktami $p_i$ i~$p_j$.
W~zależności od wyniku porównania $i$ z~$j$ wypisane zostają punkty z~podścieżki biegnącej w~lewo albo, w~kolejności odwrotnej do ich napotykania, punkty z~podścieżki biegnącej w~prawo.
Procedura stosuje dodatkowe warunki, aby nie dopuścić do podwójnego wypisania punktu $p_1$.

Dla każdego $j>1$ co najwyżej 2 pozycje $b[i,j]$ obliczane są w~algorytmie \proc{Bitonic-TSP} w~wierszach \ref{li:bitonic-tsp-min-begin}\nbendash\ref{li:bitonic-tsp-min-end} przy użyciu $O(n)$ iteracji, a~każda inna pozycja tablicy $b$ wyznaczona zostaje w~czasie stałym.
Ponieważ sortowanie punktów w~wierszu \ref{li:bitonic-tsp-sorting} można wykonać w~czasie $O(n\lg n)$, to czasem działania tego algorytmu jest $\Theta(n^2)$.
Wypisanie znalezionej ścieżki bitonicznej za pomocą procedury \proc{Print-Bitonic-Path} odbywa się w~czasie $\Theta(n)$, gdyż każdy punkt wypisywany jest dokładnie raz, a~na każdy z~nich przypada stała liczba wykonywanych operacji.
