\problem{Przesuwanie pionka} %15-6
Ponumerujmy kolejnymi liczbami całkowitymi od 1 do $n$ wiersze z~dołu do góry, oraz kolumny z~lewej do prawej.
Każde pole szachownicy utożsamimy z~jego położeniem na szachownicy w~postaci pary $\langle i,j\rangle$, gdzie $i$ to numer wiersza tego pola, a~$j$ -- numer jego kolumny.
Zdefiniujmy $g[i,j]$ dla $1\le i$, $j\le n$ jako największą liczbę złotych możliwą do uzyskania, przechodząc pionkiem z~pewnego pola na dolnym brzegu szachownicy na pole $\langle i,j\rangle$, wykonując przy tym tylko dopuszczalne ruchy.
Oczywiście $g[1,j]=0$ dla $j=1$, 2, \dots, $n$.
Maksymalny zysk w~polu $\langle i,j\rangle$ możemy osiągnąć, przechodząc na nie z~pola $\langle i-1,j-1\rangle$ (o~ile $j>1$), $\langle i-1,j\rangle$ albo $\langle i-1,j+1\rangle$ (o~ile $j<n$).
Każda możliwość do sumarycznego zysku dodaje częściowy zysk związany z~przejściem na pole $\langle i,j\rangle$, zdefiniowanym przez funkcję $p$.
Dla $i=2$, 3, \dots, $n$ zachodzi następująca zależność:
\[
	g[i,j] = \max\begin{cases}
		g[i-1,j-1]+p(\langle i-1,j-1\rangle,\langle i,j\rangle), & \text{jeśli $j>1$,} \\
		g[i-1,j]+p(\langle i-1,j\rangle,\langle i,j\rangle), & \\
		g[i-1,j+1]+p(\langle i-1,j+1\rangle,\langle i,j\rangle), & \text{jeśli $j<n$.}
	\end{cases}
\]

W~naszym algorytmie zastosujemy programowanie dynamiczne w~celu obliczenia wartości w~tablicy $g$.
Ponadto w~tablicy $m$ będziemy przechowywać dane ułatwiające rekonstrukcję znalezionej ścieżki pionka.
Pozycja $m[i,j]$ będzie zawierać numer kolumny pola z~wiersza $i-1$, z~którego prowadzi optymalna ścieżka do pola $\langle i,j\rangle$.
Ponadto w~zmiennej $m^*\!$ po zakończeniu działania algorytmu znajdzie się numer kolumny pola na górnym brzegu szachownicy, na którym kończy się optymalna ścieżka pionka.
\begin{codebox}
\Procname{$\proc{Checkerboard}(n,p)$}
\li	\For $j\gets1$ \To $n$
\li		\Do $g[1,j]\gets0$
		\End
\li	\For $i\gets2$ \To $n$
\li		\Do \For $j\gets1$ \To $n$
\li				\Do $g[i,j]\gets g[i-1,j]+p(\langle i-1,j\rangle,\langle i,j\rangle)$
\li					$m[i,j]\gets j$
\li					\If $j>1$ i~$g[i-1,j-1]+p(\langle i-1,j-1\rangle,\langle i,j\rangle)>g[i,j]$
\li						\Then $g[i,j]\gets g[i-1,j-1]+p(\langle i-1,j-1\rangle,\langle i,j\rangle)$
\li							$m[i,j]\gets j-1$
						\End
\li					\If $j<n$ i~$g[i-1,j+1]+p(\langle i-1,j+1\rangle,\langle i,j\rangle)>g[i,j]$
\li						\Then $g[i,j]\gets g[i-1,j+1]+p(\langle i-1,j+1\rangle,\langle i,j\rangle)$
\li							$m[i,j]\gets j+1$
						\End
				\End
		\End
\li	$\id{result}\gets-\infty$
\li	$m^*\!\gets1$
\li	\For $j\gets1$ \To $n$
\li		\Do \If $g[n,j]>\id{result}$
\li				\Then $\id{result}\gets g[n,j]$
\li					$m^*\!\gets j$
				\End
		\End
\li	\Return $\id{result}$, $m$ i~$m^*\!$
\end{codebox}
Wykorzystując zwrócone wartości przez powyższy algorytm, możemy wypisać optymalną ścieżkę jako ciąg odwiedzonych pól, używając w~tym celu wywołania $\proc{Print-Moves}(m,n,m^*\!)$.
Pseudokod tej procedury przedstawiono poniżej.
\begin{codebox}
\Procname{$\proc{Print-Moves}(m,i,j)$}
\li	\If $i>1$
\li		\Then $\proc{Print-Moves}(m,i-1,m[i,j])$
		\End
\li	wypisz ,,$\langle$'' $i$ ,,{}, '' $j$ ,,$\rangle$''
\end{codebox}

Algorytm \proc{Checkerboard} wypełnia tablicę $g$, dla każdej z~$n^2$ jej komórek wykonując operacje w~czasie ograniczonym przez stałą.
Stąd czasem działania algorytmu jest $\Theta(n^2)$.
