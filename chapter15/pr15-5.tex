\problem{Algorytm Viterbiego} %15-5

\subproblem %15-5(a)
Ścieżka w~grafie $G$ z~wierzchołka $v_0$, mająca zadaną etykietę $\langle\sigma_1,\sigma_2,\dots,\sigma_k\rangle$ istnieje wtedy i~tylko wtedy, gdy istnieje ścieżka o~etykiecie $\langle\sigma_2,\dots,\sigma_k\rangle$ z~pewnego sąsiada $v_1$ wierzchołka $v_0$ oraz $\sigma(v_0,v_1)=\sigma_1$.
Innymi słowy, w~problemie tym dla zadanego ciągu etykiet wykorzystywane są podproblemy dla o~1 krótszego sufiksu tego ciągu.

Procedura \proc{Find-Path}, której pseudokod znajduje się poniżej, przyjmuje na wejściu graf $G$ reprezentowany przez tablicę list sąsiedztwa \id{Adj}, wierzchołek początkowy $v$ grafu $G$, tablicę $s$ przechowującą ciąg etykiet oraz początkowy indeks $i$ tej tablicy.
Wywołanie $\proc{Find-Path}(G,v,s,\sigma,i)$ będzie znajdować ścieżkę w~grafie $G$, zaczynającą się w~wierzchołku $v$ oraz mającą etykietę złożoną z~kolejnych elementów z~$s[i\twodots\attrib{s}{length}]$.
Przeglądani będą wszyscy sąsiedzi $w$ wierzchołka $v$ i~w~razie odnalezienia krawędzi $\langle v,w\rangle$ o~etykiecie $\sigma_i$, procedura wywołana zostanie rekurencyjnie dla wierzchołka $w$ oraz ze zwiększonym o~1 indeksem $i$.
W~przypadku odnalezienia ścieżki dla któregokolwiek sąsiada $v$, zostanie ona skonkatenowana z~$v$ i~zwrócona jako wynik procedury.
W~przeciwnym wypadku wynikiem jest specjalna wartość \const{nie-ma-takiej-ścieżki}.
\begin{codebox}
\Procname{$\proc{Find-Path}(G,v,s,\sigma,i)$}
\li	\If $i=\attrib{s}{length}+1$
\li		\Then \Return $\langle v\rangle$
		\End
\li	\For każdy $w\in\id{Adj}[v]$
\li		\Do \If $\sigma(v,w)=s[i]$
\li				\Then $\id{path}\gets\proc{Find-Path}(G,w,s,\sigma,i+1)$
\li					\If $\id{path}\ne\const{nie-ma-takiej-ścieżki}$
\li						\Then \Return $\langle v\rangle\id{path}$ \hspace{11mm}\Comment konkatenacja elementu $v$ z~ciągiem \id{path}
						\End
				\End
		\End
\li	\Return \const{nie-ma-takiej-ścieżki}
\end{codebox}
Aby rozwiązać problem dla grafu $G=\langle V,E\rangle$ z~wyróżnionym wierzchołkiem $v_0\in V$ oraz etykiety przechowywanej w~tablicy $s$, należy wywołać $\proc{Find-Path}(G,v_0,s,\sigma,1)$.

Każdy podproblem, jaki może się pojawić, polega na znalezieniu ścieżki o~zadanej długości $k$, zaczynającej się w~pewnym wierzchołku grafu $G=\langle V,E\rangle$.
Istnieje zatem $k|V|$ możliwych podproblemów, a~podczas rozwiązywania dowolnego z~nich może być sprawdzana krawędź do każdego innego wierzchołka grafu $G$.
Czas działania algorytmu wynosi stąd $O(k|V|^2)$.

\subproblem %15-5(b)
W~tej modyfikacji, zamiast kończyć wyszukiwanie ścieżki w~momencie odnalezienia jakiejkolwiek o~zadanej etykiecie, będziemy sprawdzać możliwe krawędzie wychodzące z~wierzchołka wejściowego.
Optymalna podstruktura wygląda podobnie do tej z~oryginalnej wersji problemu z~punktu (a).
Następujący algorytm opiera się na \proc{Find-Path} i~realizuje opisane podejście.
Zwraca on ścieżkę o maksymalnym możliwym prawdopodobieństwie, a~także samą wartość tego prawdopodobieństwa.
\begin{codebox}
\Procname{$\proc{Viterbi}(G,v,s,\sigma,p,i)$}
\li	\If $i=\attrib{s}{length}+1$
\li		\Then \Return $\langle\langle v\rangle,1\rangle$
		\End
\li	$\id{path}\gets\const{nie-ma-takiej-ścieżki}$
\li	$\id{prob}\gets0$
\li	\For każdy $w\in\id{Adj}[v]$
\li		\Do \If $\sigma(v,w)=s[i]$
\li				\Then $\langle\id{path}',\id{prob}'\rangle\gets\proc{Viterbi}(G,w,s,\sigma,p,i+1)$
\li					\If $p(v,w)\cdot\id{prob}'\ge\id{prob}$
\li						\Then $\id{prob}\gets p(v,w)\cdot\id{prob}'$
\li							$\id{path}\gets\langle v\rangle\id{path}'$ \hspace{8mm}\Comment konkatenacja elementu $v$ z~ciągiem \id{path}$'$
						\End
				\End
		\End
\li	\Return $\langle\id{path},\id{prob}\rangle$
\end{codebox}

Czas działania tego algorytmu wynosi $O(k|V|^2)$, na podstawie identycznej analizy jak w~punkcie (a).
