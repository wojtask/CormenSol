\problem{Planowanie prac} %15-7
Kluczowy w~naszym rozwiązaniu będzie następujący lemat:

\medskip
\noindent\textsf{\textbf{Lemat.}} \textit{W~uporządkowaniu o~maksymalnym zysku prace uszeregowane są według terminów ich ukończenia.}
\begin{proof}
Rozważmy optymalne uszeregowanie prac, w~którym praca $a_i$ jest wykonywana przed pracą $a_j$, ale $d_i>d_j$.
Zamiana $a_i$ z~$a_j$ w~tym uporządkowaniu może jedynie powiększyć zysk, dlatego możemy zamienić każdą taką parę, uzyskując w~wyniku zysk niemniejszy niż przez zamianą.
Ciąg, w~którym nie można już wykonać takiej zamiany, jest uporządkowaniem prac według terminów ich ukończenia.
\end{proof}

Nasz algorytm będzie konstruował optymalny harmonogram prac jako ciąg pewnego podzbioru prac wejściowych.
Aby uzyskać maksymalny sumaryczny zysk, należy wykonać na maszynie prace kolejno z~tego ciągu bezzwłocznie po rozpoczęciu działania maszyny i~bez przerw między pracami.
Zauważmy, że żadna praca o~czasie wykonania dłuższym niż termin jej ukończenia, nie wprowadza zysku do optymalnego harmonogramu, dlatego będziemy zakładać, że algorytm operuje na pracach $a_j$, dla których zachodzi $t_j\le d_j$.

Posortujemy prace według ich terminów tak, aby zachodziło $d_1\le d_2\le\dots\le d_n$.
Z~założenia, że czas wykonania każdej pracy wynosi co najwyżej $n$ jednostek czasu, możemy też każdy termin wykonania $d_i$, który jest większy niż $n^2$, zredukować do $n^2$.
Zdefiniujmy $P_{i,j}$ dla $0\le i\le n$, $0\le j\le d_n$ jako maksymalny zysk możliwy do osiągnięcia przez planowanie prac $a_1$, $a_2$, \dots, $a_i$ w~przedziale czasowym $[0,j]$.
Rozwiązaniem problemu jest wówczas $P_{n,d_n}$, gdyż wykonanie dowolnej pracy po terminie jej ukończenia, a~w~szczególności po $d_n$, nie prowadzi do powiększenia sumarycznego zysku.

Zastanówmy się teraz, jak obliczyć wartości $P_{i,j}$.
Jeśli $i=0$ lub $j=0$, to przyjmiemy $P_{i,j}=0$.
Niech $i\ge1$.
Jeśli $j<t_i$, to praca $a_i$ nie może być wykonana w~przedziale $[0,j]$, dlatego $P_{i,j}=P_{i-1,j}$.
Gdy jednak $j\ge t_i$, to mamy wówczas dwie możliwości -- możemy pominąć pracę $a_i$ w~naszym uporządkowaniu, co daje $P_{i,j}=P_{i-1,j}$, albo włączyć ją do uporządkowania, w~której to sytuacji, aby otrzymać zysk za wykonanie pracy $a_i$, musimy wykonać ją przed terminem jej wykonania, dlatego pozostały przedział czasowy, jaki pozostaje do rozplanowania pozostałych prac to $[0,\min(j,d_i)-t_i]$.
Udowodniony powyżej lemat mówi nam, że pozostałymi pracami, jakie należy rozplanować w~tym przedziale, są $a_1$, $a_2$, \dots, $a_{i-1}$.
Stąd, jeżeli praca $a_i$ będzie włączona do uporządkowania, to zachodzi $P_{i,j}=P_{i-1,\min(j,d_i)-t_i}+p_i$.
Dostajemy zależność rekurencyjną:
\[
	P_{i,j} = \begin{cases}
		0, & \text{jeśli $i=0$ lub $j=0$}, \\
		P_{i-1,j}, & \text{jeśli $i\ge1$, $1\le j<t_i$}, \\
		\max\bigl(P_{i-1,j},P_{i-1,\min(j,d_i)-t_i}+p_i\bigr), & \text{jeśli $i\ge1$, $t_i\le j\le d_n$}.
	\end{cases}
\]

Zauważmy, że tak naprawdę nie trzeba przechowywać równocześnie wszystkich wartości $P_{i,j}$.
Wystarczy w~danym momencie pamiętać wszystkie $d_n+1$ tylko dla ustalonego $i$.
Jeśli $i<n$, to ciąg $P_{i+1,j}$ można uzyskać, iterując zmienną $j$ po coraz mniejszych liczbach od $d_n$ do 0 i~wykorzystując do obliczeń aktualny ciąg $P_{i,j}$ oraz obserwację, że wartość $P_{i+1,j}$ nie zależy od żadnego $P_{i,j'}$, gdzie $j'>j$.
Do zapamiętania wartości $P_{i,j}$ wystarczy zatem $O(n^2)$ komórek pamięci.

Można teraz łatwo ułożyć algorytm wykorzystujący programowanie dynamiczne, który wyznacza wartości $P_{i,j}$ i~zwraca w~wyniku $P_{n,d_n}$\!.
Czas wymagany na policzenie ich wszystkich jest proporcjonalny do $O(nd_n)=O(n^3)$.
Jeśli od algorytmu oczekujemy także wypisania znalezionego rozwiązania, to należy użyć dodatkowej tablicy, w~której zapamiętana zostanie informacja o~tym, czy praca $a_i$ jest włączana do optymalnego uporządkowania w~przedziale czasowym $[0,j]$, a~następnie posłużenie się tą tablicą w~celu zrekonstruowania tego uporządkowania.

Założenie, z~którego korzystaliśmy w~zadaniu pozwoliło na zaprojektowanie wielomianowego algorytmu dla tego problemu.
W~ogólności, czyli przy dopuszczeniu dowolnie dużych czasów wykonania, problem staje się \singledash{NP}{zupełny} i~prawdopodobnie nie istnieje dla niego algorytm o~czasie niższym niż wykładniczy względem rozmiaru danych wejściowych.
