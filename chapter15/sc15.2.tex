\subchapter{Mnożenie ciągu macierzy}

\exercise %15.2-1
Posługując się algorytmami \proc{Matrix-Chain-Order} i~\proc{Print-Optimal-Parens}, dostajemy, że dla ciągu macierzy $\langle A_1,\dots,A_6\rangle$ o~zadanych rozmiarach, optymalnym nawiasowaniem jest $((A_1A_2)((A_3A_4)(A_5A_6)))$.
Mnożąc macierze zgodnie z~tym nawiasowaniem, wykonamy 2010 mnożeń skalarnych.

\exercise %15.2-2
Nasz algorytm oprzemy na \proc{Print-Optimal-Parens}, ale zamiast wypisywać nawiasowanie, będziemy przekazywać macierze zwrócone przez wywołania rekurencyjne do procedury odpowiedzialnej za rzeczywiste ich pomnożenie.
\begin{codebox}
\Procname{$\proc{Matrix-Chain-Multiply}(A,s,i,j)$}
\li	\If $i=j$
\li		\Then \Return $A_i$
		\End
\li	\Return $\proc{Matrix-Multiply}($
\zi	\>\>\> $\proc{Matrix-Chain-Multiply}(A,s,i,s[i,j]),$
\zi	\>\>\> $\proc{Matrix-Chain-Multiply}(A,s,s[i,j]+1,j))$
\end{codebox}

\exercise %15.2-3
\exercise %15.2-4
\exercise %15.2-5
