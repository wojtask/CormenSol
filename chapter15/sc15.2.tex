\subchapter{Mnożenie ciągu macierzy}

\exercise %15.2-1
Posługując się algorytmami \proc{Matrix-Chain-Order} i~\proc{Print-Optimal-Parens}, dostajemy, że dla ciągu macierzy $\langle A_1,\dots,A_6\rangle$ o~zadanych rozmiarach, optymalnym nawiasowaniem jest $((A_1A_2)((A_3A_4)(A_5A_6)))$.
Mnożąc macierze zgodnie z~tym nawiasowaniem, wykonamy 2010 mnożeń skalarnych.

\exercise %15.2-2
Nasz algorytm oprzemy na \proc{Print-Optimal-Parens}, ale zamiast wypisywać nawiasowanie, będziemy przekazywać macierze zwrócone przez wywołania rekurencyjne do procedury odpowiedzialnej za rzeczywiste ich pomnożenie.
\begin{codebox}
\Procname{$\proc{Matrix-Chain-Multiply}(A,s,i,j)$}
\li	\If $i=j$
\li		\Then \Return $A_i$
		\End
\li	\Return $\proc{Matrix-Multiply}($
\zi	\>\>\> $\proc{Matrix-Chain-Multiply}(A,s,i,s[i,j]),$
\zi	\>\>\> $\proc{Matrix-Chain-Multiply}(A,s,s[i,j]+1,j))$
\end{codebox}

\exercise %15.2-3
Udowodnimy przez indukcję, że istnieje stała $c>0$ taka, że $P(n)\ge c2^n$ dla każdego $n\ge1$.

Okazuje się, że w~celu pokazania tego oszacowania na podstawę indukcji musimy przyjąć $n=1$, 2, 3, 4.
Mamy:
\begin{align*}
	P(1) &= 1 \ge c2^1 \quad\Leftrightarrow\quad c \le 1/2, \\
	P(2) &= P(1)P(1) = 1 \ge c2^2 \quad\Leftrightarrow\quad c \le 1/4, \\
	P(3) &= P(1)P(2)+P(2)P(1) = 1+1 = 2 \ge c2^3 \quad\Leftrightarrow\quad c \le 1/4, \\
	P(4) &= P(1)P(3)+P(2)P(2)+P(3)P(1) = 2+1+2 = 5 \ge c2^4 \quad\Leftrightarrow\quad c \le 5/16.
\end{align*}
Największym $c$ spełniającym każdą powyższą nierówność jest $c=1/4$.

Niech teraz $n\ge5$.
Dla każdego $i=1$, \dots, $n-1$ przyjmiemy, że zachodzi $P(i)\ge c2^i$.
Mamy:
\[
	P(n) = \sum_{k=1}^{n-1}P(k)P(n-k) \ge \sum_{k=1}^{n-1}c2^kc2^{n-k} = \sum_{k=1}^{n-1}c^22^n = c^22^n(n-1).
\]
Nierówność $c^22^n(n-1)\ge c2^n$ będzie zachodzić, o~ile $c\ge1/(n-1)$.
A~zatem najmniejszą wartością $c$ spełniającą tę nierówność dla wszystkich $n\ge5$, jest $c=1/4$.
Widać więc, że wybór takiego $c$ pozwala spełnić oszacowanie $P(n)\ge c2^n$ dla każdego $n$, co oznacza, że $P(n)=\Omega(2^n)$.

\exercise %15.2-4
\exercise %15.2-5
