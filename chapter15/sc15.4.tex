\subchapter{Najdłuższy wspólny podciąg}

\exercise %15.4-1
Dla zadanych ciągów NWP wyznaczonym przez procedury \proc{LCS-Length} i~\proc{Print-LCS} jest $\langle1,0,0,1,1,0\rangle$.
Poza nim istnieje jeszcze 7 innych NWP tych ciągów:
\begin{gather*}
	\langle0,0,1,0,1,0\rangle, \langle0,0,1,0,1,1\rangle, \langle0,0,1,1,0,1\rangle, \langle0,1,0,1,0,1\rangle, \\
	\langle1,0,1,0,1,0\rangle, \langle1,0,1,0,1,1\rangle, \langle1,0,1,1,0,1\rangle.
\end{gather*}

\exercise %15.4-2
Poniższy pseudokod stanowi implementację zmodyfikowanej wersji procedury \proc{Print-LCS}, która wypisuje NWP ciągów $X$, $Y$ bez korzystania z~tablicy $b$.
\begin{codebox}
\Procname{$\proc{Print-LCS}'(c,X,Y,i,j)$}
\li	\If $i=0$ lub $j=0$
\li		\Then \Return
		\End
\li	\If $x_i=y_j$
\li		\Then $\proc{Print-LCS}'(c,X,Y,i-1,j-1)$
\li			wypisz $x_i$
\li		\ElseIf $c[i,j]=c[i-1,j]$
\li			\Then $\proc{Print-LCS}'(c,X,Y,i-1,j)$
\li		\ElseNoIf $\proc{Print-LCS}'(c,X,Y,i,j-1)$
		\End
\end{codebox}

\exercise %15.4-3
\exercise %15.4-4
\exercise %15.4-5
\exercise %15.4-6
