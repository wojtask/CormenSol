\subchapter{Najdłuższy wspólny podciąg}

\exercise %15.4-1
Dla zadanych ciągów NWP wyznaczonym przez procedury \proc{LCS-Length} i~\proc{Print-LCS} jest $\langle1,0,0,1,1,0\rangle$.
Poza nim istnieje jeszcze 7 innych NWP tych ciągów:
\begin{gather*}
	\langle0,0,1,0,1,0\rangle, \langle0,0,1,0,1,1\rangle, \langle0,0,1,1,0,1\rangle, \langle0,1,0,1,0,1\rangle, \\
	\langle1,0,1,0,1,0\rangle, \langle1,0,1,0,1,1\rangle, \langle1,0,1,1,0,1\rangle.
\end{gather*}

\exercise %15.4-2
Poniższy pseudokod stanowi implementację zmodyfikowanej wersji procedury \proc{Print-LCS}, która wypisuje NWP ciągów $X$, $Y$ bez korzystania z~tablicy $b$.
\begin{codebox}
\Procname{$\proc{Print-LCS}'(c,X,Y,i,j)$}
\li	\If $i=0$ lub $j=0$
\li		\Then \Return
		\End
\li	\If $x_i=y_j$
\li		\Then $\proc{Print-LCS}'(c,X,Y,i-1,j-1)$
\li			wypisz $x_i$
\li		\ElseIf $c[i,j]=c[i-1,j]$
\li			\Then $\proc{Print-LCS}'(c,X,Y,i-1,j)$
\li		\ElseNoIf $\proc{Print-LCS}'(c,X,Y,i,j-1)$
		\End
\end{codebox}

\exercise %15.4-3
W~naszej implementacji będziemy obliczać kolejne wartości rekurencyjnie bezpośrednio ze wzoru (15.14), ale z~wykorzystaniem tablicy $c[1\twodots m,1\twodots n]$ i~mechanizmu spamiętywania.
Aby zaznaczyć, że dane pole tablicy $c$ nie zostało jeszcze obliczone, użyjemy specjalnej wartości $\infty$.
Pierwsza z~procedur inicjalizuje tablicę $c$, po czym wywołuje właściwy algorytm odpowiedzialny za obliczenie wartości $c[m,n]$.
\begin{codebox}
\Procname{$\proc{Memoized-LCS-Length}(X,Y)$}
\li	$m\gets\attrib{X}{length}$
\li	$n\gets\attrib{Y}{length}$
\li	\For $i\gets0$ \To $m$
\li		\Do \For $j\gets0$ \To $n$
\li				\Do $c[i,j]\gets\infty$ \label{li:memoized-lcs-length-init}
				\End
		\End
\li	\Return $\proc{Lookup-LCS}(c,X,Y,m,n)$
\end{codebox}
\begin{codebox}
\Procname{$\proc{Lookup-LCS}(c,X,Y,i,j)$}
\li	\If $c[i,j]<\infty$
\li		\Then \Return $c[i,j]$
		\End
\li	\If $i=0$ lub $j=0$
\li		\Then $c[i,j]\gets0$
\li		\ElseIf $x_i=y_j$
\li			\Then $c[i,j]\gets\proc{Lookup-LCS}(c,X,Y,i-1,j-1)+1$
\li		\ElseNoIf $c[i,j]\gets\max(\proc{Lookup-LCS}(c,X,Y,i,j-1),\proc{Lookup-LCS}(c,X,Y,i-1,j))$
		\End
\li	\Return $c[i,j]$
\end{codebox}

Każde z~$(m+1)(n+1)$ pól tablicy $c$ zostaje zainicjalizowane w~wierszu \ref{li:memoized-lcs-length-init}, a~następnie zmodyfikowane przez jedno wywołanie procedury \proc{Lookup-LCS}.
Wywołania procedury \proc{Lookup-LCS} możemy podzielić na dwa typy:
\begin{enumerate}
	\item wywołania, w~których $c[i,j]=\infty$, oraz
	\item wywołania, w~których $c[i,j]<\infty$.
\end{enumerate}
Wywołań pierwszego typu jest dokładnie $\Theta(mn)$, jedno na każde pole tablicy $c$.
Wszystkie wywołania drugiego typu pojawiają się jako rekurencyjne wywołania w~wywołaniach pierwszego typu.
Kiedy w~danym wywołaniu \proc{Lookup-LCS} pojawiają się wywołania rekurencyjne, jest ich $\Theta(1)$, dlatego łącznie wywołań drugiego typu jest $\Theta(mn)$.
Każde wywołanie drugiego typu zabiera czas $\Theta(1)$, a~każde wywołanie pierwszego typu jest wykonywane w~czasie $O(1)$ plus czas spędzany na obliczenia rekurencyjne.
Dlatego łączny czas wykonania algorytmu \proc{Memoized-LCS-Length} wynosi $\Theta(mn)$.

\exercise %15.4-4
Zauważmy, że do obliczenia $c[i,j]$ w~procedurze \proc{LCS-Length} wykorzystywane są wartości $c[i-1,j]$, $c[i,j-1]$ lub $c[i-1,j-1]$.
Możemy więc utrzymywać tylko 2 wiersze tablicy $c$ -- wiersz aktualnie obliczany oraz wiersz bezpośrednio go poprzedzający.
Po każdej iteracji poprzedni wiersz będzie nadpisywany wartościami z~bieżącego wiersza, aby przygotować tablicę do kolejnej iteracji.
Aby tablica $c$ była rozmiaru $2\cdot\min(m,n)$, przed jej utworzeniem należy jeszcze sprawdzić, czy $n\le m$.
Jeśli nie, to wystarczy uruchomić procedurę z~ciągami $X$ i~$Y$ zamienionymi miejscami.

Można jednak jeszcze bardziej ograniczyć zapotrzebowanie procedury na pamięć.
Dwuwymiarowa tablica $c$ może zostać zamieniona na jednowymiarową tablicę $C[1\twodots n]$, która reprezentować będzie aktualnie obliczany wiersz tablicy $c$ w~trakcie działania procedury.
W~momencie obliczania $c[i,j]$ reprezentowanego przez $C[j]$, komórka $C[j]$ przechowuje wartość $c[i-1,j]$ z~poprzedniego wiersza.
Jeśli w~osobnej zmiennej $p$ zapamiętamy $c[i-1,j-1]$, czyli wartość $C[j-1]$ przed jej aktualizacją, to będziemy dysponować wszystkimi danymi potrzebnymi do obliczenia $C[j]$.
Jeśli $x_i=y_j$, to $C[j]$ ustawione zostanie na $p+1$, co odpowiada wierszowi 10 z~oryginalnej procedury \proc{LCS-Length}.
W~przeciwnym przypadku do $C[j]$ wpisane zostanie $\max(C[j],C[j-1])$, co odpowiada wierszom \doubledash{12}{16}.
Ostatnim krokiem w~bieżącej iteracji pętli jest aktualizacja zmiennej $p$ na wartość znajdującą się w~$C[j]$ przed modyfikacją tej komórki.
Dzięki tak wprowadzonym zmianom w~procedurze wykorzystujemy $\min(m,n)+\Theta(1)$ pamięci.

\exercise %15.4-5


\exercise %15.4-6
