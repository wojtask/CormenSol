\subchapter{Najdłuższy wspólny podciąg}

\exercise %15.4-1
Dla zadanych ciągów NWP wyznaczonym przez procedury \proc{LCS-Length} i~\proc{Print-LCS} jest $\langle1,0,0,1,1,0\rangle$.
Poza nim istnieje jeszcze 7 innych NWP tych ciągów:
\begin{gather*}
	\langle0,0,1,0,1,0\rangle, \langle0,0,1,0,1,1\rangle, \langle0,0,1,1,0,1\rangle, \langle0,1,0,1,0,1\rangle, \\
	\langle1,0,1,0,1,0\rangle, \langle1,0,1,0,1,1\rangle, \langle1,0,1,1,0,1\rangle.
\end{gather*}

\exercise %15.4-2
\exercise %15.4-3
\exercise %15.4-4
\exercise %15.4-5
\exercise %15.4-6
