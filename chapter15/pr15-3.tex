\problem{Odległość redakcyjna} %15-3

\subproblem %15-3(a)
Podobnie jak w~problemie NWP, będziemy korzystać z~notacji $X_i$ jako \singledash{$i$}{tego} prefiksu słowa $x$ oraz $Y_j$ jako \singledash{$j$}{tego} prefiksu słowa $y$.
W~podproblemach, jakie się pojawią, będziemy wyznaczać odległość redakcyjną prefiksów $X_i$ i~$Y_j$.

Niech $c[i,j]$ będzie optymalnym kosztem przekształcenia $X_i$ do $Y_j$.
Załóżmy, że $i$, $j>0$ i~że znana jest ostatnia operacja w~tym przekształceniu.
Jeśli operacją tą było kopiowanie lub zastąpienie znaku innym, to problemem do rozwiązania pozostaje znalezienie odległości redakcyjnej $X_{i-1}$ i~$Y_{j-1}$.
Można to stwierdzenie uzasadnić rozumowaniem ,,wytnij i~wklej''.
Zatem $c[i,j]=c[i-1,j-1]+\mathrm{koszt}(\text{skopiuj})$ w~przypadku, gdy znak $x[i]$ został skopiowany  do $y[j]$ oraz $c[i,j]=c[i-1,j-1]+\mathrm{koszt}(\text{zastąp})$, jeśli $x[i]$ został zastąpiony przez inny znak $y[j]$.
W~przypadku usuwania znaku podproblem stanowi wyznaczenie odległości redakcyjnej między $X_{i-1}$ a~$Y_j$, skąd $c[i,j]=c[i-1,j]+\mathrm{koszt}(\text{usuń})$.
Podobnie, jeśli ostatnio wstawiany był nowy znak, to podproblemem jest odległość $X_i$ od $Y_{j-1}$ i~wówczas $c[i,j]=c[i,j-1]+\mathrm{koszt}(\text{wstaw})$.
Do zamiany znaków doszło, gdy $x[i]=y[j-1]$ oraz $x[i-1]=y[j]$, o~ile $i$, $j\ge2$.
Wystarczy w~tym przypadku obliczyć odległość między słowami $X_{i-2}$ i~$Y_{j-2}$, a~następnie dodać do niej koszt operacji zamiany: $c[i,j]=c[i-2,j-2]+\mathrm{koszt}(\text{zamień})$.
W~końcu, jeśli ostatnią operacją było wyrzucenie reszty słowa $X_i$, to znaczy to, że konwersja z~$X_m$ do $Y_n$ została zakończona, czyli musi być $i=m$ oraz $j=n$.
Krok ten mógł zostać wykonany, gdy słowami wejściowymi był jakikolwiek prefiks $X_k$, gdzie $n<k<m$, oraz $Y_m$ -- mamy zatem $c[m,n]=\min_{n<k<m}(c[k,n])+\mathrm{koszt}(\text{wyrzuć})$.

Gdy $i$, $j>0$, to prawdziwa jest następująca zależność:
\[
	c[i,j] = \min\begin{cases}
		c[i-1,j-1]+\mathrm{koszt}(\text{skopiuj}), & \text{jeśli $x[i]=y[j]$}, \\
		c[i-1,j-1]+\mathrm{koszt}(\text{zastąp}), & \text{jeśli $x[i]\ne y[j]$}, \\
		c[i-1,j]+\mathrm{koszt}(\text{usuń}), \\
		c[i,j-1]+\mathrm{koszt}(\text{wstaw}), \\
		c[i-2,j-2]+\mathrm{koszt}(\text{zamień}), & \text{jeśli $i$, $j\ge2$, $x[i]=y[j-1]$ i~$x[i-1]=y[j]$}, \\
		\displaystyle\min_{n<k<m}(c[k,n])+\mathrm{koszt}(\text{wyrzuć}), & \text{jeśli $i=m$ i~$j=n$}.
	\end{cases}
\]

Rozważmy teraz sytuacje z~$i=0$ lub $j=0$.
Jeśli $i=0$, to prefiks $X_i$ jest słowem pustym i~przekształcenie go do słowa $Y_j$ polega na wykonaniu $j$ operacji wstawienia znaku, czyli $c[0,j]=j\cdot\mathrm{koszt}(\text{wstaw})$.
Podobnie, jeśli $j=0$, to słowo $Y_j$ jest puste i~można je uzyskać poprzez \singledash{$i$}{krotne} usunięcie znaku ze słowa $X_i$, czyli $c[i,0]=i\cdot\mathrm{koszt}(\text{usuń})$.

Poniższy pseudokod wypełnia tablicę $c$ od najwyższego do najniższego wiersza oraz od lewej do prawej w~obrębie wierszy.
Równocześnie konstruowana jest także tablica $\id{op}[1\twodots m,1\twodots n]$, w~której po zakończeniu działania algorytmu na pozycji $\id{op}[i,j]$ znajdzie się ostatnia operacja użyta do przekształcenia $X_i$ do $Y_j$.

\begin{codebox}
\Procname{$\proc{Edit-Distance}(x,y,m,n)$}
\li	\For $i\gets0$ \To $m$
\li		\Do $c[i,0]\gets i\cdot\mathrm{koszt}(\text{usuń})$
\li			$\id{op}[i,0]\gets{}$,,usuń''
		\End
\li	\For $j\gets1$ \To $n$
\li		\Do $c[0,j]\gets j\cdot\mathrm{koszt}(\text{wstaw})$
\li			$\id{op}[0,j]\gets{}$,,wstaw '' $y[j]$
		\End
\li	\For $i\gets1$ \To $m$
\li		\Do \For $j\gets1$ \To $n$
\li				\Do $c[i,j]\gets\infty$
\li					\If $x[i]=y[j]$
\li						\Then $c[i,j]\gets c[i-1,j-1]+\mathrm{koszt}(\text{skopiuj})$
\li							$\id{op}[i,j]\gets{}$,,skopiuj''
						\End
\li					\If $x[i]\ne y[j]$ i~$c[i-1,j-1]+\mathrm{koszt}(\text{zastąp})<c[i,j]$
\li						\Then $c[i,j]\gets c[i-1,j-1]+\mathrm{koszt}(\text{zastąp})$
\li							$\id{op}[i,j]\gets{}$,,zastąp przez '' $y[j]$
						\End
\li					\If $c[i-1,j]+\mathrm{koszt}(\text{usuń})<c[i,j]$
\li						\Then $c[i,j]\gets c[i-1,j]+\mathrm{koszt}(\text{usuń})$
\li							$\id{op}[i,j]\gets{}$,,usuń''
						\End
\li					\If $c[i,j-1]+\mathrm{koszt}(\text{wstaw})<c[i,j]$
\li						\Then $c[i,j]\gets c[i,j-1]+\mathrm{koszt}(\text{wstaw})$
\li							$\id{op}[i,j]\gets{}$,,wstaw '' $y[j]$
						\End
\li					\If $i\ge2$ i~$j\ge2$ i~$x[i]=y[j-1]$ i~$x[i-1]=y[j]$
\zi	\phantom{\kw{if} $i\ge2$} i~$c[i-2,j-2]+\mathrm{koszt}(\text{zamień})<c[i,j]$
\li						\Then $c[i,j]\gets c[i-2,j-2]+\mathrm{koszt}(\text{zamień})$
\li							$\id{op}[i,j]\gets{}$,,zamień''
						\End
				\End
		\End
\li	\For $i\gets0$ \To $m-1$
\li		\Do \If $c[i,n]+\mathrm{koszt}(\text{wyrzuć})<c[m,n]$
\li				\Then $c[m,n]\gets c[i,n]+\mathrm{koszt}(\text{wyrzuć})$
\li					$\id{op}[m,n]\gets{}$,,wyrzuć'' $i$
				\End
		\End
\li	\Return $c$ i~\id{op}
\end{codebox}

\begin{codebox}
\Procname{$\proc{Op-Sequence}(\id{op},i,j)$}
\li	\If $i=0$ i~$j=0$
\li		\Then \Return
		\End
\li	\If $\id{op}[i,j]={}$,,skopiuj'' lub $\id{op}[i,j]={}$,,zastąp''
\li		\Then $i'\gets i-1$
\li			$j'\gets j-1$
\li		\ElseIf $\id{op}[i,j]={}$,,skopiuj''
\li			\Then $i'\gets i-2$
\li				$j'\gets j-2$
\li		\ElseIf $\id{op}[i,j]={}$,,usuń''
\li			\Then $i'\gets i-1$
\li				$j'\gets j$
\li		\ElseIf $\id{op}[i,j]={}$,,wstaw''
\li			\Then $i'\gets i$
\li				$j'\gets j-1$
\li		\ElseNoIf $i'\gets\id{op}[i,j][8\twodots\attrib{\id{op}[i,j]}{length}]$
		\End
\li			\phantom{\kw{else}} $j'\gets j$
\li	$\proc{Op-Sequence}(\id{op},i',j')$
\li	wypisz $\id{op}[i,j]$
		\End
\end{codebox}

\subproblem %15-3(b)
Możemy sprowadzić problem optymalnego uliniowienia do problemu odległości redakcyjnej w~następujący sposób.
Załóżmy, że $x'$ i~$y'$ są wynikowymi sekwencjami DNA powstałymi w~wyniku optymalnego uliniowienia, odpowiednio, sekwencji $x$ i~$y$.
Jeśli $x'[j]=y'[j]$, to w~problemie odległości redakcyjnej możemy tę sytuację rozumieć jako skopiowanie znaku $x'[j]$.
Jeśli $x'[j]\ne y'[j]$ i~żaden z~tych znaków nie jest spacją, to mamy tutaj sytuację po zastąpieniu znaku $x'[j]$ przez znak $y'[j]$.
W~końcu, spację na pozycję $x'[j]$ można uzyskać przez wykonanie operacji wstawienia znaku $y'[j]$, a~spację na pozycji $y'[j]$ -- przez wykonanie operacji usunięcia znaku $x'[j]$.

Teraz wystarczy jeszcze przypisać odpowiednie koszty dla poszczególnych operacji elementarnych.
W~problemie optymalnego uliniowienia odpowiednio zdefiniowana ocena jest maksymalizowana, zaś problem odległości redakcyjnej ma na celu minimalizację odpowiadającego tej ocenie kosztu.
Wystarczy więc jako koszt wziąć liczbę przeciwną do odpowiadającej mu oceny.
Poszczególne operacje wartościujemy następująco:
\begin{align*}
	\mathrm{koszt}(\text{skopiuj}) &= -1, \\
	\mathrm{koszt}(\text{zamień}) &= +1, \\
	\mathrm{koszt}(\text{usuń}) &= +2, \\
	\mathrm{koszt}(\text{wstaw}) &= +2.
\end{align*}
Operacje zamień i~wyrzuć nie są wykorzystywane, więc na ich koszt możemy przyjąć wartość $\infty$.
