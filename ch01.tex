\chapter{Rola algorytmów w~obliczeniach}

\subchapter{Algorytmy}

\exercise %1.1-1
\begin{description}
	\item[Sortowanie:] Jest to problem bardzo powszechny -- znajduje zastosowanie w~wielu zagadnieniach obliczeniowych. Najczęstszym powodem sortowania danych jest przygotowanie ich do dalszego przetwarzania, które wówczas jest na ogół bardziej efektywne.
	\item[Najlepsza kolejność mnożenia macierzy:] Problem występuje podczas wyznaczania transformacji graficznych (np.\ skalowań, obrotów); przekształcenia te opisane są za pomocą macierzy, a~ich składanie oznacza obliczanie iloczynu tych macierzy.
	\item[Otoczka wypukła:] Mając zbiór wbitych w~ziemię palików, chcemy otoczyć pewien obszar siatką ogrodzeniową opierając ją na niektórych palikach tak, by obszar ten zmaksymalizować.
\end{description}

\exercise %1.1-2
Miary efektywności algorytmu inne niż jego szybkość:
\begin{itemize}
	\item zużycie pamięci (operacyjnej i~masowej);
	\item stopień wykorzystania systemu operacyjnego;
	\item efektywność dostępu do bazy danych;
	\item stopień eksploatacji łącza sieciowego;
	\item dostosowanie do konkretnej architektury sprzętowo-programowej;
	\item efektywność działania w~architekturze równoległej lub rozproszonej.
\end{itemize}

\exercise %1.1-3
Poniżej zestawiono zalety i~wady listy dwukierunkowej w~porównaniu ze zwykłą tablicą.

\bigskip
\noindent\textbf{Zalety:}
\begin{itemize}
	\item przy definiowaniu listy nie trzeba z~góry znać jej maksymalnej pojemności, jak to jest w~przypadku definiowania tablicy;
	\item lista jest elastyczna -- może się dowolnie powiększać i~kurczyć w~trakcie wykonywania na niej operacji wstawiania i~usuwania;
	\item wstawianie i~usuwanie elementów z~dowolnej pozycji listy odbywa się w~czasie stałym.
\end{itemize}
\textbf{Wady:}
\begin{itemize}
	\item nie można uzyskać dostępu do dowolnego elementu listy w~stałym czasie;
	\item lista potrzebuje nieco więcej pamięci niż tablica (oprócz danych pamiętane są wskaźniki na poprzedni i~następny element listy);
	\item w~przeciwieństwie do tablicy lista na ogół nie zajmuje spójnego obszaru pamięci, co jest istotne na przykład w~systemach ze stronicowaną pamięcią wirtualną.
\end{itemize}

\exercise %1.1-4
\noindent\textbf{Podobieństwa:}
\begin{itemize}
	\item oba są problemami grafowymi;
	\item oba mają na celu minimalizację pewnej ścieżki w~grafie.
\end{itemize}
\textbf{Różnice:}
\begin{itemize}
	\item w~problemie najkrótszej ścieżki poszukuje się minimalnej ścieżki między dwoma wierzchołkami, a~w~problemie komiwojażera -- minimalnego cyklu uwzględniającego wszystkie wierzchołki grafu (minimalny cykl Hamiltona);
	\item problem najkrótszej ścieżki jest wielomianowy (istnieje dla niego szybki algorytm), podczas gdy problem komiwojażera jest \NPclass-zupełny (prawdopodobnie nie istnieje efektywny algorytm rozwiązujący ten problem).
\end{itemize}

\exercise %1.1-5
% W~wielu problemach należy podjąć inne działania w~zależności od obecności ustalonego elementu w~pewnym zbiorze. Oczekujemy zatem dokładnego wyniku od problemu wyszukiwania liniowego lub binarnego.
Problemem o~pożądanym rozwiązaniu dokładnym jest np.\ wyszukiwanie w~nieposortowanej tablicy. Obecność jednego elementu w~tablicy jest na ogół niezależna od obecności innych, zatem po sprawdzeniu kilku z~nich, jeśli nie odnaleziono poszukiwanego elementu, nadal nie można przewidzieć, czy znajduje się on w~tablicy.

Znalezienie rozwiązania przybliżonego jest natomiast wystarczające w~wielu praktycznych zastosowaniach dla problemu komiwojażera.

\subchapter{Algorytmy jako technologia}

\exercise %1.2-1
Przykładem aplikacji, w~której wykorzystywane są różnorodne algorytmy, jest współczesna gra komputerowa. Jej silnik grafiki trójwymiarowej jest zaawansowanym środowiskiem, w~którym stosowane są algorytmy geometrii obliczeniowej i~renderowania grafiki 3D, jak również wiele algorytmów numerycznych do wyznaczania interpolacji oraz algorytmy grafowe. Także w~dziedzinie sztucznej inteligencji opracowano wiele zaawansowanych algorytmów. Ponadto powszechne problemy, takie jak wyszukiwanie elementu w~tablicy czy sortowanie, są rozwiązywane w~niemal każdej aplikacji.

\exercise %1.2-2
Jeśli $n$ jest liczbą naturalną, to nierówność $8n^2<64n\lg n$ jest spełniona dla $2\le n\le43$. Tablice o~takich rozmiarach porządkowane są przez sortowanie przez wstawianie szybciej niż przez sortowanie przez scalanie.

\exercise %1.2-3
Najmniejszą dodatnią liczbą całkowitą $n$ spełniającą nierówność $100n^2<2^n$ jest $n=15$.

\problems

\problem{Porównanie czasów działania} %1-1
W~tabeli~\ref{tab:1-1} zebrano wyznaczone wartości. Liczby przekraczające $10^7$ zostały podane w~przybliżeniu. Przyjęto, że na każdy miesiąc przypada 30 dni, a~na każdy rok 365 dni.

\begin{table}[ht]
	\begin{center}
		\[
			\begin{array}{c|c|c|c|c|c|c|c}
				&1&1&1&1&1&1&1 \\
				\raisebox{1.5ex}[0cm][0cm]{$f(n)$} & \text{sekunda} & \text{minuta} & \text{godzina} & \text{dzień} & \text{miesiąc} & \text{rok} & \text{wiek} \\
				\hline
				\lg n & 2^{10^6} & 2^{6\cdot10^7} & 2^{3{,}6\cdot10^9} & 2^{8{,}64\cdot10^{10}} & 2^{2{,}59\cdot10^{12}} & 2^{3{,}15\cdot10^{13}} & 2^{3{,}15\cdot10^{15}} \\
				\hline
				\sqrt{n} & 10^{12} & 3{,}6\cdot10^{15} & 1{,}3\cdot10^{19} & 7{,}46\cdot10^{21} & 6{,}72\cdot10^{24} & 9{,}95\cdot10^{26} & 9{,}95\cdot10^{30} \\
				\hline
				n & 10^6 & 6\cdot10^7 & 3{,}6\cdot10^9 & 8{,}64\cdot10^{10} & 2{,}59\cdot10^{12} & 3{,}15\cdot10^{13} & 3{,}15\cdot10^{15} \\
				\hline
				n\lg n & 62746 & 2{,}8\cdot10^6 & 1{,}33\cdot10^8 & 2{,}76\cdot10^9 & 7{,}19\cdot10^{10} & 7{,}98\cdot10^{11} & 6{,}86\cdot10^{13} \\
				\hline
				n^2 & 1000 & 7745 & 60000 & 293938 & 1{,}61\cdot10^6 & 5{,}62\cdot10^6 & 5{,}62\cdot10^7 \\
				\hline
				n^3 & 100 & 391 & 1532 & 4420 & 13736 & 31593 & 146645 \\
				\hline
				2^n & 19 & 25 & 31 & 36 & 41 & 44 & 51 \\
				\hline
				n! & 9 & 11 & 12 & 13 & 15 & 16 & 17
			\end{array}
		\]
	\end{center}
	\caption{Ograniczenia rozmiarów problemów.} \label{tab:1-1}
\end{table}

\endinput
