\subchapter{Przywracanie własności kopca}

\exercise %6.2-1
Rys.\ \ref{fig:6.2-1} przedstawia działanie procedury $\proc{Max-Heapify}(A,3)$.
\begin{figure}[!ht]
	\centering \begin{tikzpicture}[
	pic matrix/.style = {matrix of nodes, column sep=15mm, row sep=5mm, draw=none, fill=none, rectangle},
	level/.style = {level distance=10mm, sibling distance=60mm/2^#1},
	every node/.style = {align=center, inner sep=-1pt, minimum size=5mm, circle, draw, fill=black!20},
	every label/.style = {font=\footnotesize, minimum size=4mm, draw=none, fill=none},
	marked node/.style = {fill=black!40},
	picture label/.style = {font=\small\bfseries\sffamily, draw=none, fill=none}
]

\matrix[pic matrix] (pic) {
	\node[label=1] {27}
		child {node[label=2] {17}
			child {node[label=4] {16}
				child {node[label=8] {5}}
				child {node[label=9] {7}}
			}
			child {node[label=5] {13}
				child {node[label=10] {12}}
				child {node[label=11] (x) {4}}
			}
		}
		child {node[marked node, label=3, label=left:$i$] {3}
			child {node[label=6] {10}
				child {node[label=12] (y) {8}}
				child {node[label=13] {9}}
			}
			child {node[label=7] {1}
				child {node[label=14] {0}}
				child[missing]
			}
		};
	\node[picture label, below=5mm] at ($ (x)!0.5!(y) $) {(a)};
	&
	\node[label=1] {27}
		child {node[label=2] {17}
			child {node[label=4] {16}
				child {node[label=8] {5}}
				child {node[label=9] {7}}
			}
			child {node[label=5] {13}
				child {node[label=10] {12}}
				child {node[label=11] (x) {4}}
			}
		}
		child {node[label=3] {10}
			child {node[marked node, label=6, label=left:$i$] {3}
				child {node[label=12] (y) {8}}
				child {node[label=13] {9}}
			}
			child {node[label=7] {1}
				child {node[label=14] {0}}
				child[missing]
			}
		};
	\node[picture label, below=5mm] at ($ (x)!0.5!(y) $) {(b)};
	\\
	\node[label=1] {27}
		child {node[label=2] {17}
			child {node[label=4] {16}
				child {node[label=8] {5}}
				child {node[label=9] {7}}
			}
			child {node[label=5] {13}
				child {node[label=10] {12}}
				child {node[label=11] (x) {4}}
			}
		}
		child {node[label=3] {10}
			child {node[label=6] {9}
				child {node[label=12] (y) {8}}
				child {node[marked node, label=13, label=below left:{$i$}] {3}}
			}
			child {node[label=7] {1}
				child {node[label=14] {0}}
				child[missing]
			}
		};
	\node[picture label, below=5mm] at ($ (x)!0.5!(y) $) {(c)};
	& \\
};

\end{tikzpicture}

	\caption{Działanie procedury $\proc{Max-Heapify}(A,3)$ dla tablicy $A=\langle$27,\! 17,\! 3,\! 16,\! 13,\! 10,\! 1,\! 5,\! 7,\! 12,\! 4,\! 8,\! 9,\! 0$\rangle$.
{\sffamily\bfseries\doubledash{(a)}{(b)}} Drzewo binarne reprezentujące $A$, w~którym przywracana jest własność kopca typu max, odpowiednio, w~węzłach $i=3$ oraz $i=6$.
{\sffamily\bfseries(c)} Wynikowy kopiec z~przywróconą własnością kopca.} \label{fig:6.2-1}
\end{figure}

\exercise %6.2-2
Poniższy pseudokod prezentuje procedurę przywracania własności kopca typu min.
Ponieważ jedyną modyfikacją w~porównaniu z~procedurą \proc{Max-Heapify} jest zmiana znaków nierówności na przeciwne w~warunkach w~wierszach \ref{li:min-heapify-check1} i~\ref{li:min-heapify-check2}, to czas działania tej procedury jest identyczny z~czasem działania \proc{Max-Heapify}, czyli $\Theta(\lg n)$.
\begin{codebox}
\Procname{$\proc{Min-Heapify}(A,i)$}
\li	$l\gets\proc{Left}(i)$
\li	$r\gets\proc{Right}(i)$
\li	\If $l\le\attrib{A}{heap-size}$ i~$A[l]<A[i]$ \label{li:min-heapify-check1}
\li		\Then $\id{smallest}\gets l$
\li		\Else $\id{smallest}\gets i$
		\End
\li	\If $r\le\attrib{A}{heap-size}$ i~$A[r]<A[\id{smallest}]$ \label{li:min-heapify-check2}
\li		\Then $\id{smallest}\gets r$
		\End
\li	\If $\id{smallest}\ne i$
\li		\Then zamień $A[i]\leftrightarrow A[\id{smallest}]$
\li			$\proc{Min-Heapify}(A,\id{smallest})$
		\End
\end{codebox}

\exercise %6.2-3
Jeśli element $A[i]$ jest większy niż jego synowie, to \id{largest} jest ustawiane na $i$ i~warunek z~wiersza 8 nie jest spełniony.
Procedura zakończy więc działanie, nie dokonując żadnej zamiany elementów.

\exercise %6.2-4
Z~\refExercise{6.1-7} mamy, że element o~indeksie $i>\attrib{A}{heap-size}/2$ jest liściem kopca, czyli nie istnieją jego synowie.
W~dwóch pierwszych wierszach procedury \proc{Max-Heapify} obliczone zostaną wartości przekraczające \attrib{A}{heap-size}, więc zmienna \id{largest} przyjmie wartość $i$.
Warunek w~wierszu 8 będzie więc fałszywy i~natychmiast po jego sprawdzeniu procedura zakończy działanie.

\exercise %6.2-5
Iteracyjna wersja procedury \proc{Max-Heapify} została przedstawiona poniżej.
\begin{codebox}
\Procname{$\proc{Iterative-Max-Heapify}(A,i)$}
\li	\While \const{true}
\li		\Do $l\gets\proc{Left}(i)$ \label{li:iterative-max-heapify-begin}
\li			$r\gets\proc{Right}(i)$
\li			\If $l\le\attrib{A}{heap-size}$ i~$A[l]>A[i]$
\li				\Then $\id{largest}\gets l$
\li				\Else $\id{largest}\gets i$
				\End
\li			\If $r\le\attrib{A}{heap-size}$ i~$A[r]>A[\id{largest}]$
\li				\Then $\id{largest}\gets r$
				\End \label{li:iterative-max-heapify-end}
\li			\If $\id{largest}=i$ \label{li:iterative-max-heapify-cond}
\li				\Then \Return
				\End
\li			zamień $A[i]\leftrightarrow A[\id{largest}]$
\li			$i\gets\id{largest}$
		\End
\end{codebox}
Działania wykonywane w~wierszach \doubledash{\ref{li:iterative-max-heapify-begin}}{\ref{li:iterative-max-heapify-end}} są identyczne jak w~oryginalnej implementacji procedury.
W~zależności od wyniku testu z~wiersza \ref{li:iterative-max-heapify-cond} procedura kończy działanie albo zamienia elementy $A[i]$ i~$A[\id{largest}]$, po czym symuluje wywołanie rekurencyjne, aktualizując wartość zmiennej $i$ i~wykonując kolejną iterację pętli \kw{while}.

\exercise %6.2-6
Najgorszy przypadek dla procedury \proc{Max-Heapify} zachodzi wówczas, gdy zostanie ona wywołana dla korzenia kopca i~schodzi rekurencyjnie aż do jego ostatniego poziomu.
Najdłuższa ścieżka od korzenia do liścia składa się z~$h=\lfloor\lg n\rfloor$ krawędzi (z~\refExercise{6.1-2}) i~tyle będzie wywołań rekurencyjnych procedury w~najgorszym przypadku.
Koszt pracy wykonanej na każdym poziomie rekursji jest stały, a~więc procedura \proc{Max-Heapify} działa wtedy w~czasie $\Omega(\lg n)$.
Przykładowym drzewem, dla którego procedura wykona opisane operacje, jest takie, w~którym korzeń ma wartość 0, a~każdy inny węzeł ma wartość 1.
