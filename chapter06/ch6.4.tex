\subchapter{Algorytm sortowania przez kopcowanie (heapsort)}

\exercise %6.4-1
Na rys.\ \ref{fig:6.4-1} zilustrowano działanie sortowania przez kopcowanie dla tablicy $A$.
\begin{figure}[!ht]
	\centering \begin{tikzpicture}[
	marked node/.style = {tree node, fill=black!40},
	every label/.style = {index node, draw=none, fill=none},
	outer/.append style = {node distance=13mm and 8mm},
	outer scaled/.style = {outer, scale=0.7}
]

\node[outer scaled] (pic a) {
\begin{tikzpicture}[
	every node/.style = {tree node},
	anchor = center
]
	\node {25}
		child {node {13}
			child {node {8}
				child {node {5}}
				child {node {4}}
			}
			child {node {7}}
		}
		child {node {20}
			child {node {17}}
			child {node {2}
				child[missing]
				child {node[draw=none, fill=none] {} edge from parent[draw=none]}
			}
		};
\end{tikzpicture}
};

\node[outer scaled, right=of pic a] (pic b) {
\begin{tikzpicture}[
	every node/.style = {tree node},
	anchor = center
]
	\node {20}
		child {node {13}
			child {node {8}
				child {node {5}}
				child {node[marked node, label=right:{$i$}] {25} edge from parent[draw=none]}
			}
			child {node {7}}
		}
		child {node {17}
			child {node {4}}
			child {node {2}
				child[missing]
				child {node[draw=none, fill=none] {} edge from parent[draw=none]}
			}
		};
\end{tikzpicture}
};

\node[outer scaled, right=of pic b] (pic c) {
\begin{tikzpicture}[
	every node/.style = {tree node},
	anchor = center
]
	\node {17}
		child {node {13}
			child {node {8}
				child {node[marked node, label=left:{$i$}] {20} edge from parent[draw=none]}
				child {node[marked node] {25} edge from parent[draw=none]}
			}
			child {node {7}}
		}
		child {node {5}
			child {node {4}}
			child {node {2}
				child[missing]
				child {node[draw=none, fill=none, label=right:{}] {} edge from parent[draw=none]}
			}
		};
\end{tikzpicture}
};

\node[outer scaled, below=of pic a] (pic d) {
\begin{tikzpicture}[
	every node/.style = {tree node},
	anchor = center
]
	\node {13}
		child {node {8}
			child {node {2}
				child {node[marked node] {20} edge from parent[draw=none]}
				child {node[marked node] {25} edge from parent[draw=none]}
			}
			child {node {7}}
		}
		child {node {5}
			child {node {4}}
			child {node[marked node, label=left:{$i$}] {17} edge from parent[draw=none]
				child[missing]
				child {node[draw=none, fill=none] {} edge from parent[draw=none]}
			}
		};
\end{tikzpicture}
};

\node[outer scaled, below=of pic b] (pic e) {
\begin{tikzpicture}[
	every node/.style = {tree node},
	anchor = center
]
	\node {8}
		child {node {7}
			child {node {2}
				child {node[marked node] {20} edge from parent[draw=none]}
				child {node[marked node] {25} edge from parent[draw=none]}
			}
			child {node {4}}
		}
		child {node {5}
			child {node[marked node, label=left:{$i$}] {13} edge from parent[draw=none]}
			child {node[marked node] {17} edge from parent[draw=none]
				child[missing]
				child {node[draw=none, fill=none] {} edge from parent[draw=none]}
			}
		};
\end{tikzpicture}
};

\node[outer scaled, below=of pic c] (pic f) {
\begin{tikzpicture}[
	every node/.style = {tree node},
	anchor = center
]
	\node {7}
		child {node {4}
			child {node {2}
				child {node[marked node] {20} edge from parent[draw=none]}
				child {node[marked node] {25} edge from parent[draw=none]}
			}
			child {node[marked node, label=left:{$i$}] {8} edge from parent[draw=none]}
		}
		child {node {5}
			child {node[marked node] {13} edge from parent[draw=none]}
			child {node[marked node] {17} edge from parent[draw=none]
				child[missing]
				child {node[draw=none, fill=none] {} edge from parent[draw=none]}
			}
		};
\end{tikzpicture}
};

\node[outer scaled, below=of pic d] (pic g) {
\begin{tikzpicture}[
	every node/.style = {tree node},
	anchor = center
]
	\node {5}
		child {node {4}
			child {node[marked node, label=left:{$i$}] {7} edge from parent[draw=none]
				child {node[marked node] {20} edge from parent[draw=none]}
				child {node[marked node] {25} edge from parent[draw=none]}
			}
			child {node[marked node] {8} edge from parent[draw=none]}
		}
		child {node {2}
			child {node[marked node] {13} edge from parent[draw=none]}
			child {node[marked node] {17} edge from parent[draw=none]
				child[missing]
				child {node[draw=none, fill=none] {} edge from parent[draw=none]}
			}
		};
\end{tikzpicture}
};

\node[outer scaled, below=of pic e] (pic h) {
\begin{tikzpicture}[
	every node/.style = {tree node},
	anchor = center
]
	\node {4}
		child {node {2}
			child {node[marked node] {7} edge from parent[draw=none]
				child {node[marked node] {20} edge from parent[draw=none]}
				child {node[marked node] {25} edge from parent[draw=none]}
			}
			child {node[marked node] {8} edge from parent[draw=none]}
		}
		child {node[marked node, label=left:{$i$}] {5} edge from parent[draw=none]
			child {node[marked node] {13} edge from parent[draw=none]}
			child {node[marked node] {17} edge from parent[draw=none]
				child[missing]
				child {node[draw=none, fill=none] {} edge from parent[draw=none]}
			}
		};
\end{tikzpicture}
};

\node[outer scaled, below=of pic f] (pic i) {
\begin{tikzpicture}[
	every node/.style = {tree node},
	anchor = center
]
	\node {2}
		child {node[marked node, label=left:{$i$}] {4} edge from parent[draw=none]
			child {node[marked node] {7} edge from parent[draw=none]
				child {node[marked node] {20} edge from parent[draw=none]}
				child {node[marked node] {25} edge from parent[draw=none]}
			}
			child {node[marked node] {8} edge from parent[draw=none]}
		}
		child {node[marked node] {5} edge from parent[draw=none]
			child {node[marked node] {13} edge from parent[draw=none]}
			child {node[marked node] {17} edge from parent[draw=none]
				child[missing]
				child {node[draw=none, fill=none] {} edge from parent[draw=none]}
			}
		};
\end{tikzpicture}
};

\node[outer, below=of pic g] (pic j) {
\begin{tikzpicture}[
	row 1/.style = {nodes={fill=black!20}}
]
	\matrix[array] (arr) {2 & 4 & 5 & 7 & 8 & 13 & 17 & 20 & 25 \\};
	\node[left=3mm of arr-1-1] {$A$};
\end{tikzpicture}
};

\foreach \x in {a,b,c,d,e,f,g,h,i,j} {
	\node[subpicture label, below=2mm of pic \x] {(\x)};
}

\end{tikzpicture}

	\caption{Działanie procedury \proc{Heapsort} dla tablicy $A=\langle5,13,2,25,7,17,20,8,4\rangle$.
{\sffamily\bfseries(a)} Kopiec zaraz po jego zbudowaniu przez \proc{Build-Max-Heap}.
{\sffamily\bfseries\doubledash{(b)}{(i)}} Kopiec i~elementy z~niego usunięte po każdym wywołaniu \proc{Max-Heapify} w~wierszu 5.
{\sffamily\bfseries(j)} Wynikowa posortowana tablica.} \label{fig:6.4-1}
\end{figure}

\exercise %6.4-2
Pokażemy, że podany niezmiennik jest spełniony przed pierwszą iteracją pętli \kw{for}, że każda iteracja tej pętli zachowuje niezmiennik i~że po zakończeniu pętli z~niezmiennika można wywnioskować poprawność procedury.
\begin{description}
	\item[Inicjowanie:] Przed pierwszą iteracją pętli mamy $i=\attrib{A}{length}=n$.
Wówczas fragment $A[1\twodots i]$ jest całą tablicą $A$, która stanowi kopiec typu max, utworzony w~wyniku działania procedury \proc{Build-Max-Heap}, natomiast fragment $A[i+1\twodots n]$ jest pusty.
	\item[Utrzymanie:] Załóżmy, że niezmiennik jest prawdziwy przed wykonaniem kolejnej iteracji pętli.
Podtablica $A[1\twodots i]$ tworzy więc kopiec typu max, którego korzeniem jest $A[1]$, czyli największy element w~tej podtablicy.
Po wykonaniu wiersza 3 znajdzie się on na pozycji $i$.
Dekrementacja \attrib{A}{heap-size} powoduje, że element $A[i]$ nie wchodzi teraz w~skład kopca, ale podtablica $A[i\twodots n]$ zawiera teraz $n-i+1$ największych elementów z~$A[1\twodots n]$ posortowanych niemalejąco, ponieważ element $A[i]$ jest równy lub mniejszy od wcześniej umieszczonych tam elementów.
W~tym momencie korzeń może naruszać własność kopca typu max, dlatego w~wierszu 5 zostaje wywołana dla niego procedura \proc{Max-Heapify} przywracająca tę własność.
Uaktualnienie $i$ powoduje odtworzenie niezmiennika.
	\item[Zakończenie:] Po zakończeniu działania pętli jest $i=1$, zatem podtablica $A[1\twodots i]$ składa się z~jednego elementu, który jest najmniejszym elementem tablicy $A$.
Ponadto $n-1$ pozostałych elementów jest uszeregowanych w~kolejności niemalejącej w~podtablicy $A[2\twodots n]$.
Stąd mamy, że tablica $A$ jest posortowana.
\end{description}

\exercise %6.4-3
Na podstawie analizy zamieszczonej w~Podręczniku czasem działania algorytmu heapsort dla tablicy o~rozmiarze $n$ jest $O(n\lg n)$.
Z~\refExercise{6.4-5} mamy, że jest to w~rzeczywistości oszacowanie dokładne.
A~zatem w~szczególności dla tablicy posortowanej rosnąco i~tablicy posortowanej malejąco heapsort działa w~czasie $\Theta(n\lg n)$.

\exercise %6.4-4
Przypadek pesymistyczny algorytmu heapsort ma miejsce wtedy, gdy każde wywołanie \proc{Max-Heapify} z~wiersza 5 schodzi rekurencyjnie aż do ostatniego poziomu drzewa.
Na mocy wyniku z~\refExercise{6.2-6} oraz wzoru (3.18) wywołania te zabierają łącznie czas
\[
	\sum_{i=2}^n\Omega(\lg i) = \Omega(\lg(n!)) = \Omega(n\lg n)
\]
i~taki jest też czas działania algorytmu heapsort w~pesymistycznym przypadku.

\exercise %6.4-5
Dokonamy analizy liczby wykonywanych instrukcji z~linii 9 procedury \proc{Max-Heapify} podczas działania algorytmu sortowania przez kopcowanie w~przypadku optymistycznym.

Załóżmy, że algorytm heapsort działa na tablicy $A$ o~rozmiarze $n=2^{h+1}-1$, gdzie $h$ jest dodatnią liczbą całkowitą.
A~zatem kopiec zbudowany z~$A$ stanowi pełne drzewo binarne o~wysokości $h$.
Rozważanie tylko takich kopców nie powoduje zmniejszenia ogólności analizy.
Przez \singledash{$j$}{ty} etap działania algorytmu heapsort, gdzie $j=0$, 1, \dots, $h-1$, będziemy rozumieć działania wykonywane podczas iteracji pętli \kw{for} z~procedury \proc{Heapsort}, w~których $2^{h-j}\le i\le2^{h-j+1}-1$.
Inaczej mówiąc, \singledash{$j$}{ty} etap pozbawia kopiec \singledash{$(h-j)$}{tego} poziomu.

\medskip
\noindent\textsf{\textbf{Lemat.}} \textit{Podczas\/ \singledash{$j$}{tego} etapu działania algorytmu heapsort na kopcu\/ $A$ o~rozmiarze\/ $n=2^{h+1}-1$,\/ $h\ge5$, którego wszystkie elementy są różne, liczba wykonanych zamian elementów w~linii 9 procedury \proc{Max-Heapify} jest większa niż\/ $(h-j-5)2^{h-j-3}$.}
\begin{proof}
Oznaczmy przez $m_j$ badaną liczbę zamian elementów wykonywanych w~\singledash{$j$}{tym} etapie.
Niech $j=0$ i~bez utraty ogólności załóżmy, że $\langle A[1],A[2],\dots,A[n]\rangle$ jest permutacją $\langle1,2,\dots,n\rangle$.
Liczbę $k$ będziemy nazywać \textbf{dużą}, jeśli $k\ge(n+1)/2$.
Niech $S$ będzie zbiorem indeksów dużych elementów w~kopcu $A$, które nie są liśćmi, czyli
\[
    S = \biggl\{\,i\in\Bigl\{1,2,\dots,\frac{n-1}{2}\Bigr\}:A[i]\ge\frac{n+1}{2}\,\biggr\}.
\]
Zauważmy, że wszystkie elementy, których pozycjami w~$A$ są indeksy ze zbioru $S$, zostaną usunięte z~kopca w~etapie $j=0$.
A~zatem muszą wpierw znaleźć się w~korzeniu kopca za sprawą wykonania pewnej liczby zamian z~linii 9 procedury \proc{Max-Heapify}.
Stąd $m_0$ spełnia nierówność
\[
    m_0 \ge \sum_{i\in S}d_i,
\]
gdzie $d_i$ oznacza głębokość węzła o~początkowej pozycji $i$ w~kopcu $A$.

Węzły o~indeksach ze zbioru $S$ tworzą w~kopcu $A$ poddrzewo $T$ o~korzeniu w~$A[1]$.
Jest tak dlatego, że jeśli węzeł $A[i]$ jest duży, to $A[\proc{Parent}(i)]$ również jest duży, a~więc także wszystkie węzły na ścieżce od $A[i]$ do korzenia kopca, czyli $A[1]$.
Jeśli zastąpimy każde puste poddrzewo w~$T$ pojedynczym węzłem, to dostaniemy regularne drzewo binarne, którego długość ścieżki wewnętrznej (patrz \refExercise{B.5-5}) wynosi $m_0$.
W~zbiorze wszystkich drzew binarnych o~$|S|$ węzłach wewnętrznych najmniejsza możliwa długość ścieżki wewnętrznej jest osiągana dla pełnego drzewa binarnego (przy czym ostatni poziom tego drzewa może nie być wypełniony) i~wynosi $\sum_{k=1}^{|S|}\lfloor\lg k\rfloor$.
Korzystając ze wzoru (3.3) i~\refExercise{8.1-2}, mamy
\[
    m_0 \ge \sum_{k=1}^{|S|}\lfloor\lg k\rfloor > \sum_{k=1}^{|S|}(\lg k-1) = \sum_{k=1}^{|S|}\lg k-|S| \ge \frac{|S|}{2}\lg\frac{|S|}{2}-|S| = \frac{|S|}{2}\lg|S|-\frac{3}{2}|S|.
\]

Pokażemy teraz, że $|S|\ge2^{h-2}$.
Rozważmy w~tym celu permutację $\pi$ elementów kopca $A$ na początku zerowego etapu w~kolejności ich odwiedzania podczas przechodzenia kopca metodą inorder.
Jeśli $\pi(i)$, gdzie $i\ge2$, jest liściem kopca, to $\pi(i-1)$ nie może być liściem kopca.
Jeśli w~dodatku $\pi(i)$ jest dużym liściem, to $\pi(i-1)$ jest dużym węzłem wewnętrznym.
Stąd indeks elementu $\pi(i-1)$ należy do $S$.
Mamy więc, że $l$ -- liczba dużych liści -- nie przekracza $|S|+1$, nawet jeśli $\pi(1)$ jest dużym liściem.
Ponieważ liczba dużych elementów w~kopcu wynosi $(n+1)/2=2^h$, to otrzymujemy, że $|S|=2^h-l\ge2^h-(|S|+1)$, skąd $|S|\ge2^{h-1}-1/2\ge2^{h-2}$.

Powracając teraz do oszacowania na $m_0$, mamy
\[
    m_0 > \frac{|S|}{2}\lg|S|-\frac{3}{2}|S| = \frac{|S|}{2}(\lg|S|-3) \ge 2^{h-3}(h-5),
\]
czyli lemat jest prawdziwy, gdy $j=0$.

Na początku \singledash{$j$}{tego} etapu kopiec ma wysokość $h-j$, więc dowód lematu dla \singledash{$j$}{tego} etapu, gdzie $1\le j\le h-1$, sprowadza się do dowodu oszacowania $m_0$ dla kopca o~rozmiarze $n=2^{h-j+1}-1$.
\end{proof}

Załóżmy teraz, że $h\ge5$ i~wykorzystajmy oznaczenie $m_j$ z~powyższego dowodu.
Sumaryczną liczbę zamian elementów podczas sortowania $n$ liczb, na podstawie lematu, ograniczamy od dołu:
\[
    \sum_{j=0}^{h-1}m_j > \sum_{j=0}^{h-5}m_j > \sum_{j=0}^{h-5}(h-j-5)2^{h-j-3} = \sum_{j=0}^{h-5}j2^{j+2} = 4\sum_{j=0}^{h-5}j2^j.
\]
Ostatnią sumę obliczamy poprzez skorzystanie ze wzoru (A.5):
\[
    \sum_{j=0}^{h-5}jx^j = x\cdot\frac{d}{dx}\biggl(\sum_{j=0}^{h-5}x^j\biggr) = x\cdot\frac{d}{dx}\biggl(\frac{x^{h-4}-1}{x-1}\biggr) = x\,\frac{(h-4)x^{h-5}(x-1)-(x^{h-4}-1)}{(x-1)^2}.
\]
Przyjmując teraz $x=2$ i~korzystając z~nierówności $2^h>n/2$ i~$h>\lg n-1$, mamy ostatecznie
\[
    4\sum_{j=0}^{h-5}j2^j = (h-4)2^{h-2}-2^{h-1}+8 > (h-6)2^{h-2} > \frac{1}{8}n\lg n-\frac{7}{8}n = \Omega(n\lg n).
\]
Otrzymany wynik stanowi oszacowanie czasu działania algorytmu heapsort w~przypadku optymistycznym.
