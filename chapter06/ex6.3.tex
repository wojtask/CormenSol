\subchapter{Budowanie kopca}

\exercise %6.3-1
Ilustracja działania procedury \proc{Build-Max-Heap} dla tablicy $A$ znajduje się na rys.\ \ref{fig:6.3-1}.
\begin{figure}[!ht]
	\centering \begin{tikzpicture}[
	marked node/.style = {tree node, med gray},
	every label/.style = {index node, draw=none, fill=none},
	outer/.append style = {node distance=9mm and 15mm},
	inner/.style = {draw=none, fill=none, node distance=3mm},
]

\node[outer] (pic a) {
\begin{tikzpicture}
	\node[inner] (pic a1) {
	\begin{tikzpicture}[
		row 1/.style = {nodes={light gray}}
	]
		\matrix[array] (arr) {5 & 3 & 17 & 10 & 84 & 19 & 6 & 22 & 9 \\};
		\node[left=of arr-1-1.west] {$A$};
	\end{tikzpicture}
	};
	\node[inner, below=of pic a1] {
	\begin{tikzpicture}[
		every node/.style = {tree node},
		anchor = center
	]
		\node[label=1] {5}
			child {node[label=2] {3}
				child {node[marked node, label=4, label=left:{$i$}] {10}
					child {node[label=8] {22}}
					child {node[label=9] {9}}
				}
				child {node[label=5] {84}}
			}
			child {node[label=3] {17}
				child {node[label=6] {19}}
				child {node[label=7] {6}
					child[missing]
					child {node[draw=none, fill=none] {} edge from parent[draw=none]}
				}
			};
	\end{tikzpicture}
	};
\end{tikzpicture}
};

\node[outer, right=of pic a.south east, anchor=south west] (pic b) {
\begin{tikzpicture}[
	every node/.style = {tree node},
	anchor = center
]
	\node[label=1] {5}
		child {node[label=2] {3}
			child {node[label=4] {22}
				child {node[label=8] {10}}
				child {node[label=9] {9}}
			}
			child {node[label=5] {84}}
		}
		child {node[marked node, label=3, label=left:{$i$}] {17}
			child {node[label=6] {19}}
			child {node[label=7] {6}
				child[missing]
				child {node[draw=none, fill=none] {} edge from parent[draw=none]}
			}
		};
\end{tikzpicture}
};

\node[outer, below=of pic a] (pic c) {
\begin{tikzpicture}[
	every node/.style = {tree node},
	anchor = center
]
	\node[label=1] {5}
		child {node[marked node, label=2, label=left:{$i$}] {3}
			child {node[label=4] {22}
				child {node[label=8] {10}}
				child {node[label=9] {9}}
			}
			child {node[label=5] {84}}
		}
		child {node[label=3] {19}
			child {node[label=6] {17}}
			child {node[label=7] {6}
				child[missing]
				child {node[draw=none, fill=none] {} edge from parent[draw=none]}
			}
		};
\end{tikzpicture}
};

\node[outer, right=of pic c] (pic d) {
\begin{tikzpicture}[
	every node/.style = {tree node},
	anchor = center
]
	\node[marked node, label=1, label=left:{$i$}] {5}
		child {node[label=2] {84}
			child {node[label=4] {22}
				child {node[label=8] {10}}
				child {node[label=9] {9}}
			}
			child {node[label=5] {3}}
		}
		child {node[label=3] {19}
			child {node[label=6] {17}}
			child {node[label=7] {6}
				child[missing]
				child {node[draw=none, fill=none] {} edge from parent[draw=none]}
			}
		};
\end{tikzpicture}
};

\node[outer, below=of pic c] (pic e) {
\begin{tikzpicture}[
	every node/.style = {tree node},
	anchor = center
]
	\node[label=1] {84}
		child {node[label=2] {22}
			child {node[label=4] {10}
				child {node[label=8] {5}}
				child {node[label=9] {9}}
			}
			child {node[label=5] {3}}
		}
		child {node[label=3] {19}
			child {node[label=6] {17}}
			child {node[label=7] {6}
				child[missing]
				child {node[draw=none, fill=none] {} edge from parent[draw=none]}
			}
		};
\end{tikzpicture}
};

\foreach \x in {a,b,c,d,e} {
	\node[subpicture label, below=-3mm of pic \x] {(\x)};
}

\end{tikzpicture}

	\caption{Działanie procedury \proc{Build-Max-Heap} dla tablicy $A=\langle5,3,17,10,84,19,6,22,9\rangle$.
{\sffamily\bfseries(a)} Tablica $A$ i~reprezentowane przez nią drzewo binarne przed pierwszym wywołaniem \proc{Max-Heapify} z~wiersza~3.
{\sffamily\bfseries(b)\nbendash(d)} Kopiec przed każdym kolejnym wywołaniem \proc{Max-Heapify}.
{\sffamily\bfseries(e)} Wynikowy kopiec typu max.} \label{fig:6.3-1}
\end{figure}

\exercise %6.3-2
Gdy wywołujemy $\proc{Max-Heapify}(A,i)$, to zakładamy, że poddrzewa o~korzeniach $\proc{Left}(i)$ i~$\proc{Right}(i)$ (o~ile istnieją) są kopcami typu max.
Jeżeli podczas budowy kopca procedura \proc{Max-Heapify} byłaby wywoływana dla węzłów o~rosnących indeksach, to nie moglibyśmy zagwarantować spełnienia tego warunku.

\exercise %6.3-3
Oznaczmy kopiec przez $T$, a~przez $n_h$ -- ilość węzłów kopca $T$ znajdujących się na wysokości $h$.
Udowodnimy fakt przez indukcję względem $h$.

W~pierwszym kroku indukcji musimy pokazać, że $n_0\le\lceil n/2\rceil$.
W~rzeczywistości udowodnimy, że $n_0=\lceil n/2\rceil$.
Korzystając z~\refExercise{6.1-7}, mamy, że węzły znajdujące się w~$T$ na wysokości 0, czyli jego liście, zajmują pozycje $\lfloor n/2\rfloor+1\twodots n$.
Jest ich zatem
\[
    n_0 = n-(\lfloor n/2\rfloor+1)+1 = n-\lfloor n/2\rfloor = \lceil n/2\rceil.
\]
A~więc przypadek bazowy indukcji jest spełniony.

Załóżmy teraz, że $h>0$ i~że twierdzenie jest spełnione dla węzłów na wysokości $h-1$.
Ponadto niech $T'$ będzie kopcem powstałym z~$T$ po usunięciu z~niego wszystkich jego liści.
Nowy kopiec ma zatem $n'=n-n_0$ węzłów.
Ponieważ w~kroku bazowym pokazaliśmy, że $n_0=\lceil n/2\rceil$, to stąd $n'=n-\lceil n/2\rceil=\lfloor n/2\rfloor$.
Węzły, które w~kopcu $T$ znajdują się na wysokości $h$, w~$T'$ zajmują wysokość $h-1$, więc jeśli oznaczymy przez $n_{h-1}'$ liczbę węzłów na wysokości $h-1$ w~kopcu $T'$, to wówczas będzie $n_h=n_{h-1}'$.
Wykorzystując założenie indukcyjne, dostajemy
\[
    n_h = n_{h-1}' \le \lceil n'\!/2^h\rceil = \lceil\lfloor n/2\rfloor/2^h\rceil \le \lceil(n/2)/2^h\rceil = \lceil n/2^{h+1}\rceil,
\]
co kończy dowód.
