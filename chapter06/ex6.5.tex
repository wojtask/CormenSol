\subchapter{Kolejki priorytetowe}

\exercise %6.5-1
Na rys.\ \ref{fig:6.5-1} został przedstawiony kopiec wejściowy $A$ i~wynikowy kopiec otrzymany w~wyniku działania procedury \proc{Heap-Extract-Max}.
W~wierszu 3 zmiennej \id{max} przypisywana jest maksymalna wartość kopca, czyli 15.
Następnie korzeń otrzymuje wartość 1 i~rozmiar kopca jest pomniejszany o~1.
Po przywróceniu własności kopca w~linii 6 procedura zwraca wartość \id{max}.
\begin{figure}[!ht]
	\centering \begin{tikzpicture}

\node[outer] (pic a) {
\begin{tikzpicture}[
	every node/.style = {tree node},
	anchor = center
]
	\node {15}
		child {node {13}
			child {node {5}
				child {node {4}}
				child {node {0}}
			}
			child {node {12}
				child {node {6}}
				child {node {2}}
			}
		}
		child {node {9}
			child {node {8}
				child {node {1}}
				child[missing]
			}
			child {node {7}
				child[missing]
				child {node[draw=none, fill=none] {} edge from parent[draw=none]}
			}
		};
\end{tikzpicture}
};

\node[outer, right=15mm of pic a] (pic b) {
\begin{tikzpicture}[
	every node/.style = {tree node},
	anchor = center
]
	\node {13}
		child {node {12}
			child {node {5}
				child {node {4}}
				child {node {0}}
			}
			child {node {6}
				child {node {1}}
				child {node {2}}
			}
		}
		child {node {9}
			child {node {8}}
			child {node {7}
				child[missing]
				child {node[draw=none, fill=none] {} edge from parent[draw=none]}
			}
		};
\end{tikzpicture}
};

\node[subpicture label, below=2mm of pic a] {(a)};
\node[subpicture label, below=2mm of pic b] {(b)};

\end{tikzpicture}

	\caption{Działanie procedury \proc{Heap-Extract-Max} dla kopca $A=\langle15,13,9,5,12,8,7,4,0,6,2,1\rangle$.
{\sffamily\bfseries(a)} Kopiec wejściowy $A$.
{\sffamily\bfseries(b)} Kopiec $A$ po usunięciu maksymalnej wartości i~przywróceniu własności kopca naruszonej przez korzeń, któremu wcześniej przypisano wartość 1.} \label{fig:6.5-1}
\end{figure}

\exercise %6.5-2
Procedura \proc{Max-Heap-Insert} rozpoczyna działanie od dodania do kopca nowego elementu o~wartości $-\infty$.
Wartość ta jest następnie odpowiednio modyfikowana i~element jest umieszczany na właściwej pozycji w~kopcu dzięki wywołaniu \proc{Heap-Increase-Key}.
Działanie procedury $\proc{Max-Heap-Insert}(A,10)$ zostało przedstawione na rys.\ \ref{fig:6.5-2}.
\begin{figure}[!ht]
	\centering \begin{tikzpicture}[
	marked node/.style = {tree node, fill=black!40},
	every label/.style = {index node, draw=none, fill=none}
]

\node[outer] (pic a) {
\begin{tikzpicture}[
	every node/.style = {tree node},
	anchor = center
]
	\node {15}
		child {node {13}
			child {node {5}
				child {node {4}}
				child {node {0}}
			}
			child {node {12}
				child {node {6}}
				child {node {2}}
			}
		}
		child {node {9}
			child {node {8}
				child {node {1}}
				child {node[marked node, font=\footnotesize] {$-\infty$}}
			}
			child {node {7}
				child[missing]
				child {node[draw=none, fill=none] {} edge from parent[draw=none]}
			}
		};
\end{tikzpicture}
};

\node[outer, right=of pic a] (pic b) {
\begin{tikzpicture}[
	every node/.style = {tree node},
	anchor = center
]
	\node {15}
		child {node {13}
			child {node {5}
				child {node {4}}
				child {node {0}}
			}
			child {node {12}
				child {node {6}}
				child {node {2}}
			}
		}
		child {node {9}
			child {node {8}
				child {node {1}}
				child {node[marked node, font=\footnotesize, label=right:{$i$}] {10}}
			}
			child {node {7}
				child[missing]
				child {node[draw=none, fill=none] {} edge from parent[draw=none]}
			}
		};
\end{tikzpicture}
};

\node[outer, below=15mm of pic a] (pic c) {
\begin{tikzpicture}[
	every node/.style = {tree node},
	anchor = center
]
	\node {15}
		child {node {13}
			child {node {5}
				child {node {4}}
				child {node {0}}
			}
			child {node {12}
				child {node {6}}
				child {node {2}}
			}
		}
		child {node {9}
			child {node[marked node, font=\footnotesize, label=right:{$i$}] {10}
				child {node {1}}
				child {node {8}}
			}
			child {node {7}
				child[missing]
				child {node[draw=none, fill=none] {} edge from parent[draw=none]}
			}
		};
\end{tikzpicture}
};

\node[outer, right=of pic c] (pic d) {
\begin{tikzpicture}[
	every node/.style = {tree node},
	anchor = center
]
	\node {15}
		child {node {13}
			child {node {5}
				child {node {4}}
				child {node {0}}
			}
			child {node {12}
				child {node {6}}
				child {node {2}}
			}
		}
		child {node[marked node, font=\footnotesize, label=right:{$i$}] {10}
			child {node {9}
				child {node {1}}
				child {node {8}}
			}
			child {node {7}
				child[missing]
				child {node[draw=none, fill=none] {} edge from parent[draw=none]}
			}
		};
\end{tikzpicture}
};

\node[subpicture label, below=2mm of pic a] {(a)};
\node[subpicture label, below=2mm of pic b] {(b)};
\node[subpicture label, below=2mm of pic c] {(c)};
\node[subpicture label, below=2mm of pic d] {(d)};

\end{tikzpicture}

	\caption{Działanie procedury $\proc{Max-Heap-Insert}(A,10)$ dla kopca $A=\langle$15,\! 13,\! 9,\! 5,\! 12,\! 8,\! 7,\! 4,\! 0,\! 6,\! 2,\! 1$\rangle$.
{\sffamily\bfseries(a)} Kopiec po dodaniu nowego elementu o~wartości początkowej $-\infty$.
{\sffamily\bfseries(b)} W~trakcie działania procedury \proc{Heap-Increase-Key}.
Wartość nowego elementu została zwiększona i~wynosi teraz 10.
{\sffamily\bfseries(c)} Po wykonaniu pierwszej iteracji pętli \kw{while} procedury \proc{Heap-Increase-Key} nowy element został zamieniony ze swoim ojcem.
{\sffamily\bfseries(d)} Po drugiej iteracji pętli nastąpiła jeszcze jedna zamiana nowego elementu i~jego aktualnego ojca, dzięki czemu $A$ spełnia już własność kopca i~procedura kończy działanie.} \label{fig:6.5-2}
\end{figure}

\exercise %6.5-3
Zakładamy, że tablica $A$ stanowi kopiec typu min.
Poniższe procedury stanowią implementację kolejki priorytetowej typu min i~działają analogicznie do odpowiadających im procedur dla kolejki priorytetowej typu max.
\begin{codebox}
\Procname{$\proc{Heap-Minimum}(A)$}
\li	\Return $A[1]$
\end{codebox}
\begin{codebox}
\Procname{$\proc{Heap-Extract-Min}(A)$}
\li	\If $\attrib{A}{heap-size}<1$
\li		\Then \Error ,,kopiec pusty''
		\End
\li	$\id{min}\gets A[1]$
\li	$A[1]\gets A[\attrib{A}{heap-size}]$
\li	$\attrib{A}{heap-size}\gets\attrib{A}{heap-size}-1$
\li	$\proc{Min-Heapify}(A,1)$
\li	\Return \id{min}
\end{codebox}
\begin{codebox}
\Procname{$\proc{Heap-Decrease-Key}(A,i,\id{key})$}
\li	\If $\id{key}>A[i]$
\li		\Then \Error ,,nowy klucz jest większy niż klucz aktualny''
		\End
\li	$A[i]\gets\id{key}$
\li	\While $i>1$ i~$A[\proc{Parent}(i)]>A[i]$
\li		\Do zamień $A[i]\leftrightarrow A[\proc{Parent}(i)]$
\li			$i\gets\proc{Parent}(i)$
		\End
\end{codebox}
\begin{codebox}
\Procname{$\proc{Min-Heap-Insert}(A,\id{key})$}
\li	$\attrib{A}{heap-size}\gets\attrib{A}{heap-size}+1$
\li	$A[\attrib{A}{heap-size}]\gets\infty$
\li	$\proc{Heap-Decrease-Key}(A,\attrib{A}{heap-size},\id{key})$
\end{codebox}

\exercise %6.5-4
Po wykonaniu wiersza 1 procedury \proc{Max-Heap-Insert} element $A[\attrib{A}{heap-size}]$ mógłby być większy od \id{key} i~wywołanie \proc{Heap-Increase-Key} zakończyłoby się błędem.
Radzimy sobie z~tą sytuacją poprzez nadanie elementowi wartości $-\infty$.

\exercise %6.5-5
Dowodzimy, że niezmiennik spełniony jest przed każdą iteracją pętli \kw{while} oraz że po zakończeniu tej pętli z~niezmiennika wynika, że procedura jest poprawna.
\begin{description}
	\item[Inicjowanie:] Przed wykonaniem wiersza 3 tablica $A[1\twodots\attrib{A}{heap-size}]$ jest kopcem typu max.
Zwiększenie wartości $A[i]$ może naruszyć własność kopca tylko dla elementów $A[i]$ oraz $A[\proc{Parent}(i)]$.
	\item[Utrzymanie:] Dokonując zamiany elementów w~wierszu 5 w~bieżącej iteracji pętli, przywracamy własność kopca dla elementów $A[i]$ oraz $A[\proc{Parent}(i)]$.
Jednak operacja ta może wygenerować nową parę elementów niespełniających własności kopca: $A[\proc{Parent}(i)]$ oraz $A[\proc{Parent}(\proc{Parent}(i))]$.
Aktualizacja wartości $i$ powoduje zachowanie niezmiennika, albowiem nowa para elementów jest jedyną, która może naruszać własność kopca.
	\item[Zakończenie:] Pętla kończy działanie, gdy $i\le1$ lub $A[\proc{Parent}(i)]\ge A[i]$.
W~pierwszym przypadku $\proc{Parent}(i)\le0$, co jest niepoprawną wartością dla indeksów tablicy $A$.
W~drugim natomiast jedyna para, która mogłaby naruszać własność kopca, w~rzeczywistości ją spełnia.
A~zatem po zakończeniu wykonywania pętli tablica $A[1\twodots\attrib{A}{heap-size}]$ stanowi kopiec typu max.
\end{description}
Z~prawdziwości niezmiennika pętli wynika, że procedura \proc{Heap-Increase-Key} poprawnie zwiększa wartość węzła $i$, pozostawiając kopiec typu max.

\exercise %6.5-6
Kolejkę FIFO implementujemy, wykorzystując do tego celu kolejkę priorytetową typu min.
Przyjmiemy, że elementy kolejki oprócz kluczy zawierają też dodatkowe dane, natomiast klucze posłużą jedynie do symulowania działania kolejki FIFO.
Podczas inicjalizacji kolejki ustawimy dodatkowy jej atrybut \id{rank} na 1.
Przed dodaniem nowego elementu do kolejki FIFO ustawimy jego klucz na aktualną wartość pola \id{rank}, następnie wstawimy element do kolejki priorytetowej procedurą \proc{Min-Heap-Insert} i~wreszcie zinkrementujemy wartość pola \id{rank}.
Z~kolei usuwanie elementów sprowadza się do zwykłego wywołania procedury \proc{Heap-Extract-Min}.
Taka implementacja operacji kolejki FIFO zapewnia, że w~danym momencie w~strukturze danych nie ma dwóch różnych elementów z~tym samym kluczem i~elementy pobierane są w~odpowiedniej kolejności.

Realizacja stosu jest podobna, ale używamy do tego celu kolejki priorytetowej typu max.
Podczas wstawiania elementów na stos korzystamy z~procedury \proc{Max-Heap-Insert} i~inkrementujemy jego pole \id{rank} (zainicjalizowane na 1 w~momencie utworzenia stosu).
Usuwanie polega na odnalezieniu i~pobraniu elementu z~największym kluczem przy pomocy \proc{Heap-Extract-Max}.
Dzięki temu elementy pobierane są w~kolejności odwrotnej do tej, w~której były wstawiane.

\exercise %6.5-7
\note{Zmienimy nazwę operacji z~sugerowanej w~Podręczniku \proc{Heap-Delete} na \proc{Max-Heap-Delete}, aby odróżnić ją od analogicznej procedury dla kopca typu min.}

\noindent Przedstawiona poniżej procedura \proc{Max-Heap-Delete} zamienia element $A[i]$ w~$n$\nbhyphen elementowym kopcu $A$ typu max z~jego liściem $A[\attrib{A}{heap-size}]$, po czym dekrementuje \attrib{A}{heap-size}.
Po tym kroku własność kopca może być naruszona przez węzeł $A[i]$ na dwa sposoby.
W~pierwszym przypadku może zachodzić $A[i]<A[\proc{Left}(i)]$ lub $A[i]<A[\proc{Right}(i)]$ -- przywracamy więc własność kopca za pomocą wywołania \proc{Max-Heapify}.
W~drugim przypadku może być $A[i]>A[\proc{Parent}(i)]$, więc w~celu odbudowy struktury kopca wykonujemy podobne operacje, jak w~procedurze \proc{Heap-Increase-Key}.
Ostatni krok procedury to zwrócenie elementu, który początkowo zajmował w~kopcu pozycję $i$.
\begin{codebox}
\Procname{$\proc{Max-Heap-Delete}(A,i)$}
\li	zamień $A[i]\leftrightarrow A[\attrib{A}{heap-size}]$
\li	$\attrib{A}{heap-size}\gets\attrib{A}{heap-size}-1$
\li	$\proc{Max-Heapify}(A,i)$ \label{li:max-heap-delete-heapify}
\li	\While $i>1$ i~$A[\proc{Parent}(i)]<A[i]$ \label{li:max-heap-delete-while-begin}
\li		\Do zamień $A[i]\leftrightarrow A[\proc{Parent}(i)]$
\li			$i\gets\proc{Parent}(i)$
		\End \label{li:max-heap-delete-while-end}
\li	\Return $A[\attrib{A}{heap-size}+1]$
\end{codebox}

Zarówno wywołanie z~wiersza~\ref{li:max-heap-delete-heapify}, jak i~pętla \kw{while} w~wierszach \ref{li:max-heap-delete-while-begin}\nbendash\ref{li:max-heap-delete-while-end} zajmuje czas $O(\lg n)$, a~więc czasem działania procedury \proc{Max-Heap-Delete} jest również $O(\lg n)$.

\exercise %6.5-8
W~algorytmie wykorzystamy kolejkę priorytetową typu min jako strukturę pomocniczą.
Na początku do kolejki zostaną przeniesione elementy z~głowy każdej listy.
Jest oczywiste, że wśród tych elementów znajduje się najmniejszy element listy wynikowej.
Aby go uzyskać, wystarczy wywołać na kolejce operację \proc{Extract-Min}.
Kolejnego elementu należy szukać wśród aktualnych węzłów kolejki lub w~głowie listy, do której początkowo należało usunięte przed chwilą minimum.
Głowę tej listy, o~ile istnieje, przenosimy do kolejki.
W~kolejnych iteracjach powtarzamy te operacje aż do opróżnienia kolejki -- pobieramy najmniejszy element kolejki i~wstawiamy na listę wynikową, po czym uzupełniamy kolejkę głową listy, do której należał pobrany element, o~ile lista ta nie jest jeszcze pusta.
Po zakończeniu wykonywania tych działań wszystkie listy zostają przetworzone.
Aby zachować kolejność niemalejącą na liście wynikowej, musimy wstawiać elementy na jej koniec, pamiętając wskaźnik do ogona tej listy.

Podczas działania algorytmu wykonamy $n$ razy operację wstawienia węzła do kolejki zawierającej co najwyżej $k$ elementów i~tyleż samo operacji \proc{Extract-Min}.
Otrzymujemy zatem górne oszacowanie $O(n\lg k)$ na czas działania algorytmu, przy założeniu, że kolejka priorytetowa została zaimplementowana w~oparciu o~kopiec typu min.
