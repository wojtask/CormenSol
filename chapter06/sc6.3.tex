\subchapter{Budowanie kopca}

\exercise %6.3-1
Ilustracja działania procedury \proc{Build-Max-Heap} dla tablicy $A$ znajduje się na rys.\ \ref{fig:6.3-1}.
\begin{figure}[ht!]
	\begin{center}
		\includegraphics{fig_6.3-1}
	\end{center}
	\caption{Działanie procedury \proc{Build-Max-Heap} dla tablicy $A=\langle5,3,17,10,84,19,6,22,9\rangle$.
{\sffamily\bfseries(a)} Tablica $A$ i~reprezentowane przez nią drzewo binarne przed pierwszym wywołaniem \proc{Max-Heapify} z~wiersza 3.
{\sffamily\bfseries\doubledash{(b)}{(d)}} Drzewo przed każdym kolejnym wywołaniem \proc{Max-Heapify}.
{\sffamily\bfseries(e)} Wynikowy kopiec typu max} \label{fig:6.3-1}
\end{figure}

\exercise %6.3-2
Wywołując $\proc{Max-Heapify}(A,i)$, zakładamy, że drzewa o~korzeniach w~$\proc{Left}(i)$ i~$\proc{Right}(i)$ (o~ile istnieją) są kopcami typu max.
Jeżeli podczas budowy kopca procedura \proc{Max-Heapify} byłaby wywoływana dla węzłów o~rosnących indeksach, to nie moglibyśmy zagwarantować, że założenie to jest spełnione, dlatego przetwarzanie odbywa się w~kolejności malejących indeksów.

\exercise %6.3-3
Oznaczmy kopiec przez $T$, a~przez $n_h$ -- ilość węzłów kopca $T$ znajdujących się na wysokości $h$.
Udowodnimy fakt przez indukcję względem $h$.

W~pierwszym kroku indukcji musimy pokazać, że $n_0\le\lceil n/2\rceil$.
W~rzeczywistości udowodnimy, że $n_0=\lceil n/2\rceil$.
Korzystając z~\refExercise{6.1-7}, mamy, że węzły znajdujące się w~$T$ na wysokości 0, czyli jego liście, zajmują pozycje $\lfloor n/2\rfloor+1\twodots n$.
Jest ich zatem
\[
    n_0 = n-(\lfloor n/2\rfloor+1)+1 = n-\lfloor n/2\rfloor = \lceil n/2\rceil.
\]
A~więc przypadek bazowy indukcji jest spełniony.

Załóżmy teraz, że $h>0$ i~że twierdzenie jest spełnione dla węzłów na wysokości $h-1$.
Ponadto niech $T'$ będzie kopcem powstałym z~$T$ po usunięciu z~niego wszystkich jego liści.
Nowy kopiec ma zatem $n'=n-n_0$ węzłów.
Ponieważ w~kroku bazowym pokazaliśmy, że $n_0=\lceil n/2\rceil$, to stąd $n'=n-\lceil n/2\rceil=\lfloor n/2\rfloor$.
Węzły, które w~kopcu $T$ znajdują się na wysokości $h$, w~$T'$ zajmują wysokość $h-1$, więc jeśli oznaczymy przez $n_{h-1}'$ liczbę węzłów na wysokości $h-1$ w~kopcu $T'$, to wówczas będzie $n_h=n_{h-1}'$.
Wykorzystując założenie indukcyjne, dostajemy
\[
    n_h = n_{h-1}' \le \lceil n'\!/2^h\rceil = \lceil\lfloor n/2\rfloor/2^h\rceil \le \lceil(n/2)/2^h\rceil = \lceil n/2^{h+1}\rceil,
\]
co kończy dowód.
