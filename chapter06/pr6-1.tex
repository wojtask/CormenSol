\problem{Budowa kopca przez wstawianie} %6-1

\subproblem %6-1(a)
Procedury te nie zawsze generują identyczne kopce dla tej samej tablicy wejściowej.
Jeśli na przykład rozważymy tablicę $A=\langle1,2,3\rangle$, to kopce budowane przez obie procedury różnią się, jak to widać na rys.\ \ref{fig:6-1(a)}.
\begin{figure}[ht]
	\centering \begin{tikzpicture}[
	level/.append style = {sibling distance=30mm/2^#1},
	outer/.append style = {node distance=30mm}
]

\node[outer] (pic a) {
\begin{tikzpicture}[
	every node/.style = tree node,
	anchor = center
]
	\node {3}
		child {node {2}}
		child {node {1}};
\end{tikzpicture}
};

\node[outer, right=of pic a] (pic b) {
\begin{tikzpicture}[
	every node/.style = tree node,
	anchor = center
]
	\node {3}
		child {node {1}}
		child {node {2}};
\end{tikzpicture}
};

\node[subpicture label, below=2mm of pic a] {(a)};
\node[subpicture label, below=2mm of pic b] {(b)};

\end{tikzpicture}

	\caption{Porównanie kopców budowanych przez obie procedury dla tablicy $A=\langle1,2,3\rangle$.
{\sffamily\bfseries(a)} Wynik działania \proc{Build-Max-Heap}.
{\sffamily\bfseries(b)} Wynik działania \proc{Build-Max-Heap}$'$.} \label{fig:6-1(a)}
\end{figure}

\subproblem %6-1(b)
Najgorszym przypadkiem dla procedury \proc{Build-Max-Heap}$'$ jest tablica uporządkowana rosnąco.
W~każdym z~$n-1$ wywołań \proc{Max-Heap-Insert} z~wiersza 3 nowy węzeł transportowany jest wówczas aż do korzenia kopca, co wymaga $\Theta(\lg i)$ operacji przy \singledash{$i$}{elementowym} kopcu.
Stąd czas działania \proc{Build-Max-Heap}$'$ w~przypadku pesymistycznym wynosi
\[
	\sum_{i=1}^{n-1}\Theta(\lg i) = \Theta(\lg(n!)) = \Theta(n\lg n).
\]
