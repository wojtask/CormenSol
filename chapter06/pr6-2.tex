\problem{Analiza kopców rzędu $d$} %6-2

\subproblem %6-2(a)
\singledash{$d$}{kopiec} będziemy reprezentować w~tablicy w~następujący sposób.
Podobnie jak w~reprezentacji tablicowej kopców binarnych tablica $A$ reprezentująca \singledash{$d$}{kopiec} będzie mieć atrybuty \attrib{A}{length} oraz \attrib{A}{heap-size}.
Na pierwszej pozycji tablicy znajdzie się korzeń kopca, a~pozycje $2\twodots d+1$ będą zajmowane przez $d$ synów korzenia.
Synowie pierwszego z~lewej syna korzenia zajmą pozycje $d+2\twodots2d+1$, synowie drugiego od lewej syna korzenia -- pozycje $2d+2\twodots3d+1$ itd.
Ogólnie, mając dany indeks węzła $i$, można wyznaczyć indeks jego ojca, korzystając ze wzoru $\lceil(i-1)/d\rceil$.
Uogólnienie procedury \proc{Parent} dla kopca rzędu $d$ wygląda zatem następująco:
\begin{codebox}
\Procname{$\proc{Multiary-Parent}(d,i)$}
\zi	\Return $\lceil(i-1)/d\rceil$
\end{codebox}

Łatwo pokazać, że indeks \singledash{$k$}{tego} od lewej syna węzła o~indeksie $i$, gdzie $k=1$, 2, \dots, $d$, jest opisany wzorem $d(i-1)+k+1$.
Poniższa procedura stanowi uogólnienie procedur \proc{Left} i~\proc{Right} dla kopca rzędu $d$ -- w~porównaniu do nich przyjmuje dodatkowy parametr $k$ oznaczający numer szukanego syna węzła $i$.
\begin{codebox}
\Procname{$\proc{Multiary-Child}(d,k,i)$}
\zi	\Return $d(i-1)+k+1$
\end{codebox}

Można sprawdzić, że zachodzi $\proc{Multiary-Parent}(d,\proc{Multiary-Child}(d,k,i))=i$ dla każdego $k=1$, 2, \dots, $d$.

\subproblem %6-2(b)
Uogólnimy rozumowanie z~\refExercise{6.1-1} na kopce rzędu $d$.
Potraktujmy taki kopiec jak drzewo \singledash{$d$}{arne} o~wysokości $h$ i~$n$ węzłach.
Na \singledash{$i$}{tym} poziomie tego drzewa, gdzie $i=0$, 1, \dots, $h-1$, znajduje się $d^i$ węzłów.
Najniższy, \singledash{$h$}{ty} poziom, może zawierać od 1 do $d^h$ węzłów.
Mamy zatem
\[
    \sum_{i=0}^{h-1}d^i+1 \le n \le \sum_{i=0}^hd^i.
\]
Na podstawie punktu (c) problemu \refProblem{3-1} sumę po lewej stronie można oszacować przez $\Theta(d^{h-1})$, a~sumę po prawej -- przez $\Theta(d^h)$.
Oba te oszacowania dają w~wyniku $h=\Theta(\log_dn)$.

\subproblem %6-2(c)
Przedstawimy najpierw implementację procedury \proc{Max-Heapify} dla kopców rzędu $d$.
Ogólny zarys jej działania pozostaje niezmieniony w~porównaniu z~oryginalną procedurą \proc{Max-Heapify}.
Na każdym poziomie rekursji musimy wyznaczyć maksimum z~$d+1$ wartości -- bieżącego węzła i~jego $d$ synów.
W~tym celu stosujemy pętlę przeglądającą wszystkich synów bieżącego węzła.
\begin{codebox}
\Procname{$\proc{Multiary-Max-Heapify}(A,d,i)$}
\li	$\id{largest}\gets i$
\li	$k\gets1$
\li	$\id{child}\gets\proc{Multiary-Child}(d,1,i)$
\li	\While $k\le d$ i~$\id{child}\le\attrib{A}{heap-size}$ \label{li:multiary-max-heapify-while-begin}
\li		\Do
			\If $A[\id{child}]>A[\id{largest}]$
\li				\Then $\id{largest}\gets\id{child}$
				\End
\li			$k\gets k+1$
\li			$\id{child}\gets\proc{Multiary-Child}(d,k,i)$ \label{li:multiary-max-heapify-child}
		\End \label{li:multiary-max-heapify-while-end}
\li	\If $\id{largest}\ne i$
\li		\Then
			zamień $A[i]\leftrightarrow A[\id{largest}]$
\li			$\proc{Multiary-Max-Heapify}(A,d,\id{largest})$
		\End
\end{codebox}
Zauważmy, że podczas wykonywania pętli \kw{while} w~wierszach \doubledash{\ref{li:multiary-max-heapify-while-begin}}{\ref{li:multiary-max-heapify-while-end}} zmienna $k$ może przyjąć wartość $d+1$ i~wówczas w~wierszu \ref{li:multiary-max-heapify-child} zostaje wyznaczony indeks nieistniejącego, \singledash{$(d+1)$}{szego} syna węzła $i$.
Jednak wartości tej nigdzie później nie wykorzystujemy, ponieważ następną operacją jest przerwanie pętli \kw{while}.

Na każdym poziomie rekursji (z~wyjątkiem być może ostatniego) wykonywanych jest $\Theta(d)$ operacji.
Na mocy poprzedniego punktu mamy $\Theta(\log_dn)$ wywołań rekurencyjnych, a~zatem czasem działania powyższej procedury jest $\Theta(d\log_dn)$.

Procedura \proc{Multiary-Heap-Extract-Max}, która implementuje operację \proc{Extract-Max} dla \singledash{$d$}{kopca}, przyjmuje jako parametry kopiec $A$ oraz jego rząd $d$.
Działa ona identyczne jak operacja \proc{Extract-Max} dla kopca binarnego, jednak w~wierszu 6 zamiast procedury \proc{Max-Heapify} wywołuje procedurę \proc{Multiary-Max-Heapify} przedstawioną powyżej.
Czas działania tej operacji wynosi $\Theta(d\log_dn)$.

\subproblem %6-2(d)
Procedura \proc{Multiary-Max-Heap-Insert} implementująca operację \proc{Insert} dla \singledash{$d$}{kopca} przyjmuje na wejściu kopiec $A$, rząd kopca $d$ oraz wartość \id{key}, która będzie wstawiana do $A$.
Jej działanie jest analogiczne do działania procedury wstawiania węzła do kopca binarnego.
Jedyną różnicą jest wiersz \ref{li:multiary-max-heap-insert-increase-key}, który zamiast \proc{Heap-Increase-Key} zawiera analogiczne wywołanie procedury \proc{Multiary-Heap-Increase-Key} zwiększającej wartość węzła w~kopcu rzędu $d$.
\begin{codebox}
\Procname{$\proc{Multiary-Max-Heap-Insert}(A,d,\id{key})$}
\li	$\attrib{A}{heap-size}\gets\attrib{A}{heap-size}+1$
\li	$A[\attrib{A}{heap-size}]\gets-\infty$
\li	$\proc{Multiary-Heap-Increase-Key}(A,d,\attrib{A}{heap-size},\id{key})$ \label{li:multiary-max-heap-insert-increase-key}
\end{codebox}

Czas działania operacji \proc{Insert} dla \singledash{$d$}{kopca} jest tego samego rzędu co czas działania wywołania z~wiersza 3.
Implementacja wywoływanej procedury \proc{Multiary-Heap-Increase-Key} i~analiza jej czasu działania zostały opisane w~następnym punkcie.

\subproblem %6-2(e)
Implementacja tej operacji dla \singledash{$d$}{kopca} jest analogiczna do jej implementacji dla kopca binarnego.
Jednak zamiast sprawdzania poprawności parametru $k$, do $A[i]$ przypisujemy natychmiast odpowiednią wartość.
\begin{codebox}
\Procname{$\proc{Multiary-Heap-Increase-Key}(A,d,i,k)$}
\li	$A[i]\gets\max(A[i],k)$ \label{li:multiary-heap-increase-key}
\li	\While $i>1$ i~$A[\proc{Multiary-Parent}(d,i)]<A[i]$
\li		\Do
			zamień $A[i]\leftrightarrow A[\proc{Multiary-Parent}(d,i)]$
\li			$i\gets\proc{Multiary-Parent}(d,i)$
		\End
\end{codebox}

Po wykonaniu wiersza \ref{li:multiary-heap-increase-key} wartość węzła $i$ może być większa niż wartość jego ojca.
W~najgorszym przypadku, jeśli węzeł $i$ jest liściem i~jego nowa wartość jest największą wartością w~kopcu, to zostanie on przetransportowany aż do korzenia, co zajmie czas proporcjonalny do wysokości kopca, czyli, na mocy punktu (b), $\Theta(\log_dn)$.
