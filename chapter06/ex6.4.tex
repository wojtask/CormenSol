\subchapter{Algorytm sortowania przez kopcowanie (heapsort)}

\exercise %6.4-1
Na rys.\ \ref{fig:6.4-1} zilustrowano działanie sortowania przez kopcowanie dla tablicy $A$.
\begin{figure}[!ht]
	\centering \begin{tikzpicture}[
	marked node/.style = {tree node, med gray},
	every label/.style = {index node, draw=none, fill=none},
	outer/.append style = {node distance=12mm and 6mm},
	level/.append style = {level distance=7mm, sibling distance=42mm/2^#1},
	tree node/.append style = {minimum size=4mm},
]

\node[outer] (pic a) {
\begin{tikzpicture}[
	every node/.style = {tree node},
	anchor = center
]
	\node {25}
		child {node {13}
			child {node {8}
				child {node {5}}
				child {node {4}}
			}
			child {node {7}}
		}
		child {node {20}
			child {node {17}}
			child {node {2}
				child[missing]
				child {node[draw=none, fill=none] {} edge from parent[draw=none]}
			}
		};
\end{tikzpicture}
};

\node[outer, right=of pic a] (pic b) {
\begin{tikzpicture}[
	every node/.style = {tree node},
	anchor = center
]
	\node {20}
		child {node {13}
			child {node {8}
				child {node {5}}
				child {node[marked node, label=right:{$i$}] {25} edge from parent[draw=none]}
			}
			child {node {7}}
		}
		child {node {17}
			child {node {4}}
			child {node {2}
				child[missing]
				child {node[draw=none, fill=none] {} edge from parent[draw=none]}
			}
		};
\end{tikzpicture}
};

\node[outer, right=of pic b] (pic c) {
\begin{tikzpicture}[
	every node/.style = {tree node},
	anchor = center
]
	\node {17}
		child {node {13}
			child {node {8}
				child {node[marked node, label=left:{$i$}] {20} edge from parent[draw=none]}
				child {node[marked node] {25} edge from parent[draw=none]}
			}
			child {node {7}}
		}
		child {node {5}
			child {node {4}}
			child {node {2}
				child[missing]
				child {node[draw=none, fill=none, label=right:{}] {} edge from parent[draw=none]}
			}
		};
\end{tikzpicture}
};

\node[outer, below=of pic a] (pic d) {
\begin{tikzpicture}[
	every node/.style = {tree node},
	anchor = center
]
	\node {13}
		child {node {8}
			child {node {2}
				child {node[marked node] {20} edge from parent[draw=none]}
				child {node[marked node] {25} edge from parent[draw=none]}
			}
			child {node {7}}
		}
		child {node {5}
			child {node {4}}
			child {node[marked node, label=left:{$i$}] {17} edge from parent[draw=none]
				child[missing]
				child {node[draw=none, fill=none] {} edge from parent[draw=none]}
			}
		};
\end{tikzpicture}
};

\node[outer, below=of pic b] (pic e) {
\begin{tikzpicture}[
	every node/.style = {tree node},
	anchor = center
]
	\node {8}
		child {node {7}
			child {node {2}
				child {node[marked node] {20} edge from parent[draw=none]}
				child {node[marked node] {25} edge from parent[draw=none]}
			}
			child {node {4}}
		}
		child {node {5}
			child {node[marked node, label=left:{$i$}] {13} edge from parent[draw=none]}
			child {node[marked node] {17} edge from parent[draw=none]
				child[missing]
				child {node[draw=none, fill=none] {} edge from parent[draw=none]}
			}
		};
\end{tikzpicture}
};

\node[outer, below=of pic c] (pic f) {
\begin{tikzpicture}[
	every node/.style = {tree node},
	anchor = center
]
	\node {7}
		child {node {4}
			child {node {2}
				child {node[marked node] {20} edge from parent[draw=none]}
				child {node[marked node] {25} edge from parent[draw=none]}
			}
			child {node[marked node, label=left:{$i$}] {8} edge from parent[draw=none]}
		}
		child {node {5}
			child {node[marked node] {13} edge from parent[draw=none]}
			child {node[marked node] {17} edge from parent[draw=none]
				child[missing]
				child {node[draw=none, fill=none] {} edge from parent[draw=none]}
			}
		};
\end{tikzpicture}
};

\node[outer, below=of pic d] (pic g) {
\begin{tikzpicture}[
	every node/.style = {tree node},
	anchor = center
]
	\node {5}
		child {node {4}
			child {node[marked node, label=left:{$i$}] {7} edge from parent[draw=none]
				child {node[marked node] {20} edge from parent[draw=none]}
				child {node[marked node] {25} edge from parent[draw=none]}
			}
			child {node[marked node] {8} edge from parent[draw=none]}
		}
		child {node {2}
			child {node[marked node] {13} edge from parent[draw=none]}
			child {node[marked node] {17} edge from parent[draw=none]
				child[missing]
				child {node[draw=none, fill=none] {} edge from parent[draw=none]}
			}
		};
\end{tikzpicture}
};

\node[outer, below=of pic e] (pic h) {
\begin{tikzpicture}[
	every node/.style = {tree node},
	anchor = center
]
	\node {4}
		child {node {2}
			child {node[marked node] {7} edge from parent[draw=none]
				child {node[marked node] {20} edge from parent[draw=none]}
				child {node[marked node] {25} edge from parent[draw=none]}
			}
			child {node[marked node] {8} edge from parent[draw=none]}
		}
		child {node[marked node, label=left:{$i$}] {5} edge from parent[draw=none]
			child {node[marked node] {13} edge from parent[draw=none]}
			child {node[marked node] {17} edge from parent[draw=none]
				child[missing]
				child {node[draw=none, fill=none] {} edge from parent[draw=none]}
			}
		};
\end{tikzpicture}
};

\node[outer, below=of pic f] (pic i) {
\begin{tikzpicture}[
	every node/.style = {tree node},
	anchor = center
]
	\node {2}
		child {node[marked node, label=left:{$i$}] {4} edge from parent[draw=none]
			child {node[marked node] {7} edge from parent[draw=none]
				child {node[marked node] {20} edge from parent[draw=none]}
				child {node[marked node] {25} edge from parent[draw=none]}
			}
			child {node[marked node] {8} edge from parent[draw=none]}
		}
		child {node[marked node] {5} edge from parent[draw=none]
			child {node[marked node] {13} edge from parent[draw=none]}
			child {node[marked node] {17} edge from parent[draw=none]
				child[missing]
				child {node[draw=none, fill=none] {} edge from parent[draw=none]}
			}
		};
\end{tikzpicture}
};

\node[outer, below=of pic g] (pic j) {
\begin{tikzpicture}[
	row 1/.style = {nodes={light gray}}
]
	\matrix[array] (arr) {2 & 4 & 5 & 7 & 8 & 13 & 17 & 20 & 25 \\};
	\node[left=3mm of arr-1-1.west] {$A$};
\end{tikzpicture}
};

\foreach \x in {a,b,c,d,e,f,g,h,i,j} {
	\node[subpicture label, below=1mm of pic \x] {(\x)};
}

\end{tikzpicture}

	\caption{Działanie procedury \proc{Heapsort} dla tablicy $A=\langle5,13,2,25,7,17,20,8,4\rangle$.
{\sffamily\bfseries(a)} Kopiec zaraz po jego zbudowaniu przez \proc{Build-Max-Heap}.
{\sffamily\bfseries(b)\nbendash(i)} Kopiec i~elementy z~niego usunięte po każdym wywołaniu \proc{Max-Heapify} w~wierszu 5.
{\sffamily\bfseries(j)} Wynikowa posortowana tablica.} \label{fig:6.4-1}
\end{figure}

\exercise %6.4-2
Poprawność algorytmu wynika z~dowodu podanego niezmiennika pętli \kw{for}:
\begin{description}
	\item[Inicjowanie:] Przed pierwszą iteracją pętli mamy $i=\attrib{A}{length}=n$.
Wówczas fragment $A[1\twodots i]$ jest całą tablicą $A$, która stanowi kopiec typu max, utworzony w~wyniku działania procedury \proc{Build-Max-Heap}, natomiast fragment $A[i+1\twodots n]$ jest pusty, a~więc trywialnie posortowany.
	\item[Utrzymanie:] Załóżmy, że niezmiennik jest prawdziwy przed wykonaniem kolejnej iteracji pętli.
Podtablica $A[1\twodots i]$ tworzy więc kopiec typu max, którego korzeniem jest $A[1]$, czyli największy element w~tej podtablicy.
W~wyniku wykonania wiersza 3 zostaje on przeniesiony na pozycję $i$, zaś dekrementacja \attrib{A}{heap-size} powoduje, że przestaje on wchodzić w~skład kopca.
Z~kolei podtablica $A[i\twodots n]$ składa się $n-i+1$ największych elementów tablicy $A$ posortowanych niemalejąco, ponieważ element $A[i]$ jest mniejszy lub równy pozostałym elementom z~tej podtablicy.
W~tym momencie korzeń może naruszać własność kopca typu max, dlatego w~wierszu 5 zostaje wywołana na nim procedura \proc{Max-Heapify} przywracająca tę własność.
Uaktualnienie $i$ powoduje odtworzenie niezmiennika.
	\item[Zakończenie:] Po zakończeniu działania pętli jest $i=1$, zatem podtablica $A[1\twodots i]$ składa się z~jednego elementu, który jest najmniejszym elementem tablicy $A$.
Ponadto $n-1$ pozostałych elementów jest uszeregowanych w~kolejności niemalejącej w~podtablicy $A[2\twodots n]$.
To oznacza, że tablica $A$ jest posortowana.
\end{description}

\exercise %6.4-3
Na podstawie analizy w~Podręczniku czasem działania algorytmu heapsort dla tablicy $n$\nbhyphen elementowej jest $O(n\lg n)$.
W~\refExercise{6.4-5} pokazane jest natomiast dolne oszacowanie $\Omega(n\lg n)$ na optymistyczny czas działania.
A~zatem dla dowolnej tablicy o~długości $n$, w~szczególności dla tablicy posortowanej rosnąco lub malejąco, algorytm heapsort działa w~czasie $\Theta(n\lg n)$.

\exercise %6.4-4
Przypadek pesymistyczny algorytmu heapsort ma miejsce wtedy, gdy każde wywołanie \proc{Max-Heapify} z~wiersza 5 schodzi rekurencyjnie aż do ostatniego poziomu drzewa.
Na mocy wyniku z~\refExercise{6.2-6} oraz wzoru (3.18) wywołania te zabierają łącznie czas
\[
	\sum_{i=2}^n\Omega(\lg i) = \Omega(\lg(n!)) = \Omega(n\lg n)
\]
i~taki jest też pesymistyczny czas działania algorytmu heapsort.

\exercise %6.4-5
Udowodnimy dolne oszacowanie $\Omega(n\lg n)$ na czas działania algorytmu heapsort przez pokazanie, że zachodzi ono dla kopców o~wysokości $h\ge4$ będących pełnymi drzewami binarnymi, czyli takich, w~których $n=2^{h+1}-1$.

Niech $L$ będzie zbiorem $2^h$ największych elementów kopca.
Algorytm usunie te elementy z~kopca w~pierwszej kolejności, pozbawiając go tym samym jego $h$\nbhyphen tego poziomu.
Elementy z~$L$ w~początkowym kopcu tworzą poddrzewo binarne, tzn.\ jeśli dany węzeł kopca należy do $L$, to jego ojciec także, co wynika z~własności kopca typu max.
Oznaczmy teraz przez $L'$ podzbiór elementów z~$L$ zajmujących w~kopcu poziomy 0, 1, \dots, $h-1$.
Elementy z~$L'$ mogą zostać usunięte z~kopca jedynie dzięki przetransportowaniu ich do korzenia za pomocą zamian z~linii 9 procedury \proc{Max-Heapify}.
Jednym ze sposobów na znalezienie dolnego oszacowania na czas działania algorytmu heapsort jest oszacowanie sumarycznej liczby $s$ tych zamian wykonanych na elementach z~$L'$.

W~tym celu musimy znaleźć rozmiar zbioru $L'$.
Załóżmy, że na poziomie $h$ znajduje się co najmniej $2^{h-1}+1$ węzłów z~$L$.
Ojciec każdego takiego węzła znajdujący się na poziomie $h-1$ także należy do $L$, więc w~najlepszym przypadku na poziomie $h-1$ jest ich co najmniej $2^{h-2}+1$.
To z~kolei oznacza, że na poziomie $h-2$ jest co najmniej $2^{h-3}+1$ węzłów z~$L$ itd.
Sumując węzły z~$L$ ze wszystkich poziomów, mamy ich $2^{h-1}+2^{h-2}+2^{h-3}+\dots+2^0+(h+1)=2^h+h>2^h$.
Wynik ten jest jednak sprzeczny z~definicją zbioru $L$, co oznacza, że początkowe założenie było błędne i~na poziomie $h$ jest maksymalnie $2^{h-1}$ węzłów.
Stąd wynika, że $|L'|\ge|L|-2^{h-1}=2^{h-1}$.

Aby wynieść element $j$ z~poziomu $d_j$ na poziom 0, musi zostać wykonanych $d_j$ zamian w~procedurze \proc{Max-Heapify}.
Sumaryczna ich liczba dla elementów z~$L'$ wynosi zatem $s=\sum_{j\in L'}d_j$.
Suma ta minimalizowana jest, gdy elementy z~$L'$ tworzą pełne drzewo binarne z~ewentualnie niepełnym ostatnim poziomem.
W~takiej postaci $L'$ wypełnia $h-2$ początkowych poziomów kopca, skąd
\[
	s \ge \sum_{k=0}^{h-2}k2^k.
\]
Sumę tę obliczamy przez skorzystanie ze wzoru (A.5):
\[
    \sum_{k=0}^{h-2}kx^k = x\cdot\frac{d}{dx}\biggl(\sum_{k=0}^{h-2}x^k\biggr) = x\cdot\frac{d}{dx}\biggl(\frac{x^{h-1}-1}{x-1}\biggr) = x\,\frac{(h-1)x^{h-2}(x-1)-(x^{h-1}-1)}{(x-1)^2}.
\]
Przyjmując teraz $x=2$ i~korzystając z~zależności $h=\lg(n+1)-1>\lg n-1$, mamy ostatecznie
\[
    s \ge \sum_{k=0}^{h-2}k2^k = (h-1)2^{h-1}-2^h+2 > (h-3)2^{h-1} > (\lg n-4)2^{\lg n-2} = \frac{n\lg n}{4}-n = \Omega(n\lg n).
\]
