\subchapter{Kopce}

\exercise %6.1-1
Kopiec reprezentowany jest przez prawie pełne drzewo binarne, tzn.\ takie, w~którym wszystkie poziomy, być może z~wyjątkiem najniższego, zawierają komplet węzłów.
Jeśli drzewo to ma wysokość $h$, to jego maksymalna liczba węzłów wynosi $2^{h+1}-1$ (gdy jest drzewem pełnym), a~minimalna $(2^h-1)+1=2^h$ (gdy jego najniższy poziom składa się z~tylko jednego węzła).

\exercise %6.1-2
Niech $h$ oznacza wysokość kopca.
Z~poprzedniego zadania mamy, że $2^h\le n<2^{h+1}$, skąd dostajemy $h\le\lg n<h+1$.
Ponieważ $h$ jest całkowite, to $h=\lfloor\lg n\rfloor$.

\exercise %6.1-3
Załóżmy, że największy element danego poddrzewa kopca typu max nie znajduje się w~korzeniu tego poddrzewa.
Jeśli w~poddrzewie jest więcej niż jeden największy element, to rozważmy ten o~najniższej pozycji w~kopcu.
Wtedy ojciec tego elementu także należy do poddrzewa.
Jednak to prowadzi do sprzeczności, bo ojciec jest mniejszy niż jego syn i~naruszona zostaje własność kopca typu max.

\exercise %6.1-4
Z~własności kopca typu max o~różnych elementach, dowolny element kopca będący ojcem jest mniejszy od swoich synów, a~więc nie może być najmniejszy w~kopcu.
Wynika stąd, że najmniejszy element kopca znajduje się wśród jego liści.

\exercise %6.1-5
Stosując rozumowanie z~\refExercise{6.1-3} dla kopca typu min, wnioskujemy, że jego korzeń stanowi jego najmniejszy element, czyli zajmuje pierwszą pozycję w~posortowanej tablicy $A$.
Dla każdego indeksu tablicy $i>1$ zachodzi $\proc{Parent}(i)<i$, a~więc $A[\proc{Parent}(i)]\le A[i]$.
Własność kopca typu min jest zatem spełniona i~tablica posortowana $A$ stanowi taki kopiec.

\exercise %6.1-6
Potraktujmy ten ciąg jak tablicę $A$.
Elementy na pozycjach $i=9$ oraz $\proc{Parent}(i)=4$ nie spełniają własności $A[\proc{Parent}(i)]\ge A[i]$, zatem tablica $A$ nie jest kopcem typu max.

\exercise %6.1-7
Element kopca na pozycji $i$ nie jest liściem wtedy i~tylko wtedy, gdy istnieje jego lewy syn.
W~kopcu o~$n$ elementach wierzchołki wewnętrzne znajdują się zatem na pozycjach $i$ takich, że $\proc{Left}(i)\le n$.
Warunek ten sprowadza się do nierówności $2i\le n$, skąd $i\le\lfloor n/2\rfloor$, bo $i$ jest całkowite.
Pozostałe wierzchołki są liśćmi i~zajmują pozycje $\lfloor n/2\rfloor+1\twodots n$.
