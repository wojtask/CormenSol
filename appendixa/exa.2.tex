\subchapter{Szacowanie sum}

\exercise %A.2-1
Wykorzystując własności szeregu teleskopowego, dostajemy
\[
	\sum_{k=1}^n\frac{1}{k^2} \le 1+\sum_{k=2}^n\frac{1}{k(k-1)} = 1+\sum_{k=2}^n\biggl(\frac{1}{k-1}-\frac{1}{k}\biggr) = 1+1-\frac{1}{n} < 2.
\]

\exercise %A.2-2
Dzięki nierówności $\lceil x\rceil<x+1$ (wzór (3.3)) oraz tożsamości (A.6), możemy napisać:
\begin{align*}
	\sum_{k=0}^{\lfloor\lg n\rfloor}\biggl\lceil\frac{n}{2^k}\biggr\rceil &< \sum_{k=0}^{\lfloor\lg n\rfloor}\biggl(\frac{n}{2^k}+1\biggr) \le \lg n+1+\sum_{k=0}^{\lfloor\lg n\rfloor}\frac{n}{2^k} < \lg n+1+\sum_{k=0}^\infty\frac{n}{2^k} \\[2mm]
	&= \lg n+1+\frac{n}{1-1/2} = \lg n+1+2n = O(n).
\end{align*}

\exercise %A.2-3
Postępując podobnie jak podczas badania oszacowania górnego $n$\nbhyphen tej liczby harmonicznej, dzielimy zakres indeksów od 1 do $n$ na $\lfloor\lg n\rfloor$ części i~ograniczamy sumę każdej części przez $1/2$:
\[
    \sum_{k=1}^n\frac{1}{k} \ge \sum_{i=0}^{\lfloor\lg n\rfloor-1}\sum_{j=0}^{2^i-1}\frac{1}{2^i+j} \ge \sum_{i=0}^{\lfloor\lg n\rfloor-1}\sum_{j=0}^{2^i-1}\frac{1}{2^{i+1}} = \sum_{i=0}^{\lfloor\lg n\rfloor-1}\frac{1}{2} = \frac{\lfloor\lg n\rfloor}{2} = \Omega(\lg n).
\]

\exercise %A.2-4
Funkcja $f(k)=k^3$, gdzie $k$ to dodatnia liczba całkowita, jest monotonicznie rosnąca, zatem szacujemy sumę, korzystając z~nierówności (A.11):
\[
	\int_0^nx^3\,dx \le \sum_{k=1}^nk^3 \le \int_1^{n+1}x^3\,dx.
\]
Dolnym oszacowaniem sumy jest więc
\[
	\int_0^nx^3\,dx = \frac{n^4}{4},
\]
a~górnym
\[
	\int_1^{n+1}x^3\,dx = \frac{(n+1)^4-1}{4}.
\]

\exercise %A.2-5
Zastosowanie nierówności (A.12) do sumy $\sum_{k=1}^n1/k$ doprowadza do całki niewłaściwej:
\[
	\sum_{k=1}^n\frac{1}{k} \le \int_0^n\frac{dx}{x} = \lim_{a\to0^+}\int_a^n\frac{dx}{x} = \infty.
\]
W~rezultacie nie uzyskujemy żadnej informacji o~oszacowaniu górnym badanej sumy.
Dzięki zapisaniu jej w~postaci $1+\sum_{k=2}^n1/k$, można zastosować wzór (A.12) do drugiego składnika i~otrzymać oszacowanie górne $\ln n+1$ na sumę wyjściową.
