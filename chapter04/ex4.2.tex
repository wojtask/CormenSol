\subchapter{Metoda drzewa rekursji}

\exercise %4.2-1
Dokonamy pewnego uproszczenia, pomijając podłogę w~argumencie rekurencji i~rozważając zależność $T(n)=3T(n/2)+n$.
\begin{figure}[!ht]
	\centering \begin{tikzpicture}[
	level/.style = {level distance=15mm, sibling distance=100mm/3^#1},
	level 3/.style = {level distance=7mm, sibling distance=4mm},
	arrow/.style = {->, >=stealth, dashed},
	node distance = 6mm and 0mm
]

\node (root) {$n$}
	child {node (0) {$\frac{n}{2}$}
		child {node (00) {$\frac{n}{4}$}
			child {node (000) {}}
			child {node {}}
			child {node {}}
		}
		child {node (01) {$\frac{n}{4}$}
			child {node {}}
			child {node {}}
			child {node {}}	
		}
		child {node (02) {$\frac{n}{4}$}
			child {node {}}
			child {node {}}
			child {node {}}
		}
	}
	child {node (1) {$\frac{n}{2}$}
		child {node (10) {$\frac{n}{4}$}
			child {node {}}
			child {node {}}
			child {node {}}
		}
		child {node (11) {$\frac{n}{4}$}
			child {node {}}
			child {node {}}
			child {node {}}
		}
		child {node (12) {$\frac{n}{4}$}
			child {node {}}
			child {node {}}
			child {node {}}
		}
	}
	child {node (2) {$\frac{n}{2}$}
		child {node (20) {$\frac{n}{4}$}
			child {node {}}
			child {node {}}
			child {node {}}
		}
		child {node (21) {$\frac{n}{4}$}
			child {node {}}
			child {node {}}
			child {node {}}
		}
		child {node (22) {$\frac{n}{4}$}
			child {node {}}
			child {node {}}
			child {node (222) {}}
		}
	};
	
\node[below=of 000, font=\scriptsize] (t0) {$T(1)$};
\foreach[count=\i] \x in {0, ..., 9} {
	\node[right=of t\x, font=\scriptsize] (t\i) {$T(1)$};
}
\node[below=of 222, font=\scriptsize] (tlast) {$T(1)$};
\path (t10) -- (tlast) node[midway] {$\dots$};
\foreach \x in {0, ..., 9, last} {
	\draw (t\x.north) -- +(0mm, 1mm);
}

\node[right=60mm of root] (level 0 total) {$n$};
\node at (level 0 total |- 2) (level 1 total) {$\frac{3}{2}n$};
\node at (level 0 total |- 22) (level 2 total) {$\bigl(\frac{3}{2}\bigr)^2n$};
\node at (level 0 total |- tlast) (last level total) {$\Theta(n^{\lg3})$};

\path (01) -- (01 |- t0) node[midway] {$\vdots$};
\path (10) -- (10 |- t0) node[midway] {$\vdots$};
\path (12) -- (12 |- t0) node[midway] {$\vdots$};
\path (21) -- (21 |- t0) node[midway] {$\vdots$};
\path (level 2 total) -- (last level total) node[midway] {$\vdots$};

\draw[arrow] (root) -- (level 0 total);
\draw[arrow] (2) -- (level 1 total);
\draw[arrow] (22) -- (level 2 total);
\draw[arrow] (tlast) -- (last level total);

\draw[decorate, decoration={brace, amplitude=10pt, mirror}] (t0.south west) -- (tlast.south east) node[midway, yshift=-7mm] {\footnotesize $n^{\lg3}$};

\node[below=3mm of last level total.south east, anchor=north east] (total) {Łącznie: $O(n^{\lg3})$};
\draw (total.north east) -- ++(-30mm, 0);

\coordinate[left=3mm of t0] (bottom);
\draw[<->, >=stealth, thick] (bottom) -- (bottom |- root) node[midway, fill=white] {$\lg n$};

\end{tikzpicture}

	\caption{Drzewo rekursji dla równania rekurencyjnego $T(n)=3T(n/2)+n$.} \label{fig:4.2-1}
\end{figure}
Drzewo tej rekursji przedstawione na rys.\ \ref{fig:4.2-1} ma wysokość równą $\lg n$, czyli jest w~nim $\lg n+1$ poziomów.
Na \singledash{$i$}{tym} poziomie znajduje się $3^i$ węzłów, zatem jest $n^{\lg3}$ liści.
Koszt węzła na poziomie $i$ wynosi $n/2^i$, skąd wynika, że łączny koszt wszystkich węzłów na \singledash{$i$}{tej} głębokości jest równy $(3/2)^in$.
Przyjmujemy, że $T(1)=1$, a~więc na ostatnim poziomie jest $\Theta(n^{\lg3})$.
Na podstawie tych wartości rozwiązujemy rekurencję:
\begin{align*}
	T(n) &= n+\frac{3}{2}\,n+\biggl(\frac{3}{2}\biggr)^2n+\dots+\biggl(\frac{3}{2}\biggr)^{\lg n-1}n+\Theta(n^{\lg3}) \\
	&= \sum_{i=0}^{\lg n-1}\biggl(\frac{3}{2}\biggr)^in+\Theta(n^{\lg3}) \\
	&= \frac{(3/2)^{\lg n}-1}{(3/2)-1}\,n+\Theta(n^{\lg3}) \\[1mm]
	&= 2n(n^{\lg3-1}-1)+\Theta(n^{\lg3}) \\
	&= O(n^{\lg3}).
\end{align*}

Pokażemy teraz, że otrzymany wynik stanowi górne oszacowanie dla oryginalnej rekurencji.
Przyjmujemy założenie, że
\[
    T(\lfloor n/2\rfloor)\le c\lfloor n/2\rfloor^{\lg3}
\]
dla pewnej stałej $c>0$.
Dowodzimy oszacowania dla oryginalnej rekurencji metodą podstawiania:
\[
	T(n) \le 3c\lfloor n/2\rfloor^{\lg3}+n \le 3c(n/2)^{\lg3}+n = cn^{\lg3}+n.
\]
Nie możemy jednak na podstawie tego wyniku wywnioskować szukanego oszacowania.
Wzmocnijmy zatem nasze założenie, niech
\[
	T(\lfloor n/2\rfloor) \le c\lfloor n/2\rfloor^{\lg3}-b\lfloor n/2\rfloor,
\]
gdzie $b\ge0$ jest nową stałą.
Korzystając ze wzoru (3.3), mamy teraz:
\begin{align*}
	T(n) &\le 3c\lfloor n/2\rfloor^{\lg3}-3b\lfloor n/2\rfloor+n \\
	&< 3c(n/2)^{\lg3}-3b(n/2-1)+n \\
	&= cn^{\lg3}-3bn/2+3b+n \\
	&\le cn^{\lg3}-bn,
\end{align*}
co zachodzi dla $n\ge7$, o~ile $b\ge14$.
Przyjmujemy wszystkie $n=3$, 4, 5, 6 na podstawę indukcji, ponieważ dla dowolnie ustalonego $b$ można dobrać dla stałej $c$ odpowiednią wartość tak, aby pokazane oszacowanie zachodziło dla takich $n$.
To kończy dowód faktu, że górnym oszacowaniem rekurencji $T(n)$ jest $O(n^{\lg3})$.

\exercise %4.2-2
Drzewo rekursji $T(n)$ nie jest pełnym drzewem binarnym.
Najkrótszą ścieżką od korzenia do liścia jest $cn\to cn/3\to cn/9\to\dots\to cn/3^i\to\dots\to T(1)$.
Liść tej gałęzi znajduje się na poziomie $h=\log_3n$.
Gdybyśmy pominęli wszystkie poziomy tego drzewa poniżej \singledash{$h$}{tego}, to uzyskalibyśmy pełne drzewo binarne o~wysokości $h$ i~łącznym koszcie $\Theta(n\lg n)$.
Ale tak zmodyfikowane drzewo jest mniejsze niż drzewo rekursji $T(n)$, przez co jego koszt stanowi oszacowanie dolne rekurencji $T(n)$, czyli $T(n)=\Omega(n\lg n)$.

\exercise %4.2-3
W~metodzie drzewa rekursji dla uproszczenia pominiemy branie części całkowitych.
\begin{figure}[!ht]
	\centering \begin{tikzpicture}[
	level/.style = {level distance=15mm, sibling distance=100mm/4^#1},
	level 3/.style = {edge from parent/.style = {halved path={solid}{very densely dashed}}},
	level arrow/.style = {arrow, thick, dashed},
	total/.style = {text width=9mm, align=right}
]

\node (root) {$cn$}
	child {node (0) {$\frac{cn}{2}$}
		child {node (00) {$\frac{cn}{4}$}
			child {node (000) {}}
			child {node {}}
			child {node {}}
			child {node {}}
		}
		child {node (01) {$\frac{cn}{4}$}
			child {node {}}
			child {node {}}
			child {node {}}
			child {node {}}	
		}
		child {node (02) {$\frac{cn}{4}$}
			child {node {}}
			child {node {}}
			child {node {}}
			child {node {}}
		}
		child {node (03) {$\frac{cn}{4}$}
			child {node {}}
			child {node {}}
			child {node {}}
			child {node {}}
		}
	}
	child {node (1) {$\frac{cn}{2}$}
		child {node (10) {$\frac{cn}{4}$}
			child {node {}}
			child {node {}}
			child {node {}}
			child {node {}}
		}
		child {node (11) {$\frac{cn}{4}$}
			child {node {}}
			child {node {}}
			child {node {}}
			child {node {}}
		}
		child {node (12) {$\frac{cn}{4}$}
			child {node {}}
			child {node {}}
			child {node {}}
			child {node {}}
		}
		child {node (13) {$\frac{cn}{4}$}
			child {node {}}
			child {node {}}
			child {node {}}
			child {node {}}
		}
	}
	child {node (2) {$\frac{cn}{2}$}
		child {node (20) {$\frac{cn}{4}$}
			child {node {}}
			child {node {}}
			child {node {}}
			child {node {}}
		}
		child {node (21) {$\frac{cn}{4}$}
			child {node {}}
			child {node {}}
			child {node {}}
			child {node {}}
		}
		child {node (22) {$\frac{cn}{4}$}
			child {node {}}
			child {node {}}
			child {node {}}
			child {node {}}
		}
		child {node (23) {$\frac{cn}{4}$}
			child {node {}}
			child {node {}}
			child {node {}}
			child {node {}}
		}
	}
	child {node (3) {$\frac{cn}{2}$}
		child {node (30) {$\frac{cn}{4}$}
			child {node {}}
			child {node {}}
			child {node {}}
			child {node {}}
		}
		child {node (31) {$\frac{cn}{4}$}
			child {node {}}
			child {node {}}
			child {node {}}
			child {node {}}
		}
		child {node (32) {$\frac{cn}{4}$}
			child {node {}}
			child {node {}}
			child {node {}}
			child {node {}}
		}
		child {node (33) {$\frac{cn}{4}$}
			child {node {}}
			child {node {}}
			child {node {}}
			child {node (333) {}}
		}
	};
	
\node[below=5mm of 000] (leaf 0) {$T(1)$};
\foreach[count=\i] \x in {0, ..., 7} {
	\node[right=0mm of leaf \x] (leaf \i) {$T(1)$};
}
\node[below=5mm of 333] (leaf n) {$T(1)$};
\path (leaf 8) -- (leaf n) node[midway] {$\dots$};
\foreach \x in {0, ..., 8, n} {
	\draw[halved path={draw=none}{very densely dashed}] (leaf \x |- 000) -- (leaf \x.north);
}

\node[right=58mm of root, total] (level 0 total) {$cn$};
\node[total] at (level 0 total |- 2) (level 1 total) {$2cn$};
\node[total] at (level 0 total |- 22) (level 2 total) {$2^2cn$};
\node[total] at (level 0 total |- leaf n) (last level total) {$\Theta(n^2)$};

\path (level 2 total) -- (last level total) node[midway] {$\vdots$};

\draw[level arrow] (root) -- (level 0 total);
\draw[level arrow] (3) -- (level 1 total);
\draw[level arrow] (33) -- (level 2 total);
\draw[level arrow] (leaf n) -- (last level total);

\draw[thick, decorate, decoration={brace, amplitude=8pt, mirror}] (leaf 0.south west) -- (leaf n.south east) node[midway, yshift=-6mm] {$n^2$};

\node[anchor=east] at (last level total.east |- leaves total) (total) {Łącznie: $\Theta(n^2)$};
\draw[thick] (total.north west) -- (total.north east);

\coordinate[left=3mm of leaf 0] (bottom);
\draw[arrow, <->, thick] (bottom) -- (bottom |- root) node[midway, fill=white] {$\lg n$};

\end{tikzpicture}

	\caption{Drzewo rekursji dla równania rekurencyjnego $T(n)=4T(n/2)+cn$.} \label{fig:4.2-3}
\end{figure}
W~drzewie rekursji z~rys.\ \ref{fig:4.2-3} na \singledash{$i$}{tym} poziomie jest $4^i$ węzłów, z~których każdy wnosi koszt równy $cn/2^i$.
Stąd kosztem całego poziomu jest $2^icn$.
Współczynnik przy $n$ w~koszcie węzła maleje dwukrotnie wraz z~głębokością drzewa, więc jego wysokością jest $\lg n$.
Wnioskujemy zatem, że liczbą liści w~tym drzewie jest $4^{\lg n}=n^2$ i~że koszt ostatniego poziomu wynosi $\Theta(n^2)$.
Sumując koszty z~każdego poziomu, otrzymujemy:
\begin{align*}
	T(n) &= cn+2cn+2^2cn+\dots+2^{\lg n-1}cn+\Theta(n^2) \\
	&= cn\sum_{i=0}^{\lg n-1}2^i+\Theta(n^2) \\
	&= cn(2^{\lg n}-1)+\Theta(n^2) \\
	&= cn(n-1)+\Theta(n^2) \\
	&= \Theta(n^2).
\end{align*}

Sprawdzimy teraz otrzymany wynik, używając w~tym celu metody podstawiania dla oryginalnej rekurencji.
Przyjmiemy ponadto, że $T(1)=1$ stanowi jej przypadek brzegowy.
Badamy najpierw oszacowanie dolne $T(n)$, wykorzystując założenie
\[
	T(\lfloor n/2\rfloor) \ge c_1\lfloor n/2\rfloor^2
\]
dla pewnej stałej $c_1>0$.
Stąd
\[
	T(n) \ge 4c_1\lfloor n/2\rfloor^2+cn > 4c_1(n/2-1)^2+cn = c_1n^2-4c_1n+4c_1+cn \ge c_1n^2,
\]
co jest prawdą dla $n\ge2$, jeśli przyjmiemy, że $c_1\le c/4$.
Podstawą indukcji jest $n=1$, bo $T(1)$ spełnia oszacowanie dla $c_1\le1$.
A~więc istotnie $T(n)=\Omega(n^2)$.

Pokażemy teraz, że $T(n)=O(n^2)$.
W~tym celu można przyjąć analogiczne założenie indukcyjne jak przy dowodzie dolnego oszacowania.
Niestety założenie to okazuje się zbyt słabe i~nie prowadzi do żądanego wyniku.
Przyjmijmy zatem, że dla stałych $c_2>0$ i~$c_3\ge0$ zachodzi mocniejszy warunek
\[
	T(\lfloor n/2\rfloor) \le c_2\lfloor n/2\rfloor^2-c_3\lfloor n/2\rfloor.
\]
Wówczas:
\begin{align*}
	T(n) &\le 4(c_2\lfloor n/2\rfloor^2-c_3\lfloor n/2\rfloor)+cn \\
	&< 4(c_2(n/2)^2-c_3(n/2-1))+cn \\
	&= c_2n^2-2c_3n+4c_3+cn \\
	&\le c_2n^2-c_3n.
\end{align*}
Ostatnia nierówność jest prawdziwa dla $n\ge5$, o~ile $c_3\ge5c$.

Zajmiemy się teraz ustaleniem stałych $c_2$ i~$c_3$, aby spełnić podstawę tej indukcji.
Początkowe wyrazy rekurencji wynoszą $T(1)=1$, $T(2)=4+2c$, $T(3)=4+3c$, $T(4)=16+12c$.
Jeśli przyjmiemy $c_3=5c$, to dla dowolnego $c_2\ge5c+1$ wyrażenie $c_2n^2-c_3n$ będzie stanowić oszacowanie górne dla tych wartości.
Zachodzi zatem $T(n)=O(n^2)$ i~dowód dokładnego oszacowania na $T(n)$ jest zakończony.

\exercise %4.2-4
Dla uproszczenia przyjmijmy, że $T(n)=ca$ w~przypadku, gdy $n\le a$, czyli że dla dostatecznie małego $n$ rekurencja przyjmuje wartość stałą równą $ca$.
Drzewo rekursji $T(n)$ zostało zilustrowane na rys.\ \ref{fig:4.2-4}.
\begin{figure}[!ht]
	\centering \begin{tikzpicture}[
	level/.style = {level distance=10mm, sibling distance=100mm/2^#1},
	arrow/.style = {->, >=stealth, dashed},
	node distance = 2mm and 8mm
]

\scoped[every node/.style = {inner sep=3pt, minimum width=10mm}]
\node (root) {$cn$}
	child {node (0) {$c(n-a)$}
		child {node (00) {$c(n-2a)$}
			child {node (000) {$c(n-3a)$}
				child {node (0000) {$T(1)$} edge from parent[draw=none, label={$\vdots$}]}
			}
			child {node (001) {$ca$}}
		}
		child {node (01) {$ca$}}
	}
	child {node (1) {$ca$}};
\path (000) -- (0000) node[midway] {$\vdots$};

\scoped[every node/.style = {text width=17mm, align=right}]
\node[right=35mm of root] (level 0 total) {$cn$}
	[edge from parent/.style = {draw=none}]
	child {node (level 1 total) {$cn$}
		child {node (level 2 total) {$c(n-a)$}
			child {node (level 3 total) {$c(n-2a)$}
				child {node (last level total) {$\Theta(1)$}}
			}
		}
	};
\path (level 3 total) -- (last level total) node[midway] {$\vdots$};

\draw[arrow] (root) -- (level 0 total);
\draw[arrow] (1) -- (level 1 total);
\draw[arrow] (01) -- (level 2 total);
\draw[arrow] (001) -- (level 3 total);
\draw[arrow] (0000) -- (last level total);

\node[below=of last level total.south east, anchor=north east] {Łącznie: $\Theta(n^2)$};
\draw (last level total.south east) -- ++(-30mm, 0);

\coordinate[left=of 0000] (bottom);
\draw[<->, >=stealth, thick] (bottom) -- (bottom |- root) node[midway, fill=white] {$n/a$};

\end{tikzpicture}

	\caption{Drzewo rekursji dla równania rekurencyjnego $T(n)=T(n-a)+T(a)+cn$.} \label{fig:4.2-4}
\end{figure}
Wysokością tego drzewa jest $n/a$.
Koszt wnoszony przez \singledash{$i$}{ty} poziom (z~wyjątkiem zerowego i~ostatniego) wynosi $c(n-a(i-1))$, przy czym na ostatnim poziomie jest tylko jeden liść, który kosztuje $\Theta(1)$.
Mamy zatem:
\begin{align*}
	T(n) &= cn+\sum_{i=1}^{n/a-1}c(n-a(i-1))+\Theta(1) \\
	&= cn+c\sum_{i=0}^{n/a-2}(n-ai)+\Theta(1) \\
	&= cn+cn\sum_{i=0}^{n/a-2}1-ca\sum_{i=0}^{n/a-2}i+\Theta(1) \\
	&= cn+cn(n/a-1)-\frac{ca(n/a-2)(n/a-1)}{2}+\Theta(1) \\
	&= \Theta(n^2).
\end{align*}

\exercise %4.2-5
Zauważmy, że drzewo rekurencji $T(n)$ z~rys.\ \ref{fig:4.2-5} dla parametru $\alpha$ jest symetryczne do tego samego drzewa z~parametrem $1-\alpha$ -- przyjmijmy więc, że $0<\alpha\le1/2$.
\begin{figure}[!ht]
	\centering \begin{tikzpicture}[
	level/.style = {level distance=10mm, sibling distance=95mm/2^#1},
	level 3/.style = {level distance=8mm, edge from parent/.style={halved path={solid}{very densely dashed}}},
	level 4/.style = {level distance=8mm, sibling distance=24mm, edge from parent/.style={halved path={draw=none}{very densely dashed}}},
	level 5/.style = {level distance=8mm, sibling distance=6mm, edge from parent/.style={halved path={solid}{very densely dashed}}},
	level arrow/.style = {arrow, thick, dashed},
	total/.style = {text width=8mm, align=right}
]

\node (root) {$cn$}
	child {node (0) {$c\alpha n$}
		child {node (00) {$c\alpha^2n$}
			child {node (000) {}}
			child {node {}}
		}
		child {node (01) {$c\alpha(1-\alpha)n$}
			child {node (010) {}}
			child {node (011) {}}
		}
	}
	child {node (1) {$c(1-\alpha)n$}
		child {node (10) {$c\alpha(1-\alpha)n$}
			child {node (100) {}}
			child {node (101) {}}
		}
		child {node (11) {$c(1-\alpha)^2n$}
			child {node (110) {}}
			child {node (111) {}
				child {node (i0) {$c\alpha(1-\alpha)^{i-1}n$}
					child {node (i00) {}}
					child {node (i01) {}}
				}
				child {node (i1) {$c(1-\alpha)^in$}
					child {node (i10) {}}
					child {node (i11) {}}
				}
			}
		}
	};

\node[below=3mm of 000] (leaf 0) {$T(1)$};
\node[below=8mm of 010] (leaf 1) {$T(1)$};
\node[below=13.5mm of 100] (leaf 2) {$T(1)$};
\path (i00) -- (i01) node[below=4mm, midway] (leaf 3) {$T(1)$};
\node[below=8mm of i11] (leaf 4) {$T(1)$};
\foreach \x in {0, ..., 4} {
	\draw[very densely dashed] (leaf \x.north) -- +(0mm, 2.5mm);
}

\node[right=65mm of root, total] (level 0 total) {$cn$};
\node[total] at (level 0 total |- 1) (level 1 total) {$cn$};
\node[total] at (level 0 total |- 11) (level 2 total) {$cn$};
\node[total] at (level 0 total |- i1) (level i total) {$\le cn$};
\node[total] at (level 0 total |- leaf 4) (last level total) {$\Theta(1)$};

\path (level 2 total) -- (level i total) node[midway] {$\vdots$};
\path (level i total) -- (last level total) node[midway] {$\vdots$};

\draw[level arrow] (root) -- (level 0 total);
\draw[level arrow] (1) -- (level 1 total);
\draw[level arrow] (11) -- (level 2 total);
\draw[level arrow] (i1) -- (level i total);
\draw[level arrow] (leaf 4) -- (last level total);

\node[below=2mm of last level total.south east, anchor=north east] (total) {Łącznie: $\Theta(n\lg n)$};
\draw[thick] (total.north west) -- (total.north east);

\coordinate (bottom) at (leaf 0.west |- leaf 4);
\coordinate[left=2mm of leaf 0] (bottom 1);
\coordinate[left=10mm of bottom] (bottom 2);
\draw[arrow, <->, thick] (bottom 1) -- (bottom 1 |- root) node[pos=0.5, fill=white] {$h$};
\draw[arrow, <->, thick] (bottom 2) -- (bottom 2 |- root) node[pos=0.5, fill=white] {$H$};

\end{tikzpicture}

	\caption{Drzewo rekursji dla równania rekurencyjnego $T(n)=T(\alpha n)+T((1-\alpha)n)+cn$, gdzie $0<\alpha\le1/2$.} \label{fig:4.2-5}
\end{figure}

Wyznaczmy najpierw oszacowanie dolne rekurencji.
Na \singledash{$i$}{tym} poziomie drzewa najmniejszy koszt wnoszą elementy o~wartościach $c\alpha^in$, a~więc węzły ze skrajnie lewej gałęzi.
Najgłębszym poziomem o~komplecie węzłów w~tym drzewie jest ten, na którym element ze skrajnie lewej gałęzi drzewa osiąga wartość stałą $d>0$.
Oznaczmy przez $h$ głębokość tego poziomu.
Wtedy $c\alpha^hn=d$, skąd otrzymujemy $h=\log_{1/\alpha}(cn/d)$.
Ponieważ $\alpha$ jest stałe, to $h=\Theta(\lg n)$.
Sumując koszty węzłów drzewa $T(n)$ znajdujące się na poziomach wyższych niż \singledash{$h$}{ty}, uzyskujemy oszacowanie dolne rekurencji, które wynosi $T(n)\ge cnh=\Omega(n\lg n)$.

Badając teraz skrajnie prawą gałąź, której elementy wnoszą największy koszt na każdym poziomie, możemy dojść do oszacowania górnego dla $T(n)$.
Na głębokości $H$ równej wysokości drzewa mamy $c(1-\alpha)^Hn=d$, skąd $H=\log_{1/(1-\alpha)}(cn/d)=\Theta(\lg n)$, a~więc $T(n)\le cn(H+1)=O(n\lg n)$.
Asymptotycznie dokładnym oszacowaniem rekurencji jest zatem $\Theta(n\lg n)$.
