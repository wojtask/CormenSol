\problem{Więcej przykładów rekurencji} %4-4

\subproblem %4-4(a)
Wykorzystując twierdzenie o~rekurencji uniwersalnej, mamy $n^{\log_ba}=n^{\lg3}$ oraz $f(n)=n\lg n=O(n^{\lg3-\epsilon})$, gdzie $\epsilon<\lg3-1$, a~stąd $T(n)=\Theta(n^{\lg3})$.

\subproblem %4-4(b)
Z~twierdzenia o~rekurencji uniwersalnej obliczamy $n^{\log_ba}=n^{\log_55}=n$, jednak dla żadnego $\epsilon>0$ nie jest prawdą, że
\[
	f(n) = \frac{n}{\lg n} = O(n^{1-\epsilon}),
\]
ponieważ dla pewnej stałej $c>0$ i~dostatecznie dużych $n$ musiałoby zachodzić
\[
	\frac{n^\epsilon}{\lg n} \le c,
\]
a~ponieważ $n^\epsilon=\omega(\lg n)$, to niezależnie od wyboru $\epsilon$ dla dużych wartości $n$ licznik tego ułamka jest dowolnie większy od mianownika i~ułamka nie da się z~tego powodu ograniczyć stałą.

Skorzystajmy zatem z~innego sposobu na obliczenie $T(n)$, rozważając rodzinę rekurencji postaci
\[
	T_a(n) = \begin{cases}
		\Theta(1), & \text{jeśli $1\le n<a$}, \\
		aT_a(n/a)+n/\!\lg n, & \text{jeśli $n\ge a$},
	\end{cases}
\]
gdzie $a\ge2$ jest liczbą całkowitą.
Wykorzystamy technikę zamiany zmiennych, podstawiając $m=\log_an$, skąd $n=a^m$, dzięki czemu otrzymujemy
\[
	T_a(a^m) = aT_a(a^{m-1})+\frac{a^m}{m\lg a}.
\]
Możemy teraz podstawić $S_a(m)=T_a(a^m)$, otrzymując nową rekurencję
\[
	S_a(m) = aS_a(m-1)+\frac{a^m}{m\lg a},
\]
którą rozwiązujemy następująco (po przyjęciu $S_a(0)=T_a(1)=d$ dla pewnej stałej $d>0$):
\begin{align*}
	S_a(m) &= \frac{1}{\lg a}\biggl(\frac{a^m}{m}+a\cdot\frac{a^{m-1}}{m-1}+a^2\cdot\frac{a^{m-2}}{m-2}+\dots+a^{m-1}\cdot\frac{a^1}{1}+a^md\biggr) \\[1mm]
	&= \frac{1}{\lg a}\biggl(\sum_{k=1}^m\frac{a^m}{k}+a^md\biggr) \\[1mm]
	&= \frac{a^m(H_m+d)}{\lg a} \\[1mm]
	&= \Theta(a^m\lg m).
\end{align*}
Zamieniając z~powrotem $S_a(m)$ na $T_a(n)$, otrzymujemy rozwiązanie
\[
	T_a(n) = T_a(a^m) = S_a(m) = \Theta(a^m\lg m) = \Theta(n\lg\log_a n) = \Theta(n\lg\lg n).
\]

Stosując znalezione oszacowanie do rekurencji $T(n)\equiv T_5(n)$ z~treści zadania, dostajemy oczywiście $T(n)=\Theta(n\lg\lg n)$.

\subproblem %4-4(c)
Z~twierdzenia o~rekurencji uniwersalnej mamy $n^{\log_ba}=n^{\lg4}=n^2$ oraz $f(n)=n^{5/2}=\Omega(n^{2+\epsilon})$, gdzie $\epsilon\le1/2$.
Badamy warunek regularności:
\begin{align*}
	4f(n/2) &\le cf(n), \\
	\frac{4n^{5/2}}{2^{5/2}} &\le cn^{5/2}, \\
	c &\ge \frac{1}{\sqrt{2}}
\end{align*}
i~stwierdzamy, że $T(n)=\Theta(n^{5/2})$.

\subproblem %4-4(d)
Wykażemy, że stała 5 w~argumencie rekurencji $T(n)$ nie wpływa na postać rozwiązania.
Rozważając rekurencję $T'(n)=3T'(n/3)+n/2$ i~rozwiązując ją za pomocą twierdzenia o~rekurencji uniwersalnej, dostajemy $T'(n)=\Theta(n\lg n)$.
Udowodnimy teraz metodą podstawiania, że identyczny wynik jest rozwiązaniem $T(n)$.

Wykorzystując założenie
\[
	T(n/3+5) \le c_1(n/3+5)\lg(n/3+5)
\]
dla pewnej stałej $c_1>0$, otrzymujemy:
\begin{align*}
	T(n) &\le 3c_1(n/3+5)\lg(n/3+5)+n/2 \\
	&\le c_1(n+15)\lg(2n/5)+n/2 \\
	&< c_1n\lg n+c_1n\lg(2/5)+15c_1\!\lg n+n/2 \\
	&\le c_1n\lg n.
\end{align*}
Powyższe wnioskowanie jest prawdziwe, o~ile $n/3+5\le2n/5$, skąd $n\ge75$, oraz
\[
	c_1n\lg(2/5)+15c_1\!\lg n+n/2 \le 0.
\]
Przeprowadzając podobną analizę jak w~\refExercise{4.1-5}, dostajemy, że dowolne $c_1\ge7$ spełnia ostatnią nierówność dla $n\ge75$.

Analogicznie dla dolnego oszacowania przyjmujemy, że
\[
	T(n/3+5) \ge c_2(n/3+5)\lg(n/3+5),
\]
gdzie $c_2>0$ jest pewną stałą.
Wówczas:
\begin{align*}
	T(n) &\ge 3c_2(n/3+5)\lg(n/3+5)+n/2 \\
	&> 3c_2(n/3)\lg(n/3)+n/2 \\
	&= c_2n\lg n-c_2n\lg3+n/2 \\
	&\ge c_2n\lg n,
\end{align*}
przy czym ostatnia nierówność zachodzi, o~ile
\[
	-c_2n\lg3+n/2 \ge 0.
\]
Warunek ten spełniamy poprzez przyjęcie $c_2\le0{,}3$.

Dowód asymptotycznie dokładnego oszacowania dla $T(n)$ kończymy, wybierając odpowiednie wartości dla podstaw obu indukcji.
Zauważmy, że obliczanie wartości rekurencji ze wzoru $T(n)=3T(n/3+5)+n/2$ dla $n\le7$ nie ma sensu, bo wtedy $T(n)$ nie zależy od niższych wyrazów.
Przyjmijmy zatem, że $T(n)=1$ dla wszystkich $n\le7$ będzie przypadkiem brzegowym rekurencji.
W~dowodzie górnego oszacowania założyliśmy, że $n\ge75$, więc podstawą obu indukcji możemy uczynić wszystkie $n=30$, 31, \dots, 74.
Można sprawdzić, że dla takich wartości oba oszacowania są spełnione, co uzasadnia poprawny dobór stałych $c_1$ i~$c_2$.
A~zatem $T(n)=\Theta(n\lg n)$.

\subproblem %4-4(e)
Mamy do czynienia z~rekurencją $T_a(n)$ dla $a=2$, którą rozważaliśmy w~punkcie (b).
Zgodnie z~przedstawionym tam rozumowaniem wnioskujemy, że $T(n)=\Theta(n\lg\lg n)$.

\subproblem %4-4(f)
W~celu rozwiązania rekurencji wykorzystamy metodę drzewa rekursji do odgadnięcia rozwiązania, które następnie udowodnimy.
Drzewo zostało przedstawione na rys.\ \ref{fig:4-4f}.
\begin{figure}[ht]
	\centering \begin{tikzpicture}[
	level/.style = {level distance=10mm, sibling distance=100mm/3^#1},
	level 3/.style = {edge from parent/.style={halved path={solid}{very densely dashed}}},
	level 4/.style = {sibling distance=8mm, edge from parent/.style={halved path={draw=none}{very densely dashed}}},
	level 5/.style = {level distance=8mm, sibling distance=3mm, edge from parent/.style={halved path={solid}{very densely dashed}}},
	level arrow/.style = {arrow, thick, dashed},
	total/.style = {text width=13mm, align=right}
]

\node (root) {$n$}
	child {node (0) {$\frac{n}{2}$}
		child {node (00) {$\frac{n}{4}$}
			child {node (000) {}
				child {node (i0) {$\frac{n}{2^i}$}
					child {node (i00) {}}
					child {node (i01) {}}
					child {node (i02) {}}
				}
				child {node (i1) {$\frac{n}{2^{i+1}}$}
					child {node (i10) {}}
					child {node (i11) {}}
					child {node (i12) {}}
				}
				child {node (i2) {$\frac{n}{2^{i+2}}$}
					child {node (i20) {}}
					child {node (i21) {}}
					child {node (i22) {}}
				}
			}
			child {node (001) {}}
			child {node (002) {}}
		}
		child {node (01) {$\frac{n}{8}$}
			child {node (010) {}}
			child {node (011) {}}
			child {node (012) {}}
		}
		child {node (02) {$\frac{n}{16}$}
			child {node (020) {}}
			child {node (021) {}}
			child {node (022) {}}
		}
	}
	child {node (1) {$\frac{n}{4}$}
		child {node (10) {$\frac{n}{8}$}
			child {node (100) {}}
			child {node (101) {}}
			child {node (102) {}}
		}
		child {node (11) {$\frac{n}{16}$}
			child {node (110) {}}
			child {node (111) {}}
			child {node (112) {}}
		}
		child {node (12) {$\frac{n}{32}$}
			child {node (120) {}}
			child {node (121) {}}
			child {node (122) {}}
		}
	}
	child {node (2) {$\frac{n}{8}$}
		child {node (20) {$\frac{n}{16}$}
			child {node (200) {}}
			child {node (201) {}}
			child {node (202) {}}
		}
		child {node (21) {$\frac{n}{32}$}
			child {node (210) {}}
			child {node (211) {}}
			child {node (212) {}}
		}
		child {node (22) {$\frac{n}{64}$}
			child {node (220) {}}
			child {node (221) {}}
			child {node (222) {}}
		}
	};

\node[below=3mm of 222] (leaf 0) {$T(1)$};
\node[below=13mm of 200] (leaf 1) {$T(1)$};
\node[below=18.5mm of 101] (leaf 2) {$T(1)$};
\node[below=4.5mm of i22] (leaf 3) {$T(1)$};
\node[below=8mm of i00] (leaf 4) {$T(1)$};
\foreach \x in {0, ..., 4} {
	\draw[very densely dashed] (leaf \x.north) -- +(0mm, 2.5mm);
}

\node[right=55mm of root, total] (level 0 total) {$n$};
\node[total] at (level 0 total |- 1) (level 1 total) {$\frac{7}{8}n$};
\node[total] at (level 0 total |- 11) (level 2 total) {$\bigl(\frac{7}{8}\bigr)^2n$};
\node[total] at (level 0 total |- i2) (level i total) {$\le\bigl(\frac{7}{8}\bigr)^in$};
\node[total] at (level 0 total |- leaf 4) (last level total) {$\Theta(1)$};

\path (level 2 total) -- (level i total) node[midway] {$\vdots$};
\path (level i total) -- (last level total) node[midway] {$\vdots$};

\draw[level arrow] (root) -- (level 0 total);
\draw[level arrow] (2) -- (level 1 total);
\draw[level arrow] (22) -- (level 2 total);
\draw[level arrow] (i2) -- (level i total);
\draw[level arrow] (leaf 4) -- (last level total);

\node[below=2mm of last level total.south east, anchor=north east] (total) {Łącznie: $\Theta(n)$};
\draw[thick] (total.north west) -- (total.north east);

\coordinate (bottom 1) at (leaf 4.west |- leaf 0);
\coordinate[left=6mm of leaf 4] (bottom 2);
\draw[arrow, <->, thick] (bottom 1) -- (bottom 1 |- root) node[midway, fill=white] {$\log_8n$};
\draw[arrow, <->, thick] (bottom 2) -- (bottom 2 |- root) node[pos=0.2, fill=white] {$\lg n$};

\end{tikzpicture}

	\caption{Drzewo rekursji dla równania rekurencyjnego $T(n)=T(n/2)+T(n/4)+T(n/8)+n$.} \label{fig:4-4f}
\end{figure}

Zauważmy, że najszybciej maleją argumenty na skrajnie prawej gałęzi.
Niech $T(1)=1$.
Ponieważ koszt węzła z~tej gałęzi znajdującego się na \singledash{$i$}{tym} poziomie wynosi $n/8^i$, to liść zajmuje poziom $h=\log_8n$.
Łącznym kosztem na \singledash{$i$}{tym} poziomie jest $(7/8)^in$, więc sumując te wartości od korzenia aż do poziomu \singledash{$h$}{tego}, otrzymujemy oszacowanie rekurencji od dołu:
\[
	T(n) \ge \sum_{i=0}^{\log_8n}\biggl(\frac{7}{8}\biggr)^in = \frac{1-(7/8)^{\log_8n+1}}{1-7/8}\,n = 8n(1-(7/8)n^{\log_8(7/8)}) \ge 8n(1-7/8) = \Omega(n).
\]

Zbadajmy teraz górne oszacowanie rekurencji $T(n)$ poprzez dokonanie obserwacji, że najwolniej malejącymi węzłami drzewa są te ze skrajnie lewej gałęzi.
Węzły te wnoszą koszt równy $n/2^i$ na \singledash{$i$}{tym} poziomie, więc liść znajduje się na poziomie $H=\lg n$.
Mamy
\[
	T(n) \le \sum_{i=0}^{\lg n}\biggl(\frac{7}{8}\biggr)^in < \sum_{i=0}^\infty\biggl(\frac{7}{8}\biggr)^in = \frac{n}{1-7/8} = O(n),
\]
co pozwala przypuszczać, że oszacowaniem dokładnym na $T(n)$ jest $\Theta(n)$.

Przeprowadźmy teraz dowód tego wyniku metodą podstawiania.
Załóżmy, że:
\begin{gather*}
	c_1(n/2) \le T(n/2) \le c_2(n/2), \\
	c_1(n/4) \le T(n/4) \le c_2(n/4), \\
	c_1(n/8) \le T(n/8) \le c_2(n/8),
\end{gather*}
gdzie $c_1$, $c_2>0$ to pewne stałe.
Mamy zatem
\[
	T(n) \ge c_1n/2+c_1n/4+c_1n/8+n = 7c_1n/8+n \ge c_1n,
\]
co jest spełnione dla dowolnego $n$, o~ile $c_1\le8$.
Dowód górnego oszacowania przebiega analogicznie, przy czym zakładamy, że $c_2\ge8$.

Jako warunek brzegowy rekurencji przyjmijmy $T(1)=T(2)=\dots=T(7)=1$.
Aby spełnić przypadek bazowy indukcji dla $n=1$, 2, \dots, 7, musimy przyjąć dodatkowo, że $c_1\le1/7$.
Udowodniliśmy zatem, że dokładnym rozwiązaniem rekurencji jest $T(n)=\Theta(n)$.

\subproblem %4-4(g)
Załóżmy dla uproszczenia rachunków, że $T(1)=1$.
Rozwijając rekurencję, otrzymujemy
\[
	T(n) = \frac{1}{n}+\frac{1}{n-1}+\dots+\frac{1}{2}+\frac{1}{1} = H_n,
\]
a~zatem, korzystając ze wzoru (A.7), mamy $T(n)=\Theta(\lg n)$.

\subproblem %4-4(h)
Dla ułatwienia przyjmijmy, że $T(1)=0$.
Wówczas
\[
	T(n) = \lg n+\lg(n-1)+\dots+\lg2+\lg1 = \lg\biggl(\prod_{i=1}^ni\biggr) = \lg(n!)
\]
i~ze wzoru (3.18) dostajemy $T(n)=\Theta(n\lg n)$.

\subproblem %4-4(i)
Przyjmijmy, że $T(2)=2$ i~rozważmy przypadek, gdy $n$ jest parzyste.
Rozwijamy rekurencję, otrzymując
\[
	T(n) = 2\lg n+2\lg(n-2)+\dots+2\lg4+2\lg2 = 2\lg\biggl(\prod_{i=1}^{n/2}2i\biggr) = 2\bigl(\lg((n/2)!)+n/2\bigr).
\]
Wykorzystując wzór (3.18), dostajemy
\[
	T(n) = 2\cdot\Theta((n/2)\lg (n/2))+n = \Theta(n\lg n).
\]

Dla $n$ nieparzystego, po przyjęciu $T(1)=0$ sprowadzamy rekurencję do sumy
\[
	T(n) = 2\lg n+2\lg(n-2)+\dots+2\lg3+2\lg1 = 2\lg\biggl(\prod_{i=1}^{(n+1)/2}(2i-1)\biggr),
\]
którą można ograniczyć z~góry przez $2\lg\Bigl(\prod_{i=1}^{(n+1)/2}2i\Bigr)$, a~z~dołu przez $2\lg\Bigl(\prod_{i=1}^{(n-1)/2}2i\Bigr)$.
Po pewnych przekształceniach i~po skorzystaniu ze wzoru (3.18) oba te wyrażenia sprowadzamy do postaci $\Theta(n\lg n)$.
Na podstawie tego faktu i~uzasadnienia z~poprzedniego paragrafu wnioskujemy, że $T(n)$ jest klasy $\Theta(n\lg n)$.

\subproblem %4-4(j)
Po podzieleniu rekurencji $T(n)$ przez $n$ dostajemy
\[
	\frac{T(n)}{n} = \frac{T(\!\sqrt{n})}{\sqrt{n}}+1.
\]
Podstawmy teraz $S(n)=T(n)/n$, otrzymując nową rekurencję
\[
	S(n) = S(\!\sqrt{n})+1.
\]
Potraktujmy $n$ jako $2^m$, skąd $m=\lg n$ i~podstawmy $R(m)=S(2^m)$:
\[
	R(m) = R(m/2)+1.
\]
Rozwiązanie ostatniej rekurencji zostało wyznaczone w~\refExercise{4.3-3} i~wynosi $R(m)=\Theta(\lg m)$.
Powracamy do oryginalnej rekurencji $T(n)$, dostając ostatecznie
\[
	T(n) = nS(n) = nR(\lg n) = n\cdot\Theta(\lg\lg n) = \Theta(n\lg\lg n).
\]
