\problem{Więcej przykładów rekurencji} %4-4

\subproblem %4-4(a)
Wykorzystując twierdzenie o~rekurencji uniwersalnej, mamy $n^{\log_ba}=n^{\lg3}$ oraz $f(n)=n\lg n=O(n^{\lg3-\epsilon})$, gdzie $\epsilon<\lg3-1$, a~stąd $T(n)=\Theta(n^{\lg3})$.

\subproblem %4-4(b)
Z~twierdzenia o~rekurencji uniwersalnej obliczamy $n^{\log_ba}=n^{\log_55}=n$, jednak dla żadnego $\epsilon>0$ nie jest prawdą, że
\[
	f(n) = \frac{n}{\lg n} = O(n^{1-\epsilon}),
\]
ponieważ dla pewnej stałej $c>0$ i~dostatecznie dużych $n$ musiałoby zachodzić
\[
	\frac{n^\epsilon}{\lg n} \le c,
\]
a~ponieważ $n^\epsilon=\omega(\lg n)$, to niezależnie od wyboru $\epsilon$ dla dużych wartości $n$ licznik tego ułamka jest dowolnie większy od mianownika i~ułamka nie da się z~tego powodu ograniczyć stałą.

Skorzystajmy zatem z~innego sposobu na obliczenie $T(n)$, rozważając rodzinę rekurencji postaci
\[
	T_a(n) = \begin{cases}
		\Theta(1), & \text{jeśli $1\le n<a$}, \\
		aT_a(n/a)+n/\!\lg n, & \text{jeśli $n\ge a$},
	\end{cases}
\]
gdzie $a\ge2$ jest liczbą całkowitą.
Wykorzystamy technikę zamiany zmiennych, podstawiając $m=\log_an$, skąd $n=a^m$, dzięki czemu otrzymujemy
\[
	T_a(a^m) = aT_a(a^{m-1})+\frac{a^m}{m\lg a}.
\]
Możemy teraz podstawić $S_a(m)=T_a(a^m)$, otrzymując nową rekurencję
\[
	S_a(m) = aS_a(m-1)+\frac{a^m}{m\lg a}.
\]

Przyjmiemy, że $S_a(0)=T_a(1)=d$ dla pewnej stałej $d>0$ i~udowodnimy, że dla $m\ge1$, $S_a(m)=a^m(H_m+d)/\lg a$, gdzie $H_m$ jest $m$\nbhyphen tą liczbą harmoniczną.
W~kroku bazowym:
\[
	S_a(1) = ad+\frac{a}{\lg a} = \frac{a^1(H_1+d)}{\lg a}.
\]
W~kroku indukcyjnym, gdy $m\ge2$, polegamy na założeniu $S_a(m-1)=a^{m-1}(H_{m-1}+d)/\lg a$:
\[
	S_a(m) = a\cdot\frac{a^{m-1}(H_{m-1}+d)}{\lg a}+\frac{a^m}{m\lg a} = \frac{a^m(H_{m-1}+d)+\frac{a^m}{m}}{\lg a} = \frac{a^m(H_m+d)}{\lg a}.
\]

Wzór (A.7) implikuje więc zależność $S_a(m)=\Theta(a^m\lg m)$.
Zamieniając z~powrotem $S_a(m)$ na $T_a(n)$, otrzymujemy rozwiązanie
\[
	T_a(n) = T_a(a^m) = S_a(m) = \Theta(a^m\lg m) = \Theta(n\lg\log_a n) = \Theta(n\lg\lg n).
\]

Stosując znalezione oszacowanie do rekurencji $T(n)\equiv T_5(n)$ z~treści zadania, dostajemy oczywiście $T(n)=\Theta(n\lg\lg n)$.

\subproblem %4-4(c)
Z~twierdzenia o~rekurencji uniwersalnej mamy $n^{\log_ba}=n^2$ i~$f(n)=n^{5/2}=\Omega(n^{2+\epsilon})$, gdzie $\epsilon\le1/2$.
Dla $c\ge2^{-1/2}$ mamy
\[
	4f(n/2) = \frac{4n^{5/2}}{2^{5/2}} \le cn^{5/2} = cf(n),
\]
zatem warunek regularności zachodzi i~$T(n)=\Theta(n^{5/2})$.

\subproblem %4-4(d)
Stosując metodę podstawiania, udowodnimy, że $T(n)=O(n\lg n)$, a~dokładniej, że istnieje $c>0$, że dla każdego $n\ge9$ zachodzi $T(n)\le c(n-15/2)\log_3(n-15/2)$.
W~tym celu wykorzystamy założenie indukcyjne
\[
	T(n/3+5) \le c(n/3+5-15/2)\log_3(n/3+5-15/2) = c(n/3-5/2)\log_3(n/3-5/2).
\]
Mamy następnie:
\begin{align*}
	T(n) &= 3T(n/3+5)+n/2 \\
	&\le 3c(n/3-5/2)\log_3(n/3-5/2)+n/2 \\
	&= c(n-15/2)\log_3(n-15/2)-c(n-15/2)+n/2 \\
	&\le c(n-15/2)\log_3(n-15/2),
\end{align*}
o~ile $-c(n-15/2)+n/2\le0$, czyli $c\ge n/(2n-15)$, co dla każdego $n\ge9$ jest spełnione, gdy $c\ge3$.

Przyjmiemy, że w~przypadku, gdy argument rekurencji jest niecałkowity, będziemy brać jego podłogę.
Założymy, że przypadkami brzegowymi rekurencji są $T(n)=1$ dla $n=1$, 2, \dots, 7.
Wówczas $T(8)=3T(7)+4=7$.
Krokiem bazowym indukcji możemy uczynić przypadki $n=9$, 10, 11, bo wyrazy następujące po nich nie zależą już od $T(8)$.
Wszystkie poniższe nierówności
\begin{align*}
	T(9) &= 3T(8)+\phantom{1}9/2 = 51/2 \le c(3/2)\log_3(3/2), \\
	T(10) &= 3T(8)+10/2 = 52/2 \le c(5/2)\log_3(5/2), \\
	T(11) &= 3T(8)+11/2 = 53/2 \le c(7/2)\log_3(7/2)
\end{align*}
spełnia np.\ $c=47$ i~dowód górnego oszacowania na $T(n)$ jest zakończony.

Aby uzyskać dolne oszacowanie, skorzystamy z~obserwacji, że funkcja $T(n)$ jest monotonicznie rosnąca, więc od dołu ogranicza ją rekurencja $T'(n)=3T'(n/3)+n/2$.
Wystarczy teraz powołać się na metodę rekurencji uniwersalnej dla $T'(n)$.
Według oznaczeń z~tw.\ 4.1 mamy $n^{\log_ba}=n$ oraz $f(n)=n/2=\Theta(n)$, zatem $T'(n)=\Theta(n\lg n)$, skąd $T(n)=\Omega(n\lg n)$.

\subproblem %4-4(e)
Mamy do czynienia z~rekurencją $T_a(n)$ dla $a=2$, którą rozważaliśmy w~punkcie (b).
Zgodnie z~przedstawionym tam rozumowaniem wnioskujemy, że $T(n)=\Theta(n\lg\lg n)$.

\subproblem %4-4(f)
Oczywiście wyrazy rekurencji są nieujemne, więc
\[
	T(n) = T(n/2)+T(n/4)+T(n/8)+n \ge n = \Omega(n).
\]

Udowodnimy teraz metodą podstawiania, że $T(n)\le cn$ dla pewnego $c>0$.
Załóżmy, że $T(k)\le ck$ dla każdego $1\le k<n$.
Wtedy
\[
	T(n) = T(n/2)+T(n/4)+T(n/8)+n \le cn/2+cn/4+cn/8+n = 7cn/8+n \le cn,
\]
co jest spełnione dla dowolnego $n$, o~ile $c\ge8$.
Jako warunek brzegowy rekurencji przyjmijmy $T(1)=T(2)=\dots=T(7)=1$.
Wówczas krokiem bazowym indukcji mogą być przypadki $n=1$, 2, \dots, 7 i~oszacowanie zachodzi w~każdym z~nich dla dowolnego $c\ge1$.

Na podstawie powyższych uzasadnień otrzymujemy $T(n)=\Theta(n)$.

\subproblem %4-4(g)
Załóżmy dla uproszczenia rachunków, że $T(1)=1$.
Pokażemy przez indukcję, że $T(n)=H_n$, gdzie $H_n$ to $n$\nbhyphen ta liczba harmoniczna.
Mamy $T(1)=1=H_1$, więc krok bazowy zachodzi.
Gdy $n>1$, przy założeniu, że $T(n-1)=H_{n-1}$:
\[
	T(n) = T(n-1)+\frac{1}{n} = H_{n-1}+\frac{1}{n} = H_n.
\]
Z~(A.7) dostajemy $T(n)=\Theta(\lg n)$.

\subproblem %4-4(h)
Dla ułatwienia przyjmijmy, że $T(1)=0$.
Pokażemy przez indukcję, że $T(n)=\lg(n!)$.
W~kroku bazowym mamy $T(1)=0=\lg(0!)$.
Dla $n>1$ i~z~założenia, że $T(n-1)=\lg((n-1)!)$:
\[
	T(n) = T(n-1)+\lg n = \lg((n-1)!)+\lg n = \lg\biggl(\prod_{i=1}^{n-1}i\biggr)+\lg n = \lg\biggl(\prod_{i=1}^ni\biggr) = \lg(n!).
\]
Z~(3.18) dostajemy $T(n)=\Theta(n\lg n)$.

\subproblem %4-4(i)
Przyjmiemy, że przypadkami brzegowymi rekurencji są $T(1)=0$ i~$T(2)=2$.
Gdy $n=2k-1$ dla pewnego $k\ge1$, to $T(2k-1)=T(2(k-1)-1)+2\lg(2k-1)$.
Podstawiając następnie $S_1(k)=T(2k-1)$, dostajemy nową rekurencję
\[
	S_1(k) = S_1(k-1)+2\lg(2k-1).
\]
Podobnie, gdy $n=2k$ dla pewnego $k\ge1$, to $T(2k)=T(2(k-1))+2\lg(2k)$, więc po podstawieniu $S_2(k)=T(2k)$, dostajemy rekurencję
\[
	S_2(k) = S_2(k-1)+2\lg(2k).
\]
Ponieważ $S_1(1)=0$, $S_2(1)=2$, a~$S_1(k)-S_1(k-1)\le S_2(k)-S_2(k-1)$, to dla każdego $k\ge1$ zachodzi $S_1(k)\le S_2(k)$.

Pokażemy przez indukcję, że $S_1(k)\ge\lg(k!)$.
Oczywiście dla $k=1$ nierówność zachodzi.
Niech teraz $k\ge2$ i~rozważmy założenie indukcyjne $S_1(k-1)\ge\lg((k-1)!)$.
Korzystając z~faktu, że dla $k\ge1$, $2\lg(2k-1)\ge\lg(k)$, mamy
\[
	S_1(k) = S_1(k-1)+2\lg(2k-1) \ge \lg((k-1)!)+\lg(k) = \lg(k!).
\]
Na podstawie obserwacji z~poprzedniego paragrafu, $\lg(k!)$ jest także dolnym oszacowaniem dla $S_2(k)$.

Teraz udowodnimy indukcyjnie, że $S_2(k)=2\lg(k!)+2k$, co będzie stanowić też górne oszacowanie dla $S_1(k)$.
Równość jest prawdziwa, gdy $k=1$, niech więc $k\ge2$.
Z~założenia indukcyjnego, że $S_2(k-1)=2\lg((k-1)!)+2(k-1)$, dostajemy
\[
	S_2(k) = S_2(k-1)+2\lg(2k) = 2\lg((k-1)!)+2(k-1)+2\lg k+2 = 2\lg(k!)+2k.
\]

Ze~wzoru (3.18) dostajemy, że oszacowaniem dokładnym zarówno $S_1(k)$, jak i~$S_2(k)$, jest $\Theta(k\lg k)$.
Powracając do $T(n)$, otrzymujemy, że niezależnie od parzystości $n$, $T(n)=\Theta(n\lg n)$.

\subproblem %4-4(j)
Po podzieleniu rekurencji $T(n)$ przez $n$ dostajemy
\[
	\frac{T(n)}{n} = \frac{T(\!\sqrt{n})}{\sqrt{n}}+1.
\]
Podstawmy teraz $S(n)=T(n)/n$, otrzymując nową rekurencję
\[
	S(n) = S(\!\sqrt{n})+1.
\]
Potraktujmy $n$ jako $2^m$, skąd $m=\lg n$ i~podstawmy $R(m)=S(2^m)$:
\[
	R(m) = R(m/2)+1.
\]
Rozwiązanie ostatniej rekurencji zostało wyznaczone w~\refExercise{4.3-3} i~wynosi $R(m)=\Theta(\lg m)$.
Powracamy do oryginalnej rekurencji $T(n)$, dostając ostatecznie
\[
	T(n) = nS(n) = nR(\lg n) = n\cdot\Theta(\lg\lg n) = \Theta(n\lg\lg n).
\]
