\problem{Testowanie układów VLSI} %4-6

\subproblem %4-6(a)
Niech $D$ i~$Z$ będą zbiorami, odpowiednio, układów dobrych i~układów złych.
Podczas testowania każdy zły układ może twierdzić, że każdy inny układ ze zbioru $Z$ jest dobry, a~każdy układ ze zbioru $D$ jest zły.
Z~kolei dobry układ może orzekać o~każdym układzie z~$Z$, że jest zły, natomiast o~każdym innym z~$D$, że jest dobry.
Inaczej ujmując, zbiór $Z$ będzie wskazywał, że sam jest zbiorem układów dobrych, a~zbiór $D$ złych i~symetrycznie dla zbioru $D$.
W~ogólności $Z$ może składać się z~podzbiorów układów, które będą twierdzić, że wyłącznie one są dobre.
Nie da się natomiast rozdzielić w~taki sposób zbioru $D$, ponieważ jego układy zawsze orzekają prawdziwie.
Aby zatem jednoznacznie wskazać zbiór dobrych układów, wymagane jest założenie, że $|D|>|Z|$.

\subproblem %4-6(b)
Potraktujmy wynik każdego testu dwóch układów jako parę $\langle a,b\rangle\in\{{\scriptstyle\rm D},{\scriptstyle\rm Z}\}^2$.
Pierwszy element pary jest stwierdzeniem pierwszego układu o~drugim, a~drugi element -- drugiego o~pierwszym, przy czym ${\scriptstyle\rm D}$ oznacza pozytywny wynik testu, a~${\scriptstyle\rm Z}$ -- negatywny.
Możliwe jest uzyskanie jednego z~czterech wyników:
\begin{itemize}
	\item $\langle{\scriptstyle\rm D},{\scriptstyle\rm D}\rangle$ -- oba układy są dobre albo oba są złe;
	\item $\langle{\scriptstyle\rm D},{\scriptstyle\rm Z}\rangle$ -- tylko pierwszy z~układów jest zły;
	\item $\langle{\scriptstyle\rm Z},{\scriptstyle\rm D}\rangle$ -- tylko drugi z~układów jest zły;
	\item $\langle{\scriptstyle\rm Z},{\scriptstyle\rm Z}\rangle$ -- co najmniej jeden z~układów jest zły.
\end{itemize}

Podzielmy zbiór układów na pary i~przetestujmy wzajemnie układy w~każdej takiej parze, wykonując przy tym $\lfloor n/2\rfloor$ testów.
Zauważmy, że otrzymawszy dla pewnej pary wynik inny niż $\langle{\scriptstyle\rm D},{\scriptstyle\rm D}\rangle$, możemy ją odrzucić, gdyż co najmniej jeden układ z~tej pary jest zły i~wśród nieodrzuconych układów będzie nadal więcej dobrych niż złych.
Dostaniemy w~rezultacie od jednej do $\lfloor n/2\rfloor$ par o~tej własności, że w~każdej z~nich oba układy są dobre albo oba są złe.
Odrzucając zatem po jednym układzie z~każdej pozostawionej pary, nadal zachowujemy własność o~większej liczbie dobrych układów w~pozostawionym zbiorze układów, którego rozmiar wynosi co najwyżej $\lceil n/2\rceil$.

Powyżej opisany proces przeprowadzamy rekurencyjnie, dostając w~końcu zbiór jednoelementowy, który na mocy założenia zawiera dobry układ.

\subproblem %4-6(c)
Wykorzystując wynik poprzedniego punktu, dostajemy następującą rekurencję opisującą liczbę testów koniecznych do znalezienia jednego dobrego układu w~pesymistycznym przypadku (tzn.\ kiedy każda para układów zwraca wynik $\langle{\scriptstyle\rm D},{\scriptstyle\rm D}\rangle$):
\[
	T(n) =
	\begin{cases}
		\Theta(1), & \text{jeśli $n=1$}, \\
		T(\lceil n/2\rceil)+\lfloor n/2\rfloor, & \text{jeśli $n>1$}.
	\end{cases}
\]
Stosując twierdzenie o~rekurencji uniwersalnej, dostajemy rozwiązanie: $T(n)=\Theta(n)$.
Wynik ten jest prawdziwy także w~przypadku optymistycznym -- wówczas dobry układ znajdujemy w~jednym zejściu rekurencyjnym po uprzednim wykonaniu $\Theta(n)$ testów.

Potrafimy wyznaczyć dobry układ $u$ -- wykorzystajmy go więc do znalezienia kolejnych.
Testujemy tenże układ z~pozostałymi $n-1$.
Wynikiem testu $u$ z~pewnym innym układem $v$ nie może być $\langle{\scriptstyle\rm D},{\scriptstyle\rm Z}\rangle$, a~więc możliwe są trzy sytuacje.
Jeśli otrzymamy $\langle{\scriptstyle\rm D},{\scriptstyle\rm D}\rangle$, to oznacza to, że oba układy są tak samo dobre, a~więc $v$ również jest dobry.
Uzyskując wynik $\langle{\scriptstyle\rm Z},{\scriptstyle\rm D}\rangle$, mamy natychmiast, że $v$ jest zły, podobnie w~wypadku, gdy wynikiem testu będzie $\langle{\scriptstyle\rm Z},{\scriptstyle\rm Z}\rangle$ -- wtedy co najmniej jeden z~testowanych układów jest zły, ale nie może nim być $u$.
Wynika stąd, że po wykonaniu $n-1$ takich testów wyznaczymy pozostałe dobre układy.
A~zatem łączna liczba testów wymaganych do zidentyfikowania wszystkich dobrych układów wynosi $\Theta(n)$.
