\problem{Koszty przekazywania parametrów} %4-3

\subproblem %4-3(a)
W~pierwszej strategii rekurencja przyjmuje postać z~\refExercise{2.3-5}, której rozwiązaniem jest $T(n)=\Theta(\lg n)$, skąd $T(N)=\Theta(\lg N)$.

W~przypadku drugiej strategii na każdym poziomie rekursji należy dodać składnik $\Theta(N)$ odpowiedzialny za przekazywanie tablicy do wywołań rekurencyjnych.
Otrzymujemy zatem
\[
	T(n) = \begin{cases}
		\Theta(1), & \text{jeśli $n\le1$}, \\
		T(\lfloor n/2\rfloor)+\Theta(N), & \text{jeśli $n>1$}.
	\end{cases}
\]
Ponieważ $\Theta(N)$ nie zależy od rozmiaru podproblemu $n$, to stąd rozwiązaniem powyższej rekurencji jest $T(n)=\Theta(N\lg n)$, a~więc $T(N)=\Theta(N\lg N)$.

W~ostatnim przypadku przekazujemy podtablicę o~rozmiarze równym rozmiarowi podproblemu, co prowadzi do rekurencji
\[
	T(n) = \begin{cases}
		\Theta(1), & \text{jeśli $n\le1$}, \\
		T(\lfloor n/2\rfloor)+\Theta(\lfloor n/2\rfloor), & \text{jeśli $n>1$},
	\end{cases}
\]
którą można rozwiązać przy użyciu twierdzenia o~rekurencji uniwersalnej.
Otrzymujemy wynik $T(n)=\Theta(n)$, skąd $T(N)=\Theta(N)$.

\subproblem %4-3(b)
W~przypadku zwykłego przekazywania wskaźnika mamy oryginalną postać rekurencji, której rozwiązaniem dla $n=N$ jest $T(N)=\Theta(N\lg N)$.

Ponieważ w~drugiej strategii do każdego z~obu wywołań rekurencyjnych przekazywana jest cała tablica, to czas działania wygląda następująco:
\[
	T(n) = \begin{cases}
		\Theta(1), & \text{jeśli $n=1$}, \\
		2T(\lfloor n/2\rfloor)+\Theta(n)+2\Theta(N), & \text{jeśli $n>1$}.
	\end{cases}
\]
Rozwiązaniem tej rekurencji jest suma rozwiązania rekurencji z~pierwszej strategii oraz iloczynu składnika $2\Theta(N)$ i~łącznej liczby wywołań rekurencyjnych, których jest
\[
	2^1+2^2+\dots+2^{\lfloor\lg n\rfloor} < \frac{2^{\lg n+1}}{2-1} = 2n.
\]
Mamy zatem $T(n)=\Theta(n\lg n)+\Theta(Nn)$, a~stąd $T(N)=\Theta(N^2)$.

W~ostatniej strategii do każdego wywołania rekurencyjnego przekazujemy podtablicę o~rozmiarze podproblemu, jednak czas na to poświęcany jest identyczny ze składnikiem liniowym odpowiadającym za czas przeznaczany na procedurę \proc{Merge}.
Rekurencja ma zatem identyczną postać jak w~oryginalnej analizie tego algorytmu i~jej rozwiązaniem jest $T(N)=\Theta(N\lg N)$.
