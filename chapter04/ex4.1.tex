\subchapter{Metoda podstawiania}

\exercise %4.1-1
Niech $c>0$ będzie stałą.
Przyjmujemy założenie, że
\[
    T(\lceil n/2\rceil)\le c\lg\lceil n/2\rceil
\]
i~że $n\ge4$.
Na podstawie założeń i~wzoru (3.3) otrzymujemy:
\begin{align*}
	T(n) &\le c\lg\lceil n/2\rceil+1 \\
	&< c\lg(n/2+1)+1 \\
	&\le c\lg(3n/4)+1 \\
	&= c\lg n+c\lg(3/4)+1 \\
	&\le c\lg n,
\end{align*}
przy czym ostatnia nierówność jest spełniona, gdy $c\ge\log_{4/3}2$.

Niech $T(1)=1$.
Wówczas $T(2)=2$, $T(3)=3$.
Nierówności $T(2)\le c\lg2$ oraz $T(3)\le c\lg3$ zachodzą dla $c\ge2$, a~więc w~szczególności dla $c\ge\log_{4/3}2$.
Można zatem przyjąć $n=2$ i~$n=3$ za podstawę indukcji, gdyż dla $n\ge4$ rekurencja nie zależy bezpośrednio od $T(1)$.
Na mocy dowodu indukcyjnego wynika zatem, że $T(n)=O(\lg n)$.

\exercise %4.1-2
Przyjmijmy założenie, że
\[
	T(\lfloor n/2\rfloor) \ge c\lfloor n/2\rfloor\lg\lfloor n/2\rfloor
\]
dla pewnej dodatniej stałej $c$.
Korzystając ze wzoru (3.3), mamy:
\begin{align*}
	T(n) &\ge 2c\lfloor n/2\rfloor\lg\lfloor n/2\rfloor+n \\
	&> 2c(n/2-1)\lg(n/4)+n \\
	&= 2c((n/2)\lg n-n-\lg n+2)+n \\
	&= cn\lg n-2cn-2c\lg n+4c+n \\
	&> cn\lg n-2c(n+\lg n)+n \\
	&\ge cn\lg n.
\end{align*}
Ostatni krok zachodzi, o~ile $-2c(n+\lg n)+n\ge0$, czyli
\[
	c \le \frac{n}{2(n+\lg n)} \le \frac{n}{2n} = \frac{1}{2}.
\]
Wybierając dowolne $0<c\le1/2$, spełniamy ostatnią nierówność wyprowadzenia.
Można przyjąć $T(1)=1$ i~$n=1$ za podstawę indukcji, bo wówczas $T(1)\ge c\cdot1\cdot\lg1=0$.
Wykazaliśmy, że $T(n)=\Omega(n\lg n)$, więc na mocy tw.\ 3.1 mamy $T(n)=\Theta(n\lg n)$.

\exercise %4.1-3
Udowodnimy, że $T(n)\le cn\lg n+1$ dla pewnej stałej $c>0$.
W~kroku bazowym dla $n=1$ mamy $T(1)=1\le c\cdot1\cdot\lg1+1$.
Dla $n\ge2$ przyjmijmy założenie indukcyjne
\[
    T(\lfloor n/2\rfloor) \le c\lfloor n/2\rfloor\lg\lfloor n/2\rfloor+1.
\]
Mamy teraz:
\begin{align*}
    T(n) &\le 2(c\lfloor n/2\rfloor\lg\lfloor n/2\rfloor+1)+n \\
    &= 2c\lfloor n/2\rfloor\lg\lfloor n/2\rfloor+2+n \\
	&\le cn\lg(n/2)+2+n \\
	&= cn\lg n-cn+2+n \\
	&\le cn\lg n-cn+n+n \\
	&\le cn\lg n,
\end{align*}
przy czym ostatnia nierówność zachodzi, o~ile $c\ge2$.

\exercise %4.1-4
Rekurencję (4.2) można przedstawić w~alternatywny sposób:
\[
	T(n) =
	\begin{cases}
		a, & \text{jeśli $n=1$}, \\
		T(\lceil n/2\rceil)+T(\lfloor n/2\rfloor)+bn, & \text{jeśli $n>1$},
	\end{cases}
\]
dla stałych $a$, $b>0$.

Niech $n\ge4$.
Udowodnimy oszacowania górne i~dolne dla $T(n)$ przy użyciu indukcji i~korzystając z~obserwacji, że $T(n)$ jest funkcją niemalejącą.
Dla stałych $c_1$, $c_2>0$ przyjmijmy założenia
\[
	T(\lfloor n/2\rfloor) \ge c_1\lfloor n/2\rfloor\lg\lfloor n/2\rfloor \quad\text{oraz}\quad T(\lceil n/2\rceil) \le c_2\lceil n/2\rceil\lg\lceil n/2\rceil.
\]
Dla oszacowania dolnego mamy:
\begin{align*}
	T(n) &= T(\lceil n/2\rceil)+T(\lfloor n/2\rfloor)+bn \\
	&\ge T(\lfloor n/2\rfloor)+T(\lfloor n/2\rfloor)+bn \\
	&\ge 2c_1\lfloor n/2\rfloor\lg\lfloor n/2\rfloor+bn \\
	&> 2c_1(n/2-1)\lg(n/4)+bn \\
	&= c_1n\lg n-2c_1n-2c_1\!\lg(n/4)+bn \\
	&\ge c_1n\lg n,
\end{align*}
ponieważ zachodzi $\lfloor n/2\rfloor>n/2-1$ i~$\lfloor n/2\rfloor\ge n/4$ oraz można tak dobrać stałą $c_1$, aby prawdziwa była ostatnia nierówność powyższego wyprowadzenia.
Po podstawieniu $c_1=b/4$ sprowadza się ona do postaci $n\ge\lg(n/4)$, co jest prawdą dla wszystkich $n$ dodatnich.
Podstawę indukcji stanowi $n=1$, gdyż $T(1)=a\ge c_1\cdot 1\cdot\lg1=0$.
Stąd $T(n)=\Omega(n\lg n)$.

Wykorzystując nierówność $\lceil n/2\rceil<n/2+1$, dowodzimy górnego oszacowania:
\begin{align*}
	T(n) &= T(\lceil n/2\rceil)+T(\lfloor n/2\rfloor)+bn \\
	&\le T(\lceil n/2\rceil)+T(\lceil n/2\rceil)+bn \\
	&\le 2c_2\lceil n/2\rceil\lg\lceil n/2\rceil+bn \\
	&< 2c_2(n/2+1)\lg\frac{n\lceil n/2\rceil}{n}+bn \\
	&= c_2(n+2)\biggl(\lg n+\lg\frac{\lceil n/2\rceil}{n}\biggr)+bn \\
	&\le c_2n\lg n+2c_2\lg n+c_2(n+2)\lg\frac{\lceil n/2\rceil}{n}+bn \\
	&\le c_2n\lg n.
\end{align*}
W~ostatniej nierówności skorzystano z~faktu, że dla $n\ge2$, $\lceil n/2\rceil<n$, więc $\lg\frac{\lceil n/2\rceil}{n}<0$, a~zatem, dobierając odpowiednio duże $c_2$, można uzasadnić nierówność, gdyż funkcja liniowa rośnie szybciej od logarytmicznej.
Dokładniej, okazuje się, że przyjęcie $c_2\ge10b$ wystarcza, aby spełnić nierówność dla $n\ge4$.

Ponieważ dla $n\ge4$ rekurencja nie zależy bezpośrednio od $T(1)$, to za podstawę indukcji można przyjąć $n=2$ i~$n=3$.
Zachodzi $T(2)=2a+2b$, $T(3)=3a+5b$.
Łatwo sprawdzić, że wartości te spełniają otrzymane oszacowanie, o~ile $a$ jest dostatecznie małe.
W~przeciwnym przypadku wybranie większej stałej $c_2$ pozwala na dowolne ograniczenie rekurencji od góry w~zależności od stałych $a$ i~$b$.
Niezależnie od ich doboru oszacowaniem górnym rekurencji $T(n)$ jest $O(n\lg n)$, co na mocy wcześniejszego wyniku dolnego oszacowania implikuje $T(n)=\Theta(n\lg n)$.

\exercise %4.1-5
Wykorzystując założenie
\[
	T(\lfloor n/2\rfloor+17) \le c(\lfloor n/2\rfloor+17)\lg(\lfloor n/2\rfloor+17)
\]
dla pewnej stałej $c>0$, otrzymujemy:
\begin{align*}
	T(n) &\le 2c(\lfloor n/2\rfloor+17)\lg(\lfloor n/2\rfloor+17)+n \\
	&\le 2c(n/2+17)\lg(n/2+17)+n \\
	&\le c(n+34)\lg(11n/20)+n \\
	&< cn\lg n+cn\lg(11/20)+34c\lg n+n \\
	&\le cn\lg n.
\end{align*}
Nierówności zachodzą, o~ile $n/2+17\le 11n/20$, czyli $n\ge340$, oraz
\[
	cn\lg(11/20)+34c\lg n+n \le 0.
\]
Badając ostatnią nierówność, można dojść do rezultatu, że przyjęcie $c\ge47$ wystarcza, aby spełnić ją dla wszystkich $n\ge n_0$, gdzie $n_0=340$.

Zauważmy, że stosowanie wzoru $T(n)=2T(\lfloor n/2\rfloor+17)+n$ dla $n\le34$ nie ma sensu, bo wtedy $T(n)$ nie zależy od niższych wyrazów.
Przyjmijmy zatem, że $T(n)=1$ dla wszystkich $n\le34$ i~niech stanowi to przypadek brzegowy rekurencji.
Za podstawę indukcji musimy wtedy przyjąć wszystkie $n=187$, 188, \dots, 339.
Można pokazać, że dla $c\ge47$ wszystkie te wartości spełniają oszacowanie, a~zatem $T(n)=O(n\lg n)$.

Analiza tej rekurencji z~każdą inną stałą w~miejscu 17 przebiega analogicznie, zmianie ulegają natomiast wartości $n_0$ i~$c$ oraz wartości brzegowe rekursji i~podstawa indukcji, jednak w~każdym takim przypadku rekurencja jest klasy $O(n\lg n)$.

\exercise %4.1-6
Przyjmijmy, że $n=2^m$, skąd $m=\lg n$.
Rekurencja przyjmuje teraz postać
\[
	T(2^m) = 2T(2^{m/2})+1.
\]
Z~kolei podstawiając $S(m)$ za $T(2^m)$, dostajemy nową rekurencję
\[
	S(m) = 2S(m/2)+1,
\]
dla której udowodnimy rozwiązanie $\Theta(\lg m)$.

Ponieważ całkowitość argumentów nie jest dla nas istotna, to możemy rekurencję potraktować jako $S(m)=S(\lfloor m/2\rfloor)+S(\lceil m/2\rceil)+1$, co, według Podręcznika, jest $O(m)$.

Aby uzyskać oszacowanie dokładne, pozostaje udowodnić, że $S(m)=\Omega(m)$.
Przyjmujemy zatem założenie, że $S(m/2)\ge cm/2$ dla $c>0$ i~na jego podstawie otrzymujemy:
\[
	S(m) \ge 2cm/2+1 = cm+1 > cm,
\]
co zachodzi dla dowolnej wartości $c$.
Niech $S(1)=1$.
Przyjęcie $m=1$ na podstawę indukcji wystarcza, bo warunek $S(1)\ge c$ spełnia każde $c\le1$, a~zatem $S(m)=\Omega(m)$.

Stosując tw.\ 3.1, mamy $S(m)=\Theta(m)$.
Wracając teraz do oryginalnej rekurencji i~starej zmiennej, dostajemy $T(n)=T(2^m)=S(m)=\Theta(m)=\Theta(\lg n)$.
