\problem{Przykłady rekurencji} %4-1
W~punktach (a)\nbendash(f) skorzystano z~twierdzenia o~rekurencji uniwersalnej.

\subproblem %4-1(a)
Mamy $n^{\log_ba}=n^{\log_22}=n$ oraz $f(n)=n^3=\Omega(n^{1+\epsilon})$ dla $\epsilon\le2$.
Gdy $c\ge1/4$, to
\[
	2f(n/2) = 2n^3\!/8 \le cn^3 = cf(n),
\]
więc warunek regularności jest spełniony i~$T(n)=\Theta(n^3)$.

\subproblem %4-1(b)
Mamy $n^{\log_ba}=n^{\log_{10/9}1}=n^0=1$ oraz $f(n)=n=\Omega(n^\epsilon)$ dla $\epsilon\le1$.
Gdy $c\ge9/10$, to
\[
	f(9n/10) = 9n/10 \le cn = cf(n),
\]
więc warunek regularności jest spełniony i~$T(n)=\Theta(n)$.

\subproblem %4-1(c)
Mamy $n^{\log_ba}=n^{\log_416}=n^2$ oraz $f(n)=n^2=\Theta(n^2)$, a~stąd $T(n)=\Theta(n^2\lg n)$.

\subproblem %4-1(d)
Mamy $n^{\log_ba}=n^{\log_37}$ oraz $f(n)=n^2=\Omega(n^{\log_37+\epsilon})$ dla $\epsilon\le2-\log_37$.
Gdy $c\ge7/9$, to
\[
	7f(n/3) = 7n^2\!/9 \le cn^2 = cf(n),
\]
więc warunek regularności jest spełniony i~$T(n)=\Theta(n^2)$.

\subproblem %4-1(e)
Mamy $n^{\log_ba}=n^{\lg7}$ oraz $f(n)=n^2=O(n^{\lg7-\epsilon})$ dla $\epsilon\le\lg7-2$, a~stąd $T(n)=\Theta(n^{\lg7})$.

\subproblem %4-1(f)
Mamy $n^{\log_ba}=n^{\log_42}=n^{1/2}$ oraz $f(n)=\sqrt{n}=\Theta(n^{1/2})$, a~stąd $T(n)=\Theta\bigl(\!\sqrt{n}\lg n\bigr)$.

\subproblem %4-1(g)
Dla uproszczenia rachunków ustalmy przypadki brzegowe $T(1)=1$ i~$T(2)=3$.
Przyjęcie innych wartości nie wpływa na postać oszacowania rekurencji.

Udowodnimy przez indukcję, że $T(n)=n(n+1)/2$.
Równość oczywiście zachodzi dla $n=1$ i~$n=2$, niech więc $n>2$ i~przyjmijmy założenie, że $T(n-1)=n(n-1)/2$.
Mamy
\[
	T(n) = T(n-1)+n = \frac{n(n-1)}{2}+n = \frac{n(n+1)}{2}.
\]
Stąd $T(n)=\Theta(n^2)$.

\subproblem %4-1(h)
Niech $n=2^m$, skąd $m=\lg n$.
Rekurencja przyjmuje postać
\[
	T(2^m) = T(2^{m/2})+1.
\]
Podstawiając $S(m)$ za $T(2^m)$, otrzymujemy
\[
	S(m) = S(m/2)+1.
\]
Ponieważ rozwiązaniem ostatniej rekurencji jest $\Theta(\lg m)$ (co pokazano w~\refExercise{4.3-3}), to stąd mamy, że $T(n)=T(2^m)=S(m)=\Theta(\lg m)=\Theta(\lg\lg n)$.
