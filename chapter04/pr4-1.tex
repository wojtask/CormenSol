\problem{Przykłady rekurencji} %4-1
W~punktach (a)\nbendash(f) skorzystano z~twierdzenia o~rekurencji uniwersalnej.

\subproblem %4-1(a)
Mamy $n^{\log_ba}=n^{\log_22}=n$ oraz $f(n)=n^3=\Omega(n^{1+\epsilon})$ dla $\epsilon\le2$.
Ponieważ warunek regularności jest spełniony:
\begin{align*}
	2f(n/2) &\le cf(n), \\
	2n^3\!/8 &\le cn^3, \\
	c &\ge 1/4,
\end{align*}
to stąd wnioskujemy, że $T(n)=\Theta(n^3)$.

\subproblem %4-1(b)
Mamy $n^{\log_ba}=n^{\log_{10/9}1}=n^0=1$ oraz $f(n)=n=\Omega(n^\epsilon)$ dla $\epsilon\le1$.
Badamy warunek regularności:
\begin{align*}
	f(9n/10) &\le cf(n), \\
	9n/10 &\le cn, \\
	c &\ge 9/10
\end{align*}
i~stwierdzamy, że $T(n)=\Theta(n)$.

\subproblem %4-1(c)
Mamy $n^{\log_ba}=n^{\log_416}=n^2$ oraz $f(n)=n^2=\Theta(n^2)$, a~stąd $T(n)=\Theta(n^2\lg n)$.

\subproblem %4-1(d)
Mamy $n^{\log_ba}=n^{\log_37}$ oraz $f(n)=n^2=\Omega(n^{\log_37+\epsilon})$ dla $\epsilon\le2-\log_37$.
Warunek regularności zachodzi:
\begin{align*}
	7f(n/3) &\le cf(n), \\
	7n^2\!/9 &\le cn^2, \\
	c &\ge 7/9,
\end{align*}
a~zatem $T(n)=\Theta(n^2)$.

\subproblem %4-1(e)
Mamy $n^{\log_ba}=n^{\lg7}$ oraz $f(n)=n^2=O(n^{\lg7-\epsilon})$ dla $\epsilon\le\lg7-2$, a~stąd $T(n)=\Theta(n^{\lg7})$.

\subproblem %4-1(f)
Mamy $n^{\log_ba}=n^{\log_42}=n^{1/2}$ oraz $f(n)=\sqrt{n}=\Theta(n^{1/2})$, a~stąd $T(n)=\Theta\bigl(\!\sqrt{n}\lg n\bigr)$.

\subproblem %4-1(g)
Zauważmy, że rekurencja rozwija się następująco:
\begin{align*}
	T(n) &= T(n-1)+n \\
	&= T(n-2)+(n-1)+n \\
	&\hspace{.5in}\vdots \\
	&= c+3+4+\dots+n \\
	&= c+\sum_{i=3}^ni \\
	&= c+\frac{n(n+1)}{2}-3,
\end{align*}
gdzie $c=T(2)$ jest pewną stałą.
Otrzymujemy zatem, że $T(n)=\Theta(n^2)$.

\subproblem %4-1(h)
Niech $n=2^m$, skąd $m=\lg n$.
Rekurencja przyjmuje postać
\[
	T(2^m) = T(2^{m/2})+1.
\]
Podstawiając $S(m)$ za $T(2^m)$, otrzymujemy
\[
	S(m) = S(m/2)+1.
\]
Ponieważ rozwiązaniem ostatniej rekurencji jest $\Theta(\lg m)$ (co pokazano w~\refExercise{4.3-3}), to stąd mamy, że $T(n)=T(2^m)=S(m)=\Theta(\lg m)=\Theta(\lg\lg n)$.
