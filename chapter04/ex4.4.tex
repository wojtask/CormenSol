\subchapter{Dowód twierdzenia o~rekurencji uniwersalnej}

\exercise %4.4-1
Wykażemy metodą indukcji, że $n_j=\bigl\lceil n/b^j\bigr\rceil$.
Z~definicji (4.12) bezpośrednio wynika prawdziwość tego wzoru dla $j=0$.
Przyjmijmy zatem, że dla $j>0$ zachodzi $n_{j-1}=\bigl\lceil n/b^{j-1}\bigr\rceil$.
Wykorzystując tożsamość (3.4), otrzymujemy
\[
	n_j = \lceil n_{j-1}/b\rceil = \bigl\lceil\bigl\lceil n/b^{j-1}\bigr\rceil/b\bigr\rceil = \bigl\lceil n/b^j\bigr\rceil,
\]
co należało pokazać.

\exercise %4.4-2
Zastąpmy przypadek 2 tw.\ 4.1 ogólniejszym warunkiem, tzn.\ jeśli $f(n)=\Theta(n^{\log_ba}\lg^kn)$ dla $k\ge0$, to zachodzi $T(n)=\Theta(n^{\log_ba}\lg^{k+1}n)$.
Dla tak zmodyfikowanego twierdzenia przeprowadzimy dowody analogicznie zmodyfikowanego przypadku 2 w~lemacie 4.3 i~w~lemacie 4.4, skąd bezpośrednio wyniknie teza twierdzenia.

\begin{proof}[Dowód zmodyfikowanego przypadku 2 z~lematu 4.3]
	Skoro $f(n)=\Theta(n^{\log_ba}\lg^kn)$, to dostajemy oszacowanie
	\[
		f(n/b^j)=\Theta\bigl((n/b^j)^{\log_ba}\lg^k(n/b^j)\bigr)
	\]
	i~podstawiamy je do wzoru (4.7):
	\[
		g(n) = \Theta\biggl(\sum_{j=0}^{\log_bn-1}a^j\biggl(\frac{n}{b^j}\biggr)^{\log_ba}\lg^k\frac{n}{b^j}\biggr).
	\]
	Mamy dalej:
	\begin{align*}
		\sum_{j=0}^{\log_bn-1}a^j\biggl(\frac{n}{b^j}\biggr)^{\log_ba}\lg^k\frac{n}{b^j} &= n^{\log_ba}\sum_{j=0}^{\log_bn-1}\biggl(\frac{a}{b^{\log_ba}}\biggr)^j\lg^k\frac{n}{b^j} \\
		&= n^{\log_ba}\sum_{j=0}^{\log_bn-1}(\lg n-j\lg b)^k \\
		&= n^{\log_ba}\sum_{j=0}^{\log_bn-1}\Theta(\lg^kn) \\
		&= n^{\log_ba}\cdot\log_bn\cdot\Theta(\lg^kn) \\[1mm]
		&= \Theta(n^{\log_ba}\lg^{k+1}n).
	\end{align*}
	W~wyprowadzeniu skorzystaliśmy ze wzoru (3.2), podstawiając $\lg n$ w~miejsce $n$.
	Zachodzi więc $g(n)=\Theta(n^{\log_ba}\lg^{k+1}n)$, skąd wynika teza.
\end{proof}

\begin{proof}[Dowód zmodyfikowanego przypadku 2 z~lematu 4.4]
	Z~lematu 4.2 i~zmodyfikowanego przypadku 2 lematu 4.3:
	\[
		T(n) = \Theta(n^{\log_ba})+\Theta(n^{\log_ba}\lg^{k+1}n) = \Theta(n^{\log_ba}\lg^{k+1}n). \qedhere
	\]
\end{proof}

\exercise %4.4-3
\note{Aby udowodnić to wynikanie, musimy założyć dodatkowo, że\/ $c>0$ i~że funkcja\/ $f(n)$ jest asymptotycznie dodatnia.
W~tw.\ 4.1 oba te warunki wynikają z~faktu, że\/ $f(n)=\Omega(n^{\log_ba+\epsilon})$.}

\noindent Załóżmy, że dla danej funkcji asymptotycznie dodatniej $f(n)$ oraz stałych $a\ge1$, $b>1$, zachodzi warunek regularności, tzn.\ istnieje stała $0<c<1$ i~stała $n_0>0$, że dla każdego $n\ge n_0$ spełnione jest $af(n/b)\le cf(n)$.
Przyjmiemy, że $n_0$ zostało wybrane w~taki sposób, że dla wszystkich $n\ge n_0/b$ funkcja $f(n)$ osiąga wartości dodatnie.
Dla $n\ge n_0$ mamy
\[
	f(n) \ge (a/c)f(n/b),
\]
a~zatem, iterując tę nierówność, dostajemy
\[
	f(n) \ge (a/c)f(n/b) \ge (a/c)^2f(n/b^2) \ge \dots \ge (a/c)^if(n/b^i).
\]
Załóżmy, że nierówność iterowana była dokładnie $i$ razy, tzn.\ $n/b^{i-1}\ge n_0$ oraz $n/b^i<n_0$.
Stąd $\log_b(n/n_0)<i\le\log_b(n/n_0)+1$, czyli, ze wzoru (3.3), $i=\lfloor\log_b(n/n_0)+1\rfloor$.
Zachodzi więc
\[
	f(n) \ge (a/c)^{\lfloor\log_b(n/n_0)+1\rfloor}f\bigl(n/b^{\lfloor\log_b(n/n_0)+1\rfloor}\bigr).
\]
Zauważmy, że przy zadanych ograniczeniach na $a$ i~$c$, $a/c>1$, więc $f(n)$ musi być funkcją rosnącą dla $n\ge n_0/b$, skąd
\[
	f\bigl(n/b^{\lfloor\log_b(n/n_0)+1\rfloor}\bigr) > f(n/b^{\log_b(n/n_0)+1}) = f(n_0/b).
\]
Mamy następnie
\begin{align*}
	f(n) &> (a/c)^{\log_b(n/n_0)}\,f(n/b^{\log_b(n/n_0)+1}) \\[1mm]
	&= \frac{(n/n_0)^{\log_ba}}{(n/n_0)^{\log_bc}}\,f(n_0/b) \\[1mm]
	&= n^{\log_ba-\log_bc}\cdot n_0^{\log_bc-\log_ba}\cdot f(n_0/b).
\end{align*}
Dwa ostatnie czynniki powyższego wyrażenia są dodatnie i~niezależne od $n$, możemy więc potraktować ich iloczyn jak stałą $d>0$.
Ponieważ $0<c<1$ oraz $b>1$, to $\log_bc<0$, przyjmijmy więc $\epsilon=-\log_bc$.
Otrzymujemy zatem, że dla dowolnego $n\ge n_0$
\[
	f(n) > n^{\log_ba+\epsilon}\cdot d = \Omega(n^{\log_ba+\epsilon}),
\]
co należało wykazać.
