\problem{Tablice Monge'a} %4-7

\subproblem %4-7(a)
Elementy tablicy Monge'a $A$ spełniają nierówność
\[
	A[i,j]+A[k,l] \le A[i,l]+A[k,j],
\]
gdzie $1\le i<k\le m$ oraz $1\le j<l\le n$.
W~szczególności więc może być $k=i+1$ oraz $l=j+1$, zatem implikacja w~prawo zachodzi.

Dowód implikacji w~przeciwną stronę poprzedzimy wykazaniem nierówności
\[
	A[i,j]+A[k,j+1] \le A[i,j+1]+A[k,j] \tag{$*$}\label{intermediary}
\]
przez indukcję po $k=i+1$, $i+2$, \dots, $m$.
Dla $k=i+1$ nierówność~\eqref{intermediary} jest identyczna z~założeniem, a~więc krok bazowy trywialnie zachodzi.
Niech teraz $k>i+1$.
Założenie zachodzi w~szczególności dla $i=k-1$:
\[
	A[k-1,j]+A[k,j+1] \le A[k-1,j+1]+A[k,j].
\]
Dodając tę nierówność stronami do tej stanowiącej założenie indukcyjne (dla $k-1$), dostajemy
\[
	A[k-1,j]+A[k,j+1]+A[i,j]+A[k-1,j+1] \le A[k-1,j+1]+A[k,j]+A[i,j+1]+A[k-1,j]
\]
i~po redukcji powtarzających się wyrazów po obu stronach uzyskujemy~\eqref{intermediary}.

Możemy teraz przejść do dowodu nierówności
\[
	A[i,j]+A[k,l] \le A[i,l]+A[k,j],
\]
który przeprowadzimy przez indukcję po $l=j+1$, $j+2$, \dots, $n$.
Dla $l=j+1$ nierówność ta sprowadza się do~\eqref{intermediary}, załóżmy więc, że $l>j+1$.
Nierówność~\eqref{intermediary} zachodzi w~szczególności dla $j=l-1$:
\[
	A[i,l-1]+A[k,l] \le A[i,l]+A[k,l-1].
\]
Po dodaniu jej stronami do założenia indukcyjnego (dla $l-1$) mamy
\[
	A[i,l-1]+A[k,l]+A[i,j]+A[k,l-1] \le A[i,l]+A[k,l-1]+A[i,l-1]+A[k,j],
\]
skąd wynika teza.

\subproblem %4-7(b)
Na podstawie poprzedniego punktu można sprawdzić, że nierówność
\[
	A[1,2]+A[2,3] \le A[1,3]+A[2,2]
\]
jest fałszywa, przez co własność tablicy Monge'a jest zaburzona.
By przywrócić tę własność, można zamienić $A[1,3]$ np.\ na 24.

\subproblem %4-7(c)
Załóżmy, że dla pewnej tablicy Monge'a $A$ istnieje $1\le i<m$, takie że $f(i)>f(i+1)$.
Z~definicji tablicy Monge'a mamy
\[
	A[i,f(i+1)]+A[i+1,f(i)] \le A[i,f(i)]+A[i+1,f(i+1)],
\]
jednak nierówność ta stoi w~sprzeczności z~faktami $A[i,f(i+1)]>A[i,f(i)]$ oraz $A[i+1,f(i)]\ge A[i+1,f(i+1)]$ wynikającymi z~definicji $f$.

\subproblem %4-7(d)
Aby odnaleźć minimum wiersza $i$\nbhyphen tego (nieparzystego), sprawdzamy indeksy minimów wierszy $(i-1)$\nbhyphen szego oraz $(i+1)$\nbhyphen szego (parzystych).
Wartości te zostały wyznaczone w~pierwszej części algorytmu.
Na podstawie poprzedniego punktu mamy, że $f(i-1)\le f(i)\le f(i+1)$, wystarczy więc przejrzeć podtablicę $A[i,f(i-1)\twodots f(i+1)]$ i~wyznaczyć jej element minimalny.
Podczas przetwarzania pierwszego wiersza nie szukamy minimum poprzedniego (nieistniejącego), ale przyjmujemy dla uproszczenia procedury, że $f(0)=1$.
Podobny przypadek może się zdarzyć dla ostatniego nieparzystego wiersza, jeśli jest on ostatnim wierszem tablicy -- wtedy wystarczy przyjąć $f(m+1)=n$.

Poszukując minimum wiersza $i$\nbhyphen tego, sprawdzamy $f(i+1)-f(i-1)+1$ komórek w~tym wierszu.
Podczas działania procedury w~pesymistycznym przypadku zostanie sprawdzonych
\begin{align*}
	\sum_{\substack{1\le i\le m\\2\,\nmid\,i}}(f(i+1)-f(i-1)+1) &= \sum_{k=0}^{\lceil m/2\rceil-1}(f(2k+2)-f(2k)+1) \\[-4mm]
	&= \lceil m/2\rceil+\sum_{k=0}^{\lceil m/2\rceil-1}(f(2k+2)-f(2k)) \\[1mm]
	&= \lceil m/2\rceil+f(2\lceil m/2\rceil)-f(0) \\[2mm]
	&\le \lceil m/2\rceil+n-1
\end{align*}
komórek, co jest rzędu $O(m+n)$.

\subproblem %4-7(e)
Na ostatnim poziomie rekursji wyznaczenie minimum jednego wiersza tablicy wymaga sprawdzenia co najwyżej $O(n)$ komórek.
Na podstawie tego faktu i~oszacowania pokazanego w~poprzednim punkcie formułujemy następującą rekurencję opisującą pesymistyczny czas działania algorytmu:
\[
	T(m,n) =
	\begin{cases}
		O(n), & \text{jeśli $m=1$}, \\
		T(\lfloor m/2\rfloor,n)+O(m+n), & \text{jeśli $m>1$}.
	\end{cases}
\]

Pokażemy metodą podstawiania ze względu na $m$, że $T(m,n)\le c(2m+n\lg(2m))$, gdzie $c>0$ jest największą stałą ukrytą w~notacjach $O$ z~powyższej definicji.
W~kroku bazowym, gdy $m=1$,
\[
	T(1,n) \le cn < c(2+n\lg2).
\]
Dla $m\ge2$, przy założeniu, że $T(\lfloor m/2\rfloor,n)\le c(2\lfloor m/2\rfloor+n\lg(2\lfloor m/2\rfloor))\le c(m+n\lg m)$, mamy
\[
	T(m,n) \le c(m+n\lg m)+c(m+n) = c(2m+n(\lg m+1)) = c(2m+n\lg(2m)).
\]
Stąd $T(m,n)=O(m+n\lg m)$.
