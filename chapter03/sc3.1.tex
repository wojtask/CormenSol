\subchapter{Notacja asymptotyczna}

\exercise %3.1-1
Załóżmy, że dziedziną funkcji $f(n)$ i~$g(n)$ jest $\mathbb{N}$.
Niech $A=\{\,n\in\mathbb{N}:f(n)\ge g(n)\,\}$.
Dla odpowiednio dużych $n\in A$, dla których obie funkcje przyjmują wartości nieujemne, mamy
\[
    \max(f(n),g(n)) = f(n) \le f(n)+g(n) = O(f(n)+g(n)).
\]
Z~kolei
\[
    \max(f(n),g(n)) = f(n) \ge \frac{f(n)+g(n)}{2} = \Omega(f(n)+g(n))
\]
i~na mocy tw.\ 3.1 otrzymujemy, że $\max(f(n),g(n))=\Theta(f(n)+g(n))$.
Identyczny rezultat można uzyskać dla $n\in\mathbb{N}\setminus A$, tzn.\ gdy $f(n)<g(n)$.

\exercise %3.1-2
Aby pokazać, że $(n+a)^b=\Theta(n^b)$, należy znaleźć stałe $c_1$, $c_2$, $n_0>0$ takie, że
\[
	0 \le c_1n^b \le (n+a)^b \le c_2n^b
\]
dla wszystkich $n\ge n_0$.
Zauważmy, że $n+a\le n+|a|\le2n$, gdy $|a|\le n$ oraz $n+a\ge n-|a|\ge n/2$, o~ile $|a|\le n/2$.
Stąd, jeśli $n\ge 2|a|$, to zachodzi
\[
	0 \le n/2 \le n+a \le 2n.
\]
Ponieważ $b>0$, to powyższe nierówności możemy podnieść do potęgi $b$:
\begin{gather*}
	0 \le (n/2)^b \le (n+a)^b \le (2n)^b, \\
	0 \le 2^{-b}n^b \le (n+a)^b \le 2^bn^b.
\end{gather*}
Widać zatem, że szukanym stałym można nadać wartości $c_1=2^{-b}$, $c_2=2^b$ oraz $n_0=2|a|$.
Prawdą jest zatem, że $(n+a)^b=\Theta(n^b)$.

\exercise %3.1-3
Niech $T(n)$ będzie czasem działania algorytmu $A$.
Stwierdzenie ,,$T(n)$ wynosi co najmniej $O(n^2)$'' oznacza, że począwszy od pewnego $n$, $T(n)\ge f(n)$ dla pewnej funkcji $f(n)$ z~klasy $O(n^2)$.
Zdanie to pozostaje prawdziwe dla dowolnego $T$, wystarczy bowiem wybrać funkcję $f(n)$ tożsamościowo równą 0, która oczywiście jest w~$O(n^2)$.
Widać więc, że takie określenie nie przekazuje żadnej użytecznej informacji o~czasie działania algorytmu.

\exercise %3.1-4
Znajdziemy stałe $c$, $n_0>0$ takie, że $0\le2^{n+1}\le c2^n$ dla każdego $n\ge n_0$.
Ponieważ $2^{n+1}=2\cdot2^n$ dla każdego $n\ge1$, to można przyjąć $c=2$ oraz $n_0=1$.
A~zatem $2^{n+1}=O(2^n)$.

Spróbujmy teraz wyznaczyć te same stałe, ale spełniające zależność $0\le2^{2n}\le c2^n$ dla wszystkich $n\ge n_0$.
Mamy $2^{2n}=2^n\cdot2^n\le c2^n$, z~czego wynika, że $c\ge2^n$, co jednak uzależnia $c$ od funkcji zmiennej $n$ przyjmującej dowolnie duże wartości, a~więc $c$ nie może być stałą.
Stąd otrzymujemy, że $2^{2n}\ne O(2^n)$.

\exercise %3.1-5
Z~definicji notacji $\Theta$ mamy, że $f(n)=\Theta(g(n))$ wtedy i~tylko wtedy, gdy istnieją takie stałe $c_1$, $c_2$, $n_0>0$, że nierówności
\[
	0 \le c_1g(n) \le f(n) \le c_2g(n)
\]
zachodzą dla wszystkich $n\ge n_0$.
Możemy je zapisać w~formie układu
\[
	\begin{cases}
		0 \le c_1g(n) \le f(n) \\
		0 \le f(n) \le c_2g(n)
	\end{cases}.
\]
Z~pierwszej nierówności układu dostajemy, że $f(n)=\Omega(g(n))$, a~z~drugiej, że $f(n)=O(g(n))$.

\exercise %3.1-6
Niech $c_1$, $c_2$, $n_1$, $n_2$ będą pewnymi stałymi dodatnimi.
Pesymistyczny czas działania algorytmu wynosi $O(g(n))$ wtedy i~tylko wtedy, gdy dla dowolnych danych wejściowych rozmiaru $n\ge n_1$ czas jego działania $f(n)$ nie przekracza $c_1g(n)$.
Z~kolei to, że optymistyczny czas wynosi $\Omega(g(n))$, oznacza, że dla dowolnych danych wejściowych rozmiaru $n\ge n_2$ czas działania algorytmu $f(n)$ jest nie mniejszy niż $c_2g(n)$.
Widać zatem, że dla dowolnych danych rozmiaru $n\ge\max(n_1,n_2)$ mamy $0\le c_2g(n)\le f(n)\le c_1g(n)$, a~to jest równoważne z~tym, że $f(n)=\Theta(g(n))$.

\exercise %3.1-7
Załóżmy niepustość tego zbioru i~rozważmy pewne $f(n)\in o(g(n))\cap\omega(g(n))$.
Zachodzi zatem zarówno $f(n)=o(g(n))$, jak i~$f(n)=\omega(g(n))$, co oznacza, że dla każdych dodatnich stałych $c_1$ i~$c_2$ istnieje pewne dodatnie $n_0$, że
\[
	c_1g(n) < f(n) < c_2g(n)
\]
dla wszystkich $n\ge n_0$.
Dochodzimy do sprzeczności, bowiem nieprawdą jest, że każde liczby $c_1$ i~$c_2$ spełniają $c_1<c_2$.
Stąd $o(g(n))\cap\omega(g(n))=\emptyset$.

Udowodniona własność pokazuje, że nie ma potrzeby definiowania notacji $\theta$ odpowiadającej $\Theta$ i~analogicznej do $o$ i~$\omega$.

\exercise %3.1-8
\note{Notacja\/ $O$ dla funkcji dwóch zmiennych jest błędnie zdefiniowana -- warunek powinien zachodzić dla wszystkich\/ $n\ge n_0$ \textbf{lub}\/ $m\ge m_0$.}

\noindent Definicje notacji $\Omega$ i~$\Theta$ dla funkcji dwóch zmiennych:
\[
	\begin{split}
		\Omega(g(n,m)) &= \bigl\{\,f(n,m):\text{istnieją dodatnie stałe $c$, $n_0$, $m_0$ takie, że} \\
		&\qquad 0 \le cg(n,m) \le f(n,m) \text{ dla wszystkich $n \ge n_0$ lub $m \ge m_0$}\,\bigr\}, \\[2mm]
		\Theta(g(n,m)) &= \bigl\{\,f(n,m):\text{istnieją dodatnie stałe $c_1$, $c_2$, $n_0$, $m_0$ takie, że} \\
		&\qquad 0 \le c_1g(n,m) \le f(n,m) \le c_2g(n,m) \text{ dla wszystkich $n \ge n_0$ lub $m \ge m_0$}\,\bigr\}.
	\end{split}
\]
