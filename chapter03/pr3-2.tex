\problem{Względny rząd asymptotyczny} %3-2

\subproblem %3-2(a)
Ponieważ każdy dodatni wielomian rośnie szybciej niż dowolna funkcja polilogarytmiczna, czyli $\lg^kn=o(n^\epsilon)$, to stąd wynika, że również $\lg^kn=O(n^\epsilon)$.

\subproblem %3-2(b)
Podobnie, ze wzoru (3.9), mamy, że $n^k=o(c^n)$, co implikuje również $n^k=O(c^n)$.

\subproblem %3-2(c)
Wyrażenie $\sqrt{n}$ nie jest w~żadnej rozważanej relacji z~wyrażeniem $n^{\sin n}$, gdyż wartość wykładnika tego ostatniego przyjmuje wszystkie wartości między $-1$ a~1, podczas gdy $\sqrt{n}\equiv n^{1/2}$.

\subproblem %3-2(d)
Zachodzi $2^n=\omega(2^{n/2})$, bo
\[
	\lim_{n\to\infty}\frac{2^n}{2^{n/2}} = \lim_{n\to\infty}2^{n/2} = \infty,
\]
a~stąd wynika też $2^n=\Omega(2^{n/2})$.

\subproblem %3-2(e)
Funkcje $n^{\lg c}$ i~$c^{\lg n}$ dla $n>0$ są równoważne na podstawie tożsamości (3.15).

\subproblem %3-2(f)
Ze wzoru (3.18) mamy $\lg(n!)=\Theta(n\lg n)$, z~kolei $\lg n^n=n\lg n=\Theta(n\lg n)$, a~zatem obie funkcje są asymptotycznie równoważne.

\bigskip
\noindent Na podstawie powyższych uzasadnień dostajemy tabelę \ref{tab:3-2}.
\begin{table}[!ht]
	\centering
		\begin{tabular}{cc|c|c|c|c|c}
			$A$ & $B$ & $O$ & $o$ & $\Omega$ & $\omega$ & $\Theta$ \\
			\hline
			$\lg^kn$ & $n^\epsilon$ & tak & tak & nie & nie & nie \\
			\hline
			$n^k$ & $c^n$ & tak & tak & nie & nie & nie \\
			\hline
			$\sqrt{n}$ & $n^{\sin n}$ & nie & nie & nie & nie & nie \\
			\hline
			$2^n$ & $2^{n/2}$ & nie & nie & tak & tak & nie \\
			\hline
			$n^{\lg c}$ & $c^{\lg n}$ & tak & nie & tak & nie & tak \\
			\hline
			$\lg(n!)$ & $\lg n^n$ & tak & nie & tak & nie & tak
		\end{tabular}
		\caption{Porównanie funkcji na podstawie rzędu asymptotycznego.} \label{tab:3-2}
\end{table}
