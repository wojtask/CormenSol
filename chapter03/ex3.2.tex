\subchapter{Standardowe notacje i~typowe funkcje}

\exercise %3.2-1
Z~założenia, jeśli $n_1\le n_2$, to zachodzi $f(n_1)\le f(n_2)$ oraz $g(n_1)\le g(n_2)$, więc po dodaniu tych nierówności stronami otrzymujemy $f(n_1)+g(n_1)\le f(n_2)+g(n_2)$, czyli że funkcja $f(n)+g(n)$ jest monotonicznie rosnąca.

Traktując wartości funkcji $g(n)$ jako argumenty funkcji $f(n)$, otrzymamy $f(g(n_1))\le f(g(n_2))$, zatem $f(g(n))$ także jest funkcją monotonicznie rosnącą.

Jeśli ponadto założymy, że funkcje $f(n)$ i~$g(n)$ są nieujemne, to początkowe nierówności można pomnożyć stronami, co daje $f(n_1)\cdot g(n_1)\le f(n_2)\cdot g(n_2)$, a~to oznacza, że również funkcja $f(n)\cdot g(n)$ jest monotonicznie rosnąca.

\exercise %3.2-2
\[
	a^{\log_bc} = \left(b^{\log_ba}\right)^{\log_bc} = \left(b^{\log_bc}\right)^{\log_ba} = c^{\log_ba}
\]

\exercise %3.2-3
\begin{proof}[Dowód wzoru (3.18)]
	Z~definicji $n!$ i~własności logarytmów:
	\begin{align*}
		\lg(n!) &= \lg\biggl(\prod_{i=1}^ni\biggr) = \sum_{i=1}^n\lg i \le \sum_{i=1}^n\lg n = n\lg n = O(n\lg n), \\[1mm]
		\lg(n!) &= \lg\biggl(\prod_{i=1}^ni\biggr) = \sum_{i=1}^n\lg i \ge \sum_{i=\lceil n/2\rceil}^n\!\!\lg\lfloor n/2\rfloor = \lfloor n/2\rfloor\lg\lfloor n/2\rfloor = \Omega(n\lg n).
	\end{align*}
	Po skorzystaniu z~twierdzenia 3.1 dostajemy $\lg(n!)=\Theta(n\lg n)$.
\end{proof}

\begin{proof}[Dowód wzoru $n!=\omega(2^n)$]
	Równoważnie należy pokazać, że zachodzi
	\[
		\lim_{n\to\infty}\frac{n!}{2^n} = \infty.
	\]
	Zauważmy, że
	\[
	    \frac{n!}{2^n} = \biggl(\frac{1}{2}\biggr)\biggl(\frac{2}{2}\biggr)\dots\biggl(\frac{n}{2}\biggr).
	\]
	Wszystkie czynniki powyższego iloczynu są dodatnie, a~przy wzrastającym $n$ końcowe czynniki mogą być dowolnie duże, zatem cały iloczyn dąży do $\infty$.
\end{proof}

\begin{proof}[Dowód wzoru $n!=o(n^n)$]
	Na podstawie wzoru (3.1) dowód sprowadza się do pokazania, że
	\[
		\lim_{n\to\infty}\frac{n!}{n^n} = 0.
	\]
	Mamy
	\[
	    \frac{n!}{n^n} = \biggl(\frac{1}{n}\biggr)\biggl(\frac{2}{n}\biggr)\dots\biggl(\frac{n}{n}\biggr).
	\]
	Każdy czynnik po prawej stronie jest dodatni i~nie przekracza 1, a~dla wystarczająco dużego $n$ początkowe czynniki mogą być dowolnie blisko 0, zatem granicą tego iloczynu jest 0.
\end{proof}

\exercise %3.2-4
Z~definicji, funkcja $f(n)$ jest ograniczona wielomianowo, gdy istnieją stałe $c$, $k$, $n_0>0$ takie, że dla każdego $n\ge n_0$ zachodzi $f(n)\le cn^k$.
Po zlogarytmowaniu obu stron mamy
\[
	\lg f(n) \le \lg(cn^k) = \lg c+k\lg n \le (k+1)\lg n,
\]
o~ile $n\ge c$, co oznacza, że $\lg f(n)=O(\lg n)$.

Zanim przejdziemy do głównego dowodu, zauważmy, że $\lceil\lg n\rceil=\Theta(\lg n)$.
Dla każdego $n\ge2$ zachodzi bowiem
\[
	\lg n \le \lceil\lg n\rceil < \lg n+1 \le 2\lg n.
\]

Logarytmując pierwszą badaną funkcję przy wykorzystaniu wzoru (3.18), dostajemy
\[
	\lg(\lceil\lg n\rceil!) = \Theta(\lceil\lg n\rceil\lg\lceil\lg n\rceil) = \Theta(\lg n\lg\lg n) = \omega(\lg n),
\]
a~zatem $\lg(\lceil\lg n\rceil!)\ne O(\lg n)$ i~funkcja $\lceil\lg n\rceil!$ nie jest ograniczona wielomianowo.

Dla drugiej funkcji mamy
\[
	\lg(\lceil\lg\lg n\rceil!) = \Theta(\lceil\lg\lg n\rceil\lg\lceil\lg\lg n\rceil) = \Theta(\lg\lg n\lg\lg\lg n) = o((\lg\lg n)^2) = o(\lg n).
\]
Ostatni krok wynika z~tożsamości $\lg^bn=o(n^a)$ prawdziwej dla stałych $a$, $b>0$, w~której podstawiono $\lg n$ w~miejsce $n$ oraz przyjęto $a=1$ i~$b=2$.
Otrzymany rezultat potwierdza, że $\lg(\lceil\lg\lg n\rceil!)=O(\lg n)$, a~zatem funkcja $\lceil\lg\lg n\rceil!$ jest ograniczona wielomianowo.

\exercise %3.2-5
Każdy dodatni wielomian rośnie szybciej niż każda funkcja polilogarytmiczna, w~szczególności $m-1=\omega(\lg m)$.
Zatem dla każdego $c>0$ istnieje $m_0>0$, takie że $0\le c\lg m<m-1$ dla wszystkich $m\ge m_0$.
Zauważmy, że dla
\[
	\setlength{\abovedisplayskip}{2pt}
	\setlength{\belowdisplayskip}{2pt}
	n \ge n_0 = 2\mathbin{\uparrow\uparrow}m_0 = \underbrace{2^{2^{\cdot^{\cdot^{\cdot^{2}}}}}}_{m_0}
\]
wyrażenie $\lg^*n$ przyjmuje wszystkie wartości rzeczywiste większe lub równe $m_0$.
W~miejsce $m$ możemy więc wstawić $\lg^*n$.
Dokładniej, dla każdego $c>0$ istnieje $n_0>0$, takie że
\[
	0 \le c\lg\lg^*n < \lg^*n-1 = \lg^*\lg n
\]
dla wszystkich $n\ge n_0$, co oznacza, że $\lg^*\lg n=\omega(\lg\lg^*n)$.

\exercise %3.2-6
W~przypadkach bazowych, gdy $i=0$ lub $i=1$, mamy
\[
	\frac{\phi^0-\widehat\phi^0}{\sqrt{5}} = 0 = F_0 \quad\text{oraz}\quad \frac{\phi^1-\widehat\phi^1}{\sqrt{5}} = \frac{1+\sqrt{5}-\left(1-\sqrt{5}\right)}{2\sqrt{5}} = 1 = F_1.
\]
Załóżmy teraz, że $i\ge2$ i~że zachodzi
\[
	F_{i-2} = \frac{\phi^{i-2}-\widehat\phi^{i-2}}{\sqrt{5}} \quad\text{oraz}\quad F_{i-1} = \frac{\phi^{i-1}-\widehat\phi^{i-1}}{\sqrt{5}}.
\]
Dzięki zależnościom $1+\phi=\phi^2$ i~$1+\widehat\phi=\widehat\phi^2$ otrzymujemy
\[
	F_i = F_{i-2}+F_{i-1} = \frac{\phi^{i-2}-\widehat\phi^{i-2}}{\sqrt{5}}+\frac{\phi^{i-1}-\widehat\phi^{i-1}}{\sqrt{5}} = \frac{\phi^{i-2}(1+\phi)-\widehat\phi^{i-2}\bigl(1+\widehat\phi\bigr)}{\sqrt{5}} = \frac{\phi^i-\widehat\phi^i}{\sqrt{5}},
\]
co kończy dowód.

\exercise %3.2-7
Korzystając z~wyniku z~poprzedniego zadania, mamy
\begin{align*}
    F_{i+2}-\phi^i &= \frac{\phi^{i+2}-\widehat\phi^{i+2}}{\sqrt{5}}-\phi^i \\[1mm]
	&= \frac{\phi^i(\phi^2-\sqrt{5})-\widehat\phi^{i+2}}{\sqrt{5}} \\[1mm]
	&= \frac{\phi^i\cdot\frac{3-\sqrt{5}}{2}-\widehat\phi^i\cdot\frac{3-\sqrt{5}}{2}}{\sqrt{5}} \\
	&= \frac{3-\sqrt{5}}{2\sqrt{5}}\,\bigl(\phi^i-\widehat\phi^i\bigr).
\end{align*}
Ponieważ $\phi>|\widehat\phi|$, to otrzymane wyrażenie jest nieujemne dla każdego $i\ge0$ (równość zachodzi tylko wówczas, gdy $i=0$), a~zatem $F_{i+2}\ge\phi^i$ dla dowolnego $i\ge0$.
