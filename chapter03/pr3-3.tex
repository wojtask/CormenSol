\problem{Porządkowanie ze względu na rząd wielkości funkcji} %3-3

\subproblem %3-3(a)
W~uzasadnieniach będziemy wykorzystywać następujące obserwacje:
\begin{enumerate}[({O}1)]
	\item \textit{Jeśli\/ $f(n)$ jest funkcją asymptotycznie dodatnią i\/~$g(n)=\omega(h(n))$, to zachodzi\/ $f(n)g(n)=\omega(f(n)h(n))$.}
	\begin{proof}
		Niech $n_1$ będzie taką liczbą, że $f(n)>0$ dla wszystkich $n\ge n_1$.
		Z~kolei warunek $g(n)=\omega(h(n))$ oznacza, że dla każdego $c>0$ istnieje $n_2>0$, że dla $n\ge n_2$,
		\[
			0 \le ch(n) < g(n).
		\]
		Po przyjęciu $n\ge\max(n_1,n_2)$ i~pomnożeniu przez $f(n)$ wszystkich stron powyższej nierówności, otrzymujemy tezę.
	\end{proof}
	\item \textit{Jeśli\/ $f(n)=\omega(g(n))$ i\/~$\lim\limits_{n\to\infty}g(n)=\infty$, to\/ $2^{f(n)}=\omega\bigl(2^{g(n)}\bigr)$.}
	\begin{proof}
		Z~definicji notacji $\omega$, dla każdego $c>0$ istnieje $n_0>0$, takie że $f(n)>cg(n)$ dla wszystkich $n\ge n_0$.
		Niech $d$ będzie liczbą spełniającą nierówność $cg(n)\ge d+g(n)$.
		Wówczas $d\le(c-1)g(n)$, więc dzięki założeniu o~nieograniczoności $g(n)$, $d$ może przyjmować dowolną wartość rzeczywistą.
		Stąd $2^d$ może przyjmować dowolną wartość dodatnią oraz
		\[
			2^{f(n)} > 2^{cg(n)} \ge 2^{d+g(n)} = 2^d\cdot2^{g(n)} > 0,
		\]
		czyli $2^{f(n)}=\omega\bigl(2^{g(n)}\bigr)$.
	\end{proof}
	\item \textit{Dla\/ $f(n)=\Theta(g(n))$,\/ $h(n)=\omega(f(n))$ wtedy i~tylko wtedy, gdy\/ $h(n)=\omega(g(n))$.}
	\begin{proof}
		Z~definicji, $f(n)=\Theta(g(n))$ oznacza, że istnieją dodatnie stałe $c_1$, $c_2$ i~$n_0$, takie że
		\[
			0 \le c_1g(n) \le f(n) \le c_2g(n).
		\]

		Jeśli $h(n)=\omega(f(n))$, to dla każdego $a>0$ istnieje $n_a>0$, takie że $0\le af(n)<h(n)$ dla wszystkich $n\ge n_a$.
		Ustalmy $a$ i~niech $n\ge\max(n_0,n_a)$.
		Wówczas
		\[
			0 \le ac_1g(n) \le af(n) < h(n),
		\]
		czyli $h(n)=\omega(g(n))$, ponieważ czynnik $ac_1$ przyjmuje dowolną wartość dodatnią.

		W~dowodzie w~drugą stronę, dla każdego $b>0$ istnieje $n_b>0$, takie że $0\le bg(n)<h(n)$ dla wszystkich $n\ge n_b$.
		Ustalmy $b$ i~niech $n\ge\max(n_0,n_b)$.
		Mamy
		\[
			0 \le (b/c_2)f(n) \le bg(n) < h(n),
		\]
		a~stąd $h(n)=\omega(f(n))$, bo $b/c_2$ jest dowolną stałą dodatnią.
	\end{proof}
\end{enumerate}

Poniżej znajduje się lista dowodów zależności $g_i(n)=\omega(g_{i+1}(n))$ (które implikują $g_i(n)=\Omega(g_{i+1}(n))$) albo $g_i(n)=\Theta(g_{i+1}(n))$, dla $i=1$, 2, \dots, 29, w~których $g_i$ są rozważanymi funkcjami po ich uporządkowaniu względem notacji $\Omega$.
\begin{description}[font=\textnormal, topsep=2ex, itemsep=2ex]
	\item[$2^{2^{n+1}}=\omega\bigl(2^{2^n}\bigr)$:]
	Z~(O1) dla $f(n)=g(n)=2^{2^n}$ i~$h(n)=1$.
	\item[$2^{2^n}=\omega((n+1)!)$:]
	Z~(3.9) wynika $2^n=\omega(n^2)$, a~z~(O1), $n^2=\omega(n\lg n)$, bo $n=\omega(\lg n)$, więc z~przechodniości $\omega$, $2^n=\omega(n\lg n)$.
	Z~(3.18), $\lg((n+1)!)=\lg(n+1)+\lg(n!)=\Theta(n\lg n)$, dlatego dzięki (O3) możemy napisać $2^n=\omega(\lg(n+1)!)$ i~zależność wynika z~(O2).
	\item[$(n+1)!=\omega(n!)$:]
	Z~(O1) dla $f(n)=n!$, $g(n)=n+1$ i~$h(n)=1$.
	\item[$n!=\omega(e^n)$:]
	Ze wzoru (3.18) i~z~(O1), $\lg(n!)=\Theta(n\lg n)=\omega(n)$, natomiast $\lg e^n=\Theta(n)$.
	Dzięki (O3) mamy $\lg(n!)=\omega(\lg e^n)$ i~zależność wynika z~(O2).
	\item[$e^n=\omega(n\cdot2^n)$:]
	Z~(3.9) mamy $(e/2)^n=\omega(n)$.
	Zależność wynika więc z~(O1) dla $f(n)=2^n$, $g(n)=(e/2)^n$ i~$h(n)=n$.
	\item[$n\cdot2^n=\omega(2^n)$:]
	Z~(O1) dla $f(n)=2^n$, $g(n)=n$ i~$h(n)=1$.
	\item[$2^n=\omega((3/2)^n)$:]
	Z~(O1) dla $f(n)=(3/2)^n$, $g(n)=(4/3)^n$ i~$h(n)=1$.
	\item[$(3/2)^n=\omega\bigl(n^{\lg\lg n}\bigr)$:]
	Zachodzi:
	\[
		\lg(3/2)^n = n\lg(3/2) \quad\text{oraz}\quad \lg\bigl(n^{\lg\lg n}\bigr) = \lg n\lg\lg n < \lg n\lg n = \lg^2n.
	\]
	Zgodnie ze stwierdzeniem podanym w~Podręczniku, że dodatnie wielomiany rosną szybciej niż funkcje polilogarytmiczne, $\lg(3/2)^n=\omega\bigr(\!\lg\bigl(n^{\lg\lg n}\bigr)\bigl)$, więc wystarczy skorzystać z~(O2).
	\item[$n^{\lg\lg n}=\Theta\bigl((\lg n)^{\lg n}\bigr)$:]
	Z~(3.15).
	\item[$(\lg n)^{\lg n}=\omega((\lg n)!)$:]
	Ze wzoru Stirlinga:
	\[
		(\lg n)! = \Theta\biggl(\!\sqrt{2\pi\lg n}\,\Bigl(\frac{\lg n}{e}\Bigr)^{\lg n}\biggr) = \Theta\biggl(\!(\lg n)^{\lg n}\,\frac{\sqrt{\lg n}}{n^{\lg e}}\biggr).
	\]
	Niech
	\[
		f(n) = (\lg n)^{\lg n}\,\frac{\sqrt{\lg n}}{n^{\lg e}} \quad\text{oraz}\quad g(n) = \frac{n^{\lg e}}{\sqrt{\lg n}}.
	\]
	Zachodzi $g(n)=\omega(1)$, więc z~(O1), $f(n)g(n)=(\lg n)^{\lg n}=\omega(f(n))$.
	Wracając do początkowego oszacowania, z~(O3) dostajemy $(\lg n)^{\lg n}=\omega((\lg n)!)$.
	\item[$(\lg n)!=\omega(n^3)$:]
	Z~(3.18) i~(O1) mamy
	\[
		\lg((\lg n)!) = \Theta(\lg n\lg\lg n) = \omega(\lg n) \quad\text{oraz}\quad \lg n^3 = 3\lg n,
	\]
	więc $\lg((\lg n)!)=\omega(\lg n^3)$ i~korzystamy z~(O2).
	\item[$n^3=\omega(n^2)$:]
	Z~punktu (e) problemu \refProblem{3-1}.
	\item[$n^2=\Theta\bigl(4^{\lg n}\bigr)$:]
	Z~(3.15).
	\item[$4^{\lg n}=\omega(n\lg n)$:]
	Z~(3.15), $4^{\lg n}=n^2$, więc korzystamy z~(O1) dla $f(n)=g(n)=n$ i~$h(n)=\lg n$.
	\item[$n\lg n=\Theta(\lg(n!))$:]
	Z~(3.18).
	\item[$\lg(n!)=\omega(n)$:]
	Z~(3.18) i~(O1), $\lg(n!)=\Theta(n\lg n)=\omega(n)$.
	\item[$n=\Theta\bigl(2^{\lg n}\bigr)$:]
	Z~(3.15).
	\item[$2^{\lg n}=\omega\bigl(\!\bigl(\!\sqrt{2}\bigr)^{\lg n}\bigr)$:]
	Z~(3.15), $2^{\lg n}=n$, a~także $\bigl(\!\sqrt{2}\bigr)^{\lg n}=n^{\lg\sqrt{2}}=\sqrt{n}$.
	Zależność zachodzi więc na podstawie punktu (e) problemu \refProblem{3-1}.
	\item[$\bigl(\!\sqrt{2}\bigr)^{\lg n}=\omega\Bigl(2^{\sqrt{2\lg n}}\Bigr)$:]
	Zachodzi
	\[
		\lg\bigl(\!\sqrt{2}\bigr)^{\lg n} = \lg n\lg\sqrt{2} = \frac{\lg n}{2} = \sqrt{2\lg n}\cdot\frac{\sqrt{\lg n}}{2\sqrt{2}} = \sqrt{2\lg n}\cdot\omega(1),
	\]
	zatem z~(O1) wynika $\lg\bigl(\!\sqrt{2}\bigr)^{\lg n}=\omega\Bigl(\lg2^{\sqrt{2\lg n}}\Bigr)$.
	Wystarczy więc skorzystać z~(O2).
	\item[$2^{\sqrt{2\lg n}}=\omega(\lg^2n)$:]
	Rozważmy
	\[
		\lg\lg2^{\sqrt{2\lg n}} = \Theta(\lg\lg n) \quad\text{oraz}\quad \lg\lg\lg^2n = \Theta(\lg\lg\lg n).
	\]

	Można uzasadnić, że $\lg\lg n=\omega(\lg\lg\lg n)$, zaczynając od zależności $\lg m=\omega(m)$ oznaczającej, że dla każdego $c>0$ istnieje $m_0>0$, takie że $m>c\lg m$ dla wszystkich $m\ge m_0$.
	Następnie, rozumując podobnie jak w~\refExercise{3.2-5}, zastępujemy $m$ wyrażeniem $\lceil\lg\lg n\rceil$ po stwierdzeniu, że dla $n\ge2^{2^{m_0}}$\!, wyrażenie to przyjmuje każde całkowite $m\ge m_0$.
	Wtedy dla każdego $c>0$ istnieje $n_0>0$, że dla wszystkich $n\ge n_0$ zachodzi
	\[
		\lg\lg n \ge \frac{\lceil\lg\lg n\rceil}{2} > \frac{c\lg\lceil\lg\lg n\rceil}{2} \ge \frac{c\lg\lg\lg n}{2}.
	\]

	Powracając do głównego dowodu, mamy $\lg\lg2^{\sqrt{2\lg n}}=\omega(\lg\lg\lg^2n)$, więc dwukrotnie korzystamy z~(O2).
	\item[$\lg^2n=\omega(\ln n)$:]
	Z~(O1), $\lg^2n=\omega(\lg n)$, zaś $\ln n=\Theta(\lg n)$, więc zależność zachodzi dzięki (O3).
	\item[$\ln n=\omega\bigl(\!\sqrt{\lg n}\bigr)$:]
	Zachodzi
	\[
		\ln n = \frac{\lg n}{\lg e} = \sqrt{\lg n}\cdot\frac{\sqrt{\lg n}}{\lg e} = \sqrt{\lg n}\cdot\omega(1),
	\]
	dlatego zależność zachodzi na podstawie (O1).
	\item[$\sqrt{\lg n}=\omega(\ln\ln n)$:]
	Zachodzi
	\[
		\lg\sqrt{\lg n} = \Theta(\lg\lg n) \quad\text{oraz}\quad \lg\ln\ln n = \Theta(\lg\lg\lg n).
	\]
	Prawdą jest, że $\lg\lg n=\omega(\lg\lg\lg n)$, czyli $\lg\sqrt{\lg n}=\omega(\lg\ln\ln n)$ i~zależność wynika z~(O2).
	\item[$\ln\ln n=\omega\bigl(2^{\lg^*\!n}\bigr)$:]
	Zachodzi $\lg\ln\ln n=\Theta(\lg\lg\lg n)=\omega(\lg^*n)$, więc zależność wynika z~(O2).
	\item[$2^{\lg^*\!n}=\omega(\lg^*n)$:]
	Zależność wynika z~(O2), bo $\lg2^{\lg^*\!n}=\lg^*n=\omega(\lg\lg^*n)$, przy czym ostatni krok jest uzasadniony w~kolejnych dowodach.
	\item[$\lg^*n=\Theta(\lg^*\lg n)$:]
	Dla $n\ge2$, $\lg^*\lg n=\lg^*n-1$.
	\item[$\lg^*\lg n=\omega(\lg\lg^*n)$:]
	Z~rozwiązania \refExercise{3.2-5}.
	\item[$\lg\lg^*n=\omega\bigl(n^{1/\!\lg n}\bigr)$:]
	Z~(3.14), $1/\!\lg n=\log_n2$, a~z~(3.15), $n^{1/\!\lg n}=n^{\log_n2}=2^{\log_nn}=2=\Theta(1)$.
	Z~kolei $\lg\lg^*n=\omega(1)$, zatem zależność zachodzi na podstawie (O3).
	\item[$n^{1/\!\lg n}=\Theta(1)$:]
	Zależność jest spełniona na podstawie poprzedniego uzasadnienia.
\end{description}

Tabela~\ref{tab:3-3} przedstawia badane funkcje uporządkowane względem notacji $\Omega$ na podstawie powyższych dowodów.
\begin{table}[!ht]
	\renewcommand*{\arraystretch}{1.4}
	\centering
	\[
		\begin{array}{|rl|rl|rl|} \hline
		g_1(n)= & 2^{2^{n+1}} & g_{11}(n)= & (\lg n)! & g_{21}(n)= & \lg^2n \\ \hline
		g_2(n)= & 2^{2^n} & g_{12}(n)= & n^3 & g_{22}(n)= & \ln n \\ \hline
		g_3(n)= & (n+1)! & g_{13}(n)= & n^2 & g_{23}(n)= & \sqrt{\lg n} \\ \cline{1-2}\cline{5-6}
		g_4(n)= & n! & g_{14}(n)= & 4^{\lg n} & g_{24}(n)= & \ln\ln n \\ \hline
		g_5(n)= & e^n & g_{15}(n)= & n\lg n & g_{25}(n)= & 2^{\lg^*\!n} \\ \cline{1-2}\cline{5-6}
		g_6(n)= & n\cdot2^n & g_{16}(n)= & \lg(n!) & g_{26}(n)= & \lg^*n \\ \cline{1-4}
		g_7(n)= & 2^n & g_{17}(n)= & n & g_{27}(n)= & \lg^*\lg n \\ \cline{1-2}\cline{5-6}
		g_8(n)= & (3/2)^n & g_{18}(n)= & 2^{\lg n} & g_{28}(n)= & \lg\lg^*n \\ \hline
		g_9(n)= & n^{\lg\lg n} & g_{19}(n)= & \bigl(\!\sqrt{2}\bigr)^{\lg n} & g_{29}(n)= & n^{1/\!\lg n} \\ \cline{3-4}
		g_{10}(n)= & (\lg n)^{\lg n} & g_{20}(n)= & 2^{\sqrt{2\lg n}} & g_{30}(n)= & 1 \\ \hline
		\end{array}
	\]
	\caption{Uporządkowanie funkcji względem asymptotycznego tempa wzrostu.
	Funkcje znajdujące się w~tej samej komórce są asymptotycznie równoważne.} \label{tab:3-3}
\end{table}

\subproblem %3-3(b)
Oto przykład funkcji, która nie jest ani $O(g_i(n))$, ani $\Omega(g_i(n))$, gdzie $i=1$, 2, \dots, 30:
\[
	f(n) = \bigl(2^{2^{n+1}}\bigr)^{n\bmod2} =
	\begin{cases}
		\hfill g_1(n), & \text{jeśli $n$ jest nieparzyste}, \\
		g_{30}(n), & \text{jeśli $n$ jest parzyste}.
	\end{cases}
\]
