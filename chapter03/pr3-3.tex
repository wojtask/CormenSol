\problem{Porządkowanie ze względu na rząd wielkości funkcji} %3-3

\subproblem %3-3(a)
Poniższe uzasadnienia stanowią dowody, że $g_i(n)=\Omega(g_{i+1}(n))$ dla $i=1$, 2, \dots, 29, gdzie $g_i(n)$ to rozważane kolejno funkcje.
W~niektórych dowodach korzystamy z~obserwacji, że jeśli $f(n)=g(n)h(n)$ i~$h(n)=\omega(1)$, to $f(n)=\omega(g(n))$.
Ponadto wykorzystujemy fakt, że $\lg f(n)=\omega(\lg g(n))$ implikuje $f(n)=\omega(g(n))$, co można łatwo wykazać.
\begin{itemize}
\item $2^{2^{n+1}}=\Omega\bigl(2^{2^n}\bigr)$
	\[
		2^{2^{n+1}} = 2^{2^n}\cdot2^{2^n} = \omega\bigl(2^{2^n}\bigr), \quad\text{bo $2^{2^n} = \omega(1)$}.
	\]
\item $2^{2^n}=\Omega((n+1)!)$ \\
	Logarytmując obie funkcje i~wykorzystując wzór (3.18), otrzymujemy
	\[
		\lg 2^{2^n} = 2^n \quad\text{oraz}\quad \lg((n+1)!) = \Theta((n+1)\lg(n+1)) = \Theta(n\lg n).
	\]
	Ponieważ $2^n=\omega(n^2)$ i~$n^2=\omega(n\lg n)$, to mamy, że $2^n=\omega(n\lg n)$.
Powracając do początkowych funkcji, dostajemy $2^{2^n}=\omega((n+1)!)$, skąd wynika prawdziwość dowodzonej zależności.
\item $(n+1)!=\Omega(n!)$
	\[
		(n+1)! = (n+1)\cdot n! = \omega(n!), \quad\text{bo $n+1 = \omega(1)$}.
	\]
\item $n!=\Omega(e^n)$ \\
	Zbadajmy logarytmy obu funkcji.
Ze wzoru (3.18) otrzymujemy $\lg(n!)=\Theta(n\lg n)=\omega(n)$, a~$\lg e^n=\Theta(n)$.
Prawdą jest zatem, że $\lg(n!)=\omega(\lg e^n)$ i~stąd wynika zależność $n!=\omega(e^n)$.
\item $e^n=\Omega(n\cdot2^n)$
	\[
		e^n = (e/2)^n2^n = \omega(n\cdot2^n), \quad\text{bo $(e/2)^n = \omega(n)$},
	\]
	ponieważ funkcje wykładnicze rosną szybciej niż wielomiany.
\item $n\cdot2^n=\Omega(2^n)$ \\
	Tożsamość zachodzi, bo $n=\omega(1)$.
\item $2^n=\Omega((3/2)^n)$
	\[
		2^n = (4/3)^n(3/2)^n = \omega((3/2)^n), \quad\text{bo $(4/3)^n = \omega(1)$}.
	\]
\item $(3/2)^n=\Omega\bigl(n^{\lg\lg n}\bigr)$ \\
	Logarytmując obie funkcje, dostajemy
	\[
		\lg(3/2)^n = \Theta(n) \quad\text{oraz}\quad \lg n^{\lg\lg n} = \lg n\lg\lg n = o(\lg^2n).
	\]
	Wystarczy pokazać, że $n=\omega(\lg^2n)$.
Po podstawieniu $n=2^h$ wzór przyjmuje postać $2^h=\omega(h^2)$, co jest prawdą, ponieważ funkcja wykładnicza rośnie szybciej niż wielomian.
\item $n^{\lg\lg n}=\Omega\bigl((\lg n)^{\lg n}\bigr)$ \\
	Na mocy tożsamości (3.15) funkcje są równoważne.
\item $(\lg n)^{\lg n}=\Omega((\lg n)!)$ \\
	Jeśli podstawimy $n=2^h$, to otrzymamy wzór $h^h=\Omega(h!)$, który jest prawdziwy na podstawie zależności $n!=o(n^n)$, zaprezentowanej w~Podręczniku.
\item $(\lg n)!=\Omega(n^3)$ \\
	Korzystając z~logarytmu pierwszej funkcji oszacowanego w~poprzednim uzasadnieniu oraz z~tego, że $\lg n^3=\Theta(\lg n)$, dostajemy żądany wynik, ponieważ $\lg\lg n=\omega(1)$.
\item $n^3=\Omega(n^2)$ \\
	Tożsamość zachodzi wprost z~punktu (b) problemu \refProblem{3-1}.
\item $n^2=\Omega\bigl(4^{\lg n}\bigr)$ \\
	Funkcje są tożsame -- na podstawie wzoru (3.15) mamy $4^{\lg n}=n^{\lg4}=n^2$.
\item $4^{\lg n}=\Omega(n\lg n)$ \\
	Wzór zachodzi, bo $4^{\lg n}=n^2$, a~$n=\omega(\lg n)$.
\item $n\lg n=\Omega(\lg(n!))$ \\
	Obie funkcje są asymptotycznie równoważne na podstawie wzoru (3.18).
\item $\lg(n!)=\Omega(n)$ \\
	Ze wzoru (3.18) mamy, że $\lg(n!)=\Theta(n\lg n)$, więc tożsamość jest prawdziwa, bo $\lg n=\omega(1)$.
\item $n=\Omega\bigl(2^{\lg n}\bigr)$ \\
	Na mocy wzoru (3.15) zachodzi $2^{\lg n}=n^{\lg2}=n$, a~więc funkcje są tożsame.
\item $2^{\lg n}=\Omega\bigl(\!\bigl(\!\sqrt{2}\bigr)^{\lg n}\bigr)$ \\
	Z~poprzedniego uzasadnienia mamy, że $2^{\lg n}=n$, a~$\bigl(\!\sqrt{2}\bigr)^{\lg n}=n^{\lg\sqrt{2}}=\sqrt{n}$ (ze wzoru (3.15)), więc tożsamość zachodzi, ponieważ $\sqrt{n}=\omega(1)$.
\item $\bigl(\!\sqrt{2}\bigr)^{\lg n}=\Omega\bigl(2^{\sqrt{2\lg n}}\bigr)$ \\
	Rozważmy tożsamość $2^{\lg n}=n$ i~podnieśmy ją do potęgi $\sqrt{2/\!\lg n}$.
Otrzymujemy $2^{\sqrt{2\lg n}}=n^{\sqrt{2/\!\lg n}}$, a~zatem $2^{\sqrt{2\lg n}}=\Theta\bigl(n^{\sqrt{2/\!\lg n}}\bigr)$.
Ponieważ $\bigl(\!\sqrt{2}\bigr)^{\lg n}=n^{1/2}$, to wystarczy pokazać, że $1/2=\Omega\bigl(\!\sqrt{2/\!\lg n}\bigr)$.
Wzór oczywiście zachodzi, gdyż funkcja z~prawej strony jest malejąca i~dąży do 0 wraz ze wzrostem $n$.
\item $2^{\sqrt{2\lg n}}=\Omega(\lg^2 n)$ \\
	Biorąc logarytmy obu funkcji, dostajemy
	\[
		\lg2^{\sqrt{2\lg n}} = \sqrt{2\lg n} = \Theta\bigl(\lg^{1/2}n\bigr) \quad\text{oraz}\quad \lg\lg^2n = \Theta(\lg\lg n).
	\]
	Pozostaje zatem zbadać prawdziwość wzoru $\lg^{1/2}n=\Omega(\lg\lg n)$.
Przyjmując $n=2^{4^h}$, sprowadzamy go do postaci $2^h=\Omega(h)$, co oczywiście zachodzi.
\item $\lg^2n=\Omega(\ln n)$ \\
	Zależność jest prawdziwa, ponieważ $\ln n=\Theta(\lg n)$ oraz $\lg n=\omega(1)$.
\item $\ln n=\Omega\bigl(\!\sqrt{\lg n}\bigr)$ \\
	Wystarczy przyjąć $n=e^h$, aby otrzymać tożsamość $h=\Omega\bigl(\!\sqrt{h}\bigr)$, która zachodzi na mocy tego, że $\sqrt{h}=\omega(1)$.
\item $\sqrt{\lg n}=\Omega(\ln\ln n)$ \\
	Prawa strona jest identyczna z~$\Omega(\lg\lg n)$, zatem przyjmując $n=2^{4^h}$, dostajemy tożsamość $2^h=\Omega(h)$.
\item $\ln\ln n=\Omega\bigl(2^{\lg^*n}\bigr)$ \\
	Logarytmując funkcje, dostajemy
	\[
		\lg\ln\ln n = \Theta\bigl(\lg^{(3)}n\bigr) \quad\text{oraz}\quad \lg\bigl(2^{\lg^*n}\bigr) = \lg^*n.
	\]
	By wykazać prawdziwość tożsamości, dokonajmy podstawienia
	\[
		n = 2^{2^{\cdot^{\cdot^{\cdot^{2^\epsilon}}}}}\vbox{\hbox{$\Big\}\scriptstyle k$}\kern0pt},
	\]
	gdzie $0<\epsilon\le1$, a~$k\ge3$ jest liczbą dwójek.
Wyliczając wartości obu funkcji dla takiego argumentu, otrzymujemy
	\[
		\lg^{(3)}n = 2^{2^{\cdot^{\cdot^{\cdot^{2^\epsilon}}}}}\vbox{\hbox{$\Big\}\scriptstyle k-3$}\kern0pt} \quad\text{oraz}\quad \lg^*n=k.
	\]
	Oczywistym jest, że funkcja po lewej stronie jest asymptotycznie większa od funkcji po prawej stronie.
\item $2^{\lg^*n}=\Omega(\lg^*n)$ \\
	Po zlogarytmowaniu obu funkcji i~wykorzystaniu wzoru (3.15), otrzymujemy
	\[
		\lg2^{\lg^*n} = \lg^*n \quad\text{oraz}\quad \lg\lg^*n.
	\]
	Biorąc $h=\lg^*n$, sprowadzamy tożsamość do udowodnionej wcześniej $h=\Omega(\lg h)$, a~zatem dowodzona zależność jest spełniona.
\item $\lg^*n=\Omega(\lg^*\lg n)$ \\
	Funkcje są asymptotycznie równoważne, ponieważ $\lg^*\lg n=\lg^*n-1$ dla $n\ge2$.
\item $\lg^*\lg n=\Omega(\lg\lg^*n)$ \\
	Tożsamość zachodzi na podstawie rozwiązania \refExercise{3.2-5}.
\item $\lg\lg^*n=\Omega\bigl(n^{1/\!\lg n}\bigr)$ \\
	Z~własności logarytmów mamy, że $1/\!\lg n=\log_n2$, a~więc wykorzystując wzór (3.15), dostajemy $n^{1/\!\lg n}=n^{\log_n2}=2^{\log_nn}=2=\Theta(1)$, skąd wynika prawdziwość zależności.
\item $n^{1/\!\lg n}=\Omega(1)$ \\
	Tożsamość zachodzi, gdyż z~poprzedniego uzasadnienia $n^{1/\!\lg n}=\Theta(1)$.
\end{itemize}

Tabela \ref{tab:3-3} przedstawia badane funkcje uporządkowane względem notacji $\Omega$ na podstawie powyższych dowodów.
\begin{table}[!ht]
	\centering
		\[
			\begin{array}{|lc|lc|lc|} \hline
				g_1(n)= & 2^{2^{n+1}} & g_{11}(n)= & (\lg n)! & g_{21}(n)= & \lg^2n \\ \hline
				g_2(n)= & 2^{2^n} & g_{12}(n)= & n^3 & g_{22}(n)= & \ln n \\ \hline
				g_3(n)= & (n+1)! & g_{13}(n)= & n^2 & g_{23}(n)= & \sqrt{\lg n} \\ \cline{1-2}\cline{5-6}
				g_4(n)= & n! & g_{14}(n)= & 4^{\lg n} & g_{24}(n)= & \ln\ln n \\ \hline
				g_5(n)= & e^n & g_{15}(n)= & n\lg n & g_{25}(n)= & 2^{\lg^*n} \\ \cline{1-2}\cline{5-6}
				g_6(n)= & n\cdot2^n & g_{16}(n)= & \lg(n!) & g_{26}(n)= & \lg^*n \\ \cline{1-4}
				g_7(n)= & 2^n & g_{17}(n)= & n & g_{27}(n)= & \lg^*\lg n \\ \cline{1-2}\cline{5-6}
				g_8(n)= & (3/2)^n & g_{18}(n)= & 2^{\lg n} & g_{28}(n)= & \lg\lg^*n \\ \hline
				g_9(n)= & n^{\lg\lg n} & g_{19}(n)= & \bigl(\!\sqrt{2}\bigr)^{\lg n} & g_{29}(n)= & n^{1/\!\lg n} \\ \cline{3-4}
				g_{10}(n)= & (\lg n)^{\lg n} & g_{20}(n)= & 2^{\sqrt{2\lg n}} & g_{30}(n)= & 1 \\ \hline
			\end{array}
		\]
	\caption{Uporządkowanie funkcji względem asymptotycznego tempa wzrostu.
Funkcje znajdujące się w~tej samej komórce są asymptotycznie równoważne.} \label{tab:3-3}
\end{table}

\subproblem %3-3(b)
Oto przykład funkcji, która nie jest ani $O(g_i(n))$, ani $\Omega(g_i(n))$, gdzie $i=1$, 2, \dots, 30:
\[
	f(n) =
	\begin{cases}
		2^{2^{n+2}}, & \text{jeśli $n$ jest parzyste}, \\
		0, & \text{jeśli $n$ jest nieparzyste}.
	\end{cases}
\]
Gdyby rozważać funkcję $f(n)$ w~dziedzinie liczb parzystych, to byłaby ona $\Omega(g_1(n))$, z~kolei po obcięciu dziedziny do liczb nieparzystych, $f(n)$ byłoby na końcu listy funkcji w~uporządkowaniu z~poprzedniego punktu.
Dlatego wraz ze wzrostem $n$, w~zależności od jego parzystości, $f(n)$ jest asymptotycznie większe od wszystkich funkcji $g_i(n)$ albo od nich asymptotycznie mniejsze.
