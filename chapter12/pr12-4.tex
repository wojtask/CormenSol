\problem{Zliczanie różnych drzew binarnych} %12-4

\subproblem %12-4(a)
Oczywiście jest tylko jedno drzewo binarne niezawierające żadnego węzła, zatem $b_0=1$.

Rozważmy drzewo binarne o~$n\ge1$ węzłach.
Jeśli jego lewe poddrzewo ma $k$ węzłów, gdzie $k=0$, 1, \dots, $n-1$, to jego prawe poddrzewo ma $n-1-k$ węzłów.
Wynika stąd, że liczba drzew binarnych o~$n$ węzłach i~o~ustalonej liczbie $k$ węzłów w~lewym poddrzewie, jest równa $b_kb_{n-1-k}$.
Liczba wszystkich drzew binarnych o~$n$ węzłach wynosi zatem
\[
	b_n = \sum_{k=0}^{n-1}b_kb_{n-1-k}.
\]

\subproblem %12-4(b)
Wyznaczmy $B^2(x)$.
Kwadrat sumy wyrazów ciągu jest sumą iloczynów każdej pary wyrazów tego ciągu (iloczyn Cauchy'ego szeregów).
Dzięki odpowiedniemu pogrupowaniu iloczynów możemy napisać:
\[
	B^2(x) = \biggl(\sum_{n=0}^\infty b_nx^n\biggr)^2 = \sum_{n=0}^\infty\sum_{k=0}^nb_kx^kb_{n-k}x^{n-k} = \sum_{n=0}^\infty\sum_{k=0}^nb_kb_{n-k}x^n.
\]
Wewnętrzną sumę na podstawie punktu (a) zamieniamy na $b_{n+1}$ i~przy założeniu, że $x\ne0$:
\[
	B^2(x) = \sum_{n=0}^\infty b_{n+1}x^n = \frac{1}{x}\sum_{n=0}^\infty b_{n+1}x^{n+1} = \frac{1}{x}\sum_{n=1}^\infty b_nx^n = \frac{1}{x}\biggl(\sum_{n=0}^\infty b_nx^n-1\biggr) = \frac{B(x)-1}{x}.
\]
Stąd $B(x)=xB^2(x)+1$ i~rozwiązaniem tego równania ze względu na $B(x)$ jest
\[
	B(x) = \frac{1-\sqrt{1-4x}}{2x},
\]
co weryfikujemy przez podstawienie:
\[
	xB^2(x)+1 = x\cdot\frac{1+1-4x-2\sqrt{1-4x}}{4x^2}+1 = \frac{2-2\sqrt{1-4x}}{4x}-1+1 = \frac{1-\sqrt{1-4x}}{2x} = B(x).
\]

\subproblem %12-4(c)
\note{Rozwinięciem Taylora funkcji\/ $f(x)$ wokół punktu\/ $x=a$ jest\/ $f(x)=\sum_{k=0}^\infty\frac{f^{(k)}(a)}{k!}(x-a)^k$.}

\noindent Aby skorzystać z~rozwinięcia Taylora funkcji $f(x)$, musimy wyznaczyć jej kolejne pochodne.

\medskip
\noindent\textsf{\textbf{Lemat.}} \textit{$k$\nbhyphen tą pochodną funkcji\/ $f(x)=\sqrt{1-4x}$ jest
\[
	f^{(k)}(x) = -\frac{2(2k-2)!}{(k-1)!}\,(1-4x)^{1/2-k}.
\]
}
\begin{proof}
Wzór udowodnimy przez indukcję.
Jeśli $k=1$, to $f'(x)=\frac{-2}{\sqrt{1-4x}}$ i~łatwo sprawdzić, że wzór zachodzi.
Niech teraz $k\ge2$.
Wykorzystując założenie indukcyjne, mamy:
\begin{align*}
	f^{(k)}(x) &= \bigl(f^{(k-1)}(x)\bigr)' \\
	&= \biggl(-\frac{2(2k-4)!}{(k-2)!}\,(1-4x)^{3/2-k}\biggr)' \\[1mm]
	&= -\frac{2(2k-4)!}{(k-2)!}\cdot\biggl(\frac{3}{2}-k\biggr)\cdot(-4)\cdot(1-4x)^{1/2-k} \\[1mm]
	&= -\frac{2(2k-4)!}{(k-2)!}\cdot\frac{(2k-3)(2k-2)}{(k-1)}\cdot(1-4x)^{1/2-k} \\[1mm]
	&= -\frac{2(2k-2)!}{(k-1)!}\,(1-4x)^{1/2-k}. \qedhere
\end{align*}
\end{proof}

Rozwinięcie Taylora funkcji $f(x)=\sqrt{1-4x}$ wokół punktu $x=0$ ma postać:
\[
	f(x) = \sum_{k=0}^\infty\frac{f^{(k)}(0)}{k!}\,x^k = 1-2\sum_{k=1}^\infty\frac{(2k-2)!}{(k-1)!\,k!}\,x^k = 1-2\sum_{k=1}^\infty\binom{2k-2}{k-1}\frac{x^k}{k} = 1-2\sum_{k=0}^\infty\binom{2k}{k}\frac{x^{k+1}}{k+1}.
\]
Podstawiając uzyskany wynik do postaci $B(x)$ wyprowadzonej w~części (b), dostajemy
\[
	B(x) = \frac{1-\sqrt{1-4x}}{2x} = \sum_{n=0}^\infty\binom{2n}{n}\frac{x^n}{n+1}.
\]
co po przyrównaniu do wzoru z~definicji funkcji $B(x)$, daje
\[
	b_n = \frac{1}{n+1}\binom{2n}{n}.
\]

\subproblem %12-4(d)
Na mocy poprzedniego punktu i~z~\refExercise{C.1-13} mamy
\begin{align*}
	b_n &= \frac{1}{n+1}\binom{2n}{n} \\
	&= \frac{1}{n+1}\cdot\frac{2^{2n}}{\sqrt{\pi n}}\,(1+O(1/n)) \\
	&= \frac{n}{n+1}\cdot\frac{4^n}{\sqrt{\pi}\,n^{3/2}}\,(1+O(1/n)) \\
	&= \biggl(1-\frac{1}{n+1}\biggr)\frac{4^n}{\sqrt{\pi}\,n^{3/2}}\,(1+O(1/n)) \\
	&= \frac{4^n}{\sqrt{\pi}\,n^{3/2}}\,(1+O(1/n)).
\end{align*}
Ostatnia równość zachodzi, bo
\[
	\biggl(1-\frac{1}{n+1}\biggr)(1+O(1/n)) = 1-\frac{1}{n+1}+O(1/n)-O(1/n^2) = 1+O(1/n).
\]
