\subchapter{Wyszukiwanie w~drzewie wyszukiwań binarnych}

\exercise %12.2-1
Ciągi z~przykładów (a), (b) i~(d) są możliwe do uzyskania podczas wyszukiwania klucza 363 w~drzewie BST.
W~przykładzie (c) po odwiedzeniu węzła o~kluczu 911 przechodzimy do lewego poddrzewa, w~którym znajdują się węzły o~kluczach nie większych niż 911.
Jednakże później napotykamy 912, a~to nie jest możliwe w~drzewie BST.
W~przykładzie (e) występuje podobna sytuacja -- po odwiedzeniu węzła o~kluczu 347 przetwarzane jest prawe poddrzewo.
Klucz 299 napotykany później jest jednak mniejszy niż 347, co w~drzewie BST nie może mieć miejsca.

\exercise %12.2-2
Rekurencyjne wersje procedur zostały przedstawione poniżej.
\begin{codebox}
\Procname{$\proc{Recursive-Tree-Minimum}(x)$}
\li	\If $\attrib{x}{left}\ne\const{nil}$
\li		\Then \Return $\proc{Recursive-Tree-Minimum}(\attrib{x}{left})$
\li		\Else \Return $x$
		\End
\end{codebox}
\begin{codebox}
\Procname{$\proc{Recursive-Tree-Maximum}(x)$}
\li	\If $\attrib{x}{right}\ne\const{nil}$
\li		\Then \Return $\proc{Recursive-Tree-Maximum}(\attrib{x}{right})$
\li		\Else \Return $x$
		\End
\end{codebox}

\exercise %12.2-3
Procedura \proc{Tree-Predecessor} jest symetryczna do \proc{Tree-Successor}.
Jej pseudokod znajduje się poniżej.
\begin{codebox}
\Procname{$\proc{Tree-Predecessor}(x)$}
\li	\If $\attrib{x}{left}\ne\const{nil}$
\li		\Then \Return $\proc{Tree-Maximum}(\attrib{x}{left})$
		\End
\li	$y\gets\attrib{x}{p}$
\li	\While $y\ne\const{nil}$ i~$x=\attrib{y}{left}$
\li		\Do $x\gets y$
\li			$y\gets\attrib{y}{p}$
		\End
\li	\Return $y$
\end{codebox}

\exercise %12.2-4
Rys.\ \ref{fig:12.2-4} przedstawia najmniejsze (w~sensie liczby węzłów) drzewo BST, które obala postawioną hipotezę.
\begin{figure}[!ht]
	\centering \begin{tikzpicture}[
	every node/.style = tree node,
	marked node/.style = {fill=black!40}
]
\node[marked node] {1}
	child[missing]
	child {node[marked node] {3}
		child {node {2}}
		child {node[marked node] {4}}
	};
\end{tikzpicture}

	\caption{Najmniejszy kontrprzykład dla rozumowania profesora Bunyana.
W~drzewie tym szukano klucza $k=4$.
Jasnym kolorem zaznaczono jedyny węzeł o~kluczu należącym do zbioru $A$, a~ciemnym kolorem -- węzły o~kluczach ze zbioru $B$.
Zbiór $C$ w~tym przykładzie jest pusty.
Klucze 1 i~2 nie spełniają nierówności postawionej w~hipotezie.} \label{fig:12.2-4}
\end{figure}

\exercise %12.2-5
Niech $x$ będzie węzłem o~dwóch synach w~drzewie BST.
Następnik $y$ węzła $x$, zgodnie z~definicją, jest węzłem występującym bezpośrednio po $x$ w~porządku inorder wszystkich węzłów drzewa -- należy on zatem do prawego poddrzewa $x$.
Załóżmy teraz, że $z$ jest lewym synem $y$.
W~porządku inorder pojawi się on przed $y$, ale po $x$, bo też znajduje się w~prawym poddrzewie $x$.
Istnienie węzła $z$ stoi jednak w~sprzeczności z~faktem, że $y$ jest następnikiem $x$.

Analogiczne rozumowanie przeprowadza się dla poprzednika węzła $x$.

\exercise %12.2-6
Wpierw udowodnimy, że $y$ musi być przodkiem $x$.
Załóżmy więc, że $y$ nie jest przodkiem $x$ i~rozważmy węzeł $z$ -- najniższy wspólny przodek $x$ i~$y$.
Z~własności drzew wyszukiwań binarnych zachodzi wtedy $\attrib{x}{key}<\attrib{z}{key}<\attrib{y}{key}$, co stoi w~sprzeczności z~założeniem, że $y$ jest następnikiem $x$.

Zauważmy teraz, że \attrib{y}{left} musi być przodkiem $x$, gdyż w~przeciwnym razie, to \attrib{y}{right} byłby przodkiem $x$, skąd byłoby $\attrib{x}{key}>\attrib{y}{key}$.
W~końcu załóżmy, że to nie $y$, a~inny węzeł $z$ jest najniższym przodkiem węzła $x$, którego lewy syn jest także przodkiem $x$.
Wówczas $z$ należy do lewego poddrzewa $y$, skąd $\attrib{z}{key}<\attrib{y}{key}$, co przeczy temu, że $y$ jest następnikiem $x$.

\exercise %12.2-7
Niech $h$ będzie wysokością drzewa.
Wywołanie \proc{Tree-Minimum} zajmuje oczywiście czas $O(h)$.
Z~\refExercise{12.2-8} mamy, że kolejne $n-1$ wywołań \proc{Tree-Successor} potrzebuje czasu $O(n+h)$.
Oczywiście $h=O(n)$, zatem algorytm ten działa w~czasie $O(n)$.
Dolne oszacowanie $\Omega(n)$ wynika z~faktu, że algorytm musi odwiedzić każdy z~$n$ węzłów drzewa.

\exercise %12.2-8
Czas działania tego ciągu operacji jest proporcjonalny do liczby przejść po krawędziach drzewa, dlatego skupimy się na znalezieniu oszacowania dla tej wielkości.

W~trakcie wyszukiwania $k$ kolejnych następników dowolnego węzła w~drzewie o~wysokości $h$, każda krawędź będzie pokonywana co najwyżej dwukrotnie -- co najwyżej raz podczas przechodzenia w~dół drzewa i~co najwyżej raz podczas przechodzenia w~górę drzewa.
Przejście krawędzią $\langle u,v\rangle$ w~dół rozpoczyna wyszukiwanie następników w~poddrzewie o~korzeniu w~$v$.
Węzły tego poddrzewa tworzą spójny fragment w~porządku inorder, a~więc zanim krawędź $\langle u,v\rangle$ zostanie pokonana ponownie w~drodze w~górę drzewa, kolejne wywołania \proc{Tree-Successor} zwrócą wszystkie węzły poddrzewa o~korzeniu w~$v$, po czym krawędź ta nie będzie już wykorzystywana.
Liczba krawędzi pokonywanych dwukrotnie wynosi więc co najwyżej $k$, zatem wykonanych zostanie co najwyżej $2k$ przejść po tych krawędziach.
Z~kolei liczba krawędzi odwiedzanych jednokrotnie nie przekracza $2h$, gdyż $2h$ jest maksymalną długością ścieżki prostej w~drzewie.
Jeśli po pokonaniu takiej ścieżki należy przejść do kolejnego następnika, to w~tym celu wykorzystywane są już odwiedzone krawędzie, pokonywane po raz drugi.

Zatem łączna liczba przejść po krawędziach drzewa jest ograniczona przez $2k+2h=O(k+h)$.

\exercise %12.2-9
\note{Wymagane jest założenie, że wszystkie klucze w~drzewie $T$ są parami różne.}

\noindent Dowód sprowadza się do pokazania, że $y$ jest albo następnikiem albo poprzednikiem węzła $x$.

Załóżmy, że $x$ jest lewym synem węzła $y$.
Zauważmy, że w~drzewie $T$ istnieje następnik węzła $x$, ponieważ $x$ nie jest skrajnie prawym węzłem drzewa $T$.
Możemy więc skorzystać z~\refExercise{12.2-6}, z~którego wynika, że następnikiem $x$ jest w~rzeczywistości $y$.
Korzystając z~analogicznego rozumowania z~rozwiązania tamtego zadania, można udowodnić symetryczne twierdzenie charakteryzujące poprzednika węzła o~pustym lewym poddrzewie.
W~przypadku, gdy $x$ jest prawym synem $y$, na podstawie twierdzenia symetrycznego do tego z~\refExercise{12.2-6} pokazujemy, że $y$ jest poprzednikiem $x$.
