\subchapter{Wyszukiwanie w~drzewie wyszukiwań binarnych}

\exercise %12.2-1
Ciągi z~przykładów (a), (b) i~(d) są możliwe do uzyskania podczas wyszukiwania klucza 363 w~drzewie BST.
W~przykładzie (c) po odwiedzeniu węzła o~kluczu 911 przechodzimy do lewego poddrzewa, w~którym znajdują się węzły o~kluczach nie większych niż 911.
Jednakże później napotykamy 912, a~to nie jest możliwe w~drzewie BST.
W~przykładzie (e) występuje podobna sytuacja -- po odwiedzeniu węzła o~kluczu 347 przetwarzane jest prawe poddrzewo.
Klucz 299 napotykany później jest jednak mniejszy niż 347, co w~drzewie BST nie może mieć miejsca.

\exercise %12.2-2
Rekurencyjne wersje procedur zostały przedstawione poniżej.
\begin{codebox}
\Procname{$\proc{Recursive-Tree-Minimum}(x)$}
\li	\If $\attrib{x}{left}\ne\const{nil}$
\li		\Then \Return $\proc{Recursive-Tree-Minimum}(\attrib{x}{left})$
\li		\Else \Return $x$
		\End
\end{codebox}
\begin{codebox}
\Procname{$\proc{Recursive-Tree-Maximum}(x)$}
\li	\If $\attrib{x}{right}\ne\const{nil}$
\li		\Then \Return $\proc{Recursive-Tree-Maximum}(\attrib{x}{right})$
\li		\Else \Return $x$
		\End
\end{codebox}

\exercise %12.2-3
Procedura \proc{Tree-Predecessor} jest symetryczna do \proc{Tree-Successor}.
Jej pseudokod znajduje się poniżej.
\begin{codebox}
\Procname{$\proc{Tree-Predecessor}(x)$}
\li	\If $\attrib{x}{left}\ne\const{nil}$
\li		\Then \Return $\proc{Tree-Maximum}(\attrib{x}{left})$
		\End
\li	$y\gets\attrib{x}{p}$
\li	\While $y\ne\const{nil}$ i~$x=\attrib{y}{left}$
\li		\Do $x\gets y$
\li			$y\gets\attrib{y}{p}$
		\End
\li	\Return $y$
\end{codebox}

\exercise %12.2-4
Rys.\ \ref{fig:12.2-4} przedstawia najmniejsze drzewo BST, które obala postawioną hipotezę.
\begin{figure}[!ht]
	\centering \begin{tikzpicture}[
	every node/.style = tree node,
	marked node/.style = {fill=black!40}
]
\node[marked node] {1}
	child[missing]
	child {node[marked node] {3}
		child {node {2}}
		child {node[marked node] {4}}
	};
\end{tikzpicture}

	\caption{Najmniejszy kontrprzykład dla rozumowania profesora Bunyana.
W~drzewie tym szukano klucza $k=4$.
Jasnym kolorem zaznaczono jedyny węzeł o~kluczu należącym do zbioru $A$, a~ciemnym kolorem -- węzły o~kluczach ze zbioru $B$.
Zbiór $C$ w~tym przykładzie jest pusty.
Klucze 1 i~2 nie spełniają nierówności postawionej w~hipotezie.} \label{fig:12.2-4}
\end{figure}

\exercise %12.2-5
Niech $x$ będzie węzłem o~dwóch synach w~drzewie BST.
Następnik $y$ węzła $x$, zgodnie z~definicją, jest węzłem występującym bezpośrednio po $x$ w~porządku inorder wszystkich węzłów drzewa -- należy on zatem do prawego poddrzewa $x$.
Załóżmy teraz, że $z$ jest lewym synem $y$.
W~porządku inorder pojawi się on przed $y$, ale po $x$, bo też znajduje się w~prawym poddrzewie $x$.
Istnienie węzła $z$ stoi jednak w~sprzeczności z~faktem, że $y$ jest następnikiem $x$.

Analogiczne rozumowanie przeprowadza się dla poprzednika węzła $x$.

\exercise %12.2-6
Udowodnimy to stwierdzenie bez użycia założenia o~unikalności kluczy w~drzewie.

Oczywiście następnik węzła $x$ nie należy do jego lewego poddrzewa.
Nie jest nim także żaden przodek $x$, którego prawe poddrzewo zawiera $x$, ani żaden węzeł z~lewego poddrzewa takiego przodka.
Wszystkie te węzły bowiem występują w~porządku inorder przed $x$.
Następnika $x$ należy więc szukać wśród tych przodków $x$, które zawierają $x$ w~swoich lewych poddrzewach.
Niech $y_1$, $y_2$ będą dwoma takimi przodkami $x$ i~przyjmijmy, że $y_1$ zajmuje głębszy poziom od $y_2$.
Wynika stąd, że $y_1$ należy do lewego poddrzewa $y_2$, a~więc $y_1$ występuje pomiędzy $x$ a~$y_2$ w~porządku inorder.
Wnioskujemy zatem, że następnikiem węzła $x$ jest najniższy przodek $x$ mający $x$ w swoim lewym poddrzewie, a~to jest równoważne temu, że jego lewy syn także jest przodkiem $x$.

\exercise %12.2-7
Wywołanie \proc{Tree-Minimum}, po którym następuje $n-1$ wywołań \proc{Tree-Successor}, spowoduje oczywiście wypisanie kluczy wszystkich $n$ węzłów drzewa w~tej samej kolejności, co wywołanie na tym drzewie \proc{Inorder-Tree-Walk}.
Jest w~sumie $n$ wywołań procedur -- algorytm wymaga więc czasu $\Omega(n)$.
Pokażemy jeszcze, że każda krawędź drzewa jest pokonywana co najwyżej dwukrotnie, skąd wynika górne oszacowanie $O(n)$ na czas działania algorytmu.

Niech $u$ będzie węzłem drzewa, a~$v$ jego lewym synem.
W~trakcie działania algorytmu przed wypisaniem klucza $u$ wpierw wypisane zostaną klucze wszystkich węzłów z~jego lewego poddrzewa, którego korzeniem jest $v$, co zostanie zapoczątkowane przez przejście krawędzią $\langle u,v\rangle$.
Następnie w~celu wypisania klucza $u$ procedura \proc{Tree-Successor} przejdzie w~górę drzewa ścieżką do $u$ od maksymalnego elementu poddrzewa o~korzeniu w~$v$.
Na tej ścieżce oczywiście znajduje się krawędź $\langle u,v\rangle$.
Po pokonaniu ścieżki krawędź ta nie zostanie ponownie wykorzystana, ponieważ wszystkie węzły w~lewym poddrzewie $u$ zostały już odwiedzone.

Załóżmy teraz, że $v$ jest prawym synem $u$.
Tuż po wypisaniu klucza $u$ wywoływane jest $\proc{Tree-Successor}(u)$.
Aby dostać się do następnika węzła $u$ znajdującego się w~jego prawym poddrzewie, algorytm musi zejść krawędzią $\langle u,v\rangle$.
Przejście nią w~drodze powrotnej nastąpi po odwiedzeniu wszystkich węzłów z~prawego poddrzewa $u$, o~ile będą wówczas jeszcze nieodwiedzone węzły w~drzewie.
Będzie to miało miejsce podczas wywołania \proc{Tree-Successor} dla węzła o~maksymalnym kluczu w~prawym poddrzewie $u$, kiedy to algorytm będzie przechodził w~górę drzewa do węzła następującego po $u$ w~porządku inorder.
Krawędź $\langle u,v\rangle$ należy do tej ścieżki i~z~racji tego, że wszystkie węzły prawego poddrzewa $u$ zostały odwiedzone, nie zostanie już użyta ponownie.

\exercise %12.2-8
Niech $x$ będzie początkowym węzłem, a~$y$ -- jego \singledash{$k$}{tym} następnikiem.
Z~rozwiązania poprzedniego zadania mamy, że każda krawędź drzewa pokonywana jest w~wywołaniach \proc{Tree-Successor} co najwyżej dwukrotnie.
A~zatem każdy z~$k$ węzłów od $x$ do $y$ w~porządku inorder będzie odwiedzany w~serii wywołań \proc{Tree-Successor} co najwyżej 3 razy.
Oznaczmy przez $z$ najniższego wspólnego przodka węzłów $x$ i~$y$.
W~ciągu wywołań \proc{Tree-Successor} badane są także węzły będące poprzednikami $x$ i~węzły będące następnikami $y$.
Wszystkie one występują na prostej ścieżce z~$x$ do $z$ lub na prostej ścieżce z~$z$ do $y$, dlatego ich liczbę można ograniczyć od góry przez $2h$.
Górne ograniczenie na czas działania danego ciągu wywołań wynosi zatem $3k+2h=O(k+h)$.

\exercise %12.2-9
\note{Wymagane jest założenie, że wszystkie klucze w~drzewie $T$ są parami różne.}

\noindent Dowód sprowadza się do pokazania, że $y$ jest albo następnikiem albo poprzednikiem węzła $x$.

Załóżmy, że $x$ jest lewym synem węzła $y$.
Zauważmy, że w~drzewie $T$ istnieje następnik węzła $x$, ponieważ $x$ nie jest skrajnie prawym węzłem drzewa $T$.
Możemy więc skorzystać z~\refExercise{12.2-6}, z~którego wynika, że następnikiem $x$ jest w~rzeczywistości $y$.
Korzystając z~analogicznego rozumowania z~rozwiązania tamtego zadania, można udowodnić symetryczne twierdzenie charakteryzujące poprzednika węzła o~pustym lewym poddrzewie.
W~przypadku, gdy $x$ jest prawym synem $y$, na podstawie twierdzenia symetrycznego do tego z~\refExercise{12.2-6} pokazujemy, że $y$ jest poprzednikiem $x$.
