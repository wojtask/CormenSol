\subchapter{Co to jest drzewo wyszukiwań binarnych?}

\exercise %12.1-1
Przykładowe drzewa BST przedstawiono na rys.\ \ref{fig:12.1-1}.
\begin{figure}[!ht]
	\centering \begin{tikzpicture}[
	level 4/.style = {sibling distance=5mm},
	level 5/.style = {sibling distance=4mm},
	level 6/.style = {sibling distance=3.5mm}
]

\node[outer] (pic a) {
\begin{tikzpicture}[
	every node/.style = {tree node},
	anchor = center
]
	\node {10}
		child {node {4}
			child {node {1}}
			child {node {5}}
		}
		child {node {17}
			child {node {16}}
			child {node {21}}
		};
\end{tikzpicture}
};

\node[outer, right=18mm of pic a.north east, anchor=north west] (pic b) {
\begin{tikzpicture}[
	every node/.style = {tree node},
	anchor = center
]
	\node {17}
		child {node {4}
			child {node {1}}
			child {node {10}
				child {node {5}}
				child {node {16}}
			}
		}
		child {node {21}};
\end{tikzpicture}
};

\node[outer, below=22mm of pic a.south west, anchor=north west] (pic c) {
\begin{tikzpicture}[
	every node/.style = tree node,
	anchor = center
]
	\node {16}
		child {node {10}
			child {node {1}
				child[missing]
				child {node {5}
					child {node {4}}
					child[missing]
				}
			}
			child[missing]
		}
		child {node {21}
			child {node {17}}
			child[missing]
		};
\end{tikzpicture}
};

\node[outer, right=of pic c.north east, anchor=north west] (pic d) {
\begin{tikzpicture}[
	every node/.style = tree node,
	anchor = center
]
	\node {1}
		child[missing]
		child {node {4}
			child[missing]
			child {node {21}
				child {node {10}
					child {node {5}}
					child {node {17}
						child {node {16}}
						child[missing]
					}
				}
				child[missing]
			}
		};
\end{tikzpicture}
};

\node[outer, right=of pic d.north east, anchor=north west] (pic e) {
\begin{tikzpicture}[
	every node/.style = tree node,
	anchor = center
]
	\node {1}
		child[missing]
		child {node {21}
			child {node {17}
				child {node {16}
					child {node {5}
						child[missing]
						child {node {10}
							child {node {4}}
							child[missing]
						}
					}
					child[missing]
				}
				child[missing]
			}
			child[missing]
		};
\end{tikzpicture}
};

\foreach \x in {a,b,c,d,e} {
	\node[subpicture label, below=2mm of pic \x] {(\x)};
}

\end{tikzpicture}

	\caption{Drzewa BST o~różnych wysokościach zawierające zbiór kluczy $\{1,4,5,10,16,17,21\}$.} \label{fig:12.1-1}
\end{figure}

\exercise %12.1-2
\note{Wyrażenie ,,właściwa kolejność'' użyte w~treści zadania oznacza kolejność niemalejącą.}

\noindent Niech $x$ będzie węzłem drzewa $T$, a~$y$, $z$, odpowiednio, lewym i~prawym synem $x$.
Jeśli drzewo $T$ byłoby drzewem BST, czyli spełniałoby własność drzewa BST, to wówczas prawdziwe byłyby nierówności $\attrib{y}{key}\le\attrib{x}{key}\le\attrib{z}{key}$.
Jeśli z~kolei drzewo $T$ stanowiłoby kopiec typu min, czyli spełniałoby własność kopca typu min, to zachodziłoby wtedy $\attrib{x}{key}\le\attrib{y}{key}$ oraz $\attrib{x}{key}\le\attrib{z}{key}$.
Ta różnica sprawia, że podczas przechodzenia kopca typu min w~kolejności niemalejących węzłów, po odwiedzeniu węzła $x$ nie jest wiadomo, które jego poddrzewo należałoby następnie odwiedzić, ponieważ nie jest znana relacja między $y$ a~$z$.
Oznacza to, że wypisanie wszystkich $n$ węzłów kopca typu min w~kolejności niemalejącej zajmuje czas wyższy niż $O(n)$.
W~przeciwnym przypadku moglibyśmy zaimplementować algorytm heapsort w~czasie liniowym.

\exercise %12.1-3
Nierekurencyjna wersja algorytmu przechodzenia drzewa metodą inorder, która wykorzystuje stos, została przedstawiona w~\refExercise{10.4-3}.
Nierekurencyjny algorytm o~identycznej funkcjonalności, ale niewykorzystujący stosu, został opisany w~\refExercise{10.4-5}.

\exercise %12.1-4
Algorytmy przechodzenia drzewa metodami preorder i~postorder przedstawiono poniżej.

\begin{codebox}
\Procname{$\proc{Preorder-Tree-Walk}(x)$}
\li	\If $x\ne\const{nil}$
\li		\Then wypisz \attrib{x}{key}
\li			$\proc{Preorder-Tree-Walk}(\attrib{x}{left})$
\li			$\proc{Preorder-Tree-Walk}(\attrib{x}{right})$
		\End
\end{codebox}

\begin{codebox}
\Procname{$\proc{Postorder-Tree-Walk}(x)$}
\li	\If $x\ne\const{nil}$
\li		\Then $\proc{Postorder-Tree-Walk}(\attrib{x}{left})$
\li			$\proc{Postorder-Tree-Walk}(\attrib{x}{right})$
\li			wypisz \attrib{x}{key}
		\End
\end{codebox}

Dla każdego z~tych algorytmów można udowodnić twierdzenie analogiczne do tw.\ 12.1.
Jeśli więc $T$ jest drzewem $n$\nbhyphen wierzchołkowym, to oba wywołania $\proc{Preorder-Tree-Walk}(\attrib{T}{root})$ oraz $\proc{Postorder-Tree-Walk}(\attrib{T}{root})$ działają w~czasie $\Theta(n)$.

\exercise %12.1-5
Załóżmy nie wprost, że istnieje algorytm konstruowania drzewa BST z~dowolnej listy $n$ elementów za pomocą porównań, którego pesymistyczny czas działania jest rzędu $o(n\lg n)$.
Na drzewie BST utworzonym za pomocą tego algorytmu moglibyśmy następnie uruchomić algorytm przechodzenia drzewa metodą inorder i~wypisać wszystkie jego klucze w~kolejności niemalejącej w~czasie $\Theta(n)$.
Posortowanie wejściowej listy $n$ elementów można byłoby wtedy przeprowadzić w~czasie niższym niż liniowo-logarytmiczny, co przeczyłoby tw.\ 8.1.
