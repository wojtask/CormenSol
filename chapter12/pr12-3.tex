\problem{Średnia głębokość węzła w~losowo zbudowanym drzewie wyszukiwań binarnych} %12-3
\note{Przyjmiemy, że zapis $x\in T$ oznacza, że węzeł\/ $x$ znajduje się w~drzewie\/ $T$.
Notacji tej nie zdefiniowano wyraźnie w~Podręczniku, także w~wersji oryginalnej.}
\bignegskip

\subproblem %12-3(a)
Wzór wynika wprost z~definicji średniej głębokości węzła w~drzewie $T$ i~z~definicji $P(T)$.

\subproblem %12-3(b)
Zauważmy, że dla każdego $x\in T_L$ zachodzi $d(x,T_L)=d(x,T)-1$ i~podobnie, dla każdego $x\in T_R$ zachodzi $d(x,T_R)=d(x,T)-1$.
Stąd, jeśli $r$ jest korzeniem drzewa $T$, to
\begin{align*}
	P(T) &= \sum_{x\in T}d(x,T) \\
	&= d(r,T)+\sum_{x\in T_L}d(x,T)+\sum_{x\in T_R}d(x,T) \\[1mm]
	&= \sum_{x\in T_L}(d(x,T_L)+1)+\sum_{x\in T_R}(d(x,T_R)+1) \\[1mm]
	&= \sum_{x\in T_L}d(x,T_L)+\sum_{x\in T_R}d(x,T_R)+n-1 \\[1mm]
	&= P(T_L)+P(T_R)+n-1.
\end{align*}

\subproblem %12-3(c)
Jeśli $T$ jest losowo skonstruowanym drzewem wyszukiwań binarnych o~$n$ węzłach, to klucz pierwszego wstawianego do $T$ węzła jest z~jednakowym prawdopodobieństwem dowolnym w~kolejności rosnącej w~ciągu $n$ kluczy wstawianych węzłów.
Pierwszy węzeł staje się korzeniem drzewa $T$, a~więc po zbudowaniu $T$ jego lewe poddrzewo $T_L$ może z~jednakowym prawdopodobieństwem być drzewem o~$i$ węzłach, gdzie $i=0$, 1, \dots, $n-1$.
Dla ustalonego $i$ prawe poddrzewo $T_R$ ma wówczas rozmiar $n-i-1$.
Z~definicji $P(n)$ jako średniej wartości $P(T)$ oraz z~punktu (b) otrzymujemy
\[
	P(n) = \frac{1}{n}\sum_{i=0}^{n-1}(P(i)+P(n-i-1)+n-1).
\]

\subproblem %12-3(d)
Na podstawie poprzedniego punktu mamy:
\begin{align*}
	P(n) &= \frac{1}{n}\sum_{k=0}^{n-1}(P(k)+P(n-k-1)+n-1) \\
	&= \frac{1}{n}\biggl(\sum_{k=0}^{n-1}P(k)+\sum_{k=0}^{n-1}P(n-k-1)+\sum_{k=0}^{n-1}(n-1)\biggr) \\
	&= \frac{1}{n}\biggl(2\sum_{k=0}^{n-1}P(k)+n(n-1)\biggr) \\
	&= \frac{2}{n}\sum_{k=1}^{n-1}P(k)+\Theta(n).
\end{align*}
W~ostatnim przekształceniu skorzystaliśmy z~tego, że $P(0)=0$, i~opuściliśmy ten składnik sumy.

\subproblem %12-3(e)
Podobnie jak w~problemie \refProblem{7-2} dla $\E(T(n))$, tu pokażemy, że $P(n)\le an\lg n$ dla pewnej dodatniej stałej $a$.
Podstawą indukcji niech będzie $n=1$.
Oczywiście $P(1)=0$ oraz $a\cdot1\cdot\lg1=0$, więc podstawa jest spełniona dla każdego $a$.
Niech teraz $n>1$.
Przyjmijmy założenie, że dla dowolnego $k=1$, 2, \dots, $n-1$ zachodzi $P(k)\le ak\lg k$.
Wówczas:
\[
	P(n) = \frac{2}{n}\sum_{k=1}^{n-1}P(k)+\Theta(n) \le \frac{2}{n}\sum_{k=1}^{n-1}ak\lg k+\Theta(n) = \frac{2a}{n}\sum_{k=1}^{n-1}{k\lg k}+\Theta(n).
\]
Posługując się nierównością (7.7) i~dobierając odpowiednio duże $a$ tak, aby wyrażenie $an/4$ było równe co najmniej składnikowi $\Theta(n)$, mamy
\[
	P(n) \le \frac{2a}{n}\biggl(\frac{n^2\lg n}{2}-\frac{n^2}{8}\biggr)+\Theta(n) = an\lg n-\frac{an}{4}+\Theta(n) \le an\lg n = O(n\lg n).
\]

\subproblem %12-3(f)
Zauważmy, że po tym, jak węzeł $x$ zostaje wybrany jako korzeń drzewa $T$, każdy kolejny węzeł wstawiany do $T$ jest porównywany z~$x$.
Analogicznie, po tym, jak klucz $y$ zostaje wybrany jako element rozdzielający w~tablicy $A$, każdy inny klucz z~tablicy $A$ porównywany jest z~$y$.
Opisaną analogię możemy powtórzyć dla lewego poddrzewa $T$ i~podtablicy $A$ zawierającej elementy mniejsze od $y$ oraz dla prawego poddrzewa $T$ i~podtablicy $A$ zawierającej elementy większe bądź równe $y$.
Aby jednak klucze równe elementowi rozdzielającemu trafiały do drugiej podtablicy, musimy w~wierszu 4 procedury \proc{Partition} zmienić nierówność na ostrą.
Ponadto podczas budowy losowo skonstruowanego drzewa wyszukiwań binarnych korzeniem staje się pierwszy węzeł z~permutacji wejściowej.
Implementacja quicksorta o~identycznych porównaniach elementów powinna zatem wybierać na element rozdzielający pierwszy element tablicy wejściowej.
