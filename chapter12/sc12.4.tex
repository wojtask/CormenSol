\subchapter{Losowo skonstruowane drzewa wyszukiwań binarnych}

\exercise %12.4-1
Wzór udowodnimy przez indukcję względem $n$.
Jeśli $n=1$, to po obu stronach znaku równości jest 1.
Załóżmy teraz, że $n>1$ i~że spełnione jest założenie indukcyjne
\[
	\sum_{i=0}^{n-2}\binom{i+3}{3} = \binom{n+2}{4}.
\]
Mamy
\[
	\sum_{i=0}^{n-1}\binom{i+3}{3} = \sum_{i=0}^{n-2}\binom{i+3}{3}+\binom{n+2}{3} = \binom{n+2}{4}+\binom{n+2}{3} = \binom{n+3}{4},
\]
przy czym w~ostatniej równości wykorzystaliśmy \refExercise{C.1-7}.

\exercise %12.4-2
Rozważmy drzewo binarne o~$n\ge16$ węzłach, w~którym początkowe poziomy stanowią pełne drzewo binarne składające się z~$n-\sqrt{n\lg n}$ węzłów (dla zwiększenia czytelności pomijamy branie części całkowitych), natomiast pozostałe $\sqrt{n\lg n}$ węzłów znajduje się na ścieżce w~dół od jednego z~węzłów pełnego drzewa i~rozciąga się na $\sqrt{n\lg n}$ poziomów.
Drzewo takie ma wysokość
\[
	\Theta\bigl(\lg\bigl(n-\sqrt{n\lg n}\bigr)\bigr)+\sqrt{n\lg n} = \Theta\bigl(\!\sqrt{n\lg n}\bigr) = \omega(\lg n).
\]

O~tym, że średnia głębokość węzła w~tym drzewie wynosi $\Theta(\lg n)$, przekonamy się, wyprowadzając najpierw górne, a~potem dolne oszacowanie tej wartości.
W~oszacowaniu górnym użyjemy $O(\lg n)$ jako ograniczenia na głębokość każdego z~$n-\sqrt{n\lg n}$ węzłów z~części stanowiącej pełne drzewo binarne oraz $O\bigl(\lg n+\sqrt{n\lg n}\bigr)$ jako ograniczenia na głębokość każdego z~pozostałych $\sqrt{n\lg n}$ węzłów.
Średnia głębokość węzła w~drzewie wynosi więc co najwyżej
\[
	\frac{1}{n}\cdot O\bigl(\bigl(n-\sqrt{n\lg n}\bigr)\lg n+\sqrt{n\lg n}\,\bigl(\lg n+\sqrt{n\lg n}\bigr)\bigr) = \frac{1}{n}\cdot O(n\lg n) = O(\lg n).
\]
Z~kolei zauważmy, że najniższy poziom pełnego drzewa binarnego składa się z~$n-\sqrt{n\lg n}$ węzłów, z~których każdy ma głębokość $\Theta\bigl(\lg\bigl(n-\sqrt{n\lg n}\bigr)\bigr)$.
Dla $n\ge16$ zachodzi $\sqrt{n\lg n}\le n/2$, dlatego liczbę węzłów w~pełnym drzewie binarnym można ograniczyć przez $n-\sqrt{n\lg n}\ge n-n/2=\Omega(n)$, a~wysokość każdego z~nich przez $\Omega(\lg n)$.
Stąd średnia głębokość węzła ograniczona jest z~dołu przez
\[
	\frac{1}{n}\cdot\Omega(n\lg n) = \Omega(\lg n).
\]

W~drugiej części zadania udowodnimy, że wysokość drzewa binarnego o~$n$ węzłach i~średniej głębokości węzła $\Theta(\lg n)$ wynosi $O\bigl(\!\sqrt{n\lg n}\bigr)$.
Niech dane będzie drzewo binarne o~$n$ węzłach i~wysokości $h$ ze średnią głębokością węzła $\Theta(\lg n)$.
Istnieje więc ścieżka od korzenia w~dół tego drzewa, na której głębokościami węzłów są kolejno 0, 1, \dots, $h$.
Oznaczmy przez $U$ zbiór węzłów na jednej z~takich ścieżek, przez $U'$ zbiór wszystkich pozostałych węzłów drzewa, a~przez $d(x)$ głębokość węzła $x$.
Wówczas średnia głębokość węzła w~tym drzewie wynosi
\[
	\frac{1}{n}\biggl(\sum_{x\in U}d(x)+\sum_{x\in U'}d(x)\biggr) \ge \frac{1}{n}\sum_{x\in U}d(x) = \frac{1}{n}\sum_{i=0}^hi = \frac{1}{n}\cdot\Theta(h^2).
\]
Jeśli byłoby $h=\omega\bigl(\!\sqrt{n\lg n}\bigr)$, to wtedy mielibyśmy
\[
	\frac{1}{n}\cdot\Theta(h^2) = \frac{1}{n}\cdot\omega(n\lg n) = \omega(\lg n),
\]
co stoi w~sprzeczności z~założeniem, że średnią głębokością węzła jest $\Theta(\lg n)$.
A~zatem wysokość drzewa musi być ograniczona przez $O\bigl(\!\sqrt{n\lg n}\bigr)$.

\exercise %12.4-3
Istnieje 5 różnych drzew BST przechowujących zbiór kluczy $\{1,2,3\}$, podczas gdy permutacji tego zbioru jest $3!=6$.
Jedno z~tych drzew można więc zbudować z~więcej niż jednej permutacji.
Rys.~\ref{fig:12.4-3} ilustruje każde drzewo, podając permutacje, z~których one powstają.
\begin{figure}[!ht]
	\centering \begin{tikzpicture}

\node[outer] (pic a) {
\begin{tikzpicture}[
	every node/.style = tree node,
	anchor = center
]
	\node {1}
		child[missing]
		child {node {2}
			child[missing]
			child {node {3}}
		};
\end{tikzpicture}
};

\node[outer, right=of pic a] (pic b) {
\begin{tikzpicture}[
	every node/.style = tree node,
	anchor = center
]
	\node {1}
		child[missing]
		child {node {3}
			child {node {2}}
			child[missing]
		};
\end{tikzpicture}
};

\node[outer, right=of pic b] (pic c) {
\begin{tikzpicture}[
	every node/.style = tree node,
	anchor = center
]
	\node {2}
		child {node {1}}
		child {node {3}};
\end{tikzpicture}
};

\node[outer, right=of pic c] (pic d) {
\begin{tikzpicture}[
	every node/.style = tree node,
	anchor = center
]
	\node {3}
		child {node {1}
			child[missing]
			child {node {2}}
		}
		child[missing];
\end{tikzpicture}
};

\node[outer, right=of pic d] (pic e) {
\begin{tikzpicture}[
	every node/.style = tree node,
	anchor = center
]
	\node {3}
		child {node {2}
			child {node {1}}
			child[missing]
		}
		child[missing];
\end{tikzpicture}
};

\node[subpicture label, below=2mm of pic a] (label a) {(a)};
\foreach \x in {b,c,d,e} {
	\node[subpicture label] at (label a -| pic \x) {(\x)};
}

\end{tikzpicture}

	\caption{Drzewa BST powstałe w~wyniku wstawienia kluczy 1, 2, 3 w~kolejności wyznaczonej przez permutację {\sffamily\bfseries(a)} $\langle1,2,3\rangle$,
{\sffamily\bfseries(b)} $\langle1,3,2\rangle$,
{\sffamily\bfseries(c)} $\langle2,1,3\rangle$ oraz $\langle2,3,1\rangle$, {\sffamily\bfseries(d)} $\langle3,1,2\rangle$, {\sffamily\bfseries(e)} $\langle3,2,1\rangle$.} \label{fig:12.4-3}
\end{figure}

\exercise %12.4-4
Udowodnimy silniejsze twierdzenie, że każda funkcja $f(x)=c^x$, gdzie $c$ jest stałą dodatnią, jest wypukła.
Z~definicji wypukłości funkcji z~dodatku C musimy pokazać, że dla każdych $x$, $y$ i~dla każdego $0\le\lambda\le1$ prawdziwa jest nierówność
\[
	c^{\lambda x+(1-\lambda)y} \le \lambda c^x+(1-\lambda)c^y.
\]

\medskip
\noindent\textsf{\textbf{Lemat.}} \textit{Dla dowolnych liczb rzeczywistych\/ $a$,\/ $b$ i~dla dowolnej dodatniej liczby rzeczywistej\/ $c$ zachodzi
\[
	c^a \ge c^b+(a-b)c^b\ln c.
\]
}
\begin{proof}
Na podstawie wzoru (3.11), $e^x\ge1+x$ dla dowolnego $x$.
Jeśli przyjmiemy $x=r\ln c$, to $e^x=e^{r\ln c}=(e^{\ln c})^r=c^r$ i~nierówność sprowadza się do postaci $c^r\ge1+r\ln c$.
Po podstawieniu $r=a-b$ dostajemy $c^{a-b}\ge1+(a-b)\ln c$ i~aby otrzymać żądaną nierówność, wystarczy pomnożyć obie strony przez $c^b$.
\end{proof}

Niech $z=\lambda x+(1-\lambda)y$.
Skorzystajmy z~powyższego lematu dwukrotnie, najpierw podstawiając $a=x$ i~$b=z$ i~otrzymując nierówność $c^x\ge c^z+(x-z)c^z\ln c$, a~następnie przyjmując $a=y$ i~$b=z$, skąd dostajemy $c^y\ge c^z+(y-z)c^z\ln c$.
Pierwszą otrzymaną nierówność pomnożymy przez $\lambda$, a~drugą przez $1-\lambda$, a~następnie dodamy do siebie:
\begin{align*}
	\lambda c^x+(1-\lambda)c^y &\ge \lambda(c^z+(x-z)c^z\ln c)+(1-\lambda)(c^z+(y-z)c^z\ln c) \\
	&= \lambda c^z+\lambda xc^z\ln c-\lambda zc^z\ln c+(1-\lambda)c^z+(1-\lambda)yc^z\ln c-(1-\lambda)zc^z\ln c \\
	&= (\lambda+(1-\lambda))c^z+(\lambda x+(1-\lambda)y)c^z\ln c-(\lambda+(1-\lambda))zc^z\ln c \\
	&= c^z+zc^z\ln c-zc^z\ln c \\
	&= c^z \\
	&= c^{\lambda x+(1-\lambda)y}.
\end{align*}
Otrzymany wynik dowodzi wypukłości funkcji $f(x)=c^x$, a~więc w~szczególności $f(x)=2^x$.

\exercise %12.4-5
\note{Wymagane jest założenie, że elementy tablicy wejściowej algorytmu \proc{Randomized-Quicksort} są parami różne.}
