\subchapter{Wstawianie i~usuwanie}
\note{Linia 16 procedury \proc{Tree-Delete} mówi o~kopiowaniu wartości wszystkich pól\/ $y$ do pól\/ $z$, podczas gdy kopiowane powinny być jedynie dodatkowe dane, jakie są przechowywane w~węźle\/ $y$.}
\bignegskip

\exercise %12.3-1
Rekurencyjna wersja operacji wstawiania będzie przyjmować na wejściu korzeń $x$ poddrzewa, do~którego odbywa się wstawianie, oraz nowy węzeł $z$.
Jeśli $x=\const{nil}$, to procedura natychmiast zakończy działanie, zwracając $z$.
W~przeciwnym przypadku będzie schodzić rekurencyjnie w~dół poddrzewa o~korzeniu w~$x$ odpowiednią ścieżką tak, aby na końcu tego procesu uczynić $z$ synem liścia znajdującego się na tej ścieżce.
Następnie, po aktualizacji lewego lub prawego syna $x$ na wynik poprzedniego wywołania rekurencyjnego, zwrócone zostanie $x$.
\begin{codebox}
\Procname{$\proc{Recursive-Tree-Insert}(x,z)$}
\li	\If $x=\const{nil}$
\li		\Then \Return $z$
		\End
\li	\If $\attrib{z}{key}<\attrib{x}{key}$
\li		\Then $\attrib{x}{left}\gets\proc{Recursive-Tree-Insert}(\attrib{x}{left},z)$
\li			$\attribb{x}{left}{p}\gets x$
\li		\Else $\attrib{x}{right}\gets\proc{Recursive-Tree-Insert}(\attrib{x}{right},z)$
\li			$\attribb{x}{right}{p}\gets x$
		\End
\li	\Return $x$
\end{codebox}

Procedura ta nie powinna być używana bezpośrednio, gdyż nie aktualizuje ona pola \id{root} drzewa $T$, do którego wstawiany jest nowy węzeł $z$.
Czynność tę wykonuje poniższy pseudokod, który powinien być wywoływany, aby skorzystać z~procedury \proc{Recursive-Tree-Insert}.
\begin{codebox}
\Procname{$\proc{Recursive-Tree-Insert-Wrapper}(T,z)$}
\li	$\attrib{T}{root}\gets\proc{Recursive-Tree-Insert}(\attrib{T}{root},z)$
\end{codebox}

\exercise %12.3-2
Podczas wstawiania do drzewa nowego węzła $z$ o~kluczu $k$ pokonana zostanie ścieżka od korzenia w~dół drzewa w~celu odnalezienia liścia, który stanie się ojcem węzła $z$.
Z~kolei wyszukiwanie węzła o~kluczu $k$ wiąże się z~przejściem po podobnej ścieżce w~celu jego odnalezienia w~drzewie.
Porównajmy instrukcje odpowiedzialne za poruszanie się po tych ścieżkach w~\proc{Tree-Insert} oraz w~iteracyjnej wersji wyszukiwania, \proc{Iterative-Tree-Search}.
W~pierwszej procedurze są to wiersze \doubledash{3}{7}, a~w~drugiej -- wiersze \doubledash{1}{4}.
W~\proc{Tree-Insert} obecność linii 4 nie wpływa na wybór ścieżki, a~$\attrib{z}{key}=k$, więc $\attrib{x}{key}$ w~obu procedurach porównywane jest z~tą samą wartością.
Ostatnią różnicą jest występowanie w~\proc{Iterative-Tree-Search} dodatkowego warunku $k\ne\attrib{x}{key}$ w~pętli \kw{while}.
Z~założenia o~unikalności kluczy w~drzewie wnioskujemy, że warunek ten będzie prawdziwy dla każdego węzła na ścieżce, zanim osiągnięty zostanie szukany węzeł.
Ścieżki pokonywane w~obu procedurach są zatem identyczne, przy czym procedura \proc{Tree-Search} porówna jeszcze poszukiwany klucz $k$ z~węzłem o~tym kluczu na końcu tej ścieżki.

\exercise %12.3-3
Pesymistyczny przypadek dla takiego sposobu sortowania $n$ liczb zachodzi wtedy, gdy tablica jest uporządkowana rosnąco bądź malejąco.
Wówczas drzewo powstałe po serii $n$ operacji \proc{Tree-Insert} ma wysokość $n-1$.
Budowa takiego drzewa zajmuje $O(n^2)$ i~taki też jest czas sortowania w~tym przypadku.

Z~kolei przypadek optymistyczny ma miejsce, gdy zbudowane drzewo ma minimalną wysokość.
Z~\refExercise{B.5-4} wiemy, że drzewo zawierające $n$ węzłów ma wysokość co najmniej $\lfloor\lg n\rfloor$.
Najmniejszy możliwy czas działania algorytmu sortowania jest ograniczony przez czas budowy takiego drzewa, czyli $O(n\lg n)$.

\exercise %12.3-4
Procedura \proc{Tree-Delete} wywołana dla węzła o~dwóch synach w~rzeczywistości usunie następnik tego węzła.
Problem pojawia się, gdy pewna struktura danych przechowuje wskaźnik do węzła $y$ i~wywołane zostanie \proc{Tree-Delete} celem usunięcia węzła $z$, będącego poprzednikiem $y$.
Struktura ta może wciąż zakładać, że $y$ należy do drzewa.
Tak jednak nie jest, ponieważ to węzeł wskazywany przez $y$ został ostatecznie usunięty, a~jego wszystkie atrybuty skopiowane zostały do węzła wskazywanego przez $z$.

Rozwiązanie tego problemu polega na podmianie węzła $z$ przez $y$ tuż przed zakończeniem operacji usuwania w~przypadku, gdy węzeł $z$ początkowo miał dwóch synów.
Wówczas węzłem efektywnie usuwanym byłby za każdym razem ten wskazywany przez $z$, dlatego w~bezpiecznej wersji procedury usuwania, której pseudokod prezentujemy poniżej, możemy zrezygnować ze zwracania jakiejkolwiek wartości.
\begin{codebox}
\Procname{$\proc{Safe-Tree-Delete}(T,z)$}
\li	$y\gets\proc{Tree-Delete}(T,z)$
\li	\If $y\ne z$
\li		\Then \If $\attrib{z}{left}\ne\const{nil}$
\li				\Then $\attribb{z}{left}{p}\gets y$
				\End
\li			\If $\attrib{z}{right}\ne\const{nil}$
\li				\Then $\attribb{z}{right}{p}\gets y$
				\End
\li			\If $\attrib{z}{p}=\const{nil}$
\li				\Then $\attrib{T}{root}\gets y$
\li				\Else \If $z=\attribb{z}{p}{left}$
\li						\Then $\attribb{z}{p}{left}\gets y$
\li						\Else $\attribb{z}{p}{right}\gets y$
						\End
				\End
\li			skopiuj zawartość wszystkich pól z~$z$ do $y$
		\End
\end{codebox}

Powyższa procedura deleguje operację usuwania węzła $z$ do \proc{Tree-Delete}.
Jeśli zwrócony przez to wywołanie węzeł $y$ jest różny od $z$, to w~dalszej części procedury następuje modyfikacja odpowiednich atrybutów synów i~ojca $z$ na węzeł $y$, przepisanie wszystkich pól z~$z$ do $y$ oraz ewentualnie aktualizacja korzenia drzewa $T$.
Po zakończeniu jej działania struktura danych korzystająca z~drzewa może bezpiecznie założyć, że usuniętym węzłem jest ten, który stanowił argument operacji usuwania.

\exercise %12.3-5
Operacja usuwania z~drzewa wyszukiwań binarnych nie jest przemienna. Kontrprzykład został zilustrowany na rys.\ \ref{fig:12.3-5}.
\begin{figure}[!ht]
	\centering \begin{tikzpicture}[
	node distance = 15mm
]

\node[outer] (pic a) {
\begin{tikzpicture}[every node/.style = tree node, anchor=center]
	\node {4}
		child {node {3}}
		child {node {7}
			child {node {6}}
			child[missing]
		};
\end{tikzpicture}
};

\node[outer, right=of pic a] (pic b) {
\begin{tikzpicture}[every node/.style = tree node, anchor=center]
	\node {7}
		child {node {6}
			child[missing]
			child {node[draw=none, fill=none] {} edge from parent[draw=none]}
		}
		child[missing];
\end{tikzpicture}
};

\node[outer, right=of pic b] (pic c) {
\begin{tikzpicture}[every node/.style = tree node, anchor=center]
	\node {6}
		child[missing]
		child {node {7}
			child {node[draw=none, fill=none] {} edge from parent[draw=none]}
			child[missing]
		};
\end{tikzpicture}
};

\foreach \x in {a,b,c} {
	\node[subpicture label, below=2mm of pic \x] {(\x)};
}

\end{tikzpicture}

	\caption{Kontrprzykład dla przemienności operacji usuwania węzła z~drzewa BST.
{\sffamily\bfseries(a)} W~przykładowym drzewie $T$ najpierw usuwany jest węzeł o~kluczu 3, a~następnie węzeł o~kluczu 4.
{\sffamily\bfseries(b)} To samo drzewo $T$, z~którego usuwane są te same węzły, ale w~odwrotnej kolejności.
Wynikowe drzewa są różne.} \label{fig:12.3-5}
\end{figure}

\exercise %12.3-6
Wybór między poprzednikiem a~następnikiem uzależnimy od wyniku rzutu monetą, który będziemy symulować wywołaniem $\proc{Random}(0,1)$.
Instrukcję przypisania w~wierszu~3 procedury \proc{Tree-Delete} zastąpimy więc następującym fragmentem:
\begin{codebox}
\zi	\If $\proc{Random}(0,1)=0$
\zi		\Then $y\gets\proc{Tree-Predecessor}(z)$
\zi		\Else $y\gets\proc{Tree-Successor}(z)$
\zi		\End
\end{codebox}
