\subchapter{Zbiory}

\exercise %B.1-1
Pierwsze prawo rozdzielności zostało zilustrowane na rys.\ \ref{fig:B.1-1}.
\begin{figure}[!ht]
	\centering \begin{tikzpicture}[
	outer/.append style = {node distance=10mm},
	filled/.style = {fill=black!20},
	node distance = 3mm
]

\def\circleA{(0,0) circle (7mm) node[circle, inner sep=2pt, above left] {$A$}}
\def\circleB{(0:7mm) circle (7mm) node[circle, inner sep=2pt, above right] {$B$}}
\def\circleC{(-60:7mm) circle (7mm) node[circle, inner sep=2pt, below] {$C$}}
	
\node[outer] (set 1) {
\begin{tikzpicture}
   	\draw[filled] \circleA;
    \draw \circleB;
    \draw \circleC;
\end{tikzpicture}
};

\node[outer, right=of set 1] (set 2) {
\begin{tikzpicture}
    \draw[filled] \circleB \circleC;
    \draw \circleA;
\end{tikzpicture}
};

\node[outer, right=of set 2] (set 3) {
\begin{tikzpicture}
	\begin{scope}
       	\clip \circleB;
       	\draw[filled] \circleA;
   	\end{scope}
    \begin{scope}
       	\clip \circleC;
       	\draw[filled] \circleA;
   	\end{scope}
	\draw \circleA;
   	\draw \circleB \circleC;
\end{tikzpicture}
};

\node[outer, right=of set 3] (set 4) {
\begin{tikzpicture}
	\begin{scope}
       	\clip \circleB;
       	\draw[filled] \circleA;
   	\end{scope}
	\draw \circleA;
   	\draw \circleB \circleC;
\end{tikzpicture}
};

\node[outer, right=of set 4] (set 5) {
\begin{tikzpicture}
	\begin{scope}
       	\clip \circleC;
       	\draw[filled] \circleA;
    \end{scope}
	\draw \circleA;
    \draw \circleB \circleC;
\end{tikzpicture}
};

\node[below=of set 1] (label 1) {$A$};
\node[below=of set 2] (label 2) {$(B\cup C)$};
\node[below=of set 3] (label 3) {$A\cap(B\cup C)$};
\node[below=of set 4] (label 4) {$(A\cap B)$};
\node[below=of set 5] (label 5) {$(A\cap C)$};

\path (set 1) -- (set 2) node[midway] (midlabel 1) {$\cap$}
      (set 2) -- (set 3) node[midway] (midlabel 2) {$=$}
      (set 3) -- (set 4) node[midway] (midlabel 3) {$=$}
      (set 4) -- (set 5) node[midway] (midlabel 4) {$\cup$};
      
\path (label 1) -| (midlabel 1) node[midway] {$\cap$}
      (label 2) -| (midlabel 2) node[midway] {$=$}
      (label 3) -| (midlabel 3) node[midway] {$=$}
      (label 4) -| (midlabel 4) node[midway] {$\cup$};

\end{tikzpicture}

	\caption{Diagramy Venna ilustrujące pierwsze prawo rozdzielności.} \label{fig:B.1-1}
\end{figure}

\exercise %B.1-2
Przeprowadzimy dowód pierwszego wzoru przez indukcję względem $n$.
Dla $n=1$ dowód jest trywialny, a~dla $n=2$ wzór stanowi pierwsze prawo de Morgana.
Załóżmy więc, że $n>2$ i~że wzór zachodzi dla rodziny $n-1$ zbiorów.
Mamy
\begin{align*}
	\overline{A_1\cap A_2\cap\dots\cap A_{n-1}\cap A_n} &= \overline{(A_1\cap A_2\cap\dots\cap A_{n-1})\cap A_n} \\
	&= \overline{A_1\cap A_2\cap\dots\cap A_{n-1}}\cup\overline{A_n} \\
	&= \overline{A_1}\cup\overline{A_2}\cup\dots\cup\overline{A_{n-1}}\cup\overline{A_n}.
\end{align*}
W~drugiej równości skorzystano z~pierwszego prawa de Morgana, a~w~trzeciej -- z~założenia indukcyjnego.

Dowód drugiego wzoru przebiega analogicznie.
Jeśli $n=2$, to wzór jest drugim prawem de Morgana, a~jeśli $n>2$, to wystarczy zastosować powyższe rozumowanie z~zamienionymi symbolami sumy i~przecięcia zbiorów oraz skorzystać z~drugiego prawa de Morgana.

\exercise %B.1-3
Udowodnimy zasadę włączania i~wyłączania przez indukcję względem $n$.
Dla $n=1$ dowód jest trywialny, a~dla $n=2$ zasada stanowi wzór (B.3).
Jeśli $n>2$, to na mocy tegoż wzoru, jak i~uogólnionego pierwszego prawa rozdzielności, mamy
\begin{align*}
    |A_1\cup A_2\cup\dots\cup A_n| &= \bigl|(A_1\cup A_2\cup\dots\cup A_{n-1})\cup A_n\bigr| \\
	&= |A_1\cup A_2\cup\dots\cup A_{n-1}|+|A_n|-\bigl|(A_1\cup A_2\cup\dots\cup A_{n-1})\cap A_n\bigr| \\
	&= |A_1\cup A_2\cup\dots\cup A_{n-1}|+|A_n|-\bigl|(A_1\cap A_n)\cup\dots\cup(A_{n-1}\cap A_n)\bigr|.
\end{align*}
Stosujemy teraz założenie indukcyjne do pierwszego i~ostatniego składnika, w~wyniku czego otrzymujemy
\begin{align*}
	|A_1\cup A_2\cup\dots\cup A_{n-1}| &= \sum_{1\le i_1<n}|A_{i_1}|-\sum_{1\le i_1<i_2<n}|A_{i_1}\cap A_{i_2}|+\sum_{1\le i_1<i_2<i_3<n}|A_{i_1}\cap A_{i_2}\cap A_{i_3}| \\[1mm]
	&\quad {}-\dots+(-1)^{n-2}|A_1\cap A_2\cap\dots\cap A_{n-1}|
\end{align*}
oraz
\begin{align*}
	\bigl|(A_1\cap A_n)\cup\dots\cup(A_{n-1}\cap A_n)\bigr| &= \sum_{1\le i_1<n}|A_{i_1}\cap A_n|-\sum_{1\le i_1<i_2<n}|A_{i_1}\cap A_{i_2}\cap A_n|\\[1mm]
	&\quad {}+\dots+(-1)^{n-2}|A_1\cap A_2\cap\dots\cap A_{n-1}\cap A_n|.
\end{align*}
Wystarczy wstawić otrzymane wyrażenia do początkowego wzoru:
\begin{align*}
	|A_1\cup A_2\cup\dots\cup A_n| &= \sum_{1\le i_1\le n}|A_{i_1}|-\sum_{1\le i_1<i_2\le n}|A_{i_1}\cap A_{i_2}|+\sum_{1\le i_1<i_2<i_3\le n}|A_{i_1}\cap A_{i_2}\cap A_{i_3}| \\[1mm]
	&\quad {}-\dots+(-1)^{n-1}|A_1\cap A_2\cap\dots\cap A_n|.
\end{align*}
A~zatem zasada zachodzi dla dowolnej skończonej rodziny zbiorów.

\exercise %B.1-4
\note{Poniższe rozwiązanie dowodzi przeliczalności zbioru nieparzystych liczb naturalnych, czego dotyczy oryginalna treść zadania.
Tłumaczenie pyta natomiast o~przeliczalność zbioru wszystkich liczb nieparzystych.}

\noindent Aby wykazać ten fakt, należy znaleźć wzajemnie jednoznaczne odwzorowanie zbioru $\mathbb{N}$ w~zbiór $\bigl\{\,2n+1:n\in\mathbb{N}\,\bigr\}$.
Każdej liczbie naturalnej $n$ przyporządkujmy liczbę $2n+1$.
Bijektywność tego odwzorowania jest oczywista, wnioskujemy zatem, że zbiór nieparzystych liczb naturalnych jest przeliczalny.

\exercise %B.1-5
Dowodzimy przez indukcję względem liczby elementów zbioru $S$.
Jeśli $|S|=0$, czyli $S=\emptyset$, to $|2^S|=2^{|S|}=1$, bo $S$ ma tylko jeden podzbiór -- zbiór pusty.
Niech teraz $|S|>0$ i~załóżmy, że $|2^S|=2^{|S|}$.
Ustalmy $p\notin S$ i~rozważmy zbiór $S'=S\cup\{p\}$.
Podzbiory zbioru $S'$ można podzielić na takie, które zawierają $p$ i~na takie, które nie zawierają $p$.
Tych ostatnich jest $|2^S|=\bigl|2^{S'\setminus\{p\}}\bigr|=2^{|S'\setminus\{p\}|}$ na mocy założenia indukcyjnego.
Okazuje się, że podzbiorów zawierających $p$ jest tyle samo, ponieważ każdy powstaje przez zsumowanie singletonu $\{p\}$ z~pewnym podzbiorem niezawierającym $p$.
Mamy zatem
\[
	\bigl|2^{S'}\bigr|=2\cdot2^{|S'\setminus\{p\}|} = 2^{|S'\setminus\{p\}|+1} = 2^{|S'|}.
\]
Na mocy indukcji twierdzenie jest spełnione dla dowolnego zbioru skończonego.

\exercise %B.1-6
\[
	\langle a_1,a_2,\dots,a_n\rangle =
	\begin{cases}
		\emptyset, & \text{jeśli $n=0$}, \\
		\{a_1\}, & \text{jeśli $n=1$}, \\
		\{a_1,\{a_1,a_2\}\}, & \text{jeśli $n=2$}, \\
		\langle a_1,\langle a_2,\dots,a_n\rangle\rangle, & \text{jeśli $n\ge3$}.
	\end{cases}
\]
