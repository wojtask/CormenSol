\subchapter{Grafy}

\exercise %B.4-1
Jeśli będziemy reprezentować zbiór pracowników przez zbiór wierzchołków $V$, a~dla każdych $u$, $v\in V$ relację ,,pracownik $u$ podał rękę pracownikowi $v$'' przez zbiór krawędzi $E$, to otrzymamy graf nieskierowany $G=\langle V,E\rangle$.
Sumując stopnie wszystkich wierzchołków tego grafu, otrzymamy podwojoną liczbę krawędzi, gdyż każdą krawędź policzymy dwa razy (każda krawędź jest incydentna z~dokładnie dwoma wierzchołkami).
Mamy więc
\[
	\sum_{v\in V}\deg(v) = 2|E|.
\]

\exercise %B.4-2
Ścieżka z~wierzchołka $u$ do wierzchołka $v$ w~dowolnym grafie jest skończonym ciągiem wierzchołków kolejno odwiedzanych na tej ścieżce, $\langle v_0,v_1,\dots,v_n\rangle$, przy czym $v_0=u$ i~$v_n=v$.
Jeśli ścieżka jest prosta, to wyrazy tego ciągu nie powtarzają się.
W~przeciwnym przypadku, jeśli podciągiem spójnym ścieżki z~$u$ do $v$ jest $\langle v_i,v_{i+1},\dots,v_{i+k},v_i\rangle$, to eliminując jego podciąg $\langle v_i,v_{i+1},\dots,v_{i+k}\rangle$, odrzucamy jedno powtórzenie $v_i$, a~tym samym podcykl ścieżki, który sprawia, że nie jest ona prosta.
Po eliminacji wszystkich takich podcykli otrzymujemy ścieżkę prostą.
Oznacza to, że każda ścieżka zawiera ścieżkę prostą reprezentowaną przez ciąg wierzchołków pozostawiony z~początkowego ciągu po zastosowaniu opisanej procedury.

Dowód dla cykli przeprowadzamy analogicznie z~$v_n=u$, pamiętając jednak, by nie eliminować ostatniego powtórzenia $u$, które jest wymagane do tego, by ścieżka stanowiła cykl.

\exercise %B.4-3
Z~twierdzenia B.2 mamy, że graf $G=\langle V,E\rangle$ będący drzewem, jest spójny i~acykliczny oraz że ma $|E|=|V|-1$ krawędzi.
Gdy dodamy do $E$ nową krawędź, to $G$ nie będzie już drzewem, ale nadal będzie spójny -- może być zatem $|E|>|V|-1$.
Z~kolei gdy usuniemy z~$E$ jakąkolwiek krawędź, to rozspójnimy $G$, przez co nie może zachodzić $|E|<|V|-1$.

\exercise %B.4-4
Każdy wierzchołek grafu skierowanego lub nieskierowanego jest osiągalny z~samego siebie, ponieważ istnieje ścieżka o~długości równej 1 zawierająca tylko ten wierzchołek, zatem relacja osiągalności jest zwrotna.

Dla dowolnych wierzchołków $u$, $v$ i~$w$ grafu skierowanego lub nieskierowanego z~faktu, że $u\leadsto v$ i~$v\leadsto w$ wynika, że $u\leadsto w$.
Istnieje bowiem ścieżka z~$u$ do $w$ będąca konkatenacją ciągów reprezentujących ścieżki z~$u$ do $v$ i~z~$v$ do $w$ (z~pominięciem powtórzenia $v$ między nimi).

Relacja osiągalności jest symetryczna jedynie w~grafach nieskierowanych, gdyż dla dowolnych wierzchołków $u$ i~$v$, jeśli $u\leadsto v$, to $v\leadsto u$.
Ścieżka z~$v$ do $u$ powstaje przez lustrzane odbicie ścieżki z~$u$ do $v$; powstały ciąg reprezentuje poprawną ścieżkę, bo każdą krawędzią można poruszać się w~obie strony.
W~grafie skierowanym krawędzie są jednokierunkowe, więc symetria nie zachodzi.

\exercise %B.4-5
\begin{figure}[!ht]
	\centering \begin{tikzpicture}[
	outer/.append style = {node distance=30mm}
]

\node[outer] (pic a) {
\begin{tikzpicture}[
	every node/.style = {tree node},
	anchor = center
]
	\node (node 1) {1};
	\node[right=15mm of node 1] (node 2) {2};
	\node[right=5mm of node 2] (node 3) {3};
	\node[below=15mm of node 1] (node 4) {4};
	\node[right=15mm of node 4] (node 5) {5};
	\node[right=5mm of node 5] (node 6) {6};
	\path (node 2) edge (node 1) edge (node 4) edge (node 5);
	\path (node 4) edge (node 1) edge (node 5);
	\path (node 3) edge (node 6);
\end{tikzpicture}
};

\node[outer, right=of pic a] (pic b) {
\begin{tikzpicture}[
	every node/.style = {tree node},
	anchor = center,
	every edge/.style = {draw, bend left=15, -latex}
]
	\node (node 1) {1};
	\node[right=15mm of node 1] (node 2) {2};
	\node[right=5mm of node 2] (node 3) {3};
	\node[below=15mm of node 1] (node 4) {4};
	\node[right=15mm of node 4] (node 5) {5};
	\node[right=5mm of node 5] (node 6) {6};
	\path (node 2) edge (node 1) edge (node 5);
	\path (node 1) edge (node 2) edge (node 5);
	\path (node 5) edge (node 1) edge (node 2);
	\path (node 3) edge (node 6);
	\path (node 6) edge (node 3);
\end{tikzpicture}
};

\node[subpicture label, below=2mm of pic a] {(a)};
\node[subpicture label, below=2mm of pic b] {(b)};

\end{tikzpicture}

	\caption{{\sffamily\bfseries(a)} Wersja nieskierowana grafu skierowanego z~rysunku B.2(a).
{\sffamily\bfseries(b)} Wersja skierowana grafu nieskierowanego z~rysunku B.2(b).} \label{fig:B.4-5}
\end{figure}

\exercise %B.4-6
Hipergraf $H=\langle V_H,E_H\rangle$ można reprezentować jako graf dwudzielny $G=\langle V_1\cup V_2,E\rangle$, w~którym $V_1=V_H$ oraz $V_2=E_H$.
Krawędź $\langle u,v\rangle\in V_1\times V_2$ w~grafie $G$ istnieje wtedy i~tylko wtedy, gdy hiperkrawędź $v$ jest incydentna z~$u$ (hiperkrawędzie mogą być incydentne z~więcej niż dwoma wierzchołkami).
W~grafie $G$ nie istnieją krawędzie pomiędzy elementami z~$V_1$ ani pomiędzy elementami z~$V_2$, zatem $G$ istotnie jest dwudzielny.
