\problem{Podziały drzew} %B-3

\subproblem %B-3(a)
Fakt oczywiście zachodzi dla drzew o~co najwyżej trzech węzłach, dlatego w~dalszej części dowodu założymy, że $n\ge4$.
Rozważmy ciąg węzłów $\langle x_0,x_1,\dots,x_k\rangle$, w~którym $x_0$ jest korzeniem, $x_k$ jest liściem, a~dla $i=1$, 2, \dots, $k$, $x_i$ jest korzeniem poddrzewa węzła $x_{i-1}$ o~większym rozmiarze (w~przypadku, gdy lewe i~prawe poddrzewo $x_{i-1}$ są tego samego rozmiaru, $x_i$ jest korzeniem dowolnego z~nich).
Oznaczmy przez $s(x)$ rozmiar poddrzewa o~korzeniu w~$x$.
Dla każdego $i=0$, 1, \dots, $k-1$ zachodzi $s(x_i)\le2s(x_{i+1})+1$.
Oczywiście ciąg $\langle s(x_0),s(x_1),\dots,s(x_k)\rangle$ maleje od $n$ do 1, dlatego musi istnieć takie $j$, że $s(x_j)>n/4$ i~$s(x_{j+1})\le n/4$.
Zatem usuwając krawędź $\{x_{j-1},x_j\}$, dzielimy węzły drzewa na dwa rozłączne zbiory o~rozmiarach $s(x_j)\le2n/4+1\le3n/4$ oraz $n-s(x_j)<n-n/4=3n/4$.

\subproblem %B-3(b)
Stała $3/4$ jest wystarczająca do dokonywania zrównoważonych podziałów, jak to wykazaliśmy w~punkcie (a).
Przykład drzewa binarnego z~rys.\ \ref{fig:B-3b} pokazuje, że nie można przyjąć na jej miejsce mniejszej wartości.
Usuwając dowolną krawędź tego drzewa, dzielimy zbiór jego wierzchołków na podzbiory, z~których jeden ma trzy elementy.
\begin{figure}[!ht]
	\centering \begin{tikzpicture}[every node/.style = tree node]
\node {}
	child {node {}
		child {node {}}
		child {node {}}
	}
	child[missing];
\end{tikzpicture}

	\caption{Drzewo binarne, w~którym najbardziej zrównoważony podział tworzy podzbiór zawierający 3 wierzchołki.} \label{fig:B-3b}
\end{figure}

\subproblem %B-3(c)
Pokażemy, jak usuwać krawędzie w~drzewie binarnym $T$ o~$n$ wierzchołkach tak, aby wydzielić podzbiór $A$ jego wierzchołków o~rozmiarze $\lfloor n/2\rfloor$.
Początkowo przyjmujemy $A=\emptyset$ i~powtarzamy następujące kroki.
Usunięcie dowolnej krawędzi z~drzewa $T$ dzieli go na poddrzewa $T_1$ i~$T_2$, z~których $T_1$ ma rozmiar $s$ nieprzekraczający rozmiaru drzewa $T_2$.
Wybierzmy do usunięcia krawędź, dla której $s$ jest możliwie największe.
Jeśli $s+|A|\le\lfloor n/2\rfloor$, to do zbioru $A$ dodajemy $s$ wierzchołków poddrzewa $T_1$.
Procedurę kontynuujemy dla $T_2$ w~miejscu $T$, aż do uzyskania $|A|=\lfloor n/2\rfloor$.

Na podstawie punktu (a), liczba usuwanych krawędzi będzie maksymalizowana, gdy za każdym razem rozmiar drzewa $T_2$ będzie stanowić dokładnie $3/4$ rozmiaru drzewa $T$.
Będzie wówczas usuniętych $\log_{4/3}n=O(\lg n)$ krawędzi.
