\problem{Kolorowanie grafów} %B-1

\subproblem %B-1(a)
Zamiast dowolnych drzew rozważmy bez straty ogólności drzewa ukorzenione.
Krawędzie w~takim drzewie są incydentne z~węzłami z~sąsiednich poziomów, a~więc można węzłom nadać kolory na podstawie parzystości ich głębokości w~drzewie.

\subproblem %B-1(b)
$1.\Rightarrow 2.$: Skoro $G$ jest grafem dwudzielnym, to można podzielić jego zbiór wierzchołków $V$ na 2 rozłączne zbiory $V_1$ i~$V_2$, między którymi nie ma krawędzi, i~nadać wierzchołkom z~$V_1$ kolor 0, a~wierzchołkom z~$V_2$ -- kolor 1, uzyskując prawidłowe \singledash{2}{kolorowanie}.
\medskip

$2.\Rightarrow 3.$: Załóżmy, że graf $G$ jest \singledash{2}{kolorowalny} i~że ma cykl nieparzysty $\langle v_1,v_2,\dots,v_{2k+1},v_1\rangle$ dla pewnego $k\ge1$.
Bez utraty ogólności niech $c(v_1)=0$.
Wtedy musi być $c(v_{2i})=1$ oraz $c(v_{2i+1})=0$, gdzie $i=1$, 2, \dots, $k$.
Jednak wówczas dwa sąsiednie wierzchołki mają ten sam kolor, $c(v_1)=c(v_{2k+1})=0$, co przeczy założeniu, że $G$ jest \singledash{2}{kolorowalny}.
Wnioskujemy zatem, że $G$ nie zawiera cyklu o~długości nieparzystej.
\medskip

$3.\Rightarrow 1.$: Ustalmy pewne $v\in V$.
Niech $V_1$ będzie zbiorem wszystkich wierzchołków grafu $G$, które znajdują się w~odległości parzystej od $v$ oraz niech $V_2=V\setminus V_1$.
Ponieważ $G$ nie zawiera cyklu nieparzystego, to żaden wierzchołek z~$V_1$ nie sąsiaduje z~innym wierzchołkiem z~$V_1$ i~analogicznie dla zbioru $V_2$, a~to oznacza, że graf $G=\langle V_1\cup V_2,E\rangle$ jest dwudzielny.

\subproblem %B-1(c)
Dowód przeprowadzimy indukcyjnie ze względu na liczbę wierzchołków w~grafie $G$.
Jeśli graf ma jeden wierzchołek, to oczywiście wystarcza jeden kolor.

Załóżmy teraz, że $|V|\ge2$.
Wybierzmy dowolny wierzchołek $v\in V$ i~oznaczmy przez $G'$ podgraf $G$ indukowany przez zbiór $V\setminus\{v\}$.
Na mocy założenia indukcyjnego $G'$ da się pokolorować $d'+1$ barwami, gdzie $d'$ to maksymalny stopień wierzchołka w~$G'$.
Ale $d'\le d$, więc tym bardziej wystarczy $d+1$ kolorów.
Ponieważ wierzchołek $v$ ma co najwyżej $d$ sąsiadów, to wśród $d+1$ kolorów użytych w~kolorowaniu podgrafu $G'$ jest jeden nieprzypisany żadnemu sąsiadowi wierzchołka $v$.
Wybierając ten kolor dla $v$, uzyskujemy poprawne \singledash{$(d+1)$}{kolorowanie} grafu $G$.

\subproblem %B-1(d)
Jeżeli $k$ barw wystarcza do optymalnego pokolorowania (czyli takiego, które wykorzystuje możliwie najmniej kolorów) grafu $G$, to dla każdych dwóch różnych barw istnieje krawędź w~$E$ incydentna z~wierzchołkami o~takich barwach.
W~przeciwnym przypadku istniałyby dwie różne barwy, których moglibyśmy nie rozróżniać i~w~ten sposób zmniejszyć liczbę wymaganych barw, co przeczyłoby optymalności kolorowania.

Na podstawie powyższego rozumowania mamy $\binom{k}{2}\le|E|$.
Korzystamy z~tego, że dla $k\ge2$ zachodzi $k/2\le k-1$ i~stąd
\[
    k^2 = 4\cdot\frac{k}{2}\cdot\frac{k}{2} \le 4\cdot\frac{k(k-1)}{2} = 4\binom{k}{2} \le 4|E| = O(|V|),
\]
czyli $k=O\bigl(\!\sqrt{|V|}\bigr)$.
