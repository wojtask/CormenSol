\subchapter{Relacje}

\exercise %B.2-1
Pokażemy, że relacja $\subseteq$ w~$2^\mathbb{Z}$ jest zwrotna, antysymetryczna i~przechodnia.
Niech $A$, $B$, $C\in2^\mathbb{Z}$.
Oczywiście $x\in A$ implikuje $x\in A$, czyli zwrotność trywialnie zachodzi.
Jeśli $A\subseteq B$, czyli $x\in A$ implikuje $x\in B$, oraz $B\subseteq A$, czyli $x\in B$ implikuje $x\in A$, to z~prawa przechodniości implikacji $x\in A$ wtedy i~tylko wtedy, gdy $x\in B$, a~zatem $A=B$.
Koniunkcja $A\subseteq B$ i~$B\subseteq C$ oznacza, że $x\in A$ implikuje $x\in B$, a~$x\in B$ implikuje $x\in C$.
Odwołując się ponownie do prawa przechodniości implikacji mamy, że $x\in A$ implikuje $x\in C$, czyli $A\subseteq C$.

Porządek $\subseteq$ w~$2^\mathbb{Z}$ nie jest liniowy, bo np.\ $\{0,1\}\nsubseteq\{1,2\}$ i~$\{1,2\}\nsubseteq\{0,1\}$.

\exercise %B.2-2
Oznaczmy przez $R_n$, dla $n\in\mathbb{N}\setminus\{0\}$, relację ,,przystaje modulo $n$'':
\[
	R_n = \{\,\langle a,b\rangle\in\mathbb{Z}\times\mathbb{Z}\;:\;a\equiv b\!\!\!\pmod{n}\,\}.
\]
Dla dowolnego $a\in\mathbb{Z}$ mamy $a\equiv a\pmod{n}$, bo $a-a=0$, więc relacja $R_n$ jest zwrotna.
Dla dowolnych $a$, $b\in\mathbb{Z}$, jeśli istnieje $q\in\mathbb{Z}$, że $a-b=qn$, to $b-a=-qn$, a~zatem z~faktu, że $a\equiv b\pmod{n}$ wynika, że $b\equiv a\pmod{n}$, co dowodzi symetrii $R_n$.
Dla dowodu przechodniości wybierzmy dowolne $a$, $b$, $c\in\mathbb{Z}$ i~załóżmy, że zachodzi $a\equiv b\pmod{n}$ oraz $b\equiv c\pmod{n}$.
Oznacza to, że istnieją $q$, $r\in\mathbb{Z}$, że $a-b=qn$ oraz $b-c=rn$.
Stąd $a-c=a-b+b-c=qn+rn=(q+r)n$, a~zatem $a\equiv c\pmod{n}$.

Na mocy powyższych faktów $R_n$ jest relacją równoważności i~dzieli zbiór $\mathbb{Z}$ na $n$ klas abstrakcji; \singledash{$i$}{ta} klasa, gdzie $i=1$, 2, \dots, $n$, jest zbiorem takich liczb całkowitych, które przy dzieleniu przez $n$ dają resztę $i-1$.

\exercise %B.2-3
\subexercise
Relacja $\{\langle a,a\rangle,\langle b,b\rangle,\langle c,c\rangle,\langle a,b\rangle,\langle b,a\rangle,\langle a,c\rangle,\langle c,a\rangle\}$ w~zbiorze $\{a,b,c\}$.

\subexercise
Relacja $\{\langle a,a\rangle, \langle a,b\rangle,\langle b,b\rangle\}$ w~zbiorze $\{a,b\}$.

\subexercise
Relacja pusta w~dowolnym zbiorze niepustym.

\exercise %B.2-4
Jeśli $R$ jest relacją równoważności, to dla każdego $s\in S$ zachodzi $s\in[s]$.
Na mocy antysymetrii $R$, jeśli zachodzi $s'\,R\,s$ oraz $s\,R\,s'$, to $s=s'$, a~więc nie istnieją takie elementy $s'$, że $s'\in[s]\setminus\{s\}$.
To oznacza, że klasy abstrakcji $S$ względem relacji $R$ są singletonami.

\exercise %B.2-5
Symetria i~przechodniość relacji zdefiniowane są za pomocą implikacji, do spełnienia których nie jest konieczne spełnienie ich poprzedników.
Relacja pozostanie symetryczna i~przechodnia, jeśli w~zbiorze, w~którym jest określona, istnieje element niebędący w~relacji z~żadnym elementem z~tego zbioru.
Zwrotność wymaga natomiast, aby każdy element był w~relacji z~samym sobą.
Istnieją zatem relacje symetryczne i~przechodnie, ale nie zwrotne (patrz punkt (c) \refExercise{B.2-3}).
