\problem{Grafy znajomości} %B-2

\subproblem %B-2(a)
\textsf{\textbf{Twierdzenie.}} \textit{W~grafie nieskierowanym\/ $G=\langle V,E\rangle$, w~którym\/ $|V|\ge2$, istnieją dwa wierzchołki o~tym samym stopniu.}
\begin{proof}
W~grafie $G$ o~$n\ge2$ wierzchołkach możliwymi stopniami wierzchołków są liczby 0, 1, \dots, $n-1$.
Jeśli jednak pewien wierzchołek ma stopień równy 0, to żaden z~pozostałych nie ma stopnia $n-1$.
Oznacza to, że jest $n$ wierzchołków, ale tylko co najwyżej $n-1$ liczb mogących jednocześnie być ich stopniami w~$G$, zatem pewne dwa wierzchołki mają równe stopnie.
\end{proof}

\subproblem %B-2(b)
\textsf{\textbf{Twierdzenie.}} \textit{Graf nieskierowany\/ $G=\langle V,E\rangle$, w~którym\/ $|V|=6$, lub jego dopełnienie\/ $\overline{G}$ zawierają klikę o~3 wierzchołkach.}
\begin{proof}
Wybierzmy pewne $v\in V$.
Istnieją wtedy w~$V$ trzy inne wierzchołki $v_1$, $v_2$, $v_3$ wszystkie sąsiednie z~$v$ albo wszystkie niesąsiednie z~$v$.
Ponieważ przypadki te są symetryczne, rozważmy pierwszy z~nich.
Jeśli nie istnieje wśród $v_1$, $v_2$, $v_3$ para wierzchołków sąsiednich, to twierdzenie zachodzi.
Załóżmy więc, że istnieje krawędź między pewnymi dwoma.
Wtedy jednak tworzą one wraz z~$v$ klikę, a~więc również w~tym przypadku twierdzenie jest prawdziwe.
\end{proof}

Problem rozważany w~tym punkcie jest związany z~\textbf{liczbami Ramseya} $R(m,n)$~\cite{ramseynumber}; powyższe twierdzenie pokazuje, że $R(3,3)\le6$.

\subproblem %B-2(c)
\textsf{\textbf{Twierdzenie.}} \textit{Zbiór wierzchołków dowolnego grafu nieskierowanego\/ $G=\langle V,E\rangle$ można podzielić na dwa rozłączne zbiory tak, aby dla dowolnego wierzchołka\/ $v\in V$ co najmniej połowa jego sąsiadów nie należała do zbioru, do którego należy\/ $v$.}
\begin{proof}
Pokażemy, że istnieje taki podział zbioru wierzchołków grafu nieskierowanego $G=\langle V,E\rangle$, że dla każdego $v\in V$\! zachodzi $d(v)\ge0$, gdzie $d(v)$ jest liczbą sąsiadów $v$ z~części podziału niezawierającej $v$ pomniejszoną o~liczbę sąsiadów $v$ z~części podziału zawierającą $v$.

Zdefiniujmy $D=\sum_{v\in V}d(v)$ i~zastanówmy się, jak tę wartość można zmaksymalizować, wyznaczając podział $V=V_1\cup V_2$.
Załóżmy, że dokonaliśmy już pewnego takiego podziału i~wybierzmy pewien wierzchołek $v$ należący do zbioru $V_1$.
Dowód w~przypadku gdy $v\in V_2$ przebiega symetrycznie.
Zauważmy, że jeśli przenieślibyśmy wierzchołek $v$ do $V_2$, to jego waga $d(v)$ zmieniłaby znak na przeciwny.
Ponadto waga każdego sąsiada $v$ należącego do zbioru $V_1$ wzrosłaby o~2, a~waga każdego sąsiada $v$ z~$V_2$ zmalałaby o~2.
W~wyniku przeniesienia $v$ wartość $D$ wzrosłaby o~$-4d(v)$.
Widać więc, że aby zwiększyć $D$, należy przenosić wierzchołki o~ujemnych wagach.
Ponieważ $D$ nie może rosnąć w~nieskończoność (jest ograniczone od góry przez $2|E|$ w~grafach dwudzielnych), to po skończonej liczbie takich operacji wszystkie wierzchołki grafu $G$ będą mieć wagi nieujemne, czego należało dowieść.
\end{proof}

\subproblem %B-2(d)
\textsf{\textbf{Twierdzenie} (Dirac)\textbf{.}} \textit{Jeśli dla każdego wierzchołka\/ $v$ grafu nieskierowanego\/ $G=\langle V,E\rangle$, w~którym\/ $|V|\ge3$, zachodzi\/ $\deg(v)\ge|V|/2$, to\/ $G$ jest hamiltonowski.}

\medskip
\noindent Zanim zajmiemy się dowodem twierdzenia, udowodnimy następujący lemat.

\medskip
\noindent\textsf{\textbf{Lemat} (Ore)\textbf{.}} \textit{Jeśli w~grafie nieskierowanym\/ $G=\langle V,E\rangle$, gdzie\/ $|V|\ge3$, dla każdej pary niesąsiednich wierzchołków\/ $u$ i\/~$v$ zachodzi nierówność\/ $\deg(u)+\deg(v)\ge|V|$, to\/ $G$ jest hamiltonowski.}
\begin{proof}
Oznaczmy przez $n$ liczbę wierzchołków grafu $G$.
Przypuśćmy, że lemat jest fałszywy, czyli że dla pewnego $n$ istnieje kontrprzykład -- graf $G=\langle V,E\rangle$, w~którym $|V|=n$ i~który spełnia założenie lematu, ale nie jest hamiltonowski.
Spośród wszystkich takich grafów rozpatrzmy ten, dla którego $|E|$ jest maksymalne.
Wówczas $G$ musi mieć ścieżkę Hamiltona $\langle v_1,v_2,\dots,v_n\rangle$ -- w~przeciwnym przypadku moglibyśmy dodać do niego brakujące krawędzie, nie naruszając warunku z~założenia lematu i~otrzymując w~wyniku graf o~więcej niż $|E|$ krawędziach.
Ponieważ $G$ nie ma cyklu Hamiltona, to nie istnieje krawędź łącząca $v_1$ z~$v_n$.
Z~kolei z~założenia wiemy, że $\deg(v_1)+\deg(v_n)\ge n$.

Zdefiniujemy teraz podzbiory $S_1$ i~$S_n$ zbioru $\{2,3,\dots,n\}$ takie, że
\[
	S_1 = \bigl\{\,i:\{v_1,v_i\}\in E\,\bigr\} \quad\text{oraz}\quad S_n = \bigl\{\,i:\{v_{i-1},v_n\}\in E\,\bigr\}.
\]
Wtedy $|S_1|=\deg(v_1)$ i~$|S_n|=\deg(v_n)$.
Ponieważ $|S_1|+|S_n|\ge n$ i~zbiór $S_1\cup S_n$ ma co najwyżej $n-1$ elementów, to zbiór $S_1\cap S_n$ musi być niepusty.
Istnieje więc $i$, dla którego istnieją krawędzie $\{v_1,v_i\}$ oraz $\{v_{i-1},v_n\}$.
Stąd ścieżka $\langle v_1,\dots,v_{i-1},v_n,v_{n-1},\dots,v_i,v_1\rangle$ jest cyklem Hamiltona w~grafie $G$.
Sprzeczność -- lemat jest prawdziwy.
\end{proof}

Można teraz udowodnić główne twierdzenie.
\begin{proof}
Jeśli dla każdego $v\in V$ zachodzi $\deg(v)\ge|V|/2$, to $\deg(u)+\deg(v)\ge|V|$ dla każdych $u$, $v\in V$ niezależnie od tego, czy są sąsiednie, czy nie, a~więc $G$ spełnia założenia powyższego lematu, czyli jest hamiltonowski.
\end{proof}
