\chapter{Drzewa wyszukiwań binarnych}

\subchapter{Co to jest drzewo wyszukiwań binarnych?}

\exercise %12.1-1
Przykładowe drzewa BST przedstawiono na rys.~\ref{fig:12.1-1}.
\begin{figure}[ht]
	\begin{center}
		\includegraphics{fig12.1}
	\end{center}
	\caption{Drzewa BST o~różnych wysokościach zawierające zbiór kluczy $\{1,4,5,10,16,17,21\}$.} \label{fig:12.1-1}
\end{figure}

\exercise %12.1-2
\note{Wyrażenie ,,właściwa kolejność'' użyte w~tłumaczeniu oznacza kolejność niemalejącą.}

\noindent Niech $x$ będzie węzłem drzewa $T$, a~$y$, $z$, odpowiednio, lewym i~prawym synem $x$. Jeśli drzewo $T$ byłoby drzewem BST, czyli spełniałoby własność drzewa BST, to wówczas prawdziwe byłyby nierówności $\id{key}[y]\le\id{key}[x]\le\id{key}[z]$. Jeśli z~kolei drzewo $T$ stanowiłoby kopiec typu min, czyli spełniałoby własność kopca typu min, to zachodziłoby wtedy $\id{key}[x]\le\id{key}[y]$ oraz $\id{key}[x]\le\id{key}[z]$.

Ta różnica sprawia, że podczas przechodzenia kopca typu min w~kolejności niemalejących węzłów, po odwiedzeniu węzła $x$ nie jest wiadomo, które jego poddrzewo należałoby następnie odwiedzić, ponieważ nie jest znana relacja między $y$ a~$z$. Oznacza to, że wypisanie wszystkich $n$ węzłów kopca typu min w~kolejności niemalejącej zajmuje czas wyższy niż $O(n)$. W~przeciwnym przypadku moglibyśmy zaimplementować algorytm heapsort w~czasie liniowym.

\exercise %12.1-3
Nierekurencyjna wersja algorytmu przechodzenia drzewa metodą inorder, która wykorzystuje stos, została przedstawiona w~\refExercise{10.4-3}. Nierekurencyjny algorytm o~identycznej funkcjonalności, ale niewykorzystujący stosu, został opisany w~\refExercise{10.4-5}.

\exercise %12.1-4
Algorytmy przechodzenia drzewa metodami preorder i~postorder przedstawiono poniżej. 
\begin{codebox}
\Procname{$\proc{Preorder-Tree-Walk}(x)$}
\li	\If $x\ne\const{nil}$
\li		\Then
			wypisz $\id{key}[x]$
\li			$\proc{Preorder-Tree-Walk}(\id{left}[x])$
\li			$\proc{Preorder-Tree-Walk}(\id{right}[x])$
		\End
\end{codebox}

\begin{codebox}
\Procname{$\proc{Postorder-Tree-Walk}(x)$}
\li	\If $x\ne\const{nil}$
\li		\Then
			$\proc{Postorder-Tree-Walk}(\id{left}[x])$
\li			$\proc{Postorder-Tree-Walk}(\id{right}[x])$
\li			wypisz $\id{key}[x]$
		\End
\end{codebox}

Dla każdego z~tych algorytmów można udowodnić twierdzenie analogiczne do tw.~12.1, a~więc każdy z~nich działa w~czasie $\Theta(n)$ na drzewie \onedash{$n$}{wierzchołkowym}.

\exercise %12.1-5
Załóżmy nie-wprost, że istnieje algorytm konstruowania drzewa BST z~dowolnej listy $n$ elementów za pomocą porównań, którego pesymistyczny czas działania wynosi $o(n\lg n)$. Na drzewie BST utworzonym za pomocą tego algorytmu moglibyśmy następnie uruchomić algorytm przechodzenia drzewa metodą inorder i~wypisać wszystkie jego klucze w~kolejności niemalejącej w~czasie $\Theta(n)$. Posortowanie wejściowej listy $n$ elementów można byłoby wtedy przeprowadzić w~czasie niższym niż liniowo-logarytmiczny, co przeczyłoby tw.~8.1.

\subchapter{Wyszukiwanie w~drzewie wyszukiwań binarnych}

\exercise %12.2-1
Ciągi z~przykładów (a), (b) i~(d) są możliwe do uzyskania podczas wyszukiwania klucza 363 w~drzewie BST. W~przykładzie~(c) po odwiedzeniu węzła o~kluczu 911 przechodzimy do lewego poddrzewa, w~którym znajdują się węzły o~kluczach nie większych niż 911. Jednakże później napotykamy 912, a~to nie jest możliwe w~drzewie BST. W~przykładzie~(e) występuje podobna sytuacja -- po odwiedzeniu węzła o~kluczu 347 przetwarzane jest prawe poddrzewo. Klucz 299 napotykany później jest jednak mniejszy niż 347, co w~drzewie BST nie może mieć miejsca.

\exercise %12.2-2
Rekurencyjne wersje procedur zostały przedstawione poniżej.
\begin{codebox}
\Procname{$\proc{Recursive-Tree-Minimum}(x)$}
\li	\If $\id{left}[x]\ne\const{nil}$
\li		\Then \Return $\proc{Recursive-Tree-Minimum}(\id{left}[x])$
\li		\Else \Return $x$
		\End
\end{codebox}
\begin{codebox}
\Procname{$\proc{Recursive-Tree-Maximum}(x)$}
\li	\If $\id{right}[x]\ne\const{nil}$
\li		\Then \Return $\proc{Recursive-Tree-Maximum}(\id{right}[x])$
\li		\Else \Return $x$
		\End
\end{codebox}

\exercise %12.2-3
Procedura \proc{Tree-Predecessor} jest symetryczna do \proc{Tree-Successor}. Jej pseudokod znajduje się poniżej.
\begin{codebox}
\Procname{$\proc{Tree-Predecessor}(x)$}
\li	\If $\id{left}[x]\ne\const{nil}$
\li		\Then \Return $\proc{Tree-Maximum}(\id{left}[x])$
		\End
\li	$y\gets p[x]$
\li	\While $y\ne\const{nil}$ i~$x=\id{left}[y]$
\li		\Do
			$x\gets y$
\li			$y\gets p[y]$
		\End
\li	\Return $y$
\end{codebox}

\exercise %12.2-4
Rys.~\ref{fig:12.2-4} przedstawia najmniejsze drzewo BST, które obala postawioną hipotezę.
\begin{figure}[ht]
	\begin{center}
		\includegraphics{fig12.2}
	\end{center}
	\caption{Najmniejszy kontrprzykład dla rozumowania profesora Bunyana. W~drzewie tym szukano klucza $k=4$. Jasnym kolorem zaznaczono jedyny węzeł o~kluczu należącym do zbioru $A$, a~ciemnym kolorem -- węzły o~kluczach ze zbioru $B$. Zbiór $C$ w~tym przykładzie jest pusty. Klucze 1 i~2 nie spełniają nierówności postawionej w~hipotezie.} \label{fig:12.2-4}
\end{figure}

\exercise %12.2-5
\exercise %12.2-6
\exercise %12.2-7
\exercise %12.2-8
\exercise %12.2-9

\subchapter{Wstawianie i~usuwanie}

\exercise %12.3-1
\begin{codebox}
\Procname{$\proc{Recursive-Tree-Insert}(T,x,z)$}
\li	$y\gets\const{nil}$
\li	\If $x=\const{nil}$
\li		\Then
			$p[z]\gets y$
\li			\If $y=\const{nil}$
\li				\Then $\id{root}[T]\gets z$
\li				\Else
					\If $\id{key}[z]<\id{key}[y]$
\li						\Then $\id{left}[y]\gets z$
\li						\Else $\id{right}[y]\gets z$
						\End
				\End
\li		\Else
			$y\gets x$
\li			\If $\id{key}[z]<\id{key}[x]$
\li				\Then $\proc{Recursive-Tree-Insert}(T,\id{left}[x],z)$
\li				\Else $\proc{Recursive-Tree-Insert}(T,\id{right}[x],z)$
				\End
		\End
\end{codebox}

\exercise %12.3-2
\exercise %12.3-3
\exercise %12.3-4
\exercise %12.3-5
\exercise %12.3-6

\subchapter{Losowo skonstruowane drzewa wyszukiwań binarnych}

\exercise %12.4-1
\exercise %12.4-2
\exercise %12.4-3
\exercise %12.4-4
\exercise %12.4-5
\note{Wymagane jest założenie, że elementy tablicy wejściowej algorytmu \proc{Randomized-Quicksort} są parami różne.}

\problems

\problem{Drzewa wyszukiwań binarnych z~powtarzającymi się kluczami} %12-1

\subproblem %12-1(a)
\subproblem %12-1(b)
\subproblem %12-1(c)
\subproblem %12-1(d)

\problem{Drzewa pozycyjne} %12-2

\problem{Średnia głębokość węzła w~losowo zbudowanym drzewie wyszukiwań binarnych} %12-3

\subproblem %12-3(a)
\subproblem %12-3(b)
\subproblem %12-3(c)
\subproblem %12-3(d)
\subproblem %12-3(e)
\subproblem %12-3(f)

\problem{Zliczanie różnych drzew binarnych} %12-4

\subproblem %12-4(a)
\subproblem %12-4(b)
\subproblem %12-4(c)
\subproblem %12-4(d)

\endinput