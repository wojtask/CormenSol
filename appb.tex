\chapter{Zbiory i~nie tylko}

\subchapter{Zbiory}

\exercise %B.1-1
\begin{figure}[ht]
	\begin{center}
		\includegraphics{figb.1}
	\end{center}
	\caption{Diagramy Venna ilustrujące pierwsze prawo rozdzielności.}
\end{figure}

\exercise %B.1-2
Przeprowadzimy dowód dla dopełnienia przecięcia zbiorów przez indukcję względem $n$. Dla $n=1$ dowód jest trywialny, a~dla $n=2$ wzór jest podany w~podręczniku. Załóżmy więc, że $n>2$ i~że twierdzenie zachodzi dla rodziny $n-1$ zbiorów. Mamy
\begin{align*}
	\overline{A_1\cap A_2\cap\cdots\cap A_{n-1}\cap A_n} &= \overline{(A_1\cap A_2\cap\cdots\cap A_{n-1})\cap A_n} \\
	&= \overline{A_1\cap A_2\cap\cdots\cap A_{n-1}}\cup\overline{A_n} \\
	&= \overline{A_1}\cup\overline{A_2}\cup\cdots\cup\overline{A_{n-1}}\cup\overline{A_n}.
\end{align*}
W~drugiej równości skorzystano z~prawa de~Morgana dla dopełnienia przecięcia dwóch zbiorów, a~w~trzeciej -- z~założenia indukcyjnego.

Dla dopełnienia sumy zbiorów dowód przebiega analogicznie -- wystarczy tylko w~powyższym wyprowadzeniu zamienić ze~sobą symbole sumy i~przecięcia zbiorów.

\exercise %B.1-3
Udowodnimy zasadę włączania i~wyłączania przez indukcję względem liczby zbiorów $n$. Dla $n=1$ dowód jest trywialny, a~dla $n=2$ wynika z~wzoru~(B.3). Na mocy tego wzoru dla $n>2$ mamy
\begin{align*}
    |A_1\cup A_2\cup\dots\cup A_n| &= \bigl|(A_1\cup A_2\cup\dots\cup A_{n-1})\cup A_n\bigr| \\
	&= |A_1\cup A_2\cup\dots\cup A_{n-1}|+|A_n|-\bigl|(A_1\cup A_2\cup\dots\cup A_{n-1})\cap A_n\bigr| \\
	&= |A_1\cup A_2\cup\dots\cup A_{n-1}|+|A_n|-\bigl|(A_1\cap A_n)\cup\dots\cup(A_{n-1}\cap A_n)\bigr|.
\end{align*}
Stosujemy teraz założenie indukcyjne do pierwszego i~ostatniego składnika sumy, w~wyniku czego otrzymujemy
\begin{align*}
	|A_1\cup A_2\cup\dots\cup A_{n-1}| &= \sum_{1\le i_1<n}|A_{i_1}|-\sum_{1\le i_1<i_2<n}|A_{i_1}\cap A_{i_2}|+\sum_{1\le i_1<i_2<i_3<n}|A_{i_1}\cap A_{i_2}\cap A_{i_3}| \\[1mm]
	&\quad {}-\dots+(-1)^{n-2}|A_1\cap A_2\cap\dots\cap A_{n-1}|
\end{align*}
oraz
\begin{align*}
	\bigl|(A_1\cap A_n)\cup\dots\cup(A_{n-1}\cap A_n)\bigr| &= \sum_{1\le i_1<n}|A_{i_1}\cap A_n|-\sum_{1\le i_1<i_2<n}|A_{i_1}\cap A_{i_2}\cap A_n|\\[1mm]
	&\quad {}+\dots+(-1)^{n-1}|A_1\cap A_2\cap\dots\cap A_{n-1}\cap A_n|.
\end{align*}
Wystarczy wstawić otrzymane wyrażenia do początkowego wzoru, który przyjmuje postać
\begin{align*}
	|A_1\cup A_2\cup\dots\cup A_n| &= \sum_{1\le i_1\le n}|A_{i_1}|-\sum_{1\le i_1<i_2\le n}|A_{i_1}\cap A_{i_2}|+\sum_{1\le i_1<i_2<i_3\le n}|A_{i_1}\cap A_{i_2}\cap A_{i_3}| \\[1mm]
	&\quad {}-\dots+(-1)^{n-1}|A_1\cap A_2\cap\dots\cap A_n|,
\end{align*}
a~zatem zasada zachodzi dla dowolnej skończonej ilości zbiorów.

\exercise %B.1-4
\note{Poniższe rozwiązanie dowodzi przeliczalności zbioru nieparzystych liczb naturalnych, czego dotyczy oryginalna treść zadania. Tłumaczenie pyta natomiast o~przeliczalność zbioru wszystkich liczb nieparzystych.}

\noindent Aby wykazać ten fakt, należy znaleźć wzajemnie jednoznaczne odwzorowanie ze zbioru $\mathbb{N}$ w~zbiór $\bigl\{\,2k+1:k\in\mathbb{N}\,\bigr\}$. Zauważmy, że sposób odwzorowania wynika wprost z~zapisu tego ostatniego zbioru -- danej liczbie naturalnej $k$ wystarczy przyporządkować liczbę $2k+1$. Wnioskujemy zatem, że zbiór nieparzystych liczb naturalnych jest przeliczalny.

\exercise %B.1-5
Dowód przez indukcję względem liczby elementów $S$. Jeśli $|S|=0$, czyli $S=\emptyset$, to $|2^S|=2^{|S|}=1$, bo $S$ ma tylko jeden podzbiór -- zbiór pusty. Załóżmy teraz, że $|S|>0$ oraz że $|2^S|=2^{|S|}$. Niech $p\not\in S$ i~rozważmy zbiór $S'=S\cup\{p\}$. Zauważmy, że podzbiory zbioru $S'$ można podzielić na takie, które zawierają $p$ i~na takie, które nie zawierają $p$. Tych ostatnich jest $|2^S|=\bigl|2^{S'\setminus\{p\}}\bigr|=2^{|S'\setminus\{p\}|}$ (z założenia indukcyjnego). Okazuje się, że podzbiorów zawierających $p$ jest tyle samo, ponieważ każdy powstaje przez zsumowanie singletonu $\{p\}$ z~pewnym z~$2^{|S'\setminus\{p\}|}$ podzbiorów niezawierających $p$. Mamy zatem
\[
	\bigl|2^{S'}\bigr|=2\cdot2^{|S'\setminus\{p\}|} = 2^{|S'\setminus\{p\}|+1} = 2^{|S'|}.
\]
Na mocy indukcji twierdzenie jest spełnione dla dowolnego zbioru skończonego.

\exercise %B.1-6
\[
	\langle a_1,a_2,\dots,a_n\rangle =
	\begin{cases}
		\emptyset, & \text{jeśli $n=0$}, \\
		\{a_1\}, & \text{jeśli $n=1$}, \\
		\{a_1,\{a_1,a_2\}\}, & \text{jeśli $n=2$}, \\
		\langle a_1,\langle a_2,\dots,a_n\rangle\rangle, & \text{jeśli $n\ge3$}.
	\end{cases}
\]

\subchapter{Relacje}

\exercise %B.2-1
Porządek częściowy jest relacją zwrotną, antysymetryczną i~przechodnią. Zauważmy, że relacja $\subseteq$ w~$2^\mathbb{Z}$ posiada każdą z~tych cech. Dla zbioru $A\in2^\mathbb{Z}$ zachodzi $A\subseteq A$ (zwrotność). Dla zbiorów $A$, $B\in2^\mathbb{Z}$, jeśli spełniają one $A\subseteq B$ i~$B\subseteq A$, to zachodzi $A=B$ (antysymetria). Wreszcie, dla zbiorów $A$, $B$, $C\in2^\mathbb{Z}$, jeżeli $A\subseteq B$ i~$B\subseteq C$, to $A\subseteq C$ (przechodniość). Jednak w~$2^\mathbb{Z}$ porządek $\subseteq$ nie jest liniowy, bo np.\ $\{0,1\}\not\subseteq\{1,2\}$ i~$\{1,2\}\not\subseteq\{0,1\}$.

\exercise %B.2-2
Oznaczmy przez $R_n$ dla $n\in\mathbb{N}\setminus\{0\}$ relację ``przystaje modulo $n$'':
\[
	R_n = \bigl\{\,\langle a,b\rangle\in\mathbb{Z}\times\mathbb{Z}\;:\;a\equiv b\!\!\!\pmod{n}\,\bigr\}.
\]

Dla dowolnego $a\in\mathbb{Z}$ mamy $a\equiv a\pmod{n}$, bo $a-a=0$, więc relacja $R_n$ jest zwrotna. Dla dowolnych $a$, $b\in\mathbb{Z}$, jeśli istnieje $q\in\mathbb{Z}$, że $a-b=qn$, to $b-a=-qn$, a~zatem z~faktu, że $a\equiv b\pmod{n}$ wynika, że $b\equiv a\pmod{n}$, co dowodzi symetrii $R_n$. Dla dowodu przechodniości wybierzmy dowolne $a$, $b$, $c\in\mathbb{Z}$ i~załóżmy, że zachodzi $a\equiv b\pmod{n}$ oraz $b\equiv c\pmod{n}$. Oznacza to, że istnieją $q$, $r\in\mathbb{Z}$, że $a-b=qn$ oraz $b-c=rn$. Stąd $a-c=a-b+b-c=qn+rn=(q+r)n$, a~zatem $a\equiv c\pmod{n}$.

Na mocy powyższych faktów $R_n$ jest relacją równoważności i~dzieli zbiór $\mathbb{Z}$ na $n$ klas abstrakcji; \compound{$i$}{ta} klasa, gdzie $1\le i\le n$, jest zbiorem takich liczb całkowitych, które przy dzieleniu przez $n$ dają resztę $i-1$.

\exercise %B.2-3
\subexercise
Relacja $R=\bigl\{\langle a,a\rangle,\langle b,b\rangle,\langle c,c\rangle,\langle a,b\rangle,\langle b,a\rangle,\langle a,c\rangle,\langle c,a\rangle\bigr\}$ określona w~zbiorze $\{a,b,c\}$.

\subexercise
Relacja $\le$ określona w~zbiorze liczb rzeczywistych.

\subexercise
Relacja $T=\bigl\{\langle a,a\rangle,\langle b,b\rangle,\langle a,b\rangle,\langle b,a\rangle\bigr\}$ określona w~zbiorze $\{a,b,c\}$.

\exercise %B.2-4
Jeśli $R$ jest relacją równoważności, to dla każdego $s\in S$ zachodzi $s\in[s]$. Na mocy antysymetrii $R$, jeśli zachodzi $s'\,R\,s$ oraz $s\,R\,s'$, to $s=s'$, a~więc nie istnieją takie elementy $s'$, że $s'\in[s]\setminus\{s\}$, zatem klasy abstrakcji są singletonami.

\exercise %B.2-5
W~definicji symetrii i~przechodniości relacji mamy implikacje. Aby były one prawdziwe, nie muszą być spełnione ich poprzedniki. Zwrotność wymaga natomiast, aby każdy element ze zbioru, w~którym określamy relację, był ze sobą w~relacji. Istnieją zatem relacje symetryczne i~przechodnie, ale nie zwrotne. Jednym z~przykładów jest relacja z~punktu~(c) w~\refExercise{B.2-3}.

\subchapter{Funkcje}

\exercise %B.3-1
\subexercise
Zbiór wartości funkcji $f\colon A\to B$, czyli obraz jej dziedziny, jest zdefiniowany następująco:
\[
	f(A) = \bigl\{\,b\in B:b=f(a)\text{ dla pewnego $a\in A$}\,\bigr\}.
\]
Z~tego, że $f$ jest injekcją, mamy, że $|A|=|f(A)|$. Z~kolei $|f(A)|\le|B|$, bo w~$B$ mogą być takie elementy $b$, dla których nie istnieje $a\in A$ takie, że $b=f(a)$. Stąd $|A|\le|B|$.

\subexercise
Dla surjekcji $f\colon A\to B$ zachodzi $f(A)=B$, więc $|f(A)|=|B|$. Dla pewnych elementów $a_1$,~$a_2\in A$ może zachodzić $f(a_1)=f(a_2)$, mamy zatem $|A|\ge|f(A)|$, a~stąd $|A|\ge|B|$.
\bigskip

Zauważmy, że z~powyższych faktów wynika, że jeśli $f\colon A\to B$ jest bijekcją, to $|A|=|B|$.

\exercise %B.3-2
Funkcja $f$ o~dziedzinie i~przeciwdziedzinie $\mathbb{N}$ nie jest bijekcją, gdyż dla żadnego $x\in\mathbb{N}$ nie zachodzi $f(x)=0$. Jeśli zamiast $\mathbb{N}$ rozważamy $\mathbb{Z}$, to $f$ jest bijekcją -- każda liczba całkowita jest wartością $f$ w~pewnej jednoznacznie wyznaczonej liczbie całkowitej.

\exercise %B.3-3
Niech $R$ będzie relacją binarną w~zbiorze $A$. Relację $R^{-1}$ w~zbiorze $A$ nazywamy relacją odwrotną do $R$, jeżeli dla dowolnych $a$,~$b\in A$, $a\,R^{-1}\,b$ wtedy i~tylko wtedy, gdy $b\,R\,a$. Łatwo zauważyć, że jeśli $R$ jest bijekcją, to $R^{-1}$ jest jej funkcją odwrotną.

\exercise %B.3-4
Ponieważ każda bijekcja posiada funkcję odwrotną, która także jest bijekcją, to znajdziemy funkcję $F\colon\mathbb{Z}\times\mathbb{Z}\to\mathbb{Z}$ będącą odwrotnością szukanego odwzorowania. Wyznaczenie $F$ jest równoważne znalezieniu sposobu ponumerowania liczbami całkowitymi każdej pary o~elementach całkowitych tak, aby żadne dwie pary nie miały tego samego numeru i~żeby każda liczba całkowita była wykorzystana jako numer pewnej pary. Opiszemy teraz konstrukcję jednej z~takich funkcji.

Dokonajmy pewnego uproszczenia -- zamiast numerować pary liczbami całkowitymi, ograniczmy się do liczb naturalnych. Niech $g\colon\mathbb{Z}\times\mathbb{Z}\to\mathbb{N}$ oraz $h\colon\mathbb{N}\to\mathbb{Z}$ będą pewnymi bijekcjami oraz niech $F=h\circ g$. Łatwo wykazać, że $h(n)=(-1)^n\lceil n/2\rceil$ jest bijekcją, pozostaje więc jeszcze znaleźć funkcję $g$.

Rozważmy numerację par o~elementach całkowitych przedstawioną na rys.~\ref{fig:B.3-4} w~formie spirali.
\begin{figure}[ht]
	\begin{center}
		\includegraphics{figb.2}
	\end{center}
	\caption{Bijekcja ze zbioru $\mathbb{Z}\times\mathbb{Z}$ w~zbiór $\mathbb{N}$. Poszczególne liczby naturalne oznaczają wartości tej bijekcji dla punktów o~współrzędnych całkowitych w~układzie kartezjańskim.} \label{fig:B.3-4}
\end{figure}
Ponieważ każdemu punktowi o~współrzędnych całkowitych przypisywana jest unikalna liczba naturalna, to możemy tę spiralę potraktować jako opis funkcji~$g$. Przyjmijmy wpierw oznaczenia: $d=\max(|x|,|y|)$ oraz $D=(2d-1)^2-1$. Nieformalnie liczby te oznaczają, odpowiednio, numer ``okrążenia'' punktu $\langle0,0\rangle$ pokonywanego przez spiralę w~momencie przechodzenia przez punkt $\langle x,y\rangle$ oraz największą wartość przyjmowaną przez spiralę podczas pokonywania poprzedniego ``okrążenia''. Można teraz przyjąć następującą definicję funkcji~$g$:
\[
	g(x,y) =
	\begin{cases}
		0, & \text{jeśli $d=|x|=|y|=0$}, \\
		D+d+y, & \text{jeśli $d\ne|y|$ i~$d=x$}, \\
		D+3d-x, & \text{jeśli $d\ne0$ i~$d=y$}, \\
		D+5d-y, & \text{jeśli $d\ne|y|$ i~$d=-x$}, \\
		D+7d+x, & \text{jeśli $d\ne0$ i~$d=-y$}.
	\end{cases}
\]

Zaprezentowana tutaj spirala przypomina znaną w~literaturze \emph{spiralę Ulama}, opisaną w~\cite{ulamspiral} i~wykorzystywaną do znajdowania pewnych własności liczb pierwszych.

\subchapter{Grafy}

\exercise %B.4-1
Jeśli będziemy reprezentować zbiór pracowników przez zbiór wierzchołków $V$, a~dla każdych $u$, $v\in V$ relację ``pracownik $u$ podał rękę pracownikowi $v$'' przez zbiór krawędzi $E$, to otrzymamy graf nieskierowany $G=\langle V,E\rangle$. Sumując stopnie wszystkich wierzchołków tego grafu, otrzymamy podwojoną liczbę krawędzi, gdyż każdą krawędź policzymy dwa razy (każda krawędź jest incydentna z~dokładnie dwoma wierzchołkami). Mamy więc
\[
	\sum_{v\in V}\deg(v) = 2|E|.
\]

\exercise %B.4-2
Ścieżka z~wierzchołka $u$ do wierzchołka $v$ w~dowolnym grafie jest ciągiem wierzchołków kolejno odwiedzanych na tej ścieżce, $\langle v_0,v_1,\dots,v_n\rangle$, przy czym $v_0=u$ i~$v_n=v$. Jeśli ścieżka jest prosta, to wyrazy tego ciągu nie powtarzają się. W~każdej ścieżce można wyeliminować pewne spójne podciągi, otrzymując w~wyniku ścieżkę prostą. Jeśli podciągiem ścieżki z~$u$ do $v$ jest ciąg $\langle v_i,v_{i+1},\dots,v_{i+k},v_i\rangle$, to eliminując jego podciąg $\langle v_i,v_{i+1},\dots,v_{i+k}\rangle$, odrzucimy jedno powtórzenie $v_i$, a~tym samym podcykl ścieżki, który sprawia, że nie jest ona prosta. Po eliminacji wszystkich takich podcykli otrzymamy ścieżkę prostą. Oznacza to, że każda ścieżka zawiera ścieżkę prostą reprezentowaną przez ciąg wierzchołków pozostałych z~początkowego ciągu po eliminacji podcykli. Procedura wyznaczania ścieżki prostej jest poprawna, gdyż jeśli z~$v_i$ istnieje krawędź do $v_{i+k+1}$, to możemy wcześniej przejść tą krawędzią, zapobiegając ponownemu odwiedzeniu $v_i$.

Dowód dla cykli przeprowadzamy analogicznie z~$v_n=u$, pamiętając jednak, by nie eliminować ostatniego powtórzenia $u$, które jest wymagane do tego, by ścieżka stanowiła cykl.

\exercise %B.4-3
Z~twierdzenia~B.2 mamy, że graf $G=\langle V,E\rangle$ będący drzewem jest spójny i~acykliczny oraz że ma $|E|=|V|-1$ krawędzi. Gdy dodamy do $E$ nową krawędź, to $G$ nie będzie już drzewem, ale nadal będzie spójny -- może być zatem $|E|\ge|V|-1$. Z~kolei, gdy usuniemy z~$E$ jakąkolwiek krawędź, to rozspójnimy $G$, przez co nie może zachodzić $|E|<|V|-1$.

\exercise %B.4-4
Każdy wierzchołek grafu skierowanego lub nieskierowanego jest osiągalny z~samego siebie, ponieważ istnieje ścieżka o~długości równej 1 zawierająca tylko ten wierzchołek, zatem relacja osiągalności jest zwrotna.

Dla dowolnych wierzchołków $u$, $v$ i~$w$ grafu skierowanego lub nieskierowanego z~faktu, że $u\leadsto v$ i~$v\leadsto w$ wynika, że $u\leadsto w$. Istnieje bowiem ścieżka z~$u$ do $w$ będąca konkatenacją ciągów reprezentujących ścieżki z~$u$ do $v$ i~z~$v$ do $w$ (z~pominięciem powtórzenia $v$ między nimi).

Relacja osiągalności jest symetryczna jedynie w~grafach nieskierowanych, gdyż dla dowolnych wierzchołków $u$ i~$v$, jeśli $u\leadsto v$, to $v\leadsto u$. Ścieżka z~$v$ do $u$ powstaje przez lustrzane odbicie ścieżki z~$u$ do $v$; powstały ciąg reprezentuje poprawną ścieżkę, bo każdą krawędzią można poruszać się w~obie strony. W~grafie skierowanym krawędzie są jednokierunkowe, więc symetria nie zachodzi.

\exercise %B.4-5
\begin{figure}[ht]
	\begin{center}
		\includegraphics{figb.3}
	\end{center}
	\caption{{\sffamily\bfseries(a)} Wersja nieskierowana grafu skierowanego z~rysunku~B.2(a). {\sffamily\bfseries(b)} Wersja skierowana grafu nieskierowanego z~rysunku~B.2(b).} \label{fig:B.4-5}
\end{figure}

\exercise %B.4-6
Hipergraf $H=\langle V_H,E_H\rangle$ można reprezentować jako graf dwudzielny $G=\langle V_1\cup V_2,E\rangle$, w~którym $V_1=V_H$ oraz $V_2=E_H$. Krawędź $\langle u,v\rangle\in V_1\times V_2$ w~grafie $G$ istnieje wtedy i~tylko wtedy, gdy hiperkrawędź $v$ jest incydentna z~$u$ (hiperkrawędzie mogą być incydentne z~więcej niż dwoma wierzchołkami). Graf $G$ rzeczywiście jest dwudzielny, ponieważ nie~istnieją krawędzie pomiędzy elementami $V_1$ ani pomiędzy elementami $V_2$.

\subchapter{Drzewa}

\exercise %B.5-1
\begin{figure}[ht]
	\begin{center}
		\includegraphics{figb.4}
	\end{center}
	\caption{{\sffamily\bfseries(a)} Drzewa wolne złożone z~3 wierzchołków $A$, $B$ i~$C$. {\sffamily\bfseries(b)} Drzewa ukorzenione o~węzłach $A$, $B$ i~$C$, w~których $A$ jest korzeniem. {\sffamily\bfseries(c)} Drzewa uporządkowane o~węzłach $A$, $B$ i~$C$, w~których $A$ jest korzeniem. {\sffamily\bfseries(d)} Drzewa binarne o~węzłach $A$, $B$ i~$C$, w~których $A$ jest korzeniem.} \label{fig:B.5-1}
\end{figure}

\exercise %B.5-2
Przypuśćmy, że twierdzenie jest fałszywe, czyli że wersja nieskierowana grafu $G$ nie tworzy drzewa, a~więc posiada cykl, w~szczególności cykl prosty (\refExercise{B.4-2}). Niech $\langle v_1,v_2,\dots,v_k,v_1\rangle$ będzie takim cyklem. Graf $G$ jest acykliczny, zatem dla pewnego $1\le l\le k$ istnieją krawędzie $\langle v_l,v_{l+1}\rangle$, $\langle v_{l+2},v_{l+1}\rangle\in E$, przy czym $v_{k+1}$ utożsamiamy z~$v_1$, a~$v_{k+2}$ z~$v_2$. Wiemy z~założenia, że $v_0\leadsto v_l$ oraz $v_0\leadsto v_{l+2}$, zatem istnieją dwie różne ścieżki z~$v_0$ do $v_{l+1}$:
\[
	\langle v_0,\dots,v_l,v_{l+1}\rangle \quad\text{oraz}\quad \langle v_0,\dots,v_{l+2},v_{l+1}\rangle.
\]
Otrzymana sprzeczność prowadzi do wniosku, że wersja nieskierowana grafu $G$ istotnie stanowi drzewo.

\exercise %B.5-3
W~drzewie o~jednym węźle jest jeden liść i~brak węzłów wewnętrznych, więc przypadek bazowy zachodzi. Zauważmy, że wszystkie krawędzie pomiędzy węzłami stopnia~1, a~ich synami można ściągnąć, nie powodując zmian w~liczbie węzłów stopnia~2. W~rzeczywistości operacja ściągnięcia po wszystkich takich krawędziach pozbawia drzewo wszystkich węzłów stopnia~1, które teraz jest drzewem regularnym. Dalej będziemy zatem rozważać tylko takie drzewa.

\medskip
\noindent\textsf{\textbf{Lemat.}} \textit{Niepuste regularne drzewo binarne ma nieparzystą liczbę węzłów.}
\begin{proof}
Niech $T=\langle V,E\rangle$ będzie niepustym regularnym drzewem binarnym, a~$L\subseteq V$ -- zbiorem liści tego drzewa. Obliczmy sumę wszystkich stopni węzłów $T$ (w sensie grafowym, czyli uwzględniając ojca węzła):
\[
	\sum_{v\in V}\deg(v) = \sum_{v\in L}1+\sum_{v\in V\setminus L}\!\!\!3\;-1=3|V|-2|L|-1.
\]
Z~lematu o~podawaniu rąk (\refExercise{B.4-1}) mamy, że $\sum_{v\in V}\deg(v) = 2|E|$, a~stąd
\[
	|V| = \frac{2|E|+2|L|+1}{3}.
\]
Licznik ułamka jest nieparzysty, zatem liczba węzłów $T$ także jest nieparzysta.
\end{proof}

Korzystając z~powyższego lematu, założymy, że twierdzenie jest prawdziwe dla drzewa o~$2k-1$ węzłach ($k\ge1$) i~wykażemy jego prawdziwość dla drzewa o~$2k+1$ węzłach. Mamy więc, że liczba węzłów $w$ stopnia 2 w~regularnym drzewie binarnym o~$2k-1$ węzłach jest o~1 mniejsza od liczby jego liści $l$. Wybierając dowolny liść i~czyniąc z~niego węzeł wewnętrzny, poprzez dołączenie do niego dwóch synów, tworzymy regularne drzewo binarne o~$2k+1$ węzłach. W~nowym drzewie mamy teraz $w'=w+1$ węzłów stopnia 2 oraz $l'=(l-1)+2=l+1$ liści, więc z~założenia indukcyjnego, że $w=l-1$ dostajemy $w'=l'-1$, a~zatem twierdzenie jest prawdziwe.

\exercise %B.5-4
Udowodnimy nierówność $h\ge\lfloor\lg n\rfloor$ przez indukcję względem $h$. Dla $h=0$ drzewo posiada tylko jeden węzeł, zatem $n=1$ i~nierówność zachodzi. Załóżmy zatem, że $h\ge1$ oraz że nierówność jest spełniona dla drzewa $T$ o~wysokości $h-1$ i~posiadającego $m$ węzłów. Na głębokości $h-1$ znajduje się pewna ilość liści, powiedzmy $l$. Utworzymy teraz nowe drzewo $T'$ o~wysokości $h$, dodając pewną ilość nowych węzłów, które będą potomkami niektórych liści drzewa $T$. Oznaczmy przez $n$ liczbę węzłów drzewa $T'$. Ponieważ $l\le2^{h-1}$, to $n\le m+2l\le m+2^h$, bo każdy liść drzewa $T$ może stać się ojcem dla co najwyżej dwóch nowych węzłów. Korzystając z~założenia indukcyjnego, mamy $h-1\ge\lfloor\lg m\rfloor$, a~ponieważ dla dowolnego $h\ge1$ zachodzi $h-1=\bigl\lfloor\lg(2^h-1)\bigr\rfloor$, to musi być $m\le 2^h-1$. Wynika stąd ograniczenie na ilość węzłów drzewa $T'$:
\[
	n \le m+2^h \le 2^h-1+2^h = 2^{h+1}-1.
\]
Logarytmując i~biorąc podłogi obu stron nierówności, dostajemy
\[
	\lfloor\lg n\rfloor \le \bigl\lfloor\lg(2^{h+1}-1)\bigr\rfloor = h,
\]
a~zatem twierdzenie jest prawdziwe dla wszystkich drzew binarnych, bo każde drzewo o~wysokości $h$ można otrzymać z~pewnego drzewa o~wysokości $h-1$ poprzez opisaną wyżej operację dołączania nowych węzłów.

\exercise %B.5-5
Dowód przez indukcję względem $n$. Gdy $n=0$, to drzewo jest puste, więc $i=e=0$, zatem baza zachodzi. Załóżmy więc, że $n>0$ i~że równanie $e=i+2(n-1)$ jest spełnione przez drzewo regularne o~$n-1$ węzłach wewnętrznych. Zauważmy, że aby zwiększyć liczbę węzłów wewnętrznych o~1, należy z~dowolnego liścia na pewnej głębokości $d$ utworzyć węzeł wewnętrzny poprzez dołączenie do niego dwóch nowych liści.

Zbadajmy co się dzieje z~długościami ścieżek wewnętrznej i~zewnętrznej po takiej modyfikacji. Oznaczmy przez $e'$ oraz $i'$, odpowiednio, długość nowej ścieżki zewnętrznej i~długość nowej ścieżki wewnętrznej. Zachodzi $e'=e-d+(d+1)+(d+1)=i+2(n-1)+d+2$ oraz $i'=i+d$, a~zatem $e'=i'+2n$ i~twierdzenie jest spełnione, gdyż teraz w~drzewie jest $n$ węzłów wewnętrznych.

\exercise %B.5-6
Niech $h$ będzie wysokością drzewa binarnego $T$. Zauważmy, że możemy zwiększyć sumę wag liści drzewa $T$ poprzez dołączenie do każdego węzła o~stopniu~1 nowego węzła będącego jego potomkiem, a~jednocześnie będącego nowym liściem drzewa $T$. W~wyniku tej operacji utworzymy z~$T$ pewne regularne drzewo binarne o~wysokości $h$. Można następnie zauważyć, że liść na głębokości $d$ wnosi do sumy składnik równy $2^{-d}=2\cdot2^{-(d+1)}$, zatem uczynienie z~takiego liścia węzła wewnętrznego, poprzez dołączenie do niego nowych węzłów, nie spowoduje zmian w~sumie wag liści. Powtarzając tę czynność dla każdego liścia o~głębokości mniejszej niż $h$, utworzymy z~$T$ pełne drzewo binarne $T'$. Mamy w~nim $2^h$ liści, wszystkie na głębokości $h$, a~zatem
\[
	\sum_{x}w(x) = 2^h\cdot2^{-h} = 1,
\]
gdzie sumujemy po wszystkich liściach $x$ z~$T'$. Ponieważ suma wag liści przyjmuje maksymalną wartość w~pełnym drzewie binarnym, to na mocy powyższego wyniku w~dowolnym drzewie nie przekroczy ona 1.

\exercise %B.5-7
\note{Zarówno w~oryginalnym tekście zadania jak i~w~jego tłumaczeniu występuje błąd. Twierdzenie nie zachodzi bowiem dla drzew o~jednym liściu, dlatego poniższy dowód dotyczy drzew binarnych posiadających co najmniej dwa liście.}

\noindent Niech $T$ będzie drzewem binarnym o~$L\ge2$ liściach oraz niech $LT$ i~$RT$ będą, odpowiednio, jego lewym i~prawym poddrzewem. Niech ponadto $L_1$ i~$L_2$ stanowią, odpowiednio, liczbę liści $LT$ i~liczbę liści $RT$. Bez straty ogólności załóżmy, że $L_1\le L_2$. Jeśli $2L_1\ge L_2$, to $LT$ i~$RT$ są szukanymi poddrzewami. W~przeciwnym przypadku zachodzi $L_2>2L/3$, powtarzamy zatem powyższe rozumowanie rekurencyjnie dla drzewa $RT$. W~ciągu tej procedury w~pewnym momencie będziemy rozważać drzewo $T'$, którego większe poddrzewo $RT'$ będzie mieć nie więcej niż $2L/3$ liści. Ale ponieważ $T'$ ma więcej niż $2L/3$ liści, to liczba liści $RT'$ jest większa niż $L/3$, zatem $RT'$ jest szukanym poddrzewem.

\problems

\problem{Kolorowanie grafów} %B-1

\subproblem %B-1(a)
Zamiast dowolnych drzew rozważmy bez straty ogólności drzewa ukorzenione. Krawędzie są incydentne z~węzłami z~sąsiednich poziomów, a~więc można pokolorować drzewo w~taki sposób, żeby węzły miały kolor równy parzystości swojej głębokości.

\subproblem %B-1(b)
Przyjmijmy bez straty ogólności, że będziemy rozważać tylko grafy spójne, gdyż stwierdzenia te pozostaną prawdziwe, jeżeli będą zachodzić osobno dla każdej składowej.
\bigskip

$1.\Rightarrow 2.\!\!:$ W~grafie dwudzielnym krawędzie łączą wierzchołki między dwoma rozłącznymi zbiorami, zatem można pokolorować wierzchołki dwoma barwami w~zależności od ich przynależności do danego zbioru, uzyskując prawidłowe \compound{2}{kolorowanie}.
\bigskip

$2.\Rightarrow 3.\!\!:$ Niech $G$ będzie grafem \compound{2}{kolorowalnym} i~niech zawiera pewien cykl nieparzysty $\langle v_1,v_2,\dots,v_{2k+1},v_1\rangle$ dla pewnego $k\ge1$. Bez utraty ogólności niech $c(v_1)=0$ i~wtedy musi być $c(v_{2i})=1$ oraz $c(v_{2i+1})=0$, gdzie $1\le i\le k$. Zauważmy jednak, że dwa sąsiednie wierzchołki mają ten sam kolor, $c(v_{2k+1})=c(v_1)=0$, co przeczy założeniu o~tym, że $G$ jest \compound{2}{kolorowalny}. Wnioskujemy zatem, że $G$ nie zawiera cyklu o~długości nieparzystej.
\bigskip

$3.\Rightarrow 1.\!\!:$ Wybierzmy pewien wierzchołek $v\in V$ i~uczyńmy elementami zbioru $V_1$ wszystkie wierzchołki o~odległości parzystej od $v$ (w~sensie najkrótszej ścieżki) wraz z~nim samym. Pozostałe niech tworzą zbiór $V_2$. Ponieważ $G$ nie zawiera cyklu nieparzystego, to żaden jego wierzchołek nie sąsiaduje z~innym wierzchołkiem ze swojego zbioru, a~to oznacza, że graf $G=\langle V_1\cup V_2,E\rangle$ jest dwudzielny.

\subproblem %B-1(c)
Dowód przeprowadzimy indukcyjnie ze względu na liczbę wierzchołków w~grafie $G$. Jeśli graf ma jeden wierzchołek, to oczywiście wystarcza jeden kolor. Załóżmy więc, że $|V|\ge2$. Wybierzmy dowolny wierzchołek $v\in V$ i~rozważmy graf $G'=\bigl\langle V\setminus\{v\},E\bigr\rangle$. Na mocy założenia indukcyjnego da się go pokolorować $d+1$ barwami, gdzie $d$ to maksimum stopni wierzchołków w~$G'$. Zauważmy, że $v$ ma co najwyżej $d$ sąsiadów. Wśród $d+1$ kolorów użytych w~kolorowaniu grafu $G'$ jest więc kolor nieprzypisany żadnemu sąsiadowi wierzchołka $v$. Wybierając ten kolor dla $v$, poprawnie kolorujemy graf $G$ $d+1$ kolorami, więc dowód jest zakończony.

\subproblem %B-1(d)
W~optymalnym kolorowaniu grafu (czyli takim, które wykorzystuje możliwie najmniej kolorów), jeżeli $k$ barw wystarcza do pokolorowania grafu $G$, to dla każdej pary różnych barw muszą istnieć wierzchołki sąsiednie o~takich barwach. W~przeciwnym przypadku istniałyby dwa różne kolory, które nie miałyby sąsiednich wierzchołków, a~zatem można byłoby potraktować je jako jeden kolor, co przeczyłoby minimalności $k$. Wszystkich możliwych par wierzchołków o~różnych kolorach spośród $k$ jest zatem nie mniej niż
\[
	\binom{k}{2} = \frac{k(k-1)}{2} \ge |E|,
\]
skąd $k=O\bigl(\!\sqrt{|E|}\bigr)$ i~twierdzenie zachodzi na mocy tego, że $|E|=O(|V|)$.

\problem{Grafy znajomości} %B-2

\subproblem %B-2(a)
\textsf{\textbf{Twierdzenie.}} \textit{W~prostym grafie nieskierowanym\/ $G=\langle V,E\rangle$, w~którym\/ $|V|\ge2$, istnieją dwa wierzchołki o~tym samym stopniu.}
\begin{proof}
W~grafie $G$ o~$n\ge2$ wierzchołkach możliwymi stopniami wierzchołków są liczby 0, 1,~\dots,~$n-1$. Jeśli jednak pewien wierzchołek ma stopień równy~0, to żaden z~pozostałych nie ma stopnia $n-1$. Oznacza to, że jest $n$ wierzchołków, ale tylko co najwyżej $n-1$ liczb mogących jednocześnie być stopniami wierzchołków w~$G$, zatem istnieją pewne dwa wierzchołki o~tym samym stopniu.
\end{proof}

\subproblem %B-2(b)
\textsf{\textbf{Twierdzenie.}} \textit{Graf pełny\/ $K_3$ jest podgrafem dowolnego prostego grafu nieskierowanego\/ $G=\langle V,E\rangle$, w~którym\/ $|V|=6$, lub jego dopełnienia\/ $\overline{G}$.}
\begin{proof}
Wybierzmy pewne $v\in V$. Istnieją wtedy w~$V$ trzy inne wierzchołki $v_1$, $v_2$,~$v_3$ wszystkie sąsiednie z~$v$ albo wszystkie niesąsiednie z~$v$. Ponieważ przypadki te są symetryczne, rozważmy pierwszy z~nich. Jeśli nie istnieje wśród $v_1$, $v_2$,~$v_3$ para wierzchołków sąsiednich, to twierdzenie zachodzi. Załóżmy więc, że istnieje krawędź między pewnymi dwoma. Wtedy jednak tworzą one wraz z~$v$ graf $K_3$, a~więc również w~tym przypadku twierdzenie jest prawdziwe.
\end{proof}

Problem rozważany w~tym punkcie jest związany z~\emph{liczbami Ramseya} $R(q_1,q_2,\dots,q_k)$; powyższe twierdzenie stanowi dowód, że $R(3,3)\le6$.

\subproblem %B-2(c)
\textsf{\textbf{Twierdzenie.}} \textit{Zbiór wierzchołków\/ $V$ dowolnego prostego grafu nieskierowanego\/ $G=\langle V,E\rangle$, można podzielić na dwa rozłączne zbiory tak, aby co najmniej połowa wierzchołków sąsiednich z~wierzchołkiem\/ $v\in V$ nie należała do zbioru, do którego należy\/ $v$.}
\begin{proof}
Przypiszmy każdemu wierzchołkowi $v\in V$ wagę $d(v)$ równą różnicy liczby wierzchołków ze zbioru, do którego należy $v$ sąsiednich z~nim i~liczby wierzchołków sąsiednich z~$v$ spoza jego zbioru. Dowód sprowadza się do pokazania, że $d(v)\ge0$ dla każdego $v\in V$\!.

Zdefiniujmy teraz liczbę $\sigma=\sum_{v\in V}d(v)$ i~zastanówmy się jak można ją zmaksymalizować, wyznaczając podziały $V$ na dwa rozłączne podzbiory $V_1$ i~$V_2$. Załóżmy, że dokonaliśmy już pewnego takiego podziału i~wybierzmy pewien wierzchołek $v$ należący do zbioru $V_1$. Dowód w~przypadku gdy $v\in V_2$ przebiega symetrycznie. Zauważmy, że przenosząc wierzchołek $v$ do $V_2$ zmieniamy $d(v)$ na liczbę przeciwną. Ponadto dla każdego sąsiedniego do $v$ wierzchołka $v_1\in V_1$ jego waga $d(v_1)$ wzrasta o~2, a~dla każdego $v_2\in V_2$ sąsiedniego z~$v$ $d(v_2)$ maleje o~2. Operacja przeniesienia zwiększa zatem wartość $\sigma$ o~$-4d(v)$. Widać więc, że aby zmaksymalizować $\sigma$, należy przenosić wierzchołki o~ujemnych wagach. Ponieważ $\sigma$ nie może rosnąć w~nieskończoność (jest ograniczone od góry przez $2|E|$ w~grafach dwudzielnych), to po skończonej liczbie przenosin w~grafie $G$ wszystkie wierzchołki będą mieć wagi nieujemne, czego należało dowieść.
\end{proof}

\subproblem %B-2(d)
\textsf{\textbf{Twierdzenie} (Dirac)\textbf{.}} \textit{Jeśli dla każdego wierzchołka\/ $v$ prostego grafu nieskierowanego\/ $G=\langle V,E\rangle$, w~którym\/ $|V|=n\ge3$, zachodzi\/ $\deg(v)\ge n/2$, to\/ $G$ jest hamiltonowski.}

\medskip
\noindent Zanim zajmiemy się dowodem twierdzenia, udowodnimy następujący lemat.

\medskip
\noindent\textsf{\textbf{Lemat} (Ore)\textbf{.}} \textit{Jeśli w~prostym grafie nieskierowanym\/ $G=\langle V,E\rangle$, gdzie\/ $|V|=n\ge3$, zachodzi nierówność\/ $\deg(u)+\deg(v)\ge n$ dla każdej pary niesąsiednich wierzchołków\/ $u$ i~\/$v$, to\/ $G$ jest hamiltonowski.}
\begin{proof}
Przypuśćmy, że lemat jest fałszywy, czyli dla pewnego $n$ istnieje kontrprzykład -- graf $G$, który spełnia założenie lematu, ale nie jest hamiltonowski. Spośród wszystkich takich grafów rozpatrzmy ten, dla którego $|E|$ jest maksymalne. Jest to podgraf pełnego grafu hamiltonowskiego $K_n$. Dodanie do $G$ krawędzi z~grafu $K_n$ daje w~wyniku graf, który nadal spełnia założenie lematu i~który ma więcej niż $|E|$ krawędzi, a~więc ze względu na wybór grafu $G$, tak powstały graf będzie miał cykl Hamiltona. To znaczy, że $G$ musi mieć przynajmniej drogę Hamiltona określoną przez pewien ciąg wierzchołków $\langle v_1,v_2,\dots,v_n\rangle$. Ponieważ $G$ nie ma cyklu Hamiltona, to nie istnieje krawędź łącząca $v_1$ z~$v_n$. Z~kolei z~założenia wiemy, że $\deg(v_1)+\deg(v_n)\ge n$.

Można teraz zdefiniować podzbiory $S_1$ i~$S_n$ zbioru $\{2,3,\dots,n\}$ takie, że
\[
	S_1 = \bigl\{\,i:\{v_1,v_i\}\in E\,\bigr\} \quad\text{oraz}\quad S_n = \bigl\{\,i:\{v_{i-1},v_n\}\in E\,\bigr\}.
\]
Wtedy $|S_1|=\deg(v_1)$ i~$|S_n|=\deg(v_n)$. Ponieważ $|S_1|+|S_n|\ge n$ i~zbiór $S_1\cup S_n$ ma co najwyżej $n-1$ elementów, to zbiór $S_1\cap S_n$ musi być niepusty. Istnieje więc $i$, dla którego istnieją krawędzie $\{v_1,v_i\}$ oraz $\{v_{i-1},v_n\}$. Wtedy droga $\langle v_1,\dots,v_{i-1},v_n,v_{n-1},\dots,v_i,v_1\rangle$ jest cyklem Hamiltona w~grafie $G$. Sprzeczność -- lemat jest prawdziwy.
\end{proof}

Można teraz udowodnić główne twierdzenie.
\begin{proof}
Jeśli dla każdego $v\in V$ zachodzi $\deg(v)\ge n/2$, to $\deg(u)+\deg(v)\ge n$ dla każdych $u$,~$v\in V$ niezależnie od tego, czy są sąsiednie, czy nie, a~więc $G$ spełnia założenia powyższego lematu, czyli jest hamiltonowski.
\end{proof}

\problem{Podziały drzew} %B-3

\subproblem %B-3(a)
Niech $T=\langle V,E\rangle$ będzie drzewem binarnym, w~którym $|V|=n\ge2$. Przez \emph{krawędź dzielącą} będziemy rozumieć krawędź, po usunięciu której zbiór wierzchołków drzewa $T$ dzieli się na zbiory $A$ i~$B$ takie, że $|A|\le3n/4$ oraz $|B|\le3n/4$. Udowodnimy przez indukcję względem $n$, że w~każdym takim drzewie istnieje krawędź dzieląca.

Dla $n=2$ twierdzenie zachodzi, ponieważ w~drzewie $T$ istnieje tylko jedna krawędź, po usunięciu której dostajemy zbiory jednoelementowe. Niech zatem $n>2$ i~załóżmy, że w~drzewie o~$n-1$ wierzchołkach istnieje krawędź dzieląca $e\in E$ taka, że po podziale każdy ze zbiorów $A$ i~$B$ ma co najwyżej $3(n-1)/4$ elementów. Przyjmijmy bez utraty ogólności, że $|A|\le|B|$, co oznacza, że $|A|\le(n-1)/2$. Utwórzmy teraz nowe drzewo $T'$, dodając do $V$ nowy wierzchołek $v'$ oraz nową krawędź $\{v',v\}$ do $E$ dla pewnego $v\in V$. Niech teraz $A'$ oraz $B'$ będą zbiorami wierzchołków w~nowym drzewie utworzonymi w~wyniku podziału krawędzią $e$. Jeśli $v\in A$, to $A'=A\cup\{v'\}$ oraz $B'=B$. Oczywiście $|B'|<3n/4$, zbadajmy zatem $A'$:
\[
	|A'| = |A|+1 \le \frac{n-1}{2}+1 = \frac{n+1}{2} \le \frac{3n}{4},
\]
co jest prawdą, o~ile $n\ge2$, zatem w~tym przypadku twierdzenie zachodzi.

Niech teraz $v\in B$. Stąd $A'=A$ i~$B'=B\cup \{v'\}$, ale z~założenia $|B'|=|B|+1\le(3n+1)/4$, a~zatem $|B'|$ może przekroczyć $3n/4$, co oznacza, że musimy znaleźć inną krawędź dzielącą dla drzewa $T'$ w~przypadku, gdy $|B|=(3n-3)/4$, przy czym $n\ge5$.

Rozważmy drzewo $T'$ przedstawione na rys.~\ref{fig:B-3a}.
\begin{figure}[ht]
	\begin{center}
		\includegraphics{figb.5}
	\end{center}
	\caption{Drzewo $T'$ z~drugiego przypadku dowodu.} \label{fig:B-3a}
\end{figure}
Niech $u_1\in B$ oraz $e=\{u_1,u_2\}$. Oprócz $u_1$ do zbioru $B$ należą wierzchołki ze zbiorów $V_1$ i~$V_2$, a~do zbioru $A$ -- wierzchołek $u_2$ oraz wierzchołki ze zbiorów $V_3$ i~$V_4$. Załóżmy bez straty ogólności, że $|V_1|\le|V_2|$. Zbiór $V_2$ jest niepusty, gdyż $|B'|\ge4$, istnieje zatem krawędź $e'=\{u_1,w\}$, gdzie $w\in V_2$. Pokażemy, że jest to krawędź dzieląca drzewa $T'$. Rozważmy w~tym celu zbiory $A''$ i~$B''$, na które krawędź $e'$ dzieli zbiór $V\cup\{v'\}$. Mamy
\[
	|B''| = |V_2| \le |B| = \frac{3n-3}{4} < \frac{3n}{4}
\]
oraz
\[
	|A''| = \bigl|A\cup\{u_1\}\cup V_1\bigr| \le (n-1-|B|)+1+\frac{|B|}{2} = n-\frac{|B|}{2} = \frac{5n+4}{8}.
\]
Skorzystaliśmy z~tego, że $|A|+|B|=n-1$ oraz $|V_1|\le|B|/2$. Nierówność $|A''|\le3n/4$ zachodzi, o~ile $n\ge4$, więc istotnie $e'$ jest krawędzią dzielącą drzewa $T'$.

Rozpatrzyliśmy wszystkie przypadki, zatem na mocy indukcji twierdzenie zachodzi dla każdego drzewa binarnego $T$.

\subproblem %B-3(b)
Stała $3/4$ jest wystarczająca do dokonywania zrównoważonych podziałów, jak to wykazaliśmy w~punkcie~(a). Przykład drzewa binarnego z~rys.~\ref{fig:B-3b} pokazuje, że nie można rozważać mniejszej stałej niż $3/4$. Usuwając dowolną krawędź tego drzewa, dzielimy zbiór jego wierzchołków na podzbiory, z~których jeden ma trzy elementy.
\begin{figure}[ht]
	\begin{center}
		\includegraphics{figb.6}
	\end{center}
	\caption{Drzewo binarne, w~którym najbardziej zrównoważony podział tworzy podzbiór zawierający 3 wierzchołki.} \label{fig:B-3b}
\end{figure}

\subproblem %B-3(c)
Rozważmy następującą procedurę podziału zbioru wierzchołków. Na początku wynikowe zbiory $A$ i~$B$ są puste. Usuwając jedną krawędź, możemy podzielić \compound{$n$}{elementowy} zbiór~wierzchołków drzewa na dwa podzbiory, z~których większy będzie składać się z~co najwyżej $3n/4$ wierzchołków, co wynika na podstawie punktu~(a). Mniejszy podzbiór sumujemy z~jednym ze zbiorów wynikowych, natomiast większy z~nich będzie podlegał dalszemu podziałowi. Podczas działania procedury pilnujemy, aby liczby elementów $A$ i~$B$ nie przekroczyły $\lceil n/2\rceil$. Procedurę podziału zakończymy w~momencie, gdy jeden z~tych zbiorów będzie zawierał $\lceil n/2\rceil$ elementów, gdyż drugi zbiór zawiera wtedy $\lfloor n/2\rfloor$ elementów, czyli jest to podział, jakiego wymagano.

Zauważmy, że maksymalną liczbę podziałów dla zadanego drzewa wykonamy w~przypadku, gdy po~każdym kroku zostanie do podziału zbiór stanowiący $3/4$ zbioru z~poprzedniego kroku. Niech $k$ będzie taką maksymalną liczbą podziałów drzewa o~$n$ wierzchołkach. Zachodzi wtedy
$(3/4)^kn = 1,$
ponieważ singletonu nie trzeba dalej dzielić. Stąd $k=\log_{4/3}n$, a~zatem należy usunąć co najwyżej $k=O(\lg n)$ krawędzi.

\endinput
