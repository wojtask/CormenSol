\problem{Dzbanki} %8-4

\subproblem %8-4(a)
Dla każdego czerwonego dzbanka wystarczy przejrzeć wszystkie dotychczas niesparowane niebieskie dzbanki.
Po znalezieniu pasującego możemy wyłączyć go z~poszukiwań dla kolejnych czerwonych dzbanków.
Łatwo sprawdzić, że ta metoda pozwala na pogrupowanie wszystkich dzbanków w~pary za pomocą $\Theta(n^2)$ porównań.

\subproblem %8-4(b)
Ponumerujmy czerwone dzbanki kolejnymi liczbami całkowitymi od 1 do $n$ i~analogicznie dzbanki niebieskie.
Problem sprowadza się do znalezienia takiej permutacji $\pi$ niebieskich dzbanków, w~której $i$\nbhyphen ty dzbanek czerwony jest tej samej pojemności, co $\pi(i)$\nbhyphen ty dzbanek niebieski.

Zilustrujemy za pomocą drzewa decyzyjnego działanie algorytmu wyznaczającego tę permutację.
Każdy węzeł w~tym drzewie będzie odpowiadać porównaniu pewnego dzbanka czerwonego z~pewnym dzbankiem niebieskim.
Wynikiem każdego takiego porównania jest jedna z~trzech możliwości: dzbanek czerwony ma mniejszą pojemność niż dzbanek niebieski, dzbanek czerwony ma większą pojemność niż dzbanek niebieski albo oba dzbanki mają tę samą pojemność.
Z~tego powodu drzewo decyzyjne jest drzewem 3\nbhyphen arnym o~$n!$ osiągalnych liściach, bo tyle jest permutacji niebieskich dzbanków.
Wysokość $h$ tego drzewa oznacza pesymistyczną liczbę porównań wykonywanych przez algorytm.
Każdy węzeł w~tym drzewie ma co najwyżej 3 synów, prawdziwa jest zatem nierówność $3^h\ge n!$, skąd
\[
	h \ge \log_3(n!) = \frac{\lg(n!)}{\lg3} = \Omega(n\lg n).
\]
A~zatem dowolny algorytm rozwiązujący ten problem wykonuje co najmniej $\Omega(n\lg n)$ porównań.

\subproblem %8-4(c)
Rozwiązanie tego problemu będzie opierać się na idei algorytmu quicksort.
Zauważmy, że posortowanie obu ciągów dzbanków względem ich pojemności natychmiast prowadzi do rozwiązania problemu -- grupujemy wówczas w~pary dzbanek czerwony z~dzbankiem niebieskim stojące na tych samych pozycjach w~swoich ciągach.
Niech $R$ będzie tablicą dzbanków czerwonych, a~$B$ tablicą dzbanków niebieskich.
Poniższa procedura stanowi adaptację algorytmu quicksort do omawianego problemu.
\begin{codebox}
\Procname{$\proc{Jugs-Match}(R,B,p,r)$}
\li	\If $p<r$
\li		\Then $q\gets\proc{Jugs-Partition}(R,B,p,r)$
\li			$\proc{Jugs-Match}(R,B,p,q-1)$
\li			$\proc{Jugs-Match}(R,B,q+1,r)$
		\End
\end{codebox}

Poniżej znajduje się zmodyfikowana procedura \proc{Randomized-Partition}.
Przyjmuje ona dwie tablice $R$ i~$B$, z~których jedna składa się z~dzbanków o~takich samych pojemnościach jak druga i~w~każdej z~nich wszystkie pojemności są parami różne.
Potraktujemy jakiekolwiek porównanie dwóch dzbanków jak odpowiednie porównanie ich pojemności.
Procedura dokonuje podziału obu tablic na podstawie losowo wybranego elementu rozdzielającego, przy czym nie porównuje elementów z~tej samej tablicy.
Zwracanym wynikiem jest pozycja elementu rozdzielającego (która jest identyczna w~obu wynikowych tablicach).
\begin{codebox}
\Procname{$\proc{Jugs-Partition}(R,B,p,r)$}
\li	zamień $R[r]\leftrightarrow R[\proc{Random}(p,r)]$
\li	$x\gets R[r]$
\li	$i\gets p$
\li	\While $B[i]\ne x$
\li		\Do $i\gets i+1$
		\End
\li	zamień $B[i]\leftrightarrow B[r]$
\li	$i\gets p-1$ \label{li:jugs-partition-first-partition-begin}
\li	\For $j\gets p$ \To $r-1$ \label{li:jugs-partition-for1-begin}
\li		\Do \If $B[j]<x$
\li				\Then $i\gets i+1$
\li					zamień $B[i]\leftrightarrow B[j]$
				\End
		\End \label{li:jugs-partition-for1-end}
\li	zamień $B[i+1]\leftrightarrow B[r]$ \label{li:jugs-partition-first-partition-end}
\li	$x\gets B[i+1]$
\li	powtórz kroki z~wierszy \ref{li:jugs-partition-first-partition-begin}\nbendash\ref{li:jugs-partition-first-partition-end} dla tablicy $R$ \label{li:jugs-partition-second-partition}
\li	\Return $i+1$
\end{codebox}

Procedura wybiera najpierw losowo element rozdzielający $x$ z~tablicy $R[p\twodots r]$.
Dzbanki o~pojemnościach równych pojemności dzbanka $x$ są dla uproszczenia późniejszych kroków przenoszone na ostatnie pozycje swoich tablic.
Następnie pętla \kw{for} z~wierszy \ref{li:jugs-partition-for1-begin}\nbendash\ref{li:jugs-partition-for1-end} dokonuje podziału tablicy $B[p\twodots r]$ względem $x$.
Po zakończeniu pętli $B[r]=x$, podtablica $B[p\twodots i]$ zawiera dzbanki o~pojemnościach mniejszych niż $x$, a~podtablica $B[i+1\twodots r-1]$ -- dzbanki o~pojemnościach większych niż $x$.
Wystarczy jeszcze zamienić element rozdzielający z~$B[i+1]$, aby zakończyć podział tablicy $B$.
W~celu przeprowadzenia podziału tablicy $R$ algorytm wykonuje analogiczne kroki z~wierszy \ref{li:jugs-partition-first-partition-begin}\nbendash\ref{li:jugs-partition-first-partition-end}, ale z~elementem rozdzielającym $x$ pobranym z~tablicy $B$.

Udowodnimy teraz, że oczekiwana liczba porównań wykonywanych przez algorytm \proc{Jugs-Match} wynosi $O(n\lg n)$.
Nasza analiza będzie opierać się na analizie średniego przypadku algorytmu quicksort.
Niech $\langle r_1,r_2,\dots,r_n\rangle$ i~$\langle b_1,b_2,\dots,b_n\rangle$ będą ciągami dzbanków, odpowiednio, czerwonych i~niebieskich, o~rosnących pojemnościach.
Definiujemy zmienną losową wskaźnikową
\[
    X_{ij} = \I(\text{$r_i$ jest porównywane z~$b_j$}).
\]
Zauważmy, że dane elementy $r_i$ i~$b_j$ są porównywane ze sobą co najwyżej raz.
Po porównaniu $r_i$ z~elementami fragmentu $B$ (w~tym być może z~$b_j$) $r_i$ jest umieszczane na właściwym miejscu w~tablicy $R$ i~nie uczestniczy w~późniejszych wywołaniach procedury \proc{Jugs-Partition}.
Analogiczne rozumowanie stosuje się do $b_j$.

Z~powyższej obserwacji wynika, że całkowita liczba porównań wykonywanych w~algorytmie wynosi
\[
    X = \sum_{i=1}^{n-1}\sum_{j=i+1}^nX_{ij}.
\]
Biorąc wartości oczekiwane obu stron, dostajemy
\[
    \E(X) = \sum_{i=1}^{n-1}\sum_{j=i+1}^n\Pr(\text{$r_i$ jest porównywane z~$b_j$}).
\]

Pozostaje teraz tylko obliczyć $\Pr(\text{$r_i$ jest porównywane z~$b_j$})$.
Jeśli jako element rozdzielający wybrane zostanie $x$ takie, że $r_i<x<b_j$, to $r_i$ zostanie umieszczone w~lewej części podziału podtablicy $R$, a~$b_j$ -- w~prawej części podziału podtablicy $B$.
A~więc elementy $r_i$ i~$b_j$ nigdy więcej nie będą porównywane.
Oznaczmy przez $R_{ij}$ zbiór $\{r_i,r_{i+1},\dots,r_j\}$.
Wówczas elementy $r_i$ i~$b_j$ są porównywane wtedy i~tylko wtedy, gdy pierwszym elementem wybranym jako rozdzielający ze zbioru $R_{ij}$ jest $r_i$ albo $r_j$.
To, że zostanie nim dowolny ustalony element zbioru $R_{ij}$, zachodzi z~jednakowym prawdopodobieństwem i~na mocy faktu, że $|R_{ij}|=j-i+1$, szanse na zajście tego zdarzenia wynoszą $1/(j-i+1)$.
Mamy zatem
\begin{align*}
    \Pr(\text{$r_i$ jest porównywane z~$b_j$}) &= \Pr(\text{$r_i$ lub $r_j$ jest pierwszym elementem rozdzielającym z~$R_{ij}$}) \\
	&= \Pr(\text{$r_i$ jest pierwszym elementem rozdzielającym z~$R_{ij}$}) \\
	&\quad {}+\Pr(\text{$r_j$ jest pierwszym elementem rozdzielającym z~$R_{ij}$}) \\
	&= \frac{1}{j-i+1}+\frac{1}{j-i+1} \\
	&= \frac{2}{j-i+1}.
\end{align*}
Wstawiając ten wynik do wzoru na $\E(X)$ i~szacując od góry, dostajemy $\E(X)=O(n\lg n)$, co stanowi oczekiwaną liczbę porównań wykonywanych w~algorytmie \proc{Jugs-Match}.

Pesymistyczny przypadek zachodzi wtedy, gdy na każdym poziomie rekursji wybór elementów rozdzielających prowadzi do utworzenia podziałów najbardziej niezrównoważonych, czyli z~tablicy o~$n$ elementach zostaje utworzona w~wyniku podziału podtablica o~$n-1$ elementach i~podtablica pusta.
Wówczas algorytm wykonuje $O(n^2)$ porównań.
