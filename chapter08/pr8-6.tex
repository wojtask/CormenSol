\problem{Dolna granica dla scalania posortowanych list} %8-6

\subproblem %8-6(a)
Spośród $2n$ różnych liczb możemy wybrać $n$ z~nich do pierwszej listy, a~pozostałe $n$ -- do drugiej listy.
Ponieważ każdy taki podział jednoznacznie determinuje parę posortowanych list i~na odwrót, to liczba sposobów utworzenia tej pary list wynosi $\binom{2n}{n}$.

\subproblem %8-6(b)
Działanie algorytmu scalania list można przedstawić, korzystając z~drzewa decyzyjnego, w~którym każdy węzeł odpowiada jednemu porównaniu elementów list.
Wszystkie elementy są różne, mamy więc tylko dwa możliwe wyniki każdego z~porównań, a~zatem drzewo decyzyjne jest drzewem binarnym.
Na podstawie poprzedniego punktu mamy, że liczba możliwych układów dwóch \singledash{$n$}{elementowych} list wejściowych, z~połączenia których powstaje wynikowa lista o~$2n$ elementach, wynosi $\binom{2n}{n}$.
W~drzewie decyzyjnym jest więc tyleż osiągalnych liści.
Wyznaczając oszacowanie wysokości $h$ tego drzewa, znajdziemy ograniczenie dla maksymalnej liczby porównań wykonanych przez algorytm scalania list.
Mamy $2^h\ge\binom{2n}{n}$ i~korzystając z~\refExercise{C.1-13}, dostajemy
\[
    h \ge \lg\binom{2n}{n} = \lg\biggl(\frac{2^{2n}}{\sqrt{\pi n}}(1+O(1/n))\biggr) \ge \lg\frac{2^{2n}}{\sqrt{\pi n}} = 2n-\frac{\lg\pi}{2}-\frac{\lg n}{2} = 2n-o(n).
\]

\subproblem %8-6(c)
Niech $\langle a_1,a_2,\dots,a_n\rangle$ oraz $\langle b_1,b_2,\dots,b_n\rangle$ będą listami wejściowymi.
Załóżmy, że w~liście wynikowej element $a_i$ sąsiaduje z~elementem $b_j$ dla $1\le i$, $j\le n$.
Jeśli algorytm, scalając obie listy wejściowe, porównywałby $a_i$ z~elementem $b_k$, gdzie $1\le k\le j-1$, to wynikiem porównania byłoby $b_k<a_i$, ale stąd nie wynikałoby ani $b_j<a_i$, ani $a_i<b_j$.
Analogicznie przy porównaniu $a_i$ z~elementem $b_k$, gdzie $j+1\le k\le n$.
A~zatem, aby rozstrzygnąć, w~jakiej kolejności powinny wystąpić $a_i$ i~$b_j$ w~liście wynikowej, algorytm musi porównać te dwa elementy ze sobą.

\subproblem %8-6(d)
Jeśli do wynikowej listy będą trafiać elementy z~pierwszej listy na przemian z~elementami z~drugiej, to każde dwa sąsiednie elementy w~wynikowej liście będą pochodzić z~dwóch różnych list wejściowych.
Takich par sąsiadujących elementów będzie w~sumie $2n-1$.
Na mocy poprzedniego punktu wnioskujemy, że w~tym przypadku algorytm scalania list wykona tyleż porównań.
