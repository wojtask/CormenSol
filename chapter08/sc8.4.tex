\subchapter{Sortowanie kubełkowe}

\exercise %8.4-1
Na rys.\ \ref{fig:8.4-1} zostało przedstawione działanie sortowania kubełkowego dla tablicy $A$.
\begin{figure}[!ht]
	\centering \begin{tikzpicture}[
	array/.append style = {nodes={fill=black!20, inner sep=2pt, anchor=center}},
	double cell/.style = {array, nodes={inner sep=1pt, minimum size=4.5mm}, node distance=5mm},
	outer/.append style = {node distance=10mm},
	arrow/.style = {->, >=stealth}
]

\node[outer] (pic a) {
\begin{tikzpicture}
	\matrix[array] (arr) {0{,}79 \\ 0{,}13 \\ 0{,}16 \\ 0{,}64 \\ 0{,}39 \\ 0{,}20 \\ 0{,}89 \\ 0{,}53 \\ 0{,}71 \\ 0{,42} \\};
	\foreach \x in {1, ..., 10} {
		\node[index node, left=1mm of arr-\x-1] {\x};
	}
	\node[above=2mm of arr-1-1.north] {$A$};
\end{tikzpicture}
};

\node[outer, right=of pic a] (pic b) {
\begin{tikzpicture}
	\matrix[array] (arr) { \\ \\ \\ \\ \\ \\ \\ \\ \\ \\};
	\foreach[count=\i] \x in {0, ..., 9} {
		\node[index node, left=1mm of arr-\i-1] {\x};
	}
	\node[above=2mm of arr-1-1.north] {$B$};
	
	\foreach \x/\val in {2/13, 3/20, 4/39, 5/42, 6/53, 7/64, 8/71, 9/89} {
		\matrix[double cell, right=of arr-\x-1, ampersand replacement=\&] (cell \x) {0{,}\val \& \\};
		\draw[arrow] (arr-\x-1.center) -- (cell \x-1-1);
	}
	\matrix[double cell, right=of cell 2-1-2] (cell 2a) {0{,}16 & \\};
	\matrix[double cell, right=of cell 8-1-2] (cell 8a) {0{,}79 & \\};
	\draw[arrow] (cell 2-1-2.center) -- (cell 2a-1-1);
	\draw[arrow] (cell 8-1-2.center) -- (cell 8a-1-1);
	
	\draw (arr-1-1.south west) + (1mm, 1mm) -- +(4mm, 4mm);
	\draw (arr-10-1.south west) + (1mm, 1mm) -- +(4mm, 4mm);
	\foreach \x in {3, ..., 7, 9, 2a, 8a} {
		\draw (cell \x-1-2.south west) + (1mm, 1mm) -- +(3.5mm, 3.5mm);
	}
\end{tikzpicture}
};

\node[subpicture label, below=2mm of pic a] {(a)};
\node[subpicture label, below=2mm of pic b] {(b)};

\end{tikzpicture}

	\caption{Działanie procedury \proc{Bucket-Sort} dla tablicy $A=\langle0{,}79$, $0{,}13$, $0{,}16$, $0{,}64$, $0{,}39$, $0{,}20$, $0{,}89$, $0{,}53$,\! $0{,}71$,\! $0{,}42\rangle$.
{\sffamily\bfseries(a)} Wejściowa tablica $A$.
{\sffamily\bfseries(b)} Tablica $B$ zawierająca posortowane listy (kubełki) po wykonaniu pętli w~wierszach \doubledash{4}{5}.} \label{fig:8.4-1}
\end{figure}

\exercise %8.4-2
Pesymistyczny przypadek dla algorytmu sortowania kubełkowego zachodzi, gdy wszystkie elementy z~wejściowej tablicy trafią do tego samego kubełka.
Mamy wtedy do posortowania pojedynczą listę o~długości $n$, co zajmuje czas $O(n^2)$.

W~celu uporządkowania kubełków, zamiast sortowania przez wstawianie, można użyć algorytmu sortującego, który wykonuje $O(n\lg n)$ operacji w~przypadku pesymistycznym, np.\ sortowanie przez scalanie.
Aby obliczyć średni czas działania sortowania kubełkowego w~takim wariancie, podstawiamy do wzoru (8.1) ograniczenie na czas działania sortowania przez scalanie:
\[
	\E(T(n)) = \Theta(n)+\sum_{i=0}^{n-1}O(\E(n_i\lg n_i)) \le O(n)+\sum_{i=0}^{n-1}O(\E(n_i^2)) = O(n).
\]
Ponadto pętla \kw{for} w~wierszach \doubledash{2}{3} zajmuje czas liniowy, skąd mamy $\E(T(n))=\Omega(n)$.
A~zatem oczekiwany czas działania sortowania kubełkowego po modyfikacji pozostaje liniowy.

\exercise %8.4-3
Szanse nieuzyskania orła w~dwóch rzutach monetą wynoszą $1/4$, co jest równe szansie uzyskania dwóch orłów w~dwóch rzutach.
Dokładnie jednego orła można uzyskać z~prawdopodobieństwem równym $1/2$.
Na podstawie tych wartości obliczamy:
\begin{align*}
	\E(X^2) &= 0^2\cdot\Pr(X=0)+1^2\cdot\Pr(X=1)+2^2\cdot\Pr(X=2) = 3/2, \\
	\E^2(X) &= \bigl(0\cdot\Pr(X=0)+1\cdot\Pr(X=1)+2\cdot\Pr(X=2)\bigr)^2 = 1.
\end{align*}

\exercise %8.4-4
Algorytm będzie opierał się na pomyśle z~sortowania kubełkowego z~$n$ kubełkami.
Podzielimy koło jednostkowe na $n$ obszarów o~równych polach reprezentujących przedziały odległości od środka koła i~z~każdym takim obszarem skojarzymy inny kubełek.
Dzięki temu punkty będą umieszczane w~poszczególnych kubełkach z~jednakowym prawdopodobieństwem.
Łatwo zauważyć, że obszary te będą pierścieniami kołowymi (z~wyjątkiem jednego, który będzie kołem) o~jednakowych powierzchniach równych $\pi/n$.
Pozostaje tylko wyznaczyć ich promienie.

Ponumerujmy obszary (kubełki) kolejnymi liczbami całkowitymi od 0 do $n-1$ w~kolejności od środka do brzegu koła jednostkowego i~oznaczmy przez $r_j$ długość promienia wewnętrznego pierścienia kołowego stanowiącego \singledash{$j$}{ty} obszar.
Obszar zerowy jest kołem o~polu $\pi/n$, więc $r_1=\sqrt{1/n}$.
Ponieważ suma tego koła z~pierścieniem do niego przylegającym jest kołem o~polu $2\pi/n$, to stąd mamy $r_2=\sqrt{2/n}$.
Rozumowanie przeprowadzamy dla pozostałych pierścieni, otrzymując $r_j=\sqrt{j/n}$ dla każdego $j=1$, 2, \dots, $n-1$.
Przyjmijmy ponadto $r_0=0$ oraz $r_n=1$.

Punkty sortowane są względem ich odległości od środka koła jednostkowego przy wykorzystaniu opisanych kubełków.
Dla każdego punktu $p_i=\langle x_i,y_i\rangle$ obliczane jest $d_i=\sqrt{x_i^2+y_i^2}$ i~jeśli zachodzi $r_j<d_i\le r_{j+1}$, to punkt umieszczany jest w~kubełku o~numerze $j$.

\exercise %8.4-5
Niech $X$ będzie zmienną losową o~ciągłej dystrybuancie $P_X$.
Definiujemy nową zmienną losową $Y=P_X(X)$ o~dystrybuancie $P_Y$.
Wówczas, dla $0<y<1$, mamy
\[
    P_Y(y) = \Pr(Y\le y) = \Pr(P_X(X)\le y) = \Pr(X\le P_X^{-1}(y)) = P_X(P_X^{-1}(y)) = y.
\]
W~uzasadnieniu korzystamy z~funkcji odwrotnej do dystrybuanty $P_X$, jednak ta ostatnia może nie być ściśle rosnąca i~funkcja do niej odwrotna może nie być poprawnie zdefiniowana.
Dlatego funkcję $P_X^{-1}$ definiujemy tutaj jako
\[
	P_X^{-1}(y) = \inf\{\,x\in\mathbb{R}:P_X(x)\ge y\,\} \quad\text{dla $0<y<1$}.
\]
Pokazaliśmy, że $P_Y$ jest identycznością na $(0,1)$, a~zatem $Y$ jest zmienną losową o~ciągłym rozkładzie jednostajnym w~tym przedziale.

W~naszym problemie dla każdej zmiennej losowej $X_i$ o~dystrybuancie $P$, gdzie $i=1$, 2, \dots, $n$, wyznaczamy nową zmienną losową $Y_i=P(X_i)$.
Zgodnie z~powyższym opisem każda nowa zmienna ma rozkład jednostajny w~przedziale $(0,1)$, możemy zatem posortować je, stosując sortowanie kubełkowe o~$n$ kubełkach.
Jako wynik dostaniemy pewną permutację $\langle Y_{\pi(1)},Y_{\pi(2)},\dots,Y_{\pi(n)}\rangle$.
Rozwiązanie problemu stanowi wówczas ciąg $\langle X_{\pi(1)},X_{\pi(2)},\dots,X_{\pi(n)}\rangle$.

Wyznaczenie nowego ciągu zmiennych losowych zabiera czas $\Theta(n)$, gdyż zakładamy, że wartości dystrybuanty $P$ można obliczać w~czasie stałym.
Również sortowanie kubełkowe $n$ elementów działa średnio w~czasie $\Theta(n)$, zatem w~średnim przypadku cała procedura zajmuje czas liniowy.
