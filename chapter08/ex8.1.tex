\subchapter{Dolne ograniczenia dla problemu sortowania}

\exercise %8.1-1
Łatwo zauważyć, że do znalezienia uporządkowania $n$ elementów ciągu wejściowego potrzebnych jest w~najlepszym przypadku $n-1$ porównań.
Sytuacja zachodzi np.\ dla ciągu niemalejącego.
A~zatem najmniejszą głębokością liścia w~drzewie decyzyjnym jest $n-1$.

\exercise %8.1-2
Górne oszacowanie znajdujemy w~prosty sposób:
\[
	\lg(n!) = \sum_{k=1}^n\lg k \le \sum_{k=1}^n\lg n = n\lg n = O(n\lg n).
\]
Aby uzyskać oszacowanie dolne, rozdzielamy sumę, korzystając z~faktu, że $n=\lfloor n/2\rfloor+\lceil n/2\rceil$:
\[
	\lg(n!) = \sum_{k=1}^n\lg k = \sum_{k=1}^{\lfloor n/2\rfloor}\lg k+\sum_{k=\lceil n/2\rceil}^n\lg k.
\]
W~pierwszej sumie po prawej ograniczamy $\lg k$ od dołu przez $\lg1$, a~w~drugiej -- przez $\lg\lceil n/2\rceil$:
\[
	\lg(n!) \ge \sum_{k=1}^{\lfloor n/2\rfloor}\lg1+\sum_{k=\lceil n/2\rceil}^n\lg\lceil n/2\rceil \ge 0+(n/2)\lg\lceil n/2\rceil \ge (n/2)\lg(n/2) = \Omega(n\lg n).
\]

\exercise %8.1-3
Jeśli sortowanie działa w~czasie $\Theta(n)$ dla $l$ permutacji wejściowych długości $n$, to drzewo decyzyjne składające się z~gałęzi odpowiadających tylko tym permutacjom, ma wysokość $h=\Theta(n)$.
Powtarzając rozumowanie z~dowodu tw.\ 8.1, dostajemy nierówność $2^h\ge l$, a~stąd $h\ge\lg l$.
Pozostaje więc sprawdzić, czy $h$ jest asymptotycznie ograniczone funkcją liniową względem $n$ dla poszczególnych wartości przyjmowanych przez $l$:
\begin{itemize}
	\item jeśli $l=n!/2$, to $h\ge\lg l=\lg(n!/2)=\lg(n!)-1=\Omega(n\lg n)$;
	\item jeśli $l=n!/n$, to $h\ge\lg l=\lg(n!/n)=\lg(n!)-\lg n=\Omega(n\lg n)$;
	\item jeśli $l=n!/2^n$, to $h\ge\lg l=\lg(n!/2^n)=\lg(n!)-n=\Omega(n\lg n)$.
\end{itemize}
W~każdym badanym przypadku $h=\Omega(n\lg n)$, zatem nie istnieje algorytm sortujący za pomocą porównań i~działający w~czasie liniowym dla poszczególnych ilości permutacji wejściowych.

\exercise %8.1-4
Rozważmy drzewo decyzyjne reprezentujące sortowanie takiego częściowo uporządkowanego ciągu.
Każdy podciąg może wystąpić na wyjściu jako jedna z~$k!$ możliwych permutacji.
Ponieważ jest $n/k$ podciągów, to w~drzewie decyzyjnym znajduje się co najmniej $(k!)^{n/k}$ osiągalnych liści.
Na mocy faktu, że drzewo decyzyjne o~wysokości $h$ nie może mieć więcej niż $2^h$ liści, dostajemy
\[
	2^h \ge (k!)^{n/k},
\]
skąd uzyskujemy dolne oszacowanie na ilość porównań wykonywanych w~tym algorytmie:
\[
	h \ge \lg\bigl((k!)^{n/k}\bigr) = (n/k)\lg(k!) = (n/k)\cdot\Omega(k\lg k) = \Omega(n\lg k).
\]
