\subchapter{Dolne ograniczenia dla problemu sortowania}

\exercise %8.1-1
Rozważmy $n$\nbhyphen elementowy ciąg wejściowy $\langle a_1,a_2,\dots,a_n\rangle$.
Jeśli w~szczególności ciąg byłby uporządkowany rosnąco, to w~celu stwierdzenia tego faktu należałoby porównać elementy w~każdej parze postaci $\langle a_i,a_{i+1}\rangle$ dla $i=1$, 2, \dots, $n-1$, czyli wykonać co najmniej $n-1$ porównań.
Dopóki dana para elementów nie jest porównana, dopóty nie jest znana informacja o~wzajemnej kolejności elementów z~tej pary.

Stąd najmniejszą głębokością liścia w~drzewie decyzyjnym dla ciągu $n$\nbhyphen elementowego jest $n-1$.

\exercise %8.1-2
Oszacowanie górne:
\[
	\lg(n!) = \sum_{k=1}^n\lg k \le \sum_{k=1}^n\lg n = n\lg n = O(n\lg n).
\]
Oszacowanie dolne:
\[
	\lg(n!) = \sum_{k=1}^n\lg k \ge \sum_{k=\lceil n/2\rceil}^n\lg k \ge \sum_{k=\lceil n/2\rceil}^n\lg\lceil n/2\rceil \ge \lfloor n/2\rfloor\lg\lceil n/2\rceil = \Omega(n\lg n).
\]

\exercise %8.1-3
Załóżmy, że istnieje algorytm sortujący za pomocą porównań, który dla $m$ permutacji wejściowych o~długości $n$ działa w~czasie $O(n)$.
Rozważmy drzewo decyzyjne o~liściach odpowiadającym tylko tym $m$ permutacjom mające, zgodnie z~założeniem, wysokość $h=O(n)$.
Powtarzając rozumowanie z~dowodu tw.\ 8.1, dostajemy nierówność $2^h\ge m$, a~stąd $h\ge\lg m$.
Zbadajmy asymptotyczne tempo wzrostu $h$ w~zależności od $n$ dla poszczególnych wartości $m$:
\begin{enumerate}
	\renewcommand{\labelenumi}{(\roman{enumi})}
	\item jeśli $m=n!/2$, to $h\ge\lg m=\lg(n!/2)=\lg(n!)-1=\Omega(n\lg n)$;
	\item jeśli $m=n!/n$, to $h\ge\lg m=\lg(n!/n)=\lg(n!)-\lg n=\Omega(n\lg n)$;
	\item jeśli $m=n!/2^n$, to $h\ge\lg m=\lg(n!/2^n)=\lg(n!)-n=\Omega(n\lg n)$.
\end{enumerate}
W~każdym badanym przypadku $h=\Omega(n\lg n)$, co stoi w~sprzeczności z~założeniem.

\exercise %8.1-4
Rozważmy algorytm sortowania za pomocą porównań takiego częściowo uporządkowanego ciągu oraz odpowiadające mu drzewo decyzyjne.
Każdy podciąg może wystąpić na wyjściu tego algorytmu jako jedna z~$k!$ możliwych permutacji.
Ponieważ jest $n/k$ podciągów, to w~drzewie decyzyjnym znajduje się co najmniej $(k!)^{n/k}$ osiągalnych liści.
Na mocy faktu, że drzewo decyzyjne o~wysokości $h$ ma co najwyżej $2^h$ liści, dostajemy
\[
	2^h \ge (k!)^{n/k},
\]
skąd uzyskujemy dolne oszacowanie na ilość porównań wykonywanych w~tym algorytmie:
\[
	h \ge \lg\bigl((k!)^{n/k}\bigr) = (n/k)\lg(k!) = (n/k)\cdot\Omega(k\lg k) = \Omega(n\lg k).
\]
