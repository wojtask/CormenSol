\subchapter{Sortowanie pozycyjne}

\exercise %8.3-1
Przebieg działania procedury \proc{Radix-Sort} dla podanej listy słów został przedstawiony na rys.\ \ref{fig:8.3-1}.
\begin{figure}[ht]
	\begin{center}
		\includegraphics{fig_8.3-1}
	\end{center}
	\caption{Działanie procedury \proc{Radix-Sort} dla zbioru słów trzyliterowych.} \label{fig:8.3-1}
\end{figure}

\exercise %8.3-2
Algorytmami stabilnymi są sortowanie przez wstawianie i~sortowanie przez scalanie.
W~pierwszym z~nich, wstawiając element $A[j]$ do podtablicy $A[1\twodots j-1]$, zatrzymujemy się na pierwszym elemencie z~tej podtablicy, który jest mniejszy lub równy od $A[j]$.
Zatem elementy równe sobie nie zostaną wymieszane.
Podobnie w~procedurze \proc{Merge}, jeśli porównywane elementy będą sobie równe, to do wynikowej tablicy zostanie wstawiony najpierw element z~tablicy $L$ i~dopiero potem ten z~tablicy $R$.
Stabilność algorytmu wynika na podstawie indukcji po scalanych fragmentach.
Natomiast zarówno heapsort, jak i~quicksort nie sortuje stabilnie -- przykładem danych wejściowych, które ilustrują ten fakt w~obu tych algorytmach, jest ciąg $A=\langle2,2,1\rangle$.

Aby dowolny algorytm sortowania za pomocą porównań uczynić stabilnym, można zapamiętywać z~każdym elementem jego początkową pozycję w~tablicy wejściowej i~przy każdym teście dającym odpowiedź, że elementy są sobie równe, porządkować je za pomocą ich początkowych pozycji.
Do realizacji takiego podejścia wymagana jest dodatkowa pamięć rzędu $\Theta(n)$, gdzie $n$ jest długością sortowanej tablicy.
Nie zwiększa się natomiast asymptotyczne oszacowanie na czas działania zmodyfikowanego sortowania, ponieważ przeprowadzenie dodatkowego testu odbywa się w~czasie stałym.

\exercise %8.3-3
Przeprowadzimy dowód przez indukcję względem liczby cyfr $d$ elementów wejściowych.
Dla $d=1$ algorytm \proc{Radix-Sort} sprowadza się do wywołania pomocniczej procedury do posortowania pojedynczych cyfr, a~zatem poprawność algorytmu wynika z~poprawności pomocniczego sortowania.

Załóżmy teraz, że $d>1$ i~że wywołanie algorytmu \proc{Radix-Sort} po $d-1$ fazach zwróciło tablicę elementów, które, obcięte do $d-1$ ostatnich pozycji, wyznaczają porządek niemalejący.
Teraz elementy sortowane są względem pozycji $d$.
Niech $a$ i~$b$ będą cyframi porównywanymi podczas tej fazy.
Jeśli $a\ne b$, to niezależnie od pozostałych cyfr elementów, których częściami są $a$ i~$b$, elementy te zostaną ustawione w~odpowiedniej kolejności.
Jeśli jednak $a=b$, to elementy nie zostaną zamienione miejscami, bo pomocnicza procedura sortuje stabilnie.
Elementy pozostają jednak we właściwym porządku, bo jest on wyznaczony przez ich $d-1$ ostatnich pozycji, a~te, na mocy założenia indukcyjnego, zostały poprawnie uporządkowane w~poprzednich fazach algorytmu.

\exercise %8.3-4
Każdą liczbę z~zakresu $0\twodots n^2-1$ można potraktować jak liczbę dwucyfrową w~systemie o~podstawie $n$.
Liczby w~takiej postaci sortujemy, wywołując algorytm \proc{Radix-Sort} o~dwóch fazach.
W~lemacie 8.3 mamy $d=2$ oraz $k=n$, zatem opisane sortowanie działa w~czasie $\Theta(2(n+n))=\Theta(n)$.

\exercise %8.3-5
Rozważmy sortowanie pozycyjne w~wersji intuicyjnej dla liczb trzycyfrowych.
W~pierwszej fazie sortujemy po najbardziej znaczącej cyfrze wejściowego ciągu liczb.
Podczas drugiej fazy dokonujemy sortowania w~obrębie fragmentów tablicy zawierających liczby o~tej samej najbardziej znaczącej cyfrze.
W~najgorszym przypadku po pierwszej fazie dostaniemy 10 takich fragmentów, każdy o~innej najbardziej znaczącej cyfrze, a~zatem środkowe cyfry sortowane będą w~kolejnych 10 fazach.
Sytuacja w~najgorszym przypadku powtórzy się -- z~każdego sortowanego fragmentu utworzy się po 10 jeszcze mniejszych fragmentów o~poszczególnych kombinacjach dwóch pierwszych cyfr.
Będzie zatem maksymalnie 100 takich podtablic, których posortowanie będzie wymagało 100 kolejnych faz.
Łącznie algorytm wykona zatem 111 faz.

Najwięcej tymczasowych podtablic pozostawionych do późniejszego przetworzenia istnieje wówczas, gdy rozpoczyna się sortowanie ostatnich cyfr.
W~przypadku sortowania liczb trzycyfrowych jest łącznie 27 takich podtablic -- 9 powstałych po posortowaniu najbardziej znaczących cyfr, 9 kolejnych powstałych po posortowaniu środkowych cyfr w~obrębie pierwszego fragmentu i~9 takich, które czekają na posortowanie po najmniej znaczących cyfrach.

W~ogólności do posortowania liczb \singledash{$d$}{cyfrowych} tym algorytmem wymaganych jest co najwyżej $\sum_{i=0}^{d-1}10^i=(10^d-1)/9$ faz.
Najwięcej tymczasowych fragmentów istnieje w~momencie rozpoczęcia sortowania po najmniej znaczącej cyfrze -- jest ich wtedy maksymalnie $9d$.
