\problem{Sortowanie względem średnich} %8-5

\subproblem %8-5(a)
Zgodnie z~definicją tablica $A[1\twodots n]$ jest 1\nbhyphen posortowana, jeśli dla każdego $i=1$, 2, \dots, $n-1$ zachodzi $A[i]\le A[i+1]$, a~to jest równoważne temu, że tablica $A$ jest posortowana niemalejąco.

\subproblem %8-5(b)
Jedną z~takich permutacji jest $\langle2,1,5,3,7,4,8,6,10,9\rangle$.

\subproblem %8-5(c)
Aby udowodnić ten fakt, wystarczy obie strony nierówności
\[
	\frac{\sum_{j=i}^{i+k-1}A[j]}{k} \le \frac{\sum_{j=i+1}^{i+k}A[j]}{k}
\]
pomnożyć przez $k$ i~zredukować powtarzające się składniki.

\subproblem %8-5(d)
Jednym ze sposobów $k$\nbhyphen posortowania tablicy $A[1\twodots n]$ jest zastosowanie algorytmu quicksort, którego rekursja zatrzymuje się dla podtablic o~rozmiarach nieprzekraczających $k$.
Dokładniej, należy zamienić warunek z~wiersza 1 procedury \proc{Quicksort} na następujący:
\begin{codebox}
\li	\If $p+k-1<r$
\end{codebox}
Po zakończeniu działania algorytmu będzie zachodzić $A[i]\le A[i+k]$ dla każdego $i=1$, 2, \dots, $n-k$, co, na podstawie części (c), jest równoważne $k$\nbhyphen posortowaniu tablicy $A$.

W~\refExercise{7.4-5} pokazaliśmy, że quicksort po takiej modyfikacji działa w~oczekiwanym czasie $O(n\lg(n/k))$.
Aby było to oszacowanie na czas w~przypadku pesymistycznym, musimy na każdym poziomie rekurencji zagwarantować najlepszy przypadek wyboru elementów rozdzielających.
Można to zrealizować poprzez wybieranie do tej roli mediany bieżącej podtablicy, jak to opisano w~\refExercise{9.3-3}.

\subproblem %8-5(e)
Z~punktu (c) wynika wniosek, że w~$k$\nbhyphen posortowanej tablicy każdy z~ciągów postaci
\[
    \langle A[j],A[j+k],A[j+2k],\dots,A[j+m_jk]\rangle,
\]
gdzie $j=1$, 2 \dots, $k$ i~$j+m_jk\le n$, jest uporządkowany niemalejąco.
Ponadto każda z~$n$ liczb tablicy $A$ wchodzi w~skład dokładnie jednego takiego ciągu.
Wystarczy więc scalić te ciągi w~tablicę posortowaną, co na podstawie \refExercise{6.5-8} można wykonać w~czasie $O(n\lg k)$.

\subproblem %8-5(f)
Załóżmy, że $k$\nbhyphen posortowanie tablicy $n$\nbhyphen elementowej jest możliwe w~czasie $o(n\lg n)$.
Ponieważ $k$ jest stałą, to $k$\nbhyphen posortowaną tablicę można byłoby następnie scalić w~tablicę posortowaną w~czasie $O(n)$ (punkt (e)), dysponując tym samym algorytmem sortowania za pomocą porównań o~czasie $o(n\lg n)$.
Stoi to jednak w~sprzeczności z~tw.\ 8.1.
