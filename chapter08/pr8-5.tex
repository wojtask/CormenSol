\problem{Sortowanie względem średnich} %8-5

\subproblem %8-5(a)
Zgodnie z~definicją tablica $A[1\twodots n]$ jest 1\nbhyphen posortowana, jeśli dla każdego $i=1$, 2, \dots, $n-1$ zachodzi $A[i]\le A[i+1]$, a~to jest równoważne temu, że tablica $A$ jest posortowana niemalejąco.

\subproblem %8-5(b)
Jedną z~takich permutacji jest $\langle2,1,5,3,7,4,8,6,10,9\rangle$.

\subproblem %8-5(c)
Aby udowodnić ten fakt, wystarczy nierówność
\[
	\frac{\sum_{j=i}^{i+k-1}A[j]}{k} \le \frac{\sum_{j=i+1}^{i+k}A[j]}{k}
\]
pomnożyć przez $k$ i~zredukować te same składniki po obu stronach znaku nierówności.

\subproblem %8-5(d)
Jednym ze sposobów $k$\nbhyphen posortowania tablicy $A[1\twodots n]$ jest zastosowanie algorytmu quicksort, którego rekursja zatrzymuje się dla podtablic o~rozmiarach nieprzekraczających $k$.
Dokładniej, należy zamienić warunek z~wiersza 1 procedury \proc{Quicksort} na następujący:
\begin{codebox}
\li	\If $p+k-1<r$
\end{codebox}
Po zakończeniu działania algorytmu mamy, że $A[i]\le A[i+k]$ dla każdego $i=1$, 2, \dots, $n-k$.
Warunek ten, na podstawie części (c), jest równoważny temu, że tablica $A$ jest $k$\nbhyphen posortowana.

W~\refExercise{7.4-5} pokazaliśmy, że quicksort po takiej modyfikacji działa w~oczekiwanym czasie $O(n\lg(n/k))$.
Aby było to oszacowanie na czas w~przypadku pesymistycznym, musimy zagwarantować najlepszy przypadek wyboru elementów rozdzielających.
Można to zrealizować poprzez wybieranie do tej roli mediany bieżącej podtablicy, jak to opisano w~\refExercise{9.3-3}.

\subproblem %8-5(e)
W~$k$\nbhyphen posortowanej tablicy każdy z~ciągów postaci $\langle A[j],A[j+k],A[j+2k],\dots,A[j+m_jk]\rangle$, gdzie $j=1$, 2 \dots, $k$ i~$j+m_jk\le n$, jest uporządkowany niemalejąco i~wszystkie one zawierają łącznie $n$ elementów.
Wystarczy więc scalić je w~jedną posortowaną tablicę, co na podstawie \refExercise{6.5-8} można wykonać w~czasie $O(n\lg k)$.

\subproblem %8-5(f)
Przyjmiemy dla wygody, że $n$ jest podzielne przez $k$ -- nie spowoduje to zmniejszenia ogólności naszej analizy.

W~dowodzie dolnego ograniczenia czasowego tego problemu wykorzystamy model drzew decyzyjnych.
Niech $T$ będzie drzewem decyzyjnym pewnego algorytmu $k$\nbhyphen sortowania za pomocą porównań tablicy o~$n$ elementach.
Drzewo $T$ jest podobne do drzewa decyzyjnego algorytmu zwykłego sortowania za pomocą porównań -- każdy jego liść odpowiada pewnej permutacji tablicy wejściowej.
Jednak liczba osiągalnych liści jest na ogół mniejsza niż w~drzewie zwykłego sortowania.
Wynikowa tablica składa się z~$n/k$ podtablic $k$\nbhyphen elementowych, w~których uporządkowanie elementów nie ma znaczenia.
Osiągalnych liści w~tym drzewie jest zatem
\[
    \frac{n!}{(k!)^{n/k}}.
\]

Niech $h$ oznacza wysokość drzewa $T$.
Liczba liści w~$T$ nie przekracza $2^h$, więc
\[
    h \ge \lg\frac{n!}{(k!)^{n/k}} = \lg(n!)-(n/k)\lg(k!) = \Omega(n\lg n),
\]
ponieważ traktujemy $k$ jako stałą.
Wyznaczone oszacowanie na $h$ stanowi dolne ograniczenie na czas działania dowolnego algorytmu $k$\nbhyphen sortowania za pomocą porównań tablicy o~$n$ elementach.
