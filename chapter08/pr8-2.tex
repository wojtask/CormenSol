\problem{Sortowanie w~miejscu w~czasie liniowym} %8-2

\subproblem %8-2(a)
Sortowanie przez zliczanie dla $k=1$.

\subproblem %8-2(b)
Poniższy pseudokod implementuje algorytm o~zadanych własnościach:
\begin{codebox}
\Procname{$\proc{Bitwise-Sort}(A)$}
\li	$n\gets\attrib{A}{length}$
\li	$i\gets1$
\li	$j\gets n$
\li	\While $i<j$
\li		\Do zamień $A[i]\leftrightarrow A[j]$
\li			\While $i\le n$ i~$A[i]=0$
\li				\Do $i\gets i+1$
				\End
\li			\While $j\ge1$ i~$A[j]=1$
\li				\Do $j\gets j-1$
				\End
		\End
\end{codebox}

\subproblem %8-2(c)
Sortowanie przez wstawianie.

\subproblem %8-2(d)
Do posortowania $n$ rekordów \singledash{$b$}{bitowych} w~czasie $O(bn)$ za pomocą algorytmu sortowania pozycyjnego, potrzebna jest pomocnicza procedura, która sortuje stabilnie i~działa w~czasie $O(n)$.
Takie warunki spełnia jedynie algorytm z~części (a).
W~każdej kolejnej fazie sortowania rekordy powinny być porządkowane względem coraz bardziej znaczących bitów.

\subproblem %8-2(e)
W~algorytmie wykorzystamy tablicę pomocniczą $C[0\twodots k]$ tworzoną identycznie jak w~wierszach \doubledash{1}{7} procedury \proc{Counting-Sort}.
Ideą algorytmu jest przejrzenie tablicy wejściowej w~poszukiwaniu elementów, które należą do niewłaściwych obszarów tablicy, a~następnie umieszczenie ich na właściwych pozycjach, które znajdujemy, korzystając z~informacji zebranych w~tablicy $C$ na początku działania algorytmu.
\begin{codebox}
\Procname{$\proc{Counting-Sort-In-Place}(A,k)$}
\li	utwórz tablicę $C[0\twodots k]$ jak w~wierszach \doubledash{1}{7} procedury \proc{Counting-Sort}
\li	skopiuj zawartość tablicy $C$ do tablicy $C'$
\li	$i\gets1$
\li	\While $i\le\attrib{A}{length}-1$ \label{li:counting-sort-in-place-while-begin}
\li		\Do $\id{key}\gets A[i]$
\li			\If $C'[\id{key}-1]<i\le C'[\id{key}]$ \label{li:counting-sort-in-place-test}
\li				\Then $i\gets i+1$
\li				\Else zamień $A[i]\leftrightarrow A[C[\id{key}]]$ \label{li:counting-sort-in-place-swap}
\li					$C[\id{key}]\gets C[\id{key}]-1$
				\End
		\End \label{li:counting-sort-in-place-while-end}
\end{codebox}

Początkowy stan tablicy $C$ zostaje zapamiętany w~tablicy $C'$.
Następnie pętla \kw{while} w~wierszach \doubledash{\ref{li:counting-sort-in-place-while-begin}}{\ref{li:counting-sort-in-place-while-end}} przegląda tablicę wejściową w~poszukiwaniu elementów, które nie powinny znajdować się w~posortowanej tablicy na aktualnej pozycji.
Do sprawdzenia, czy element znajduje się we właściwym miejscu, używany jest test z~wiersza \ref{li:counting-sort-in-place-test}.
W~przypadku negatywnego wyniku tego testu element z~aktualnej pozycji zostaje przeniesiony na swoją docelową pozycję po zamianie z~elementem, który tam się znajduje.
Ten ostatni będzie sprawdzany w~następnej iteracji pętli \kw{while}.

Algorytm sortuje tablicę wejściową w~miejscu (niekoniecznie stabilnie), wykorzystując $O(k)$ dodatkowej pamięci.
Czas potrzebny na utworzenie tablic $C$ i~$C'$ wynosi $O(n+k)$.
Zauważmy, że element, który zostanie umieszczony na swojej docelowej pozycji, nie zmieni jej aż do zakończenia działania algorytmu -- zostanie więc wykonanych co najwyżej $n-1$ zamian w~wierszu \ref{li:counting-sort-in-place-swap}.
W~iteracjach pętli \kw{while}, w~których zamiany nie są dokonywane, jest inkrementowana zmienna $i$ -- sumaryczna liczba iteracji jest więc ograniczona przez $2(n-1)$.
Stąd czas działania algorytmu wynosi $O(n+k)$.
