\subchapter{Sortowanie przez zliczanie}

\exercise %8.2-1
Rys.\ \ref{fig:8.2-1} przedstawia działanie procedury \proc{Counting-Sort} dla tablicy $A$.
\begin{figure}[!ht]
	\centering \begin{tikzpicture}[
	index node/.append style = {minimum size=2mm, text height=1.5ex, node distance=0.5mm},
	row 1/.style = {nodes={fill=black!20, minimum size=3.5mm}},
	med cell/.style = {row 1 column #1/.style={nodes={fill=black!40}}},
	outer/.append style = {node distance=12mm and 8mm},
	inner/.style = {draw=none, fill=none, node distance=3mm and 2mm}
]

\node[outer] (pic a) {
\begin{tikzpicture}
	\node[inner] (pic a1) {
	\begin{tikzpicture}
		\matrix[array] (arr) {6 & 0 & 2 & 0 & 1 & 3 & 4 & 6 & 1 & 3 & 2 \\};
		\foreach \x in {1,...,11} {
			\node[index node, above=of arr-1-\x] {\x};
		}
		\node[left=of arr-1-1.west] {$A$};
	\end{tikzpicture}
	};
	\node[inner, below=of pic a1.south west, anchor=north west] {
	\begin{tikzpicture}
		\matrix[array] (arr) {2 & 2 & 2 & 2 & 1 & 0 & 2 \\};
		\foreach [count=\y] \x in {0, ..., 6} {
			\node[index node, above=of arr-1-\y] {\x};
		}
		\node[left=of arr-1-1.west] {$C$};
	\end{tikzpicture}
	};
\end{tikzpicture}
};

\node[outer, right=of pic a] (pic b) {
\begin{tikzpicture}
	\node[inner] {
	\begin{tikzpicture}
		\matrix[array, row 1/.append style = {column sep=0pt, nodes={fill=none, draw=none}}] { & & & & & & & & & & \\};
		\matrix[array] (arr) {2 & 4 & 6 & 8 & 9 & 9 & 11 \\};
		\foreach [count=\y] \x in {0, ..., 6} {
			\node[index node, above=of arr-1-\y] {\x};
		}
		\node[left=of arr-1-1.west] {$C$};
	\end{tikzpicture}
	};
\end{tikzpicture}
};

\node[outer, right=of pic b] (pic c) {
\begin{tikzpicture}
	\node[inner] (pic c1) {
	\begin{tikzpicture}[
		med cell/.list = {1,2,3,4,5,7,8,9,10,11}
	]
		\matrix[array] (arr) { & & & & & 2 & & & & & \\};
		\foreach \x in {1,...,11} {
			\node[index node, above=of arr-1-\x] {\x};
		}
		\node[left=of arr-1-1.west] {$B$};
	\end{tikzpicture}
	};
	\node[inner, below=of pic c1.south west, anchor=north west] {
	\begin{tikzpicture}
		\matrix[array] (arr) {2 & 4 & 5 & 8 & 9 & 9 & 11 \\};
		\foreach [count=\y] \x in {0, ..., 6} {
			\node[index node, above=of arr-1-\y] {\x};
		}
		\node[left=of arr-1-1.west] {$C$};
	\end{tikzpicture}
	};
\end{tikzpicture}
};

\node[outer, below=of pic a] (pic d) {
\begin{tikzpicture}
	\node[inner] (pic d1) {
	\begin{tikzpicture}[
		med cell/.list = {1,2,3,4,5,7,9,10,11}
	]
		\matrix[array] (arr) { & & & & & 2 & & 3 & & & \\};
		\foreach \x in {1,...,11} {
			\node[index node, above=of arr-1-\x] {\x};
		}
		\node[left=of arr-1-1.west] {$B$};
	\end{tikzpicture}
	};
	\node[inner, below=of pic d1.south west, anchor=north west] {
	\begin{tikzpicture}
		\matrix[array] (arr) {2 & 4 & 5 & 7 & 9 & 9 & 11 \\};
		\foreach [count=\y] \x in {0, ..., 6} {
			\node[index node, above=of arr-1-\y] {\x};
		}
		\node[left=of arr-1-1.west] {$C$};
	\end{tikzpicture}
	};
\end{tikzpicture}
};

\node[outer, right=of pic d] (pic e) {
\begin{tikzpicture}
	\node[inner] (pic e1) {
	\begin{tikzpicture}[
		med cell/.list = {1,2,3,5,7,9,10,11}
	]
		\matrix[array] (arr) { & & & 1 & & 2 & & 3 & & & \\};
		\foreach \x in {1,...,11} {
			\node[index node, above=of arr-1-\x] {\x};
		}
		\node[left=of arr-1-1.west] {$B$};
	\end{tikzpicture}
	};
	\node[inner, below=of pic e1.south west, anchor=north west] {
	\begin{tikzpicture}
		\matrix[array] (arr) {2 & 3 & 5 & 7 & 9 & 9 & 11 \\};
		\foreach [count=\y] \x in {0, ..., 6} {
			\node[index node, above=of arr-1-\y] {\x};
		}
		\node[left=of arr-1-1.west] {$C$};
	\end{tikzpicture}
	};
\end{tikzpicture}
};

\node[outer, right=of pic e] (pic f) {
\begin{tikzpicture}
	\node[inner] {
	\begin{tikzpicture}
		\matrix[array] (arr) { 0 & 0 & 1 & 1 & 2 & 2 & 3 & 3 & 4 & 6 & 6 \\};
		\foreach \x in {1,...,11} {
			\node[index node, above=of arr-1-\x] {\x};
		}
		\node[left=of arr-1-1.west] {$B$};
	\end{tikzpicture}
	};
\end{tikzpicture}
};

\node[subpicture label, below=2mm of pic a.south] (label a) {(a)};
\node[subpicture label] at (label a -| pic b) {(b)};
\node[subpicture label] at (label a -| pic c) {(c)};
\node[subpicture label, below=2mm of pic d.south] (label d) {(d)};
\node[subpicture label] at (label d -| pic e) {(e)};
\node[subpicture label] at (label d -| pic f) {(f)};

\end{tikzpicture}

	\caption{Działanie procedury \proc{Counting-Sort} dla tablicy $A=\langle6,0,2,0,1,3,4,6,1,3,2\rangle$, której każdy element jest nieujemną liczbą całkowitą nie większą niż $k=6$.
{\sffamily\bfseries(a)} Tablica $A$ wraz z~tablicą pomocniczą $C$ po wykonaniu pętli w~wierszach \doubledash{3}{4}.
{\sffamily\bfseries(b)} Tablica $C$ po wykonaniu pętli w~wierszach \doubledash{6}{7}.
{\sffamily\bfseries\doubledash{(c)}{(e)}} Tablica wynikowa $B$ oraz tablica pomocnicza $C$ po wykonaniu, odpowiednio, jednej, dwóch i~trzech iteracji pętli \kw{for} w~wierszach \doubledash{9}{11}.
{\sffamily\bfseries(f)} Wynikowa posortowana tablica $B$.} \label{fig:8.2-1}
\end{figure}

\exercise %8.2-2
Elementy tablicy wejściowej przetwarzane są od końca, a~wartości w~tablicy $C$ stanowiące indeksy tablicy wynikowej, na które trafią wejściowe elementy, są systematycznie zmniejszane.
A~zatem równe sobie elementy będą umieszczane na coraz niższych pozycjach w~tablicy wynikowej, dzięki czemu ich początkowa kolejność zostanie zachowana.

\exercise %8.2-3
Po wprowadzeniu takiej modyfikacji tablica $A$ będzie przeglądana od lewej do prawej, a~równe elementy będą umieszczane w~odpowiedniej części tablicy wynikowej na coraz niższych pozycjach.
Oczywiście kolejność przetwarzania elementów tablicy $A$ nie wpływa na poprawność algorytmu \proc{Counting-Sort}, jednak zastosowanie opisanej zmiany powoduje, że sortowanie nie jest stabilne.

\exercise %8.2-4
Algorytm będzie zliczać elementy z~wejściowej tablicy w~czasie $\Theta(n+k)$, zapamiętując wyniki w~pomocniczej tablicy $C[0\twodots k]$.
Następnie, zapytany o~liczbę elementów z~przedziału $a\twodots b$ na podstawie danych z~tablicy pomocniczej, zwróci liczbę elementów z~zakresu $0\twodots b$ pomniejszoną o~liczbę elementów z~zakresu $0\twodots a-1$.
Dokładniej, jego pierwsza faza jest równoważna wierszom \doubledash{1}{8} procedury \proc{Counting-Sort}, natomiast w~drugiej fazie zwracany jest odpowiedni wynik w~zależności od przypadku:
\begin{enumerate}
	\renewcommand{\labelenumi}{(\roman{enumi})}
	\item $C[b]-C[a-1]$, jeśli $0<a\le b\le k$;
	\item $C[k]-C[a-1]$, jeśli $0<a\le k<b$;
	\item $C[b]$, jeśli $a\le0\le b\le k$;
	\item $C[k]$, jeśli $a\le0\le k<b$;
	\item 0, jeśli $a>k$ lub $b<0$.
\end{enumerate}
