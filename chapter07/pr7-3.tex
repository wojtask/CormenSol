\problem{Nieefektywne sortowanie} %7-3

\subproblem %7-3(a)
Udowodnimy poprawność algorytmu przez indukcję względem rozmiaru $m=j-i+1$ tablicy wejściowej $A[i\twodots j]$.
Łatwo sprawdzić, że algorytm działa poprawnie w~przypadku, gdy $m\le2$ -- kończy on wtedy działanie w~linii 4, nie wykonując wywołań rekurencyjnych.

Załóżmy więc, że $m>2$ i~że poprawnie sortowane są tablice rozmiarów mniejszych niż $m$.
W~wierszu 5 wyznaczane jest $k=\lfloor m/3\rfloor$.
Na podstawie założenia, w~wyniku wykonania wiersza 6, w~podtablicy $A[i+\lfloor m/3\rfloor+1\twodots j-\lfloor m/3\rfloor]$ znajdują się elementy nie mniejsze od tych znajdujących się w~$A[i\twodots i+\lfloor m/3\rfloor]$.
A~zatem, po wykonaniu wiersza 7, $\lfloor m/3\rfloor$ największych elementów tablicy $A[i\twodots j]$ znajduje się w~obszarze $A[j-\lfloor m/3\rfloor+1\twodots j]$.
Aby dokończyć sortowanie tablicy $A[i\twodots j]$, wystarczy uporządkować fragment $A[i\twodots j-\lfloor m/3\rfloor]$, co realizuje wiersz 8.

Z~dowodu poprawności sortowania tablicy $A[i\twodots j]$ dla dowolnych $i\le j$ wynika w~szczególności poprawność sortowania tablicy $A[1\twodots n]$.

\subproblem %7-3(b)
Procedura jest wywoływana rekurencyjnie 3 razy, zaś każde wywołanie otrzymuje tablicę o~rozmiarze około $2/3$ rozmiaru oryginalnej tablicy.
Praca poza wywołaniami rekurencyjnymi jest wykonywana w~czasie stałym.
Stąd dostajemy równanie rekurencyjne
\[
	T(n) =
	\begin{cases}
		\Theta(1), & \text{jeśli $n\le2$}, \\
		3T(2n/3)+\Theta(1), & \text{jeśli $n>2$},
	\end{cases}
\]
które rozwiązujemy przy użyciu twierdzenia o~rekurencji uniwersalnej, otrzymując wynik $T(n)=\Theta(n^{\log_{3/2}3})\approx \Theta(n^{2{,}71})$.

\subproblem %7-3(c)
Zarówno pesymistyczne czasy działania sortowania przez scalanie i~algorytmu heapsort ($\Theta(n\lg n)$), jak i~pesymistyczne czasy sortowania przez wstawianie i~algorytmu quicksort ($\Theta(n^2)$) są asymptotycznie mniejsze od pesymistycznego czasu procedury \proc{Stooge-Sort}.
Metoda sortowania profesorów jest więc poprawna, ale nieefektywna.
