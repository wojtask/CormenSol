\problem{Nieefektywne sortowanie} %7-3

\subproblem %7-3(a)
Udowodnimy poprawność algorytmu przez indukcję względem rozmiaru tablicy $n$.
Łatwo sprawdzić, że algorytm działa poprawnie w~przypadku, gdy $n\le2$.
Załóżmy więc, że $n>2$ i~że poprawnie sortowane są tablice o~rozmiarach mniejszych niż $n$.
W~wyniku wykonania wiersza 6 na pozycjach $\lfloor n/3\rfloor+1\twodots n-\lfloor n/3\rfloor$ znajdują się elementy nie mniejsze od tych z~pozycji $1\twodots\lfloor n/3\rfloor$.
A~zatem, po wykonaniu wiersza 7, $\lfloor n/3\rfloor$ największych elementów tablicy $A$ znajduje się w~obszarze $A[n-\lfloor n/3\rfloor+1\twodots n]$.
Aby zakończyć sortowanie tablicy $A$, wystarczy uporządkować fragment $A[1\twodots n-\lfloor n/3\rfloor]$, co realizuje wiersz 8.

\subproblem %7-3(b)
Procedura jest wywoływana rekurencyjnie 3 razy, zaś każde wywołanie otrzymuje tablicę o~rozmiarze około $2/3$ rozmiaru oryginalnej tablicy.
Ponadto praca poza wywołaniami rekurencyjnymi jest wykonywana w~czasie stałym.
Stąd dostajemy równanie rekurencyjne
\[
	T(n) =
	\begin{cases}
		\Theta(1), & \text{jeśli $n\le2$}, \\
		3T(2n/3)+\Theta(1), & \text{jeśli $n>2$},
	\end{cases}
\]
które rozwiązujemy przy użyciu twierdzenia o~rekurencji uniwersalnej, otrzymując wynik $T(n)=\Theta(n^{\log_{3/2}3})\approx \Theta(n^{2{,}71})$.

\subproblem %7-3(c)
Pesymistyczny czas działania algorytmu \proc{Stooge-Sort} jest wyższego rzędu nie tylko od pesymistycznego czasu działania algorytmów sortowania przez scalanie, kopcowanie czy sortowania szybkiego, ale również od mniej efektywnego sortowania przez wstawianie.
Metoda sortowania profesorów jest więc poprawna, ale bardzo powolna.
