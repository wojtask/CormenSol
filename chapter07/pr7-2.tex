\problem{Alternatywna analiza algorytmu quicksort} %7-2

\subproblem %7-2(a)
W~procedurze \proc{Randomized-Partition} element $A[r]$ jest zamieniany z~elementem losowo wybranym z~tablicy $A$ o~rozmiarze $n$.
Stąd szanse, że pewien ustalony element tablicy $A$ zostanie elementem rozdzielającym, wynoszą $1/n$.
Wobec tego wartość oczekiwana zmiennej $X_i$, gdzie $i=1$, 2, \dots, $n$, wynosi $\E(X_i)=\Pr(X_i=1)=1/n$.

\subproblem %7-2(b)
Wywołanie procedury \proc{Randomized-Quicksort} dla tablicy wejściowej $A$ o~rozmiarze $n$ potrzebuje czasu $\Theta(n)$ na podział tablicy.
Zostaje wyznaczony pewien indeks $1\le q\le n$, po czym procedura jest wywoływana rekurencyjnie dla podtablic rozmiarów $q-1$ i~$n-q$.
Zmienna losowa $X_i$ zdefiniowana w~punkcie (a) przyjmuje wartość 1, gdy $i=q$, a~w~pozostałych przypadkach przyjmuje wartość 0.
Stąd czas potrzebny na posortowanie \singledash{$n$}{elementowej} tablicy wyraża się wzorem
\[
	T(n) = T(q-1)+T(n-q)+\Theta(n) = \sum_{i=1}^nX_i(T(i-1)+T(n-i)+\Theta(n)).
\]
Biorąc wartości oczekiwane skrajnych wyrażeń, otrzymujemy wzór (7.5) na oczekiwany czas działania algorytmu quicksort.

\subproblem %7-2(c)
\note{Przyjmiemy, że we wzorze (7.6) sumowanie przebiega od\/ $q=2$ do\/ $q=n-1$.}

\noindent Wykorzystując części (a) i~(b) oraz liniowość wartości oczekiwanej, otrzymujemy
\begin{align*}
	\E(T(n)) &= \sum_{q=1}^n\E\bigl(X_q(T(q-1)+T(n-q)+\Theta(n))\bigr) \\
	&= \sum_{q=1}^n\E(X_q)\bigl(\E(T(q-1))+\E(T(n-q))+\E(\Theta(n))\bigr) \\
	&= \sum_{q=1}^n\frac{1}{n}\bigl(\E(T(q-1))+\E(T(n-q))+\Theta(n)\bigr) \\
	&= \frac{1}{n}\sum_{q=0}^{n-1}2\E(T(q))+\frac{1}{n}\sum_{q=1}^n\Theta(n) \\
	&= \frac{2}{n}\sum_{q=2}^{n-1}\E(T(q))+\frac{2}{n}\bigl(\E(T(0))+\E(T(1))\bigr)+\Theta(n) \\
	&= \frac{2}{n}\sum_{q=2}^{n-1}\E(T(q))+\Theta(n).
\end{align*}
Skorzystaliśmy z~faktu, że $\E(T(0))$ i~$\E(T(1))$ są stałymi i~mogą zostać wchłonięte przez $\Theta(n)$.

\subproblem %7-2(d)
Rozdzielamy sumę na dwie części:
\[
	\sum_{k=1}^{n-1}k\lg k = \sum_{k=1}^{\lceil n/2\rceil-1}k\lg k+\sum_{k=\lceil n/2\rceil}^{n-1}k\lg k,
\]
po czym zauważamy, że czynnik $\lg k$ w~pierwszej sumie po prawej stronie znaku równości możemy ograniczyć z~góry przez $\lg(n/2)=\lg n-1$, a~czynnik $\lg k$ w~drugiej sumie -- przez $\lg n$.
Stąd
\begin{align*}
	\sum_{k=1}^{n-1}k\lg k &\le (\lg n-1)\sum_{k=1}^{\lceil n/2\rceil-1}k+\lg n\sum_{k=\lceil n/2\rceil}^{n-1}k \\
	&= \lg n\sum_{k=1}^{n-1}k-\sum_{k=1}^{\lceil n/2\rceil-1}k \\[2mm]
	&\le \frac{n(n-1)\lg n}{2}-\frac{n}{4}\biggl(\frac{n}{2}-1\biggr) \\
	&= \frac{n^2\lg n}{2}-\frac{n\lg n}{2}-\frac{n^2}{8}+\frac{n}{4} \\[1mm]
	&\le \frac{n^2\lg n}{2}-\frac{n^2}{8},
\end{align*}
przy czym ostatnia nierówność zachodzi, gdy $n\ge\sqrt{2}$.

\subproblem %7-2(e)
Załóżmy, że $n>2$ i~przyjmijmy założenie indukcyjne $\E(T(q))\le aq\lg q$ dla każdego $q=2$, 3, \dots, $n-1$ oraz pewnej stałej $a>0$.
Z~punktu (c) mamy
\[
	\E(T(n)) = \frac{2}{n}\sum_{q=2}^{n-1}\E(T(q))+\Theta(n) \le \frac{2}{n}\sum_{q=2}^{n-1}aq\lg q+\Theta(n) = \frac{2a}{n}\sum_{q=1}^{n-1}q\lg q+\Theta(n).
\]
Wykorzystując teraz wynik z~punktu (d), dostajemy
\[
	\E(T(n)) \le \frac{2a}{n}\biggl(\frac{n^2\lg n}{2}-\frac{n^2}{8}\biggr)+\Theta(n) = an\lg n-\frac{an}{4}+\Theta(n) \le an\lg n,
\]
gdyż możemy wybrać $a$ wystarczająco duże, by wyrażenie $an/4$ dominowało nad składnikiem $\Theta(n)$.

Jeśli przyjmiemy, że $\E(T(2))=1$, to wówczas będziemy mieć $1\le a\cdot2\cdot\lg2=2a$, skąd $a\ge1/2$.
A~zatem można przyjąć $n=2$ na podstawę indukcji i~dowód faktu, że $\E(T(n))=O(n\lg n)$ jest zakończony.
Łącząc ten wynik z~oszacowaniem $\Omega(n\lg n)$ dla przypadku optymistycznego, które zostało wyznaczone w~\refExercise{7.4-2}, dostajemy, że oczekiwanym czasem działania algorytmu quicksort jest $\Theta(n\lg n)$.
