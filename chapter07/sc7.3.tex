\subchapter{Randomizowana wersja algorytmu quicksort}

\exercise %7.3-1
Randomizacja algorytmu sprawia, że żadne jego dane wejściowe nie stanowią przypadku pesymistycznego.
Można więc przyjąć, że dla każdych danych wejściowych algorytm zachowuje się jak w~przypadku średnim.

\exercise %7.3-2
W~przypadku pesymistycznym generator liczb losowych za każdym razem wybiera wartości, które tworzą najbardziej niezrównoważony podział podtablicy, czyli $p$ lub $r$.
W~takiej sytuacji liczba wywołań generatora jest rzędu $\Theta(n)$.

W~przypadku optymistycznym generator za każdym razem zwraca liczbę bliską $(p+r)/2$.
Tworzone podziały są zatem zrównoważone i~liczba wywołań generatora w~tym przypadku wynosi $\Theta(\lg n)$.
