\subchapter{Czas działania algorytmu quicksort}

\exercise %7.2-1
Niech $d_1$, $d_2$, $n_0>0$ będą takimi stałymi, że dla wszystkich $n\ge n_0$ składnik $\Theta(n)$ jest ograniczony od dołu przez $d_1n$ oraz od góry przez $d_2n$.
Przyjmijmy, że $n\ge n_0$.
Korzystając z~założenia, że $T(n-1)\ge c_1(n-1)^2$ dla pewnej stałej $c_1>0$, dostajemy
\[
	T(n) = T(n-1)+\Theta(n) \ge c_1(n-1)^2+d_1n = c_1n^2+(d_1-2c_1)n+c_1 \ge c_1n^2.
\]
Ostatnia nierówność jest spełniona, gdy ustalimy $c_1=d_1/2$.

Górne oszacowanie na $T(n)$ dowodzimy analogicznie, zakładając, że $T(n-1)\le c_2(n-1)^2$ dla pewnej stałej $c_2>0$:
\[
	T(n) = T(n-1)+\Theta(n) \le c_2(n-1)^2+d_2n = c_2n^2+(d_2-2c_2)n+c_2 \le c_2n^2.
\]
Po przyjęciu $c_2=d_2$, spełniamy ostatnią nierówność dla wszystkich $n\ge1$.

Przypadek brzegowy $n=n_0$ można przyjąć za podstawę obu powyższych indukcji dla podanych wartości stałych, co kończy dowód, że $T(n)=\Theta(n^2)$.

\exercise %7.2-2
Procedura \proc{Partition} w~takim przypadku zwraca $q=r$ (\refExercise{7.1-2}), a~więc w~następnych wywołaniach rekurencyjnych procedury \proc{Quicksort} będą przetwarzane podtablice o~rozmiarach 0 i~$n-1$.
Tworząc za każdym razem najgorszy przypadek podziałów, algorytm działa w~czasie opisanym przez rekurencję z~\refExercise{7.2-1}, której rozwiązaniem jest $\Theta(n^2)$.

\exercise %7.2-3
W~tym przypadku podczas każdego wywołania procedury \proc{Partition} na element rozdzielający zostaje wybrany najmniejszy element przetwarzanej podtablicy $A[p\twodots r]$.
Warunek z~wiersza 4 jest zawsze fałszywy, dlatego jedynymi zamienionymi elementami są $A[p]$ i~$A[r]$ (wiersz 7), i~zwracana jest wartość $q=p$.
Podobnie jak dla tablicy o~identycznych elementach rozważanej w~poprzednim zadaniu, tworzone podziały są za każdym razem najbardziej niezrównoważone, a~więc czas działania algorytmu opisuje rekurencja z~\refExercise{7.2-1}.

\exercise %7.2-4
Prawie posortowany ciąg jest niemal najgorszym przypadkiem dla algorytmu quicksort.
W~wielu wywołaniach rekurencyjnych na element rozdzielający wybierany jest bowiem największy element przetwarzanej podtablicy.
Wykonywane są wówczas niezrównoważone podziały, przez co czas działania procedury \proc{Quicksort} zbliża się do kwadratowego.

Z~kolei dla sortowania przez wstawianie ciąg taki jest przypadkiem zbliżonym do optymistycznego, ponieważ czas działania procedury \proc{Insertion-Sort} jest tego samego rzędu, co ilość inwersji w~ciągu wejściowym (punkt (c) problemu \refProblem{2-4}).
W~ciągu prawie posortowanym ilość inwersji jest niewielka, możemy zatem mówić o~co najwyżej liniowym czasie działania sortowania przez wstawianie dla tego przypadku.

\exercise %7.2-5
Rozważmy drzewo rekursji dla opisanego przypadku dokonywania podziałów.
Najkrótsza gałąź w~tym drzewie składa się z~węzłów o~wartościach $\alpha^in$, gdzie $i$ jest poziomem węzła, zaś najdłuższa gałąź na $i$\nbhyphen tym poziomie posiada węzeł o~wartości $(1-\alpha)^in$.
Niech $h$ będzie głębokością liścia najkrótszej gałęzi.
Mamy wtedy $\alpha^hn=1$, skąd
\[
	h = \log_\alpha\frac{1}{n} = -\frac{\lg n}{\lg\alpha}.
\]
Jeśli przez $H$ oznaczymy głębokość liścia na gałęzi najdłuższej, to mamy $(1-\alpha)^Hn=1$, a~zatem
\[
	H = \log_{1-\alpha}\frac{1}{n} = -\frac{\lg n}{\lg(1-\alpha)}.
\]

\exercise %7.2-6
Załóżmy, że procedura \proc{Partition} utworzyła podział w~stosunku $1-\beta$ do $\beta$, gdzie $0<\beta\le1/2$.
Aby był on bardziej zrównoważony niż $1-\alpha$ do $\alpha$, to musi być $\alpha<\beta$.
Zdarzenie to można modelować za pomocą ciągłego rozkładu jednostajnego dla przedziału $(\alpha,1/2]$ w~przestrzeni $(0,1/2]$.
Wykorzystując definicję prawdopodobieństwa o~ciągłym rozkładzie jednostajnym, dostajemy
\[
	\Pr((\alpha,1/2]) = \frac{1/2-\alpha}{1/2-0} = 1-2\alpha.
\]
