\problem{Głębokość stosu w~algorytmie quicksort} %7-4

\subproblem %7-4(a)
\proc{Quicksort}$'$ wykonuje te same operacje na tablicy $A$ co oryginalny algorytm quicksort.
Różnica jest tylko w~przetwarzaniu podtablicy $A[q+1\twodots r]$.
Zamiast drugiego wywołania rekurencyjnego procedura wykonuje przypisanie w~wierszu 5 i~kolejna iteracja pętli \kw{while} wykonuje identyczne operacje, jak gdyby wywołane zostało $\proc{Quicksort}'(A,q+1,r)$.
Poprawność algorytmu wynika zatem z~poprawności oryginalnego algorytmu quicksort.

\subproblem %7-4(b)
Stos rekurencji może urosnąć do rozmiaru $\Theta(n)$ w~sytuacji, gdy będzie $\Theta(n)$ wywołań rekurencyjnych \proc{Quicksort}$'$.
Dzieje się tak na przykład wtedy, gdy na każdym poziomie rekursji procedura \proc{Partition} zwraca $q=r$.
Wywołanie rekurencyjne działa wtedy na podtablicy o~rozmiarze o~1 mniejszym w~porównaniu z~początkową tablicą.
Algorytm zachowuje się w~ten sposób, jeśli na wejście dostanie tablicę posortowaną niemalejąco.

\subproblem %7-4(c)
Wykorzystamy pomysł, aby wywoływać procedurę rekurencyjnie dla mniejszej podtablicy, natomiast większą przetwarzać w~bieżącym wywołaniu.
Dzięki temu na kolejnym poziomie rekursji rozważany będzie problem mniejszy co najmniej o~połowę, więc w~najgorszym przypadku głębokość stosu rekurencji wyniesie $\Theta(\lg n)$.
Tworzone podziały będą takie same jak przed dokonaniem usprawnienia, zatem oczekiwany czas działania algorytmu nie zmieni się.
Zmodyfikowany kod procedury \proc{Quicksort}$'$ został przedstawiony poniżej.
\begin{codebox}
\Procname{$\proc{Quicksort}''(A,p,r)$}
\li	\While $p<r$
\li		\Do \Comment Dziel i~sortuj mniejszą podtablicę.
\li			$q\gets\proc{Partition}(A,p,r)$
\li			\If $q-p<r-q$
\li				\Then $\proc{Quicksort}''(A,p,q-1)$
\li					$p\gets q+1$
\li				\Else $\proc{Quicksort}''(A,q+1,r)$
\li					$r\gets q-1$
				\End
		\End
\end{codebox}
