\subchapter{Opis algorytmu}

\exercise %7.1-1
\note{Rozwiązanie dotyczy przykładu z~tekstu oryginalnego.}

\noindent Rys.\ \ref{fig:7.1-1} przedstawia działanie procedury \proc{Partition} dla tablicy $A$.
\begin{figure}[ht]
	\begin{center}
		\includegraphics{fig_7.1-1}
	\end{center}
	\caption{Działanie procedury \proc{Partition} dla tablicy $A=\langle13,19,9,5,12,8,7,4,21,2,6,11\rangle$.
{\sffamily\bfseries(a)} Tablica wejściowa z~zaznaczonymi początkowymi wartościami zmiennych.
{\sffamily\bfseries\doubledash{(b)}{(l)}} Kolejne iteracje pętli \kw{for} w~wierszach \doubledash{3}{6}.
{\sffamily\bfseries(m)} Wynikowa tablica $A$ po wykonaniu zamiany z~wiersza 7.} \label{fig:7.1-1}
\end{figure}

\exercise %7.1-2
\note{Poprawną wartością dla\/ $q$ z~treści zadania powinno być\/ $\lfloor(p+r)/2\rfloor$.}

\noindent Zauważmy, że jeśli wszystkie elementy podtablicy $A[p\twodots r]$ mają taką samą wartość, to warunek z~wiersza 4 procedury \proc{Partition} jest spełniony w~każdej iteracji pętli \kw{for}.
Oznacza to, że po wykonaniu tej pętli będzie $i=r-1$ i~procedura zwróci $q=r$.

Odpowiedniej modyfikacji procedury dokonujemy poprzez wprowadzenie licznika elementów równych elementowi rozdzielającemu w~badanej podtablicy.
W~każdej iteracji pętli \kw{for} sprawdzamy dodatkowo, czy $A[j]=x$ i~jeśli tak, to licznik ten inkrementujemy.
Jeśli na końcu procedury jego wartość jest równa $r-p+1$, czyli rozmiarowi podtablicy, to zwracamy $q=\lfloor(p+r)/2\rfloor$.

\exercise %7.1-3
Podczas przetwarzania podtablicy $A[p\twodots r]$ o~rozmiarze $n=r-p+1$ wykonywanych jest $n-1$ iteracji pętli \kw{for}, a~każda z~nich przeprowadza operacje zajmujące czas stały.
Stąd czas działania procedury \proc{Partition} wynosi $\Theta(n)$.

\exercise %7.1-4
W~warunku z~wiersza 4 procedury \proc{Partition} wystarczy zmienić znak nierówności na przeciwny.
