\problem{Rozmyte sortowanie przedziałów} %7-6

\subproblem %7-6(a)
Niech $A$ będzie tablicą wejściową, przy czym $A[i]=[a_i,b_i]$ dla $i=1$, 2, \dots, $n$.
Zauważmy, że jeśli $[a_i,b_i]\cap[a_j,b_j]\ne\emptyset$, czyli przedziały $A[i]$ i~$A[j]$ nachodzą na siebie, to mogą wystąpić w~tablicy wynikowej w~dowolnej kolejności.
Algorytm działa podobnie jak quicksort, ale wykorzystuje tę obserwację, znajdując zbiór przedziałów nachodzących na przedział stanowiący element rozdzielający i~pomijając wywołanie rekurencyjne dla podtablicy utworzonej przez ten zbiór przedziałów.

Procedura \proc{Fuzzy-Sort} implementuje rozmyte sortowanie przedziałów.
Aby posortować całą tablicę $A$, należy wywołać $\proc{Fuzzy-Sort}(A,1,\attrib{A}{length})$.
\begin{codebox}
\Procname{$\proc{Fuzzy-Sort}(A,p,r)$}
\li	\If $p<r$
\li		\Then $\langle q_1,q_2\rangle\gets\proc{Fuzzy-Partition}(A,p,r)$
\li			$\proc{Fuzzy-Sort}(A,p,q_1-1)$ \label{li:fuzzy-sort-recursion1}
\li			$\proc{Fuzzy-Sort}(A,q_2+1,r)$ \label{li:fuzzy-sort-recursion2}
		\End
\end{codebox}

Pomocnicza procedura \proc{Fuzzy-Partition} dokonuje podziału tablicy $A[p\twodots r]$ na 3 podtablice: $A[q_1\twodots q_2]$, która nie musi być dalej sortowana, oraz $A[p\twodots q_1-1]$ i~$A[q_2+1\twodots r]$, które następnie sortowane są w~wywołaniach rekurencyjnych w~wierszach \ref{li:fuzzy-sort-recursion1} i~\ref{li:fuzzy-sort-recursion2}.
Pseudokod tej procedury pomocniczej został przedstawiony poniżej.
\begin{codebox}
\Procname{$\proc{Fuzzy-Partition}(A,p,r)$}
\li	zamień $A[r]\leftrightarrow A[\proc{Random}(p,r)]$
\li	$x\gets a_r$
\li $i\gets p-1$
\li	\For $j\gets p$ \To $r-1$
\li		\Do \If $a_j\le x$
\li				\Then $i\gets i+1$
\li					zamień $A[i]\leftrightarrow A[j]$
				\End
		\End
\li	zamień $A[i+1]\leftrightarrow A[r]$ \label{li:fuzzy-partition-swap}
\li	$q\gets i+1$ \label{li:fuzzy-partition-q-init}
\li	\For $k\gets i$ \Downto $p$
\li		\Do \If $b_k\ge x$
\li				\Then $q\gets q-1$
\li					zamień $A[q]\leftrightarrow A[k]$
				\End
		\End \label{li:fuzzy-partition-for-end}
\li	\Return $\langle q,i+1\rangle$
\end{codebox}
Lewy koniec przedziału stanowiącego element rozdzielający, wybrany losowo spośród wszystkich elementów tablicy wejściowej, zostaje przypisany do zmiennej $x$.
Po wykonaniu wiersza \ref{li:fuzzy-partition-swap} tablica $A$ jest podzielona na dwie podtablice według lewych końców przedziałów względem $x$.
Ta część jest więc analogiczna do zwykłej procedury \proc{Partition}, w~wyniku czego dostajemy dwa obszary tablicy rozdzielone elementem $x$.
Następnie, w~wierszach \doubledash{\ref{li:fuzzy-partition-q-init}}{\ref{li:fuzzy-partition-for-end}}, wszystkie przedziały z~podtablicy $A[p\twodots i]$, które nachodzą na element rozdzielający znajdujący się teraz w~$A[i+1]$, zostają przeniesione na koniec tej podtablicy.
W~rezultacie z~przedziałów tych utworzony zostaje trzeci obszar tablicy, niewymagający dalszego sortowania.
Na końcu zwracane są indeksy początku i~końca tego obszaru.

\subproblem %7-6(b)
Algorytm został oparty o~randomizowaną wersję quicksorta, więc jego czas działania dla tablicy \singledash{$n$}{elementowej} w~przypadku średnim wynosi $\Theta(n\lg n)$.
Jeśli jednak wszystkie przedziały na siebie nachodzą, to lewa część podtablicy podzielonej w~wyniku działania procedury \proc{Fuzzy-Partition} będzie za każdym razem pusta.
Na każdym poziomie rekursji będzie więc sortowany tylko jeden obszar.
Randomizacja zapewnia, że oczekiwaną pozycją elementu rozdzielającego jest środek podtablicy $A[p\twodots r]$ (patrz \refExercise{C.3-2}) i~w~kolejnym wywołaniu rekurencyjnym sortowany fragment jest o~połowę mniejszy.
Oczekiwany czas działania algorytmu w~tym przypadku jest więc opisany przez rekurencję $T(n)=T(n/2)+\Theta(n)$, której rozwiązaniem jest $\Theta(n)$.
