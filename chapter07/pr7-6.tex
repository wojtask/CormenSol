\problem{Rozmyte sortowanie przedziałów} %7-6

\subproblem %7-6(a)
Niech $A$ będzie tablicą wejściową, przy czym $A[i]=[a_i,b_i]$ dla $i=1$, 2, \dots, $n$.
Algorytm działa podobnie jak quicksort, ale wykorzystuje obserwację, że jeśli przedziały $A[i]$ i~$A[j]$ nachodzą na siebie (czyli $[a_i,b_i]\cap[a_j,b_j]\ne\emptyset$), to w~tablicy wynikowej mogą one wystąpić w~dowolnej kolejności.
Dzięki temu zbiór przedziałów nachodzących na przedział stanowiący element rozdzielający nie musi uczestniczyć w~kolejnym wywołaniu rekurencyjnym.

Procedura \proc{Fuzzy-Sort} implementuje rozmyte sortowanie przedziałów.
Aby posortować całą tablicę $A$, należy wywołać $\proc{Fuzzy-Sort}(A,1,\attrib{A}{length})$.
\begin{codebox}
\Procname{$\proc{Fuzzy-Sort}(A,p,r)$}
\li	\If $p<r$
\li		\Then $\langle q_1,q_2\rangle\gets\proc{Fuzzy-Partition}(A,p,r)$
\li			$\proc{Fuzzy-Sort}(A,p,q_1-1)$ \label{li:fuzzy-sort-recursion1}
\li			$\proc{Fuzzy-Sort}(A,q_2+1,r)$ \label{li:fuzzy-sort-recursion2}
		\End
\end{codebox}

Pomocnicza procedura \proc{Fuzzy-Partition} dokonuje podziału tablicy $A[p\twodots r]$ na 3 podtablice: $A[q_1\twodots q_2]$, której nie trzeba już dalej sortować, oraz $A[p\twodots q_1-1]$ i~$A[q_2+1\twodots r]$, które przekazywane są do wywołań rekurencyjnych w~wierszach \ref{li:fuzzy-sort-recursion1} i~\ref{li:fuzzy-sort-recursion2}.
\begin{codebox}
\Procname{$\proc{Fuzzy-Partition}(A,p,r)$}
\li	$x\gets a_r$
\li $i\gets p-1$
\li	\For $j\gets p$ \To $r-1$
\li		\Do \If $a_j\le x$
\li				\Then $i\gets i+1$
\li					zamień $A[i]\leftrightarrow A[j]$
				\End
		\End
\li	zamień $A[i+1]\leftrightarrow A[r]$ \label{li:fuzzy-partition-swap}
\li	$q\gets i+1$ \label{li:fuzzy-partition-q-init}
\li	\For $k\gets i$ \Downto $p$
\li		\Do \If $b_k\ge x$
\li				\Then $q\gets q-1$
\li					zamień $A[q]\leftrightarrow A[k]$
				\End
		\End \label{li:fuzzy-partition-for-end}
\li	\Return $\langle q,i+1\rangle$
\end{codebox}
Przyjmujemy przedział $A[r]$ jako element rozdzielający, którego lewy koniec przypisujemy do zmiennej $x$.
Po wykonaniu wiersza \ref{li:fuzzy-partition-swap} tablica $A$ jest podzielona na dwie podtablice według lewych końców przedziałów względem $x$.
Ta część jest więc analogiczna do zwykłej procedury \proc{Partition}.
Następnie, w~wierszach \ref{li:fuzzy-partition-q-init}\nbendash\ref{li:fuzzy-partition-for-end}, wszystkie przedziały z~podtablicy $A[p\twodots i]$, które nachodzą na element rozdzielający znajdujący się teraz w~$A[i+1]$, zostają przeniesione na koniec tej podtablicy.
W~rezultacie z~przedziałów tych utworzony zostaje trzeci obszar tablicy, niewymagający dalszego sortowania, a~jako wynik procedury zwracane są krańce tego obszaru.

\subproblem %7-6(b)
W~porównaniu z~oryginalnym \proc{Partition}, procedura \proc{Fuzzy-Partition} wykonuje dodatkową pracę w~czasie stałym.
Jeśli żadne dwa przedziały z~wejściowego zbioru nie nachodzą na siebie, to algorytm, poza wspomnianą różnicą, zachowuje się identycznie jak oryginalny quicksort i~jego oczekiwany czas wynosi $\Theta(n\lg n)$.
W~przypadku, gdy wszystkie przedziały na siebie nachodzą, algorytm nie wywoła się rekurencyjnie i~zakończy działanie po czasie $\Theta(n)$.
