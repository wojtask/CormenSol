\problem{Poprawność algorytmu podziału Hoare'a} %7-1

\subproblem %7-1(a)
Działanie procedury \proc{Hoare-Partition} dla przykładowej tablicy $A$ zostało przedstawione na rys.\ \ref{fig:7-1a}.
\begin{figure}[!ht]
	\centering \begin{tikzpicture}[
	light cell/.style = {row 1 column #1/.style={nodes={fill=black!20}}},
	med cell/.style = {row 1 column #1/.style={nodes={fill=black!40}}},
	partition line/.style = {line width=2pt, line cap=rect},
	outer/.append style = {node distance=4mm},
	on grid
]

\node[outer] (pic a) {
\begin{tikzpicture}
	\matrix[array] (arr) {13 & 19 & 9 & 5 & 12 & 8 & 7 & 4 & 11 & 2 & 6 & 21 \\};
	\draw[partition line] (arr-1-1.north west) -- (arr-1-1.south west);
	\draw[partition line] (arr-1-12.north east) -- (arr-1-12.south east);
	\node[index node, above left=of arr-1-1] {$i$};
	\node[index node, above=of arr-1-1] {$p$};
	\node[index node, above=of arr-1-12] {$r$};
	\node[index node, above right=of arr-1-12] {$j$};
\end{tikzpicture}
};

\node[outer, below=of pic a.south] (pic b) {
\begin{tikzpicture}
	\matrix[
		array,
		light cell/.list = {1},
		med cell/.list = {11,12}
	] (arr) {6 & 19 & 9 & 5 & 12 & 8 & 7 & 4 & 11 & 2 & 13 & 21 \\};
	\draw[partition line] (arr-1-2.north west) -- (arr-1-2.south west);
	\draw[partition line] (arr-1-11.north west) -- (arr-1-11.south west);
	\node[index node, above=of arr-1-1] {$p,i$};
	\node[index node, above=of arr-1-11] {$j$};
	\node[index node, above=of arr-1-12] {$r$};
\end{tikzpicture}
};

\node[outer, below=of pic b.south] (pic c) {
\begin{tikzpicture}
	\matrix[
		array,
		light cell/.list = {1,2},
		med cell/.list = {10,11,12}
	] (arr) {6 & 2 & 9 & 5 & 12 & 8 & 7 & 4 & 11 & 19 & 13 & 21 \\};
	\draw[partition line] (arr-1-3.north west) -- (arr-1-3.south west);
	\draw[partition line] (arr-1-10.north west) -- (arr-1-10.south west);
	\node[index node, above=of arr-1-1] {$p$};
	\node[index node, above=of arr-1-2] {$i$};
	\node[index node, above=of arr-1-10] {$j$};
	\node[index node, above=of arr-1-12] {$r$};
\end{tikzpicture}
};

\node[outer, below=of pic c.south] (pic d) {
\begin{tikzpicture}
	\matrix[
		array,
		light cell/.list = {1,2,3,4,5,6,7,8,9},
		med cell/.list = {10,11,12}
	] (arr) {6 & 2 & 9 & 5 & 12 & 8 & 7 & 4 & 11 & 19 & 13 & 21 \\};
	\draw[partition line] (arr-1-10.north west) -- (arr-1-10.south west);
	\node[index node, above=of arr-1-1] {$p$};
	\node[index node, above=of arr-1-9] {$j$};
	\node[index node, above=of arr-1-10] {$i$};
	\node[index node, above=of arr-1-12] {$r$};
\end{tikzpicture}
};

\node[subpicture label, left=1mm of pic a.south west, anchor=south] (label a) {(a)};
\foreach \x in {b,c,d} {
	\node[subpicture label, anchor=south] at (label a |- pic \x.south west) {(\x)};
}

\end{tikzpicture}

	\caption{Działanie procedury \proc{Hoare-Partition} dla tablicy $A=\langle13,19,9,5,12,8,7,4,11,2,6,21\rangle$.
Wszystkie elementy jasnoszare stanowią obszar złożony z~wartości nie większych niż $x=A[p]=13$.
Ciemnoszare elementy tworzą obszar złożony z~wartości nie mniejszych niż $x$.
{\sffamily\bfseries(a)} Wejściowa tablica wraz z~początkowym ustawieniem zmiennych.
{\sffamily\bfseries(b)\nbendash(d)} Tablica i~bieżące wartości zmiennych po wykonaniu, odpowiednio, jednej, dwóch i~trzech iteracji pętli \kw{while} w~wierszach 4\nbendash11.} \label{fig:7-1a}
\end{figure}

\subproblem %7-1(b)
W~pierwszej iteracji pętli \kw{while} indeks $j$ zatrzyma się na pozycji $q$, gdzie $p\le q\le r$, a~indeks $i$ pozostanie na pozycji $p$, na której znajduje się wartość $x$.
Jeśli $i=j$, to procedura kończy działanie, załóżmy więc, że $i<j$.
Elementy $A[i]=A[p]=x$ i~$A[q]$ zostaną wówczas zamienione miejscami.
W~trakcie dalszego działania procedury indeks $i$ nie przekroczy $q$, a~indeks $j$ nie zostanie zmniejszony poniżej $p$, bo $A[p]$ zawiera teraz wartość mniejszą bądź równą $x$.

\subproblem %7-1(c)
\note{Zakładamy, że podtablica\/ $A[p\twodots r]$ składa się z~co najmniej dwóch elementów.}

\noindent W~pierwszej iteracji pętli \kw{while} indeks $i$ zatrzymuje się na pozycji $p$, bo $A[p]=x\ge x$.
Jeśli więc procedura kończy działanie, wykonawszy tylko jedną iterację, to musi być $j=p<r$.
W~przypadku wykonania przez procedurę więcej niż jednej iteracji również będzie $j<r$, bo indeks $j$ jest dekrementowany w~każdej iteracji.
Ponadto, na podstawie punktu (b), $j\ge p$.

\subproblem %7-1(d)
Teza wynika wprost z~niezmiennika pętli:
\begin{quote}
W~każdej iteracji pętli \kw{while} tuż przed porównaniem $i$ z~$j$ każdy element $A[p\twodots i-1]$ jest mniejszy lub równy $x$, a~każdy element $A[j+1\twodots r]$ jest większy lub równy $x$.
\end{quote}
\begin{description}
	\item[Inicjowanie:] W~pierwszej iteracji niezmiennik zachodzi, dzięki wykonaniu obu pętli \kw{repeat}.
	\item[Utrzymanie:] Załóżmy, że niezmiennik jest spełniony i~że warunek w~wierszu 9 jest prawdziwy.
Skutkuje to zamianą elementu $A[i]$ z~$A[j]$, dzięki czemu teraz każdy element $A[p\twodots i]$ jest mniejszy lub równy $x$, a~każdy element $A[j\twodots r]$ jest większy lub równy $x$.
Inkrementacja indeksu $i$ oraz dekrementacja indeksu $j$ w~kolejnej iteracji przywraca niezmiennik.
	\item[Zakończenie:] Pętla kończy działanie, gdy $i\ge j$.
Niezmiennik oczywiście pozostaje prawdziwy.
\end{description}

\subproblem %7-1(e)
Jedyną różnicą w~porównaniu z~procedurą \proc{Quicksort} jest to, że element rozdzielający nie znajduje się w~$A[q]$ po wykonaniu \proc{Hoare-Partition}, musimy więc sortować rekurencyjnie fragment $A[p\twodots q]$ zamiast $A[p\twodots q-1]$.
Pseudokod procedury został przedstawiony poniżej.
\begin{codebox}
\Procname{$\proc{Hoare-Quicksort}(A,p,r)$}
\li	\If $p<r$
\li		\Then $q\gets\proc{Hoare-Partition}(A,p,r)$
\li			$\proc{Hoare-Quicksort}(A,p,q)$
\li			$\proc{Hoare-Quicksort}(A,q+1,r)$
		\End
\end{codebox}
