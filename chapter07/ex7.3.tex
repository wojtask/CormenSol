\subchapter{Randomizowana wersja algorytmu quicksort}

\exercise %7.3-1
Randomizacja algorytmu sprawia, że dla każdych danych wejściowych szanse na zaistnienie przypadku pesymistycznego są jednakowo niskie.
Można więc przyjąć, że dla dowolnych danych wejściowych algorytm zachowuje się jak w~przypadku średnim.

\exercise %7.3-2
W~przypadku pesymistycznym generator liczb losowych za każdym razem wybiera wartości, które $n$\nbhyphen elementową tablicę dzielą na podtablice o~rozmiarach 0 i~$n-1$.
Liczba wywołań generatora spełnia więc zależność $R(n)=R(n-1)+1$, której rozwiązaniem jest $R(n)=\Theta(n)$.

W~przypadku optymistycznym tworzone są najbardziej zrównoważone podziały tablicy, zatem liczbę wywołań generatora opisuje wówczas rekurencja $R(n)=2R(n/2)+1$, również o~rozwiązaniu $R(n)=\Theta(n)$.
