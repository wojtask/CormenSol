\section*{Rozdział 4: Rekurencje}

\subsection*{4.1. Metoda podstawiania}

\paragraph{4.1-1.}
\paragraph{4.1-2.}
\paragraph{4.1-3.}
\paragraph{4.1-4.}
\paragraph{4.1-5.}
\paragraph{4.1-6.}

\subsection*{4.2. Metoda drzewa rekursji}

\paragraph{4.2-1.}
\paragraph{4.2-2.}
\paragraph{4.2-3.}
\paragraph{4.2-4.}
\paragraph{4.2-5.}

\subsection*{4.3. Metoda rekurencji uniwersalnej}

\paragraph{4.3-1.}
\paragraph{4.3-2.}
\paragraph{4.3-3.}
\paragraph{4.3-4.}
\paragraph{4.3-5.}

\subsection*{4.4. Dowód twierdzenia o rekurencji uniwersalnej}

\paragraph{4.4-1.}
\paragraph{4.4-2.}
\paragraph{4.4-3.}

\subsection*{Problemy}

\paragraph{4-1. Przykłady rekurencji}

\subparagraph{(a)}
\subparagraph{(b)}
\subparagraph{(c)}
\subparagraph{(d)}
\subparagraph{(e)}
\subparagraph{(f)}
\subparagraph{(g)}
\subparagraph{(h)}

\paragraph{4-2. Szukanie brakującej liczby całkowitej}

\paragraph{4-3. Koszty przekazywania parametrów}

\subparagraph{(a)}
\subparagraph{(b)}

\paragraph{4-4. Więcej przykładów rekurencji}

\subparagraph{(a)}
\subparagraph{(b)}
\subparagraph{(c)}
\subparagraph{(d)}
\subparagraph{(e)}
\subparagraph{(f)}
\subparagraph{(g)}
\subparagraph{(h)}
\subparagraph{(i)}
\subparagraph{(j)}

\paragraph{4-5. Liczby Fibonacciego}

\subparagraph{(a)}
\subparagraph{(b)}
\subparagraph{(c)}
\subparagraph{(d)}
\subparagraph{(e)}

\paragraph{4-6. Testowanie układów VLSI}

\subparagraph{(a)}
\subparagraph{(b)}
\subparagraph{(c)}

\paragraph{4-7. Tablice Monge'a}

\subparagraph{(a)}
\subparagraph{(b)}
\subparagraph{(c)}
\subparagraph{(d)}
\subparagraph{(e)}
