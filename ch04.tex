\chapter{Rekurencje}

\subchapter{Metoda podstawiania}

\exercise %4.1-1
Niech $c>0$ będzie stałą.
Przyjmujemy założenie, że
\[
    T(\lceil n/2\rceil)\le c\lg(\lceil n/2\rceil)
\]
i~że $n\ge4$.
Na podstawie założeń i~wzoru~(3.3) otrzymujemy:
\begin{align*}
	T(n) &\le c\lg(\lceil n/2\rceil)+1 \\
	&< c\lg(n/2+1)+1 \\
	&\le c\lg(3n/4)+1 \\
	&= c\lg n+c\lg(3/4)+1 \\
	&\le c\lg n,
\end{align*}
przy czym ostatnia nierówność jest spełniona, gdy $c\ge\log_{4/3}2$.

Niech $T(1)=1$.
Wówczas $T(2)=2$, $T(3)=3$.
Nierówności $T(2)\le c\lg2$ oraz $T(3)\le c\lg3$ zachodzą dla $c\ge2$, a~więc w~szczególności dla $c\ge\log_{4/3}2$.
Można zatem przyjąć $n=2$ i~$n=3$ za podstawę indukcji, gdyż dla $n\ge4$ rekurencja nie zależy bezpośrednio od $T(1)$.
Na mocy dowodu indukcyjnego wynika zatem, że $T(n)=O(\lg n)$.

\exercise %4.1-2
Przyjmijmy założenie, że
\[
	T(\lfloor n/2\rfloor) \ge c\lfloor n/2\rfloor\lg(\lfloor n/2\rfloor)
\]
dla pewnej dodatniej stałej~$c$.
Korzystając ze wzoru~(3.3), mamy:
\begin{align*}
	T(n) &\ge 2c\lfloor n/2\rfloor\lg(\lfloor n/2\rfloor)+n \\
	&> 2c(n/2-1)\lg(n/4)+n \\
	&= 2c((n/2)\lg n-n-\lg n+2)+n \\
	&= cn\lg n-2cn-2c\lg n+4c+n \\
	&> cn\lg n-2c(n+\lg n)+n \\
	&\ge cn\lg n.
\end{align*}
Ostatni krok uzasadniamy, rozwiązując nierówność $-2c(n+\lg n)+n\ge0$ ze względu na $c$:
\[
	c \le \frac{n}{2(n+\lg n)} \le \frac{n}{2n} = \frac{1}{2}.
\]
Wybierając dowolne $0<c\le1/2$, spełniamy ostatnią nierówność wyprowadzenia.
Można przyjąć $T(1)=1$ i~$n=1$ za podstawę indukcji, bo wówczas $T(1)\ge c\cdot1\cdot\lg1=0$.
Wykazaliśmy, że $T(n)=\Omega(n\lg n)$, więc na mocy tw.~3.1 mamy $T(n)=\Theta(n\lg n)$.

\exercise %4.1-3
Udowodnimy, że $T(n)=O(n^2)$.
Przyjmijmy w~tym celu założenie dla $n\ge2$, że
\[
    T(\lfloor n/2\rfloor) \le c\lfloor n/2\rfloor^2,
\]
gdzie $c>0$ jest stałą.
Mamy teraz
\[
	T(n) \le 2c\lfloor n/2\rfloor^2+n \le 2cn^2\!/4 + n = cn^2\!/2+n \le cn^2,
\]
co jest prawdą, o~ile $c\ge1$.
Definiując $T(1)=1$, dzięki mocniejszemu założeniu indukcyjnemu można przyjąć $n=1$ w~podstawie indukcji, bo $T(1)\le c\cdot1^2=c$.

\exercise %4.1-4
Rekurencję~(4.2) można przedstawić w~alternatywny sposób:
\[
	T(n) =
	\begin{cases}
		d_1, & \text{jeśli $n=1$}, \\
		T(\lceil n/2\rceil)+T(\lfloor n/2\rfloor)+d_2n, & \text{jeśli $n>1$},
	\end{cases}
\]
gdzie $d_1$,~$d_2>0$ są stałymi.
Udowodnimy oszacowania górne i~dolne dla $T(n)$ przy użyciu indukcji, przy czym założymy, że $n\ge4$ i~skorzystamy z~obserwacji, że $T(n)$ jest funkcją niemalejącą.
Dla stałych $c_1$,~$c_2>0$ przyjmijmy założenia
\[
	T(\lfloor n/2\rfloor) \ge c_1\lfloor n/2\rfloor\lg(\lfloor n/2\rfloor) \quad\text{oraz}\quad T(\lceil n/2\rceil) \le c_2\lceil n/2\rceil\lg(\lceil n/2\rceil).
\]

W~przypadku dolnego oszacowania mamy:
\begin{align*}
	T(n) &\ge 2T(\lfloor n/2\rfloor)+d_2n \\
	&\ge 2c_1\lfloor n/2\rfloor\lg(\lfloor n/2\rfloor)+d_2n \\
	&> 2c_1(n/2-1)\lg(n/4)+d_2n \\
	&= c_1n\lg n-2c_1n-2c_1\!\lg(n/4)+d_2n \\
	&\ge c_1n\lg n,
\end{align*}
ponieważ zachodzi $\lfloor n/2\rfloor>n/2-1$ i~$\lfloor n/2\rfloor\ge n/4$ oraz można tak dobrać stałą $c_1$, aby prawdziwa była ostatnia nierówność powyższego wyprowadzenia.
Po podstawieniu $c_1=d_2/4$ sprowadza się ona do postaci $n\ge\lg(n/4)$, co jest prawdą dla wszystkich $n$ dodatnich.
Podstawę indukcji stanowi $n=1$, gdyż $T(1)\ge c_1\cdot 1\cdot\lg1=0$, a~zatem $T(n)=\Omega(n\lg n)$.

Wykorzystując nierówność $\lceil n/2\rceil<n/2+1$, dowodzimy górnego oszacowania:
\begin{align*}
	T(n) &\le 2T(\lceil n/2\rceil)+d_2n \\
	&\le 2c_2\lceil n/2\rceil\lg(\lceil n/2\rceil)+d_2n \\
	&< 2c_2(n/2+1)\lg\frac{n\lceil n/2\rceil}{n}+d_2n \\
	&\le c_2n\lg n+c_2(n+2)\lg\frac{\lceil n/2\rceil}{n}+2c_2\lg n+d_2n \\
	&\le c_2n\lg n.
\end{align*}
W~ostatniej nierówności skorzystano z~faktu, że dla $n\ge2$ wyrażenie $\lg\frac{\lceil n/2\rceil}{n}$ jest ujemne, a~zatem, dobierając odpowiednio duże $c_2$, można uzasadnić nierówność, gdyż funkcja liniowa rośnie szybciej od logarytmicznej.
Dokładniej, okazuje się, że przyjęcie $c_2\ge10d_2$ wystarcza, aby spełnić nierówność dla $n\ge4$.

Ponieważ dla $n\ge4$ rekurencja nie zależy bezpośrednio od $T(1)$, to za podstawę indukcji można przyjąć $n=2$ i~$n=3$.
Mamy $T(2)=2d_1+2d_2$ oraz $T(3)=3d_1+5d_2$.
Łatwo sprawdzić, że otrzymane oszacowanie zachodzi dla tych dwóch wartości, o~ile $d_1$ jest dostatecznie małe.
W~przeciwnym przypadku górne oszacowanie może nie być wystarczające, jednak zwiększenie $c_2$ pozwala na dowolne ograniczenie rekurencji od góry w~zależności od wartości stałych $d_1$ i~$d_2$.
Niezależnie od ich doboru oszacowaniem górnym rekurencji $T(n)$ jest $O(n\lg n)$, co na mocy wcześniejszego wyniku dolnego oszacowania implikuje $T(n)=\Theta(n\lg n)$.

\exercise %4.1-5
Wykorzystując założenie
\[
	T(\lfloor n/2\rfloor+17) \le c(\lfloor n/2\rfloor+17)\lg(\lfloor n/2\rfloor+17)
\]
dla pewnej stałej $c>0$, otrzymujemy:
\begin{align*}
	T(n) &\le 2c(\lfloor n/2\rfloor+17)\lg(\lfloor n/2\rfloor+17)+n \\
	&\le 2c(n/2+17)\lg(n/2+17)+n \\
	&\le c(n+34)\lg(11n/20)+n \\
	&< cn\lg n+cn\lg(11/20)+34c\lg n+n \\
	&\le cn\lg n.
\end{align*}
Nierówności zachodzą, o~ile $n/2+17\le 11n/20$, czyli $n\ge340$, oraz
\[
	cn\lg(11/20)+34c\lg n+n \le 0.
\]
Badając ostatnią nierówność, można dojść do rezultatu, że przyjęcie $c\ge47$ wystarcza, aby spełnić ją dla wszystkich $n\ge n_0$, gdzie $n_0=340$.

Zauważmy, że stosowanie równania $T(n)=2T(\lfloor n/2\rfloor+17)+n$ dla $n\le34$ nie ma sensu, bo wtedy $T(n)$ nie zależy od niższych wyrazów.
Przyjmijmy zatem, że $T(n)=1$ dla wszystkich $n\le34$ i~niech stanowi to przypadek brzegowy rekurencji.
Za podstawę indukcji musimy jednak przyjąć wszystkie $n=187$, 188,~\dots,~339.
Posługując się programem komputerowym, można wykazać, że dla $c\ge47$ wszystkie te wartości spełniają oszacowanie, a~zatem $T(n)=O(n\lg n)$.

Analiza tej rekurencji z~każdą inną stałą w~miejscu~17 przebiega analogicznie, zmianie ulegają natomiast wartości $n_0$ i~$c$ oraz wartości brzegowe rekursji i~podstawa indukcji, jednak w~każdym takim przypadku rekurencja jest klasy $O(n\lg n)$.

\exercise %4.1-6
Przyjmijmy, że $n=2^m$, skąd $m=\lg n$.
Rekurencja przyjmuje teraz postać
\[
	T(2^m) = 2T(2^{m/2})+1.
\]
Z~kolei podstawiając $S(m)$ za $T(2^m)$, dostajemy nową rekurencję
\[
	S(m) = 2S(m/2)+1,
\]
dla której udowodnimy rozwiązanie $\Theta(\lg m)$.

Ponieważ całkowitość argumentów nie jest dla nas istotna, to możemy rekurencję potraktować jako $S(m)=S(\lfloor m/2\rfloor)+S(\lceil m/2\rceil)+1$, co, według Podręcznika, jest $O(m)$.

Aby uzyskać oszacowanie dokładne, pozostaje udowodnić, że $S(m)=\Omega(m)$.
Przyjmujemy zatem założenie, że $S(m/2)\ge cm/2$ dla $c>0$ i~na jego podstawie otrzymujemy:
\[
	S(m) \ge 2cm/2+1 = cm+1 > cm,
\]
co zachodzi dla dowolnej wartości $c$.
Niech $S(1)=1$.
Przyjęcie $m=1$ na podstawę indukcji wystarcza, bo warunek $S(1)\ge c$ spełnia każde $c\le1$, a~zatem $S(m)=\Omega(m)$.

Stosując tw.~3.1, mamy $S(m)=\Theta(m)$.
Wracając teraz do oryginalnej rekurencji i~starej zmiennej, dostajemy $T(n)=T(2^m)=S(m)=\Theta(m)=\Theta(\lg n)$.

\subchapter{Metoda drzewa rekursji}

\exercise %4.2-1
Dokonamy pewnego uproszczenia, pomijając podłogę w~argumencie rekurencji i~rozważając zależność $T(n)=3T(n/2)+n$.
\begin{figure}[ht]
	\begin{center}
		\includegraphics{fig_4.2-1}
	\end{center}
	\caption{Drzewo rekursji dla równania rekurencyjnego $T(n)=3T(n/2)+n$.} \label{fig:4.2-1}
\end{figure}
Drzewo tej rekursji przedstawione na rys.~\ref{fig:4.2-1} ma wysokość równą $\lg n$, czyli jest w~nim $\lg n+1$ poziomów.
Na \singledash{$i$}{tym} poziomie znajduje się $3^i$ węzłów, zatem jest $n^{\lg3}$ liści.
Koszt węzła na poziomie $i$ wynosi $n/2^i$, skąd wynika, że łączny koszt wszystkich węzłów na \singledash{$i$}{tej} głębokości jest równy $(3/2)^in$.
Przyjmujemy, że $T(1)=1$, a~więc na ostatnim poziomie jest $\Theta(n^{\lg3})$.
Na podstawie tych wartości rozwiązujemy rekurencję:
\begin{align*}
	T(n) &= n+\frac{3}{2}\,n+\biggl(\frac{3}{2}\biggr)^2n+\dots+\biggl(\frac{3}{2}\biggr)^{\lg n-1}n+\Theta(n^{\lg3}) \\
	&= \sum_{i=0}^{\lg n-1}\biggl(\frac{3}{2}\biggr)^in+\Theta(n^{\lg3}) \\
	&= \frac{(3/2)^{\lg n}-1}{(3/2)-1}\,n+\Theta(n^{\lg3}) \\[1mm]
	&= 2n(n^{\lg3-1}-1)+\Theta(n^{\lg3}) \\
	&= O(n^{\lg3}).
\end{align*}

Pokażemy teraz, że otrzymany wynik stanowi górne oszacowanie dla oryginalnej rekurencji.
Przyjmujemy założenie, że
\[
    T(\lfloor n/2\rfloor)\le c\lfloor n/2\rfloor^{\lg3}
\]
dla pewnej stałej $c>0$.
Dowodzimy oszacowania dla oryginalnej rekurencji metodą podstawiania:
\[
	T(n) \le 3c\lfloor n/2\rfloor^{\lg3}+n \le 3c(n/2)^{\lg3}+n = cn^{\lg3}+n.
\]
Nie możemy jednak na podstawie tego wyniku wywnioskować szukanego oszacowania.
Wzmocnijmy zatem nasze założenie, niech
\[
	T(\lfloor n/2\rfloor) \le c\lfloor n/2\rfloor^{\lg3}-b\lfloor n/2\rfloor,
\]
gdzie $b\ge0$ jest nową stałą.
Korzystając ze wzoru~(3.3), mamy teraz:
\begin{align*}
	T(n) &\le 3c\lfloor n/2\rfloor^{\lg3}-3b\lfloor n/2\rfloor+n \\
	&< 3c(n/2)^{\lg3}-3b(n/2-1)+n \\
	&= cn^{\lg3}-3bn/2+3b+n \\
	&\le cn^{\lg3}-bn,
\end{align*}
co zachodzi dla $n\ge7$, o~ile $b\ge14$.
Przyjmujemy wszystkie $n=3$, 4,~5,~6 na podstawę indukcji, ponieważ dla dowolnie ustalonego $b$ można dobrać dla stałej $c$ odpowiednią wartość tak, aby pokazane oszacowanie zachodziło dla takich $n$.
To kończy dowód faktu, że górnym oszacowaniem rekurencji $T(n)$ jest $O(n^{\lg3})$.

\exercise %4.2-2
Drzewo rekursji $T(n)$ nie jest pełnym drzewem binarnym.
Najkrótszą ścieżką od korzenia do liścia jest $cn\to cn/3\to cn/9\to\dots\to cn/3^i\to\dots\to T(1)$.
Liść tej gałęzi znajduje się na poziomie $h=\log_3n$.
Gdybyśmy pominęli wszystkie poziomy tego drzewa poniżej \singledash{$h$}{tego}, to uzyskalibyśmy pełne drzewo binarne o~wysokości $h$ i~łącznym koszcie $\Theta(n\lg n)$.
Ale tak zmodyfikowane drzewo jest mniejsze niż drzewo rekursji $T(n)$, przez co jego koszt stanowi oszacowanie dolne rekurencji $T(n)$, czyli $T(n)=\Omega(n\lg n)$.

\exercise %4.2-3
W~metodzie drzewa rekursji dla uproszczenia pomijamy branie części całkowitych.
\begin{figure}[ht]
	\begin{center}
		\includegraphics{fig_4.2-3}
	\end{center}
	\caption{Drzewo rekursji dla równania rekurencyjnego $T(n)=4T(n/2)+cn$.} \label{fig:4.2-3}
\end{figure}
W~drzewie rekursji z~rys.~\ref{fig:4.2-3} na \singledash{$i$}{tym} poziomie jest $4^i$ węzłów, z~których każdy wnosi koszt równy $cn/2^i$.
Stąd kosztem całego poziomu jest $2^icn$.
Współczynnik przy $n$ w~koszcie węzła maleje dwukrotnie wraz z~głębokością drzewa, więc jego wysokością jest $\lg n$.
Wnioskujemy zatem, że liczbą liści w~tym drzewie jest $4^{\lg n}=n^2$ i~że koszt ostatniego poziomu wynosi $\Theta(n^2)$.
Sumując koszty z~każdego poziomu, otrzymujemy:
\begin{align*}
	T(n) &= cn+2cn+2^2cn+\dots+2^{\lg n-1}cn+\Theta(n^2) \\
	&= cn\sum_{i=0}^{\lg n-1}2^i+\Theta(n^2) \\
	&= cn(2^{\lg n}-1)+\Theta(n^2) \\
	&= cn(n-1)+\Theta(n^2) \\
	&= \Theta(n^2).
\end{align*}

Sprawdzimy teraz otrzymany wynik, używając w~tym celu metody podstawiania dla oryginalnej rekurencji.
Przyjmiemy ponadto, że $T(1)=1$ stanowi jej przypadek brzegowy.
Badamy najpierw oszacowanie dolne $T(n)$, wykorzystując założenie
\[
	T(\lfloor n/2\rfloor) \ge c_1\lfloor n/2\rfloor^2
\]
dla pewnej stałej $c_1>0$.
Stąd
\[
	T(n) \ge 4c_1\lfloor n/2\rfloor^2+cn > 4c_1(n/2-1)^2+cn = c_1n^2-4c_1n+4c_1+cn \ge c_1n^2,
\]
co jest prawdą dla $n\ge2$, jeśli przyjmiemy, że $c_1\le c/4$.
Podstawą indukcji jest $n=1$, bo $T(1)$ spełnia oszacowanie dla $c_1\le1$.
A~więc istotnie $T(n)=\Omega(n^2)$.

Pokażemy teraz, że $T(n)=O(n^2)$.
W~tym celu można przyjąć analogiczne założenie indukcyjne jak przy dowodzie dolnego oszacowania.
Niestety założenie to okazuje się zbyt słabe i~nie prowadzi do żądanego wyniku.
Przyjmijmy zatem, że dla stałych $c_2>0$ i~$c_3\ge0$ zachodzi mocniejszy warunek
\[
	T(\lfloor n/2\rfloor) \le c_2\lfloor n/2\rfloor^2-c_3\lfloor n/2\rfloor.
\]
Wówczas:
\begin{align*}
	T(n) &\le 4(c_2\lfloor n/2\rfloor^2-c_3\lfloor n/2\rfloor)+cn \\
	&< 4(c_2(n/2)^2-c_3(n/2-1))+cn \\
	&= c_2n^2-2c_3n+4c_3+cn \\
	&\le c_2n^2-c_3n.
\end{align*}
Ostatnia nierówność jest prawdziwa dla $n\ge5$, o~ile $c_3\ge5c$.

Zajmiemy się teraz ustaleniem stałych $c_2$ i~$c_3$, aby spełnić podstawę tej indukcji.
Początkowe wyrazy rekurencji wynoszą $T(1)=1$, $T(2)=4+2c$, $T(3)=4+3c$, $T(4)=16+12c$.
Jeśli przyjmiemy $c_3=5c$, to dla dowolnego $c_2\ge5c+1$ wyrażenie $c_2n^2-c_3n$ będzie stanowić oszacowanie górne dla tych wartości.
Zachodzi zatem $T(n)=O(n^2)$ i~dowód dokładnego oszacowania na $T(n)$ jest zakończony.

\exercise %4.2-4
Dla uproszczenia przyjmijmy, że $T(n)=ca$ w~przypadku, gdy $n\le a$, czyli że dla dostatecznie małego $n$ rekurencja przyjmuje wartość stałą równą $ca$.
Drzewo rekursji $T(n)$ zostało zilustrowane na rys.~\ref{fig:4.2-4}.
\begin{figure}[ht]
	\begin{center}
		\includegraphics{fig_4.2-4}
	\end{center}
	\caption{Drzewo rekursji dla równania rekurencyjnego $T(n)=T(n-a)+T(a)+cn$.} \label{fig:4.2-4}
\end{figure}
Wysokością tego drzewa jest $n/a$.
Koszt wnoszony przez \singledash{$i$}{ty} poziom (z~wyjątkiem zerowego i~ostatniego) wynosi $c(n-a(i-1))$, przy czym na ostatnim poziomie jest tylko jeden liść, który kosztuje $\Theta(1)$.
Mamy zatem:
\begin{align*}
	T(n) &= cn+\sum_{i=1}^{n/a-1}c(n-a(i-1))+\Theta(1) \\
	&= cn+c\sum_{i=0}^{n/a-2}(n-ai)+\Theta(1) \\
	&= cn+cn\sum_{i=0}^{n/a-2}1-ca\sum_{i=0}^{n/a-2}i+\Theta(1) \\
	&= cn+cn(n/a-1)-\frac{ca(n/a-2)(n/a-1)}{2}+\Theta(1) \\
	&= \Theta(n^2).
\end{align*}

\exercise %4.2-5
Zauważmy, że drzewo rekurencji $T(n)$ z~rys.~\ref{fig:4.2-5} dla parametru $\alpha$ jest symetryczne do tego samego drzewa z~parametrem $1-\alpha$ -- przyjmijmy więc, że $0<\alpha\le1/2$.
\begin{figure}[ht]
	\begin{center}
		\includegraphics{fig_4.2-5}
	\end{center}
	\caption{Drzewo rekursji dla równania rekurencyjnego $T(n)=T(\alpha n)+T((1-\alpha)n)+cn$, gdzie $0<\alpha\le1/2$.} \label{fig:4.2-5}
\end{figure}

Wyznaczmy najpierw oszacowanie dolne rekurencji.
Na \singledash{$i$}{tym} poziomie drzewa najmniejszy koszt wnoszą elementy o~wartościach $c\alpha^in$, a~więc węzły ze skrajnie lewej gałęzi.
Najgłębszym poziomem o~komplecie węzłów w~tym drzewie jest ten, na którym element ze skrajnie lewej gałęzi drzewa osiąga wartość stałą $d>0$.
Oznaczmy przez $h$ głębokość tego poziomu.
Wtedy $c\alpha^hn=d$, skąd otrzymujemy $h=\log_{1/\alpha}(cn/d)$.
Ponieważ $\alpha$ jest stałe, to $h=\Theta(\lg n)$.
Sumując koszty węzłów drzewa $T(n)$ znajdujące się na poziomach wyższych niż \singledash{$h$}{ty}, uzyskujemy oszacowanie dolne rekurencji, które wynosi $T(n)\ge cnh=\Omega(n\lg n)$.

Badając teraz skrajnie prawą gałąź, której elementy wnoszą największy koszt na każdym poziomie, możemy dojść do oszacowania górnego dla $T(n)$.
Na głębokości $H$ równej wysokości drzewa mamy $c(1-\alpha)^Hn=d$, skąd $H=\log_{1/(1-\alpha)}(cn/d)=\Theta(\lg n)$, a~więc $T(n)\le cn(H+1)=O(n\lg n)$.
Asymptotycznie dokładnym oszacowaniem rekurencji jest zatem $\Theta(n\lg n)$.

\subchapter{Metoda rekurencji uniwersalnej}

\exercise %4.3-1

\subexercise
We wzorze~(4.5) przyjmujemy $a=4$,~$b=2$ oraz $f(n)=n$.
Ponieważ $f(n)=O(n^{2-\epsilon})$ dla $\epsilon\le1$, to z~tw.~o~rekurencji uniwersalnej mamy $T(n)=\Theta(n^2)$.

\subexercise
Tutaj mamy te same wartości $a$ i~$b$ jak w~poprzednim punkcie, ale $f(n)=n^2$.
Z~tego, że $f(n)=\Theta(n^2)$, dostajemy $T(n)=\Theta(n^2\lg n)$.

\subexercise
Dla tych samych $a$ i~$b$ oraz $f(n)=n^3$ mamy $f(n)=\Omega(n^{2+\epsilon})$, gdzie $\epsilon\le1$ oraz
\[
	4f(n/2) = 4n^3\!/8 = n^3\!/2 = f(n)/2 \le cf(n),
\]
o~ile $c\ge1/2$.
A~zatem warunek regularności jest spełniony i~$T(n)=\Theta(n^3)$.

\exercise %4.3-2
Rozwiążmy $T(n)$, korzystając z~metody rekurencji uniwersalnej.
Stosujemy przypadek~1 tw.~4.1 i~stwierdzamy, że $n^2=O(n^{\lg7-\epsilon})$ dla $\epsilon\le\lg7-2$, a~więc zachodzi $T(n)=\Theta(n^{\lg7})$.

Pozostaje teraz zbadanie nierówności $T'(n)<T(n)$ w~zależności od parametru $a$, bo w~rekurencji $T'(n)$ jest $n^{\log_ba}=n^{\log_4a}$ oraz $f(n)=n^2$.
Załóżmy, że dla pewnej stałej $\epsilon>0$, $f(n)=O(n^{\log_4a-\epsilon})$, co jest prawdą dla $a>16$ i~wtedy $T'(n)=\Theta(n^{\log_4a})$.
Algorytm $A'$ jest efektywniejszy od algorytmu $A$, gdy $\log_4a<\lg7$, skąd $16<a<49$.
Pozostałe przypadki tw.~4.1 można stosować, o~ile $a\le16$, ale pominiemy ich sprawdzanie, bo dążymy do maksymalizacji $a$.

Największym całkowitym $a$, dla którego algorytm $A'$ jest bardziej efektywny od algorytmu $A$, jest~$a=48$.

\exercise %4.3-3
Ponieważ $a=1$ oraz $b=2$, to $n^{\log_ba}=n^0=1$ jest funkcją stałą.
W~rekurencji tej $f(n)$ także jest stałe, a~więc $f(n)=\Theta(n^{\log_ba})$ i~$T(n)=\Theta(n^{\log_ba}\lg n)=\Theta(\lg n)$.

\exercise %4.3-4
Dla rekurencji $T(n)$ mamy $a=4$, $b=2$ oraz $f(n)=n^2\lg n$, a~więc $n^{\log_ba}=n^2$, ale nie istnieje takie $\epsilon>0$, że $f(n)=\Omega(n^{2+\epsilon})$.
Nie można zatem zastosować w~rozwiązaniu twierdzenia o~rekurencji uniwersalnej.
Asymptotyczne górne oszacowanie na $T(n)$ znajdziemy natomiast, zgadując rozwiązanie, a~następnie dowodząc jego poprawności metodą podstawiania.

W~pierwszym wywołaniu rekurencja wnosi koszt równy $n^2\lg n$.
Następnie mamy 4 wywołania $T(n/2)$, co daje koszt
\[
	4T(n/2) = 4(n/2)^2\lg(n/2) = n^2\lg n-n^2.
\]
Kolejne poziomy wywołań kosztują:
\begin{gather*}
	16T(n/4) = 16(n/4)^2\lg(n/4) = n^2\lg n-2n^2, \qquad\phantom{\text{itd.}} \\
	64T(n/8) = 64(n/8)^2\lg(n/8) = n^2\lg n-3n^2 \qquad\text{itd.}
\end{gather*}
Z~otrzymanych wyników wnioskujemy, że \singledash{$i$}{ty} poziom wprowadza koszt równy $n^2\lg n-in^2$.
Liście o~koszcie stałym znajdują się na poziomie $\lg n$ i~jest ich $4^{\lg n}=n^2$, więc dostajemy:
\begin{align*}
	T(n) &= \sum_{i=0}^{\lg n-1}(n^2\lg n-in^2)+\Theta(n^2) \\
	&= n^2\lg^2n-n^2\sum_{i=0}^{\lg n-1}i+\Theta(n^2) \\[1mm]
	&= n^2\lg^2n-\frac{n^2\lg n(\lg n-1)}{2}+\Theta(n^2) \\[1mm]
	&= O(n^2\lg^2n).
\end{align*}

Udowodnimy teraz metodą podstawiania, że otrzymane przypuszczenie jest istotnie oszacowaniem górnym dla $T(n)$.
Przyjmijmy założenie
\[
	T(n/2) \le c(n/2)^2\lg^2(n/2)
\]
dla pewnej stałej $c>0$.
Mamy teraz:
\begin{align*}
	T(n) &\le 4c(n/2)^2\lg^2(n/2)+n^2\lg n \\
	&= cn^2(\lg n-1)^2+n^2\lg n \\
	&= cn^2\lg^2n-2cn^2\lg n+cn^2+n^2\lg n \\
	&\le cn^2\lg^2n.
\end{align*}
Ostatnią nierówność dla wszystkich $n\ge4$ spełniamy, dobierając $c\ge2/3$.
Przyjmujemy $T(1)=1$ jako warunek brzegowy rekurencji, zaś $n=2$ oraz $n=3$ jako podstawę indukcji, ponieważ dla $n\ge4$ rekurencja nie zależy już bezpośrednio od $T(1)$, a~otrzymane oszacowanie dla $T(2)=8$ i~$T(3)=4+9\lg3$ jest spełnione, o~ile $c\ge2$.
Rozwiązaniem rekurencji $T(n)$ jest zatem $O(n^2\lg^2n)$.

\exercise %4.3-5
Przyjmując $a=1$, $b=2$ oraz $f(n)=n(2-\cos n)$, dostajemy, że $f(n)=\Omega(n^\epsilon)$ dla pewnego $0<\epsilon\le1$.
Jednak dla $n=2(2k+1)\pi$, gdzie $k=0$, 1,~\dots, mamy
\[
	af(n/b) = 3(2k+1)\pi \quad\text{oraz}\quad f(n) = 2(2k+1)\pi,
\]
a~więc nierówność $af(n/b)\le cf(n)$ zachodzi dla $c\ge3/2$ i~warunek regularności nie jest spełniony.

\subchapter{Dowód twierdzenia o~rekurencji uniwersalnej}

\exercise %4.4-1
Wykażemy metodą indukcji, że $n_j=\bigl\lceil n/b^j\bigr\rceil$.
Z~definicji~(4.12) bezpośrednio wynika prawdziwość tego wzoru dla $j=0$.
Przyjmijmy zatem, że dla $j>0$ zachodzi $n_{j-1}=\bigl\lceil n/b^{j-1}\bigr\rceil$.
Wykorzystując tożsamość~(3.4), otrzymujemy
\[
	n_j = \lceil n_{j-1}/b\rceil = \bigl\lceil\bigl\lceil n/b^{j-1}\bigr\rceil/b\bigr\rceil = \bigl\lceil n/b^j\bigr\rceil,
\]
co należało pokazać.

\exercise %4.4-2
Zastąpmy drugi przypadek tw.~4.1 ogólniejszym warunkiem, tzn.\ jeśli $f(n)=\Theta(n^{\log_ba}\lg^kn)$ dla $k\ge0$, to zachodzi $T(n)=\Theta(n^{\log_ba}\lg^{k+1}n)$.
Dla tak zmodyfikowanego twierdzenia przeprowadzimy dowody analogicznie zmodyfikowanych lematów~4.3 i~4.4 (oznaczonych poniżej przez 4.3$'$ i~4.4$'$).

\begin{proof}[Dowód lematu~4.3\/$'$]
	Przy założeniu, że $f(n)=\Theta(n^{\log_ba}\lg^kn)$, otrzymujemy
	\[
		f(n/b^j)=\Theta\bigl((n/b^j)^{\log_ba}\lg^k(n/b^j)\bigr)
	\]
	i~podstawiamy do wzoru~(4.7):
	\[
		g(n) = \Theta\biggl(\sum_{j=0}^{\log_bn-1}a^j\biggl(\frac{n}{b^j}\biggr)^{\log_ba}\lg^k\frac{n}{b^j}\biggr).
	\]
	Mamy dalej:
	\begin{align*}
		\sum_{j=0}^{\log_bn-1}a^j\biggl(\frac{n}{b^j}\biggr)^{\log_ba}\lg^k\frac{n}{b^j} &= n^{\log_ba}\sum_{j=0}^{\log_bn-1}\biggl(\frac{a}{b^{\log_ba}}\biggr)^j\lg^k\frac{n}{b^j} \\
		&= n^{\log_ba}\sum_{j=0}^{\log_bn-1}(\lg n-j\lg b)^k \\
		&= n^{\log_ba}\cdot\Theta(\lg^{k+1}n) \\[1mm]
		&= \Theta(n^{\log_ba}\lg^{k+1}n).
	\end{align*}
	Skorzystano ze wzoru~(3.2), podstawiając $\lg n$ w~miejsce $n$, a~następnie z~tego, że
	\[
		\sum_{j=0}^{\log_bn-1}\Theta(\lg^kn) = \log_bn\cdot\Theta(\lg^kn) = \Theta(\lg^{k+1}n).
	\]
	Pokazano, że $g(n)=\Theta(n^{\log_ba}\lg^{k+1}n)$, a~więc lemat jest prawdziwy.
\end{proof}

\begin{proof}[Dowód lematu~4.4\/$'$]
	Wystarczy wykazać jedynie drugi przypadek, bo $f(n)=\Theta(n^{\log_ba}\lg^kn)$.
Z~lematów~4.2 i~4.3$'$:
	\[
		T(n) = \Theta(n^{\log_ba})+\Theta(n^{\log_ba}\lg^{k+1}n) = \Theta(n^{\log_ba}\lg^{k+1}n),
	\]
	co należało udowodnić.
\end{proof}

Twierdzenie wynika bezpośrednio z~lematu~4.4$'$.

\exercise %4.4-3
\note{Aby udowodnić to wynikanie, musimy założyć dodatkowo, że\/ $c>0$ i~że funkcja\/ $f(n)$ jest asymptotycznie dodatnia.
W~tw.~4.1 oba te warunki wynikają z~faktu, że\/ $f(n)=\Omega(n^{\log_ba+\epsilon})$.}

\noindent Załóżmy, że dla danej funkcji asymptotycznie dodatniej $f(n)$ oraz stałych $a\ge1$, $b>1$, zachodzi warunek regularności, tzn.\ istnieje stała $0<c<1$ i~stała $n_0>0$, że dla każdego $n\ge n_0$ spełnione jest $af(n/b)\le cf(n)$.
Przyjmiemy, że $n_0$ zostało wybrane w~taki sposób, że dla wszystkich $n\ge n_0/b$ funkcja $f(n)$ osiąga wartości dodatnie.
Dla $n\ge n_0$ mamy
\[
	f(n) \ge (a/c)f(n/b),
\]
a~zatem, iterując tę nierówność, dostajemy
\[
	f(n) \ge (a/c)f(n/b) \ge (a/c)^2f(n/b^2) \ge \dots \ge (a/c)^if(n/b^i).
\]
Zakładamy, że nie można wykonać większej ilości iteracji, tzn.\ $n/b^{i-1}\ge n_0$ oraz $n/b^i<n_0$.
Stąd $\log_b(n/n_0)<i\le\log_b(n/n_0)+1$, czyli, ze wzoru~(3.3), $i=\lfloor\log_b(n/n_0)+1\rfloor$.
Zachodzi więc
\[
	f(n) \ge (a/c)^{\lfloor\log_b(n/n_0)+1\rfloor}f\bigl(n/b^{\lfloor\log_b(n/n_0)+1\rfloor}\bigr).
\]
Zauważmy, że przy zadanych ograniczeniach na $a$ i~$c$, $a/c>1$, więc $f(n)$ musi być funkcją rosnącą dla $n\ge n_0/b$, skąd
\[
	f\bigl(n/b^{\lfloor\log_b(n/n_0)+1\rfloor}\bigr) > f(n/b^{\log_b(n/n_0)+1}) = f(n_0/b).
\]
Mamy następnie
\begin{align*}
	f(n) &> (a/c)^{\log_b(n/n_0)}\,f(n/b^{\log_b(n/n_0)+1}) \\[1mm]
	&= \frac{(n/n_0)^{\log_ba}}{(n/n_0)^{\log_bc}}\,f(n_0/b) \\[1mm]
	&= n^{\log_ba-\log_bc}\cdot n_0^{\log_bc-\log_ba}\cdot f(n_0/b).
\end{align*}
Dwa ostatnie czynniki powyższego wyrażenia są dodatnie i~niezależne od $n$, możemy więc potraktować ich iloczyn jako stałą $d>0$.
Ponieważ $0<c<1$ oraz $b>1$, to $\log_bc<0$, przyjmijmy więc $\epsilon=-\log_bc$.
Otrzymujemy zatem, że dla dowolnego $n\ge n_0$
\[
	f(n) > n^{\log_ba+\epsilon}\cdot d = \Omega(n^{\log_ba+\epsilon}),
\]
co należało wykazać.

\problems

\problem{Przykłady rekurencji} %4-1
W~punktach \doubledash{(a)}{(f)} skorzystano z~twierdzenia o~rekurencji uniwersalnej.

\subproblem %4-1(a)
Mamy $n^{\log_ba}=n^{\log_22}=n$ oraz $f(n)=n^3=\Omega(n^{1+\epsilon})$ dla $\epsilon\le2$.
Ponieważ warunek regularności jest spełniony:
\begin{align*}
	2f(n/2) &\le cf(n), \\
	2n^3\!/8 &\le cn^3, \\
	c &\ge 1/4,
\end{align*}
to stąd wnioskujemy, że $T(n)=\Theta(n^3)$.

\subproblem %4-1(b)
Mamy $n^{\log_ba}=n^{\log_{10/9}1}=n^0=1$ oraz $f(n)=n=\Omega(n^\epsilon)$ dla $\epsilon\le1$.
Badamy warunek regularności:
\begin{align*}
	f(9n/10) &\le cf(n), \\
	9n/10 &\le cn, \\
	c &\ge 9/10
\end{align*}
i~stwierdzamy, że $T(n)=\Theta(n)$.

\subproblem %4-1(c)
Mamy $n^{\log_ba}=n^{\log_416}=n^2$ oraz $f(n)=n^2=\Theta(n^2)$, a~stąd $T(n)=\Theta(n^2\lg n)$.

\subproblem %4-1(d)
Mamy $n^{\log_ba}=n^{\log_37}$ oraz $f(n)=n^2=\Omega(n^{\log_37+\epsilon})$ dla $\epsilon\le2-\log_37$.
Warunek regularności zachodzi:
\begin{align*}
	7f(n/3) &\le cf(n), \\
	7n^2\!/9 &\le cn^2, \\
	c &\ge 7/9,
\end{align*}
a~zatem $T(n)=\Theta(n^2)$.

\subproblem %4-1(e)
Mamy $n^{\log_ba}=n^{\lg7}$ oraz $f(n)=n^2=O(n^{\lg7-\epsilon})$ dla $\epsilon\le\lg7-2$, a~stąd $T(n)=\Theta(n^{\lg7})$.

\subproblem %4-1(f)
Mamy $n^{\log_ba}=n^{\log_42}=n^{1/2}$ oraz $f(n)=\sqrt{n}=\Theta(n^{1/2})$, a~stąd $T(n)=\Theta\bigl(\!\sqrt{n}\lg n\bigr)$.

\subproblem %4-1(g)
Zauważmy, że rekurencja rozwija się następująco:
\begin{align*}
	T(n) &= T(n-1)+n \\
	&= T(n-2)+(n-1)+n \\
	&\hspace{.5in}\vdots \\
	&= c+3+4+\dots+n \\
	&= c+\sum_{i=3}^ni \\
	&= c+\frac{n(n+1)}{2}-3,
\end{align*}
gdzie $c=T(2)$ jest pewną stałą.
Otrzymujemy zatem, że $T(n)=\Theta(n^2)$.

\subproblem %4-1(h)
Niech $n=2^m$, skąd $m=\lg n$.
Rekurencja przyjmuje postać
\[
	T(2^m) = T(2^{m/2})+1.
\]
Podstawiając $S(m)$ za $T(2^m)$, otrzymujemy
\[
	S(m) = S(m/2)+1.
\]
Ponieważ rozwiązaniem ostatniej rekurencji jest $\Theta(\lg m)$ (co pokazano w~\refExercise{4.3-3}), to stąd mamy, że $T(n)=T(2^m)=S(m)=\Theta(\lg m)=\Theta(\lg\lg n)$.

\problem{Szukanie brakującej liczby całkowitej} %4-2
Poprzez badanie najmniej znaczącego bitu liczby całkowitej możemy sprawdzić parzystość tej liczby.
Pobierzemy więc najmniej znaczące bity wszystkich liczb z~tablicy $A$ i~policzymy, ile z~nich jest zerami.
W~zakresie $0\twodots n$ jest $\lfloor n/2\rfloor+1$ liczb parzystych.
Jeśli więc wśród pobranych bitów znajdziemy dokładnie tyle samo zer, to wiemy, że szukana liczba jest nieparzysta, a~jeśli zer będzie $\lfloor n/2\rfloor$, to brakuje liczby parzystej.
W~zależności od przypadku z~dalszej analizy odrzucimy liczby o~parzystości innej niż parzystość brakującej liczby, a~pozostałe będziemy przeszukiwać w~kolejnym wywołaniu rekurencyjnym, badając kolejny najmniej znaczący bit tych liczb.
W~momencie osiągnięcia pustego zbioru będziemy znać wszystkie bity brakującej liczby.

Spośród $n$ liczb, do wywołania rekurencyjnego zostaje ich przekazanych $n-(\lfloor n/2\rfloor+1)=\lfloor(n-1)/2\rfloor$ albo $\lfloor n/2\rfloor$.
Pesymistyczny czas działania opisanego tutaj algorytmu można więc zapisać w~postaci rekurencji $T(n)=T(\lfloor n/2\rfloor)+O(n)$, której rozwiązaniem jest $\Theta(n)$ na podstawie tw.~4.1.

\problem{Koszty przekazywania parametrów} %4-3

\subproblem %4-3(a)
W~pierwszej strategii rekurencja przyjmuje postać z~\refExercise{2.3-5}, której rozwiązaniem jest $T(n)=\Theta(\lg n)$.
Po przyjęciu $n=N$ dostajemy $T(N)=\Theta(\lg N)$.

W~przypadku drugiej strategii na każdym poziomie rekursji należy dodać składnik $\Theta(N)$ odpowiedzialny za przekazywanie tablicy do wywołań rekurencyjnych.
Otrzymujemy zatem
\[
	T(n) = \begin{cases}
		\Theta(1), & \text{jeśli $n\le1$}, \\
		T(\lfloor n/2\rfloor)+\Theta(N), & \text{jeśli $n>1$}.
	\end{cases}
\]
Ponieważ $\Theta(N)$ nie zależy od rozmiaru podproblemu $n$, to stąd rozwiązaniem powyższej rekurencji jest $T(n)=\Theta(N\lg n)$, a~więc $T(N)=\Theta(N\lg N)$.

W~ostatnim przypadku przekazujemy podtablicę o~rozmiarze równym rozmiarowi podproblemu, co prowadzi do rekurencji
\[
	T(n) = \begin{cases}
		\Theta(1), & \text{jeśli $n\le1$}, \\
		T(\lfloor n/2\rfloor)+\Theta(\lfloor n/2\rfloor), & \text{jeśli $n>1$},
	\end{cases}
\]
którą można rozwiązać przy użyciu twierdzenia o~rekurencji uniwersalnej.
Otrzymujemy wynik $T(n)=\Theta(n)$, skąd $T(N)=\Theta(N)$.

\subproblem %4-3(b)
W~przypadku zwykłego przekazywania wskaźnika mamy oryginalną postać rekurencji, której rozwiązaniem dla $n=N$ jest $T(N)=\Theta(N\lg N)$.

Ponieważ w~drugiej strategii należy przekazać całą tablicę do obu wywołań rekurencyjnych, to czas działania wygląda następująco:
\[
	T(n) = \begin{cases}
		\Theta(1), & \text{jeśli $n=1$}, \\
		2T(\lfloor n/2\rfloor)+\Theta(n)+2\Theta(N), & \text{jeśli $n>1$}.
	\end{cases}
\]
Na każdym poziomie dodajemy składnik rzędu $\Theta(N)$, zatem rozwiązaniem tej rekurencji jest iloczyn tegoż składnika i~rozwiązania rekurencji z~pierwszego przypadku, czyli $T(n)=\Theta(Nn\lg n)$, a~stąd mamy $T(N)=\Theta(N^2\lg N)$.

W~ostatniej strategii do każdego wywołania rekurencyjnego przekazujemy podtablicę o~rozmiarze podproblemu, jednak czas na to poświęcany jest identyczny ze składnikiem liniowym odpowiadającym za czas przeznaczany na procedurę \proc{Merge}.
Rekurencja ma zatem identyczną postać jak w~oryginalnej analizie tego algorytmu i~jej rozwiązaniem jest $T(N)=\Theta(N\lg N)$.

\problem{Więcej przykładów rekurencji} %4-4

\subproblem %4-4(a)
Wykorzystując twierdzenie o~rekurencji uniwersalnej, mamy $n^{\log_ba}=n^{\lg3}$ oraz $f(n)=n\lg n=O(n^{\lg3-\epsilon})$, gdzie $\epsilon<\lg3-1$, a~stąd $T(n)=\Theta(n^{\lg3})$.

\subproblem %4-4(b)
Z~twierdzenia o~rekurencji uniwersalnej obliczamy $n^{\log_ba}=n^{\log_55}=n$, jednak dla żadnego $\epsilon>0$ nie jest prawdą, że
\[
	f(n) = \frac{n}{\lg n} = O(n^{1-\epsilon}),
\]
ponieważ dla pewnej stałej $c>0$ i~dostatecznie dużych $n$ musiałoby zachodzić
\[
	\frac{n^\epsilon}{\lg n} \le c,
\]
a~ponieważ $n^\epsilon=\omega(\lg n)$, to niezależnie od wyboru $\epsilon$ dla dużych wartości $n$ licznik tego ułamka jest dowolnie większy od mianownika i~ułamka nie da się z~tego powodu ograniczyć stałą.

Skorzystajmy zatem z~innego sposobu na obliczenie $T(n)$, rozważając rodzinę rekurencji postaci
\[
	T_a(n) = \begin{cases}
		\Theta(1), & \text{jeśli $1\le n<a$}, \\
		aT_a(n/a)+n/\!\lg n, & \text{jeśli $n\ge a$},
	\end{cases}
\]
gdzie $a\ge2$ jest liczbą całkowitą.
Wykorzystamy technikę zamiany zmiennych, podstawiając $m=\log_an$, skąd $n=a^m$, dzięki czemu otrzymujemy
\[
	T_a(a^m) = aT_a(a^{m-1})+\frac{a^m}{m\lg a}.
\]
Możemy teraz podstawić $S_a(m)=T_a(a^m)$, otrzymując nową rekurencję
\[
	S_a(m) = aS_a(m-1)+\frac{a^m}{m\lg a},
\]
którą rozwiązujemy następująco (po przyjęciu $S_a(0)=T_a(1)=d$ dla pewnej stałej $d>0$):
\begin{align*}
	S_a(m) &= \frac{1}{\lg a}\biggl(\frac{a^m}{m}+a\cdot\frac{a^{m-1}}{m-1}+a^2\cdot\frac{a^{m-2}}{m-2}+\dots+a^{m-1}\cdot\frac{a^1}{1}+a^md\biggr) \\[1mm]
	&= \frac{1}{\lg a}\biggl(\sum_{k=1}^m\frac{a^m}{k}+a^md\biggr) \\[1mm]
	&= \frac{a^m(H_m+d)}{\lg a} \\[1mm]
	&= \Theta(a^m\lg m).
\end{align*}
Zamieniając z~powrotem $S_a(m)$ na $T_a(n)$, otrzymujemy rozwiązanie
\[
	T_a(n) = T_a(a^m) = S_a(m) = \Theta(a^m\lg m) = \Theta(n\lg\log_a n) = \Theta(n\lg\lg n).
\]

Stosując znalezione oszacowanie do rekurencji $T(n)\equiv T_5(n)$ z~treści zadania, dostajemy oczywiście $T(n)=\Theta(n\lg\lg n)$.

\subproblem %4-4(c)
Z~twierdzenia o~rekurencji uniwersalnej mamy $n^{\log_ba}=n^{\lg4}=n^2$ oraz $f(n)=n^{5/2}=\Omega(n^{2+\epsilon})$, gdzie $\epsilon\le1/2$.
Badamy warunek regularności:
\begin{align*}
	4f(n/2) &\le cf(n), \\
	\frac{4n^{5/2}}{2^{5/2}} &\le cn^{5/2}, \\
	c &\ge \frac{1}{\sqrt{2}}
\end{align*}
i~stwierdzamy, że $T(n)=\Theta(n^{5/2})$.

\subproblem %4-4(d)
Wykażemy, że stała~5 w~argumencie rekurencji $T(n)$ nie wpływa na postać rozwiązania.
Rozważając rekurencję $T'(n)=3T'(n/3)+n/2$ i~rozwiązując ją za pomocą twierdzenia o~rekurencji uniwersalnej, dostajemy $T'(n)=\Theta(n\lg n)$.
Udowodnimy teraz metodą podstawiania, że identyczny wynik jest rozwiązaniem $T(n)$.

Wykorzystując założenie
\[
	T(n/3+5) \le c_1(n/3+5)\lg(n/3+5)
\]
dla pewnej stałej $c_1>0$, otrzymujemy:
\begin{align*}
	T(n) &\le 3c_1(n/3+5)\lg(n/3+5)+n/2 \\
	&\le c_1(n+15)\lg(2n/5)+n/2 \\
	&< c_1n\lg n+c_1n\lg(2/5)+15c_1\!\lg n+n/2 \\
	&\le c_1n\lg n.
\end{align*}
Powyższe wnioskowanie jest prawdziwe, o~ile $n/3+5\le2n/5$, skąd $n\ge75$, oraz
\[
	c_1n\lg(2/5)+15c_1\!\lg n+n/2 \le 0.
\]
Przeprowadzając podobną analizę jak w~\refExercise{4.1-5}, dostajemy, że dowolne $c_1\ge7$ spełnia ostatnią nierówność dla $n\ge75$.

Analogicznie dla dolnego oszacowania przyjmujemy, że
\[
	T(n/3+5) \ge c_2(n/3+5)\lg(n/3+5),
\]
gdzie $c_2>0$ jest pewną stałą.
Wówczas:
\begin{align*}
	T(n) &\ge 3c_2(n/3+5)\lg(n/3+5)+n/2 \\
	&> 3c_2(n/3)\lg(n/3)+n/2 \\
	&= c_2n\lg n-c_2n\lg3+n/2 \\
	&\ge c_2n\lg n,
\end{align*}
przy czym ostatnia nierówność zachodzi, o~ile
\[
	-c_2n\lg3+n/2 \ge 0.
\]
Warunek ten spełniamy poprzez przyjęcie $c_2\le0{,}3$.

Dowód asymptotycznie dokładnego oszacowania dla $T(n)$ kończymy, wybierając odpowiednie wartości dla podstaw obu indukcji.
Zauważmy, że obliczanie wartości rekurencji ze wzoru $T(n)=3T(n/3+5)+n/2$ dla $n\le7$ nie ma sensu, bo wtedy $T(n)$ nie zależy od niższych wyrazów.
Przyjmijmy zatem, że $T(n)=1$ dla wszystkich $n\le7$ będzie przypadkiem brzegowym rekurencji.
W~dowodzie górnego oszacowania założyliśmy, że $n\ge75$, więc podstawą obu indukcji uczyńmy wszystkie $n=30$, 31,~\dots,~74.
Oszacowania w~tych przypadkach weryfikujemy programem komputerowym, uzasadniając poprawny dobór stałych $c_1$ i~$c_2$.
A~zatem $T(n)=\Theta(n\lg n)$.

\subproblem %4-4(e)
Mamy do czynienia z~rekurencją $T_a(n)$ dla $a=2$, którą rozważaliśmy w~punkcie~(b).
Zgodnie z~przedstawionym tam rozumowaniem wnioskujemy, że $T(n)=\Theta(n\lg\lg n)$.

\subproblem %4-4(f)
W~celu rozwiązania rekurencji wykorzystamy metodę drzewa rekursji do odgadnięcia rozwiązania, które następnie udowodnimy.
Drzewo zostało przedstawione na rys.~\ref{fig:4-4f}.
\begin{figure}[ht]
	\begin{center}
		\includegraphics{fig_4-4.f}
	\end{center}
	\caption{Drzewo rekursji dla równania rekurencyjnego $T(n)=T(n/2)+T(n/4)+T(n/8)+n$.} \label{fig:4-4f}
\end{figure}

Zauważmy, że najszybciej maleją argumenty na skrajnie prawej gałęzi.
Niech $T(1)=1$.
Ponieważ koszt węzła z~tej gałęzi znajdującego się na \singledash{$i$}{tym} poziomie wynosi $n/8^i$, to liść zajmuje poziom $h=\log_8n$.
Łącznym kosztem na \singledash{$i$}{tym} poziomie jest $(7/8)^in$, więc sumując te wartości od korzenia aż do poziomu \singledash{$h$}{tego}, otrzymujemy oszacowanie rekurencji od dołu:
\[
	T(n) \ge \sum_{i=0}^{\log_8n}\biggl(\frac{7}{8}\biggr)^in = \frac{1-(7/8)^{\log_8n+1}}{1-7/8}\,n = 8n(1-(7/8)n^{\log_8(7/8)}) \ge 8n(1-7/8) = \Omega(n).
\]

Zbadajmy teraz górne oszacowanie rekurencji $T(n)$ poprzez dokonanie obserwacji, że najwolniej malejącymi węzłami drzewa są te ze skrajnie lewej gałęzi.
Węzły te wnoszą koszt równy $n/2^i$ na \singledash{$i$}{tym} poziomie, więc liść znajduje się na poziomie $H=\lg n$.
Mamy
\[
	T(n) \le \sum_{i=0}^{\lg n}\biggl(\frac{7}{8}\biggr)^in < \sum_{i=0}^\infty\biggl(\frac{7}{8}\biggr)^in = \frac{n}{1-7/8} = O(n),
\]
co pozwala przypuszczać, że oszacowaniem dokładnym na $T(n)$ jest $\Theta(n)$.

Przeprowadźmy teraz dowód tego wyniku metodą podstawiania.
Załóżmy, że:
\begin{gather*}
	c_1(n/2) \le T(n/2) \le c_2(n/2), \\
	c_1(n/4) \le T(n/4) \le c_2(n/4), \\
	c_1(n/8) \le T(n/8) \le c_2(n/8),
\end{gather*}
gdzie $c_1$, $c_2>0$ to pewne stałe.
Mamy zatem
\[
	T(n) \ge c_1n/2+c_1n/4+c_1n/8+n = 7c_1n/8+n \ge c_1n,
\]
co jest spełnione dla dowolnego $n$, o~ile $c_1\le8$.
Dowód górnego oszacowania przebiega analogicznie, przy czym zakładamy, że $c_2\ge8$.

Jako warunek brzegowy rekurencji przyjmijmy $T(1)=T(2)=\dots=T(7)=1$.
Aby spełnić przypadek bazowy indukcji dla $n=1$, 2,~\dots,~7, musimy przyjąć dodatkowo, że $c_1\le1/7$.
Udowodniliśmy zatem, że dokładnym rozwiązaniem rekurencji jest $T(n)=\Theta(n)$.

\subproblem %4-4(g)
Załóżmy dla uproszczenia rachunków, że $T(1)=1$.
Rozwijając rekurencję, otrzymujemy
\[
	T(n) = \frac{1}{n}+\frac{1}{n-1}+\dots+\frac{1}{2}+\frac{1}{1} = H_n,
\]
a~zatem, korzystając ze wzoru~(A.7), mamy $T(n)=\Theta(\lg n)$.

\subproblem %4-4(h)
Dla ułatwienia przyjmijmy, że $T(1)=0$.
Wówczas
\[
	T(n) = \lg n+\lg(n-1)+\dots+\lg2+\lg1 = \lg\biggl(\prod_{i=1}^ni\biggr) = \lg(n!)
\]
i~ze wzoru~(3.18) dostajemy $T(n)=\Theta(n\lg n)$.

\subproblem %4-4(i)
Przyjmijmy, że $T(2)=2$ i~rozważmy przypadek, gdy $n$ jest parzyste.
Rozwijamy rekurencję, otrzymując
\[
	T(n) = 2\lg n+2\lg(n-2)+\dots+2\lg4+2\lg2 = 2\lg\biggl(\prod_{i=1}^{n/2}2i\biggr) = 2\bigl(\lg((n/2)!)+n/2\bigr).
\]
Wykorzystując wzór~(3.18), dostajemy
\[
	T(n) = 2\cdot\Theta((n/2)\lg (n/2))+n = \Theta(n\lg n).
\]

Dla $n$ nieparzystego, po przyjęciu $T(1)=0$ sprowadzamy rekurencję do sumy
\[
	T(n) = 2\lg n+2\lg(n-2)+\dots+2\lg3+2\lg1 = 2\lg\biggl(\prod_{i=1}^{(n+1)/2}(2i-1)\biggr),
\]
którą można ograniczyć z~góry przez $2\lg\Bigl(\prod_{i=1}^{(n+1)/2}2i\Bigr)$, a~z~dołu przez $2\lg\Bigl(\prod_{i=1}^{(n-1)/2}2i\Bigr)$.
Po pewnych przekształceniach i~po skorzystaniu ze wzoru~(3.18) oba te wyrażenia sprowadzamy do postaci $\Theta(n\lg n)$.
Na podstawie tego faktu i~uzasadnienia z~poprzedniego paragrafu wnioskujemy, że $T(n)$ jest klasy $\Theta(n\lg n)$.

\subproblem %4-4(j)
Po podzieleniu rekurencji $T(n)$ przez $n$ dostajemy
\[
	\frac{T(n)}{n} = \frac{T(\!\sqrt{n})}{\sqrt{n}}+1.
\]
Podstawmy teraz $S(n)=T(n)/n$, otrzymując nową rekurencję
\[
	S(n) = S(\!\sqrt{n})+1.
\]
Potraktujmy $n$ jako $2^m$, skąd $m=\lg n$ i~podstawmy $R(m)=S(2^m)$:
\[
	R(m) = R(m/2)+1.
\]
Rozwiązanie ostatniej rekurencji zostało wyznaczone w~\refExercise{4.3-3} i~wynosi $R(m)=\Theta(\lg m)$.
Powracamy do oryginalnej rekurencji $T(n)$, dostając ostatecznie
\[
	T(n) = nS(n) = nR(\lg n) = n\cdot\Theta(\lg\lg n) = \Theta(n\lg\lg n).
\]

\problem{Liczby Fibonacciego} %4-5

\subproblem %4-5(a)
Wprost z~definicji $\mathcal{F}(z)$ mamy:
\begin{align*}
	\mathcal{F}(z) &= \sum_{i=0}^\infty F_iz^i \\
	&= F_0+zF_1+\sum_{i=2}^\infty (F_{i-1}+F_{i-2})z^i \\
	&= z+\sum_{i=2}^\infty F_{i-1}z^i+\sum_{i=2}^\infty F_{i-2}z^i \\
	&= z+\sum_{i=1}^\infty F_iz^{i+1}+\sum_{i=0}^\infty F_iz^{i+2} \\[2mm]
	&= z+z\mathcal{F}(z)+z^2\mathcal{F}(z),
\end{align*}
a~zatem tożsamość zachodzi.

\subproblem %4-5(b)
Ze wzoru z~poprzedniego punktu wynika pierwsza równość:
\begin{align*}
	\mathcal{F}(z) &= z+z\mathcal{F}(z)+z^2\mathcal{F}(z), \\
	(1-z-z^2)\mathcal{F}(z) &= z, \\
	\mathcal{F}(z) &= \frac{z}{1-z-z^2}.
\end{align*}
Mianownik prawej strony ostatniego wyrażenia jest trójmianem kwadratowym o~miejscach zerowych $\phi$ i~$\widehat\phi$, który zapisujemy w~równoważnej postaci
\[
	1-z-z^2 = -(z+\phi)\bigl(z+\widehat\phi\bigr) = (1-\phi z)\bigl(1-\widehat\phi z\bigr),
\]
co można uzasadnić wzorem $\phi\cdot\widehat\phi=-1$ i~tym samym dowieść drugiej równości.
Ostatnią z~nich otrzymujemy, zauważając, że
\[
	\frac{1}{1-\phi z}-\frac{1}{1-\widehat\phi z} = \frac{1-\widehat\phi z-1+\phi z}{(1-\phi z)\bigl(1-\widehat\phi z\bigr)} = \frac{z\bigl(\phi-\widehat\phi\bigr)}{(1-\phi z)\bigl(1-\widehat\phi z\bigr)} = \frac{z\sqrt{5}}{(1-\phi z)\bigl(1-\widehat\phi z\bigr)},
\]
a~stąd mamy
\[
	\frac{z}{(1-\phi z)\bigl(1-\widehat\phi z\bigr)} = \frac{1}{\sqrt{5}}\cdot\frac{z\sqrt{5}}{(1-\phi z)\bigl(1-\widehat\phi z\bigr)} = \frac{1}{\sqrt{5}}\biggl(\frac{1}{1-\phi z}-\frac{1}{1-\widehat\phi z}\biggr).
\]

\subproblem %4-5(c)
Tezę otrzymujemy natychmiast, jeśli w~definicji $\mathcal{F}(z)$ podstawimy $F_i=\bigl(\phi^i-\widehat\phi^i\bigr)/\sqrt{5}$, co wykazano w~\refExercise{3.2-6}.

\subproblem %4-5(d)
Ponieważ $\bigl|\widehat\phi\bigr|<1$, to prawdą jest, że $\bigl|\widehat\phi^i\bigr|<1$ dla $i>0$ oraz $\bigl|\widehat\phi^i\bigr|/\sqrt{5}<1/\sqrt{5}<1/2$.
Mamy
\[
	F_i = \frac{\phi^i-\widehat\phi^i}{\sqrt{5}} = \frac{\phi^i}{\sqrt{5}}-\frac{\widehat\phi^i}{\sqrt{5}},
\]
skąd
\[
	\frac{\phi^i}{\sqrt{5}}-\frac{1}{2} < F_i < \frac{\phi^i}{\sqrt{5}}+\frac{1}{2},
\]
a~zatem $F_i$ jest liczbą całkowitą najbliższą wartości $\phi^i/\sqrt{5}$.

\subproblem %4-5(e)
Nierówność została udowodniona w~\refExercise{3.2-7}.

\problem{Testowanie układów VLSI} %4-6

\subproblem %4-6(a)
Niech $D$ i~$Z$ będą zbiorami, odpowiednio, układów dobrych i~układów złych.
Podczas testowania każdy zły układ może twierdzić, że każdy inny układ ze zbioru $Z$ jest dobry, a~każdy układ ze zbioru $D$ jest zły.
Z~kolei dobry układ może orzekać o~każdym układzie z~$Z$, że jest zły, natomiast o~każdym innym z~$D$, że jest dobry.
Inaczej ujmując, zbiór $Z$ będzie wskazywał, że sam jest zbiorem układów dobrych, a~zbiór $D$ złych i~symetrycznie dla zbioru $D$.
W~ogólności $Z$ może składać się z~podzbiorów układów, które będą twierdzić, że wyłącznie one są dobre.
Nie da się natomiast rozdzielić w~taki sposób zbioru $D$, ponieważ jego układy zawsze orzekają prawdziwie.
Aby zatem jednoznacznie wskazać zbiór dobrych układów, wymagane jest założenie, że $|D|>|Z|$.

\subproblem %4-6(b)
Potraktujmy wynik każdego testu dwóch układów jako parę $\langle a,b\rangle\in\{{\scriptstyle\rm D},{\scriptstyle\rm Z}\}^2$.
Pierwszy element pary jest stwierdzeniem pierwszego układu o~drugim, a~drugi element -- drugiego o~pierwszym, przy czym ${\scriptstyle\rm D}$ oznacza pozytywny wynik testu, a~${\scriptstyle\rm Z}$ -- negatywny.
Możliwe jest uzyskanie jednego z~czterech wyników:
\begin{itemize}
	\item $\langle{\scriptstyle\rm D},{\scriptstyle\rm D}\rangle$ -- oba układy są dobre albo oba są złe;
	\item $\langle{\scriptstyle\rm D},{\scriptstyle\rm Z}\rangle$ -- tylko pierwszy z~układów jest zły;
	\item $\langle{\scriptstyle\rm Z},{\scriptstyle\rm D}\rangle$ -- tylko drugi z~układów jest zły;
	\item $\langle{\scriptstyle\rm Z},{\scriptstyle\rm Z}\rangle$ -- co najmniej jeden z~układów jest zły.
\end{itemize}

Podzielmy zbiór układów na pary i~przetestujmy wzajemnie układy w~każdej takiej parze, wykonując przy tym $\lfloor n/2\rfloor$ testów.
Zauważmy, że otrzymawszy dla pewnej pary wynik inny niż $\langle{\scriptstyle\rm D},{\scriptstyle\rm D}\rangle$, możemy ją odrzucić, gdyż co najmniej jeden układ z~tej pary jest zły i~wśród nieodrzuconych układów będzie nadal więcej dobrych niż złych.
Dostaniemy w~rezultacie od jednej do $\lfloor n/2\rfloor$ par o~tej własności, że w~każdej z~nich oba układy są dobre albo oba są złe.
Odrzucając zatem po jednym układzie z~każdej pozostawionej pary, nadal zachowujemy własność o~większej liczbie dobrych układów w~pozostawionym zbiorze układów, którego rozmiar wynosi co najwyżej $\lceil n/2\rceil$.

Powyżej opisany proces przeprowadzamy rekurencyjnie, dostając w~końcu zbiór jednoelementowy, który na mocy założenia zawiera dobry układ.

\subproblem %4-6(c)
Wykorzystując wynik poprzedniego punktu, dostajemy następującą rekurencję opisującą liczbę testów koniecznych do znalezienia jednego dobrego układu w~pesymistycznym przypadku (tzn.\ kiedy każda para układów zwraca wynik $\langle{\scriptstyle\rm D},{\scriptstyle\rm D}\rangle$):
\[
	T(n) =
	\begin{cases}
		\Theta(1), & \text{jeśli $n=1$}, \\
		T(\lceil n/2\rceil)+\lfloor n/2\rfloor, & \text{jeśli $n>1$}.
	\end{cases}
\]
Stosując twierdzenie o~rekurencji uniwersalnej, dostajemy rozwiązanie: $T(n)=\Theta(n)$.
Wynik ten jest prawdziwy także w~przypadku optymistycznym -- wówczas dobry układ znajdujemy w~jednym zejściu rekurencyjnym po uprzednim wykonaniu $\Theta(n)$ testów.

Potrafimy wyznaczyć dobry układ $u$, wykorzystajmy go więc do znalezienia kolejnych.
Testujemy tenże układ z~pozostałymi $n-1$.
Wynikiem testu $u$ z~pewnym innym układem $v$ nie może być $\langle{\scriptstyle\rm D},{\scriptstyle\rm Z}\rangle$, a~więc możliwe są trzy sytuacje.
Jeśli otrzymamy $\langle{\scriptstyle\rm D},{\scriptstyle\rm D}\rangle$, to oznacza to, że oba układy są tak samo dobre, a~więc $v$ również jest dobry.
Uzyskując wynik $\langle{\scriptstyle\rm Z},{\scriptstyle\rm D}\rangle$, mamy natychmiast, że $v$ jest zły, podobnie w~wypadku, gdy wynikiem testu będzie $\langle{\scriptstyle\rm Z},{\scriptstyle\rm Z}\rangle$ -- wtedy co najmniej jeden z~testowanych układów jest zły, ale nie może nim być $u$.
Wynika stąd, że po wykonaniu $n-1$ takich testów wyznaczymy pozostałe dobre układy.
A~zatem łączna liczba testów wymaganych do zidentyfikowania wszystkich dobrych układów wynosi $\Theta(n)$.

\problem{Tablice Monge'a} %4-7

\subproblem %4-7(a)
Elementy tablicy Monge'a $A$ spełniają nierówność
\[
	A[i,j]+A[k,l] \le A[i,l]+A[k,j],
\]
gdzie $1\le i<k\le m$ oraz $1\le j<l\le n$.
W~szczególności więc może być $k=i+1$ oraz $l=j+1$, zatem implikacja w~prawo zachodzi.

Implikację w~przeciwną stronę dowodzimy przez indukcję względem liczby wierszy $m$.
Zauważmy, że warunek tablicy Monge'a nie ma większego sensu dla tablic o~jednej kolumnie lub jednym wierszu, zatem przyjmijmy $m=2$ za podstawę indukcji.
Wówczas może być tylko $i=1$ oraz $k=2$.
Na podstawie założenia spełnione są zatem wszystkie nierówności z~następującego układu:
\[
	\begin{cases}
		\hfill A[1,1]+A[2,2] &\le\,\,\, A[1,2]+A[2,1] \\
		\hfill A[1,2]+A[2,3] &\le\,\,\, A[1,3]+A[2,2] \\
		& \;\vdots \\
		A[1,n-1]+A[2,n] &\le\,\,\, A[1,n]+A[2,n-1] \\
	\end{cases}.
\]
Dodając stronami nierówności z~tego układu od \singledash{$j$}{tej} do \singledash{$l$}{tej} włącznie, a~następnie redukując powtarzające się składniki, otrzymujemy nierówność $A[1,j]+A[2,l]\le A[1,l]+A[2,j]$ z~tezy twierdzenia.

Niech teraz $m>2$.
Przyjmujemy założenie indukcyjne, że tablica $A$ pozbawiona ostatniego wiersza stanowi tablicę Monge'a.
Pozostaje zatem wykazać, że zachodzą nierówności z~definicji tablicy Monge'a, przy czym $k=m$.
Wykorzystując pomysł z~pierwszego kroku indukcji, z~układu
\[
	\begin{cases}
		\hfill A[m-1,1]+A[m,2] &\le\,\,\, A[m-1,2]+A[m,1] \\
		\hfill A[m-1,2]+A[m,3] &\le\,\,\, A[m-1,3]+A[m,2] \\
		& \;\vdots \\
		A[m-1,n-1]+A[m,n] &\le\,\,\, A[m-1,n]+A[m,n-1] \\
	\end{cases}
\]
możemy uzyskać wszystkie nierówności postaci $A[m-1,j]+A[m,l]\le A[m-1,l]+A[m,j]$, gdzie $1\le j<l\le n$.
Jeśli teraz dodamy stronami każdą z~nich do każdej nierówności postaci $A[i,j]+A[m-1,l]\le A[i,l]+A[m-1,j]$, gdzie $1\le j<l\le n$ oraz $1\le i<m-1$, zachodzących na mocy założenia indukcyjnego, to po zredukowaniu zbędnych składników dostaniemy wszystkie nierówności z~tezy, w~których $k=m$.

Widać zatem, że implikacja jest prawdziwa dla tablic o~ustalonej liczbie kolumn.
Dowód dla zmiennej liczby kolumn przeprowadza się analogicznie przez indukcję po $n\ge2$, pokazując tym samym, że twierdzenie zachodzi dla tablic o~dowolnych wymiarach.

\subproblem %4-7(b)
Na podstawie poprzedniego punktu można sprawdzić, że nierówność
\[
	A[1,2]+A[2,3] \le A[1,3]+A[2,2]
\]
jest fałszywa, przez co własność tablicy Monge'a jest zaburzona.
By przywrócić tę własność, można zamienić $A[1,3]$ np.\ na~24.

\subproblem %4-7(c)
Korzystając z~poniższej nierówności prawdziwej dla tablicy Monge'a $A$:
\[
	A[i,j]+A[i+1,j+r] \le A[i,j+r]+A[i+1,j],
\]
gdzie $0<r\le n-j$, wnioskujemy w~następujący sposób.
Znajdujemy w~pierwszym wierszu tablicy $A$ pierwsze minimum z~lewej strony, które oznaczymy przez $\mu$.
Indeksem $\mu$ jest oczywiście $f(1)$.
Wstawiając teraz $i=1$ oraz $r=f(1)-j$ do powyższej nierówności, otrzymujemy
\[
	A[1,j]+A[2,f(1)] \le \mu+A[2,j].
\]
Z~drugiej strony $\mu<A[1,j]$ dla każdego $1\le j<f(1)$.
Łącząc oba fakty, otrzymujemy, że $A[2,f(1)]<A[2,j]$, a~to oznacza, że pierwsze z~lewej strony minimum wiersza~2 występuje w~nim na pozycji nie mniejszej niż $f(1)$, skąd $f(1)\le f(2)$.

Dowód kolejnych nierówności przebiega analogicznie.

\subproblem %4-7(d)
Aby odnaleźć minimum wiersza \singledash{$i$}{tego} (nieparzystego), sprawdzamy indeksy minimów wierszy \singledash{$(i-1)$}{szego} oraz \singledash{$(i+1)$}{szego} (parzystych).
Wartości te zostały wyznaczone w~pierwszej części algorytmu.
Na podstawie poprzedniego punktu mamy, że $f(i-1)\le f(i)\le f(i+1)$, wystarczy więc sprawdzić komórki $A[i,f(i-1)]$, $A[i,f(i-1)+1]$,~\dots,~$A[i,f(i+1)]$ i~wybrać spośród nich element minimalny.
Oczywiście nie istnieje wiersz zerowy, dlatego przetwarzając pierwszy wiersz, nie szukamy minimum poprzedniego, ale przyjmujemy dla uproszczenia procedury, że $f(0)=1$.
Podobny przypadek może się zdarzyć dla ostatniego nieparzystego wiersza, jeśli jest on ostatnim wierszem tablicy -- wtedy wystarczy przyjąć $f(m+1)=n$.

Poszukując minimum wiersza \singledash{$i$}{tego}, sprawdzamy $f(i+1)-f(i-1)+1$ komórek w~tym wierszu.
Podczas działania procedury w~pesymistycznym przypadku zostanie sprawdzonych
\begin{align*}
	\sum_{\substack{1\le i\le m\\2\,\nmid\,i}}(f(i+1)-f(i-1)+1) &= \sum_{k=0}^{\lceil m/2\rceil-1}(f(2k+2)-f(2k)+1) \\[-4mm]
	&= \lceil m/2\rceil+\sum_{k=0}^{\lceil m/2\rceil-1}(f(2k+2)-f(2k)) \\[1mm]
	&= \lceil m/2\rceil+f(2\lceil m/2\rceil)-f(0) \\[2mm]
	&\le \lceil m/2\rceil+n-1
\end{align*}
komórek, co jest rzędu $O(m+n)$.

\subproblem %4-7(e)
Na ostatnim poziomie rekursji wyznaczenie minimum jednego wiersza tablicy wymaga sprawdzenia co najwyżej $O(n)$ komórek.
Na podstawie tego faktu i~oszacowania pokazanego w~poprzednim punkcie formułujemy następującą rekurencję opisującą pesymistyczny czas działania algorytmu:
\[
	T(m,n) =
	\begin{cases}
		O(n), & \text{jeśli $m=1$}, \\
		T(\lceil m/2\rceil,n)+O(m+n), & \text{jeśli $m>1$}.
	\end{cases}
\]
Dla uproszczenia pominiemy sufit w~argumencie rekurencji.
Na \singledash{$i$}{tym} poziomie rekurencja ta wprowadza koszt równy $O(m/2^i+n)$.
Łatwo zauważyć, że jest $\lg m+1$ poziomów, a~zatem całkowity koszt wynosi:
\begin{align*}
	T(m,n) &= \sum_{i=0}^{\lg m-1}O\biggl(\frac{m}{2^i}+n\biggr)+O(n) \\
	&= O\biggl(m\sum_{i=0}^{\lg m-1}\frac{1}{2^i}\biggr)+O(n\lg m) \\
	&\le O\biggl(m\sum_{i=0}^\infty\frac{1}{2^i}\biggr)+O(n\lg m) \\[1mm]
	&= O(m+n\lg m).
\end{align*}

\endinput
