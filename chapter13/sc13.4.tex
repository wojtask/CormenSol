\subchapter{Operacja usuwania}

\exercise %13.4-1
\exercise %13.4-2
\exercise %13.4-3
Na rys.\ \ref{fig:13.4-3} zilustrowano ciąg operacji usuwania węzłów z~drzewa czerwono-czarnego.
\begin{figure}[!ht]
	\centering \begin{tikzpicture}[
	level/.append style = {level distance=7mm, sibling distance=50mm/2^#1},
	every label/.style = {index node, rectangle, label distance=1mm, label position=below, draw=none, fill=none},
	outer/.append style = {node distance=15mm}
]

\node[outer] (pic a) {
\begin{tikzpicture}[
	every node/.style = {tree node},
	anchor = center
]
        \node[blackened] {38}
            child {node {19}
                child {node[blackened] {12}
                    child {node {8}}
                    child[missing]
                }
                child {node[blackened] {31}}
            }
            child {node[blackened] {41}};
\end{tikzpicture}
};

\node[outer, right=of pic a] (pic b) {
\begin{tikzpicture}[
	every node/.style = {tree node},
	anchor = center
]
        \node[blackened] {38}
            child {node {19}
                child {node[blackened] {12}}
                child {node[blackened] {31}}
            }
            child {node[blackened] {41}};
\end{tikzpicture}
};

\node[outer, right=of pic b] (pic c) {
\begin{tikzpicture}[
	every node/.style = {tree node},
	anchor = center
]
	\node[blackened] {38}
            child {node[blackened] {19}
                child[missing]
                child {node {31}}
            }
            child {node[blackened] {41}};
\end{tikzpicture}
};

\node[outer, below=of pic a] (pic d) {
\begin{tikzpicture}[
	every node/.style = {tree node},
	anchor = center
]
	\node[blackened] {38}
            child {node[blackened] {31}}
            child {node[blackened] {41}};
\end{tikzpicture}
};

\node[outer, right=20mm of pic d] (pic e) {
\begin{tikzpicture}[
	every node/.style = {tree node},
	anchor = center
]
	\node[blackened] {38}
            child[missing]
            child {node {41}};
\end{tikzpicture}
};

\node[outer, right=20mm of pic e] (pic f) {
\begin{tikzpicture}[
	every node/.style = {tree node},
	anchor = center
]
        \node[blackened, label=\phantom{\attrib{T}{nil}}] {41};
\end{tikzpicture}
};

\node[outer, right=20mm of pic f] (pic g) {
\begin{tikzpicture}[
	every node/.style = {tree node},
	anchor = center
]
        \node[blackened, label=\attrib{T}{nil}] {};
\end{tikzpicture}
};

\node[subpicture label, below=1mm of pic a] (label a) {(a)};
\foreach \x in {b,c} {
	\node[subpicture label] at (label a -| pic \x) {(\x)};
}
\node[subpicture label, below=3mm of pic d] (label d) {(d)};
\foreach \x in {e,f,g} {
	\node[subpicture label] at (label d -| pic \x) {(\x)};
}

\end{tikzpicture}

	\caption{Drzewa czerwono-czarne powstałe po usunięciu elementów 8, 12, 19, 31, 38, 41 kolejno z~drzewa czerwono-czarnego $T$ skonstruowanego w~\refExercise{13.3-2}.
	{\sffamily\bfseries(a)} Drzewo $T$ przed rozpoczęciem usuwania.
	{\sffamily\bfseries(b)} Wycięcie węzła o~kluczu 8 nie powoduje naruszenia żadnej własności drzewa czerwono-czarnego.
	{\sffamily\bfseries\doubledash{(c)}{(e)}} Po wycięciu każdego kolejnego elementu w~drzewie naruszona zostaje wyłącznie własność 5.
	Przywracana jest ona następnie w~procedurze \proc{RB-Delete-Fixup}.
	{\sffamily\bfseries{(f)}} Po pozbawieniu drzewa przedostatniego węzła pozostaje w~nim jedynie czerwony korzeń, co stanowi naruszenie własności 2.
        W~linii 23 procedury \proc{RB-Delete-Fixup} jest ona jednak natychmiast przywracana.
        {\sffamily\bfseries{(g)}} Usunięcie ostatniego węzła pozostawia puste drzewo czerwono-czarne, czyli składające się tylko z~wartownika \attrib{T}{nil} (którego pominęliśmy na pozostałych częściach rysunku dla zwiększenia czytelności).
        Wywołanie procedury \proc{RB-Delete-Fixup} na wartowniku nie powoduje żadych zmian.} \label{fig:13.4-3}
\end{figure}

\exercise %13.4-4
\exercise %13.4-5
\exercise %13.4-6
\exercise %13.4-7
