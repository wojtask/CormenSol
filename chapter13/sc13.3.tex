\subchapter{Operacja wstawiania}

\exercise %13.3-1
Pokolorowanie węzła $z$ na czarno mogłoby spowodować naruszenie własności 5 drzewa czerwono-czarnego, podczas gdy procedura \proc{RB-Insert-Fixup} operuje na drzewie, w~którym własność ta jest zachowana i~nie narusza jej w~trakcie swojego działania.

\exercise %13.3-2
Na rys.\ \ref{13.3-2} zilustrowano ciąg operacji wstawiania węzłów do drzewa czerwono-czarnego.
\begin{figure}[ht]
	\centering \begin{tikzpicture}[
	level/.append style = {level distance=7mm, sibling distance=50mm/2^#1},
	outer/.append style = {node distance=20mm and 15mm}
]

\node[outer] (pic a) {
\begin{tikzpicture}[
	every node/.style = {tree node},
	anchor = center
]
	\node[blackened] {41};
\end{tikzpicture}
};

\node[outer, right=of pic a] (pic b) {
\begin{tikzpicture}[
	every node/.style = {tree node},
	anchor = center
]
	\node[blackened] {41}
            child {node {38}}
            child[missing];
\end{tikzpicture}
};

\node[outer, right=of pic b] (pic c) {
\begin{tikzpicture}[
	every node/.style = {tree node},
	anchor = center
]
	\node[blackened] {38}
            child {node {31}}
            child {node {41}};
\end{tikzpicture}
};

\node[outer, right=of pic c] (pic d) {
\begin{tikzpicture}[
	every node/.style = {tree node},
	anchor = center
]
	\node[blackened] {38}
            child {node[blackened] {31}
                child {node {12}}
                child[missing]
            }
            child {node[blackened] {41}};
\end{tikzpicture}
};

\node[outer, below=of pic b] (pic e) {
\begin{tikzpicture}[
	every node/.style = {tree node},
	anchor = center
]
	\node[blackened] {38}
            child {node[blackened] {19}
                child {node {12}}
                child {node {31}}
            }
            child {node[blackened] {41}};
\end{tikzpicture}
};

\node[outer, right=35mm of pic e] (pic f) {
\begin{tikzpicture}[
	every node/.style = {tree node},
	anchor = center
]
	\node[blackened] {38}
            child {node {19}
                child {node[blackened] {12}
                    child {node {8}}
                    child[missing]
                }
                child {node[blackened] {31}}
            }
            child {node[blackened] {41}};
\end{tikzpicture}
};

\foreach \x in {a, ..., f} {
	\node[subpicture label, below=2mm of pic \x] {(\x)};
}

\end{tikzpicture}

	\caption{Drzewa czerwono-czarne powstałe po wstawieniu węzłów o~kluczach 41, 38, 31, 12, 19, 8 kolejno do początkowo pustego drzewa.
	{\sffamily\bfseries(a)} Po wstawieniu pierwszego węzła drzewo składa się tylko z~korzenia, który początkowo ma kolor czerwony.
	Przywrócenie własności 2 drzewa czerwono-czarnego następuje w~linii 16 procedury \proc{RB-Insert-Fixup}.
	{\sffamily\bfseries(b)} Dodanie węzła o~kluczu 38 nie powoduje naruszenia żadnej własności drzewa czerwono-czarnego.
	{\sffamily\bfseries\doubledash{(c)}{(f)}} Wstawienie każdego kolejnego węzła produkuje drzewo, w~którym naruszona jest własność 4, przywracana następnie za pomocą procedury \proc{RB-Insert-Fixup}.} \label{fig:13.3-2}
\end{figure}

\exercise %13.3-3
\exercise %13.3-4
\exercise %13.3-5
\exercise %13.3-6
