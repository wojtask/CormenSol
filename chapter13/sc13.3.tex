\subchapter{Operacja wstawiania}

\exercise %13.3-1
Pokolorowanie węzła $z$ na czarno mogłoby spowodować naruszenie własności 5 drzewa czerwono-czarnego, podczas gdy procedura \proc{RB-Insert-Fixup} operuje na drzewie, w~którym własność ta jest zachowana i~nie narusza jej w~trakcie swojego działania.

\exercise %13.3-2
Na rys.\ \ref{fig:13.3-2} zilustrowano ciąg operacji wstawiania węzłów do drzewa czerwono-czarnego.
\begin{figure}[ht]
	\centering \begin{tikzpicture}[
	level/.append style = {level distance=7mm, sibling distance=50mm/2^#1},
	outer/.append style = {node distance=20mm and 15mm}
]

\node[outer] (pic a) {
\begin{tikzpicture}[
	every node/.style = {tree node},
	anchor = center
]
	\node[blackened] {41};
\end{tikzpicture}
};

\node[outer, right=of pic a] (pic b) {
\begin{tikzpicture}[
	every node/.style = {tree node},
	anchor = center
]
	\node[blackened] {41}
            child {node {38}}
            child[missing];
\end{tikzpicture}
};

\node[outer, right=of pic b] (pic c) {
\begin{tikzpicture}[
	every node/.style = {tree node},
	anchor = center
]
	\node[blackened] {38}
            child {node {31}}
            child {node {41}};
\end{tikzpicture}
};

\node[outer, right=of pic c] (pic d) {
\begin{tikzpicture}[
	every node/.style = {tree node},
	anchor = center
]
	\node[blackened] {38}
            child {node[blackened] {31}
                child {node {12}}
                child[missing]
            }
            child {node[blackened] {41}};
\end{tikzpicture}
};

\node[outer, below=of pic b] (pic e) {
\begin{tikzpicture}[
	every node/.style = {tree node},
	anchor = center
]
	\node[blackened] {38}
            child {node[blackened] {19}
                child {node {12}}
                child {node {31}}
            }
            child {node[blackened] {41}};
\end{tikzpicture}
};

\node[outer, right=35mm of pic e] (pic f) {
\begin{tikzpicture}[
	every node/.style = {tree node},
	anchor = center
]
	\node[blackened] {38}
            child {node {19}
                child {node[blackened] {12}
                    child {node {8}}
                    child[missing]
                }
                child {node[blackened] {31}}
            }
            child {node[blackened] {41}};
\end{tikzpicture}
};

\foreach \x in {a, ..., f} {
	\node[subpicture label, below=2mm of pic \x] {(\x)};
}

\end{tikzpicture}

	\caption{Drzewa czerwono-czarne powstałe po wstawieniu węzłów o~kluczach 41, 38, 31, 12, 19, 8 kolejno do początkowo pustego drzewa.
	{\sffamily\bfseries(a)} Po wstawieniu pierwszego węzła drzewo składa się tylko z~korzenia, który początkowo ma kolor czerwony.
	Przywrócenie własności 2 drzewa czerwono-czarnego następuje w~linii 16 procedury \proc{RB-Insert-Fixup}.
	{\sffamily\bfseries(b)} Dodanie węzła o~kluczu 38 nie powoduje naruszenia żadnej własności drzewa czerwono-czarnego.
	{\sffamily\bfseries\doubledash{(c)}{(f)}} Wstawienie każdego kolejnego węzła produkuje drzewo, w~którym naruszona jest własność 4, przywracana następnie za pomocą procedury \proc{RB-Insert-Fixup}.} \label{fig:13.3-2}
\end{figure}

\exercise %13.3-3
Wersja rysunków 13.5 i~13.6 z~Podręcznika z~podanymi czarnymi wysokościami węzłów została zaprezentowana na rys.\ \ref{fig:13.3-3}.
Wynika z~niego, że własność 5 drzewa czerwono-czarnego faktycznie pozostaje zachowana podczas działania procedury \proc{RB-Insert-Fixup}.
\begin{figure}[ht]
	\centering \begin{tikzpicture}[
	level/.append style = {level distance=7mm, sibling distance=100mm/2^#1},
	every label/.style = {index node, label position=left, draw=none, fill=none},
	subtree node/.style = {tree node, draw=none, fill=none},
	outer/.append style = {node distance=0mm}
]

\node[outer] (pic a) {
\begin{tikzpicture}[
	every node/.style = {tree node},
	anchor = center
]
	\node[draw=none, fill=none] (left tree) {}
            child {node[blackened, label=$k+1$] (left subtree root) {$\boldsymbol{C}$}
                child {node[label=$k$] {$A$}
                    child {node[subtree node] {$\alpha$}}
                    child {node[label=$k$] {$B$}
                        child {node[subtree node] {$\beta$}}
                        child {node[subtree node] {$\gamma$}}
                    }
                }
                child {node[label=$k$] {$D$}
                    child {node[subtree node] {$\delta$}}
                    child {node[subtree node] {$\varepsilon$}}
                }
            };
            
        \node[right=50mm of left tree, draw=none, fill=none] {}
            child {node[label=$k+1$] (right subtree root) {$C$}
                child {node[blackened, label=$k+1$] {$\boldsymbol{A}$}
                    child {node[subtree node] {$\alpha$}}
                    child {node[label=$k$] {$B$}
                        child {node[subtree node] {$\beta$}}
                        child {node[subtree node] {$\gamma$}}
                    }
                }
                child {node[blackened, label=$k+1$] {$\boldsymbol{D}$}
                    child {node[subtree node] {$\delta$}}
                    child {node[subtree node] {$\varepsilon$}}
                }
            };
            
        \draw[arrow, shorten >=10mm, shorten <=10mm, dashed, thick] (left subtree root.east) -- (right subtree root.west);
        
        \node[subpicture label, draw=none, fill=none, left=25mm of left subtree root] {(a)};
\end{tikzpicture}
};

\node[outer, below=of pic a] (pic b) {
\begin{tikzpicture}[
	every node/.style = {tree node},
	anchor = center
]
	\node[draw=none, fill=none] (left tree) {}
            child {node[blackened, label=$k+1$] (left subtree root) {$\boldsymbol{C}$}
                child {node[label=$k$] {$B$}
                    child {node[label=$k$] {$A$}
                        child {node[subtree node] {$\alpha$}}
                        child {node[subtree node] {$\beta$}}
                    }
                    child {node[subtree node] {$\gamma$}}
                }
                child {node[label=$k$] {$D$}
                    child {node[subtree node] {$\delta$}}
                    child {node[subtree node] {$\varepsilon$}}
                }
            };
            
        \node[right=50mm of left tree, draw=none, fill=none] {}
            child {node[label=$k+1$] (right subtree root) {$C$}
                child {node[blackened, label=$k+1$] {$\boldsymbol{B}$}
                    child {node[label=$k$] {$A$}
                        child {node[subtree node] {$\alpha$}}
                        child {node[subtree node] {$\beta$}}
                    }
                    child {node[subtree node] {$\gamma$}}
                }
                child {node[blackened, label=$k+1$] {$\boldsymbol{D}$}
                    child {node[subtree node] {$\delta$}}
                    child {node[subtree node] {$\varepsilon$}}
                }
            };
            
        \draw[arrow, shorten >=10mm, shorten <=10mm, dashed, thick] (left subtree root.east) -- (right subtree root.west);
        
        \node[subpicture label, draw=none, fill=none, left=25mm of left subtree root] {(b)};
\end{tikzpicture}
};

\node[outer, below=of pic b] (pic c) {
\begin{tikzpicture}[
	every node/.style = {tree node},
	anchor = center
]
	\node[draw=none, fill=none] (left tree) {}
            child {node[blackened, label=$k+1$] (left subtree root) {$\boldsymbol{C}$}
                child {node[label=$k$] {$A$}
                    child {node[subtree node] {$\alpha$}}
                    child {node[label=$k$] {$B$}
                        child {node[subtree node] {$\beta$}}
                        child {node[subtree node] {$\gamma$}}
                    }
                }
                child {node[subtree node] {$\delta$}}
            };
            
        \node[right=40mm of left tree, draw=none, fill=none] (middle tree) {}
            child {node[blackened, label=$k+1$] (middle subtree root) {$\boldsymbol{C}$}
                child {node[label=$k$] {$B$}
                    child {node[label=$k$] {$A$}
                        child {node[subtree node] {$\alpha$}}
                        child {node[subtree node] {$\beta$}}
                    }
                    child {node[subtree node] {$\gamma$}}
                }
                child {node[subtree node] {$\delta$}}
            };
            
        \draw[arrow, shorten >=10mm, shorten <=10mm, dashed, thick] (left subtree root.east) -- (middle subtree root.west);
        
        \node[right=40mm of middle tree, draw=none, fill=none] {}
            child {node[blackened, label=$k+1$] (right subtree root) {$\boldsymbol{B}$}
                child {node[label=$k$] {$A$}
                    child {node[subtree node] {$\alpha$}}
                    child {node[subtree node] {$\beta$}}
                }
                child {node[label=$k$] {$C$}
                    child {node[subtree node] {$\gamma$}}
                    child {node[subtree node] {$\delta$}}
                }
            };
            
        \draw[arrow, shorten >=10mm, shorten <=10mm, dashed, thick] (middle subtree root.east) -- (right subtree root.west);
        
        \node[subpicture label, draw=none, fill=none, left=20mm of left subtree root] {(c)};
\end{tikzpicture}
};

\end{tikzpicture}

	\caption{{\sffamily\bfseries\doubledash{(a)}{(b)}} Rysunek 13.5 z~Podręcznika uzupełniony o~czarne wysokości poszczególnych węzłów drzewa.
	{\sffamily\bfseries(c)} Uzupełniony rysunek 13.6.} \label{fig:13.3-3}
\end{figure}

\exercise %13.3-4
\exercise %13.3-5
\exercise %13.3-6
