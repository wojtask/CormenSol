\subchapter{Operacja wstawiania}
\note{Linia 15 procedury \proc{RB-Insert-Fixup} powinna mówić również o~zamianie lewej rotacji na prawą i~vice versa.}
\bignegskip

\exercise %13.3-1
Pokolorowanie węzła $z$ na czarno mogłoby spowodować naruszenie własności 5 drzewa czerwono-czarnego, podczas gdy procedura \proc{RB-Insert-Fixup} operuje na drzewie, w~którym własność ta jest zachowana i~nie narusza jej w~trakcie swojego działania.

\exercise %13.3-2
Na rys.\ \ref{fig:13.3-2} zilustrowano ciąg drzew czerwono-czarnych powstających po każdym wywołaniu operacji \proc{Tree-Insert} dla zadanego ciągu elementów.
\begin{figure}[!ht]
	\centering \begin{tikzpicture}[
	level/.append style = {level distance=7mm, sibling distance=50mm/2^#1},
	outer/.append style = {node distance=20mm and 15mm}
]

\node[outer] (pic a) {
\begin{tikzpicture}[
	every node/.style = {tree node},
	anchor = center
]
	\node[blackened] {41};
\end{tikzpicture}
};

\node[outer, right=of pic a] (pic b) {
\begin{tikzpicture}[
	every node/.style = {tree node},
	anchor = center
]
	\node[blackened] {41}
            child {node {38}}
            child[missing];
\end{tikzpicture}
};

\node[outer, right=of pic b] (pic c) {
\begin{tikzpicture}[
	every node/.style = {tree node},
	anchor = center
]
	\node[blackened] {38}
            child {node {31}}
            child {node {41}};
\end{tikzpicture}
};

\node[outer, right=of pic c] (pic d) {
\begin{tikzpicture}[
	every node/.style = {tree node},
	anchor = center
]
	\node[blackened] {38}
            child {node[blackened] {31}
                child {node {12}}
                child[missing]
            }
            child {node[blackened] {41}};
\end{tikzpicture}
};

\node[outer, below=of pic b] (pic e) {
\begin{tikzpicture}[
	every node/.style = {tree node},
	anchor = center
]
	\node[blackened] {38}
            child {node[blackened] {19}
                child {node {12}}
                child {node {31}}
            }
            child {node[blackened] {41}};
\end{tikzpicture}
};

\node[outer, right=35mm of pic e] (pic f) {
\begin{tikzpicture}[
	every node/.style = {tree node},
	anchor = center
]
	\node[blackened] {38}
            child {node {19}
                child {node[blackened] {12}
                    child {node {8}}
                    child[missing]
                }
                child {node[blackened] {31}}
            }
            child {node[blackened] {41}};
\end{tikzpicture}
};

\foreach \x in {a, ..., f} {
	\node[subpicture label, below=2mm of pic \x] {(\x)};
}

\end{tikzpicture}

	\caption{Drzewa czerwono-czarne powstałe po wstawieniu węzłów o~kluczach 41, 38, 31, 12, 19, 8 kolejno do początkowo pustego drzewa.
	{\sffamily\bfseries(a)} Po wstawieniu pierwszego elementu drzewo składa się tylko z~korzenia, który początkowo ma kolor czerwony.
	Przywrócenie własności 2 drzewa czerwono-czarnego następuje w~ostatnim wierszu procedury \proc{RB-Insert-Fixup}.
	{\sffamily\bfseries(b)} Dodanie węzła o~kluczu 38 nie powoduje naruszenia żadnej własności drzewa czerwono-czarnego.
	{\sffamily\bfseries\doubledash{(c)}{(f)}} Wstawienie każdego kolejnego elementu produkuje drzewo, w~którym naruszona jest własność 4 przywracana następnie w~procedurze \proc{RB-Insert-Fixup}.} \label{fig:13.3-2}
\end{figure}

\exercise %13.3-3
Wersja rysunków 13.5 i~13.6 z~Podręcznika z~podanymi czarnymi wysokościami węzłów została zaprezentowana na rys.\ \ref{fig:13.3-3}.
Wartości te wyznaczone są jednoznacznie dla każdego węzła, co oznacza, że własność 5 drzewa czerwono-czarnego faktycznie pozostaje zachowana podczas działania procedury \proc{RB-Insert-Fixup}.
\begin{figure}[!ht]
	\centering \begin{tikzpicture}[
	level/.append style = {level distance=7mm, sibling distance=100mm/2^#1},
	every label/.style = {index node, label position=left, draw=none, fill=none},
	subtree node/.style = {tree node, draw=none, fill=none},
	outer/.append style = {node distance=0mm}
]

\node[outer] (pic a) {
\begin{tikzpicture}[
	every node/.style = {tree node},
	anchor = center
]
	\node[draw=none, fill=none] (left tree) {}
            child {node[blackened, label=$k+1$] (left subtree root) {$\boldsymbol{C}$}
                child {node[label=$k$] {$A$}
                    child {node[subtree node] {$\alpha$}}
                    child {node[label=$k$] {$B$}
                        child {node[subtree node] {$\beta$}}
                        child {node[subtree node] {$\gamma$}}
                    }
                }
                child {node[label=$k$] {$D$}
                    child {node[subtree node] {$\delta$}}
                    child {node[subtree node] {$\varepsilon$}}
                }
            };
            
        \node[right=50mm of left tree, draw=none, fill=none] {}
            child {node[label=$k+1$] (right subtree root) {$C$}
                child {node[blackened, label=$k+1$] {$\boldsymbol{A}$}
                    child {node[subtree node] {$\alpha$}}
                    child {node[label=$k$] {$B$}
                        child {node[subtree node] {$\beta$}}
                        child {node[subtree node] {$\gamma$}}
                    }
                }
                child {node[blackened, label=$k+1$] {$\boldsymbol{D}$}
                    child {node[subtree node] {$\delta$}}
                    child {node[subtree node] {$\varepsilon$}}
                }
            };
            
        \draw[arrow, shorten >=10mm, shorten <=10mm, dashed, thick] (left subtree root.east) -- (right subtree root.west);
        
        \node[subpicture label, draw=none, fill=none, left=25mm of left subtree root] {(a)};
\end{tikzpicture}
};

\node[outer, below=of pic a] (pic b) {
\begin{tikzpicture}[
	every node/.style = {tree node},
	anchor = center
]
	\node[draw=none, fill=none] (left tree) {}
            child {node[blackened, label=$k+1$] (left subtree root) {$\boldsymbol{C}$}
                child {node[label=$k$] {$B$}
                    child {node[label=$k$] {$A$}
                        child {node[subtree node] {$\alpha$}}
                        child {node[subtree node] {$\beta$}}
                    }
                    child {node[subtree node] {$\gamma$}}
                }
                child {node[label=$k$] {$D$}
                    child {node[subtree node] {$\delta$}}
                    child {node[subtree node] {$\varepsilon$}}
                }
            };
            
        \node[right=50mm of left tree, draw=none, fill=none] {}
            child {node[label=$k+1$] (right subtree root) {$C$}
                child {node[blackened, label=$k+1$] {$\boldsymbol{B}$}
                    child {node[label=$k$] {$A$}
                        child {node[subtree node] {$\alpha$}}
                        child {node[subtree node] {$\beta$}}
                    }
                    child {node[subtree node] {$\gamma$}}
                }
                child {node[blackened, label=$k+1$] {$\boldsymbol{D}$}
                    child {node[subtree node] {$\delta$}}
                    child {node[subtree node] {$\varepsilon$}}
                }
            };
            
        \draw[arrow, shorten >=10mm, shorten <=10mm, dashed, thick] (left subtree root.east) -- (right subtree root.west);
        
        \node[subpicture label, draw=none, fill=none, left=25mm of left subtree root] {(b)};
\end{tikzpicture}
};

\node[outer, below=of pic b] (pic c) {
\begin{tikzpicture}[
	every node/.style = {tree node},
	anchor = center
]
	\node[draw=none, fill=none] (left tree) {}
            child {node[blackened, label=$k+1$] (left subtree root) {$\boldsymbol{C}$}
                child {node[label=$k$] {$A$}
                    child {node[subtree node] {$\alpha$}}
                    child {node[label=$k$] {$B$}
                        child {node[subtree node] {$\beta$}}
                        child {node[subtree node] {$\gamma$}}
                    }
                }
                child {node[subtree node] {$\delta$}}
            };
            
        \node[right=40mm of left tree, draw=none, fill=none] (middle tree) {}
            child {node[blackened, label=$k+1$] (middle subtree root) {$\boldsymbol{C}$}
                child {node[label=$k$] {$B$}
                    child {node[label=$k$] {$A$}
                        child {node[subtree node] {$\alpha$}}
                        child {node[subtree node] {$\beta$}}
                    }
                    child {node[subtree node] {$\gamma$}}
                }
                child {node[subtree node] {$\delta$}}
            };
            
        \draw[arrow, shorten >=10mm, shorten <=10mm, dashed, thick] (left subtree root.east) -- (middle subtree root.west);
        
        \node[right=40mm of middle tree, draw=none, fill=none] {}
            child {node[blackened, label=$k+1$] (right subtree root) {$\boldsymbol{B}$}
                child {node[label=$k$] {$A$}
                    child {node[subtree node] {$\alpha$}}
                    child {node[subtree node] {$\beta$}}
                }
                child {node[label=$k$] {$C$}
                    child {node[subtree node] {$\gamma$}}
                    child {node[subtree node] {$\delta$}}
                }
            };
            
        \draw[arrow, shorten >=10mm, shorten <=10mm, dashed, thick] (middle subtree root.east) -- (right subtree root.west);
        
        \node[subpicture label, draw=none, fill=none, left=20mm of left subtree root] {(c)};
\end{tikzpicture}
};

\end{tikzpicture}

	\caption{{\sffamily\bfseries\doubledash{(a)}{(b)}} Rys.\ 13.5 i~{\sffamily\bfseries(c)} rys.\ 13.6 z~Podręcznika z~wyznaczonymi czarnymi wysokościami poszczególnych węzłów.} \label{fig:13.3-3}
\end{figure}

\exercise %13.3-4
Jedyny węzeł, który jest kolorowany na czerwono w~trakcie działania procedury \proc{RB-Insert-Fixup}, to \attribb{z}{p}{p}.
Pętla \kw{while} wykonuje się tylko wtedy, gdy \attrib{z}{p} jest czerwonym węzłem.
Jeśli \attrib{z}{p} jest korzeniem drzewa, to warunek w~wierszu 2 (oraz symetryczny z~wiersza 15) jest fałszywy i~pętla natychmiast kończy swe działanie.
Do zmiany koloru \attribb{z}{p}{p} dochodzi więc tylko wtedy, gdy \attrib{z}{p} nie jest korzeniem, czyli gdy $\attribb{z}{p}{p}\ne\attrib{T}{nil}$.

\exercise %13.3-5
Pokażemy, że nowy węzeł wstawiony do niepustego drzewa czerwono-czarnego za pomocą procedury \proc{RB-Insert} zachowuje kolor czerwony.
Bezpośrednio przed wywołaniem pomocniczej procedury \proc{RB-Insert-Fixup} nowo wstawiony węzeł $z$ jest czerwonym liściem.
Jeśli ojciec węzła $z$ jest czarny, to procedura zakończy działanie, a~$z$ zachowa swój kolor.
Załóżmy więc, że ojciec węzła $z$ jest czerwony i~rozważmy przypadki, jak przy omawianiu działania procedury \proc{RB-Insert}.
Zmiany kolorów węzłów w~drzewie odbywają się tylko w~przypadkach 1 i~3, ale~w~żadnym z~nich kolor węzła $z$ nie ulega zmianie -- pole \id{color} może być zmodyfikowane jedynie dla dziadka (ojca ojca) węzła $z$ i~synów dziadka węzła $z$.
Drzewo, do którego $z$ zostało wstawione, nie jest puste, więc $z$ nie może być korzeniem.
Dlatego również ostatni wiersz procedury \proc{RB-Insert-Fixup} nie zaktualizuje koloru $z$ na czarny.

\exercise %13.3-6
Podczas wstawiania węzła do drzewa czerwono-czarnego w~reprezentacji bez wskaźników do ojców, do zapamiętania ścieżki od korzenia do nowego węzła użyjemy stosu.
Wstawimy do niego każdy kolejno odwiedzony węzeł drzewa, zanim wywołana zostanie procedura \proc{RB-Insert-Fixup}.
Dokładniej, między linią 4 a~5 w~procedurze \proc{RB-Insert} będziemy wywoływać $\proc{Push}(S,y)$, gdzie $S$ jest początkowo stosem zawierającym tylko \attrib{T}{nil} (czyli ojca korzenia drzewa).
Pominiemy w~niej także linię 8.

Elementy zebrane na stosie $S$ są wystarczające dla procedury \proc{RB-Insert-Fixup}, ponieważ korzysta ona ze wskaźnika $p$ jedynie dla węzłów znajdujących się w~$S$.
Musimy następnie zmodyfikować procedurę \proc{RB-Insert-Fixup} tak, aby każde odwołanie do przodka danego węzła przy pomocy wskaźnika $p$ było zamienione na odwołanie do odpowiedniego elementu stosu.
Implementacja tej procedury została przedstawiona na poniższym pseudokodzie.
\begin{codebox}
\Procname{$\proc{RB-Parentless-Insert-Fixup}(T,S,z)$}
\li	$p\gets\proc{Pop}(S)$
\li	\While $\attrib{p}{color}=\const{red}$
\li		\Do $r\gets\proc{Pop}(S)$
\li			\If $p=\attrib{r}{left}$
\li				\Then $y\gets\attrib{r}{right}$
\li					\If $\attrib{y}{color}=\const{red}$
\li						\Then $\attrib{y}{color}\gets\attrib{p}{color}\gets\const{black}$
\li							$\attrib{r}{color}\gets\const{red}$
\li							$z\gets r$
\li							$p\gets\proc{Pop}(S)$
\li						\Else \If $z=\attrib{p}{right}$
\li							\Then $z\leftrightarrow p$
\li								$\proc{RB-Parentless-Left-Rotate}(T,z,r)$
							\End
\li							$\attrib{p}{color}\gets\const{black}$
\li							$\attrib{r}{color}\gets\const{red}$
\li							$\proc{RB-Parentless-Right-Rotate}(T,r,\proc{Pop}(S))$ \label{li:no-parents-insert-fixup-left-rotate}
						\End
\li				\Else (to samo co po \kw{then} z~zamienionymi ,,right'' i~,,left'')
				\End
		\End
\li	$\attribb{T}{root}{color}\gets\const{black}$
\end{codebox}
W~każdej iteracji pętli \kw{while}, oprócz węzła $z$, potrzebujemy znać także jego ojca i~dziadka (ojca ojca).
Węzły te są pobierane ze stosu $S$ przekazywanego do procedury i~umieszczane w~zmiennych, odpowiednio, $p$ i~$r$.
Ponieważ rotacje także wykorzystują wskaźniki do ojca, to wyeliminujemy te użycia poprzez przekazanie im dodatkowego parametru -- ojca węzła, dla którego rotację wykonujemy.
W~wierszu \ref{li:no-parents-insert-fixup-left-rotate} powyższej procedury ojciec węzła $r$ jest ściągany ze stosu i~przekazywany do prawej rotacji.
Pobranie dodatkowego elementu ze stosu $S$ nie powoduje jednak problemów, bo pętla \kw{while} nie wykona więcej iteracji i~żaden element nie będzie już ściągany z~$S$.

Poniżej znajduje się operacja lewej rotacji na węźle $x$ w~drzewie $T$, która nie korzysta ze wskaźników na ojca.
Zamiast tego otrzymuje ona trzeci parametr, będący ojcem węzła $x$.
\begin{codebox}
\Procname{$\proc{RB-Parentless-Left-Rotate}(T,x,p)$}
\li	$y\gets\attrib{x}{right}$
\li	$\attrib{x}{right}\gets\attrib{y}{left}$
\li	\If $p=\attrib{T}{nil}$
\li		\Then $\attrib{T}{root}\gets y$
\li		\Else \If $x=\attrib{p}{left}$
\li			\Then $\attrib{p}{left}\gets y$
\li			\Else $\attrib{p}{right}\gets y$
			\End
		\End
\li	$\attrib{y}{left}\gets x$
\end{codebox}
Implementacja prawej rotacji nie wykorzystującej wskaźników na ojca jest analogiczna.
