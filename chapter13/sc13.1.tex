\subchapter{Własności drzew czerwono-czarnych}

\exercise %13.1-1
Drzewa zostały przedstawione na rys.\ \ref{fig:13.1-1}.
\begin{figure}[ht]
	\centering \begin{tikzpicture}[
	level/.append style = {level distance=7mm, sibling distance=100mm/2^#1},
	every label/.style = {index node, label position=left, draw=none, fill=none},
	outer/.append style = {node distance=15mm and 15mm}
]

\newcommand\nilnode{%
	node[align=center, inner sep=1pt, minimum size=3mm, rectangle, rounded corners=3pt, draw, font=\scriptsize, blackened] {\const{nil}}
}

\node[outer] (pic a) {
\begin{tikzpicture}[
	every node/.style = {tree node},
	anchor = center
]
	\node[label=177:2, blackened] {8}
		child {node[label=2] {4}
			child {node[label=1, blackened] {2}
				child {node[label=1] {1}
					child {\nilnode}
					child {\nilnode}
				}
				child {node[label=1] {3}
					child {\nilnode}
					child {\nilnode}
				}
			}
			child {node[label=1, blackened] {6}
				child {node[label=1] {5}
					child {\nilnode}
					child {\nilnode}
				}
				child {node[label=1] {7}
					child {\nilnode}
					child {\nilnode}
				}
			}
		}
		child {node[label=2] {12}
			child {node[label=1, blackened] {10}
				child {node[label=1] {9}
					child {\nilnode}
					child {\nilnode}
				}
				child {node[label=1] {1}
					child {\nilnode}
					child {\nilnode}
				}
			}
			child {node[label=1, blackened] {14}
				child {node[label=1] {13}
					child {\nilnode}
					child {\nilnode}
				}
				child {node[label=1] {15}
					child {\nilnode}
					child {\nilnode}
				}
			}
		};
\end{tikzpicture}
};

\node[outer, below=of pic a] (pic b) {
\begin{tikzpicture}[
	every node/.style = {tree node},
	anchor = center
]
	\node[label=177:3, blackened] {8}
		child {node[label=2, blackened] {4}
			child {node[label=2] {2}
				child {node[label=1, blackened] {1}
					child {\nilnode}
					child {\nilnode}
				}
				child {node[label=1, blackened] {3}
					child {\nilnode}
					child {\nilnode}
				}
			}
			child {node[label=2] {6}
				child {node[label=1, blackened] {5}
					child {\nilnode}
					child {\nilnode}
				}
				child {node[label=1, blackened] {7}
					child {\nilnode}
					child {\nilnode}
				}
			}
		}
		child {node[label=2, blackened] {12}
			child {node[label=2] {10}
				child {node[label=1, blackened] {9}
					child {\nilnode}
					child {\nilnode}
				}
				child {node[label=1, blackened] {1}
					child {\nilnode}
					child {\nilnode}
				}
			}
			child {node[label=2] {14}
				child {node[label=1, blackened] {13}
					child {\nilnode}
					child {\nilnode}
				}
				child {node[label=1, blackened] {15}
					child {\nilnode}
					child {\nilnode}
				}
			}
		};
\end{tikzpicture}
};

\node[outer, below=of pic b] (pic c) {
\begin{tikzpicture}[
	every node/.style = {tree node},
	anchor = center
]
	\node[label=177:4, blackened] {8}
		child {node[label=3, blackened] {4}
			child {node[label=2, blackened] {2}
				child {node[label=1, blackened] {1}
					child {\nilnode}
					child {\nilnode}
				}
				child {node[label=1, blackened] {3}
					child {\nilnode}
					child {\nilnode}
				}
			}
			child {node[label=2, blackened] {6}
				child {node[label=1, blackened] {5}
					child {\nilnode}
					child {\nilnode}
				}
				child {node[label=1, blackened] {7}
					child {\nilnode}
					child {\nilnode}
				}
			}
		}
		child {node[label=3, blackened] {12}
			child {node[label=2, blackened] {10}
				child {node[label=1, blackened] {9}
					child {\nilnode}
					child {\nilnode}
				}
				child {node[label=1, blackened] {1}
					child {\nilnode}
					child {\nilnode}
				}
			}
			child {node[label=2, blackened] {14}
				child {node[label=1, blackened] {13}
					child {\nilnode}
					child {\nilnode}
				}
				child {node[label=1, blackened] {15}
					child {\nilnode}
					child {\nilnode}
				}
			}
		};
\end{tikzpicture}
};

\foreach \x in {a,b,c} {
	\node[subpicture label, below=2mm of pic \x] {(\x)};
}

\end{tikzpicture}

	\caption{Drzewa czerwono-czarne o~wysokości 3 zawierające zbiór kluczy $\langle$1,\! 2,\! \dots,\! 15$\rangle$ z~dodanymi węzłami \const{nil}.
	Czarna wysokość tych drzew wynosi, odpowiednio, {\sffamily\bfseries(a)} 2, {\sffamily\bfseries(b)} 3 oraz {\sffamily\bfseries(c)} 4.} \label{fig:13.1-1}
\end{figure}

\exercise %13.1-2
Rys.\ \ref{fig:13.1-2} przedstawia drzewo po dodaniu nowego węzła o~nieustalonym kolorze.
Pokolorowanie go na czerwono naruszy jednak własność 4 drzewa czerwono-czarnego, natomiast nadanie mu koloru czarnego naruszy własność 5 -- na ścieżce od korzenia drzewa do nowego węzła będą 3 czarne węzły, a~na ścieżce od korzenia do każdego innego liścia -- po 2 czarne węzły.
A~zatem niezależnie od wyboru koloru dla nowego węzła, wynikowe drzewo nie będzie drzewem czerwono-czarnym.
\begin{figure}[ht]
	\centering \begin{tikzpicture}[
	every node/.style = {tree node},
	level/.append style = {level distance=7mm, sibling distance=100mm/2^#1}
]
\node[blackened] {26}
	child {node {17}
		child {node[blackened] {14}
			child {node {10}
				child {node[blackened] {7}
					child {node {3}}
					child[missing]
				}
				child {node[blackened] {12}}
			}
			child {node[blackened] {16}
				child {node {15}}
				child[missing]
			}
		}
		child {node[blackened] {21}
			child {node[blackened] {19}
				child[missing]
				child {node {20}}
			}
			child {node[blackened] {23}}
		}
	}
	child {node[blackened] {41}
		child {node {30}
			child {node[blackened] {28}}
			child {node[blackened] {38}
				child {node {35}
					child[missing]
					child {node[med grayed] {36}}
				}
				child {node {39}}
			}
		}
		child {node[blackened] {47}}
	};
\end{tikzpicture}

	\caption{Drzewo z~rys.\ 13.1 z~Podręcznika po dodaniu węzła o~kluczu 36 procedurą \proc{Tree-Insert}.} \label{fig:13.1-2}
\end{figure}

\exercise %13.1-3
Po zmianie koloru korzenia na czarny w~uproszczonym drzewie czerwono-czarnym własności 1--4 drzewa czerwono-czarnego oczywiście będą spełnione.
Liczba czarnych węzłów na każdej prostej ścieżce od korzenia do liścia zwiększy się o~1, a~liczba czarnych węzłów na prostych ścieżkach od dowolnego innego węzła do ich potomnych liści nie zmieni się.
To oznacza, że także własność 5 nie będzie naruszona i~otrzymane drzewo będzie drzewem czerwono-czarnym.

\exercise %13.1-4
W~wyniku wykonania operacji pochłaniania każdy ojciec $x$ czerwonego węzła $y$ stanie się ojcem dwóch czarnych synów $y$.
Stopień węzła $x$ będzie więc nadal wynosił 2 (jeśli obaj synowie $x$ są czarni), zwiększy się do 3 (jeśli dokładnie jeden z~synów $x$ był czerwony) albo do 4 (gdy obaj synowie $x$ byli czerwoni).

Po pochłonięciu wszystkich czerwonych węzłów dowolna prosta ścieżka od korzenia do liścia będzie składać się z~tej samej liczby czarnych węzłów.
Głębokość każdego liścia będzie równa czarnej wysokości początkowego drzewa czerwono-czarnego.

\exercise %13.1-5
Skorzystamy z~faktu, że każda prosta ścieżka od węzła $x$ do jego potomnego liścia w~drzewie czerwono-czarnym składa się z~tej samej liczby czarnych węzłów.
Na najkrótszej z~nich są same czarne węzły, a~na najdłuższej oprócz nich znajduje się maksymalna możliwa liczba czerwonych węzłów.
Oczywiście dwa czerwone węzły nie mogą następować bezpośrednio po sobie na takiej ścieżce, skąd wynika, że jest ich co najwyżej tyle, ile czarnych węzłów.
Najdłuższa ścieżka jest więc co najwyżej dwukrotnie dłuższa od najkrótszej.

\exercise %13.1-6
Drzewo czerwono-czarne o~czarnej wysokości $k$ ma najmniejszą możliwą liczbę węzłów wewnętrznych, jeśli każda ścieżka od korzenia do liścia jest możliwie najkrótsza.
Podobnie, największa liczba węzłów wewnętrznych w~tym drzewie jest osiągana, gdy każda taka ścieżka jest możliwie najdłuższa.
Na podstawie poprzedniego zadania mamy, że ich długość, a~co za tym idzie także wysokość drzewa, wynosi co najmniej $k$ i~co najwyżej $2k$.
Drzewo składa się zatem z~co najmniej $2^k-1$ i~co najwyżej $2^{2k}-1$ węzłów wewnętrznych.

\exercise %13.1-7
W~\refExercise{13.1-5} zobaczyliśmy, że na prostej ścieżce od korzenia do liścia w~drzewie czerwono-czarnym może znajdować się maksymalnie tyle samo węzłów czerwonych co czarnych.
Jeśli każda taka ścieżka zawiera maksymalną liczbę czerwonych węzłów, to drzewo jest pełnym drzewem binarnym o~wysokości parzystej, w~którym na parzystych poziomach są węzły czarne, a~na nieparzystych -- czerwone.
Przykładem takiego drzewa jest to zobrazowane na rys.\ \ref{fig:13.1-1}(a).
W~drzewach takich czerwonych węzłów wewnętrznych jest 2 razy więcej niż czarnych węzłów wewnętrznych.

Z~drugiej strony zauważmy, że pełne drzewo wyszukiwań binarnych, w~którym wszystkie węzły są czarne, jest poprawnym drzewem czerwono-czarnym (np.\ to z~rys.\ \ref{fig:13.1-1}(c)).
Minimalny stosunek czerwonych węzłów wewnętrznych do czarnych węzłów wewnętrznych wynosi więc 0.
