\subchapter{Własności drzew czerwono-czarnych}

\exercise %13.1-1
Drzewa zostały przedstawione na rys.\ \ref{fig:13.1-1}.
\begin{figure}[ht]
	\centering \begin{tikzpicture}[
	level/.append style = {level distance=7mm, sibling distance=100mm/2^#1},
	every label/.style = {index node, label position=left, draw=none, fill=none},
	outer/.append style = {node distance=10mm}
]

\newcommand\nilnode{%
	node[align=center, inner sep=1pt, minimum size=3mm, rectangle, rounded corners=3pt, draw, font=\scriptsize, blackened] {\const{nil}}
}

\node[outer] (pic a) {
\begin{tikzpicture}[
	every node/.style = {tree node},
	anchor = center
]
	\node[label=177:2, blackened] (root) {8}
		child {node[label=2] {4}
			child {node[label=1, blackened] {2}
				child {node[label=1] {1}
					child {\nilnode}
					child {\nilnode}
				}
				child {node[label=1] {3}
					child {\nilnode}
					child {\nilnode}
				}
			}
			child {node[label=1, blackened] {6}
				child {node[label=1] {5}
					child {\nilnode}
					child {\nilnode}
				}
				child {node[label=1] {7}
					child {\nilnode}
					child {\nilnode}
				}
			}
		}
		child {node[label=2] {12}
			child {node[label=1, blackened] {10}
				child {node[label=1] {9}
					child {\nilnode}
					child {\nilnode}
				}
				child {node[label=1] {11}
					child {\nilnode}
					child {\nilnode}
				}
			}
			child {node[label=1, blackened] {14}
				child {node[label=1] {13}
					child {\nilnode}
					child {\nilnode}
				}
				child {node[label=1] {15}
					child {\nilnode}
					child {\nilnode}
				}
			}
		};
		
	\node[subpicture label, draw=none, fill=none, left=50mm of root] {(a)};
\end{tikzpicture}
};

\node[outer, below=of pic a] (pic b) {
\begin{tikzpicture}[
	every node/.style = {tree node},
	anchor = center
]
	\node[label=177:3, blackened] (root) {8}
		child {node[label=2, blackened] {4}
			child {node[label=2] {2}
				child {node[label=1, blackened] {1}
					child {\nilnode}
					child {\nilnode}
				}
				child {node[label=1, blackened] {3}
					child {\nilnode}
					child {\nilnode}
				}
			}
			child {node[label=2] {6}
				child {node[label=1, blackened] {5}
					child {\nilnode}
					child {\nilnode}
				}
				child {node[label=1, blackened] {7}
					child {\nilnode}
					child {\nilnode}
				}
			}
		}
		child {node[label=2, blackened] {12}
			child {node[label=2] {10}
				child {node[label=1, blackened] {9}
					child {\nilnode}
					child {\nilnode}
				}
				child {node[label=1, blackened] {11}
					child {\nilnode}
					child {\nilnode}
				}
			}
			child {node[label=2] {14}
				child {node[label=1, blackened] {13}
					child {\nilnode}
					child {\nilnode}
				}
				child {node[label=1, blackened] {15}
					child {\nilnode}
					child {\nilnode}
				}
			}
		};
		
	\node[subpicture label, draw=none, fill=none, left=50mm of root] {(b)};
\end{tikzpicture}
};

\node[outer, below=of pic b] (pic c) {
\begin{tikzpicture}[
	every node/.style = {tree node},
	anchor = center
]
	\node[label=177:4, blackened] (root) {8}
		child {node[label=3, blackened] {4}
			child {node[label=2, blackened] {2}
				child {node[label=1, blackened] {1}
					child {\nilnode}
					child {\nilnode}
				}
				child {node[label=1, blackened] {3}
					child {\nilnode}
					child {\nilnode}
				}
			}
			child {node[label=2, blackened] {6}
				child {node[label=1, blackened] {5}
					child {\nilnode}
					child {\nilnode}
				}
				child {node[label=1, blackened] {7}
					child {\nilnode}
					child {\nilnode}
				}
			}
		}
		child {node[label=3, blackened] {12}
			child {node[label=2, blackened] {10}
				child {node[label=1, blackened] {9}
					child {\nilnode}
					child {\nilnode}
				}
				child {node[label=1, blackened] {11}
					child {\nilnode}
					child {\nilnode}
				}
			}
			child {node[label=2, blackened] {14}
				child {node[label=1, blackened] {13}
					child {\nilnode}
					child {\nilnode}
				}
				child {node[label=1, blackened] {15}
					child {\nilnode}
					child {\nilnode}
				}
			}
		};
		
	\node[subpicture label, draw=none, fill=none, left=50mm of root] {(c)};
\end{tikzpicture}
};

\end{tikzpicture}

	\caption{Drzewa czerwono-czarne o~wysokości 3 zawierające zbiór kluczy $\langle$1,\! 2,\! \dots,\! 15$\rangle$ z~dodanymi węzłami \const{nil}.
	Czarna wysokość tych drzew wynosi, odpowiednio, {\sffamily\bfseries(a)} 2, {\sffamily\bfseries(b)} 3 oraz {\sffamily\bfseries(c)} 4.} \label{fig:13.1-1}
\end{figure}

\exercise %13.1-2
\exercise %13.1-3
\exercise %13.1-4
\exercise %13.1-5
\exercise %13.1-6
\exercise %13.1-7
