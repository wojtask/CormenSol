\subchapter{Operacje rotacji}
\note{Linia 3 w~procedurze \proc{Left-Rotate} powinna być zastąpiona fragmentem:}
\bignegskip
\begin{codebox}
\zi	\If $\attrib{y}{left}\ne\attrib{T}{nil}$
\zi		\Then $\attribb{y}{left}{p}\gets x$
		\End
\end{codebox}
\note{W~rozwiązaniach przyjmujemy, że procedury \proc{Left-Rotate} i~\proc{Right-Rotate} działają poprawnie także dla zwykłych drzew wyszukiwań binarnych.
W~tych wersjach procedur odwołania do\/ \attrib{T}{nil} w~wierszu 5 oraz we fragmencie zastępującym linię 3 (patrz wyżej) zastąpione są przez \const{nil}.}
\bignegskip

\exercise %13.2-1
Pseudokod operacji prawej rotacji wygląda analogicznie do procedury \proc{Left-Rotate}, w~której zamieniono ze sobą pola \id{left} i~\id{right}.
\begin{codebox}
\Procname{$\proc{Right-Rotate}(T,x)$}
\li	$y\gets\attrib{x}{left}$
\li	$\attrib{x}{left}\gets\attrib{y}{right}$
\li	\If $\attrib{y}{right}\ne\attrib{T}{nil}$
\li		\Then $\attribb{y}{right}{p}\gets x$
		\End
\li	$\attrib{y}{p}\gets\attrib{x}{p}$
\li	\If $\attrib{x}{p}=\attrib{T}{nil}$
\li		\Then $\attrib{T}{root}\gets y$
\li		\Else \If $x=\attribb{x}{p}{right}$
\li				\Then $\attribb{x}{p}{right}\gets y$
\li				\Else $\attribb{x}{p}{left}\gets y$
				\End
		\End
\li	$\attrib{y}{right}\gets x$
\li	$\attrib{x}{p}\gets y$
\end{codebox}

\exercise %13.2-2
Na dowolnym węźle $x$ w~drzewie wyszukiwań binarnych $T$ można wykonać lewą rotację, o~ile prawy syn $x$ jest różny od \const{nil}, a~prawą rotację -- o~ile jego lewy syn jest różny od \const{nil}.
Oznacza to, że istnieje wzajemna jednoznaczność między rotacjami a~krawędziami drzewa, wokół których można przeprowadzić rotacje.
W~drzewie o~$n$ węzłach jest dokładnie $n-1$ krawędzi i~tyle samo różnych rotacji można w~nim wykonać.

\exercise %13.2-3
Wewnętrzna struktura węzłów w~poddrzewach $\alpha$, $\beta$, $\gamma$ nie zmienia się w~wyniku wykonania rotacji.
A~zatem głębokość dowolnego węzła $a$ z~$\alpha$ zwiększa się o~1, głębokość dowolnego węzła $b$ z~$\beta$ nie zmienia się, a~głębokość dowolnego węzła $c$ z~$\gamma$ zmniejsza się o~1.

\exercise %13.2-4
\note{Rozwiązanie zakłada, że zadanie dotyczy drzew wyszukiwań binarnych, zgodnie z~treścią oryginalną.
Ponadto doprecyzujemy pojęcie (prawego) łańcucha z~treści zadania -- jest to drzewo wyszukiwań binarnych, w~którym każdy węzeł znajduje się na jego prawym kręgosłupie (patrz problem 13.4).}

\noindent Wykażemy stwierdzenie podane we wskazówce.
Jeśli drzewo $T$ nie jest prawym łańcuchem, to istnieje węzeł $y$ na prawym kręgosłupie $T$ mający lewego syna $x$ spoza prawego kręgosłupa.
Zauważmy, że wykonanie prawej rotacji na węźle $y$ dodaje $x$ do prawego kręgosłupa, jednocześnie nie usuwając z~niego żadnego innego węzła.
Prawa rotacja zwiększa więc długość prawego kręgosłupa o~1, dlatego wystarczy wykonać co najwyżej $n-1$ rotacji, aby przekształcić $T$ w~prawy łańcuch.

Jeśli znamy ciąg prawych rotacji przekształcających drzewo $T$ w~prawy łańcuch $T'$, to możemy powrócić od $T'$ do $T$ poprzez wykonanie tego ciągu w~odwrotnej kolejności, zamieniając każdą prawą rotację w~odpowiadającą jej lewą rotację.

Niech $T_1$, $T_2$ będą drzewami wyszukiwań binarnych o~$n$ węzłach, a~$T'$ -- jedynym prawym łańcuchem powstałym z~$T_1$ bądź z~$T_2$ po przeprowadzeniu powyżej opisanej transformacji.
Niech $\langle r_1,r_2,\dots,r_k\rangle$ i~$\langle r_1',r_2',\dots,r_{k'}'\rangle$ będą ciągami prawych rotacji przekształcających, odpowiednio $T_1$ w~$T'$ i~$T_2$ w~$T'$.
Wcześniej pokazaliśmy, że istnieją takie ciągi, w~których $k$, $k'\le n-1$.
Dla każdej prawej rotacji $r_i'$ niech $l_i'$ będzie odpowiadającą jej lewą rotacją.
Wówczas ciąg $\langle r_1,r_2,\dots,r_k,l_{k'}',l_{k'-1}',\dots,l_1'\rangle$ co najwyżej $2n-2$ rotacji pozwala przekształcić drzewo $T_1$ w~$T_2$.

\exercise %13.2-5
\note{Treść zadania nie definiuje\/ $n$ -- przyjmiemy, że jest to rozmiar drzew\/ $T_1$ i\/~$T_2$.}

\noindent W~drzewie $T_1$ stanowiącym prawy łańcuch (definicja w~komentarzu do rozwiązania \refExercise{13.2-4}) nie można wykonać żadnej prawej rotacji, gdyż nie istnieje w~nim węzeł mający lewego syna.
A~zatem $T_1$ nie jest prawostronnie przekształcalne do jakiegokolwiek innego drzewa wyszukiwań binarnych $T_2$.
Innym przykładem drzew $T_1$ i~$T_2$ są dowolne 2 drzewa, które przechowują różne zbiory kluczy, gdyż rotacje zmieniają strukturę drzewa, a~nie klucze w~węzłach.

Niech teraz $T_1$, $T_2$ będą drzewami wyszukiwań binarnych o~$n$ węzłach takimi, że $T_1$ jest prawostronnie przekształcalne do $T_2$.
Załóżmy, że $n>0$ -- w~przeciwnym przypadku bowiem drzewa są puste i~nie jest potrzebna żadna rotacja, aby przekształcić $T_1$ w~$T_2$.

Wykonanie prawej rotacji na węźle $x$ skutkuje tym, że jego lewy syn zostaje przeniesiony na mniejszą głębokość w~drzewie, stając się nowym synem ojca $x$.
Wynika stąd, że jedynie węzły leżące na lewym kręgosłupie drzewa mogą być przeniesione w~miejsce korzenia.
Korzeń $r$ drzewa $T_2$ leży zatem na lewym kręgosłupie $T_1$.
Można więc z~$T_1$ utworzyć drzewo $T_1'$ poprzez wykonywanie prawych rotacji na aktualnym korzeniu, aż $r$ stanie się korzeniem.
Oczywiście $T_1'$ jest prawostronnie przekształcalne do $T_2$, więc także lewe i~prawe poddrzewo $T_1'$ jest prawostronnie przekształcalne do, odpowiednio, lewego i~prawego poddrzewa $T_2$.
W~celu otrzymania drzewa $T_2$ wystarczy więc powtórzyć opisane ciągi prawych rotacji rekurencyjnie na lewym i~prawym poddrzewie drzewa $T_1'$.

Liczbę rotacji potrzebnych do przekształcenia drzewa $T_1$ w~$T_1'$ można ograniczyć od góry przez długość lewego kręgosłupa drzewa $T_1$, która wynosi co najwyżej $n-1$.
Wywołania rekurencyjne mogą następnie wykonywać podobne ciągi rotacji na drzewach o~łącznej liczbie $n-1$ węzłów.
Jeśli przez $R(n)$ oznaczymy liczbę wywołań operacji \proc{Right-Rotate} potrzebnych do przekształcenia $n$\nbhyphen węzłowego drzewa $T_1$ do $T_2$, to otrzymujemy zależność $R(n)\le R(n-1)+n-1$.
Rozwiązaniem tej rekurencji jest oczywiście $R(n)=O(n^2)$.
