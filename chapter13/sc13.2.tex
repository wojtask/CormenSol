\subchapter{Operacje rotacji}
\note{Linia 3 w~procedurze \proc{Left-Rotate} powinna być zastąpiona liniami:}
\bignegskip
\begin{codebox}
\zi	\If $\attrib{y}{left}\ne\attrib{T}{nil}$
\zi		\Then $\attrib{\attrib{y}{left}}{p}\gets x$
		\End
\end{codebox}
\note{W~kilku zadaniach z~tego podrozdziału mowa jest o~rotacjach na drzewach wyszukiwań binarnych, a~nie tylko na bardziej wyspecjalizowanych drzewach czerwono-czarnych.
Operacje te działają identycznie jak w~procedurach \proc{Left-Rotate} i~\proc{Right-Rotate}, w~których zamieniono\/ \attrib{T}{nil} z~wiersza 5 na \const{nil}.}
\bignegskip

\exercise %13.2-1
Oto pseudokod prawej rotacji na drzewie czerwono-czarnym:
\begin{codebox}
\Procname{$\proc{Right-Rotate}(T,x)$}
\li	$y\gets\attrib{x}{left}$
\li	$\attrib{x}{left}\gets\attrib{y}{right}$
\li	\If $\attrib{y}{right}\ne\attrib{T}{nil}$
\li		\Then $\attrib{\attrib{y}{right}}{p}\gets x$
		\End
\li	$\attrib{y}{p}\gets\attrib{x}{p}$
\li	\If $\attrib{x}{p}=\attrib{T}{nil}$
\li		\Then $\attrib{T}{root}\gets y$
\li		\Else \If $x=\attrib{\attrib{x}{p}}{left}$
\li				\Then $\attrib{\attrib{x}{p}}{left}\gets y$
\li				\Else $\attrib{\attrib{x}{p}}{right}\gets y$
				\End
		\End
\li	$\attrib{y}{right}\gets x$
\li	$\attrib{x}{p}\gets y$
\end{codebox}

\exercise %13.2-2
Na dowolnym węźle $x$ w~drzewie wyszukiwań binarnych $T$ można wykonać lewą rotację, o~ile prawy syn $x$ jest różny od \const{nil} a~prawą rotację -- o~ile lewy syn jest różny od \const{nil}.
Oznacza to, że istnieje wzajemna jednoznaczność między rotacjami a~krawędziami drzewa, wokół których można przeprowadzić rotacje.
W~drzewie o~$n$ węzłach jest dokładnie $n-1$ krawędzi i~tyle samo różnych rotacji można w~nim wykonać.

\exercise %13.2-3
Zauważmy, że wewnętrzna struktura węzłów w~poddrzewach $\alpha$, $\beta$, $\gamma$ nie zmienia się w~wyniku wykonania rotacji.
A~zatem głębokość dowolnego węzła $a$ z~$\alpha$ zwiększa się o~1, głębokość dowolnego węzła $b$ z~$\beta$ nie zmienia się, a~głębokość dowolnego węzła $c$ z~$\gamma$ zmniejsza się o~1.

\exercise %13.2-4
\note{Rozwiązanie zakłada, że zadanie dotyczy drzew wyszukiwań binarnych, zgodnie z~treścią oryginalną.
Ponadto doprecyzujemy pojęcie (prawego) łańcucha z~treści zadania -- jest to drzewo wyszukiwań binarnych, w~którym lewe poddrzewo każdego węzła jest puste.}

\noindent Wykażemy stwierdzenie ze wskazówki.
Dla danego drzewa wyszukiwań binarnych $T$ o~$n$ węzłach, niech $R_T$ będzie zbiorem zawierającym korzeń drzewa $T$ oraz wszystkie węzły osiągalne z~korzenia podczas poruszania się wyłącznie w~kierunku prawych synów (czyli po wskaźnikach \id{right}).
Dopóki drzewo $T$ nie jest prawym łańcuchem, czyli dopóki są w~nim węzły nienależące do $R_T$, to istnieje węzeł $y\in R_T$, którego lewy syn $x\not\in R_T$.
Zauważmy, że wykonanie prawej rotacji na węźle $y$ dodaje $x$ do $R_T$, jednocześnie nie usuwając z~niego żadnego innego węzła.
Prawa rotacja zwiększa więc rozmiar $R_T$ o~1, dlatego wystarczy wykonać co najwyżej $n-1$ rotacji, aby przekształcić $T$ w~prawy łańcuch.

Jeśli znamy ciąg prawych rotacji przekształcających drzewo $T$ w~prawy łańcuch $T'$, to możemy powrócić od $T'$ do $T$ poprzez wykonanie tego ciągu w~odwrotnej kolejności, zamieniając każdą prawą rotację w~odpowiadającą jej lewą rotację.

Niech $T_1$, $T_2$ będą drzewami wyszukiwań binarnych o~$n$ węzłach, a~$T'$ -- jedynym prawym łańcuchem powstałym z~$T_1$ bądź z~$T_2$ po przeprowadzeniu powyżej opisanej transformacji.
Niech $r=\langle r_1,r_2,\dots,r_k\rangle$ i~$r'=\langle r_1',r_2',\dots,r_{k'}'\rangle$ będą ciągami prawych rotacji przekształcających, odpowiednio $T_1$ w~$T'$ i~$T_2$ w~$T'$.
Z~poprzedniego paragrafu wiemy, że istnieją takie ciągi $r$, $r'$, gdzie $k$, $k'\le n-1$.
Dla każdej prawej rotacji $r_i'$ niech $l_i'$ będzie odpowiadającą jej lewą rotacją.
Wówczas ciąg $\langle r_1,r_2,\dots,r_k,l_{k'}',l_{k'-1}',\dots,l_1'\rangle$ co najwyżej $2n-2$ rotacji pozwala przekształcić drzewo $T_1$ w~$T_2$.

\exercise %13.2-5
\note{Treść zadania nie definiuje\/ $n$ -- przyjmiemy, że jest to rozmiar drzew\/ $T_1$ i\/~$T_2$.}

\noindent W~drzewie $T_1$ stanowiącym prawy łańcuch (definicja w~komentarzu do rozwiązania \refExercise{13.2-4}) nie można wykonać żadnej prawej rotacji, gdyż nie istnieje w~nim węzeł o~niepustym lewym synu.
A~zatem $T_1$ nie jest prawostronnie przekształcalne do jakiegokolwiek drzewa wyszukiwań binarnych $T_2$ różnego od $T_1$.
Innym przykładem drzew $T_1$ i~$T_2$ są dowolne 2 drzewa, które reprezentują różne zbiory kluczy, gdyż rotacje zmieniają strukturę drzewa, a~nie klucze w~węzłach.

Niech teraz $T_1$, $T_2$ będą drzewami wyszukiwań binarnych o~$n$ węzłach takimi, że $T_1$ jest prawostronnie przekształcalne do $T_2$.
Załóżmy, że $n>0$ -- w~przeciwnym przypadku bowiem drzewa są puste i~nie jest potrzebna żadna rotacja, aby przekształcić $T_1$ w~$T_2$.

Dla drzewa binarnego $T$ zdefiniujmy zbiór $L_T$ analogicznie do zbioru $R_T$ z~\refExercise{13.2-4} -- jako składający się z~korzenia drzewa $T$ oraz wszystkich węzłów osiągalnych z~korzenia $T$ przy poruszaniu się po wyłącznie po wskaźnikach \id{left}.
Prawa rotacja na węźle $x$ w~drzewie $T_1$ przenosi lewego syna $x$ w~miejsce $x$, dlatego jedynie węzły należące do $L_{T_1}$ mogą być przeniesione w~miejsce korzenia drzewa $T_1$.
Korzeń $r$ drzewa $T_2$ należy zatem do $L_{T_1}$.
Można więc z~$T_1$ utworzyć drzewo $T_1'$ poprzez wykonanie ciągu prawych rotacji na aktualnym korzeniu, aż $r$ stanie się korzeniem.
Oczywiście $T_1'$ jest prawostronnie przekształcalne do $T_2$, więc także lewe i~prawe poddrzewo $T_1'$ jest prawostronnie przekształcalne do, odpowiednio, lewego i~prawego poddrzewa $T_2$.
W~celu otrzymania drzewa $T_2$ wystarczy więc powtórzyć opisane ciągi prawych rotacji rekurencyjnie na lewym i~prawym poddrzewie drzewa $T_1'$.

Liczbę wykonanych rotacji przekształcających drzewo $T_1$ w~$T_1'$, można ograniczyć od góry przez rozmiar zbioru $L_{T_1}$, który wynosi co najwyżej $n$.
Wywołania rekurencyjne mogą następnie wykonywać podobne ciągi rotacji na każdym spośród $n-1$ pozostałych węzłów w~poddrzewach drzewa $T_1'$.
Wynika stąd, że sumaryczną liczbę rotacji wykonywanych przy przekształceniu $T_1$ do $T_2$, można ograniczyć od góry przez $O(n^2)$.
