\subchapter{Operacje rotacji}

\exercise %13.2-1
Oto pseudokod prawej rotacji na drzewie czerwono-czarnym:
\begin{codebox}
\Procname{$\proc{Right-Rotate}(T,x)$}
\li	$y\gets\attrib{x}{left}$
\li	$\attrib{x}{left}\gets\attrib{y}{right}$
\li	$\attrib{\attrib{y}{right}}{p}\gets x$
\li	$\attrib{y}{p}\gets\attrib{x}{p}$
\li	\If $\attrib{x}{p}=\attrib{T}{nil}$
\li		\Then $\attrib{T}{root}\gets y$
\li		\Else
			\If $x=\attrib{\attrib{x}{p}}{left}$
\li				\Then $\attrib{\attrib{x}{p}}{left}\gets y$
\li				\Else $\attrib{\attrib{x}{p}}{right}\gets y$
				\End
		\End
\li	$\attrib{y}{right}\gets x$
\li	$\attrib{x}{p}\gets y$
\end{codebox}

\exercise %13.2-2
Na dowolnym węźle $x$ w~drzewie binarnym $T$ można wykonać lewą rotację, o~ile prawy syn $x$ jest różny od \attrib{T}{nil} oraz prawą rotację, o~ile lewy syn jest różny od \attrib{T}{nil}.
Oznacza to, że istnieje wzajemna jednoznaczność między rotacjami a~krawędziami drzewa, wokół których można przeprowadzić rotacje.
W~drzewie o~$n$ węzłach jest $n-1$ krawędzi, a~więc tyle samo rotacji.

\exercise %13.2-3
Zauważmy, że wewnętrzna struktura węzłów w~poddrzewach $\alpha$, $\beta$, $\gamma$ nie zmienia się w~wyniku wykonania rotacji.
A~zatem głębokość dowolnego węzła $a$ z~$\alpha$ zwiększa się o~1, głębokość dowolnego węzła $b$ z~$\beta$ nie zmienia się, a~głębokość dowolnego węzła $c$ z~$\gamma$ zmniejsza się o~1.

\exercise %13.2-4
\exercise %13.2-5
