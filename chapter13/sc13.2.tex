\subchapter{Operacje rotacji}

\exercise %13.2-1
Oto pseudokod prawej rotacji na drzewie czerwono-czarnym:
\begin{codebox}
\Procname{$\proc{Right-Rotate}(T,x)$}
\li	$y\gets\attrib{x}{left}$
\li	$\attrib{x}{left}\gets\attrib{y}{right}$
\li	$\attrib{\attrib{y}{right}}{p}\gets x$
\li	$\attrib{y}{p}\gets\attrib{x}{p}$
\li	\If $\attrib{x}{p}=\attrib{T}{nil}$
\li		\Then $\attrib{T}{root}\gets y$
\li		\Else
			\If $x=\attrib{\attrib{x}{p}}{left}$
\li				\Then $\attrib{\attrib{x}{p}}{left}\gets y$
\li				\Else $\attrib{\attrib{x}{p}}{right}\gets y$
				\End
		\End
\li	$\attrib{y}{right}\gets x$
\li	$\attrib{x}{p}\gets y$
\end{codebox}

\exercise %13.2-2
Na dowolnym węźle $x$ w~drzewie binarnym $T$ można wykonać lewą rotację, o~ile prawy syn $x$ jest różny od \attrib{T}{nil} oraz prawą rotację, o~ile lewy syn jest różny od \attrib{T}{nil}.
Oznacza to, że istnieje wzajemna jednoznaczność między rotacjami a~krawędziami drzewa, wokół których można przeprowadzić rotacje.
W~drzewie o~$n$ węzłach jest dokładnie $n-1$ krawędzi i~tyle samo różnych rotacji można w~nim wykonać.

\exercise %13.2-3
Zauważmy, że wewnętrzna struktura węzłów w~poddrzewach $\alpha$, $\beta$, $\gamma$ nie zmienia się w~wyniku wykonania rotacji.
A~zatem głębokość dowolnego węzła $a$ z~$\alpha$ zwiększa się o~1, głębokość dowolnego węzła $b$ z~$\beta$ nie zmienia się, a~głębokość dowolnego węzła $c$ z~$\gamma$ zmniejsza się o~1.

\exercise %13.2-4
\note{Rotacje, o~których mówimy w~rozwiązaniu są operacjami na drzewach binarnych, których szczególną postać dla drzew czerwono-czarnych implementują procedury \proc{Left-Rotate} i~\proc{Right-Rotate}.
Ponadto doprecyzujemy pojęcie (prawego) łańcucha z~treści zadania -- jest to drzewo binarne, w~którym lewe poddrzewo każdego węzła jest puste.}

\noindent Wykażemy stwierdzenie podane we wskazówce w~treści zadania.
Dla danego drzewa binarnego $T$ o~$n$ węzłach, niech $R_T$ będzie zbiorem zawierającym korzeń drzewa $T$ oraz wszystkie węzły osiągalne z~korzenia podczas poruszania się wyłącznie w~kierunku prawych synów (czyli po wskaźnikach \id{right}).
Dopóki drzewo $T$ nie jest prawym łańcuchem, czyli dopóki są w~nim węzły nienależące do $R_T$, to istnieje węzeł $y\in R_T$, którego lewy syn $x\not\in R_T$.
Zauważmy, że wykonanie prawej rotacji na węźle $y$ dodaje $x$ do $R_T$, jednocześnie nie usuwając z~niego żadnego innego węzła.
Prawa rotacja zwiększa więc rozmiar $R_T$ o~1, dlatego wystarczy wykonać co najwyżej $n-1$ rotacji, aby przekształcić $T$ w~prawy łańcuch.

Jeśli znamy ciąg prawych rotacji przekształcających drzewo $T$ w~prawy łańcuch $T'$, to możemy powrócić od $T'$ do $T$ poprzez wykonanie tego ciągu w~odwrotnej kolejności, zamieniając każdą prawą rotację w~odpowiadającą jej lewą rotację.

Niech $T_1$, $T_2$ będą drzewami binarnymi o~$n$ węzłach, a~$T'$ -- jedynym prawym łańcuchem powstałym z~$T_1$ bądź z~$T_2$ po przeprowadzeniu powyżej opisanej transformacji.
Niech $r=\langle r_1,r_2,\dots,r_k\rangle$ i~$r'=\langle r_1',r_2',\dots,r_{k'}'\rangle$ będą ciągami prawych rotacji przekształcających, odpowiednio $T_1$ w~$T'$ i~$T_2$ w~$T'$.
Z~poprzedniego paragrafu wiemy, że istnieją takie ciągi $r$, $r'$, gdzie $k$, $k'\le n-1$.
Dla każdej prawej rotacji $r_i'$ niech $l_i'$ będzie odpowiadającą jej lewą rotacją.
Wówczas ciąg $\langle r_1,r_2,\dots,r_k,l_{k'}',l_{k'-1}',\dots,l_1'\rangle$ co najwyżej $2n-2$ rotacji pozwala przekształcić drzewo $T_1$ w~$T_2$.

\exercise %13.2-5
