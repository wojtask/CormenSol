\problem{Zbiory dynamiczne z~historią} %13-1

\subproblem %13-1(a)
Gdy do drzewa wstawiany jest klucz $k$, zmianie ulegają wszystkie węzły na prostej ścieżce od korzenia drzewa do nowego węzła z~kluczem $k$, który staje się nowym liściem drzewa.

W~operacji usuwania, niech $y$ będzie rzeczywiście usuniętym węzłem, a~$z$ -- argumentem operacji.
Jeśli $z$ ma co najwyżej jednego syna, to $y=z$ i~będzie on usunięty albo wycięty, tak jak to ilustruje rys.\ 12.4, części (a) i~(b), w~Podręczniku.
Zaktualizowani zostaną wszyscy właściwi przodkowie $y$.
Gdy $z$ ma dwóch synów, to jego następnik $y$ będzie wycięty i~przeniesiony w~miejsce $z$ (patrz rys.\ 12.4(c) z~Podręcznika).
Podobnie jak wcześniej zmianie ulegną tylko właściwi przodkowie $y$ (bo $y$ początkowo był potomkiem $z$).
Potomkowie węzła $y$ nie zmienią się, bo założyliśmy, że węzły nie mają wskaźników do ojca.

\subproblem %13-1(b)
Zdefiniujmy dwie pomocnicze operacje, które wykorzystane zostaną w~procedurze:
\begin{itemize}
\item $\proc{New-Node}(k)$ -- tworzy nowy węzeł, którego pole \id{key} ma wartość $k$, a~pola \id{left} i~\id{right} są ustawione na \const{nil}; zwraca wskaźnik do nowo utworzonego węzła;
\item $\proc{Copy-Node}(x)$ -- tworzy nowy węzeł, którego pola \id{key}, \id{left} oraz \id{right} mają identyczne wartości jak odpowiadające pola węzła $x$; zwraca wskaźnik do nowo utworzonego węzła.
\end{itemize}

Główną pracę wykona rekurencyjna procedura \proc{Persistent-Subtree-Insert} przyjmująca na wejściu dwa parametry -- korzeń $x$ drzewa, do którego wstawiany jest nowy węzeł, oraz klucz $k$ nowego węzła.
Będzie ona kopiować ścieżkę od $x$ do jednego z~jego potomnych liści, a~na końcu kopii tej ścieżki umieści nowy węzeł o~kluczu $k$.
Wynikem zwracanym przez tę procedurę będzie nowo wstawiony węzeł albo kopia węzła $x$ będąca jego przodkiem.
\begin{codebox}
\Procname{$\proc{Persistent-Subtree-Insert}(x,k)$}
\li	\If $x=\const{nil}$
\li		\Then $z\gets\proc{New-Node}(k)$
\li		\Else $z\gets\proc{Copy-Node}(x)$
\li			\If $k<\attrib{x}{key}$
\li				\Then $\attrib{z}{left}\gets\proc{Persistent-Subtree-Insert}(\attrib{x}{left},k)$
\li				\Else $\attrib{z}{right}\gets\proc{Persistent-Subtree-Insert}(\attrib{x}{right},k)$
				\End
		\End
\li	\Return $z$		
\end{codebox}

Procedura \proc{Persistent-Tree-Insert} przyjmuje na wejściu drzewo z~historią $T$ oraz klucz $k$ i~zwraca nowe drzewo z~historią $T'$ powstałe przez dodanie do $T$ klucza $k$.
\begin{codebox}
\Procname{$\proc{Persistent-Tree-Insert}(T,k)$}
\li	utwórz puste drzewo z~historią $T'$
\li	$\attrib{T'}{root}\gets\proc{Persistent-Subtree-Insert}(\attrib{T}{root},k)$
\li	\Return $T'$
\end{codebox}

\subproblem %13-1(c)
Każde kolejne wywołanie rekurencyjne procedury \proc{Persistent-Subtree-Insert} schodzi o~1 poziom w~dół drzewa $T$.
Ścieżka pokonywana przez tę procedurę, wywołaną z~\proc{Persistent-Tree-Insert}, ma długość co najwyżej $h$.
A~zatem, przy założeniu, że operacje \proc{New-Node} oraz \proc{Copy-Node} działają w~czasie stałym, czas potrzebny do wstawienia nowego klucza do $T$ wynosi $O(h)$.

Z~punktu (a) mamy, że podczas wstawiania nowego klucza do $T$ w~procedurze \proc{Persistent-Subtree-Insert} skopiowany zostanie każdy przodek nowego węzła.
Złożoność pamięciowa operacji wstawiania wynosi więc także $O(h)$.

\subproblem %13-1(d)
Jeśli w~każdym węźle byłby przechowywany wskaźnik na ojca, to skopiowanie korzenia drzewa podczas wstawiania nowego węzła pociągałoby za sobą konieczność skopiowania obydwu synów korzenia, a~to z~kolei konieczność skopiowania także ich synów -- i~tak dalej, aż do liści drzewa.
A~zatem użycie wskaźników na ojca wymagałoby wykonania kopii każdego węzła drzewa.
Dla drzewa o~$n$ węzłach operacja \proc{Persistent-Tree-Insert} potrzebowałaby więc zarówno czasu, jak i~dodatkowej pamięci rzędu $\Omega(n)$.

\subproblem %13-1(e)
Jak zobaczyliśmy w~punkcie (c), operacja wstawiania do drzewa z~historią działa w~czasie proporcjonalnym do wysokości tego drzewa.
Jeśli do implementacji drzew z~historią użylibyśmy drzew czerwono-czarnych, to moglibyśmy zagwarantować niski czas działania tej operacji.
W~implementacji tej nie możemy jednak utrzymywać w~węzłach pola $p$, gdyż wtedy bylibyśmy zmuszeni do kopiowania całego drzewa przy wstawianiu jednego węzła (patrz punkt (d)).
Możliwe jest jednak zaimplementowanie operacji wstawiania do drzewa czerwono-czarnego, niewykorzystującego wskaźników na ojca, poprzez użycie stosu -- szczegóły zostały opisane w~\refExercise{13.3-6}.
Możemy zmodyfikować nieco implementację podaną w~tamtym rozwiązaniu tak, aby zamiast dokonywać zmian na drzewie wejściowym, odpowiednie węzły były kopiowane, a~na końcu działania, by zwracane było nowe drzewo uzupełnione o~nowy węzeł, podobnie jak w~\proc{Persistent-Tree-Insert} z~punktu (b).

Wystarczy jeszcze pokazać, że w~trakcie wstawiania do drzewa z~historią $T$ o~$n$ węzłach opisana wersja procedury \proc{RB-Insert} kopiuje nie więcej niż $O(\lg n)$ węzłów w~wyniku wykonywania rotacji i~zmian kolorów.
Niech $s$ będzie ścieżką złożoną ze skopiowanych węzłów, zanim została wywołana adaptacja procedury \proc{RB-Insert-Fixup} -- czyli od kopii korzenia do nowo wstawionego węzła.
W~wywołaniu tym wykonywane są co najwyżej 2 rotacje, a~każda z~nich modyfikuje wskaźniki do synów 3 węzłów -- tego, wokół którego rotacja jest wykonywana, jego ojca oraz jednego z~synów.
Wszystkie one leżą jednak na ścieżce $s$, dlatego zostały skopiowane jeszcze wewnątrz \proc{RB-Insert}.

W~zmodyfikowanej procedurze \proc{RB-Insert-Fixup} w~przypadku 1 zmieniany jest kolor dziadka (ojca ojca) aktualnego węzła oraz synowie dziadka.
Zarówno dziadek, jak i~jeden z~jego synów znajdują się na ścieżce $s$ -- należy więc wykonać kopię drugiego syna (stryja aktualnego węzła).
Pętla tej procedury może sumarycznie wykonać $O(\lg n)$ iteracji i~tyle samo węzłów może wymagać wykonania ich kopii z~powodu aktualizacji kolorów.

Widzimy zatem, że wstawianie do drzewa z~historią w~opisywanej reprezentacji wymaga czasu $O(\lg n)$, gdzie $n$ jest liczbą węzłów w~tym drzewie.

Do usuwania węzła z~drzewa z~historią reprezentowanego przez drzewo czerwono-czarne możemy podać wersję procedury \proc{RB-Delete} z~modyfikacjami podobnymi do tych opisanych dla wstawiania.
Skopiowane węzły sprzed wywołania adaptacji \proc{RB-Delete-Fixup} -- czyli z~punktu (a) właściwych przodków efektywnie usuniętego węzła -- można przechować na stosie, co pozwoli dostać się do nich bez użycia w~tym celu wskaźników na ojca.
W~każdej iteracji pętli \kw{while} podczas tego wywołania potrzebni są tylko dwaj właściwi przodkowie $x$ -- jego ojciec i~ojciec ojca.
Węzły te przed każdą iteracją można pobierać ze stosu, a~podczas samej iteracji należy pamiętać o~poprawnym ich aktualizowaniu, także bezpośrednio po rotacjach na ojcu $x$.

Analogicznie do analizy operacji wstawiania można sprawdzić, że każda iteracja pętli zmodyfikowanej procedury \proc{RB-Delete-Fixup} zaktualizuje $O(1)$ węzłów przez wykonane rotacje i~modyfikacje kolorów.
Sumarycznie procedura ta skopiuje zatem dodatkowo co najwyżej $O(\lg n)$ węzłów, dlatego czas działania operacji usuwania także wynosi $O(\lg n)$.
