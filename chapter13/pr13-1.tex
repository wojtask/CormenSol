\problem{Zbiory dynamiczne z~historią} %13-1

\subproblem %13-1(a)
Podczas wstawiania nowego klucza $k$ zmianie muszą ulec wszystkie węzły na ścieżce od korzenia drzewa do nowego węzła z~kluczem $k$, który staje się nowym liściem drzewa.

Gdy usuwamy węzeł $z$, to rzeczywiście usuwany jest węzeł $y$, który jest równy $z$, jeśli $z$ posiada co najwyżej jednego syna, albo jest następnikiem $z$ w~przypadku, gdy $z$ ma dwóch synów.
Aktualizowane są wtedy wszyscy przodkowie $y$.

\subproblem %13-1(b)
Zdefiniujemy dwie pomocnicze operacje, które wykorzystamy w~procedurze:
\begin{itemize}
\item $\proc{New-Node}(k)$ -- tworzy nowy węzeł, którego pole \id{key} ma wartość $k$, a~pola \id{left} i~\id{right} są ustawione na \const{nil}; zwraca wskaźnik do nowo utworzonego węzła;
\item $\proc{Copy-Node}(x)$ -- tworzy nowy węzeł, którego pola \id{key}, \id{left} oraz \id{right} mają identyczne wartości jak odpowiadające pola węzła $x$; zwraca wskaźnik do nowo utworzonego węzła.
\end{itemize}

Poniższa rekurencyjna procedura \proc{Persistent-Tree-Insert} wywoływana jest z~dwoma parametrami -- korzeniem $x$ drzewa, do którego wstawiany jest nowy węzeł, oraz kluczem $k$ nowego węzła.
Węzły na ścieżce w~dół drzewa o~korzeniu w~$x$ są kopiowane, a~na końcu ścieżki złożonej ze skopiowanych węzłów umieszczany jest nowy węzeł o~kluczu $k$.
\begin{codebox}
\Procname{$\proc{Persistent-Tree-Insert}(x,k)$}
\li	\If $x=\const{nil}$
\li		\Then $z\gets\proc{New-Node}(k)$
\li		\Else $z\gets\proc{Copy-Node}(x)$
\li			\If $k<\attrib{x}{key}$
\li				\Then $\attrib{z}{left}\gets\proc{Persistent-Tree-Insert}(\attrib{x}{left},k)$
\li				\Else $\attrib{z}{right}\gets\proc{Persistent-Tree-Insert}(\attrib{x}{right},k)$
				\End
		\End
\li	\Return $z$		
\end{codebox}
Aby wstawić klucz $k$ do drzewa z~historią $T$, należy użyć wywołania $\proc{Persistent-Tree-Insert}(\attrib{T}{root},k)$, które zwróci wskaźnik na korzeń wynikowego drzewa $T'$.

\subproblem %13-1(c)
Każde kolejne wywołanie rekurencyjne procedury \proc{Persistent-Tree-Insert} schodzi o~1 poziom w~dół drzewa.
Ścieżka pokonywana przez procedurę ma długość co najwyżej $h$.
A~zatem, przy założeniu, że operacje \proc{New-Node} oraz \proc{Copy-Node} działają w~czasie stałym, czas działania procedury wynosi $O(h)$.

Z~punktu (a) mamy, że podczas wstawiania nowego węzła procedura \proc{Persistent-Tree-Insert} skopiuje każdy węzeł znajdujący się na ścieżce od korzenia do nowego węzła.
Złożoność pamięciowa procedury wynosi więc także $O(h)$.

\subproblem %13-1(d)
Jeśli w~każdym węźle byłby przechowywany wskaźnik na ojca, to skopiowanie korzenia drzewa podczas wstawiania nowego węzła sprawiłoby, że obaj synowie korzenia musieliby również zostać skopiowani, a~wówczas także ich synowie i~tak dalej, aż do liści drzewa.
A~zatem obecność wskaźników na ojca wymaga skopiowania każdego węzła w~drzewie przez procedurę \proc{Persistent-Tree-Insert}.
W~drzewie o~$n$ węzłach działałaby więc ona w~czasie $\Omega(n)$ i~potrzebowałaby $\Omega(n)$ dodatkowej pamięci.

\subproblem %13-1(e)
