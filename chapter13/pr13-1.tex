\problem{Zbiory dynamiczne z~historią} %13-1

\subproblem %13-1(a)
Gdy do drzewa wstawiany jest klucz $k$, zmianie ulegają wszystkie węzły na prostej ścieżce od korzenia drzewa do nowego węzła z~kluczem $k$, który staje się nowym liściem drzewa.

Podczas usuwania węzła $z$, z~drzewa rzeczywiście wycinany jest węzeł $y$, który jest równy $z$, gdy ten posiada co najwyżej jednego syna, albo jest następnikiem $z$, w~przypadku gdy ten ma dwóch synów.
Aktualizowani są wtedy wszyscy właściwi przodkowie $y$.

\subproblem %13-1(b)
Zdefiniujmy dwie pomocnicze operacje, z~których będziemy korzystać:
\begin{itemize}
\item $\proc{New-Node}(k)$ -- tworzy nowy węzeł, którego pole \id{key} ma wartość $k$, a~pola \id{left} i~\id{right} są ustawione na \const{nil}; zwraca wskaźnik do nowo utworzonego węzła;
\item $\proc{Copy-Node}(x)$ -- tworzy nowy węzeł, którego pola \id{key}, \id{left} oraz \id{right} mają identyczne wartości jak odpowiadające pola węzła $x$; zwraca wskaźnik do nowo utworzonego węzła.
\end{itemize}

Poniższa rekurencyjna procedura \proc{Persistent-Subtree-Insert} wywoływana jest z~dwoma parametrami -- korzeniem $x$ drzewa, do którego wstawiany jest nowy węzeł, oraz kluczem $k$ nowego węzła.
Ścieżka od $x$ do jednego z~jego potomnych liści jest kopiowana, a~na końcu kopii tej ścieżki umieszczany jest nowy węzeł o~kluczu $k$.
\begin{codebox}
\Procname{$\proc{Persistent-Subtree-Insert}(x,k)$}
\li	\If $x=\const{nil}$
\li		\Then $z\gets\proc{New-Node}(k)$
\li		\Else $z\gets\proc{Copy-Node}(x)$
\li			\If $k<\attrib{x}{key}$
\li				\Then $\attrib{z}{left}\gets\proc{Persistent-Subtree-Insert}(\attrib{x}{left},k)$
\li				\Else $\attrib{z}{right}\gets\proc{Persistent-Subtree-Insert}(\attrib{x}{right},k)$
				\End
		\End
\li	\Return $z$		
\end{codebox}
Wynikiem zwracanym przez tę procedurę jest nowo wstawiony węzeł albo kopia węzła $x$ będąca jego przodkiem.

Następujący pseudokod przyjmuje na wejściu drzewo z~historią $T$ oraz klucz $k$ i~zwraca nowe drzewo z~historią $T'$ powstałe przez dodanie do $T$ klucza $k$.
\begin{codebox}
\Procname{$\proc{Persistent-Tree-Insert}(T,k)$}
\li	utwórz puste drzewo z~historią $T'$
\li	$\attrib{T'}{root}\gets\proc{Persistent-Subtree-Insert}(\attrib{T}{root},k)$
\li	\Return $T'$
\end{codebox}

\subproblem %13-1(c)
Każde kolejne wywołanie rekurencyjne procedury \proc{Persistent-Subtree-Insert} schodzi o~1 poziom w~dół drzewa $T$.
Ścieżka pokonywana przez tę procedurę, wywołaną z~\proc{Persistent-Tree-Insert}, ma długość co najwyżej $h$.
A~zatem, przy założeniu, że operacje \proc{New-Node} oraz \proc{Copy-Node} działają w~czasie stałym, czas potrzebny do wstawienia nowego klucza do $T$ wynosi $O(h)$.

Z~punktu (a) mamy, że podczas wstawiania nowego klucza do $T$ w~procedurze \proc{Persistent-Subtree-Insert} skopiowany zostanie każdy przodek nowego węzła.
Złożoność pamięciowa operacji wstawiania wynosi więc także $O(h)$.

\subproblem %13-1(d)
Jeśli w~każdym węźle byłby przechowywany wskaźnik na ojca, to skopiowanie korzenia drzewa podczas wstawiania nowego węzła pociągałoby za sobą konieczność skopiowania obydwu synów korzenia, a~to z~kolei konieczność skopiowania także ich synów -- i~tak dalej, aż do liści drzewa.
A~zatem obecność wskaźników na ojca wymaga wykonania kopii każdego węzła w~drzewie w~procedurze \proc{Persistent-Subtree-Insert}.
Dla drzewa o~$n$ węzłach operacja \proc{Persistent-Tree-Insert} potrzebowałaby więc czasu $\Omega(n)$ oraz $\Omega(n)$ dodatkowej pamięci.

\subproblem %13-1(e)
Jak zobaczyliśmy w~punkcie (c), operacja wstawiania do drzewa z~historią działa w~czasie proporcjonalnym do wysokości tego drzewa.
Jeśli do implementacji drzew z~historią użylibyśmy drzew czerwono-czarnych, to moglibyśmy zagwarantować niski czas działania tej operacji.
W~implementacji tej w~żadnym węźle nie możemy jednak przechowywać pola wskazującego na ojca, gdyż wtedy bylibyśmy zmuszeni do kopiowania całego drzewa przy wstawianiu jednego węzła (patrz punkt (d)).
Możliwe jest jednak zaimplementowanie operacji wstawiania do drzewa czerwono-czarnego nie wykorzystującego wskaźników na ojca, poprzez użycie stosu -- szczegóły zostały opisane w~\refExercise{13.3-6}.
Możemy zmodyfikować nieco implementację podaną w~tamtym rozwiązaniu tak, aby zamiast dokonywać zmian na drzewie wejściowym, odpowiednie węzły były kopiowane, a~na końcu działania, by zwracane było nowe drzewo uzupełnione o~nowy węzeł, podobnie jak w~\proc{Persistent-Tree-Insert} z~punktu (b).

Wystarczy jeszcze pokazać, że w~trakcie wstawiania do drzewa z~historią $T$ o~$n$ węzłach opisana wersja procedury \proc{RB-Insert} kopiuje nie więcej niż $O(\lg n)$ węzłów w~wyniku wykonywania rotacji i~zmian kolorów.
Niech $s$ będzie ścieżką złożoną ze skopiowanych węzłów, zanim została wywołana procedura \proc{RB-Insert-Fixup} -- czyli od kopii korzenia do nowo wstawionego węzła.
Procedura ta wykonuje co najwyżej 2 rotacje, a~każda z~nich modyfikuje wskaźniki do synów 3 węzłów -- tego, wokół którego rotacja jest wykonywana, jego ojca oraz jednego z~jego synów.
Wszystkie one stanowią jednak fragment ścieżki $s$, dlatego zostały skopiowane jeszcze wewnątrz \proc{RB-Insert}.

W~przypadku 1 w~procedurze \proc{RB-Insert-Fixup} zmieniany jest kolor dziadka (ojca ojca) aktualnego węzła oraz synowie dziadka.
Zarówno dziadek, jak i~jeden z~jego synów znajdują się na ścieżce $s$ -- należy więc wykonać kopię drugiego syna (stryja aktualnego węzła).
Jest to jednak jedyny dodatkowo kopiowany węzeł w~aktualnej iteracji pętli w~procedurze \proc{RB-Insert-Fixup}, ponieważ jego ojciec leży już na $s$.
Pętla może sumarycznie wykonać $O(\lg n)$ iteracji i~tyle samo węzłów może wymagać wykonania ich kopii z~powodu aktualizacji kolorów.

Widzimy zatem, że wstawianie do drzewa z~historią reprezentowanego przez drzewo czerwono-czarne o~$n$ węzłach bez wskaźników do ojca wymaga czasu $O(\lg n)$.

Do usuwania węzła z~drzewa z~historią reprezentowanego przez drzewo czerwono-czarne możemy podać wersję procedury \proc{RB-Delete} z~modyfikacjami podobnymi do tych opisanych dla wstawiania.
Skopiowane węzły sprzed wywołania \proc{RB-Delete-Fixup}, czyli z~punktu (a) właściwych przodków efektywnie usuniętego węzła (i~jego syna $x$ w~przypadku, gdy $\attrib{x}{color}=\const{red}$), można przechować na stosie, co pozwoli dostać się do nich bez użycia w~tym celu wskaźników na ojca.

Analogicznie do analizy operacji wstawiania można sprawdzić, że w~wywołaniu \proc{RB-Delete-Fixup} w~każdej iteracji pętli aktualizowanych jest $O(1)$ węzłów przez wykonane rotacje i~modyfikacje kolorów.
Sumarycznie procedura ta skopiuje zatem dodatkowo co najwyżej $O(\lg n)$ węzłów, dlatego czas działania operacji usuwania także wynosi $O(\lg n)$.
