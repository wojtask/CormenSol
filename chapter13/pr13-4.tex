\problem{Drzepce, czyli drzewa ,,treaps''} %13-4

\subproblem %13-4(a)
Dla danego zbioru węzłów $x_1$, $x_2$, \dots, $x_n$ może istnieć wiele możliwych drzew wyszukiwań binarnych zbudowanych z~tych węzłów.
Jeśli $\attrib{x_i}{key}<\attrib{x_j}{key}$, to węzeł $x_i$ może znajdować się w~lewym poddrzewie węzła $x_j$ albo $x_j$ może znajdować się w~prawym poddrzewie $x_i$.
Powiązanie z~każdym węzłem priorytetów i~utrzymywanie na ich podstawie własności kopca typu min w~drzepcu sprawia, że niejednoznaczność ta znika -- węzeł o~niższym priorytecie zajmuje w~drzepcu wyższy poziom od węzła o~wyższym priorytecie.
A~zatem, jeśli dodatkowo $\attrib{x_i}{priority}<\attrib{x_j}{priority}$, to węzeł $x_i$ jest przodkiem $x_j$, a~w~przeciwnym przypadku -- odwrotnie.
Dla każdej pary węzłów z~danego zbioru istnieje jednoznacznie wyznaczone ich wzajemne rozmieszczenie w~drzepcu, dlatego istnieje dokładnie jeden drzepiec dla tego zbioru węzłów i~powiązanych z~nimi priorytetów.

\subproblem %13-4(b)
\subproblem %13-4(c)
\subproblem %13-4(d)
\subproblem %13-4(e)
\subproblem %13-4(f)
\subproblem %13-4(g)
\subproblem %13-4(h)
\subproblem %13-4(i)
\subproblem %13-4(j)
