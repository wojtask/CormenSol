\problem{Drzewa AVL} %13-3

\subproblem %13-3(a)
Udowodnimy stwierdzenie ze wskazówki, wykorzystując indukcję po wysokościach drzew AVL.
Jako podstawę indukcji przyjmiemy drzewa AVL o~wysokościach 0 i~1.
W~drzewie o~wysokości 0 jest tylko jeden węzeł, czyli więcej niż $F_0=0$, zaś~drzewo o~wysokości 1 może składać się z~minimalnie 2 węzłów, co jest większe niż $F_1=1$.

Niech teraz dane będzie drzewo AVL o~wysokości $h\ge2$ i~niech $h_L$ będzie wysokością jego lewego poddrzewa, a~$h_R$ -- wysokością jego prawego poddrzewa.
Bez utraty ogólności możemy przyjąć, że $h_L\le h_R$, skąd $h=h_R+1$.
Na podstawie definicji drzewa AVL mamy, że $|h_R-h_L|\le1$, a~więc na mocy powyższego $h_L\ge h_R-1$.
Z~kolei z~założenia indukcyjnego mamy, że liczby węzłów $n_L$ i~$n_R$ w~lewym i~prawym poddrzewie wynoszą, odpowiednio, co najmniej $F_{h_L}$ i~co najmniej $F_{h_R}$.
Stąd liczba węzłów drzewa wynosi $n=n_L+n_R+1\ge F_{h_L}+F_{h_R}+1\ge F_{h_R-1}+F_{h_R}+1=F_{h_R+1}+1=F_h+1>F_h$.
Z~\refExercise{3.2-7} mamy $F_h\ge\phi^{h-2}$ dla każdego $h\ge2$, a~zatem $n>F_h\ge\phi^{h-2}$, skąd otrzymujemy ostatecznie $h<\log_\phi n+2=O(\lg n)$.

\subproblem %13-3(b)
\note{Do procedury \proc{Balance} konieczne jest przekazanie drzewa AVL jako dodatkowego parametru\/ $T$.}

\noindent Pseudokod procedury \proc{Balance} wygląda następująco:
\begin{codebox}
\Procname{$\proc{Balance}(T,x)$}
\li	\If $\attrib{x}{left}\ne\const{nil}$ i~$\attrib{x}{h}=\attribb{x}{left}{h}+1$
\li		\Then $y\gets\attrib{x}{left}$ \label{li:balance-left-cases-begin}
\li			\If $\attrib{y}{left}\ne\const{nil}$ i~$\attrib{y}{h}=\attribb{y}{left}{h}+1$
\li				\Then $\proc{Right-Rotate}(T,x)$ \>\>\>\>\>\>\>\Comment{Przypadek ,,lewo-lewo''}
\li					\Return $y$ \>\>\>\>\>\>\>\Comment{Przypadek ,,lewo-lewo''}
\li				\Else $z\gets\attrib{y}{right}$ \>\>\>\>\>\>\>\Comment{Przypadek ,,lewo-prawo''}
\li					$\proc{Left-Rotate}(T,y)$ \>\>\>\>\>\>\>\Comment{Przypadek ,,lewo-prawo''}
\li					$\proc{Right-Rotate}(T,z)$ \>\>\>\>\>\>\>\Comment{Przypadek ,,lewo-prawo''}
\li					\Return $z$ \>\>\>\>\>\>\>\Comment{Przypadek ,,lewo-prawo''}
				\End \label{li:balance-left-cases-end}
\li		\Else (to samo co po \kw{then} z~zamienionymi ,,right'' i~,,left'')
		\End
\end{codebox}

Dla każdego węzła $x$ w~drzewie AVL zdefiniujmy \attrib{x}{balance-factor} jako $-\attribb{x}{left}{h}+\attribb{x}{right}{h}$.
Nie będziemy jednak przechowywać tego atrybutu w~węźle -- posłuży on nam jedynie w~opisie działania algorytmu jako uproszczony zapis różnicy wysokości poddrzew węzła.

Procedura $\proc{Balance}(T,x)$ wywoływana jest, gdy $\attrib{x}{balance-factor}\in\{-2,2\}$, a~dla każdego potomka $y$ węzła $x$ zachodzi $\attrib{y}{balance-factor}\in\{-1,0,1\}$.
Rozważmy 4 przypadki w~zależności od wysokości poddrzew węzła $x$ oraz jego synów.
W~przypadku, który określimy jako ,,lewo-lewo'', spełnione jest $\attrib{x}{balance-factor}=-2$ oraz $\attribb{x}{left}{balance-factor}\in\{-1,0\}$.
Aby zrównoważyć $x$, wystarczy wykonać na nim prawą rotację.
Przypadek ,,lewo-prawo'' zachodzi wtedy, gdy $\attrib{x}{balance-factor}=-2$ i~$\attribb{x}{left}{balance-factor}=1$.
Do zrównoważenia $x$ potrzebne są dwie rotacje, najpierw lewa na \attrib{x}{left}, która sprowadza przypadek ,,lewo-prawo'' dla $x$ do przypadku ,,lewo-lewo'' dla \attrib{x}{left}, a~następnie prawa rotacja na \attrib{x}{left} wprowadzająca równowagę dla tego węzła.
Działanie procedury dla opisanych przypadków przedstawione zostało na rys.\ \ref{fig:13-3b}.
\begin{figure}[!ht]
	\centering \begin{tikzpicture}[
	level/.append style = {level distance=7mm, sibling distance=50mm/2^#1},
	every label/.style = {index node, inner sep=1pt, draw=none, fill=none},
	subtree node/.style = {tree node, draw=none, fill=none},
	outer/.append style = {node distance=7mm}
]

\node[outer] (pic a) {
\begin{tikzpicture}[anchor=center]
	\node[tree node, label=left:$h+3$] (left subtree root) {$x$}
    	child {node[tree node, label=left:$h+2$] {$y$}
        	child {node[subtree node, label=left:$h+1$] {$\alpha$}}
            child {node[subtree node, label=right:\text{$h$ lub $h+1$}] {$\beta$}}
        }
        child {node[subtree node, label=left:$h$] {$\gamma$}};
        
    \draw (left subtree root.north) -- +(0, 2mm);
            
	\node[right=50mm of left subtree root, tree node, label=right:\text{$h+2$ lub $h+3$}] (right subtree root) {$y$}
        child {node[subtree node, label=right:$h+1$] {$\alpha$}}
        child {node[tree node, label=right:\text{$h+1$ lub $h+2$}] {$x$}
            child {node[subtree node, label=left:\text{$h$ lub $h+1$}] {$\beta$}}
            child {node[subtree node, label=right:$h$] {$\gamma$}}
        };
            
    \draw (right subtree root.north) -- +(0, 2mm);
    \draw[arrow, shorten >=10mm, shorten <=10mm, dashed, thick] (left subtree root.east) -- (right subtree root.west);
    
    \node[subpicture label, left=30mm of left subtree root] {(a)};
\end{tikzpicture}
};

\node[outer, below=of pic a.south west, anchor=north west] (pic b) {
\begin{tikzpicture}[anchor=center]
	\node[tree node, label=left:$h+3$] (left tree) (left subtree root) {$x$}
    	child {node[tree node, label=left:$h+2$] {$y$}
        	child {node[subtree node, label=left:$h$] {$\alpha$}}
            child {node[tree node, label=right:$h+1$, label={[label distance=7mm]below:\text{$h-1$ lub $h$}}] {$w$}
            	child {node[subtree node] {$\beta$}}
            	child {node[subtree node] {$\gamma$}}
            }
        }
        child {node[subtree node, label=left:$h$] {$\delta$}};
        
    \draw (left subtree root.north) -- +(0, 2mm);
            
    \node[right=50mm of left subtree root, tree node, label=right:$h+2$, label={[label distance=14mm]below:\text{$h-1$ lub $h$}}] {$w$}
    	child {node[tree node, label=right:$h+1$] {$y$}
        	child {node[subtree node, label=left:$h$] {$\alpha$}}
            child {node[subtree node] {$\beta$}}
        }
        child {node[tree node, label=right:$h+1$] {$x$}
            child {node[subtree node] {$\gamma$}}
            child {node[subtree node, label=right:$h$] {$\delta$}}
        };

	\draw (right subtree root.north) -- +(0, 2mm);
    \draw[arrow, shorten >=10mm, shorten <=10mm, dashed, thick] (left subtree root.east) -- (right subtree root.west);
    
    \node[subpicture label, left=30mm of left subtree root] {(b)};
\end{tikzpicture}
};

\end{tikzpicture}

	\caption{Działanie procedury \proc{Balance} wywołanej dla węzła $x$, który nie jest zrównoważony po wysokościach.
	Obok węzłów $x$, $y$, $z$ i~poddrzew $\alpha$, $\beta$, $\gamma$, $\delta$ zaznaczone zostały ich wysokości.
	{\sffamily\bfseries(a)} Przypadek ,,lewo-lewo''.
	Jeśli przyjmiemy, że wysokością poddrzewa $\gamma$ jest $h$, to wysokość poddrzewa o~korzeniu w~$y$ wynosi $h+2$.
	Węzeł $y$ jest zrównoważony i~$\attrib{y}{balance-factor}\in\{-1,0\}$, więc poddrzewo $\alpha$ ma wysokość $h+1$, a~poddrzewo $\beta$ może mieć wysokość równą $h$ lub $h+1$.
	Drzewo $T$ po lewej stronie rysunku zostaje przekształcone w~zrównoważone drzewo po prawej stronie rysunku po wykonaniu $\proc{Right-Rotate}(T,x)$.
	{\sffamily\bfseries(b)} W~przypadku ,,lewo-prawo'' po przyjęciu, że wysokością $\delta$ jest $h$, mamy, że wysokość poddrzewa o~korzeniu w~węźle $y$ wynosi $h+2$.
	Tutaj jednak $\attrib{y}{balance-factor}=1$, więc wyróżniamy $z$ -- prawego syna $y$ o~wysokości $h+1$, którego oba poddrzewa, $\beta$ i~$\gamma$, mają wysokość $h-1$ lub $h$.
	Zrównoważenie drzewa $T$ odbywa się poprzez wywołanie najpierw $\proc{Left-Rotate}(T,y)$, a~następnie $\proc{Right-Rotate}(T,z)$.} \label{fig:13-3b}
\end{figure}

Pozostałe przypadki, ''prawo-prawo'' oraz ,,prawo-lewo'', są symetryczne do poprzednich, gdzie tym razem $\attrib{x}{balance-factor}=2$.
Aby je obsłużyć, wystarczy w~wierszach \doubledash{\ref{li:balance-left-cases-begin}}{\ref{li:balance-left-cases-end}} zamienić lewych synów z~prawymi i~odwrócić kierunki rotacji.

\subproblem %13-3(c)
Idea działania operacji \proc{AVL-Insert} opiera się na procedurze \proc{Recursive-Tree-Insert} z~\refExercise{12.3-1}.
Po wstawieniu węzła $z$ do niepustego drzewa AVL o~korzeniu w~$x$, węzeł $x$ może mieć zaburzoną równowagę po wysokościach.
Można to naprawić, wywołując procedurę \proc{Balance} opisaną w~części (b).
Wynikiem zwracanym przez \proc{AVL-Insert} jest aktualny korzeń drzewa, do którego wstawiane było $z$.
Pseudokod tej operacji został przedstawiony poniżej:
\begin{codebox}
\Procname{$\proc{AVL-Insert}(x,z)$}
\li	\If $x=\const{nil}$
\li		\Then \Return $z$
		\End
\li	\If $\attrib{z}{key}<\attrib{x}{key}$
\li		\Then $\attrib{x}{left}\gets\proc{AVL-Insert}(\attrib{x}{left},z)$
\li			$\attribb{x}{left}{p}\gets x$
\li		\Else $\attrib{x}{right}\gets\proc{AVL-Insert}(\attrib{x}{right},z)$
\li			$\attribb{x}{right}{p}\gets x$
		\End
\li	\Return $\proc{Balance}(x)$
\end{codebox}

Podobnie jak \proc{Recursive-Tree-Insert}, z~powyższej procedury nie należy korzystać bezpośrednio, ale poprzez wywołanie poniższego pseudokodu, który aktualizuje pole \id{root} drzewa $T$, do którego wstawiany jest węzeł $z$.
\begin{codebox}
\Procname{$\proc{AVL-Insert}'(T,z)$}
\li	$\attrib{T}{root}\gets\proc{AVL-Insert}(\attrib{T}{root},z)$
\end{codebox}

\subproblem %13-3(d)
