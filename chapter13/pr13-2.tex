\problem{Złączanie drzew czerwono-czarnych} %13-2

\subproblem %13-2(a)
Podczas wstawiania nowego węzła do drzewa $T$ czarna wysokość $T$ może zmienić się tylko wtedy, gdy czerwony korzeń $T$ zostaje pokolorowany na czarno.
Można więc w~procedurze \proc{RB-Insert-Fixup} bezpośrednio przed ostatnim wierszem inkrementować wartość \attrib{T}{bh} w~przypadku, gdy $\attribb{T}{root}{color}=\const{red}$.

W~procedurze \proc{RB-Delete-Fixup} wszystkie przypadki zachowują własność 5, o~ile przyjmiemy, że węzeł aktualnie pokazywany przez $x$ wnosi dodatkową ,,czarną jednostkę'' (patrz \refExercise{13.4-5}).
Czarna wysokość drzewa $T$ w~tej procedurze może być więc zmodyfikowana tylko wtedy, gdy nadmiarowa ,,czarna jednostka'' jest przenoszona na korzeń drzewa i~,,zapominana'' w~przypadku 2.
Wystarczy zatem bezpośrednio po linii 11 (a~także w~analogicznym miejscu w~symetrycznej sytuacji reprezentowanej przez linię 22) zmniejszyć \attrib{T}{bh} o~1, gdy $x=\attrib{T}{root}$.
W~przypadku 4 wskaźnik $x$ także jest ustawiany na korzeń, ale zanim zostanie wykonane przypisanie z~wiersza 21, w~drzewie zachowana jest już własność 5 bez potraktowania węzła wskazywanego przez $x$ jako ,,nadmiarowo'' czarnego.
Przypisanie to służy tylko do przerwania pętli.

Czarna wysokość drzewa $T$ może się zmienić także w~sytuacji, gdy z~$T$ usuwany jest jego jedyny węzeł wewnętrzny.
Wówczas w~procedurze \proc{RB-Delete-Fixup} nie wykonuje się żadna iteracja pętli \kw{while}, więc sytuację tę należy wykrywać wcześniej, np.\ w~wierszu 9 procedury \proc{RB-Delete}.

Czarna wysokość korzenia drzewa $T$ to oczywiście \attrib{T}{bh}.
Czerwone węzły mają te same czarne wysokości co ich ojcowie, natomiast czarne węzły (z~wyjątkiem korzenia) -- o~1 mniejsze od czarnych wysokości swoich ojców.
A~zatem, podążając dowolną ścieżką od korzenia do liścia w~drzewie $T$, możemy wyznaczyć czarne wysokości wszystkich węzłów na tej ścieżce w~czasie $O(1)$ na każdy odwiedzony węzeł -- zaczynamy od \attrib{T}{bh} dla korzenia i~zmniejszamy tę liczbę o~1 za każdym razem, gdy napotykamy czarny węzeł.

\subproblem %13-2(b)
Jak wyjaśnimy w~punkcie (f), w~celu zaprojektowania efektywnej operacji złączania drzew czerwono-czarnych, musimy zrezygnować w~ich implementacji z~wartownika, a~zamiast niego korzystać z~wartości \const{nil}, podobnie jak w~zwykłych drzewach wyszukiwań binarnych.

Poniższy pseudokod realizuje opisany w~treści algorytm, przy założeniu, że drzewa czerwono-czarne zaimplementowane są bez użycia wartownika.
\begin{codebox}
\Procname{$\proc{RB-Join-Point}(T_1,T_2)$}
\li	$y\gets\attrib{T_1}{root}$ \label{li:rb-join-point-initial-node}
\li	$b\gets\attrib{T_1}{bh}$
\li	\While $b>\attrib{T_2}{bh}$ \label{li:rb-join-point-while-begin}
\li		\Do $y\gets\attrib{y}{right}$ \label{li:rb-join-point-next}
\li			\If $y=\const{nil}$ lub $\attrib{y}{color}=\const{black}$ \label{li:rb-join-point-z-color}
\li				\Then $b\gets b-1$
				\End
		\End
\li	\Return $y$
\end{codebox}
Algorytm przechodzi drzewo $T_1$ po jego prawym kręgosłupie, gdzie znajdują się węzły o~największych kluczach z~każdej czarnej wysokości występującej w~drzewie.
Algorytm kończy działanie, gdy czarna wysokość $b$ odwiedzanego węzła $y$, wyznaczona na podstawie obserwacji z~punktu~(a), zrówna się z~wartością \attrib{T_2}{bh}.
Zwracany jest wówczas czarny węzeł wskazywany aktualnie przez $y$, przy czym wskaźnik \const{nil} jest tu traktowany jak wirtualny czarny węzeł zastępujący wartownik \attrib{T_1}{nil}.

W~najgorszym przypadku mogą zostać odwiedzone wszystkie węzły z~prawego kręgosłupa drzewa.
Ponieważ $T_1$ jest drzewem czerwono-czarnym o~co najwyżej $n$ węzłach, to czas działania algorytmu można ograniczyć od góry przez $O(\lg n)$.

\subproblem %13-2(c)
Na podstawie założenia klucz korzenia $y$ drzewa $T_y$, będącego poddrzewem $T_1$, jest mniejszy lub równy od klucza węzła $x$.
Można zatem, bez naruszenia własności drzewa wyszukiwań binarnych, uczynić $x$ prawym synem węzła \attrib{y}{p} (przy założeniu, że $y$ nie jest korzeniem drzewa $T_1$), a~drzewo $T_y$ -- lewym poddrzewem węzła $x$.
Korzystając ponownie z~założenia, dochodzimy do wniosku, że $T_2$ może być prawym poddrzewem $x$.
Wykonując opisane modyfikacje, jesteśmy w~stanie zastąpić poddrzewo $T_y$ przez $T_y\cup\{x\}\cup T_2$ w~czasie $O(1)$.

\subproblem %13-2(d)
Jeżeli pokolorujemy $x$ na czerwono, to czarna wysokość węzła \attrib{x}{p} (o~ile $x\ne\attrib{T_1}{root}$) nie zmieni się i~czerwono-czarne własności 1, 3 i~5 pozostaną spełnione.
W~przypadku gdy $x=\attrib{T_1}{root}$, to naruszona będzie własność 2, a~jeśli $x\ne\attrib{T_1}{root}$ i~węzeł \attrib{x}{p} też jest czerwony, to spowoduje to naruszenie własności 4.
Z~identyczną sytuacją mieliśmy do czynienia podczas wstawiania węzła do drzewa czerwono-czarnego przy pomocy procedury \proc{RB-Insert}.
Wystarczy więc wywołać $\proc{RB-Insert-Fixup}(T,x)$, aby przywrócić naruszone własności 2 i~4 w~powstałym drzewie $T$ w~czasie $O(\lg n)$.

\subproblem %13-2(e)
W~sytuacji symetrycznej algorytm z~punktu (b) będzie wyszukiwał w~drzewie $T_2$ czarny węzeł $y$ o~możliwie najmniejszym kluczu spośród węzłów o~czarnej wysokości \attrib{T_1}{bh}, poruszając się w~dół drzewa po jego lewym kręgosłupie.
Węzeł $x$ należy następnie umieścić w~drzewie $T_2$ jako lewego syna węzła \attrib{y}{p} (o~ile $y\ne\attrib{T_2}{root}$), drzewo $T_1$ uczynić lewym poddrzewem $x$, a~$y$ -- prawym synem węzła $x$.
Wystarczy wreszcie pokolorować $x$ na czerwono i~przywrócić czerwono-czarne własności powstałego drzewa analogicznie do punktu (d).

\subproblem %13-2(f)
Algorytm \proc{RB-Join} buduje drzewo $T$ na podstawie opisu z~poprzednich punktów.
Poza wywołaniami procedur \proc{RB-Join-Point}, jej symetrycznej wersji i~\proc{RB-Insert-Fixup} wykonuje on dodatkowo stałą liczbę operacji, zatem dla drzew wejściowych o~łącznie $n$ węzłach działa on w~czasie $O(\lg n)$.

Zastanówmy się teraz, dlaczego wartownik w~reprezentacji drzew czerwono-czarnych może negatywnie wpłynąć na efektywność implementacji operacji złączania.
W~reprezentacji z~wartownikami po zakończeniu działania algorytmu wskaźniki na synów niektórych węzłów początkowo znajdujących się w~$T_1$ pokazują na \attrib{T_1}{nil}, a~niektórych węzłów początkowo znajdujących się w~$T_2$ -- na \attrib{T_2}{nil}.
Liczba wszystkich takich węzłów jest rzędu $O(n)$, dlatego aktualizacja ich pól na \attrib{T}{nil} zwiększyłaby czas działania procedury \proc{RB-Join} do $O(n)$.

Celem stosowania wartownika jest przede wszystkim uproszczenie operacji na drzewach czer\-wono-czarnych.
Wersje tych operacji wykorzystujące specjalną wartość \const{nil} zamiast wartownika powinny wykrywać sytuacje z~próbami odniesienia się do pól węzła \const{nil} i~nie dopuszczać do ich odczytania bądź zapisania.
Węzeł \const{nil} powinien być traktowany jak wirtualny czarny liść.
Pamiętajmy też, że w~oryginalnej implementacji \proc{RB-Delete} pole $p$ wartownika wykorzystywane jest do przechowania pewnej wartości odczytywanej następnie w~procedurze \proc{RB-Delete-Fixup}.
Jednym ze sposobów poradzenia sobie z~tym niuansem w~implementacji \proc{RB-Delete} bez wartowników jest tworzenie sztucznego węzła zastępującego wartownik, w~przypadku gdy $x=\const{nil}$, i~odpowiednie jego podłączenie do węzła $y$.
Po powrocie z~wywołania \proc{RB-Delete-Fixup} sztuczny węzeł byłby następnie z~powrotem zastępowany przez \const{nil}.
Widać zatem, że zmodyfikowanie operacji wstawiania i~usuwania tak, aby nie wykorzystywały wartownika, jest możliwe, a~w~dodatku nie powoduje pogorszenia ich czasów działania.
