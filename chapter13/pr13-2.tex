\problem{Złączanie drzew czerwono-czarnych} %13-2

\subproblem %13-2(a)
Podczas wstawiania nowego węzła do drzewa $T$ czarna wysokość $T$ może zmienić się tylko wtedy, gdy czerwony korzeń $T$ zostaje pokolorowany na czarno.
Można więc w~procedurze \proc{RB-Insert-Fixup} bezpośrednio przed ostatnim wierszem inkrementować wartość \attrib{T}{bh} w~przypadku, gdy $\attribb{T}{root}{color}=\const{red}$.

W~procedurze \proc{RB-Delete-Fixup} wszystkie przypadki zachowują własność 5, o~ile przyjmiemy, że węzeł aktualnie pokazywany przez $x$ wnosi dodatkową ,,czarną jednostkę''.
Czarna wysokość drzewa $T$ w~tej procedurze może być więc zmodyfikowana tylko wtedy, gdy nadmiarowa ,,czarna jednostka'' jest przenoszona na korzeń drzewa i~,,zapominana'' w~przypadku 2.
Wystarczy zatem bezpośrednio po linii 11 (a~także w~analogicznym miejscu w~symetrycznej sytuacji reprezentowanej przez linię 22) zmniejszyć \attrib{T}{bh} o~1, gdy $x=\attrib{T}{root}$.
W~przypadku 4 wskaźnik $x$ także jest ustawiany na korzeń, ale zanim zostanie wykonane przypisanie z~wiersza 21, w~drzewie zachowana jest już własność 5 bez potraktowania węzła wskazywanego przez $x$ jako ,,nadmiarowo'' czarnego.
Przypisanie to służy tylko do przerwania pętli.

Czarna wysokość drzewa $T$ może się zmienić także w~sytuacji, gdy z~$T$ usuwany jest jego jedyny węzeł wewnętrzny.
Wówczas w~procedurze \proc{RB-Delete-Fixup} nie wykonuje się żadna iteracja pętli \kw{while}, więc sytuację tę należy wykrywać wcześniej, np.\ w~wierszu 9 procedury \proc{RB-Delete}.

Czarna wysokość korzenia drzewa $T$ to oczywiście \attrib{T}{bh}.
Jeśli dany węzeł jest czerwony, to ma tę samą czarną wysokość, co jego ojciec.
Gdy natomiast jest czarny (ale nie jest korzeniem), to jego czarna wysokość jest o~1 mniejsza od czarnej wysokości ojca.
Możemy więc wyznaczyć czarne wysokości wszystkich węzłów na prostej ścieżce od korzenia drzewa $T$ do jednego z~jego liści w~czasie proporcjonalnym do długości tej ścieżki.

\subproblem %13-2(b)
Jak wyjaśnimy w~punkcie (f), w~celu zaprojektowania efektywnej operacji złączania drzew czerwono-czarnych, musimy zrezygnować w~ich implementacji z~wartownika, a~zamiast niego korzystać z~wartości \const{nil}, podobnie jak w~zwykłych drzewach wyszukiwań binarnych.

Poniższy pseudokod realizuje opisany w~treści algorytm, przy założeniu, że drzewa czerwono-czarne zaimplementowane są bez użycia wartownika.
\begin{codebox}
\Procname{$\proc{RB-Join-Point}(T_1,T_2)$}
\li	$y\gets\attrib{T_1}{root}$ \label{li:rb-join-point-initial-node}
\li	$b\gets\attrib{T_1}{bh}$
\li	\While $b>\attrib{T_2}{bh}$ \label{li:rb-join-point-while-begin}
\li		\Do \If $\attrib{y}{right}\ne\const{nil}$ \label{li:rb-join-point-if-begin}
\li				\Then $y\gets\attrib{y}{right}$
\li				\Else $y\gets\attrib{y}{left}$
				\End \label{li:rb-join-point-if-end}
\li			\If $y=\const{nil}$ lub $\attrib{y}{color}=\const{black}$ \label{li:rb-join-point-z-color}
\li				\Then $b\gets b-1$
				\End
		\End
\li	\Return $y$
\end{codebox}
Algorytm przechodzi drzewo $T_1$ od korzenia w~dół, wybierając w~pierwszej kolejności prawego syna aktualnego węzła $y$, a~jeżeli ten jest \const{nil}, to wtedy lewego syna $y$.
Dzięki temu odwiedzane są węzły o~największym możliwym kluczu na danej czarnej wysokości w~drzewie $T_1$.
Algorytm kończy działanie, gdy czarna wysokość $b$ węzła $y$, wyznaczona na podstawie obserwacji z~punktu~(a), zrówna się z~wartością \attrib{T_2}{bh}.
Gdy to się wydarzy, węzeł wskazywany aktualnie przez $y$ jest czarny i~następuje jego zwrócenie.
Zauważmy, że test z~linii \ref{li:rb-join-point-z-color} traktuje \const{nil} jak wirtualny czarny węzeł, którego napotkanie także prowadzi do dekrementacji $b$.

Algorytm może odwiedzić wszystkie węzły na prostej ścieżce od korzenia do liścia \const{nil}.
Ponieważ $T_1$ jest drzewem czerwono-czarnym o~co najwyżej $n$ węzłach, to czas działania algorytmu można ograniczyć od góry przez $O(\lg n)$.

\subproblem %13-2(c)
Na podstawie założenia klucz korzenia $y$ drzewa $T_y$, będącego poddrzewem $T_1$, jest mniejszy lub równy od klucza węzła $x$.
Można zatem, bez naruszenia własności drzewa wyszukiwań binarnych, umieścić węzeł $x$ między $y$ a~\attrib{y}{p}, tzn.\ uczynić $T_y$ lewym poddrzewem $x$.
Korzystając ponownie z~założenia, dochodzimy do wniosku, że $T_2$ może być prawym poddrzewem $x$.
Wykonując opisane modyfikacje, jesteśmy w~stanie zastąpić poddrzewo $T_y$ przez $T_y\cup\{x\}\cup T_2$ w~czasie $O(1)$.

\subproblem %13-2(d)
Jeżeli pokolorujemy $x$ na czerwono, to czarna wysokość węzła \attrib{x}{p} (o~ile $x\ne\attrib{T}{root}$) nie zmieni się i~czerwono-czarne własności 1, 3 i~5 pozostaną spełnione.
W~przypadku gdy $x=\attrib{T}{root}$, to naruszona będzie czerwono-czarna własność 2, a~jeśli $x\ne\attrib{T}{root}$ i~węzeł \attrib{x}{p} też jest czerwony, to spowoduje to naruszenie własności 4.
Z~identyczną sytuacją mieliśmy do czynienia podczas wstawiania węzła do drzewa czerwono-czarnego przy pomocy procedury \proc{RB-Insert}.
Wystarczy więc wywołać $\proc{RB-Insert-Fixup}(T,x)$, aby przywrócić naruszone własności 2 i~4 w~drzewie $T$ w~czasie $O(\lg n)$.

\subproblem %13-2(e)
W~sytuacji symetrycznej algorytm z~punktu (b) będzie wyszukiwał w~drzewie $T_2$ czarny węzeł $y$ o~możliwie najmniejszym kluczu spośród węzłów o~czarnej wysokości \attrib{T_1}{bh}, preferując lewych synów w~poruszaniu się po ścieżce w~dół drzewa.
Można to zrealizować, modyfikując pseudokod \proc{RB-Join-Point} poprzez zamianę $T_1$ z~$T_2$ w~liniach \doubledash{\ref{li:rb-join-point-initial-node}}{\ref{li:rb-join-point-while-begin}} oraz zamianę pól \id{left} i~\id{right} w~liniach \doubledash{\ref{li:rb-join-point-if-begin}}{\ref{li:rb-join-point-if-end}}.
Węzeł $x$ należy następnie umieścić w~drzewie $T_2$ jako nowego ojca węzła $y$, który teraz staje się prawym synem $x$, a~drzewo $T_1$ uczynić lewym poddrzewem $x$.
Nadanie koloru węzłowi $x$ i~przywrócenie własności drzewa czerwono-czarnego w~wynikowym drzewie $T$ odbywa się identycznie jak w~punkcie (d).

\subproblem %13-2(f)
Prześledźmy działanie algorytmu \proc{RB-Join}, którego pseudokod został podany poniżej.
\begin{codebox}
\Procname{$\proc{RB-Join}(T_1,x,T_2)$}
\li	utwórz puste drzewo czerwono-czarne $T$
\li	\If $\attrib{T_1}{bh}\ge\attrib{T_2}{bh}$ \label{li:rb-join-if-begin}
\li		\Then \If $\attrib{T_2}{root}=\const{nil}$
\li				\Then $\proc{RB-Insert}(T_1,x)$ \label{li:rb-join-rb-insert}
\li					\Return $T_1$
				\End
\li			$\attrib{T}{root}\gets x$
\li			$\attrib{T}{bh}\gets\attrib{T_1}{bh}$
\li			$y\gets\proc{RB-Join-Point}(T_1,T_2)$ \label{li:rb-join-rb-join-point}
\li			$\attrib{x}{left}\gets y$
\li			$\attrib{x}{right}\gets\attrib{T_2}{root}$
\li			\If $y\ne\attrib{T_1}{root}$
\li				\Then \If $y=\attribb{y}{p}{left}$
\li						\Then $\attribb{y}{p}{left}\gets x$
\li						\Else $\attribb{y}{p}{right}\gets x$
						\End
\li					$\attrib{T}{root}\gets\attrib{T_1}{root}$
\li					$\attrib{x}{p}\gets\attrib{y}{p}$
				\End
\li			$\attribb{T_2}{root}{p}\gets\attrib{y}{p}\gets x$ \label{li:rb-join-if-end}
\li		\Else (to samo co po \kw{then} odwrócone symetrycznie) \label{li:rb-join-symmetric-case}
		\End
\li	$\attrib{x}{color}\gets\const{red}$
\li	$\proc{RB-Insert-Fixup}(T,x)$ \label{li:rb-join-rb-insert-fixup}
\li	\Return $T$
\end{codebox}
W~wierszach \doubledash{\ref{li:rb-join-if-begin}}{\ref{li:rb-join-if-end}} wykonywane są działania w~przypadku, gdy czarna wysokość drzewa $T_2$ nie przekracza czarnej wysokości drzewa $T_1$.
Jeśli jedynym węzłem drzewa $T_2$ jest jego liść \const{nil}, to w~celu złączenia $T_1$ oraz węzła $x$, wystarczy ten ostatni wstawić do $T_1$ za pomocą zwykłej operacji wstawiania, po czym zwrócić wynikowe drzewo $T_1$.
W~przeciwnym przypadku wykonywane są operacje opisane w~punkcie (c).
W~wierszu~\ref{li:rb-join-symmetric-case} obsługiwany jest przypadek, gdy $\attrib{T_1}{bh}<\attrib{T_2}{bh}$, poprzez przeprowadzenie analogicznych działań do tych z~linii \doubledash{\ref{li:rb-join-if-begin}}{\ref{li:rb-join-if-end}}, ale z~zamienionymi $T_1$ i~$T_2$ oraz \id{left} i~\id{right}.
Wywołanie odpowiadające temu z~linii \ref{li:rb-join-rb-join-point} jest z~kolei zastąpione wersją symetryczną opisaną w~części (e).
Pod koniec działania procedura koloruje węzeł $x$ na czerwono i~przywraca naruszone własności czerwono-czarne w~drzewie $T$.

Na podstawie analizy z~poprzednich punktów nietrudno wywnioskować, że czas działania algorytmu \proc{RB-Join}, działającego na drzewach o~sumarycznie $n$ węzłach, wynosi $O(\lg n)$.

Zastanówmy się teraz, dlaczego wartownik w~reprezentacji drzew czerwono-czarnych powoduje problemy w~efektywnej implementacji operacji złączania.
W~reprezentacji z~wartownikami po zakończeniu działania algorytmu wskaźniki na synów niektórych węzłów w~$T_1$ pokazują na \attrib{T_1}{nil}, a~niektórych węzłów w~$T_2$ -- na \attrib{T_2}{nil}.
Tuż przed zakończeniem działania algorytmu moglibyśmy wykonać przypisanie $\attrib{T}{nil}\gets\attrib{T_1}{nil}$, dzięki któremu wartownik w~drzewie $T$ byłby poprawnie ustawiony dla węzłów oryginalnie znajdujących się w~$T_1$.
Jednakże wszystkie węzły początkowo należące do $T_2$, których co najmniej jeden z~synów jest liściem, nadal pokazywałyby na wartownika $\attrib{T_2}{nil}\ne\attrib{T}{nil}$.
Liczba takich węzłów może być rzędu $O(n)$, dlatego modyfikacja ich pól wskazujących na \attrib{T_2}{nil} zwiększyłaby czas działania procedury \proc{RB-Join} do $O(n)$.

Wartownik służy jedynie do uproszczenia operacji na drzewach czerwono-czarnych.
Wersje tych operacji wykorzystujące wartość \const{nil} zamiast wartownika powinny wykrywać sytuacje z~próbami odniesienia się do pól węzła \const{nil} i~nie dopuszczać do ich odczytania bądź zapisania.
Węzeł \const{nil} powinien być traktowany jak wirtualny czarny liść.
Pamiętajmy też, że w~oryginalnej implementacji \proc{RB-Delete} pole $p$ wartownika wykorzystywane jest do przechowania pewnej wartości.
Jednym ze sposobów poradzenia sobie z~tym niuansem w~implementacji \proc{RB-Delete} bez wartowników jest tworzenie sztucznego węzła zastępującego wartownik, w~przypadku, gdy $x=\const{nil}$ i~odpowiednie jego podłączenie do węzła $y$.
Po powrocie z~wywołania \proc{RB-Delete-Fixup} sztuczny węzeł jest następnie z~powrotem zastępowany przez \const{nil}.
Widać zatem, że zmodyfikowanie operacji wstawiania i~usuwania tak, aby nie wykorzystywały wartownika, jest możliwe, a~w~dodatku nie powoduje pogorszenia ich czasów działania.
