\problem{Złączanie drzew czerwono-czarnych} %13-2

\subproblem %13-2(a)
Podczas wstawiania nowego węzła do drzewa $T$ czarna wysokość $T$ może zmienić się tylko wtedy, gdy czerwony korzeń $T$ zostaje pokolorowany na czarno.
Można więc w~procedurze \proc{RB-Insert-Fixup} bezpośrednio przed ostatnim wierszem zinkrementować wartość \attrib{T}{bh} w~przypadku, gdy $\attrib{\attrib{T}{root}}{color}=\const{red}$.

W~usuwaniu węzła wszystkie przypadki w~procedurze \proc{RB-Delete-Fixup} zachowują własność 5, o~ile przyjmiemy, że węzeł aktualnie pokazywany przez $x$ wnosi dodatkową ,,czarną jednostkę''.
Czarna wysokość drzewa $T$ może być więc zmodyfikowana tylko wtedy, gdy nadmiarowa ,,czarna jednostka'' jest usuwana z~korzenia.
Wystarczy zatem bezpośrednio przed linią 23 zmniejszyć \attrib{T}{bh} o~1, gdy $x=\attrib{T}{root}$.

Czarna wysokość korzenia drzewa $T$ to oczywiście \attrib{T}{bh}.
Jeśli dany węzeł jest czerwony, to ma tę samą czarną wysokość, co jego ojciec.
Gdy natomiast jest czarny (ale nie jest korzeniem), to jego czarna wysokość jest o~1 mniejsza od czarnej wysokości ojca.
Możemy więc wyznaczyć czarne wysokości wszystkich węzłów na ścieżce od korzenia drzewa $T$ do jego liścia w~czasie proporcjonalnym do długości tej ścieżki.

\subproblem %13-2(b)
\subproblem %13-2(c)
\subproblem %13-2(d)
\subproblem %13-2(e)
\subproblem %13-2(f)
