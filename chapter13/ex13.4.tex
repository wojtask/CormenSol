\subchapter{Operacja usuwania}
\note{W~procedurze \proc{RB-Delete} w~linii 3 powinna być wywoływana wersja operacji następnika dla drzew czerwono-czarnych, która różni się od \proc{Tree-Successor} tym, że przyjmuje drzewo\/ $T$ jako dodatkowy parametr oraz używa\/ \attrib{T}{nil} zamiast \const{nil}.
Z~kolei w~linii 15 powinny być kopiowane tylko dodatkowe dane z~węzła\/ $y$ do\/ $z$.
W~szczególności kolor węzła\/ $z$ powinien pozostać bez zmian.
Ponadto w~linii 22 procedury \proc{RB-Delete-Fixup} nie wystarczy tylko zamienić wskaźniki, ale także każdą lewą rotację na prawą i~vice versa.}
\bignegskip

\exercise %13.4-1
Zanim wywołana zostanie procedura \proc{RB-Delete-Fixup}, korzeniem może zostać czerwony węzeł tylko w~przypadku, gdy usuwany jest korzeń drzewa, którego dokładnie jeden syn jest czerwonym węzłem wewnętrznym.
Wtedy jednak w~wywołaniu \proc{RB-Delete-Fixup} nie jest wykonywana ani jedna iteracja pętli \kw{while} i~korzeń jest kolorowany na czarno w~wierszu 23.

Procedura \proc{RB-Delete-Fixup} aktualizuje jednak kolor niektórych węzłów na czerwono i~dokonuje pewnych rotacji.
Przyjrzymy się tym sytuacjom i~sprawdzimy, że w~żadnej z~nich czerwony węzeł nie staje się korzeniem drzewa.
Zastosujemy identyczny podział na przypadki, jak w~omawianiu działania procedury w~Podręczniku.
W~przypadkach 1 i~3 zamiana kolorów między czarnym węzłem i~jego czerwonym synem, a~następnie wykonanie rotacji na tym węźle powoduje, że czerwony z~nich zostaje umieszczony w~drzewie głębiej niż czarny, przez co nie może on być korzeniem.
W~przypadku 2 na czerwono zostaje pokolorowany brat węzła $x$, ale oczywiście nie może być on korzeniem drzewa.
W końcu, po wykonaniu przypadku 4, $x$ jest ustawione na \attrib{T}{root} i~pętla kończy działanie, po czym koloruje $x$ na czarno.

\exercise %13.4-2
Jeśli węzeł \attrib{y}{p} jest czerwony, to $y$ jest czarny i~w~wierszu 17 operacji \proc{RB-Delete} wywoływana jest procedura \proc{RB-Delete-Fixup} dla czerwonego węzła $x$.
Pętla \kw{while} tej procedury nie wykonuje się i~węzeł $x$ jest kolorowany na czarno w~wierszu 23, co przywraca własność~4.

\exercise %13.4-3
Ciąg drzew po każdej operacji usunięcia węzła został zilustrowany na rys.\ \ref{fig:13.4-3}.
\begin{figure}[!ht]
	\centering \begin{tikzpicture}[
	level/.append style = {level distance=7mm, sibling distance=50mm/2^#1},
	every label/.style = {index node, rectangle, label distance=1mm, label position=below, draw=none, fill=none},
	outer/.append style = {node distance=15mm}
]

\node[outer] (pic a) {
\begin{tikzpicture}[
	every node/.style = {tree node},
	anchor = center
]
        \node[blackened] {38}
            child {node {19}
                child {node[blackened] {12}
                    child {node {8}}
                    child[missing]
                }
                child {node[blackened] {31}}
            }
            child {node[blackened] {41}};
\end{tikzpicture}
};

\node[outer, right=of pic a] (pic b) {
\begin{tikzpicture}[
	every node/.style = {tree node},
	anchor = center
]
        \node[blackened] {38}
            child {node {19}
                child {node[blackened] {12}}
                child {node[blackened] {31}}
            }
            child {node[blackened] {41}};
\end{tikzpicture}
};

\node[outer, right=of pic b] (pic c) {
\begin{tikzpicture}[
	every node/.style = {tree node},
	anchor = center
]
	\node[blackened] {38}
            child {node[blackened] {19}
                child[missing]
                child {node {31}}
            }
            child {node[blackened] {41}};
\end{tikzpicture}
};

\node[outer, below=of pic a] (pic d) {
\begin{tikzpicture}[
	every node/.style = {tree node},
	anchor = center
]
	\node[blackened] {38}
            child {node[blackened] {31}}
            child {node[blackened] {41}};
\end{tikzpicture}
};

\node[outer, right=20mm of pic d] (pic e) {
\begin{tikzpicture}[
	every node/.style = {tree node},
	anchor = center
]
	\node[blackened] {38}
            child[missing]
            child {node {41}};
\end{tikzpicture}
};

\node[outer, right=20mm of pic e] (pic f) {
\begin{tikzpicture}[
	every node/.style = {tree node},
	anchor = center
]
        \node[blackened] {41};
\end{tikzpicture}
};

\node[outer, right=20mm of pic f] (pic g) {
\begin{tikzpicture}[
	every node/.style = {tree node},
	anchor = center
]
        \node[blackened, label=\attrib{T}{nil}] {};
\end{tikzpicture}
};

\node[subpicture label, below=1mm of pic a] (label a) {(a)};
\foreach \x in {b,c} {
	\node[subpicture label] at (label a -| pic \x) {(\x)};
}
\node[subpicture label, below=3mm of pic d] (label d) {(d)};
\foreach \x in {e,f,g} {
	\node[subpicture label] at (label d -| pic \x) {(\x)};
}

\end{tikzpicture}

	\caption{Drzewa czerwono-czarne powstałe po usunięciu węzłów o~kluczach 8, 12, 19, 31, 38, 41 kolejno z~drzewa czerwono-czarnego $T$ skonstruowanego w~\refExercise{13.3-2}.
	{\sffamily\bfseries(a)} Drzewo $T$ przed rozpoczęciem usuwania.
	{\sffamily\bfseries(b)} Wycięcie węzła o~kluczu 8 nie powoduje naruszenia żadnej własności drzewa czerwono-czarnego.
	{\sffamily\bfseries(c)\nbendash(e)} Po wycięciu kolejnych elementów, w~drzewie naruszona zostaje wyłącznie własność 5.
	Przywracana jest ona następnie w~procedurze \proc{RB-Delete-Fixup}.
	{\sffamily\bfseries{(f)}} Po pozbawieniu drzewa przedostatniego węzła pozostaje w~nim jedynie czerwony korzeń, co stanowi naruszenie własności 2.
        W~linii 23 procedury \proc{RB-Delete-Fixup} jest ona jednak natychmiast przywracana.
        {\sffamily\bfseries{(g)}} Usunięcie ostatniego węzła pozostawia drzewo czerwono-czarne składające się tylko z~wartownika \attrib{T}{nil} (którego pominęliśmy na pozostałych częściach rysunku dla zwiększenia czytelności).
        Wywołanie procedury \proc{RB-Delete-Fixup} na wartowniku nie powoduje żadych zmian.} \label{fig:13.4-3}
\end{figure}

\exercise %13.4-4
Gdy z~drzewa czerwono-czarnego $T$ efektywnie usuwany jest czarny węzeł $y$, którego obaj synowie to \attrib{T}{nil}, do procedury \proc{RB-Delete-Fixup} przekazywany jest jako parametr wartownik \attrib{T}{nil}, którego pole $p$ pokazuje na \attrib{y}{p}.
A~zatem wszystkie odwołania do pól parametru $x$ w~liniach 1, 2, 3, 6, 7, 8, 11, 16, 17, 18, 20, 22 i~23 tej procedury są odwołaniami do pól wartownika \attrib{T}{nil}.
Ponadto w~liniach 9, 12 i~13 odczytywany bądź modyfikowany jest kolor synów brata $w$ węzła $x$, przy czym synem $w$ może być \attrib{T}{nil} i~operacje te mogą odbywać się na atrybucie \id{color} wartownika.

Zastanówmy się teraz, czy w~jakimkolwiek innym wierszu \proc{RB-Delete-Fixup} pola \attrib{T}{nil} mogą być odczytywane bądź aktualizowane.
Jeśli efektywnie usuniętym węzłem $y$ jest korzeń drzewa $T$, to procedura zostaje wywołana dla $x=\attrib{T}{root}$ i~pętla \kw{while} nie wykonuje się.
Gdy z~kolei $y$ nie jest korzeniem, to w~momencie wywołania procedury jest czarny i~ma brata $w\ne\attrib{T}{nil}$, gdyż w~przeciwnym przypadku własność 5 byłaby zaburzona dla \attrib{y}{p}.
Jak wynika z~analizy przypadków w~Podręczniku, $w$ nie zostaje nigdy zaktualizowane na \attrib{T}{nil} w~trakcie działania procedury -- zarówno w~linii 8, jak i~16 $\attribb{x}{p}{right}\ne\attrib{T}{nil}$.
Dzięki temu w~wierszach 4, 5, 10, 14 i~15 nigdy nie pojawi się wartownik \attrib{T}{nil}.
Możemy wyeliminować też linię 19 -- węzeł \attrib{w}{right} na początku przypadku 4 jest czerwony, dlatego nie może być \attrib{T}{nil} -- oraz 21.

Także w~żadnym z~wywołań \proc{Left-Rotate} i~\proc{Right-Rotate} w~procedurze \proc{RB-Delete-Fixup} nie ma odwołań do pól wartownika, ponieważ wszystkie węzły, na których operacje te pracują, są węzłami wewnętrznymi.

\exercise %13.4-5
W~tabeli \ref{tab:13.4-5} zebrano liczby czarnych węzłów znajdujących się na ścieżkach od korzenia każdego poddrzewa sprzed transformacji do poddrzew $\alpha$, $\beta$, \dots, $\zeta$.
Liczby te w~każdym przypadku zgadzają się z~ilościami czarnych węzłów na odpowiednich ścieżkach w~drzewie po transformacji.
Oznacza to, że każde przekształcenie drzewa w~procedurze \proc{RB-Delete-Fixup} zachowuje własność 5 drzewa czerwono-czarnego.
Pamiętajmy, że węzeł $x$ wnosi dodatkową ,,czarną jednostkę''.

\begin{table}[!ht]
	\centering
    	\begin{tabular}{l||c|c|c}
        	& $\alpha$, $\beta$ & $\gamma$, $\delta$ & $\varepsilon$, $\zeta$ \\
        	\hline
			\hline
            {Przypadek 1} & 3 & 2 & 2 \\
            \hline
            {Przypadek 2} & $\mathrm{count}(c)+2$ & $\mathrm{count}(c)+2$ & $\mathrm{count}(c)+2$ \\
            \hline
            {Przypadek 3} & $\mathrm{count}(c)+2$ & $\mathrm{count}(c)+1$ & $\mathrm{count}(c)+2$ \\
            \hline
            {Przypadek 4} & $\mathrm{count}(c)+2$ & $\mathrm{count}(c)+\mathrm{count}(c')+1$ & $\mathrm{count}(c)+1$ \\
        \end{tabular}
	\caption{Liczby czarnych węzłów od korzeni poddrzew z~rys.\ 13.7 z~Podręcznika do każdego z~poddrzew $\alpha$, $\beta$, \dots, $\zeta$, zarówno przed, jak i~po przekształceniu poddrzew.} \label{tab:13.4-5}
\end{table}

\exercise %13.4-6
Przypadek 1 ma miejsce wtedy, gdy brat $w$ węzła $x$ ma kolor czerwony, a~zatem węzeł $\attrib{x}{p}=\attrib{w}{p}$ ma kolor czarny na mocy własności 4.
Własność ta może być naruszona podczas usuwania węzła jedynie w~przypadku, gdy czerwone są węzły $x$ oraz \attrib{x}{p}, ale wtedy zostaje ona natychmiast przywrócona (\refExercise{13.4-2}) i~nie dochodzi do przypadku 1.

\exercise %13.4-7
Wywołanie \proc{RB-Insert}, a~następnie \proc{RB-Delete} dla tego samego węzła nie zawsze wraca do początkowego drzewa.
Rys.\ \ref{fig:13.4-7} przedstawia przykład, w~którym zmienia się struktura drzewa po takim działaniu oraz taki, w~którym struktura jest zachowana, ale zmianie ulegają kolory węzłów.
\begin{figure}[!ht]
	\centering \begin{tikzpicture}[
	level/.append style = {level distance=7mm, sibling distance=50mm/2^#1},
	index node/.append style = {auto, rectangle, draw=none, fill=none},
	outer/.append style = {node distance=10mm}
]

\node[outer] (pic a) {
\begin{tikzpicture}[
	every node/.style = {tree node},
	anchor = center
]
	\node[blackened] (left root) {3}
		child {node {2}}
		child[missing];
	
	\node[right=45mm of left root, blackened] (middle root) {2}
		child {node {1}}
		child {node {3}};
	
	\draw[arrow, shorten >=10mm, shorten <=10mm, dashed, thick] (left root.east) -- node[index node] {dodanie węzła} (middle root.west);
	
	\node[right=45mm of middle root, blackened] (right root) {2}
		child[missing]
		child {node {3}};
	
	\draw[arrow, shorten >=10mm, shorten <=10mm, dashed, thick] (middle root.east) -- node[index node] {usunięcie węzła} (right root.west);
	
	\node[subpicture label, draw=none, fill=none, left=20mm of left root] {(a)};
\end{tikzpicture}
};

\node[outer, below=of pic a] (pic b) {
\begin{tikzpicture}[
	every node/.style = {tree node},
	anchor = center
]
	\node[blackened] (left root) {3}
		child {node {2}}
		child {node {4}};
	
	\node[right=45mm of left root, blackened] (middle root) {3}
		child {node[blackened] {2}
			child {node {1}}
			child[missing]
		}
		child {node[blackened] {4}};
	
	\draw[arrow, shorten >=10mm, shorten <=10mm, dashed, thick] (left root.east) -- node[index node] {dodanie węzła} (middle root.west);
	
	\node[right=45mm of middle root, blackened] (right root) {3}
		child {node[blackened] {2}}
		child {node[blackened] {4}};
	
	\draw[arrow, shorten >=10mm, shorten <=10mm, dashed, thick] (middle root.east) -- node[index node] {usunięcie węzła} (right root.west);
	
	\node[subpicture label, draw=none, fill=none, left=20mm of left root] {(b)};
\end{tikzpicture}
};

\end{tikzpicture}

	\caption{Drzewa czerwono-czarne po wstawieniu nowego węzła $x$ o~kluczu 1, a~następnie natychmiastowym jego usunięciu.
	{\sffamily\bfseries(a)} Drzewo $T$, w~którym opisane operacje powodują zmianę struktury drzewa.
	{\sffamily\bfseries(b)} Drzewo $T$, w~którym działania te zachowują strukturę, ale zmieniają kolory węzłów.} \label{fig:13.4-7}
\end{figure}
