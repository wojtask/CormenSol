\chapter{Sortowanie w~czasie liniowym}

\subchapter{Dolne ograniczenia dla problemu sortowania}

\exercise %8.1-1
Łatwo zauważyć, że do znalezienia uporządkowania $n$ elementów ciągu wejściowego potrzebnych jest w~najlepszym przypadku $n-1$ porównań.
Sytuacja zachodzi np.\ dla ciągu niemalejącego.
A~zatem najmniejszą głębokością liścia w~drzewie decyzyjnym jest $n-1$.

\exercise %8.1-2
Górne oszacowanie znajdujemy w~prosty sposób:
\[
	\lg(n!) = \sum_{k=1}^n\lg k \le \sum_{k=1}^n\lg n = n\lg n = O(n\lg n).
\]
Aby uzyskać oszacowanie dolne, rozdzielamy sumę, korzystając z~faktu, że $n=\lfloor n/2\rfloor+\lceil n/2\rceil$:
\[
	\lg(n!) = \sum_{k=1}^n\lg k = \sum_{k=1}^{\lfloor n/2\rfloor}\lg k+\sum_{k=\lceil n/2\rceil}^n\lg k.
\]
W~pierwszej sumie po prawej ograniczamy $\lg k$ od dołu przez $\lg1$, a~w~drugiej -- przez $\lg\lceil n/2\rceil$:
\[
	\lg(n!) \ge \sum_{k=1}^{\lfloor n/2\rfloor}\lg1+\sum_{k=\lceil n/2\rceil}^n\lg\lceil n/2\rceil \ge 0+(n/2)\lg\lceil n/2\rceil \ge (n/2)\lg(n/2) = \Omega(n\lg n).
\]

\exercise %8.1-3
Jeśli sortowanie działa w~czasie $\Theta(n)$ dla $l$ permutacji wejściowych długości $n$, to drzewo decyzyjne składające się z~gałęzi odpowiadających tylko tym permutacjom, ma wysokość $h=\Theta(n)$.
Powtarzając rozumowanie z~dowodu tw.~8.1, dostajemy nierówność $2^h\ge l$, a~stąd $h\ge\lg l$.
Pozostaje więc sprawdzić, czy $h$ jest asymptotycznie ograniczone funkcją liniową względem $n$ dla poszczególnych wartości przyjmowanych przez $l$:
\begin{itemize}
	\item jeśli $l=n!/2$, to $h\ge\lg l=\lg(n!/2)=\lg(n!)-1=\Omega(n\lg n)$;
	\item jeśli $l=n!/n$, to $h\ge\lg l=\lg(n!/n)=\lg(n!)-\lg n=\Omega(n\lg n)$;
	\item jeśli $l=n!/2^n$, to $h\ge\lg l=\lg(n!/2^n)=\lg(n!)-n=\Omega(n\lg n)$.
\end{itemize}
W~każdym badanym przypadku $h=\Omega(n\lg n)$, zatem nie istnieje algorytm sortujący za pomocą porównań i~działający w~czasie liniowym dla poszczególnych ilości permutacji wejściowych.

\exercise %8.1-4
Rozważmy drzewo decyzyjne reprezentujące sortowanie takiego częściowo uporządkowanego ciągu.
Każdy podciąg może wystąpić na wyjściu jako jedna z~$k!$ możliwych permutacji.
Ponieważ jest $n/k$ podciągów, to w~drzewie decyzyjnym znajduje się co najmniej $(k!)^{n/k}$ osiągalnych liści.
Na mocy faktu, że drzewo decyzyjne o~wysokości $h$ nie może mieć więcej niż $2^h$ liści, dostajemy
\[
	2^h \ge (k!)^{n/k},
\]
skąd uzyskujemy dolne oszacowanie na ilość porównań wykonywanych w~tym algorytmie:
\[
	h \ge \lg\bigl((k!)^{n/k}\bigr) = (n/k)\lg(k!) = (n/k)\cdot\Omega(k\lg k) = \Omega(n\lg k).
\]

\subchapter{Sortowanie przez zliczanie}

\exercise %8.2-1
Rys.~\ref{fig:8.2-1} przedstawia działanie procedury \proc{Counting-Sort} dla tablicy $A$.
\begin{figure}[ht]
	\begin{center}
		\includegraphics{fig_8.2-1}
	\end{center}
	\caption{Działanie procedury \proc{Counting-Sort} dla tablicy $A=\langle6,0,2,0,1,3,4,6,1,3,2\rangle$, której każdy element jest nieujemną liczbą całkowitą nie większą niż $k=6$.
{\sffamily\bfseries(a)} Tablica $A$ wraz z~tablicą pomocniczą $C$ po wykonaniu pętli w~wierszach \doubledash{3}{4}.
{\sffamily\bfseries(b)} Tablica $C$ po wykonaniu pętli w~wierszach \doubledash{6}{7}.
{\sffamily\bfseries\doubledash{(c)}{(e)}} Tablica wynikowa $B$ oraz tablica pomocnicza $C$ po wykonaniu, odpowiednio, jednej, dwóch i~trzech iteracji pętli \kw{for} w~wierszach \doubledash{9}{11}.
{\sffamily\bfseries(f)} Wynikowa posortowana tablica $B$.} \label{fig:8.2-1}
\end{figure}

\exercise %8.2-2
Elementy tablicy wejściowej przetwarzane są od końca, a~wartości w~tablicy $C$ stanowiące indeksy tablicy wynikowej, na które trafią wejściowe elementy, są systematycznie zmniejszane.
A~zatem równe sobie elementy będą umieszczane na coraz niższych pozycjach w~tablicy wynikowej, dzięki czemu ich początkowa kolejność zostanie zachowana.

\exercise %8.2-3
Po wprowadzeniu takiej modyfikacji tablica $A$ będzie przeglądana od lewej do prawej, a~równe elementy będą umieszczane w~odpowiedniej części tablicy wynikowej na coraz niższych pozycjach.
Oczywiście kolejność przetwarzania elementów tablicy $A$ nie wpływa na poprawność algorytmu \proc{Counting-Sort}, jednak zastosowanie opisanej zmiany powoduje, że sortowanie nie jest stabilne.

\exercise %8.2-4
Algorytm będzie zliczać elementy z~wejściowej tablicy w~czasie $\Theta(n+k)$, zapamiętując wyniki w~pomocniczej tablicy $C[0\twodots k]$.
Następnie, zapytany o~liczbę elementów z~przedziału $a\twodots b$ na podstawie danych z~tablicy pomocniczej, zwróci liczbę elementów z~zakresu $0\twodots b$ pomniejszoną o~liczbę elementów z~zakresu $0\twodots a-1$.
Dokładniej, jego pierwsza faza jest równoważna wierszom \doubledash{1}{8} procedury \proc{Counting-Sort}, natomiast w~drugiej fazie zwracany jest odpowiedni wynik w~zależności od przypadku:
\begin{enumerate}
	\renewcommand{\labelenumi}{(\roman{enumi})}
	\item $C[b]-C[a-1]$, jeśli $0<a\le b\le k$;
	\item $C[k]-C[a-1]$, jeśli $0<a\le k<b$;
	\item $C[b]$, jeśli $a\le0\le b\le k$;
	\item $C[k]$, jeśli $a\le0\le k<b$;
	\item 0, jeśli $a>k$ lub $b<0$.
\end{enumerate}

\subchapter{Sortowanie pozycyjne}

\exercise %8.3-1
Przebieg działania procedury \proc{Radix-Sort} dla podanej listy słów został przedstawiony na rys.~\ref{fig:8.3-1}.
\begin{figure}[ht]
	\begin{center}
		\includegraphics{fig_8.3-1}
	\end{center}
	\caption{Działanie procedury \proc{Radix-Sort} dla zbioru słów trzyliterowych.} \label{fig:8.3-1}
\end{figure}

\exercise %8.3-2
Algorytmami stabilnymi są sortowanie przez wstawianie i~sortowanie przez scalanie.
W~pierwszym z~nich, wstawiając element $A[j]$ do~podtablicy $A[1\twodots j-1]$, zatrzymujemy się na pierwszym elemencie z~tej podtablicy, który jest mniejszy lub równy od $A[j]$.
Zatem elementy równe sobie nie zostaną wymieszane.
Podobnie w~procedurze \proc{Merge}, jeśli porównywane elementy będą sobie równe, to do wynikowej tablicy zostanie wstawiony najpierw element z~tablicy $L$ i~dopiero potem ten z~tablicy $R$.
Stabilność algorytmu wynika na podstawie indukcji po scalanych fragmentach.
Natomiast zarówno heapsort, jak i~quicksort nie sortuje stabilnie -- przykładem danych wejściowych, które ilustrują ten fakt w~obu tych algorytmach, jest ciąg $A=\langle2,2,1\rangle$.

Aby dowolny algorytm sortowania za pomocą porównań uczynić stabilnym, można zapamiętywać z~każdym elementem jego początkową pozycję w~tablicy wejściowej i~przy każdym teście dającym odpowiedź, że elementy są sobie równe, porządkować je za pomocą ich początkowych pozycji.
Do realizacji takiego podejścia wymagana jest dodatkowa pamięć rzędu $\Theta(n)$, gdzie $n$ jest długością sortowanej tablicy.
Nie zwiększa się natomiast asymptotyczne oszacowanie na czas działania zmodyfikowanego sortowania, ponieważ przeprowadzenie dodatkowego testu odbywa się w~czasie stałym.

\exercise %8.3-3
Przeprowadzimy dowód przez indukcję względem liczby cyfr $d$ elementów wejściowych.
Dla $d=1$ algorytm \proc{Radix-Sort} sprowadza się do wywołania pomocniczej procedury do posortowania pojedynczych cyfr, a~zatem poprawność algorytmu wynika z~poprawności pomocniczego sortowania.

Załóżmy teraz, że $d>1$ i~że wywołanie algorytmu \proc{Radix-Sort} po $d-1$ fazach zwróciło tablicę elementów, które, obcięte do $d-1$ ostatnich pozycji, wyznaczają porządek niemalejący.
Teraz elementy sortowane są względem pozycji $d$.
Niech $a$ i~$b$ będą cyframi porównywanymi podczas tej fazy.
Jeśli $a\ne b$, to niezależnie od pozostałych cyfr elementów, których częściami są $a$ i~$b$, elementy te zostaną ustawione w~odpowiedniej kolejności.
Jeśli jednak $a=b$, to elementy nie zostaną zamienione miejscami, bo pomocnicza procedura sortuje stabilnie.
Elementy pozostają jednak we właściwym porządku, bo jest on wyznaczony przez ich $d-1$ ostatnich pozycji, a~te, na mocy założenia indukcyjnego, zostały poprawnie uporządkowane w~poprzednich fazach algorytmu.

\exercise %8.3-4
Każdą liczbę z~zakresu $0\twodots n^2-1$ można potraktować jak liczbę dwucyfrową w~systemie o~podstawie $n$.
Liczby w~takiej postaci sortujemy, wywołując algorytm \proc{Radix-Sort} o~dwóch fazach.
W~lemacie~8.3 mamy $d=2$ oraz $k=n$, zatem opisane sortowanie działa w~czasie $\Theta(2(n+n))=\Theta(n)$.

\exercise %8.3-5
Rozważmy sortowanie pozycyjne w~wersji intuicyjnej dla liczb trzycyfrowych.
W~pierwszej fazie sortujemy po najbardziej znaczącej cyfrze wejściowego ciągu liczb.
Podczas drugiej fazy dokonujemy sortowania w~obrębie fragmentów tablicy zawierających liczby o~tej samej najbardziej znaczącej cyfrze.
W~najgorszym przypadku po pierwszej fazie dostaniemy 10 takich fragmentów, każdy o~innej najbardziej znaczącej cyfrze, a~zatem środkowe cyfry sortowane będą w~kolejnych 10 fazach.
Sytuacja w~najgorszym przypadku powtórzy się -- z~każdego sortowanego fragmentu utworzy się po 10 jeszcze mniejszych fragmentów o~poszczególnych kombinacjach dwóch pierwszych cyfr.
Będzie zatem maksymalnie 100 takich podtablic, których posortowanie będzie wymagało 100 kolejnych faz.
Łącznie algorytm wykona zatem 111 faz.


Najwięcej tymczasowych podtablic pozostawionych do późniejszego przetworzenia istnieje wówczas, gdy rozpoczyna się sortowanie ostatnich cyfr.
W~przypadku sortowania liczb trzycyfrowych jest łącznie 27 takich podtablic -- 9 powstałych po posortowaniu najbardziej znaczących cyfr, 9 kolejnych powstałych po posortowaniu środkowych cyfr w~obrębie pierwszego fragmentu i~9 takich, które czekają na posortowanie po najmniej znaczących cyfrach.

W~ogólności do posortowania liczb \singledash{$d$}{cyfrowych} tym algorytmem wymaganych jest co najwyżej $\sum_{i=0}^{d-1}10^i=(10^d-1)/9$ faz.
Najwięcej tymczasowych fragmentów istnieje w~momencie rozpoczęcia sortowania po najmniej znaczącej cyfrze -- jest ich wtedy maksymalnie $9d$.

\subchapter{Sortowanie kubełkowe}

\exercise %8.4-1
Na rys.~\ref{fig:8.4-1} zostało przedstawione działanie sortowania kubełkowego dla tablicy $A$.
\begin{figure}[ht]
	\begin{center}
		\includegraphics{fig_8.4-1}
	\end{center}
	\caption{Działanie procedury \proc{Bucket-Sort} dla tablicy $A=\langle0{,}79$,~$0{,}13$, $0{,}16$, $0{,}64$, $0{,}39$, $0{,}20$, $0{,}89$, $0{,}53$,\! $0{,}71$,\!~$0{,}42\rangle$.
{\sffamily\bfseries(a)} Wejściowa tablica $A$.
{\sffamily\bfseries(b)} Tablica $B$ zawierająca posortowane listy (kubełki) po wykonaniu pętli w~wierszach \doubledash{4}{5}.} \label{fig:8.4-1}
\end{figure}

\exercise %8.4-2
Pesymistyczny przypadek dla algorytmu sortowania kubełkowego zachodzi, gdy wszystkie elementy z~wejściowej tablicy trafią do tego samego kubełka.
Mamy wtedy do posortowania pojedynczą listę o~długości $n$, co zajmuje czas $O(n^2)$.

W~celu uporządkowania kubełków, zamiast sortowania przez wstawianie, można użyć algorytmu sortującego, który wykonuje $O(n\lg n)$ operacji w~przypadku pesymistycznym, np.\ sortowanie przez scalanie.
Aby obliczyć średni czas działania sortowania kubełkowego w~takim wariancie, podstawiamy do wzoru~(8.1) ograniczenie na czas działania sortowania przez scalanie:
\[
	\E(T(n)) = \Theta(n)+\sum_{i=0}^{n-1}O(\E(n_i\lg n_i)) \le O(n)+\sum_{i=0}^{n-1}O(\E(n_i^2)) = O(n).
\]
Ponadto pętla \kw{for} w~wierszach \doubledash{2}{3} zajmuje czas liniowy, skąd mamy $\E(T(n))=\Omega(n)$.
A~zatem oczekiwany czas działania sortowania kubełkowego po modyfikacji pozostaje liniowy.

\exercise %8.4-3
Szanse nieuzyskania orła w~dwóch rzutach monetą wynoszą $1/4$, co jest równe szansie uzyskania dwóch orłów w~dwóch rzutach.
Dokładnie jednego orła można uzyskać z~prawdopodobieństwem równym $1/2$.
Na podstawie tych wartości obliczamy:
\begin{align*}
	\E(X^2) &= 0^2\cdot\Pr(X=0)+1^2\cdot\Pr(X=1)+2^2\cdot\Pr(X=2) = 3/2, \\
	\E^2(X) &= \bigl(0\cdot\Pr(X=0)+1\cdot\Pr(X=1)+2\cdot\Pr(X=2)\bigr)^2 = 1.
\end{align*}

\exercise %8.4-4
Algorytm będzie opierał się na pomyśle z~sortowania kubełkowego z~$n$ kubełkami.
Podzielimy koło jednostkowe na $n$ obszarów o~równych polach reprezentujących przedziały odległości od środka koła i~z~każdym takim obszarem skojarzymy inny kubełek.
Dzięki temu punkty będą umieszczane w~poszczególnych kubełkach z~jednakowym prawdopodobieństwem.
Łatwo zauważyć, że obszary te będą pierścieniami kołowymi (z~wyjątkiem jednego, który będzie kołem) o~jednakowych powierzchniach równych $\pi/n$.
Pozostaje tylko wyznaczyć ich promienie.

Ponumerujmy obszary (kubełki) kolejnymi liczbami całkowitymi od 0 do $n-1$ w~kolejności od środka do brzegu koła jednostkowego i~oznaczmy przez $r_j$ długość promienia wewnętrznego pierścienia kołowego stanowiącego \singledash{$j$}{ty} obszar.
Obszar zerowy jest kołem o~polu $\pi/n$, więc $r_1=\sqrt{1/n}$.
Ponieważ suma tego koła z~pierścieniem do niego przylegającym jest kołem o~polu $2\pi/n$, to stąd mamy $r_2=\sqrt{2/n}$.
Rozumowanie przeprowadzamy dla pozostałych pierścieni, otrzymując $r_j=\sqrt{j/n}$ dla każdego $j=1$, 2,~\dots,~$n-1$.
Przyjmijmy ponadto $r_0=0$ oraz $r_n=1$.

Punkty sortowane są względem ich odległości od środka koła jednostkowego przy wykorzystaniu opisanych kubełków.
Dla każdego punktu $p_i=\langle x_i,y_i\rangle$ obliczane jest $d_i=\sqrt{x_i^2+y_i^2}$ i~jeśli zachodzi $r_j<d_i\le r_{j+1}$, to punkt umieszczany jest w~kubełku o~numerze $j$.

\exercise %8.4-5
Niech $X$ będzie zmienną losową o~ciągłej dystrybuancie $P_X$.
Definiujemy nową zmienną losową $Y=P_X(X)$ o~dystrybuancie $P_Y$.
Wówczas, dla $0<y<1$, mamy
\[
    P_Y(y) = \Pr(Y\le y) = \Pr(P_X(X)\le y) = \Pr(X\le P_X^{-1}(y)) = P_X(P_X^{-1}(y)) = y.
\]
W~uzasadnieniu korzystamy z~funkcji odwrotnej do dystrybuanty $P_X$, jednak ta ostatnia może nie być ściśle rosnąca i~funkcja do niej odwrotna może nie być poprawnie zdefiniowana.
Dlatego funkcję $P_X^{-1}$ definiujemy tutaj jako
\[
	P_X^{-1}(y) = \inf\{\,x\in\mathbb{R}:P_X(x)\ge y\,\} \quad\text{dla $0<y<1$}.
\]
Pokazaliśmy, że $P_Y$ jest identycznością na $(0,1)$, a~zatem $Y$ jest zmienną losową o~ciągłym rozkładzie jednostajnym w~tym przedziale.

W~naszym problemie dla każdej zmiennej losowej $X_i$ o~dystrybuancie $P$, gdzie $i=1$, 2,~\dots,~$n$, wyznaczamy nową zmienną losową $Y_i=P(X_i)$.
Zgodnie z~powyższym opisem każda nowa zmienna ma rozkład jednostajny w~przedziale $(0,1)$, możemy zatem posortować je, stosując sortowanie kubełkowe o~$n$ kubełkach.
Jako wynik dostaniemy pewną permutację $\langle Y_{\pi(1)},Y_{\pi(2)},\dots,Y_{\pi(n)}\rangle$.
Rozwiązanie problemu stanowi wówczas ciąg $\langle X_{\pi(1)},X_{\pi(2)},\dots,X_{\pi(n)}\rangle$.

Wyznaczenie nowego ciągu zmiennych losowych zabiera czas $\Theta(n)$, gdyż zakładamy, że wartości dystrybuanty $P$ można obliczać w~czasie stałym.
Również sortowanie kubełkowe $n$ elementów działa średnio w~czasie $\Theta(n)$, zatem w~średnim przypadku cała procedura zajmuje czas liniowy.

\problems

\problem{Dolne ograniczenia na średni czas działania sortowania za pomocą porównań} %8-1

\subproblem %8-1(a)
Podczas sortowania algorytmem $A$ żadne dwie różne permutacje wejściowe nie prowadzą do tego samego liścia w~drzewie $T_A$ -- jest w~nim zatem co najmniej $n!$ liści.
Ponieważ algorytm $A$ jest deterministyczny, to dla każdej permutacji wejściowej osiągany jest zawsze ten sam liść, a~więc w~$T_A$ jest co najwyżej $n!$ osiągalnych liści.
Wynika stąd, że algorytm $A$ może dotrzeć do dokładnie $n!$ liści.
Ponieważ każda permutacja ma szanse pojawić się na wejściu z~równym prawdopodobieństwem, to szanse na dotarcie do dowolnego osiągalnego liścia wynoszą $1/n!$.
Pozostałe liście nigdy nie zostaną odwiedzone podczas działania algorytmu $A$.

\subproblem %8-1(b)
Oznaczmy przez $L(T)$ zbiór liści drzewa $T$, a~przez $d_T(x)$ -- głębokość węzła $x$ w~drzewie $T$.
Ponieważ $k>1$, to korzeń drzewa $T$ nie jest jego liściem, więc $L(T)=L(LT)\cup L(RT)$.
Głębokość każdego liścia w~poddrzewie $LT$ jest o~1 mniejsza niż głębokość tego samego liścia w~drzewie $T$ i~analogicznie dla liści w~poddrzewie $RT$.
Mamy zatem
\begin{align*}
    D(T) &= \sum_{x\in L(T)}d_T(x) \\
	&= \sum_{x\in L(LT)}d_T(x)+\sum_{x\in L(RT)}d_T(x) \\
	&= \sum_{x\in L(LT)}(d_{LT}(x)+1)+\sum_{x\in L(RT)}(d_{RT}(x)+1) \\
	&= \sum_{x\in L(LT)}d_{LT}(x)+\sum_{x\in L(RT)}d_{RT}(x)+\sum_{x\in L(T)}1 \\[1mm]
	&= D(LT)+D(RT)+k.
\end{align*}

\subproblem %8-1(c)
\note{W~tekście zadania wykorzystywane jest\/ $d(1)$ mimo braku jego definicji; przyjmujemy zatem\/ $d(1)=0$.}

\noindent W~celu udowodnienia wzoru z~treści zadania pokażemy, że zachodzą nierówności
\[
    d(k) \le \min_{1\le i\le k-1}(d(i)+d(k-i)+k) \quad\text{oraz}\quad d(k) \ge \min_{1\le i\le k-1}(d(i)+d(k-i)+k).
\]

Ustalmy pewne $i$ ze zbioru $\{1,2,\dots,k-1\}$.
Niech $LT$ będzie takim drzewem binarnym o~$i$ liściach, że $d(i)=D(LT)$ i~niech $RT$ będzie takim drzewem binarnym o~$k-i$ liściach, że $d(k-i)=D(RT)$.
Niech $T$ będzie drzewem binarnym, w~którym $LT$ jest jego lewym poddrzewem, a~$RT$ -- jego prawym poddrzewem.
Korzystając z~definicji $d(k)$ i~wzoru z~części~(b), mamy
\[
    d(k) \le D(T) = D(LT)+D(RT)+k = d(i)+d(k-i)+k.
\]
Ponieważ otrzymana nierówność zachodzi dla każdego $i=1$, 2,~\dots,~$k-1$, to możemy napisać $d(k)\le\min_{1\le i\le k-1}(d(i)+d(k-i)+k)$.

Pokażemy teraz, że zachodzi także nierówność przeciwna.
Ustalmy w~tym celu drzewo binarne $T$ o~$k$ liściach, dla którego $d(k)=D(T)$.
Niech $LT$ i~$RT$ będą poddrzewami drzewa $T$, odpowiednio, lewym i~prawym.
Niech $i_0$, gdzie $1\le i_0\le k-1$, będzie liczbą liści w~drzewie $LT$, a~$k-i_0$ -- liczbą liści w~drzewie $RT$.
Wówczas
\[
    d(k) = D(T) = D(LT)+D(RT)+k \ge d(i_0)+d(k-i_0)+k \ge \min_{1\le i\le k-1}(d(i)+d(k-i)+k).
\]
Założyliśmy tutaj, że ani $LT$, ani $RT$ nie są puste.
Gdyby tak jednak było, to drugie poddrzewo $T$ zawierałoby wszystkie $k$ liści drzewa $T$ i~wówczas, na mocy punktu~(b), funkcja $D$ dla niego przyjęłaby wartość $D(T)-k$.
To jednak przeczyłoby wyborowi $T$ jako drzewa o~$k$ liściach z~minimalną wartością funkcji $D$.

\subproblem %8-1(d)
W~\refExercise{7.4-2} analizowana była funkcja $f(q)=q\lg q+(n-q-1)\lg(n-q-1)$.
Zostało tam pokazane, że w~przedziale $(0,n-1)$ funkcja $f$ osiąga minimum dla $q=(n-1)/2$.
Wystarczy zatem przyjąć $q=i$ oraz $n=k+1$, aby pokazać stwierdzenie z~treści zadania.
Minimalna wartość badanej funkcji wynosi $k\lg k-k$.

Pokażemy oszacowanie na $d(k)$ przez indukcję, zakładając w~tym celu, że $d(i)\ge i\lg i$ dla każdego $i=1$, 2,~\dots,~$k-1$ i~korzystając z~powyższego rezultatu oraz z~punktu~(c):
\begin{align*}
	d(k) &= \min_{1\le i\le k-1}(d(i)+d(k-i)+k) \\
	&\ge \min_{1\le i\le k-1}(i\lg i+(k-i)\lg(k-i))+k \\
	&\ge k\lg k-k+k \\
	&= k\lg k.
\end{align*}
Podstawa indukcji zachodzi trywialnie, bo $d(1)=0\ge 1\lg1$, a~zatem $d(k)=\Omega(k\lg k)$.

\subproblem %8-1(e)
Zauważmy, że możemy usunąć wszystkie nieosiągalne węzły z~drzewa $T_A$, ponieważ nie wpływają one na czas działania algorytmu $A$.
A~zatem, na mocy punktu~(a), po ich usunięciu $T_A$ zawiera dokładnie $n!$ liści.
Wykorzystując definicję $d(k)$ oraz wynik z~poprzedniej części, mamy
\[
	D(T_A) \ge d(n!) = \Omega(n!\lg(n!)).
\]

$D(T_A)$ stanowi sumę głębokości liści drzewa decyzyjnego $T_A$, a~głębokość każdego liścia jest proporcjonalna do czasu działania algorytmu $A$ dla permutacji reprezentowanej przez ten liść.
Ponieważ prawdopodobieństwo osiągnięcia dowolnego liścia drzewa $T_A$ jest równe $1/n!$, to oczekiwany czas algorytmu $A$ wynosi
\[
	\frac{D(T_A)}{n!} = \frac{\Omega(n!\lg(n!))}{n!} = \Omega(\lg(n!)) = \Omega(n\lg n).
\]

\subproblem %8-1(f)
Niech $B$ będzie dowolnym randomizowanym algorytmem sortującym za pomocą porównań, a~$T_B$ -- jego drzewem decyzyjnym.
Istnieje wówczas deterministyczny algorytm $A$ sortujący za pomocą porównań, którego drzewo decyzyjne $T_A$ powstaje z~drzewa $T_B$ poprzez zastąpienie każdego węzła zrandomizowanego minimalnym poddrzewem o~korzeniu będącym synem tegoż węzła zrandomizowanego.
Przez minimalne poddrzewo rozumiemy tutaj takie, które posiada minimalną średnią odległość od swojego korzenia od liścia.
Algorytm $A$ wybiera zatem najlepszą możliwość zawsze wtedy, gdy algorytm $B$ dokonuje losowego wyboru spośród $r$ możliwości.
Dlatego w~średnim przypadku $A$ wykonuje co najwyżej tyle porównań, co $B$.

\problem{Sortowanie w~miejscu w~czasie liniowym} %8-2

\subproblem %8-2(a)
Sortowanie przez zliczanie dla $k=1$.

\subproblem %8-2(b)
Poniższy pseudokod implementuje algorytm o~zadanych własnościach:
\begin{codebox}
\Procname{$\proc{Bitwise-Sort}(A)$}
\li	$n\gets\attrib{A}{length}$
\li	$i\gets1$
\li	$j\gets n$
\li	\While $i<j$
\li		\Do
			zamień $A[i]\leftrightarrow A[j]$
\li			\While $i\le n$ i~$A[i]=0$
\li				\Do $i\gets i+1$
				\End
\li			\While $j\ge1$ i~$A[j]=1$
\li				\Do $j\gets j-1$
				\End
		\End
\end{codebox}

\subproblem %8-2(c)
Sortowanie przez wstawianie.

\subproblem %8-2(d)
Do posortowania $n$ rekordów \singledash{$b$}{bitowych} w~czasie $O(bn)$ za pomocą algorytmu sortowania pozycyjnego, potrzebna jest pomocnicza procedura, która sortuje stabilnie i~działa w~czasie $O(n)$.
Takie warunki spełnia jedynie algorytm z~części~(a).
W~każdej kolejnej fazie sortowania rekordy powinny być porządkowane względem coraz bardziej znaczących bitów.

\subproblem %8-2(e)
W~algorytmie wykorzystamy tablicę pomocniczą $C[0\twodots k]$ tworzoną identycznie jak w~wierszach \doubledash{1}{7} procedury \proc{Counting-Sort}.
Ideą algorytmu jest przejrzenie tablicy wejściowej w~poszukiwaniu elementów, które należą do niewłaściwych obszarów tablicy, a~następnie umieszczenie ich na właściwych pozycjach, które znajdujemy, korzystając z~informacji zebranych w~tablicy $C$ na początku działania algorytmu.
\begin{codebox}
\Procname{$\proc{Counting-Sort-In-Place}(A,k)$}
\li utwórz tablicę $C[0\twodots k]$ jak w~wierszach \doubledash{1}{7} procedury \proc{Counting-Sort} 
\li skopiuj zawartość tablicy $C$ do tablicy $C'$
\li $i\gets1$
\li \While $i\le\attrib{A}{length}-1$ \label{li:counting-sort-in-place-while-begin}
\li     \Do
            $\id{key}\gets A[i]$
\li         \If $C'[\id{key}-1]<i\le C'[\id{key}]$ \label{li:counting-sort-in-place-test}
\li             \Then $i\gets i+1$
\li             \Else
                    zamień $A[i]\leftrightarrow A[C[\id{key}]]$ \label{li:counting-sort-in-place-swap}
\li                 $C[\id{key}]\gets C[\id{key}]-1$
                \End
        \End \label{li:counting-sort-in-place-while-end}
\end{codebox}

Początkowy stan tablicy $C$ zostaje zapamiętany w~tablicy $C'$.
Następnie pętla \kw{while} w~wierszach \doubledash{\ref{li:counting-sort-in-place-while-begin}}{\ref{li:counting-sort-in-place-while-end}} przegląda tablicę wejściową w~poszukiwaniu elementów, które nie powinny znajdować się w~posortowanej tablicy na aktualnej pozycji.
Do sprawdzenia, czy element znajduje się we właściwym miejscu, używany jest test z~wiersza~\ref{li:counting-sort-in-place-test}.
W~przypadku negatywnego wyniku tego testu element z~aktualnej pozycji zostaje przeniesiony na swoją docelową pozycję po zamianie z~elementem, który tam się znajduje.
Ten ostatni będzie sprawdzany w~następnej iteracji pętli \kw{while}.

Algorytm sortuje tablicę wejściową w~miejscu (niekoniecznie stabilnie), wykorzystując $O(k)$ dodatkowej pamięci.
Czas potrzebny na utworzenie tablic $C$ i~$C'$ wynosi $O(n+k)$.
Zauważmy, że element, który zostanie umieszczony na swojej docelowej pozycji, nie zmieni jej aż do zakończenia działania algorytmu -- zostanie więc wykonanych co najwyżej $n-1$ zamian w~wierszu~\ref{li:counting-sort-in-place-swap}.
W~iteracjach pętli \kw{while}, w~których zamiany nie są dokonywane, jest inkrementowana zmienna $i$ -- sumaryczna liczba iteracji jest więc ograniczona przez $2(n-1)$.
Stąd czas działania algorytmu wynosi $O(n+k)$.

\problem{Sortowanie obiektów zmiennej długości} %8-3

\subproblem %8-3(a)
Opiszemy sposób sortowania tylko dla liczb nieujemnych.
Jeśli w~tablicy znajdują się liczby ujemne, to wystarczy posortować je osobno analogicznym sposobem do opisanego poniżej, a~następnie scalić z~uporządkowaną tablicą liczb nieujemnych.

Sortowanie odbywa się w~dwóch fazach.
Najpierw liczby porządkowane są według ilości cyfr, z~których się składają -- im większa ilość cyfr, tym liczba jest większa.
Wykorzystywany jest w~tym celu algorytm sortowania przez zliczanie, w~którym $k=n$.
Następnie liczby o~tej samej ilości cyfr sortowane są pozycyjnie.

W~tablicy wejściowej może znaleźć się maksymalnie $n$ liczb, zatem pierwsza część algorytmu zajmuje czas $O(n)$.
Wyznaczmy teraz czas działania drugiej części.
W~tym celu oznaczmy przez $m_i$ ilość liczb o~$i$ cyfrach w~tablicy.
Zachodzi wówczas $\sum_{i=1}^nim_i=n$.
Sortowanie pozycyjne podtablicy liczb o~$i$ cyfrach zajmuje czas $\Theta(im_i)$, a~więc całkowity czas wymagany przez drugą fazę algorytmu wynosi
\[
    \sum_{i=1}^n\Theta(im_i) = \Theta\biggl(\sum_{i=1}^nim_i\biggr) = \Theta(n).
\]
Stąd oczywiście wynika, że algorytm działa w~czasie $O(n)$.

\subproblem %8-3(b)
Skorzystajmy z~obserwacji, że jeśli pierwsza litera napisu $x$ jest leksykograficznie mniejsza od pierwszej litery napisu $y$, to $x$ wystąpi w~wynikowej tablicy przed $y$.
Można zatem sortować napisy przez zliczanie według ich pierwszych liter, a~następnie, w~obrębie każdej grupy napisów o~takiej samej literze początkowej, sortować napisy według ich drugiej litery itd.
Pamiętajmy jednak, że napisy mają różną długość, dlatego po ich posortowaniu względem \singledash{$i$}{tych} liter, należy wyłączyć z~kolejnej fazy słowa o~dokładnie $i$ literach, umieszczając je w~tablicy wynikowej przed grupą napisów, które będą sortowane w~kolejnej fazie.

Zauważmy, że napis $x$ składający się z~$i$ liter będzie brał udział w~co najwyżej $i$ fazach sortowania.
Niech $m_i$ oznacza liczbę napisów na wejściu posiadających dokładnie $i$ liter.
Wówczas operacje odpowiedzialne za sortowanie napisów \singledash{$i$}{literowych} wymagają czasu $O(im_i)$.
Korzystając z~faktu, że $\sum_{i=1}^nim_i=n$, otrzymujemy, że czasem działania algorytmu jest
\[
    \sum_{i=1}^nO(im_i) = O\biggl(\sum_{i=1}^nim_i\biggr) = O(n).
\]

\problem{Dzbanki} %8-4

\subproblem %8-4(a)
Dla każdego czerwonego dzbanka wystarczy przejrzeć wszystkie dotychczas niesparowane niebieskie dzbanki.
Po znalezieniu pasującego możemy wyłączyć go z~poszukiwań dla kolejnych czerwonych dzbanków.
Łatwo sprawdzić, że ta metoda pozwala na pogrupowanie wszystkich dzbanków w~pary za pomocą $\Theta(n^2)$ porównań.

\subproblem %8-4(b)
Ponumerujmy czerwone dzbanki kolejnymi liczbami całkowitymi od~1 do~$n$ i~analogicznie dzbanki niebieskie.
Problem sprowadza się do znalezienia takiej permutacji $\pi$ niebieskich dzbanków, w~której \singledash{$i$}{ty} dzbanek czerwony jest tej samej pojemności, co dzbanek niebieski o~numerze $\pi(i)$.

Zilustrujemy za pomocą drzewa decyzyjnego działanie algorytmu wyznaczającego tę permutację.
Każdy węzeł w~tym drzewie będzie odpowiadać porównaniu pewnego dzbanka czerwonego z~pewnym dzbankiem niebieskim.
Wynikiem każdego takiego porównania jest jedna z~trzech możliwości: dzbanek czerwony ma mniejszą pojemność niż dzbanek niebieski, dzbanek czerwony ma większą pojemność niż dzbanek niebieski albo oba dzbanki mają tę samą pojemność.
Z~tego powodu drzewo decyzyjne jest drzewem \singledash{3}{arnym} o~$n!$ osiągalnych liściach, bo tyle jest permutacji niebieskich dzbanków.
Wysokość $h$ tego drzewa oznacza pesymistyczną liczbę porównań wykonywanych przez algorytm.
Każdy węzeł w~tym drzewie ma co najwyżej 3 synów, prawdziwa jest zatem nierówność $3^h\ge n!$, skąd
\[
	h \ge \log_3(n!) = \frac{\lg(n!)}{\lg3} = \Omega(n\lg n).
\]
A~zatem dowolny algorytm rozwiązujący ten problem wykonuje co najmniej $\Omega(n\lg n)$ porównań.

\subproblem %8-4(c)
Rozwiązanie tego problemu będzie opierać się na idei algorytmu quicksort.
Zauważmy, że posortowanie obu ciągów dzbanków względem ich pojemności natychmiast prowadzi do rozwiązania problemu -- grupujemy wówczas w~pary dzbanek czerwony z~dzbankiem niebieskim stojące na tych samych pozycjach w~swoich ciągach.
Niech $R$ będzie tablicą dzbanków czerwonych, a~$B$ tablicą dzbanków niebieskich.
Poniższa procedura stanowi adaptację algorytmu quicksort do omawianego problemu.
\begin{codebox}
\Procname{$\proc{Jugs-Match}(R,B,p,r)$}
\li	\If $p<r$
\li		\Then
			$q\gets\proc{Jugs-Partition}(R,B,p,r)$
\li			$\proc{Jugs-Match}(R,B,p,q-1)$
\li			$\proc{Jugs-Match}(R,B,q+1,r)$
		\End
\end{codebox}

Poniżej znajduje się zmodyfikowana procedura \proc{Randomized-Partition}.
Przyjmuje ona dwie tablice $R$ i~$B$, z~których jedna jest permutacją drugiej i~w~każdej z~nich wszystkie elementy są parami różne.
Przyjmujemy, że operacja porównania dwóch dzbanków jest w~rzeczywistości porównaniem ich pojemności.
Procedura dokonuje podziału obu tablic na podstawie losowo wybranego elementu rozdzielającego, przy czym nie porównuje elementów z~tej samej tablicy.
Zwracanym wynikiem jest pozycja elementu rozdzielającego (która jest identyczna w~obu wynikowych tablicach).
\begin{codebox}
\Procname{$\proc{Jugs-Partition}(R,B,p,r)$}
\li	zamień $R[r]\leftrightarrow R[\proc{Random}(p,r)]$
\li	$x\gets R[r]$
\li	$i\gets p$
\li	\While $B[i]\ne x$
\li		\Do $i\gets i+1$
		\End
\li	zamień $B[i]\leftrightarrow B[r]$
\li	$i\gets p-1$ \label{li:jugs-partition-first-partition-begin}
\li	\For $j\gets p$ \To $r-1$ \label{li:jugs-partition-for1-begin}
\li		\Do
			\If $B[j]<x$
\li				\Then
					$i\gets i+1$
\li					zamień $B[i]\leftrightarrow B[j]$
				\End
		\End \label{li:jugs-partition-for1-end}
\li	zamień $B[i+1]\leftrightarrow B[r]$ \label{li:jugs-partition-first-partition-end}
\li	$x\gets B[i+1]$
\li	powtórz kroki z~wierszy \doubledash{\ref{li:jugs-partition-first-partition-begin}}{\ref{li:jugs-partition-first-partition-end}} dla tablicy $R$ \label{li:jugs-partition-second-partition}
\li	\Return $i+1$
\end{codebox}

Procedura wybiera najpierw losowo element rozdzielający $x$ z~tablicy $R[p\twodots r]$.
Elementy równe $x$ są dla uproszczenia późniejszych kroków przenoszone na ostatnie pozycje swoich tablic.
Następnie pętla \kw{for} z~wierszy \doubledash{\ref{li:jugs-partition-for1-begin}}{\ref{li:jugs-partition-for1-end}} dokonuje podziału tablicy $B[p\twodots r]$ względem $x$.
Po zakończeniu pętli $B[r]=x$, podtablica $B[p\twodots i]$ zawiera elementy mniejsze niż $x$, a~podtablica $B[i+1\twodots r-1]$ -- elementy większe niż $x$.
Wystarczy jeszcze zamienić element rozdzielający z~$B[i+1]$, aby zakończyć podział tablicy $B$.
W~celu przeprowadzenia podziału tablicy $R$ algorytm wykonuje analogiczne kroki z~wierszy \doubledash{\ref{li:jugs-partition-first-partition-begin}}{\ref{li:jugs-partition-first-partition-end}}, uprzednio nadając zmiennej $x$ wartość dotychczasowego elementu rozdzielającego, ale pobranego z~tablicy $B$.

Udowodnimy teraz, że oczekiwana liczba porównań wykonywanych przez algorytm \proc{Jugs-Match} wynosi $O(n\lg n)$.
Nasza analiza będzie opierać się na analizie średniego przypadku algorytmu quicksort.
Niech $\langle r_1,r_2,\dots,r_n\rangle$ będzie ciągiem dzbanków czerwonych uporządkowanym rosnąco, a~$\langle b_1,b_2,\dots,b_n\rangle$ -- rosnącym ciągiem dzbanków niebieskich.
Definiujemy zmienną losową wskaźnikową
\[
    X_{ij} = \I(\text{$r_i$ jest porównywane z~$b_j$}).
\]
Zauważmy, że dane elementy $r_i$ i~$b_j$ są porównywane ze sobą co najwyżej raz.
Po porównaniu $r_i$ z~elementami fragmentu $B$ (w~tym być może z~$b_j$) $r_i$ jest umieszczane na właściwym miejscu w~tablicy $R$ i~nie uczestniczy w~późniejszych wywołaniach procedury \proc{Jugs-Partition}.
Analogiczne rozumowanie stosuje się do $b_j$.

Z~powyższej obserwacji wynika, że całkowita liczba porównań wykonywanych w~algorytmie wynosi
\[
    X = \sum_{i=1}^{n-1}\sum_{j=i+1}^nX_{ij}.
\]
Biorąc wartości oczekiwane obu stron, dostajemy
\[
    \E(X) = \sum_{i=1}^{n-1}\sum_{j=i+1}^n\Pr(\text{$r_i$ jest porównywane z~$b_j$}).
\]

Pozostaje teraz tylko obliczyć $\Pr(\text{$r_i$ jest porównywane z~$b_j$})$.
Jeśli jako element rozdzielający wybrane zostanie $x$ takie, że $r_i<x<b_j$, to $r_i$ zostanie umieszczone w~lewej części podziału podtablicy $R$, a~$b_j$ -- w~prawej części podziału podtablicy $B$.
A~więc elementy $r_i$ i~$b_j$ nigdy więcej nie będą porównywane.
Oznaczmy przez $R_{ij}$ zbiór $\{r_i,r_{i+1},\dots,r_j\}$.
Wówczas elementy $r_i$ i~$b_j$ są porównywane wtedy i~tylko wtedy, gdy pierwszym elementem wybranym jako rozdzielający ze zbioru $R_{ij}$ jest $r_i$ albo $r_j$.
To, że zostanie nim dowolny ustalony element zbioru $R_{ij}$, zachodzi z~jednakowym prawdopodobieństwem i~na mocy faktu, że $|R_{ij}|=j-i+1$, szanse na zajście tego zdarzenia wynoszą $1/(j-i+1)$.
Mamy zatem
\begin{align*}
    \Pr(\text{$r_i$ jest porównywane z~$b_j$}) &= \Pr(\text{$r_i$ lub $r_j$ jest pierwszym elementem rozdzielającym z~$R_{ij}$}) \\
	&= \Pr(\text{$r_i$ jest pierwszym elementem rozdzielającym z~$R_{ij}$}) \\
	&\quad {}+\Pr(\text{$r_j$ jest pierwszym elementem rozdzielającym z~$R_{ij}$}) \\
	&= \frac{1}{j-i+1}+\frac{1}{j-i+1} \\
	&= \frac{2}{j-i+1}.
\end{align*}
Wstawiając obliczony wynik do wzoru na $\E(X)$ i~ograniczając tę wartość od góry, dostajemy $\E(X)=O(n\lg n)$, co stanowi oczekiwaną liczbę porównań wykonywaną przez algorytm \proc{Jugs-Match}.

Pesymistyczny przypadek zachodzi wtedy, gdy na każdym poziomie rekursji elementy rozdzielające są wybierane tak, że tworzą się podziały najbardziej niezrównoważone, czyli z~tablicy o~$n$ elementach zostaje utworzona w~wyniku podziału podtablica o~$n-1$ elementach i~podtablica pusta.
Wówczas algorytm wykonuje $O(n^2)$ porównań.

\problem{Sortowanie względem średnich} %8-5

\subproblem %8-5(a)
Zgodnie z~definicją tablica $A[1\twodots n]$ jest \singledash{1}{posortowana}, jeśli dla każdego $i=1$, 2,~\dots,~$n-1$ zachodzi $A[i]\le A[i+1]$, a~to jest równoważne temu, że tablica $A$ jest posortowana niemalejąco.

\subproblem %8-5(b)
Jedną z~takich permutacji jest $\langle2,1,5,3,7,4,8,6,10,9\rangle$.

\subproblem %8-5(c)
Aby udowodnić ten fakt, wystarczy nierówność
\[
	\frac{\sum_{j=i}^{i+k-1}A[j]}{k} \le \frac{\sum_{j=i+1}^{i+k}A[j]}{k}
\]
pomnożyć przez $k$ i~zredukować te same składniki po obu stronach znaku nierówności.

\subproblem %8-5(d)
Jednym ze sposobów \singledash{$k$}{posortowania} tablicy $A[1\twodots n]$ jest zastosowanie algorytmu quicksort, którego rekursja zatrzymuje się dla podtablic o~rozmiarach nieprzekraczających $k$.
Dokładniej, należy zamienić warunek z~wiersza~1 procedury \proc{Quicksort} na następujący:
\begin{codebox}
\li	\If $p+k-1<r$
\end{codebox}
Po zakończeniu działania algorytmu mamy, że $A[i]\le A[i+k]$ dla każdego $i=1$, 2,~\dots,~$n-k$.
Warunek ten, na podstawie części~(c), jest równoważny temu, że tablica $A$ jest \singledash{$k$}{posortowana}.

W~\refExercise{7.4-5} pokazaliśmy, że quicksort po takiej modyfikacji działa w~oczekiwanym czasie $O(n\lg(n/k))$.
Aby było to oszacowanie na czas w~przypadku pesymistycznym, musimy zagwarantować najlepszy przypadek wyboru elementów rozdzielających.
Można to zrealizować poprzez wybieranie do tej roli mediany bieżącej podtablicy, jak to opisano w~\refExercise{9.3-3}.

\subproblem %8-5(e)
W~\singledash{$k$}{posortowanej} tablicy każdy z~ciągów postaci $\langle A[j],A[j+k],A[j+2k],\dots,A[j+m_jk]\rangle$, gdzie $j=1$, 2~\dots,~$k$ i~$j+m_jk\le n$, jest uporządkowany niemalejąco i~wszystkie one zawierają łącznie $n$ elementów.
Wystarczy więc scalić je w~jedną posortowaną tablicę, co na podstawie \refExercise{6.5-8} można wykonać w~czasie $O(n\lg k)$.

\subproblem %8-5(f)
Przyjmiemy dla wygody, że $n$ jest podzielne przez $k$ -- nie spowoduje to zmniejszenia ogólności naszej analizy.

W~dowodzie dolnego ograniczenia czasowego tego problemu wykorzystamy model drzew decyzyjnych.
Niech $T$ będzie drzewem decyzyjnym pewnego algorytmu \singledash{$k$}{sortowania} za pomocą porównań tablicy o~$n$ elementach.
Drzewo $T$ jest podobne do drzewa decyzyjnego algorytmu zwykłego sortowania za pomocą porównań -- każdy jego liść odpowiada pewnej permutacji tablicy wejściowej.
Jednak liczba osiągalnych liści jest na ogół mniejsza niż w~drzewie zwykłego sortowania.
Wynikowa tablica składa się z~$n/k$ podtablic \singledash{$k$}{elementowych}, w~których uporządkowanie elementów nie ma znaczenia.
Osiągalnych liści w~tym drzewie jest zatem
\[
    \frac{n!}{(k!)^{n/k}}.
\]

Niech $h$ oznacza wysokość drzewa $T$.
Liczba liści w~$T$ nie przekracza $2^h$, więc
\[
    h \ge \lg\frac{n!}{(k!)^{n/k}} = \lg(n!)-(n/k)\lg(k!) = \Omega(n\lg n),
\]
ponieważ traktujemy $k$ jako stałą.
Wyznaczone oszacowanie na $h$ stanowi dolne ograniczenie na czas działania dowolnego algorytmu \singledash{$k$}{sortowania} za pomocą porównań tablicy o~$n$ elementach.


\problem{Dolna granica dla scalania posortowanych list} %8-6

\subproblem %8-6(a)
Spośród $2n$ różnych liczb możemy wybrać $n$ z~nich do pierwszej listy, a~pozostałe $n$ -- do drugiej listy.
Ponieważ każdy taki podział jednoznacznie determinuje parę posortowanych list i~na odwrót, to liczba sposobów utworzenia tej pary list wynosi $\binom{2n}{n}$.

\subproblem %8-6(b)
Działanie algorytmu scalania list można przedstawić, korzystając z~drzewa decyzyjnego, w~którym każdy węzeł odpowiada jednemu porównaniu elementów list.
Wszystkie elementy są różne, mamy więc tylko dwa możliwe wyniki każdego z~porównań, a~zatem drzewo decyzyjne jest drzewem binarnym.
Na podstawie poprzedniego punktu mamy, że liczba możliwych układów dwóch \singledash{$n$}{elementowych} list wejściowych, z~połączenia których powstaje wynikowa lista o~$2n$ elementach, wynosi $\binom{2n}{n}$.
W~drzewie decyzyjnym jest więc tyleż osiągalnych liści.
Wyznaczając oszacowanie wysokości $h$ tego drzewa, znajdziemy ograniczenie dla maksymalnej liczby porównań wykonanych przez algorytm scalania list.
Mamy $2^h\ge\binom{2n}{n}$ i~korzystając z~\refExercise{C.1-13}, dostajemy
\[
    h \ge \lg\binom{2n}{n} = \lg\biggl(\frac{2^{2n}}{\sqrt{\pi n}}(1+O(1/n))\biggr) \ge \lg\frac{2^{2n}}{\sqrt{\pi n}} = 2n-\frac{\lg\pi}{2}-\frac{\lg n}{2} = 2n-o(n).
\]

\subproblem %8-6(c)
Niech $\langle a_1,a_2,\dots,a_n\rangle$ oraz $\langle b_1,b_2,\dots,b_n\rangle$ będą listami wejściowymi.
Załóżmy, że w~liście wynikowej element $a_i$ sąsiaduje z~elementem $b_j$ dla $1\le i$, $j\le n$.
Jeśli algorytm, scalając obie listy wejściowe, porównywałby $a_i$ z~elementem $b_k$, gdzie $1\le k\le j-1$, to wynikiem porównania byłoby $b_k<a_i$, ale stąd nie wynikałoby ani $b_j<a_i$, ani $a_i<b_j$.
Analogicznie przy porównaniu $a_i$ z~elementem $b_k$, gdzie $j+1\le k\le n$.
A~zatem, aby rozstrzygnąć, w~jakiej kolejności powinny wystąpić $a_i$ i~$b_j$ w~liście wynikowej, algorytm musi porównać te dwa elementy ze sobą.

\subproblem %8-6(d)
Jeśli do~wynikowej listy będą trafiać elementy z~pierwszej listy na przemian z~elementami z~drugiej, to każde dwa sąsiednie elementy w~wynikowej liście będą pochodzić z~dwóch różnych list wejściowych.
Takich par sąsiadujących elementów będzie w~sumie $2n-1$.
Na mocy poprzedniego punktu wnioskujemy, że w~tym przypadku algorytm scalania list wykona tyleż porównań.

\endinput
