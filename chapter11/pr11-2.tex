\problem{Długość listy w~metodzie łańcuchowej} %11-2

\subproblem %11-2(a)
Potraktujmy odwzorowywanie kluczy jako ciąg prób Bernoulliego, w~których sukcesem jest odwzorowanie klucza na ustaloną pozycję tablicy.
Prawdopodobieństwo sukcesu każdej takiej próby jest równe $1/n$.
Z~rozkładu dwumianowego wynika, że uzyskanie dokładnie $k$ sukcesów w~tej serii prób jest równe
\[
	Q_k = b(k;n,1/n) = \binom{n}{k}\biggl(\frac{1}{n}\biggr)^k\biggl(1-\frac{1}{n}\biggr)^{n-k}.
\]

\subproblem %11-2(b)
Niech $A_i$, dla $i=1$, 2, \dots, $n$, oznacza zdarzenie, że liczba elementów odwzorowanych na $i$\nbhyphen tą pozycję tablicy wynosi $k$.
Wówczas zdarzenie $A$, że $M=k$, spełnia inkluzję $A\subseteq\bigcup_{i=1}^nA_i$.
Z~własności prawdopodobieństwa i~z~punktu (a) mamy
\[
	P_k = \Pr(A) \le \Pr\biggl(\bigcup_{i=1}^nA_i\biggr) \le \sum_{i=1}^n\Pr(A_i) = \sum_{i=1}^nQ_k = nQ_k.
\]

\subproblem %11-2(c)
Załóżmy, że $k>0$.
Z~definicji współczynnika dwumianowego:
\[
	\binom{n}{k} = \frac{n!}{k!(n-k)!} = \frac{(n-k+1)(n-k+2)\dots n}{k!}.
\]
Każdy z~czynników w~liczniku ograniczamy od góry przez $n$.
Z~drugiej strony, na podstawie wzoru Stirlinga, mamy $k!=\sqrt{2\pi k}\,(k/e)^k(1+\Theta(1/k))>(k/e)^k$.
Stąd
\[
	\binom{n}{k} < \frac{n^k}{(k/e)^k} = \biggl(\frac{ne}{k}\biggr)^k.
\]
Wykorzystując tę nierówność, mamy
\[
	Q_k = \biggl(\frac{1}{n}\biggr)^k\biggl(1-\frac{1}{n}\biggr)^{n-k}\binom{n}{k} < \frac{(n-1)^{n-k}}{n^n}\biggl(\frac{ne}{k}\biggr)^k = \biggl(\frac{n-1}{n}\biggr)^{n-k}\biggl(\frac{e}{k}\biggr)^k < \frac{e^k}{k^k}.
\]

\subproblem %11-2(d)
\note{W~drugiej części zadania należy wywnioskować, że\/ $P_k<1/n^2$ dla\/ $k\ge k_0=c\lg n/\!\lg\lg n$.}

\noindent Badając wyrażenie $\lg Q_{k_0}$, wyznaczymy odpowiednie $c$ tak, aby zachodziło $Q_{k_0}<1/n^3$, gdzie $k_0=c\lg n/\!\lg\lg n$.
Z~poprzedniego punktu mamy, że $Q_{k_0}<e^{k_0}\!/{k_0}^{k_0}$, a~więc
\begin{align*}
	\lg Q_{k_0} &< k_0\lg e-k_0\lg k_0 \\
	&= \frac{c\lg n}{\lg\lg n}\biggl(\lg e-\lg\frac{c\lg n}{\lg\lg n}\biggr) \\[1mm]
	&= \frac{c\lg n}{\lg\lg n}(\lg e-\lg c-\lg\lg n+\lg\lg\lg n).
\end{align*}
Nierówność $Q_{k_0}<1/n^3$ jest równoważna $\lg Q_{k_0}<-3\lg n$, zatem na podstawie powyższego:
\[
	\frac{c\lg n}{\lg\lg n}(\lg e-\lg c-\lg\lg n+\lg\lg\lg n) < -3\lg n.
\]
Po podzieleniu obu stron tej nierówności przez $-\lg n$ dostajemy
\[
	c\biggl(1-\frac{\lg\lg\lg n}{\lg\lg n}+\frac{\lg(c/e)}{\lg\lg n}\biggr) > 3.
\]
Można pokazać, że $\lg\lg\lg n/\lg\lg n<1$.
Zatem dla $c\ge e$ wyrażenie w~nawiasie jest dodatnie i~wybierając wystarczająco duże $c$, możemy spełnić nierówność.

Ustalmy $c$ tak, aby $k_0>e$.
Wówczas $e/k_0<1$ i~dla $k\ge k_0$ wyrażenie $(e/k)^k$ maleje wraz ze wzrostem $k$.
Łącząc tę obserwację z~wynikami z~poprzednich części, otrzymujemy, że
\[
	P_k\le nQ_k < ne^k\!/k^k \le ne^{k_0}\!/k_0^{k_0} < n\cdot1/n^3 = 1/n^2
\]
dla dowolnego $k\ge k_0$.

\subproblem %11-2(e)
Niech $k_0=c\lg n/\!\lg\lg n$.
Wówczas
\begin{align*}
	\E(M) &= \sum_{k=0}^nkP_k = \sum_{k=0}^{k_0}kP_k+\sum_{k=k_0+1}^nkP_k \\
	&\le k_0\sum_{k=0}^{k_0}P_k+n\sum_{k=k_0+1}^nP_k = k_0\Pr(M\le k_0)+n\Pr(M>k_0).
\end{align*}

W~celu uzasadnienia górnego oszacowania dla $\E(M)$, wykorzystamy fakt, który udowodniliśmy w~punkcie (d):
\begin{align*}
	\E(M) &\le k_0\Pr(M\le k_0)+n\Pr(M>k_0) = k_0\cdot1+n\sum_{k=k_0+1}^nP_k \\
	&< k_0+n\sum_{k=k_0+1}^n1/n^2 \le k_0+n\cdot n\cdot1/n^2 = k_0+1 = O(\lg n/\!\lg\lg n).
\end{align*}
