\subchapter{Tablice z~adresowaniem bezpośrednim}

\exercise %11.1-1
Procedura wyszukująca największy element zbioru $S$ będzie przeglądać liniowo tablicę $T$ od końca, to znaczy od indeksu $m-1$ ku jej początkowi, i~zwróci element z~pierwszej niepustej pozycji.
Jeśli zbiór $S$ jest pusty, to na wszystkich pozycjach tablicy $T$ znajduje się wartość \const{nil} i~wówczas procedura przeglądnie całą tablicę, po czym zwróci \const{nil}.
Jest to przypadek pesymistyczny, który wymaga czasu $\Theta(m)$.

\exercise %11.1-2
W~skład elementów wchodzą tylko klucze, wystarczy więc pamiętać tylko, czy dany klucz należy do zbioru, czy nie.
Na pozycji $k$ wektora bitowego o~długości $m$ będzie znajdować się jedynka wtedy i~tylko wtedy, gdy element o~kluczu $k$ należy do zbioru dynamicznego.
Wyszukiwanie elementu o~kluczu $k$ polega na odczytaniu \singledash{$k$}{tej} pozycji wektora, dodanie tego elementu do zbioru -- na wpisaniu na tę pozycję jedynki, a~usunięcie -- na wpisaniu tam zera.

\exercise %11.1-3
W~naszej implementacji elementy przechowywane w~tablicy $T$, oprócz klucza i~dodatkowych danych, zawierać będą wskaźniki \id{prev} i~\id{next}, dzięki którym elementy będą mogły zostać uszeregowane w~listy dwukierunkowe.
Komórka o~indeksie $k$ w~tablicy $T$ będzie zawierać głowę listy elementów o~kluczu $k$ znajdujących się aktualnie w~tablicy albo \const{nil}, jeżeli lista ta jest pusta.

Dodanie nowego elementu $x$ będzie polegać na dodaniu go do listy znajdującej się w~$T[\attrib{x}{key}]$.
Ponieważ operacja \proc{Delete} przyjmuje wskaźnik do elementu $x$, a~nie jego klucz, możliwe jest zrealizowanie jej poprzez natychmiastowe usunięcie elementu $x$ z~listy w~$T[\attrib{x}{key}]$.
Wreszcie działanie operacji wyszukującej element o~kluczu $k$ będzie polegać na zwróceniu wartości $T[k]$.
Każda z~opisanych operacji działa w~czasie $\Theta(1)$.

Opisaną strukturę danych można traktować jak tablicę z~haszowaniem wykorzystującą łańcuchową metodę rozwiązywania kolizji z~identycznością jako funkcją haszującą.

\exercise %11.1-4
\note{Dodatkową strukturą danych, jaką należy użyć według wskazówki, powinna być tablica traktowana jak stos.}

\noindent Przez $T$ oznaczymy ogromną tablicę, a~przez $S$ -- korzystając ze wskazówki z~treści zadania -- dodatkową tablicę, za pomocą której będziemy odwoływać się do odpowiednich pozycji w~$T$.
Tablicę $S$ będziemy traktować jak stos elementów przechowywanych w~$T$ i~będziemy używać atrybutu \attrib{S}{top} wskazującego na ostatnią zajętą komórkę w~$S$.
Rozmiar tablicy $S$, jaki ustalimy podczas jej tworzenia, określa pojemność ogromnej tablicy -- próba umieszczenia w~tej strukturze większej ilości elementów zakończy się błędem nadmiaru wykrywanym przez odpowiednią implementację operacji \proc{Push}.
Jeśli rozmiarem ogromnej tablicy jest $m$, to można w~niej przechowywać klucze z~uniwersum $U=\{0,1,\dots,m-1\}$.

Powiemy, że klucz $k$ należy do zbioru dynamicznego reprezentowanego przez ogromną tablicę $T$ wtedy i~tylko wtedy, gdy $T[k]$ przechowuje poprawny indeks $j$ tablicy $S$, a~$S[j]$ zawiera element o~kluczu $k$.
Innymi słowy, zachodzą zależności: $1\le T[k]\le\attrib{S}{top}$, $\attrib{S[T[k]]}{key}=k$ oraz $T[\attrib{S[j]}{key}]=j$.
Tablice $T$ i~$S$ weryfikują się więc wzajemnie.

Inicjowanie tablicy $T$ sprowadza się do utworzenia tablicy $S$.
Aby wyszukać element o~kluczu $k$, wystarczy sprawdzić, czy $1\le T[k]\le\attrib{S}{top}$ i~$\attrib{S[T[k]]}{key}=k$.
Jeśli warunki te są spełnione, to należy zwrócić element $S[T[k]]$, w~przeciwnym przypadku zaś \const{nil}.
W~celu wstawienia elementu $x$ (przy założeniu, że nie ma go jeszcze w~tablicy) należy umieścić $x$ na stosie $S$, po czym zaktualizować $T[\attrib{x}{key}]$:
\begin{codebox}
\Procname{$\proc{Huge-Array-Insert}(T,S,x)$}
\li	$\proc{Push}(S,x)$
\li	$T[\attrib{x}{key}]\gets\attrib{S}{top}$
\end{codebox}
Usuwanie elementu $x$ o~kluczu $k$ polega na przeniesieniu elementu usuniętego ze szczytu stosu $S$ na pozycję $T[k]$ w~tym stosie oraz aktualizacji odpowiedniej wartości w~tablicy $T$.
Dokładniej, operacja ta sprowadza się do poniższego pseudokodu:
\begin{codebox}
\Procname{$\proc{Huge-Array-Delete}(T,S,x)$}
\li	$k\gets\attrib{x}{key}$
\li	$y\gets\proc{Pop}(S)$
\li	$S[T[k]]\gets y$
\li	$T[\attrib{y}{key}]\gets T[k]$
\end{codebox}

Wszystkie omówione operacje działają w~czasie $\Theta(1)$, a~każdy element znajdujący się w~ogromnej tablicy (nie licząc dodatkowych danych z~nim związanych) zajmuje $\Theta(1)$ pamięci.
