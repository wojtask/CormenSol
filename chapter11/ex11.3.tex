\subchapter{Funkcje haszujące}

\exercise %11.3-1
W~celu odnalezienia elementu o~kluczu $k$ obliczamy $h(k)$ i~przeglądamy listę, porównując jedynie wartości funkcji haszującej napotykanych kluczy, co jest znacznie szybsze od porównywania samych kluczy będącymi długimi ciągami znaków.
Gdy znajdziemy element, dla którego wartość funkcji haszującej jest równa $h(k)$, to dopiero wówczas sprawdzamy, czy kluczem tego elementu jest $k$.

\exercise %11.3-2
Reprezentacją napisu $x=x_0x_1\dots x_{r-1}$ jest liczba $\sum_{i=0}^{r-1}x_i128^i$ (dla uproszczenia zapisu utożsamiamy znak z~odpowiadającą mu wartością w~kodzie ASCII).
Obliczenie $h(x)$ polega na wyznaczeniu reszty z~dzielenia reprezentacji napisu $x$ przez $m$.
W~tym celu przedstawimy $h(x)$ w~postaci
\[
	h(x) = x_0+128(x_1+128(x_2+\dots+128(x_{r-2}+128x_{r-1})\cdots)),
\]
przy czym dodawanie i~mnożenie są tutaj działaniami w~zbiorze $\mathbb{Z}_m=\{0,1,\dots,m-1\}$.
Następnie stosujemy schemat Hornera (patrz problem \refProblem{2-3}).
Utrzymywanie pośrednich iloczynów i~sum w~zbiorze $\mathbb{Z}_m$ pozwala zmieścić je w~co najwyżej dwóch słowach maszynowych.

\exercise %11.3-3
Jeśli napis $x=x_0x_1\dots x_{r-1}$ jest permutacją napisu $y=y_0y_1\dots y_{r-1}$, to $x$ możemy uzyskać z~$y$, zamieniając ze sobą znaki występujące na tych samych pozycjach w~obu napisach, dopóki oba napisy są różne.
Można pokazać, że zawsze istnieje skończony ciąg pozycji, na których należy zamieniać znaki w~celu przekształcenia jednego napisu w~drugi.
Założymy zatem, że napis $x$ można uzyskać z~$y$, dokonując tylko jednej takiej zamiany, tzn.\ $x_a=y_b$ oraz $y_a=x_b$ dla pewnych $0\le a<b\le r-1$ oraz $x_i=y_i$ dla $i\ne a$ i~$i\ne b$.
Łatwo pokazać przez indukcję, że jeśli występuje kolizja dla takiej pary napisów, to ma ona miejsce również dla par napisów różniących się więcej niż jedną parą znaków.

Podobnie jak w~poprzednim zadaniu będziemy traktować zamiennie znak oraz jego reprezentację w~kodzie ASCII\@.
Mamy
\begin{align*}
	h(x) &= \biggl(\sum_{i=0}^{r-1}x_i2^{ip}\biggr)\bmod m \\
	&= \sum_{i=0}^{r-1}(x_i\bmod m)(2^{ip}\bmod(2^p-1)) \\
	&= \sum_{i=0}^{r-1}(x_i\bmod m)(2^p\bmod(2^p-1))^i \\
	&= \sum_{i=0}^{r-1}(x_i\bmod m).
\end{align*}
Znaki napisu $x$ możemy oczywiście sumować w~dowolnej kolejności, zatem
\[
	h(x) = \sum_{i=0}^{r-1}(x_i\bmod m) = \sum_{i=0}^{r-1}(y_i\bmod m) = h(y).
\]

Użycie $h$ w~roli funkcji haszującej może doprowadzić do dużej ilości kolizji w~tablicy z~haszowaniem, w~której przechowywanych jest wiele napisów składających się z~tych samych znaków, ale w~zmienionej kolejności.
Przykładami są zbiory anagramów albo zbiory sekwencji DNA o~jednakowej długości.

\exercise %11.3-4
Rozmiar tablicy $m$ nie jest potęgą 2, dlatego nie możemy wyznaczyć wartości funkcji $h$ za pomocą metody opisanej w~Podręczniku.
Obliczając je bezpośrednio ze wzoru, mamy:
\begin{gather*}
	h(61) = 700, \\
	h(62) = 318, \\
	h(63) = 936, \\
	h(64) = 554, \\
	h(65) = 172.
\end{gather*}

\exercise %11.3-5
Wprowadźmy oznaczenia $u=|U|$ oraz $b=|B|$.
Dla ustalonej funkcji haszującej $h\in\mathcal{H}$ i~dla każdego elementu $j\in B$ niech $u_j$ będzie liczbą elementów z~$U$, które funkcja $h$ odwzorowuje na $j$.
Wówczas liczba kolizji powstałych na danym elemencie $j\in B$ wynosi $\binom{u_j}{2}=u_j(u_j-1)/2$.

Pokażemy, że sumaryczna liczba kolizji generowanych przez funkcję $h$ jest minimalna, gdy $u_j=u/b$ dla każdego $j\in B$.
Załóżmy, że dla pewnych dwóch różnych $i$, $k\in B$ zachodzi $0<u_i\le u/b\le u_k$.
Łączna liczba kolizji na elementach $i$ i~$k$ wynosi $S=u_i(u_i-1)/2+u_k(u_k-1)/2$.
Niech teraz $h'$ będzie funkcją identyczną z~$h$ z~jednym wyjątkiem -- dla dokładnie jednego dowolnie wybranego elementu $x\in U$ takiego, że $h(x)=i$, niech zachodzi $h'(x)=k$.
Funkcja $h'$ tworzy $S'=(u_i-1)(u_i-2)/2+(u_k+1)u_k/2$ kolizji na elementach $i$ i~$k$.
Zbadajmy znak różnicy $S'-S$:
\begin{align*}
	S'-S &= (u_i-1)(u_i-2)/2+(u_k+1)u_k/2-u_i(u_i-1)/2-u_k(u_k-1)/2 \\
	&= (u_i^2-3u_i+2+u_k^2+u_k-u_i^2+u_i-u_k^2+u_k)/2 \\
	&= (2u_k-2u_i+2)/2 \\
	&= u_k-u_i+1 \\
	&> 0.
\end{align*}
Otrzymaliśmy, że $S'>S$, czyli że sumaryczna liczba kolizji zwiększa się, kiedy wartości $u_j$ oddalają się od $u/b$.
Stąd wniosek, że liczba kolizji jest możliwie najmniejsza, gdy dla wszystkich elementów $j\in B$ zachodzi $u_j=u/b$.

Na podstawie powyższego faktu dostajemy, że dowolnie wybrana funkcja haszująca z~$\mathcal{H}$ generuje łącznie co najmniej $b(u/b)(u/b-1)/2$ kolizji.
Stąd prawdopodobieństwo $p_{\mathcal{H}}$ wystąpienia kolizji przy losowo wybranej funkcji z~$\mathcal{H}$ można ograniczyć od dołu przez
\[
	p_{\mathcal{H}} \ge \frac{b(u/b)(u/b-1)/2}{u(u-1)/2} = \frac{u/b-1}{u-1} > \frac{u/b-1}{u} = \frac{1}{b}-\frac{1}{u}.
\]
Jeśli rodzina $\mathcal{H}$ jest $\epsilon$\nbhyphen uniwersalna, to $p_{\mathcal{H}}\le\epsilon$, a~stąd dostajemy $\epsilon>1/b-1/u=1/|B|-1/|U|$.

\exercise %11.3-6
\note{Aby przeciwdziedziną funkcji\/ $h_b$ był zbiór\/ $\mathbb{Z}_p$, powinna ona być zdefiniowana następująco:
\[
	h_b(\langle a_0,a_1,\dots,a_{n-1}\rangle) = \sum_{j=0}^{n-1}a_jb^j \bmod p.
\]
Błąd występuje też w~wersji oryginalnej.}

\noindent Niech $x=\langle x_0,x_1,\dots,x_{n-1}\rangle$, $y=\langle y_0,y_1,\dots,y_{n-1}\rangle$ będą dwiema różnymi $n$\nbhyphen tkami o~wartościach z~$\mathbb{Z}_p$.
Dla losowo wybranej funkcji $h_b\in\mathcal{H}$ mamy:
\[
	h_b(x)-h_b(y) = \sum_{j=0}^{n-1}x_jb^j\bmod p-\sum_{j=0}^{n-1}y_jb^j\bmod p = \sum_{j=0}^{n-1}(x_j-y_j)b^j\bmod p.
\]
Kolizja $x$ z~$y$ wystąpi wówczas, gdy powyższa różnica będzie zerem.
Wyrażenie po prawej stronie jest wielomianem stopnia $n-1$ zmiennej $b$ o~współczynnikach z~$\mathbb{Z}_p$, a~z~\refExercise{31.4-4} mamy, że taki wielomian posiada co najwyżej $n-1$ różnych pierwiastków modulo $p$.
A~zatem co najwyżej $n-1$ spośród $p$ funkcji z~rodziny $\mathcal{H}$ spowoduje kolizję $x$ z~$y$.
Prawdopodobieństwo kolizji nie przekracza więc $(n-1)/p$, czyli rodzina $\mathcal{H}$ jest $((n-1)/p)$\nbhyphen uniwersalna według definicji z~\refExercise{11.3-5}.
