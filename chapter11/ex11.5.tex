\subchapter{Haszowanie doskonałe}

\exercise %11.5-1
\note{W~treści zadania zamiast haszowania uniwersalnego powinno być haszowanie równomierne.}

\noindent Wzór udowodnimy przez indukcję po $n$.
Oczywiście jeden klucz sam nie utworzy kolizji, więc $p(1,m)=1$ i~oszacowanie górne zachodzi.
Niech $n\ge2$ i~niech $p(n-1,m)\le e^{-(n-1)(n-2)/2m}$.
Przy założeniu, że pierwszych $n-1$ kluczy nie spowodowało kolizji, wstawienie $n$\nbhyphen tego klucza $k$ też do tego nie doprowadzi, o~ile pierwszą pozycją w~ciągu kontrolnym klucza $k$ jest jedna z~$m-(n-1)$ wolnych pozycji tablicy.
A~zatem prawdopodobieństwo, że kolizja nie wystąpi, wynosi
\begin{align*}
	p(n,m) &= p(n-1,m)\cdot\frac{m-(n-1)}{m} \\[1mm]
	&\le e^{-(n-1)(n-2)/2m}\biggl(1-\frac{n-1}{m}\biggr) \\[1mm]
	&\le e^{-(n-1)(n-2)/2m}\cdot e^{-(n-1)/m} \tag{ze wzoru (3.11)} \\
	&= e^{-(n-1)(n-2+2)/2m} \\
	&= e^{-n(n-1)/2m}.
\end{align*}

W~drugiej części zadania przyjmijmy, że $n=s\sqrt{m}+1$ dla $s\ge1$.
Wówczas
\[
	\frac{n(n-1)}{2m} \ge \frac{(n-1)^2}{2m} = \frac{s^2}{2}
\]
i~prawdopodobieństwo uniknięcia kolizji wynosi
\[
	p(n,m) \le e^{-n(n-1)/2m} \le e^{-s^2\!/2}.
\]
Wyznaczmy pochodne funkcji $f(s)=e^{-s^2\!/2}$:
\[
	f'(s) = -se^{-s^2\!/2} \quad\text{oraz}\quad f''(s) = -e^{-s^2\!/2}+s^2e^{-s^2\!/2} = e^{-s^2\!/2}(s^2-1).
\]
Dla $s\ge1$, $f'(s)<0$, co znaczy, że funkcja $f(s)$ jest ściśle malejąca, oraz $f''(s)\ge0$, co znaczy, że jest też wypukła.
A~zatem zarówno funkcja $f(s)$, jak i~ograniczone przez nią prawdopodobieństwo $p(n,m)$ maleje i~zbiega do $\lim_{s\to\infty}f(s)=0$, choć nie jest to tak gwałtowny spadek, jak gdyby funkcja ta w~pewnych obszarach była wklęsła.
