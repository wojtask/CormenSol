\subchapter{Tablice z~haszowaniem}

\exercise %11.2-1
Dla kluczy $k$, $l$, gdzie $1\le k<l\le n$, zdefiniujemy zmienną losową $X_{kl}=\I(h(k)=h(l))$.
Przy założeniu o~prostym równomiernym haszowaniu mamy $\Pr(h(k)=h(l))=1/m$ i~na podstawie lematu 5.1, $\E(X_{kl})=1/m$.
Niech $X$ będzie zmienną losową oznaczającą liczbę kolizji w~tablicy $T$, czyli
\[
    X = \sum_{k=1}^{n-1}\sum_{l=k+1}^nX_{kl}.
\]
Oczekiwana liczba kolizji wynosi
\begin{align*}
	\E(X) &= \E\biggl(\sum_{k=1}^{n-1}\sum_{l=k+1}^nX_{kl}\biggr) = \sum_{k=1}^{n-1}\sum_{l=k+1}^n\E(X_{kl}) \\[1mm]
	&= \sum_{k=1}^{n-1}\sum_{l=k+1}^n\frac{1}{m} = \frac{1}{m}\sum_{k=1}^{n-1}(n-k) = \frac{1}{m}\sum_{k=1}^{n-1}k = \frac{n(n-1)}{2m}.
\end{align*}

\exercise %11.2-2
Ciąg wstawień elementów do tablicy z~haszowaniem został zobrazowany na rys.\ \ref{fig:11.2-2}.
\medskip
\begin{figure}[!ht]
	\centering \begin{tikzpicture}[
	array/.append style = {nodes={fill=black!20, inner sep=2pt, anchor=center, minimum width=7mm}},
	double cell/.style = {array, nodes={inner sep=1pt, minimum size=4.5mm}, node distance=5mm},
	med cell/.style = {row #1/.style={nodes={fill=black!40}}},
	outer/.append style = {node distance=10mm and 5.5mm}
]

\newcommand\arrayindexes{%
	\foreach[count=\i] \x in {0, ..., 8} {
		\node[index node, left=1mm of arr-\i-1] {\x};
	}
	\node[above=2mm of arr-1-1.north] {$T$};
}

\node[outer] (pic a) {
\begin{tikzpicture}
	\matrix[
		array,
		med cell/.list = {1,2,3,4,5,7,8,9}
	] (arr) { \\ \\ \\ \\ \\ \\ \\ \\ \\};
	\arrayindexes
	\foreach \x in {1,2,3,4,5,7,8,9} {
		\draw (arr-\x-1.south west) + (2mm, 1mm) -- +(5mm, 4mm);
	}
	
	\matrix[double cell, right=of arr-6-1] (cell 6) {5 & \\};
	\draw[arrow] (arr-6-1.center) -- (cell 6-1-1);
	\draw (cell 6-1-2.south west) + (1mm, 1mm) -- +(3.5mm, 3.5mm);
\end{tikzpicture}
};

\node[outer, right=of pic a] (pic b) {
\begin{tikzpicture}
	\matrix[
		array,
		med cell/.list = {1,3,4,5,7,8,9}
	] (arr) { \\ \\ \\ \\ \\ \\ \\ \\ \\};
	\arrayindexes
	\foreach \x in {1,3,4,5,7,8,9} {
		\draw (arr-\x-1.south west) + (2mm, 1mm) -- +(5mm, 4mm);
	}
	
	\foreach \x/\val in {2/28, 6/5} {
		\matrix[double cell, right=of arr-\x-1, ampersand replacement=\&] (cell \x) {\val \& \\};
		\draw[arrow] (arr-\x-1.center) -- (cell \x-1-1);
	}
	\foreach \x in {2,6} {
		\draw (cell \x-1-2.south west) + (1mm, 1mm) -- +(3.5mm, 3.5mm);
	}
\end{tikzpicture}
};

\node[outer, right=of pic b] (pic c) {
\begin{tikzpicture}
	\matrix[
		array,
		med cell/.list = {1,3,4,5,7,8,9}
	] (arr) { \\ \\ \\ \\ \\ \\ \\ \\ \\};
	\arrayindexes
	\foreach \x in {1,3,4,5,7,8,9} {
		\draw (arr-\x-1.south west) + (2mm, 1mm) -- +(5mm, 4mm);
	}
	
	\foreach \x/\val in {2/19, 6/5} {
		\matrix[double cell, right=of arr-\x-1, ampersand replacement=\&] (cell \x) {\val \& \\};
		\draw[arrow] (arr-\x-1.center) -- (cell \x-1-1);
	}
	\foreach \x/\val in {2/28} {
		\matrix[double cell, right=of cell \x-1-2, ampersand replacement=\&] (cell \x a) {\val \& \\};
		\draw[arrow] (cell \x-1-2.center) -- (cell \x a-1-1);
	}
	\foreach \x in {2a,6} {
		\draw (cell \x-1-2.south west) + (1mm, 1mm) -- +(3.5mm, 3.5mm);
	}
\end{tikzpicture}
};

\node[outer, right=of pic c] (pic d) {
\begin{tikzpicture}
	\matrix[
		array,
		med cell/.list = {1,3,4,5,8,9}
	] (arr) { \\ \\ \\ \\ \\ \\ \\ \\ \\};
	\arrayindexes
	\foreach \x in {1,3,4,5,8,9} {
		\draw (arr-\x-1.south west) + (2mm, 1mm) -- +(5mm, 4mm);
	}
	
	\foreach \x/\val in {2/19, 6/5, 7/15} {
		\matrix[double cell, right=of arr-\x-1, ampersand replacement=\&] (cell \x) {\val \& \\};
		\draw[arrow] (arr-\x-1.center) -- (cell \x-1-1);
	}
	\foreach \x/\val in {2/28} {
		\matrix[double cell, right=of cell \x-1-2, ampersand replacement=\&] (cell \x a) {\val \& \\};
		\draw[arrow] (cell \x-1-2.center) -- (cell \x a-1-1);
	}
	\foreach \x in {2a,6,7} {
		\draw (cell \x-1-2.south west) + (1mm, 1mm) -- +(3.5mm, 3.5mm);
	}
\end{tikzpicture}
};

\node[outer, below=of pic a.south west, anchor=north west] (pic e) {
\begin{tikzpicture}
	\matrix[
		array,
		med cell/.list = {1,4,5,8,9}
	] (arr) { \\ \\ \\ \\ \\ \\ \\ \\ \\};
	\arrayindexes
	\foreach \x in {1,4,5,8,9} {
		\draw (arr-\x-1.south west) + (2mm, 1mm) -- +(5mm, 4mm);
	}
	
	\foreach \x/\val in {2/19, 3/20, 6/5, 7/15} {
		\matrix[double cell, right=of arr-\x-1, ampersand replacement=\&] (cell \x) {\val \& \\};
		\draw[arrow] (arr-\x-1.center) -- (cell \x-1-1);
	}
	\foreach \x/\val in {2/28} {
		\matrix[double cell, right=of cell \x-1-2, ampersand replacement=\&] (cell \x a) {\val \& \\};
		\draw[arrow] (cell \x-1-2.center) -- (cell \x a-1-1);
	}
	\foreach \x in {2a,3,6,7} {
		\draw (cell \x-1-2.south west) + (1mm, 1mm) -- +(3.5mm, 3.5mm);
	}
\end{tikzpicture}
};

\node[outer, right=14mm of pic e] (pic f) {
\begin{tikzpicture}
	\matrix[
		array,
		med cell/.list = {1,4,5,8,9}
	] (arr) { \\ \\ \\ \\ \\ \\ \\ \\ \\};
	\arrayindexes
	\foreach \x in {1,4,5,8,9} {
		\draw (arr-\x-1.south west) + (2mm, 1mm) -- +(5mm, 4mm);
	}

	\foreach \x/\val in {2/19, 3/20, 6/5, 7/33} {
		\matrix[double cell, right=of arr-\x-1, ampersand replacement=\&] (cell \x) {\val \& \\};
		\draw[arrow] (arr-\x-1.center) -- (cell \x-1-1);
	}
	\foreach \x/\val in {2/28, 7/15} {
		\matrix[double cell, right=of cell \x-1-2, ampersand replacement=\&] (cell \x a) {\val \& \\};
		\draw[arrow] (cell \x-1-2.center) -- (cell \x a-1-1);
	}
	\foreach \x in {2a,3,6,7a} {
		\draw (cell \x-1-2.south west) + (1mm, 1mm) -- +(3.5mm, 3.5mm);
	}
\end{tikzpicture}
};

\node[outer, below=of pic d.south west, anchor=north west] (pic g) {
\begin{tikzpicture}
	\matrix[
		array,
		med cell/.list = {1,5,8,9}
	] (arr) { \\ \\ \\ \\ \\ \\ \\ \\ \\};
	\arrayindexes
	\foreach \x in {1,5,8,9} {
		\draw (arr-\x-1.south west) + (2mm, 1mm) -- +(5mm, 4mm);
	}

	\foreach \x/\val in {2/19, 3/20, 4/12, 6/5, 7/33} {
		\matrix[double cell, right=of arr-\x-1, ampersand replacement=\&] (cell \x) {\val \& \\};
		\draw[arrow] (arr-\x-1.center) -- (cell \x-1-1);
	}
	\foreach \x/\val in {2/28, 7/15} {
		\matrix[double cell, right=of cell \x-1-2, ampersand replacement=\&] (cell \x a) {\val \& \\};
		\draw[arrow] (cell \x-1-2.center) -- (cell \x a-1-1);
	}
	\foreach \x in {2a,3,4,6,7a} {
		\draw (cell \x-1-2.south west) + (1mm, 1mm) -- +(3.5mm, 3.5mm);
	}
\end{tikzpicture}
};

\node[outer, below=of pic e.south, anchor=north west] (pic h) {
\begin{tikzpicture}
	\matrix[
		array,
		med cell/.list = {1,5,8}
	] (arr) { \\ \\ \\ \\ \\ \\ \\ \\ \\};
	\arrayindexes
	\foreach \x in {1,5,8} {
		\draw (arr-\x-1.south west) + (2mm, 1mm) -- +(5mm, 4mm);
	}

	\foreach \x/\val in {2/19, 3/20, 4/12, 6/5, 7/33, 9/17} {
		\matrix[double cell, right=of arr-\x-1, ampersand replacement=\&] (cell \x) {\val \& \\};
		\draw[arrow] (arr-\x-1.center) -- (cell \x-1-1);
	}
	\foreach \x/\val in {2/28, 7/15} {
		\matrix[double cell, right=of cell \x-1-2, ampersand replacement=\&] (cell \x a) {\val \& \\};
		\draw[arrow] (cell \x-1-2.center) -- (cell \x a-1-1);
	}
	\foreach \x in {2a,3,4,6,7a,9} {
		\draw (cell \x-1-2.south west) + (1mm, 1mm) -- +(3.5mm, 3.5mm);
	}
\end{tikzpicture}
};

\node[outer, right=14mm of pic h] (pic i) {
\begin{tikzpicture}
	\matrix[
		array,
		med cell/.list = {1,5,8}
	] (arr) { \\ \\ \\ \\ \\ \\ \\ \\ \\};
	\arrayindexes
	\foreach \x in {1,5,8} {
		\draw (arr-\x-1.south west) + (2mm, 1mm) -- +(5mm, 4mm);
	}

	\foreach \x/\val in {2/10, 3/20, 4/12, 6/5, 7/33, 9/17} {
		\matrix[double cell, right=of arr-\x-1, ampersand replacement=\&] (cell \x) {\val \& \\};
		\draw[arrow] (arr-\x-1.center) -- (cell \x-1-1);
	}
	\foreach \x/\val in {2/19, 2a/28, 7/15} {
		\matrix[double cell, right=of cell \x-1-2, ampersand replacement=\&] (cell \x a) {\val \& \\};
		\draw[arrow] (cell \x-1-2.center) -- (cell \x a-1-1);
	}
	\foreach \x in {2aa,3,4,6,7a,9} {
		\draw (cell \x-1-2.south west) + (1mm, 1mm) -- +(3.5mm, 3.5mm);
	}
\end{tikzpicture}
};

\foreach \x in {a, ..., i} {
	\node[subpicture label, below=2mm of pic \x] {(\x)};
}

\end{tikzpicture}

	\caption{Ilustracja wstawiania do tablicy z~haszowaniem $T[0\twodots8]$ elementów o~kluczach 5, 28, 19, 15, 20, 33, 12, 17, 10.
Do rozwiązywania kolizji używana jest metoda łańcuchowa.} \label{fig:11.2-2}
\end{figure}

\exercise %11.2-3
Zakładamy, że listy są dwukierunkowe i~wartości funkcji haszującej można wyznaczać w~czasie stałym.
Wyszukiwanie zakończone porażką wymaga liniowego przejrzenia jednej z~list, co zajmuje czas $\Theta(1+\alpha)$.
Posortowanie list umożliwia wcześniejsze zakończenie wyszukiwania, które kończy się sukcesem, ale asymptotycznie też wymaga ono czasu $\Theta(1+\alpha)$.
Aby wstawić nowy element do tablicy z~haszowaniem, należy umieścić go na liście wskazanej przez funkcję haszującą, znajdując wpierw odpowiednie miejsce na tej liście, aby dodanie elementu nie zaburzyło jej uporządkowania.
Zostaną więc wykonane identyczne kroki jak podczas wyszukiwania zakończonego sukcesem, stąd czas wstawiania wynosi $\Theta(1+\alpha)$.
Natomiast usuwanie elementu z~tablicy to usunięcie go z~jednej z~list, co na podstawie wyniku z~problemu \refProblem{10-1} zajmuje czas $\Theta(1)$.

\exercise %11.2-4
Każda pozycja tablicy $T$ w~takiej reprezentacji będzie zawierać znacznik przechowujący informację o~tym, czy pozycja ta jest zajęta.
Na zajętych pozycjach oprócz elementu będzie przechowywany wskaźnik (który oznaczymy przez \id{next}) do innej pozycji tablicy $T$ lub wskaźnik pusty, dzięki czemu komórki te będą mogły formować listy jednokierunkowe.
Wolne pozycje natomiast tworzyć będą listę dwukierunkową $F$, do zrealizowania której wykorzystane zostaną dwa wskaźniki na każdej wolnej pozycji -- \id{prev} na poprzednik i~\id{next} na następnik.
Do wskazania na głowę listy $F$ zostanie użyte dodatkowe pole tablicy $T$.
Atrybut ten oraz wskaźniki \id{prev} i~\id{next} będą tak naprawdę liczbami całkowitymi określającymi indeksy tablicy $T$, przy czym pusty wskaźnik (odpowiednik \const{nil}) będziemy reprezentować wartością $-1$.

Aby wstawić do tablicy $T$ element $x$ o~kluczu $k$, wpierw należy sprawdzić, czy pozycja $h(k)$ w~$T$ jest wolna.
Jeśli tak, to usuwamy ją z~listy $F$, umieszczamy na niej element $x$ i~ustawiamy \attrib{T[h(k)]}{next} na $-1$.
W~przeciwnym razie na pozycji $h(k)$ znajduje się już inny element $y$ o~kluczu $l$.
Wówczas usuwamy głowę listy $F$ i~przenosimy na nią element $y$ wraz ze wskaźnikiem \attrib{T[h(k)]}{next}, a~do komórki $h(k)$ wstawiamy element $x$.
Pamiętamy też o~ustawieniu znacznika zajętości na pobranej pozycji z~listy $F$.

Pozostaje uaktualnić wskaźnik \attrib{T[h(k)]}{next}.
W~tym celu oznaczmy przez $j$ indeks nowej pozycji elementu $y$ i~rozważmy dwie sytuacje.
W~pierwszej z~nich na podstawie funkcji haszującej elementowi $y$ odpowiada komórka $h(k)$, tzn.\ $h(l)=h(k)$.
Wówczas wskaźnikiem \attrib{T[h(k)]}{next} należy pokazać na pozycję $j$, gdyż $x$ i~$y$ powinny znaleźć się na tej samej liście zajętych pozycji.
W~drugim przypadku element $y$ znajdował się na liście zajętych pozycji rozpoczynającej się na indeksie $h(l)\ne h(k)$.
Musimy zatem przejść po tej liście i~zmodyfikować ją tak, aby wskaźnik \id{next} poprzednika elementu $y$ pokazywał teraz na pozycję $j$.
Element $x$ będzie wówczas jedyny na swojej liście -- wystarczy więc wpisać do \attrib{T[h(k)]}{next} wartość $-1$.

Ponieważ traktujemy zajęte pozycje tablicy $T$ jak węzły list jednokierunkowych, to usuwanie elementu $x$ o~kluczu $k$ polega na usunięciu węzła, który go zawiera, z~listy o~głowie w~$h(k)$.
Musimy jednakże pamiętać o~tym, że lista ta, o~ile po usunięciu $x$ będzie zawierać jeszcze jakieś węzły, powinna mieć głowę na pozycji $h(k)$.
Jeśli $x$ leży w~głowie swojej listy, to wystarczy na jego miejsce przenieść element po nim następujący.
Wycięty węzeł z~elementem $x$ należy jeszcze wstawić na listę $F$ i~oznaczyć tę pozycję tablicy jako wolną.

Wreszcie wyszukiwanie elementu o~kluczu $k$ sprowadza się do przeglądnięcia listy o~głowie w~$h(k)$ i~zwrócenia wskaźnika do takiego elementu (o~ile istnieje) albo wartości \const{nil}.

Zauważmy, że przechowywanie list zajętych pozycji bezpośrednio w~tablicy nie zmienia czasów działania operacji słownikowych w~porównaniu z~zastosowaniem metody łańcuchowej, w~której listy te znajdują się poza tablicą.
Ponadto w~opisanej reprezentacji mamy zawsze $\alpha\le1$.
Operacje \proc{Insert}, \proc{Delete} i~\proc{Search} działają więc tutaj w~oczekiwanym czasie $\Theta(1+\alpha)=\Theta(1)$.
Aby go uzyskać, musieliśmy założyć, że lista wolnych pozycji $F$ jest listą dwukierunkową -- w~przeciwnym przypadku bowiem nie można byłoby usuwać z~niej w~czasie stałym.

\exercise %11.2-5
Załóżmy, że na każdą z~$m-1$ pewnych pozycji tablicy odwzorowywanych jest co najwyżej $n-1$ kluczy z~$U$.
Na ostatnią pozycję trafi więc co najmniej
\[
	|U|-(m-1)(n-1) > nm-(m-1)(n-1) = n+m-1 \ge n
\]
kluczy.
