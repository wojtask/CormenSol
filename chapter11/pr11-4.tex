\problem{Haszowanie \singledash{$k$}{uniwersalne} i~uwierzytelnianie} %11-4
\note{Poprawki wprowadzone w~angielskiej treści tego problemu okazały się na tyle znaczące, że problem został napisany od nowa.
Poniżej prezentujemy polskie tłumaczenie jego nowej wersji.}

\noindent Niech $\mathcal{H}$ będzie rodziną funkcji haszujących, które odwzorowują uniwersum kluczy $U$ w~zbiór $\{0,1,\dots,m-1\}$.
Powiemy, że $\mathcal{H}$ jest \textbf{\singledash{$k$}{uniwersalna}}, jeśli dla każdego ustalonego ciągu $k$ różnych kluczy $\langle x^{(1)},x^{(2)},\dots,x^{(k)}\rangle$ oraz funkcji $h$ wybranej losowo z~$\mathcal{H}$ ciąg $\langle h(x^{(1)}),h(x^{(2)}),\dots,h(x^{(k)})\rangle$ jest z~jednakowym prawdopodobieństwem równy dowolnemu spośród $m^k$ ciągów $k$ elementów ze zbioru $\{0,1,\dots,m-1\}$.
\begin{description}
	\setlength\labelsep{11pt}
	\item[{\sffamily\bfseries(a)}] Wykaż, że jeśli rodzina funkcji haszujących $\mathcal{H}$ jest \singledash{2}{uniwersalna}, to jest uniwersalna.
	\item[{\sffamily\bfseries(b)}] Załóżmy, że $U$ jest zbiorem \singledash{$n$}{tek} o~wartościach z~$\mathbb{Z}_p=\{0,1,\dots,p-1\}$, gdzie $p$ jest liczbą pierwszą.
Rozważmy element $x=\langle x_0,x_1,\dots,x_{n-1}\rangle\in U$.
Dla każdej \singledash{$n$}{tki} $a=\langle a_0,a_1,\dots,a_{n-1}\rangle\in U$ definiujemy funkcję haszującą $h_a$ jako
	\[
		h_a(x) = \biggl(\sum_{j=0}^{n-1}a_jx_j\biggr)\bmod p.
	\]
	Udowodnij, że rodzina $\mathcal{H}=\{h_a\}$ jest uniwersalna, ale nie \singledash{2}{uniwersalna}.
(\!\emph{Wskazówka:} Znajdź klucz, dla którego wszystkie funkcje z~$\mathcal{H}$ przyjmują tę samą wartość.)
	\item[{\sffamily\bfseries(c)}] Załóżmy, że zmodyfikowaliśmy nieco rodzinę $\mathcal{H}$ z~punktu (b): dla każdego $a\in U$ i~każdego $b\in\mathbb{Z}_p$ definiujemy
	\[
		h_{a,b}'(x) = \biggl(\sum_{j=0}^{n-1}a_jx_j+b\biggr)\bmod p
	\]
	oraz $\mathcal{H}'=\{h_{a,b}'\}$.
Udowodnij, że rodzina $\mathcal{H}'$ jest \singledash{2}{uniwersalna}.
(\!\emph{Wskazówka:} Rozważ ustalone $x\in U$ i~$y\in U$ spełniające $x_i\ne y_i$ dla pewnego $i$.
Co dzieje się z~$h_{a,b}'(x)$ i~$h_{a,b}'(y)$, gdy $a_i$ i~$b$ przyjmują poszczególne wartości z~$\mathbb{Z}_p$?)
	\item[{\sffamily\bfseries(d)}] Przypuśćmy, że Alicja i~Bob uzgodnili w~sekrecie funkcję haszującą $h$ z~\singledash{2}{uniwersalnej} rodziny funkcji haszujących $\mathcal{H}$.
Każda funkcja $h\in\mathcal{H}$ odwzorowuje uniwersum kluczy $U$ w~$\mathbb{Z}_p$, gdzie $p$ jest liczbą pierwszą.
Następnie Alicja przesyła do Boba przez Internet komunikat $m\in U$.
Alicja uwierzytelnia komunikat, przesyłając dodatkowo znacznik $t=h(m)$, a~Bob sprawdza, czy para $\langle m,t\rangle$, którą otrzymuje, faktycznie spełnia $t=h(m)$.
Załóżmy, że przeciwnik przechwytuje przesyłaną parę $\langle m,t\rangle$ i~próbuje oszukać Boba, zamieniając ją na inną parę $\langle m',t'\rangle$.
Wykaż, że prawdopodobieństwo, iż przeciwnikowi uda się oszukać Boba i~że zaakceptuje on parę $\langle m',t'\rangle$, wynosi co najwyżej $1/p$, niezależnie od tego, jak wielką mocą obliczeniową przeciwnik dysponuje, i~nawet wówczas, gdy przeciwnik zna rodzinę funkcji haszujących $\mathcal{H}$.
\end{description}

\bigskip
\note{Poniżej znajduje się rozwiązanie nowej wersji problemu.}

\subproblem %11-4(a)
Niech $\mathcal{H}$ będzie \singledash{2}{uniwersalną} rodziną funkcji haszujących oraz niech $\langle x,y\rangle$ będzie parą różnych kluczy z~$U$.
Wówczas, dla losowo wybranej funkcji haszującej $h_a\in\mathcal{H}$, para $\langle h(x),h(y)\rangle$ jest z~jednakowym prawdopodobieństwem dowolną spośród $m^2$ par o~elementach ze zbioru $\{0,1,\dots,m-1\}$.
A~zatem kolizja, czyli zdarzenie, że $h(x)=h(y)$, wystąpi z~prawdopodobieństwem $1/m$.
Rodzina $\mathcal{H}$ jest więc uniwersalna.

\subproblem %11-4(b)
Aby zbadać \singledash{2}{uniwersalność}, wykorzystamy wskazówkę.
Dla $x=\langle0,0,\dots,0\rangle$ wszystkie funkcje haszujące z~$\mathcal{H}$ dają w~wyniku 0, więc dla dowolnej funkcji $h\in\mathcal{H}$ i~dowolnej pary $\langle x,y\rangle$ różnych kluczy z~$U$ nigdy nie otrzymamy pary $\langle h_a(x),h_a(y)\rangle$, której pierwszym elementem jest liczba różna od zera.
To wyklucza \singledash{2}{uniwersalność} rodziny $\mathcal{H}$.

Udowodnimy teraz, że rodzina $\mathcal{H}$ jest uniwersalna.
Wybierzmy w~tym celu dowolną parę $\langle x,y\rangle$ różnych kluczy z~$U$ i~pewną funkcję $h_a$ z~$\mathcal{H}$.
Bez utraty ogólności przyjmijmy, że $x_0\ne y_0$.
Kolizja $h_a(x)=h_a(y)$ wystąpi tylko wtedy, gdy suma $\sum_{j=0}^{n-1}a_jx_j$ będzie dawać taką samą resztę z~dzielenia przez $p$, co suma $\sum_{j=0}^{n-1}a_jy_j$, lub równoważnie, kiedy spełniony będzie poniższy wzór:
\[
	\sum_{j=0}^{n-1}a_j(x_j-y_j) \equiv 0 \pmod p.
\]
Niech $S=\sum_{j=1}^{n-1}a_j(x_j-y_j)$.
Wówczas wzór przyjmuje postać
\[
	a_0(x_0-y_0) \equiv -S \pmod p,
\]
którą potraktujemy jak modularne równanie liniowe zmiennej $a_0$.
Ponieważ $x_0\ne y_0$, a~$p$ jest liczbą pierwszą, to $\gcd(x_0-y_0,p)=1$ i~na podstawie wniosku 31.25 $a_0$ jest wyznaczone jednoznacznie modulo $p$.
A~zatem dla ustalonych $a_1$, $a_2$, \dots, $a_{n-1}$ istnieje dokładnie jedno $a_0$, dla którego funkcja $h_a$, gdzie $a=\langle a_0,a_1,\dots,a_{n-1}\rangle$, generuje kolizję między $x$ a~$y$.
To oznacza, że dokładnie $p^{n-1}$ spośród $p^n$ funkcji w~$\mathcal{H}$ doprowadzi do kolizji.
Prawdopodobieństwo tego zdarzenia, przy założeniu, że funkcja $h_a$ jest wybrana losowo z~$\mathcal{H}$, wynosi $1/p$, co kończy dowód.

\subproblem %11-4(c)
Ustalmy parę $\langle x,y\rangle$ różnych kluczy z~$U$ i~wybierzmy pewną funkcję $h_{a,b}'$ z~$\mathcal{H}'$.
Bez utraty ogólności przyjmijmy, że $x_0\ne y_0$.
Wprowadźmy oznaczenia $\alpha=h_{a,b}'(x)$, $\beta=h_{a,b}'(y)$ oraz $X=\sum_{j=1}^{n-1}a_jx_j$, $Y=\sum_{j=1}^{n-1}a_jy_j$.
Zachodzi $\alpha=(a_0x_0+X+b)\bmod p$ oraz $\beta=(a_0y_0+Y+b)\bmod p$.
Zauważmy, że aby wygenerować każdą możliwą parę $\langle\alpha,\beta\rangle$, wystarczy abyśmy byli w~stanie wygenerować dowolne $\alpha-\beta$ i~dowolne $\beta$.
Mamy $\alpha-\beta=(a_0(x_0-y_0)+X-Y)\bmod p$, skąd
\[
	a_0(x_0-y_0) \equiv \alpha-\beta-X+Y \pmod p.
\]
Ustalmy dowolną wartość wyrażenia $\alpha-\beta$ i~potraktujmy powyższy wzór jak modularne równanie liniowe zmiennej $a_0$.
Oczywiście $\gcd(x_0-y_0,p)=1$, więc na podstawie wniosku 31.25 $a_0$ jest wyznaczone jednoznacznie modulo $p$.
Istnieje zatem jednoznaczna odpowiedniość między wartością $a_0$ a~wartością $\alpha-\beta$.
Mając ustalone $a_0$ i~dobierając różne wartości dla $b$, możemy z~kolei wygenerować każde $\beta$.
Jest dokładnie $p^2$ możliwych par $\langle\alpha,\beta\rangle$ i~tyleż samo możliwości wyboru $a_0$ i~$b$.
Stąd wnioskujemy, że każda para $\langle\alpha,\beta\rangle$ jest jednoznacznie generowana przez odpowiednie $a_0$ i~$b$.
Istnieje zatem $p^{n-1}$ funkcji $h_{a,b}'\in\mathcal{H}'$, które generują zadaną parę $\langle\alpha,\beta\rangle$.
Wnioskujemy stąd, że uzyskanie każdej takiej pary jest jednakowo prawdopodobne, gdy funkcja $h_{a,b}'$ jest wybrana losowo z~$\mathcal{H}'$, a~to oznacza, że rodzina $\mathcal{H}'$ jest \singledash{2}{uniwersalna}.

\subproblem %11-4(d)
Ponieważ rodzina $\mathcal{H}$ jest \singledash{2}{uniwersalna}, to dla każdej pary kluczy $\langle m,m'\rangle$, w~której $m\ne m'$, uzyskanie dowolnej pary wartości $\langle h(m),h(m')\rangle$ jest jednakowo prawdopodobne, gdy funkcja $h$ zostanie wybrana losowo z~$\mathcal{H}$.
W~szczególności każda z~$p$ par postaci $\langle t,h(m')\rangle$ ma jednakowe szanse wystąpienia.
Dlatego przeciwnik, nawet jeśli dysponuje pełną wiedzą na temat rodziny $\mathcal{H}$ i~przechwyci parę $\langle m,t\rangle$, to nie zyskuje żadnej informacji o~wartości $h(m')$, którą powinien przesłać jako $t'$ celem oszukania Boba.
Przeciwnik może więc tylko zgadywać, a~szansa, że wybierze właściwą spośród $p$ wartości, jest równa $1/p$.
