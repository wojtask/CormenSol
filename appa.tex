\chapter{Sumy}

\subchapter{Wzory i~własności dotyczące sum}

\exercise %A.1-1
\[
	\sum_{k=1}^n(2k-1) = 2\sum_{k=1}^nk-\sum_{k=1}^n1 = \frac{2n(n+1)}{2}-n = n^2.
\]

\exercise %A.1-2
Korzystając z~oszacowania na $H_n$ (wzór~(A.7)), mamy
\begin{align*}
	\sum_{k=1}^n\frac{1}{2k-1} &= H_{2n-1}-\sum_{k=1}^{n-1}\frac{1}{2k} \\
	&= H_{2n-1}-\frac{1}{2}H_{n-1} \\
	&= \ln (2n-1)+O(1)-\left(\frac{1}{2}\ln(n-1)+O(1)\right) \\
	&= \ln (2n-2)+\underbrace{\ln\frac{2n-1}{2n-2}}_{O(1)}-\ln\sqrt{n-1}+O(1) \\
	&= \ln\frac{2n-2}{\sqrt{n-1}}+O(1) \\[1mm]
	&= \ln\left(2\sqrt{n-1}\right)+O(1) \\
	&= \ln 2+\ln\sqrt{n}+\underbrace{\ln\frac{\sqrt{n-1}}{\sqrt{n}}}_{O(1)}+\,O(1) \\
	&= \ln\sqrt{n}+O(1).
\end{align*}

\exercise %A.1-3
\[
	\sum_{k=0}^\infty k^2x^k = x\frac{d}{dx}\sum_{k=0}^\infty kx^k = \frac{x(x+1)}{(1-x)^3}.
\]
W~ostatniej równości wykorzystano wzór~(A.8).

\exercise %A.1-4
\begin{align*}
	\sum_{k=0}^\infty\frac{k-1}{2^k} &= \sum_{k=0}^\infty\frac{k}{2^k}-\sum_{k=0}^\infty\frac{1}{2^k} \\
	&= \frac{\frac{1}{2}}{\left(1-\frac{1}{2}\right)^2}+\sum_{k=0}^\infty\left(\frac{1}{2}\right)^k \\
	&= 2-\frac{1}{1-\frac{1}{2}} \\
	&= 0.
\end{align*}
Skorzystano z~wzoru~(A.8), a~następnie~(A.6), w~obu przypadkach biorąc $x=1/2$.

\exercise %A.1-5
\begin{align*}
	\sum_{k=1}^\infty(2k+1)x^{2k} &= 2\sum_{k=0}^\infty kx^{2k}+\sum_{k=0}^\infty x^{2k}-1 \\
	&= \frac{2x^2}{(1-x^2)^2}+\frac{1}{1-x^2}-1 \\[1mm]
	&= \frac{x^2(3-x^2)}{(1-x^2)^2}.
\end{align*}
W~drugiej równości zastosowano wzory~(A.6) i~(A.8). Otrzymany wynik jest wartością sumy, gdy $|x^2|<1$. W~pozostałych przypadkach suma jest rozbieżna do $\infty$.

\exercise %A.1-6
Zgodnie z~definicją notacji $O$, dla dowolnej funkcji $F$ zachodzi $F(n) = O\bigl(\sum_{k=1}^nf_k(n)\bigr)$ wtedy i~tylko wtedy, gdy istnieją stałe $n_0$, $c>0$ takie, że dla każdego $n\ge n_0$ prawdą jest
\[
	0\le F(n)\le c\sum_{k=1}^n f_k(n).
\]

Niech $(F_i(n))_{1\le i\le n}$ będzie ciągiem funkcji, który spełnia układ nierówności
\[
	\begin{cases}
		0 \le F_1(n) \le cf_1(n) \\
		0 \le F_2(n) \le cf_1(n)+cf_2(n) \\
		\phantom{0 \le F_2} \vdots \\
		0 \le F_n(n) \le \sum_{k=1}^ncf_k(n), \\
	\end{cases} \tag{$*$}\label{eq:A.1-6}
\]
dla pewnych dodatnich stałych $n_0$ i~$c$ oraz dla każdego $n\ge n_0$. Wprost z~definicji notacji $O$ mamy
\begin{align*}
	F_1(n) &= O(f_1(n)), \\
	F_2(n)-F_1(n) &= O(f_2(n)), \\
	& \,\,\,\vdots \\
	F_n(n)-F_{n-1}(n) &= O(f_n(n)).
\end{align*}
Dodając powyższe równania stronami, otrzymujemy
\[
	F_1(n)+(F_2(n)-F_1(n))+\cdots+(F_n(n)-F_{n-1}(n)) = \sum_{k=1}^nO(f_k(n)).
\]
Wartość po lewej stronie jest równa $F_n(n)$. Ale z~ostatniej nierówności układu~(\ref{eq:A.1-6}) wynika $F_n(n)=O\bigl(\sum_{k=1}^nf_k(n)\bigr)$, a~stąd wynika tożsamość
\[
	\sum_{k=1}^nO(f_k(n)) = O\biggl(\sum_{k=1}^nf_k(n)\biggr).
\]

\exercise %A.1-7
\[
	\prod_{k=1}^n2\cdot 4^k = 2^n\cdot 4^{1+2+\cdots+n} = 2^n\cdot 4^{\frac{n(n+1)}{2}} = 2^{n(n+2)}.
\]

\exercise %A.1-8
\begin{align*}
	\prod_{k=2}^n\left(1-\frac{1}{k^2}\right) &= \prod_{k=2}^n\frac{k^2-1}{k^2} \\[2mm]
	&= \frac{\prod_{k=2}^n(k^2-1)}{\prod_{k=2}^nk^2} \\[2mm]
	&= \frac{\prod_{k=2}^n(k-1)\cdot\prod_{k=2}^n(k+1)}{\prod_{k=1}^nk\cdot\prod_{k=1}^nk} \\[2mm]
	&= \frac{\prod_{k=1}^{n-1}k\cdot\prod_{k=3}^{n+1}k}{(n!)^2} \\[2mm]
	&= \frac{(n-1)!\cdot\frac{(n+1)!}{2}}{(n!)^2} \\[2mm]
	&= \frac{n(n+1)((n-1)!)^2}{2n^2((n-1)!)^2} \\[2mm]
	&= \frac{n+1}{2n}.
\end{align*}

\subchapter{Szacowanie sum}

\exercise %A.2-1
Wykorzystując własności szeregu teleskopowego, dostajemy:
\begin{align*}
	\sum_{k=1}^n\frac{1}{k^2} &\le 1+\sum_{k=2}^n\frac{1}{k(k-1)} \\
	&= 1+\sum_{k=2}^n\left(\frac{1}{k-1}-\frac{1}{k}\right) \\
	&= 1+1-\frac{1}{n} \\
	&< 2,
\end{align*}
a~zatem badana suma jest ograniczona przez stałą. Okazuje się, że dla coraz większych $n$, jej wartość zbliża się do liczby $\pi^2\!/6$.

\exercise %A.2-2
Zauważmy, że gdy $n$ osiąga wartość będącą potęgą 2, to zwiększa się o~1 liczba sumowanych składników. Funkcja $F(n)=\sum_{k=0}^{\lfloor\lg n\rfloor}\left\lceil n/2^k\right\rceil$ jest rosnąca w~dziedzinie liczb naturalnych, więc dla pewnego $m$ całkowitego, $F(n)<F(2^m)$ dla wszystkich $n<2^m$. Oszacowaniem górnym tejże funkcji będzie zatem oszacowanie górne jej wartości przyjmowanych dla potęg 2. Dla $n=2^m$ zachodzi
\[
	F(n) = F(2^m) = \sum_{k=0}^m\left\lceil\frac{2^m}{2^k}\right\rceil = 2^m+2^{m-1}+\cdots+2^0 = 2^{m+1}-1 = 2n-1,
\]
z~czego wynika, że $F(n)=O(n)$ i~asymptotycznym górnym ograniczeniem sumy jest $O(n)$.

\exercise %A.2-3
Załóżmy, że $H_n$ rozwinęliśmy do sumy ułamków od $1/1$ do $1/n$. Ustawmy wyrazy tej sumy malejąco i~podzielmy je w~grupy w~taki sposób, by rozmiary kolejnych grup (być może z~wyjątkiem ostatniej) były kolejnymi potęgami~2. Umieszczamy pierwszy składnik w~grupie 1 o~rozmiarze~1, dwa kolejne w~grupie~2 o~rozmiarze~2, cztery kolejne w~grupie~3 o~rozmiarze~4, itd.\ aż do grupy~$s$, jak to widać poniżej:
\[
	\underbrace{\frac{1}{1}}_{\text{\scriptsize grupa 1}}\!\!+\;\;\underbrace{\frac{1}{2}+\frac{1}{3}}_{\text{\scriptsize grupa 2}}\;\;+\;\;\underbrace{\frac{1}{4}+\frac{1}{5}+\frac{1}{6}+\frac{1}{7}}_{\text{\scriptsize grupa 3}}\;\;+\cdots+\;\;\underbrace{\cdots +\frac{1}{n}}_{\text{\scriptsize grupa $s$}}.
\]
Wszystkie składniki w~grupie~2 przyjmują wartości od $1/3$ do~$1/2$, zatem suma składników tejże grupy mieści się pomiędzy $2\cdot1/3=2/3$ a~$2\cdot1/2=1$. Wszystkie składniki w~grupie~3 przyjmują wartości od $1/7$ do~$1/4$, więc ich suma jest między $4/7$ a~1. W~ogólności, każdy z~$2^{k-1}$ składników w~grupie~$k$ jest zawarty w~przedziale $\bigl(2^{-k},2^{1-k}\,\bigr]$, a~stąd suma składników w~każdej grupie znajduje się w~przedziale $(1/2,1\,]$. W~końcu, jeśli umieściliśmy składnik $1/n$ w~grupie $s = \lfloor\lg n\rfloor+1$, to zachodzi $s/2<H_n\le s$. Zatem $\lim_{n\to\infty}H_n=\infty$ i~istotnie
\[
	\frac{\lfloor\lg n\rfloor+1}{2} < H_n \le \lfloor\lg n\rfloor+1, \quad\text{więc $H_n=\Theta(\lg n)$}.
\]

\exercise %A.2-4
Funkcja $f(k)=k^3$ jest monotonicznie rosnąca, zatem rozdzielamy sumę za pomocą nierówności~(A.11):
\[
	\int_0^nx^3\,dx \le \sum_{k=1}^nk^3 \le \int_1^{n+1}x^3\,dx.
\]
Znajdujemy oszacowania obu całek:
\begin{gather*}
	\int_0^nx^3\,dx = \frac{n^4}{4} = \Theta(n^4), \\[2mm]
	\int_1^{n+1}x^3\,dx = \frac{(n+1)^4-1}{4} = \Theta(n^4),
\end{gather*}
a~zatem
\[
	\sum_{k=0}^nk^3 = \Theta(n^4).
\]

\exercise %A.2-5
Zastosowanie nierówności~(A.12) do $\sum_{k=1}^n(1/k)$ doprowadza do wyrażenia niezdefiniowanego
\[
	\sum_{k=1}^n\frac{1}{k} \le \int_0^n\frac{dx}{x} = \ln n-\ln0,
\]
ponieważ wartości $\ln0$ nie można wyrazić liczbą rzeczywistą. Potraktowanie sumy $\sum_{k=1}^n(1/k)$ jako $1+\sum_{k=2}^n(1/k)$ pozwala na zastosowanie wzoru~(A.12) do drugiego składnika wprowadzając jedynie stałą różnicę.

\problems

\problem{Szacowanie sum} %A-1
Wszystkie następujące funkcje: $f(k)=k^r$, $g(k)=\lg^sk$, $h(k)=k^r\lg^sk$, są niemalejące dla każdych stałych $r\ge0$ i~$s\ge0$. Pozwala to na wykorzystanie nierówności~(A.11) we wszystkich poniższych punktach.

\subproblem %A-1(a)
Z~wzoru~(A.11) dostajemy
\[
	\int_0^nx^r\,dx \le \sum_{k=1}^nk^r \le \int_1^{n+1}x^r\,dx.
\]
Wartości obu całek wynoszą
\begin{gather*}
	\int_0^nx^r\,dx = \frac{n^{r+1}}{r+1} = \Theta(n^{r+1}), \\[2mm]
	\int_1^{n+1}x^r\,dx = \frac{(n+1)^{r+1}-1}{r+1} = \Theta(n^{r+1}),
\end{gather*}
zatem oszacowaniem sumy jest
\[
	\sum_{k=1}^nk^r = \Theta(n^{r+1}).
\]

\subproblem %A-1(b)
Stosujemy wzór~(A.11) do sumy $\sum_{k=2}^n\lg^sk$, która jest równoważna sumie badanej:
\[
	\int_1^n\lg^sx\,dx \le \sum_{k=2}^n\lg^sk \le \int_2^{n+1}\lg^sx\,dx.
\]
Obliczmy całkę nieoznaczoną $I(x) = \int\lg^sx\,dx$ metodą całkowania przez części. Niech $u(x)=\lg^sx$ oraz $dv(x)/dx=dx$. Pomijając stałe całkowania otrzymujemy
\[
	\frac{du(x)}{dx} = \frac{s\lg^{s-1}x\,dx}{x\ln 2} = s\lg e\cdot\frac{\lg^{s-1}x\,dx}{x} \quad\text{oraz}\quad v(x) = x,
\]
a~następnie obliczamy
\begin{align*}
	I(x) &= u(x)v(x)-\int\frac{du(x)}{dx}v(x), \\
	I(x) &= x\lg^sx-\int s\lg e\lg^{s-1}x\,dx, \\
	I(x) &= x\lg^sx-\frac{s\lg e}{\lg x}\cdot I(x), \\
	I(x) &= \frac{x\lg^{s+1}x}{\lg x+s\lg e}.
\end{align*}
Wykorzystujemy otrzymany wynik do obliczenia całek ograniczających badaną sumę:
\begin{gather*}
	{\int_1^n\lg^sx\,dx} = \frac{n\lg^{s+1}n}{\lg n+s\lg e} = \Theta(n\lg^sn), \\[2mm]
	{\int_2^{n+1}\lg^sx\,dx} = \frac{(n+1)\lg^{s+1}(n+1)}{\lg(n+1)+s\lg e}-\dfrac{2}{1+s\lg e} = \Theta(n\lg^sn),
\end{gather*}
ponieważ wyrażenie $s\lg e$ jest stałe, a~zatem oszacowaniem sumy jest
\[
	\sum_{k=1}^n\lg^sk = \Theta(n\lg^sn).
\]

\subproblem %A-1(c)
Po zastosowaniu wzoru~(A.11) do sumy $\sum_{k=2}^nk^r\lg^sk$, dostajemy
\[
	\int_1^nx^r\lg^sx\,dx \le \sum_{k=2}^nk^r\lg^sk \le \int_2^{n+1}x^r\lg^sx\,dx.
\]
Analogicznie jak w~poprzednim punkcie, obliczmy całkę $I(x) = \int x^r\lg^sx\,dx$ przez części. Niech $u(x)=\lg^sx$ oraz $dv(x)/dx=x^r\,dx$, a~stąd (po zignorowaniu stałych całkowania)
\[
	\frac{du(x)}{dx} = s\lg e\cdot\frac{\lg^{s-1}x\,dx}{x} \quad\text{oraz}\quad v(x) = \frac{x^{r+1}}{r+1},
\]
a~więc mamy
\begin{align*}
	I(x) &= u(x)v(x)-\int\frac{du(x)}{dx}v(x), \\
	I(x) &= \frac{x^{r+1}\lg^sx}{r+1}-\frac{s\lg e}{(r+1)\lg x}\cdot I(x), \\
	I(x) &= \frac{x^{r+1}\lg^{s+1}x}{(r+1)\lg x+s\lg e}.
\end{align*}
Całki ograniczające badaną sumę wynoszą zatem
\begin{gather*}
	{\int_1^nx^r\lg^sx\,dx} = \frac{n^{r+1}\lg^{s+1}n}{(r+1)\lg n+s\lg e} = \Theta(n^{r+1}\lg^sn), \\[2mm]
	{\int_2^{n+1}x^r\lg^sx\,dx} = \frac{(n+1)^{r+1}\lg^{s+1}(n+1)}{(r+1)\lg(n+1)+s\lg e}-\dfrac{2^{r+1}}{r+1+s\lg e} = \Theta(n^{r+1}\lg^sn).
\end{gather*}
Stąd wnioskujemy, że
\[
	\sum_{k=1}^nk^r\lg^sk = \Theta(n^{r+1}\lg^sn).
\]

Przyjmując w~powyższym oszacowaniu odpowiednio $s=0$ i~$r=0$, otrzymamy sumy i~ich oszacowania z~punktów (a) i~(b).

\endinput
