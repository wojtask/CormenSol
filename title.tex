\title{Wprowadzenie do algorytmów (wyd. 2)\\rozwiązania zadań i problemów}
\author{Krzysztof Wojtas}
\date{\today}

\maketitle

%\newpage\pagenumbering{roman}
\tableofcontents
%\listoffigures
%\listoftables

%\newpage\pagenumbering{arabic}
%\renewcommand\abstractname{Podziękowania}
%\begin{abstract}
%\end{abstract}

%%%sprawdzic czy zgodnie z zasadami skladu w jezyku polskim

% \begin{abstract}
% Niniejsza pozycja prezentuje rozwiązania do zadań i problemów zawartych w książce \emph{Wprowadzenie do algorytmów} na podstawie jego wyd. 2. Autor tego opracowania korzystał z polskiego tłumaczenia w wydaniu 6.
% 
% UWAGA! Dokument nie jest gotowy, prezentuje rozwiązania jedynie z rozdziałów należących do części pierwszej (rozdz. 1--5) oraz dodatków wypełniających część ósmą. Kolejne iteracje dokumentu będą sukcesywnie uzupełniane o rozwiązania zadań z kolejnych części, poprawiając jednocześnie znalezione błędy i doskonaląc niektóre starsze rozwiązania.
% 
% Jeśli znalazłeś błąd, merytoryczny lub typograficzny, bądź twierdzisz, że potrafisz rozwiązać pewne zadanie znacząco krócej lub sprytniej, powiadom mnie o tym niezwłocznie: \url{kwojtas@student.agh.edu.pl}. Nie zapłacę Ci za to dolara szesnastkowego, jak to robi Profesor D. E. Knuth, ale zyskasz moją dozgonną wdzięczność i przyczynisz się do stworzenia najobszerniejszego i najdoskonalszego opracowania zadań do ``Cormena'' jakie kiedykolwiek powstało:)
% 
% Chciałbym podziękować Autorom \emph{Wprowadzenia do algorytmów} za masę zabawy, jakiej mi dostarczyli, Tłumaczom polskiego wydania za dobre tłumaczenie tego bestsellerowego tytułu, a także Profesorowi D.E. Knuthowi za system \TeX, w którym miałem przyjemność dokonać składu niniejszej pozycji i za jego perfekcjonizm, który pragnąłem naśladować opracowując rozwiązania zadań.
% \begin{flushright}
% 	Miłego czytania!
% \end{flushright}
% \end{abstract}