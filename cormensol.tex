\documentclass[a4paper,10pt,twoside,openany,titlepage]{book}
\usepackage[MeX,plmath]{polski}
\usepackage[utf8]{inputenc}
\usepackage{amsfonts}
\usepackage{amsmath}
\usepackage{amssymb}
\usepackage{amsthm}
\usepackage{epsf}
\usepackage[pdftex]{graphicx}
\usepackage{geometry}
\usepackage{calc}
\usepackage{clrscode}
\usepackage{url}
\usepackage{sectsty}
\usepackage{tocloft}
\usepackage[font=small,format=plain,labelfont=sf,bf,up]{caption}
\usepackage[pdfborder={0 0 0},pdftex,bookmarks=true,hypertexnames=false,unicode=true]{hyperref}
\usepackage{cormensol}

\allowdisplaybreaks[1] % to modify

\begin{document}

\frontmatter

\allsectionsfont{\sffamily\bfseries}

\title{Wprowadzenie do algorytmów (wyd. 2)\\rozwiązania zadań i problemów}
\author{Krzysztof Wojtas}
\date{\today}

\maketitle

%\newpage\pagenumbering{roman}
\tableofcontents
%\listoffigures
%\listoftables

%\newpage\pagenumbering{arabic}
%\renewcommand\abstractname{Podziękowania}
%\begin{abstract}
%\end{abstract}

%%%sprawdzic czy zgodnie z zasadami skladu w jezyku polskim

% \begin{abstract}
% Niniejsza pozycja prezentuje rozwiązania do zadań i problemów zawartych w książce \emph{Wprowadzenie do algorytmów} na podstawie jego wyd. 2. Autor tego opracowania korzystał z polskiego tłumaczenia w wydaniu 6.
% 
% UWAGA! Dokument nie jest gotowy, prezentuje rozwiązania jedynie z rozdziałów należących do części pierwszej (rozdz. 1--5) oraz dodatków wypełniających część ósmą. Kolejne iteracje dokumentu będą sukcesywnie uzupełniane o rozwiązania zadań z kolejnych części, poprawiając jednocześnie znalezione błędy i doskonaląc niektóre starsze rozwiązania.
% 
% Jeśli znalazłeś błąd, merytoryczny lub typograficzny, bądź twierdzisz, że potrafisz rozwiązać pewne zadanie znacząco krócej lub sprytniej, powiadom mnie o tym niezwłocznie: \url{kwojtas@student.agh.edu.pl}. Nie zapłacę Ci za to dolara szesnastkowego, jak to robi Profesor D. E. Knuth, ale zyskasz moją dozgonną wdzięczność i przyczynisz się do stworzenia najobszerniejszego i najdoskonalszego opracowania zadań do ``Cormena'' jakie kiedykolwiek powstało:)
% 
% Chciałbym podziękować Autorom \emph{Wprowadzenia do algorytmów} za masę zabawy, jakiej mi dostarczyli, Tłumaczom polskiego wydania za dobre tłumaczenie tego bestsellerowego tytułu, a także Profesorowi D.E. Knuthowi za system \TeX, w którym miałem przyjemność dokonać składu niniejszej pozycji i za jego perfekcjonizm, który pragnąłem naśladować opracowując rozwiązania zadań.
% \begin{flushright}
% 	Miłego czytania!
% \end{flushright}
% \end{abstract}

\ifpdf
	\def\DelBookmark{\def\@writetorep##1##2##3##4{\relax}}
\else
	\def\DelBookmark{\relax}
\fi

\begingroup
	\DelBookmark
	\chapter{Wstęp}

Niniejsze opracowanie zawiera rozwiązania zadań i~problemów pochodzących z~monografii \textsl{Introduction to Algorithms} \cite{cormen} autorstwa Thomasa~H.~Cormena, Charlesa~E.~Leisersona, Ronalda~L.~Rivesta i Clifforda~Steina, na~podstawie jej drugiego wydania. Autor niniejszej pozycji korzystał z~polskiego tłumaczenia pt.~\textsl{Wprowadzenie do algorytmów} \cite{cormenpl} w~wydaniu szóstym.

Opracowanie to nie jest jeszcze gotowe -- zawiera rozwiązania jedynie z~rozdziałów należących do części pierwszej (rozdz.~1\nobreakdash--5) oraz dodatków wypełniających część ósmą \textsl{Wprowadzenia}. Kolejne iteracje będą sukcesywnie uzupełniane o~rozwiązania zadań z~kolejnych części, wprowadzając jednocześnie poprawki znalezionych błędów i~doskonaląc niektóre starsze rozwiązania.

Moim celem było stworzenie kompletnego podręcznika z~rozwiązaniami, który pomaga w~przyswajaniu materiału z~\textsl{Wprowadzenia} i~służy jako wyrocznia po uprzedniej próbie rozwiązania konkretnego zadania przez Czytelnika. Oczywiście zachęcam do takich prób zamiast natychmiastowego zaglądania do rozwiązania -- z~pewnością nauczysz się więcej, a~samo zmierzenie się z~problemem może dać Ci dużo satysfakcji. Wyznaczony przeze mnie cel rzetelności przedstawianych tu wywodów zaowocował przeładowaniem treści językiem formalnym i~w~pewnych miejscach niewątpliwie trudnym, ale należy mieć na uwadze to, że dokładność i~precyzja są nieodłącznymi cechami pozycji matematycznych.

Czym różni się bieżąca pozycja od kilku podobnych, na które można natknąć się w Internecie? Udało mi się odnaleźć tylko dwie, przy czym jedna z~nich jest oficjalnym podręcznikiem lansowanym przez Autorów książki. Nie pokrywają jednak całości materiału, poza tym w~kilku miejscach zauważyłem, że przedstawione rozumowania odbiegają nieco od prawdy. Z~moich poszukiwań wynika także, że nie istnieje podobne opracowanie w~języku polskim. Ze znalezionych materiałów nieco skorzystałem, zawsze jednak miałem na uwadze własne zrozumienie rozwiązania i~przekształcenie go do postaci, moim zdaniem najbardziej odpowiedniej i~w~miarę możliwości krótkiej i~zwięzłej.

Dołożyłem wszelkich starań, aby każde rozwiązanie zostało dokładnie sprawdzone. Jeśli jednak znalazłeś błąd, merytoryczny lub typograficzny, bądź twierdzisz, że potrafisz rozwiązać pewien problem znacząco krócej lub sprytniej, powiadom mnie o~tym niezwłocznie, pisząc na adres \url{kwojtas@student.agh.edu.pl}.

Co ciekawe, w międzyczasie ukazało się już trzecie wydanie \textsl{Wprowadzenia} \cite{cormen3}, które zawiera wiele nowych treści i~problemów. Dlatego też, niedługo po ukończeniu przeze mnie opracowania dla wydania drugiego, o~ile pojawi się równie dobre polskie tłumaczenie, można spodziewać się publikacji rozwiązań zadań z~nowej edycji.

Chciałbym podziękować Autorom \textsl{Wprowadzenia do algorytmów} za masę zabawy i~satysfakcji, jakich mi dostarczyli oraz Tłumaczom polskiego wydania za dobry przekład tego bestsellerowego tytułu.

\bigskip
\noindent{\sl Kraków, 29 grudnia 2009}\hfill--- K. W.

\endinput

\endgroup

\ifpdf
	\addtocontents{toc}{\protect\pdfbookmark{Spis treści}{Tableofcont}}
\fi

\renewcommand{\cfttoctitlefont}{\Huge\sffamily\bfseries}
\renewcommand{\cftpartfont}{\large\sffamily\bfseries}
\renewcommand{\cftpartpagefont}{\large\sffamily\bfseries}
\renewcommand{\cftchapaftersnum}{.}
\renewcommand{\cftchapfont}{\sffamily\bfseries}
\renewcommand{\cftchappagefont}{\sffamily\bfseries}
\renewcommand{\cftsecfont}{\sffamily}
\renewcommand{\cftsecleader}{\sffamily\cftdotfill{\cftsecdotsep}}
\renewcommand{\cftsecpagefont}{\sffamily}

\tableofcontents

\let\mychapter=\chapter
\renewcommand{\chapter}[1]{%
	\mychapter{#1}
	\setcounter{subsection}{0}
	\def\thesubsection{\thechapter.\arabic{section}-\arabic{subsection}.}
	\lhead[\fancyplain{}{\small\sffamily\bfseries\thepage}]{\fancyplain{}{\small\sffamily\bfseries\rightmark}}
}

\makeatletter
	\@afterindentfalse
	\@afterheading
\makeatother

\mainmatter

\part{Podstawy}

\section*{Rozdział 1: Rola algorytmów w obliczeniach}

\subsection*{1.1. Algorytmy}

\paragraph{1.1-1.}
\begin{description}
	\item[sortowanie:] porządkowanie po rozdaniu kart w ręku względem ich wartości,
	\item[optymalna kolejność mnożenia macierzy:] wyznaczanie transformacji graficznych (np. skalowań, obrotów); przekształcenia te opisane są za pomocą macierzy, a ich składanie oznacza obliczenie iloczynu tych macierzy,
	\item[otoczka wypukła:] mając zbiór wbitych w ziemię palików, chcemy otoczyć je wszystkie siatką ogrodzeniową opierając ją na niektórych palikach tak, by zużyć jej jak najmniej.
\end{description}

\paragraph{1.1-2.}
\begin{itemize}
  \item minimalizacja zużycia pamięci (operacyjnej i masowej)
  \item minimalizacja wykorzystania systemu operacyjnego
  \item minimalizacja dostępu do bazy danych
  \item minimalizacja eksploatacji łącza sieciowego
  \item dostosowanie do konkretnej architektury sprzętowo-programowej
  \item efektywność działania w architekturze równoległej lub rozproszonej
\end{itemize}

\paragraph{1.1-3.}
W następującej tabeli porównano listę dwukierunkową z prostszymi strukturami.\\
\begin{tabular}{l|l}
Zalety & Wady \\
\hline
%\begin{itemize}
%  nie trzeba z góry znać maksymalnego rozmiaru jak w przypadku tablicy, & \\
%  może się dowolnie rozszerzać i kurczyć, & \\
%  wstawianie i usuwanie elementów z dowolnego miejsca listy odbywa się w czasie stałym. & \\
%\end{itemize} &
%\begin{itemize}
%  zajmuje nieco więcej pamięci niż tablica (narzut na wskaźniki na poprzedni i następny element), & \\
%  nie można odnieść się do dowolnego elementu listy w stałym czasie, & \\
%  w przeciwieństwie do tablicy, lista nie zajmuje ciągłego obszaru pamięci, przez co kompilator nie ma możliwości dokonania pewnych optymalizacji przez przechowanie jej w pamięci podręcznej. & \\
%\end{itemize}
\end{tabular}

\paragraph{1.1-4.}
W tabeli zestawiono porównanie probelmu znajdowania najkrótszej ścieżki i problemu komiwojażera.
\begin{tabular}{l|l}
Podobieństwa & Różnice: \\
\hline
% \begin{itemize}
%   \item oba są problemami grafowymi,
%   \item oba mają na celu minimalizację pewnej ścieżki.
% \end{itemize} &
% \begin{itemize}
%   \item problem najkrótszej ścieżki poszukuje minimalnej ścieżki między dwoma wierzchołkami, a problem komiwojażera -- minimalny cykl uwzględniający wszystkie wierzchołki (minimalny cykl Hamiltona),
%   \item problem najkrótszej ścieżki jest wielomianowy (łatwo znaleźć efektywny algorytm dla tego problemu), podczas gdy problem komiwojażera jest \mbox{NP-zupełny} (prawdopodobnie nie istnieje dla niego efektywny algorytm).
% \end{itemize}
\end{tabular}

\paragraph{1.1-5.}
Problemem, którego pożądanym rozwiązaniem jest rozwiązanie dokładne, jest np. wyszukiwanie konkretnego elementu w nieuporządkowanej tablicy. Obecność jednego elementu w tablicy jest na ogół niezależna od obecności innego, zatem po sprawdzeniu kilku z nich, nie znajdując poszukiwanego elementu, nadal dysponujemy taką samą wiedzą jak przed rozpoczęciem sprawdzania.

Znalezienie rozwiązania przybliżonego jest natomiast wystarczające w wielu praktycznych zastosowaniach dla problemu komiwojażera.

\subsection*{1.2. Algorytmy jako technologia}

\paragraph{1.2-1.}
Przykładem aplikacji, w której wykorzystywanych jest wiele różnych algorytmów jest wspołczesna gra komputerowa. Jej silnik grafiki trójwymiarowej może być zaawansowanym środowiskiem, w którym zastosowano algorytmy geometrii obliczeniowej i renderowania grafiki 3D, jak również wiele algorytmów numerycznych do wyznaczania interpolacji oraz algorytmy grafowe. Także w dziedzinie sztucznej inteligencji opracowano wiele zaawansowanych algorytmów. Ponadto często spotykane problemy takie jak wyszukiwanie binarne czy sortowanie są rozwiązywane w niemal każdej aplikacji.

\paragraph{1.2-2.}
Nierówność $8n^2 < 64n\lg n$ jest spełniona dla $0<n\le 43$, zatem sortowanie przez wstawianie posortuje tablicę o rozmiarze nie przekraczającym $43$ szybciej niż sortowanie przez scalanie.

\paragraph{1.2-3.}
Najmniejszym dodatnim $n$ spełniającym nierówność $100n^2 < 2^n$ jest $n=14$.

\subsection*{Problemy}
%%zwezic jakos te tabelke
\paragraph{1-1. Porównanie czasów działania}
\begin{table}[h]
\[
  \begin{array}{c|c|c|c|c|c|c|c}
    &1&1&1&1&1&1&1 \\
	f(n) & \mbox{sekunda} & \mbox{minuta} & \mbox{godzina} & \mbox{dzień} & \mbox{miesiąc} & \mbox{rok} & \mbox{wiek} \\
	\hline
	\lg n & 2^{10^6} & 2^{6\cdot 10^7} & 2^{3.6\cdot 10^9} & 2^{8.64\cdot 10^{10}} & 2^{2.59\cdot 10^{12}} & 2^{3.15\cdot 10^{13}} & 2^{3.15\cdot 10^{14}} \\
	\hline
	\sqrt{n} & 10^{12} & 3.6\cdot 10^{15} & 1.3\cdot 10^{19} & 7.47\cdot 10^{21} & 6.72\cdot 10^{24} & 9.95\cdot 10^{26} & 9.95\cdot 10^{30} \\
	\hline
	n & 10^6 & 6\cdot 10^7 & 3.6\cdot 10^9 & 8.64\cdot 10^{10} & 2.59\cdot 10^{12} & 3.15\cdot 10^{13} & 3.15\cdot 10^{14} \\
	\hline
	n\lg n & 62746 & 2.8\cdot 10^6 & 1.3\cdot 10^8 & 2.75\cdot 10^9 & 7.18\cdot 10^{10} & 7.97\cdot 10^{11} & 6.86\cdot 10^{13} \\
	\hline
	n^2 & 1000 & 7745 & 60000 & 293938 & 1.61\cdot 10^6 & 5.62\cdot 10^6 & 5.62\cdot 10^7 \\
	\hline
	n^3 & 100 & 391 & 1532 & 4420 & 13736 & 31736 & 146645 \\
	\hline
	2^n & 19 & 25 & 31 & 36 & 41 & 44 & 51 \\
	\hline
	n! & 9 & 11 & 12 & 13 & 15 & 16 & 17
  \end{array}
\]
\caption{Czasy działania}
\end{table}

\chapter{Zaczynamy}

\subchapter{Sortowanie przez wstawianie}

\exercise %2.1-1
Rysunek~\ref{fig:2.1-1} symuluje działanie algorytmu \proc{Insertion-Sort} dla przykładowych danych.
\begin{figure}[!h]
	\begin{center}
		\includegraphics{fig02.1}
	\end{center}
	\caption{Symulacja algorytmu \proc{Insertion-Sort}} \label{fig:2.1-1}
\end{figure}

\exercise %2.1-2
Aby sortować w~odwrotnym porządku, wystarczy zmienić znak drugiej nierówności na przeciwny w~warunku pętli \kw{while} w~linii~5 algorytmu \proc{Insertion-Sort}:
\begin{codebox}
\setcounter{codelinenumber}{4}
\li	\While $i>0$ i~$A[i]<\id{key}$
\end{codebox}

\exercise %2.1-3
Przedstawiony opis prowadzi do następującego algorytmu wyszukiwania liniowego:
\begin{codebox}
\Procname{$\proc{Linear-Search}(A,v)$}
\li	$i\gets1$
\li	\While $i\le\id{length}[A]$ i~$A[i]\ne v$ \label{li:search-while-begin}
\li		\Do $i\gets i+1$
		\End \label{li:search-while-end}
\li	\If $i>\id{length}[A]$
\li		\Then \Return \const{nil}
\li		\Else \Return $i$
		\End
\end{codebox}
Udowodnimy dla powyższej procedury niezmiennik pętli:
\begin{quote}
Na początku każdej iteracji pętli \kw{while} w~wierszach \ref{li:search-while-begin}\nobreakdash--\ref{li:search-while-end} fragment tablicy $A[1\twodots i-1]$ nie zawiera elementu $v$.
\end{quote}
\begin{description}
	\item[Inicjowanie:] Przed pierwszą iteracją $i=1$, więc fragment $A[1\twodots i-1]$ jest pusty.
	\item[Utrzymanie:] Załóżmy, że podtablica $A[1\twodots i-1]$ nie zawiera elementu $v$. W~warunku pętli \kw{while} sprawdzamy, czy $A[i]$ jest różne od $v$. Jeśli tak, to $i$ jest zwiększane o~1, więc niezmiennik jest zachowany. W~przeciwnym przypadku (odnaleziono $v$), przerywamy pętlę.
	\item[Zakończenie:] Algorytm kończy swe działanie, kiedy znajdzie indeks $i$ taki, że $A[i]=v$ albo $i=\id{length}[A]+1$. Pierwszy przypadek oznacza oczywiście odnalezienie pierwszego wystąpienia $v$ w~tablicy $A$, a~drugi -- że przejrzeliśmy całą tablicę nie znajdując $v$ ($A[1\twodots i-1]$ jest teraz całą tablicą $A$).
\end{description}

\exercise %2.1-4
\textbf{Dane wejściowe:} \twoparts{$n$}{elementowe} tablice $A$ i~$B$ zawierające reprezentacje binarne \twoparts{$n$}{bitowych} liczb całkowitych $a$ i~$b$ (w~kolejności od najbardziej do najmniej znaczącego bitu). \\
\textbf{Wynik:} \twoparts{$(n+1)$}{elementowa} tablica $C$ zawierająca reprezentację binarną \twoparts{$(n+1)$}{bitowej} liczby całkowitej $c$ takiej, że $c=a+b$.
\begin{codebox}
\Procname{$\proc{Binary-Addition}(A,B)$}
\li	$\id{carry}\gets0$ \>\>\>\>\Comment bit przeniesienia
\li	\For $i\gets\id{length}[A]$ \Downto $1$ \label{li:bin-add-for-begin}
\li		\Do
			$C[i+1]\gets A[i]$ \kw{xor} $B[i]$ \kw{xor} $\id{carry}$ \label{li:bin-add-xor}
\li			\If $\id{carry}=1$ \label{li:bin-add-if}
\li				\Then $\id{carry}\gets A[i]$ \kw{or} $B[i]$
\li				\Else $\id{carry}\gets A[i]$ \kw{and} $B[i]$
				\End
		\End \label{li:bin-add-for-end}
\li	$C[1]\gets\id{carry}$
\li	\Return $C$
\end{codebox}

W~pętli \kw{for} w~wierszach \ref{li:bin-add-for-begin}\nobreakdash--\ref{li:bin-add-for-end} realizowane jest dodawanie poszczególnych bitów liczb $a$ i~$b$ od najmniej do najbardziej znaczącego z~uwzględnieniem przeniesienia. Jest to tak naprawdę użycie operacji \kw{xor}, czyli alternatywy wykluczającej -- wynik takiej operacji na dwóch bitach jest ich sumą modulo~2. Instrukcja warunkowa z~wiersza~\ref{li:bin-add-if} wykrywa przepełnienie i~odpowiednio ustawia nową wartość bitu \id{carry}, także za pomocą operacji logicznych -- \kw{or} i~\kw{and}.

Jeśli potraktujemy tablice $A$ i~$B$ jako reprezentacje liczb $a$ i~$b$ w~kodzie uzupełnień do dwóch na $n$ bitach, to przedstawiona procedura poprawnie działa dla liczb ujemnych -- wynikiem dodawania jest tablica $C$ stanowiąca reprezentację liczby $c$ w~kodzie uzupełnień do dwóch na $n+1$ bitach.

\subchapter{Analiza algorytmów}

\exercise %2.2-1
\[
	n^3\!/1000-100n^2-100n+3 = \Theta(n^3).
\]

\exercise %2.2-2
Poniższy algorytm implementuje sortowanie przez wybieranie.
\begin{codebox}
\Procname{$\proc{Selection-Sort}(A)$}
\li	$n\gets\id{length}[A]$
\li	\For $j\gets1$ \To $n-1$ \label{li:sel-sort-for-begin}
\li		\Do
			$\id{min}\gets j$
\li			\For $i\gets j+1$ \To $n$
\li				\Do
					\If $A[i]<A[\id{min}]$
\li						\Then $\id{min}\gets i$
						\End
				\End
\li			zamień $A[\id{min}]\leftrightarrow A[j]$
		\End \label{li:sel-sort-for-end}
\end{codebox}
Zewnętrzna pętla algorytmu zachowuje poniższy niezmiennik:
\begin{quote}
Na początku każdej iteracji pętli \kw{for} w~wierszach \ref{li:sel-sort-for-begin}\nobreakdash--\ref{li:sel-sort-for-end} podtablica $A[1\twodots j-1]$ jest posortowana niemalejąco i~zawiera $j-1$ najmniejszych elementów znajdujących się początkowo w~tablicy $A$.
\end{quote}

Nie trzeba wykonywać $n$ przebiegów pętli \kw{for} z~wierszy \ref{li:sel-sort-for-begin}\nobreakdash--\ref{li:sel-sort-for-end}, gdyż po jej zakończeniu (po $n-1$ przebiegach), fragment $A[1\twodots n-1]$ zawiera $n-1$ najmniejszych elementów tablicy $A$ w~porządku niemalejącym, zatem element $A[n]$ jest większy od każdego elementu z~$A[1\twodots n-1]$, a~to oznacza, że cała tablica pozostaje posortowana niemalejąco.

Ponieważ mamy $n-1$ przebiegów zewnętrznej pętli \kw{for}, a~wewnętrzna pętla \kw{for} przebiega przez cały nieposortowany fragment tablicy szukając jego minimalnego elementu, zatem zarówno pesymistyczny jak i~optymistyczny czas działania algorytmu wynosi
\[
	T(n) = \sum_{i=1}^{n-1}(n-i) = \sum_{i=1}^{n-1}i = \frac{n(n-1)}{2} = \Theta(n^2).
\]

\exercise %2.2-3
Wykorzystując wynik \zad{C.3-2} mamy, że w~średnim przypadku należy sprawdzić $(n+1)/2$ elementów tablicy, zatem czas działania algorytmu wyszukiwania liniowego wynosi $\Theta(n)$. W~przypadku pesymistycznym procedura sprawdzi wszystkie $n$ elementów nie znajdując szukanego, a~więc otrzymujemy ten sam wynik: $\Theta(n)$.

\exercise %2.2-4
Egzemplarze danych wejściowych, które stanowią dla algorytmu przypadek optymistyczny, można wykrywać na początku jego działania i~bezpośrednio zwracać dla tych danych wynik wyznaczony wcześniej innymi sposobami, zamiast obliczania go za pomocą tego algorytmu.

\subchapter{Projektowanie algorytmów}

\exercise %2.3-1
Na rys.~\ref{fig:2.3-1} przedstawiono symulację działania algorytmu \proc{Insertion-Sort} dla przykładowych danych.
\begin{figure}[!ht]
	\begin{center}
		\includegraphics{fig02.2}
	\end{center}
	\caption{Symulacja algorytmu \proc{Merge-Sort}} \label{fig:2.3-1}
\end{figure}

\exercise %2.3-2
Poniższa procedura implementuje operację łączenia bez wykorzystania wartowników.
\begin{codebox}
\Procname{$\proc{Merge'}(A,p,q,r)$}
\li	$n_1\gets q-p+1$
\li	$n_2\gets r-q$
\li	utwórz tablice $L[1\twodots n_1]$ i~$R[1\twodots n_2]$
\li	\For $i\gets1$ \To $n_1$
\li		\Do $L[i]\gets A[p+i-1]$
		\End
\li	\For $j\gets1$ \To $n_2$
\li		\Do $R[j]\gets A[q+j]$
		\End
\li	$i\gets1$
\li	$j\gets1$
\li	$k\gets p$
\li	\While $i\le n_1$ i~$j\le n_2$ \label{li:merge-while-begin}
\li		\Do
			\If $L[i]\le R[j]$
\li				\Then
					$A[k]\gets L[i]$
\li					$i\gets i+1$
\li				\Else
					$A[k]\gets R[j]$
\li					$j\gets j+1$
				\End
\li			$k\gets k+1$
		\End \label{li:merge-while-end}
\li	\While $i\le n_1$
\li		\Do
			$A[k]\gets L[i]$
\li			$i\gets i+1$
\li			$k\gets k+1$
		\End
\li	\While $j\le n_2$
\li		\Do
			$A[k]\gets R[j]$
\li			$j\gets j+1$
\li			$k\gets k+1$
		\End
\end{codebox}
Po zakończeniu pętli \kw{while} z~wierszy \ref{li:merge-while-begin}\nobreakdash--\ref{li:merge-while-end} wszystkie elementy z~co najmniej jednej tablicy $L$ lub $R$ zostały przekopiowane do $A$, więc w~kolejnych dwóch pętlach \kw{while} realizujemy kopiowanie pozostałej części $L$ lub $R$ na koniec $A$. Tylko co najwyżej jedna z~tych pętli wykona swój kod więcej niż raz.

\exercise %2.3-3
Przeprowadzimy dowód przez indukcję względem $k$. Dla $k=1$ mamy $n=2$ i~$T(n)=2=2\lg2$, więc przypadek bazowy zachodzi. Załóżmy teraz, że $k>1$, czyli $n>2$ i~że zachodzi $T(n/2)=(n/2)\lg(n/2)$. Mamy
\[
	T(n) = 2T(n/2)+n = 2(n/2)\lg(n/2)+n = n(\lg n-1)+n = n\lg n,
\]
co dowodzi rozwiązania rekurencji dla $n$ będącego potęgą~2.

\exercise %2.3-4
Niech $T(n)$ będzie czasem potrzebnym na posortowanie tablicy $A[1\twodots n]$. Ponieważ wstawienie $A[n]$ w~posortowaną podtablicę $A[1\twodots n-1]$ odbywa się w~najgorszym przypadku w~czasie $\Theta(n)$, to stąd
\[
	T(n) =
	\begin{cases}
		\Theta(1), & \text{jeśli $n=1$}, \\
		T(n-1)+\Theta(n), & \text{jeśli $n>1$}.
	\end{cases}
\]
Rozwiązując rekurencję, dostajemy
\begin{align*}
	T(n) &= T(n-1)+\Theta(n) = T(n-2)+\Theta(n-1)+\Theta(n) = \dots{} \\
	&= \sum_{i=1}^n\Theta(i) = \Theta\biggl(\sum_{i=1}^ni\biggr) = \Theta\biggl(\frac{n(n+1)}{2}\biggr) = \Theta(n^2).
\end{align*}

\exercise %2.3-5
Procedura \proc{Binary-Search} przyjmuje na wejściu posortowaną niemalejąco tablicę $A$, szukaną wartość $v$ i~liczby \id{low} oraz \id{high} ograniczające zakres tablicy $A$, w~którym będzie szukane $v$. Procedura porównuje $v$ z~elementem środkowym zakresu $A[\id{mid}]$ i~na podstawie wyniku tego porównania eliminuje z~dalszych rozważań odpowiednią połowę zakresu.

Poniżej przedstawiono wersję rekurencyjną oraz iteracyjną algorytmu wyszukiwania binarnego. W~przypadku odnalezienia wartości $v$ w~tablicy $A$, zwracany jest taki indeks $i$, że $A[i]=v$. Jeśli elementu $v$ nie ma w~tablicy, to wynikiem procedury jest specjalna wartość \const{nil}.

\begin{codebox}
\Procname{$\proc{Recursive-Binary-Search}(A,v,\id{low},\id{high})$}
\li	\If $\id{low}>\id{high}$
\li		\Then \Return \const{nil}
		\End
\li	$\id{mid}\gets\lfloor(\id{low}+\id{high})/2\rfloor$
\li	\If $v=A[\id{mid}]$
\li		\Then \Return \id{mid}
		\End
\li	\If $v>A[\id{mid}]$
\li		\Then \Return $\proc{Recursive-Binary-Search}(A,v,\id{mid}+\,1,\id{high})$
\li		\Else \Return $\proc{Recursive-Binary-Search}(A,v,\id{low},\id{mid}-\,1)$
		\End
\end{codebox}

\begin{codebox}
\Procname{$\proc{Iterative-Binary-Search}(A,v,\id{low},\id{high})$}
\li	\While $\id{low}\le\id{high}$
\li		\Do
			$\id{mid}\gets\lfloor(\id{low}+\id{high})/2\rfloor$
\li			\If $v=A[\id{mid}]$
\li				\Then \Return $\id{mid}$
				\End
\li			\If $v>A[\id{mid}]$
\li				\Then $\id{low}\gets\id{mid}+\,1$
\li				\Else $\id{high}\gets\id{mid}-\,1$
				\End
		\End
\li	\Return \const{nil}
\end{codebox}

Obie wersje algorytmu \proc{Binary-Search} kończą swe działanie znajdując $v$ na pewnej pozycji tablicy $A$ albo nie znajdując go, w~przypadku gdy zakres poszukiwań okaże się pusty (czyli $\id{low}>\id{high}$). Po przyrównaniu $v$ do środkowego elementu zakresu, procedura odrzuca połowę zakresu i~poszukuje $v$ w~drugiej połowie. Rekurencja opisująca czas jej działania ma postać
\[
	T(n) =
	\begin{cases}
		\Theta(1), & \text{jeśli $n\le1$}, \\
		T(\lfloor n/2\rfloor)+\Theta(1), & \text{jeśli $n>1$}.
	\end{cases}
\]
Jej rozwiązaniem (z~\zad{4.3-3}) jest $T(n)=\Theta(\lg n)$.

\exercise %2.3-6
Wyszukując binarnie, można odnaleźć pozycję tablicy, gdzie należy umieścić kolejny element z~nieposortowanego fragmentu, jednak wstawienie go wymaga przesunięcia pewnej części tablicy o~jedną pozycję w~prawo, na co w~najgorszym przypadku potrzeba $\Theta(n)$ operacji. Nie można zatem obniżyć czasu działania sortowania przez wstawianie poprzez zastosowanie wyszukiwania binarnego.

\exercise %2.3-7
Dla każdego elementu $S[i]$ można wyszukiwać inny element w~tablicy $S$, który po zsumowaniu z~$S[i]$ daje $x$. Będziemy wyszukiwać binarnie po uprzednim posortowaniu $S$; procedura wyszukiwania binarnego została opisana w~\zad{2.3-5}. Dla $S[i]$ szukamy zatem elementu o~wartości $x-S[i]$ w~podtablicy $S[i+1\twodots n]$, gdzie $n=\id{length}[S]$. Podtablica ta jest pusta dla $i=n$, dlatego wyszukiwanie dla ostatniego elementu pomijamy. Zwracana jest wartość logiczna \const{true} lub \const{false} w~zależności od wyniku wyszukiwania. Algorytm zapisujemy w~postaci pseudokodu:
\begin{codebox}
\Procname{$\proc{Sum-Search}(S,x)$}
\li	$n\gets\id{length}[S]$
\li	posortuj $S$
\li	\For $i\gets1$ \To $n-1$
\li		\Do
			\If $\proc{Binary-Search}(S,x-S[i],i+1,n)\ne\const{nil}$
\li				\Then \Return \const{true}
				\End
		\End
\li	\Return \const{false}
\end{codebox}

Sortowanie $S$ za pomocą porównań (np.\ procedurą \proc{Merge-Sort}) działa w~czasie $\Theta(n\lg n)$, a~\proc{Binary-Search} jest wykonywane dla tablic o~rozmiarach kolejno 1, 2,~\dots,~$n-1$, zatem pesymistyczny czas algorytmu \proc{Sum-Search} wynosi
\[
	\Theta(n\lg n)+\sum_{i=1}^{n-1}\Theta(\lg i) = \Theta(n\lg n)+\Theta(\lg(n!)) = \Theta(n\lg n),
\]
ponieważ $\lg(n!)=\Theta(n\lg n)$ (z~\zad{3.2-3}).

\problems

\problem{Sortowanie przez wstawianie dla małych tablic podczas sortowania przez scalanie} %2-1

\subproblem %2-1(a)
Sortowanie przez wstawianie podlisty o~długości $k$ działa w~czasie pesymistycznym $\Theta(k^2)$, a~zastosowane do $n/k$ takich podlist zajmuje czas równy $(n/k)\cdot\Theta(k^2)=\Theta(nk)$.

\subproblem %2-1(b)
Uogólniając procedurę scalania dwóch podlist na scalanie $n/k$ podlist w~jedną, można osiągnąć czas $\Theta(n^2\!/k)$ (kopiujemy $n$ elementów, dla każdego sprawdzając, która z~$n/k$ podlist jest tą, w~której element powinien się znaleźć).

Lepszy czas można jednak uzyskać dzięki scalaniu podlist parami, następnie otrzymane większe podlisty również scalając parami itd., aż do uzyskania pojedynczej listy wynikowej. Na każdym poziomie scalanie wymaga czasu $\Theta(n)$, jest $\lceil\lg(n/k)\rceil$ poziomów, a~zatem czas działania scalania $n/k$ podlist wynosi $\Theta(n\lg(n/k))$.

\subproblem %2-1(c)
Czas działania zmodyfikowanego algorytmu ma ten sam rząd złożoności co czas działania sortowania przez scalanie, o~ile zachodzi $\Theta(nk+n\lg(n/k))=\Theta(n\lg n)$. Zauważmy, że jeśli $k=o(\lg n)$, to czas działania algorytmu zmodyfikowanego wynosi $o(n\lg n)$. Zbadajmy zatem, czy czasy obu algorytmów są równe dla $k=\Theta(\lg n)$.
\begin{align*}
	\Theta(nk+n\lg(n/k)) &= \Theta(nk+n\lg n-n\lg k) \\
	&= \Theta(2n\lg n-n\lg\lg n) \\
	&= \Theta(n\lg n),
\end{align*}
dzięki opuszczeniu składnika niższego rzędu i~pominięciu stałego współczynnika. Maksymalnym rzędem $k$, dla którego czas zmodyfikowanego algorytmu jest równy czasowi zwykłego sortowania przez scalanie, jest zatem $\Theta(\lg n)$.

\subproblem %2-1(d)
W~praktyce $k$ powinno być największą długością listy, dla której sortowanie przez wstawianie działa szybciej od sortowania przez scalanie.

\problem{Poprawność sortowania bąbelkowego} %2-2

\subproblem %2-2(a)
Należy jeszcze pokazać, że tablica $A'$ stanowi permutację tablicy $A$.

\subproblem %2-2(b)
Niezmiennik wewnętrznej pętli \kw{for}:
\begin{quote}
Przed każdą iteracją pętli \kw{for} w~wierszach 2\nobreakdash--4 najmniejszy element z~podtablicy $A[j\twodots n]$ znajduje się w~$A[j]$.
\end{quote}
\begin{description}
	\item[Inicjowanie:] Przed pierwszą iteracją $j=n$, więc $A[j\twodots n]$ zawiera jeden element, który oczywiście jest najmniejszy w~tej podtablicy i~znajduje się w~$A[j]$.
	\item[Utrzymanie:] Załóżmy, że $A[j]$ jest najmniejszym elementem w~$A[j\twodots n]$. Jeżeli $A[j-1]$ jest większe od $A[j]$, to $A[j]$ jest zamieniane z~$A[j-1]$ w~wierszu~4, więc w~tym momencie podtablica $A[j-1\twodots n]$ posiada swój najmniejszy element w~$A[j-1]$. Uaktualnienie $j$ powoduje odtworzenie niezmiennika. W~przeciwnym przypadku zamiana nie następuje, przez co $A[j-1]$ stanowi najmniejszy element $A[j-1\twodots n]$ i~aktualizacja $j$ powoduje, że niezmiennik także pozostaje spełniony.
	\item[Zakończenie:] Po zakończeniu wykonywania pętli $j=i$, więc $A[i]$ jest najmniejszym elementem podtablicy $A[i\twodots n]$, gdyż jeśli podczas ostatniej iteracji pętli, kiedy $j=i+1$, zachodziłoby $A[i]>A[j]$ czyli $A[i]>A[i+1]$, to elementy te byłyby z~sobą zamienione.
\end{description}

\subproblem %2-2(c)
Niezmiennik zewnętrznej pętli \kw{for}:
\begin{quote}
Przed każdą iteracją pętli \kw{for} w~wierszach 1\nobreakdash--4 podtablica $A[1\twodots i-1]$ jest posortowana niemalejąco.
\end{quote}
\begin{description}
	\item[Inicjowanie:] Przed pierwszą iteracją $i=1$, więc podtablica $A[1\twodots i-1]$ jest pusta, a~jako taka jest oczywiście posortowana niemalejąco.
	\item[Utrzymanie:] Z~założenia, że $A[1\twodots i-1]$ jest posortowana niemalejąco wynika, że $A[i-1]$ jest największym elementem tej podtablicy. Wewnętrzna pętla \kw{for} wyszukuje w~podtablicy $A[i\twodots n]$ najmniejszy element i~umieszcza go na pozycji $A[i]$ (dowód w~poprzednim punkcie). W~podtablicy $A[i\twodots n]$ nie ma mniejszych elementów od $A[i-1]$, a~zatem będzie w~szczególności zachodzić $A[i-1]\le A[i]$. Na podstawie założenia wnioskujemy, że podtablica $A[1\twodots i]$ jest posortowana niemalejąco i~po aktualizacji $i$ niezmiennik zostaje odtworzony.
	\item[Zakończenie:] Na końcu mamy $i=n+1$. Podtablica $A[1\twodots i-1]$ jest całą tablicą $A$ posortowaną niemalejąco, a~zatem algorytm sortuje poprawnie.
\end{description}

\subproblem %2-2(d)
Pętla \kw{for} z~wierszy 2\nobreakdash--4 wykonuje $n-i$ iteracji dla każdego $i=1$, 2,~\dots,~$n$. Czasem działania sortowania bąbelkowego dla wszystkich przypadków danych wejściowych jest zatem
\[
	T(n) = \sum_{i=1}^n(n-i) = \sum_{i=0}^{n-1}i = \frac{n(n-1)}{2} = \Theta(n^2).
\]
Pesymistyczny czas działania sortowania bąbelkowego jest więc równy pesymistycznemu czasowi sortowania przez wstawianie.

\problem{Poprawność schematu Hornera} %2-3

\subproblem %2-3(a)
Pętla \kw{while} w~wierszach 3\nobreakdash--5 wykonuje $n+1$ iteracji, więc czas działania tego fragmentu kodu wynosi $\Theta(n)$.

\subproblem %2-3(b)
Niech ciąg $A=\langle a_0,a_1,\dots,a_n\rangle$ składa się z~kolejnych współczynników wielomianu $P$. Następujący algorytm oblicza wartość $P(x)$.
\begin{codebox}
\Procname{$\proc{Naive-Polynomial-Evaluation}(A,x)$}
\li	$y\gets0$
\li	\For $i\gets0$ \To $n$
\li		\Do
			$s\gets a_i$
\li			\For $j\gets1$ \To $i$ \label{li:naive-pol-eval-for-begin}
\li				\Do $s\gets s\cdot\id{x}$
				\End \label{li:naive-pol-eval-for-end}
\li			$y\gets y+s$
		\End
\li	\Return $y$
\end{codebox}
Pętla \kw{for} w~wierszach \ref{li:naive-pol-eval-for-begin}\nobreakdash--\ref{li:naive-pol-eval-for-end} wykonuje się $i$ razy dla każdego $i=0$, 1,~\dots,~$n$, a~zatem czas działania powyższego algorytmu (pomijając czas stały, spędzony na przypisaniach) wynosi
\[
	T(n) = \sum_{i=0}^ni = \frac{n(n+1)}{2} = \Theta(n^2).
\]
Jest to więc mniej efektywny algorytm od schematu Hornera.

\subproblem %2-3(c)
\begin{description}
	\item[Inicjowanie:] Przed pierwszą iteracją $i=n$, więc $y=\sum_{k=0}^{-1}a_{k+n+1}x^k=0$, co zgadza się z~bieżącą wartością $y$.
	\item[Utrzymanie:] Podczas kolejnych iteracji $y$ przyjmuje wartość $a_i+xy$. Mamy zatem
	\[
		y = a_i+\sum_{k=0}^{n-(i+1)}a_{k+i+1}x^{k+1} = a_ix^0+\sum_{k=1}^{n-i}a_{k+i}x^k = \sum_{k=0}^{n-i}a_{k+i}x^k,
	\]
	a~stąd, po aktualizacji $i$, niezmiennik zostaje odtworzony.
	\item[Zakończenie:] Na końcu mamy $i=-1$, więc
	\[
		y = \sum_{k=0}^{n-(i+1)}a_{k+i+1}x^k = \sum_{k=0}^na_kx^k = P(x),
	\]
	zatem algorytm poprawnie oblicza wynik.
\end{description}

\subproblem %2-3(d)
Algorytm zwraca poprawny wynik, gdyż ustawia prawidłowe początkowe wartości, $y=0$ oraz $i=n$, a~poprawność pętli \kw{while} została wykazana w~poprzednim punkcie. Procedura posiada własność stopu, ponieważ zmienna $i$ jest zmniejszana w~kolejnych obiegach pętli, zatem po skończonej liczbie obiegów i~po skończonej liczbie kroków algorytmu, będzie zachodzić $i=0$, co jest warunkiem zakończenia pętli. Algorytm działa więc poprawnie.

\problem{Inwersje} %2-4

\subproblem %2-4(a)
$\langle1,5\rangle$, $\langle2,5\rangle$, $\langle3,4\rangle$, $\langle3,5\rangle$, $\langle4,5\rangle$.

\subproblem %2-4(b)
Największą możliwą liczbę inwersji posiada tablica o~różnych elementach posortowana malejąco. Każdy element na pozycji $i$ tworzy inwersję z~każdym z~$n-i$ elementów na prawo od niego w~tej tablicy. Liczba inwersji wynosi zatem
\[
	\sum_{i=1}^n(n-i) = \sum_{i=0}^{n-1}i = \frac{n(n-1)}{2}.
\]

\subproblem %2-4(c)
Załóżmy, że tablica $A$ posiada inwersję $(i,j)$. To znaczy, że $i<j$ oraz $A[i]>A[j]$. Wtedy w~procedurze \proc{Insertion-Sort} pewna iteracja pętli \kw{while} w~wierszach 5\nobreakdash--7 przesunie $A[i]$ o~jedną pozycję w~prawo, podczas gdy element będący pierwotnie w~$A[j]$ będzie znajdował się na lewo od niego, przez co wyeliminowana zostanie jedna z~inwersji. Tak więc każda iteracja pętli \kw{while} usuwa jedną inwersję tablicy $A$, skąd wnioskujemy, że ich liczba jest tego samego rzędu, co czas działania algorytmu sortowania przez wstawianie wykonanego na $A$.

\subproblem %2-4(d)
Załóżmy, że podczas sortowania procedurą \proc{Merge-Sort}, w~pewnym momencie jedna ze~scalanych podtablic $L$ zawiera element $a_x$ znajdujący się początkowo w~$A[x]$, a~druga podtablica $R$ -- element $a_y$ będący początkowo w~$A[y]$. Jeśli warunek z~wiersza~14 procedury \proc{Merge} zachodzi, to znaczy, że $a_x$ i~$a_y$ nie tworzą inwersji. W~przeciwnym przypadku $a_x>a_y$, a~ponieważ scalane podtablice są posortowane, to $a_y$ jest mniejsze od każdego dotychczas nieprzetworzonego elementu podtablicy $L$. Liczba elementów $A$ należących do $L$ wynosi $n_1$, zatem w~momencie przetwarzania $a_y$, jest w~niej $n_1-i+1$ nieskopiowanych elementów, a~więc tyle inwersji tworzy z~nimi $a_y$. Ponieważ od tego momentu element ten będzie z~nimi w~jednej podtablicy, to nie policzymy żadnej inwersji dwukrotnie.

W~ten sposób, modyfikując algorytm sortowania przez scalanie, policzymy liczbę wszystkich inwersji \twoparts{$n$}{elementowej} tablicy $A$ w~czasie $\Theta(n\lg n)$, czego efektem ubocznym jest jej posortowanie. W~procedurze \proc{Merge-Sort} wystarczy początkowo wyzerować licznik inwersji i~następnie sumować częściowe wyniki zwracane z~wywołań rekurencyjnych, a~w \proc{Merge} przy niespełnieniu warunku z~wiersza~14, doliczać odpowiednią liczbę inwersji tworzonych przez element $a_y$. Przeprowadzone rozumowanie prowadzi do poniższych pseudokodów.

\begin{codebox}
\Procname{$\proc{Count-Inversions}(A,p,r)$}
\li	$\id{inversions}\gets0$
\li	\If $p<r$
\li		\Then
			$q\gets\lfloor(p+r)/2\rfloor$
\li			$\id{inversions}\gets\id{inversions}+\,\proc{Count-Inversions}(A,p,q)$
\li			$\id{inversions}\gets\id{inversions}+\,\proc{Count-Inversions}(A,q+1,r)$
\li			$\id{inversions}\gets\id{inversions}+\,\proc{Merge-Inversions}(A,p,q,r)$
		\End
\li	\Return \id{inversions}
\end{codebox}

\begin{codebox}
\Procname{$\proc{Merge-Inversions}(A,p,q,r)$}
\li	$n_1\gets q-p+1$
\li	$n_2\gets r-q$
\li	utwórz tablice $L[1\twodots n_1+1]$ i~$R[1\twodots n_2+1]$
\li	\For $i\gets1$ \To $n_1$
\li		\Do $L[i]\gets A[p+i-1]$
		\End
\li	\For $j\gets1$ \To $n_2$
\li		\Do $R[j]\gets A[q+j]$
		\End
\li	$L[n_1+1]\gets\infty$
\li	$R[n_2+1]\gets\infty$
\li	$i\gets1$
\li	$j\gets1$
\li	$\id{inversions}\gets0$
\li	\For $k\gets p$ \To $r$
\li		\Do
\li			\If $L[i]\le R[j]$
\li				\Then
					$A[k]\gets L[i]$
\li					$i\gets i+1$
\li				\Else
					$A[k]\gets R[j]$
\li					$j\gets j+1$
\li					$\id{inversions}\gets\id{inversions}+\,n_1-i+1$
				\End
		\End
\li	\Return \id{inversions}
\end{codebox}

\endinput

\chapter{Rzędy wielkości funkcji}

\subchapter{Notacja asymptotyczna}

\exercise %3.1-1
Zbadajmy, czy dla pewnych stałych $c_1$,~$c_2$,~$n_0>0$ i~dla każdego $n\ge n_0$ prawdą jest
\begin{equation}
	0 \le c_1(f(n)+g(n)) \le \max(f(n),g(n)) \le c_2(f(n)+g(n)). \label{eq:3.1-1}
\end{equation}
Jeśli w~pewnym zbiorze $S$ zachodzi $f(n)\le g(n)$, to $\max(f(n),g(n))=g(n)$, więc
\[
	c_1f(n)+c_1g(n) \le c_1g(n)+c_1g(n) = 2c_1g(n) = 2c_1\max(f(n), g(n)).
\]
Wybierając dowolne $0<c_1\le1/2$ i~$n_0$ równe minimum zbioru $S$, spełniamy pierwsze dwie nierówności z~(\ref{eq:3.1-1}) dla wszystkich $n\ge n_0$ należących do $S$. Identyczny rezultat otrzymujemy w~przypadku przeciwnym, gdy $\max(f(n),g(n))=f(n)$. Ostatnia z~nierówności jest spełniona w~obu przypadkach, jeśli przyjmiemy $c_2=1$.

Z~przedstawionej argumentacji wnioskujemy, że $\max(f(n),g(n))=\Theta(f(n)+g(n))$.

\exercise %3.1-2
Aby pokazać, że $(n+a)^b=\Theta(n^b)$, należy znaleźć stałe $c_1$,~$c_2$,~$n_0>0$ takie, że
\[
	0 \le c_1n^b \le (n+a)^b \le c_2n^b
\]
dla wszystkich $n\ge n_0$. Zauważmy, że $n+a\le n+|a|\le2n$, gdy $|a|\le n$ oraz $n+a\ge n-|a|\ge n/2$, o~ile $|a|\le n/2$. Stąd, jeśli $n\ge 2|a|$, to zachodzi
\[
	0 \le n/2 \le n+a \le 2n.
\]
Ponieważ $b>0$, to powyższa nierówność jest spełniona również wtedy, gdy podniesiemy wszystkie jej składowe do potęgi $b$:
\begin{gather*}
	0 \le (n/2)^b \le (n+a)^b \le (2n)^b, \\
	0 \le (1/2)^bn^b \le (n+a)^b \le 2^bn^b.
\end{gather*}
Widać zatem, że szukanym stałym można nadać wartości $c_1=(1/2)^b$, $c_2=2^b$ oraz $n_0=2|a|$ i~prawdą jest, że $(n+a)^b=\Theta(n^b)$.

\exercise %3.1-3
Niech $T(n)$ będzie czasem działania algorytmu. $T(n)\ge O(n^2)$ oznacza, że $T(n)\ge f(n)$ dla pewnej funkcji $f(n)$ z~klasy $O(n^2)$. Stwierdzenie pozostaje prawdziwe dla każdego $T(n)$, wystarczy bowiem wybrać funkcję $f(n)=0$, która oczywiście jest w~$O(n^2)$. Widać więc, że takie określenie nic nam nie mówi o~oszacowaniu czasu działania algorytmu.

\exercise %3.1-4
Znajdźmy stałe $c$,~$n_0>0$ takie, że $0\le2^{n+1}\le c2^n$ dla każdego $n\ge n_0$. Ponieważ $2^{n+1}=2\cdot2^n$ dla każdego $n\ge0$, to można przyjąć $c=2$ oraz $n_0=1$. A~zatem $2^{n+1}=O(2^n)$.

Wyznaczmy teraz te same stałe, ale spełniające zależność $0\le2^{2n}\le c2^n$ dla wszystkich $n\ge n_0$. Mamy $2^{2n}=2^n\cdot2^n\le c2^n$, z~czego wynika, że $c\ge2^n$, co jednak uzależnia $c$ od $n$, a~zatem $c$ musiałoby być dowolnie duże i~nie może w~takim wypadku być stałą. Stąd otrzymujemy, że $2^{2n}\ne O(2^n)$.

\exercise %3.1-5
Z~definicji notacji $\Theta$ mamy, że $f(n)=\Theta(g(n))$ wtedy i~tylko wtedy, gdy istnieją takie stałe $c_1$,~$c_2$,~$n_0>0$, że
\[
	0 \le c_1g(n) \le f(n) \le c_2g(n)
\]
zachodzi dla wszystkich $n\ge n_0$. Potraktujmy powyższe wyrażenie jako koniunkcję nierówności:
\begin{equation}
	\begin{cases}
		0 \le c_1g(n) \le f(n), \\
		0 \le f(n) \le c_2g(n).
	\end{cases} \label{eq:3.1-5}
\end{equation}
Pierwsza z~nich stanowi definicję $f(n)=\Omega(g(n))$, a~druga -- $f(n)=O(g(n))$.

Dowód przeprowadzony w~odwrotnej kolejności pozwala wykazać implikację w~drugą stronę, gdyż koniunkcja~(\ref{eq:3.1-5}) doprowadza do nierówności z~definicji $f(n)=\Theta(g(n))$.

\exercise %3.1-6
Jeżeli pesymistyczny czas działania algorytmu wynosi $O(g(n))$, to dla dowolnych danych wejściowych oszacowanie czasu jego działania jest nie większe niż $c_1g(n)$ dla pewnej stałej $c_1>0$. Z~kolei optymistyczny czas $\Omega(g(n))$ oznacza, że dla dowolnych danych wejściowych oszacowanie czasu działania algorytmu jest nie mniejsze niż $c_2g(n)$ dla stałej $c_2>0$. Widać zatem, że dla dowolnych danych mamy $0\le c_2g(n)\le f(n)\le c_1g(n)$, gdzie $f(n)$ stanowi czas działania algorytmu, a~stąd otrzymujemy, że $f(n)=\Theta(g(n))$.

Przeprowadzenie powyższego rozumowania w~odwrotnej kolejności pozwala wykazać przeciwną implikację.

\exercise %3.1-7
Załóżmy, że twierdzenie jest fałszywe i~że istnieje pewna funkcja $f(n)$, która należy do zbioru $o(g(n))\cap\omega(g(n))$. Zachodzi zatem zarówno $f(n)=o(g(n))$ jak i~$f(n)=\omega(g(n))$, co oznacza, że dla każdych dodatnich stałych $c_1$ i~$c_2$ istnieje pewne dodatnie $n_0$, że
\[
	c_1g(n) < f(n) < c_2g(n)
\]
dla wszystkich $n\ge n_0$. Dochodzimy do sprzeczności, bowiem nieprawdą jest, że $c_1<c_2$ dla każdych liczb $c_1$ i~$c_2$, skąd wnioskujemy, że zbiór $o(g(n))\cap\omega(g(n))$ jest pusty.

Dzięki pustości tej klasy, nie ma potrzeby definiowania notacji $\theta$ odpowiadającej $\Theta$ i~analogicznej do $o$ i~$\omega$.

\exercise %3.1-8
\[
	\begin{split}
		\Omega(g(n,m)) &\stackrel{\text{\scriptsize def}}{=} \bigl\{\,f(n,m):\text{istnieją dodatnie stałe $c$,~$n_0$,~$m_0$ takie, że} \\
		&\qquad 0 \le cg(n,m) \le f(n,m) \text{ dla wszystkich $n \ge n_0$ oraz $m \ge m_0$}\,\bigr\} \\%[3mm]
		\Theta(g(n,m)) &\stackrel{\text{\scriptsize def}}{=} \bigl\{f(n,m):\text{istnieją dodatnie stałe $c_1$,~$c_2$,~$n_0$,~$m_0$ takie, że} \\
		&\qquad 0 \le c_1g(n,m) \le f(n,m) \le c_2g(n,m) \text{ dla wszystkich $n \ge n_0$ oraz $m \ge m_0$}\,\bigr\}
	\end{split}
\]

\raggedbottom
\subchapter{Standardowe notacje i~typowe funkcje}

\exercise %3.2-1
Z~założenia, jeśli $n_1\le n_2$, to zachodzi
\begin{equation}
	f(n_1) \le f(n_2) \quad\text{oraz}\quad g(n_1) \le g(n_2), \label{eq:3.2-1}
\end{equation}
więc po dodaniu nierówności stronami otrzymujemy $f(n_1)+g(n_1)\le f(n_2)+g(n_2)$, czyli że $f+g$ jest funkcją monotonicznie rosnącą.

Ponieważ $g(n_1)\le g(n_2)$, to traktując wartości funkcji $g$ jako argumenty funkcji $f$, otrzymamy $f(g(n_1))\le f(g(n_2))$, zatem $f\circ g$ jest funkcją monotonicznie rosnącą.

Jeśli ponadto założymy, że $f$ i~$g$ są nieujemne, to nierówności~(\ref{eq:3.2-1}) można pomnożyć stronami bez konieczności zmiany znaku nierówności, co daje $f(n_1)f(n_2)\le g(n_1)g(n_2)$, a~to oznacza, że również funkcja $f\cdot g$ jest monotonicznie rosnąca.

\exercise %3.2-2
Wykorzystując podstawowe własności logarytmów otrzymujemy
\[
	\log_ba^{\log_bc} = \log_bc\cdot\log_ba = \log_bc^{\log_ba}.
\]
Na podstawie różnowartościowości funkcji logarytmicznej wynika tożsamość
\[
	a^{\log_bc} = c^{\log_ba}.
\]

\exercise %3.2-3
\begin{proof}[Dowód wzoru~(3.18)]
	Z~wzoru Stirlinga dostajemy
	\[
		\lg(n!) = \lg\left(\sqrt{2\pi n}\left(\frac{n}{e}\right)^n\bigl(1+\Theta(1/n)\bigr)\right).
	\]
	Wykorzystując teraz definicję notacji $\Theta$ oraz twierdzenie~3.1 i~wybierając pewną stałą $c_1>0$, ograniczamy $\lg(n!)$ od dołu:
	\begin{align*}
		\lg(n!) &\ge \lg\left(\sqrt{2\pi n}\left(\frac{n}{e}\right)^n\left(1+\frac{c_1}{n}\right)\right) \\
		&= \lg\sqrt{2\pi}+\lg\sqrt{n}+\lg\left(\frac{n}{e}\right)^n+\lg\left(1+\frac{c_1}{n}\right) \\
		&= \lg\sqrt{2\pi}+\frac{\lg n}{2}+n\lg n-n\lg e+\lg\left(1+\frac{c_1}{n}\right) \\
		&\ge n\lg n-n\lg e\\
		&\ge n\lg n-\frac{n\lg n}{2} \\
		&= \frac{n\lg n}{2}.
	\end{align*}
	Przedostatnia nierówność zachodzi dla $n\ge e^2$. Otrzymany wynik dowodzi, że $\lg(n!)=\Omega(n\lg n)$. Wybierając inną stałą $c_2>0$ znajdujemy ograniczenie górne:
	\begin{align*}
		\lg(n!) &\le \lg\left(\sqrt{2\pi n}\left(\frac{n}{e}\right)^n\left(1+\frac{c_2}{n}\right)\right) \\
		&= \lg\sqrt{2\pi}+\lg\sqrt{n}+\lg\left(\frac{n}{e}\right)^n+\lg\left(1+\frac{c_2}{n}\right) \\
		&= \lg\sqrt{2\pi}+\frac{\lg n}{2}+n\lg n-n\lg e+\lg(n+c_2)-\lg n\\
		&\le \lg\sqrt{2\pi}\cdot n\lg n+\frac{n\lg n}{2}+n\lg n+n\lg n\\
		&= \Bigl(\lg\sqrt{2\pi}+\frac{5}{2}\Bigr)\cdot n\lg n,
	\end{align*}
	przy czym w~przedostatnim kroku wykorzystano \hbox{m.in.} zależność $\lg(n+c_2)\le n\lg n$, która zachodzi dla $n$ takich, że $n^n-n\ge c_2$. Mamy zatem $\lg(n!)=O(n\lg n)$. Po ponownym skorzystaniu z~twierdzenia~3.1 dostajemy $\lg(n!)=\Theta(n\lg n)$.
\end{proof}

\flushbottom
\begin{proof}[Dowód tożsamości $n!=\omega(2^n)$]
	Pokażemy, że zachodzi
	\[
		\lim_{n\to\infty}\frac{n!}{2^n} = \infty.
	\]
	Rozwijamy $n!$ wykorzystując wzór Stirlinga i~dostajemy:
	\begin{align*}
		\lim_{n\to\infty}\frac{\sqrt{2\pi n}\left(\frac{n}{e}\right)^n\bigl(1+\Theta(1/n)\bigr)}{2^n} &= \lim_{n\to\infty}\frac{n^n\sqrt{2\pi n}\;\bigl(1+\Theta(1/n)\bigr)}{(2e)^n} \\
		&= \sqrt{2\pi}\;\cdot\lim_{n\to\infty}\frac{n^{n+1/2}}{(2e)^n}\;\cdot\lim_{n\to\infty}\bigl(1+\Theta(1/n)\bigr) \\
		&= \infty.
	\end{align*}
	Ostania równość zachodzi, ponieważ wyrazy ciągu
	\[
		a_n = \left(\frac{n^{n+1/2}}{(2e)^n}\right)
	\]
	rosną nieograniczenie oraz na podstawie obserwacji, że
	\begin{equation}
		\lim_{n\to\infty}\bigl(1+\Theta(1/n)\bigr) \le \lim_{n\to\infty}\left(1+\frac{c}{n}\right) = 1, \label{eq:3.2-3}
	\end{equation}
	gdzie $c>0$ jest dowolną stałą.
\end{proof}

\begin{proof}[Dowód tożsamości $n!=o(n^n)$]
	Na podstawie~(3.1) widać, że dowód sprowadza się do wykazania
	\[
		\lim_{n\to\infty}\frac{n!}{n^n} = 0.
	\]
	Znajdujemy oszacowanie $n!$ przy pomocy wzoru Stirlinga otrzymując:
	\begin{align*}
		\lim_{n\to\infty}\frac{\sqrt{2\pi n}\left(\frac{n}{e}\right)^n\bigl(1+\Theta(1/n)\bigr)}{n^n} &= \lim_{n\to\infty}\frac{\sqrt{2\pi n}\;\bigl(1+\Theta(1/n)\bigr)}{e^n} \\
		&= \sqrt{2\pi}\;\cdot\lim_{n\to\infty}\frac{\sqrt{n}}{e^n}\;\cdot\lim_{n\to\infty}\bigl(1+\Theta(1/n)\bigr) \\
		&= 0.
	\end{align*}
	W ostatnim kroku korzystamy z~obserwacji~(\ref{eq:3.2-3}) z~poprzedniego dowodu oraz z~wzoru~(3.9) dla $a=e$ i~$b=1/2$.
\end{proof}

\exercise %3.2-4
Jeśli pewna funkcja $f(n)$ jest ograniczona wielomianowo, to istnieją stałe $c$,~$k$,~$n_0>0$ takie, że dla każdego $n\ge n_0$ zachodzi $f(n)\le cn^k$. Stąd $\lg f(n)\le k\lg n+\lg c\le(k+1)\lg n$, o~ile $n\ge c$, a~więc $\lg f(n)=O(\lg n)$. Stwierdzenie, że funkcja $f(n)$ jest ograniczona wielomianowo, jest przez to równoważne stwierdzeniu, że $\lg f(n)=O(\lg n)$.

Zanim przejdziemy do głównego dowodu, zauważmy, że $\lceil\lg n\rceil=\Theta(\lg n)$. Zachodzi bowiem $\lceil\lg n\rceil\ge\lg n$ oraz $\lceil\lg n\rceil<\lg n+1\le2\lg n$ dla każdego $n\ge2$.

Logarytmując pierwszą badaną funkcję przy wykorzystaniu wzoru~(3.18), dostajemy
\begin{align*}
	\lg(\lceil\lg n\rceil!) &= \Theta(\lceil\lg n\rceil\lg\lceil\lg n\rceil) \\
	&= \Theta(\lg n\lg\lg n) \\
	&= \omega(\lg n),
\end{align*}
a~zatem $\lg(\lceil\lg n\rceil!)\ne O(\lg n)$ i~$\lceil\lg n\rceil$ nie jest ograniczone wielomianowo.

Dla drugiej funkcji mamy
\begin{align*}
	\lg(\lceil\lg\lg n\rceil!) &= \Theta(\lceil\lg\lg n\rceil\lg\lceil\lg\lg n\rceil) \\
	&= \Theta(\lg\lg n\lg\lg\lg n) \\
	&= o(\lg^2\lg n) \\
	&= o(\lg n).
\end{align*}
Ostatni krok wynika z~tożsamości $\lg^bn=o(n^a)$, w~której podstawiono $\lg n$ w~miejsce $n$ oraz przyjęto $b=2$ i~$a=1$. Otrzymany rezultat potwierdza, że $\lg(\lceil\lg\lg n\rceil!)=O(\lg n)$, a~zatem $\lceil\lg\lg n\rceil$ jest ograniczone wielomianowo.

\exercise %3.2-5
Zachodzi równość $\lg^*\lg n=\lg^*n-1$. Jej prawa strona jest funkcją liniową $h(n)-1$ dla $h=\lg^*$, podczas gdy $\lg\lg^*n\equiv\lg h(n)$ to funkcja logarytmiczna względem $h(n)$. Ponieważ $\lg h(n)=o(h(n)-1)$, to stąd mamy
\[
	\lg\lg^*n = o(\lg^*\lg n).
\]

\exercise %3.2-6
Dla $i=0$ twierdzenie zachodzi trywialnie. Dla $i=1$ mamy $F_1=\bigl(\phi-\widehat\phi\bigr)/\sqrt{5}=1$. Załóżmy teraz, że zachodzi
\[
	F_i = \frac{\phi^i-\widehat\phi^i}{\sqrt{5}} \quad\text{oraz}\quad F_{i+1} = \frac{\phi^{i+1}-\widehat\phi^{i+1}}{\sqrt{5}}
\]
dla pewnego $i\ge0$. Ponieważ liczby Fibonacciego dla wszystkich $n\ge0$ spełniają zależność $F_{n+2}=F_{n+1}+F_n$, to stąd na mocy założenia indukcyjnego zachodzi
\[
	F_{i+2} = \frac{\phi^{i+1}-\widehat\phi^{i+1}}{\sqrt{5}}+\frac{\phi^i-\widehat\phi^i}{\sqrt{5}} = \frac{\phi^i\overbrace{(\phi+1)}^{\phi^2}-\widehat\phi^i\overbrace{\bigl(\widehat\phi+1\bigr)}^{\widehat\phi^2}}{\sqrt{5}} = \frac{\phi^{i+2}-\widehat\phi^{i+2}}{\sqrt{5}},
\]
a~zatem zależność jest prawdziwa dla każdego $i\ge0$.

\exercise %3.2-7
Korzystając z~wyniku z~poprzedniego zadania, dostajemy
\begin{align*}
	F_{i+2} &\ge \phi^i \\
	\frac{\phi^{i+2}-\widehat\phi^{i+2}}{\sqrt{5}} &\ge \phi^i \\
	\phi^i\bigl(\phi^2-\sqrt{5}\bigr) &\ge \widehat\phi^{i+2} \\
	\phi^i\cdot\frac{3-\sqrt{5}}{2} &\ge \widehat\phi^i\cdot\frac{3-\sqrt{5}}{2} \\
	\phi^i &\ge \widehat\phi^i.
\end{align*}
Ponieważ $|\phi|>|\widehat\phi|$, to otrzymana nierówność jest spełniona dla każdego $i\ge0$ i~dowód jest zakończony.

\problems

\problem{Asymptotyczne zachowanie wielomianów} %3-1
Udowodnimy najpierw fakt, że $p(n)=\Theta(n^d)$. Należy znaleźć stałe $c_1$,~$c_2$,~$n_0>0$ takie, że dla $n>n_0$ będzie prawdą
\[
	0 \le c_1n^d \le a_dn^d+a_{d-1}n^{d-1}+\dots+a_0 \le c_2n^d.
\]
Po podzieleniu nierówności przez $n^d$, dostajemy
\[
	0 \le c_1 \le a_d+\underbrace{\frac{a_{d-1}}{n}+\dots+\frac{a_0}{n^d}}_\epsilon \le c_2,
\]
a~ponieważ składnik $\epsilon$ można dowolnie zbliżyć do zera zwiększając parametr $n_0$, to stąd obie stałe $c_1$, $c_2$ mogą być dowolnie bliskie $a_d$. To kończy dowód.

\subproblem %3-1(a)
$p(n)=\Theta(n^d)$, zatem na mocy tw.~3.1 zachodzą równości $p(n)=O(n^d)$ oraz $p(n)=\Omega(n^d)$. Pierwsza z~nich oznacza, że dla pewnych stałych $c$,~$n_0>0$ prawdą jest $0\le p(n)\le cn^d$ dla wszystkich $n\ge n_0$. Z~kolei $k\ge d$ implikuje $cn^k\ge cn^d$, a~zatem $0\le p(n)\le cn^k$, skąd natychmiast otrzymujemy $p(n)=O(n^k)$.

\subproblem %3-1(b)
$k\le d$ implikuje $cn^k\le cn^d$ dla stałej $c$ z~poprzedniego punktu, a~korzystając z~tego, że $p(n)=\Omega(n^d)$ mamy $0\le cn^k\le p(n)$, skąd wynika $p(n)=\Omega(n^k)$.

\subproblem %3-1(c)
Dla $k=d$, tożsamość $p(n)=\Theta(n^d)=\Theta(n^k)$ trywialnie zachodzi.

\subproblem %3-1(d)
Jeśli $k>d$, to $cn^k>cn^d$ dla wszystkich $c>0$, skąd wynika, że $0\le p(n)<cn^k$, a~to prowadzi do równości $p(n)=o(n^k)$.

\subproblem %3-1(e)
Analogicznie jak w~poprzednim punkcie, $k<d$ implikuje $cn^k<cn^d$ dla wszystkich $c>0$, a~zatem $0\le cn^k<p(n)$ i~otrzymujemy $p(n)=\omega(n^k)$.

\problem{Względny czas asymptotyczny} %3-2

\subproblem %3-2(a)
Ponieważ każdy wielomian rośnie szybciej niż dowolna funkcja polilogarytmiczna, czyli $\lg^kn=o(n^\epsilon)$, to~stąd wynika, że również $\lg^kn=O(n^\epsilon)$.

\subproblem %3-2(b)
Podobnie, z~wzoru~(3.9) mamy, że $n^k=o(c^n)$, co implikuje również $n^k=O(c^n)$.

\subproblem %3-2(c)
Funkcja $\sqrt{n}$ nie jest w~żadnej relacji z~$n^{\sin n}$, gdyż wartość wykładnika tej ostatniej waha się między $-1$ a~1 przyjmując wszystkie pośrednie wartości, podczas gdy $\sqrt{n}\equiv n^{1/2}$.

\subproblem %3-2(d)
Zachodzi $2^n=\omega(2^{n/2})$, bo
\[
	\lim_{n\to\infty}\frac{2^n}{2^{n/2}} = \lim_{n\to\infty}2^{n/2} = \infty,
\]
a~stąd prawdą jest także $2^n=\Omega(2^{n/2})$.

\subproblem %3-2(e)
Funkcje $n^{\lg c}$ i~$c^{\lg n}$ dla $n>0$ są równoważne na podstawie tożsamości~(3.15).

\subproblem %3-2(f)
Z~wzoru~(3.18), $\lg(n!)=\Theta(n\lg n)$, a~$\lg(n^n)=n\lg n=\Theta(n\lg n)$, zatem obie funkcje są asymptotycznie równoważne.

\bigskip
\noindent Na podstawie powyższych faktów dostajemy tabelę~\ref{tab:3-2}.
\begin{table}[ht]
	\begin{center}
		\begin{tabular}{cc|c|c|c|c|c|}
			$A$ & $B$ & $O$ & $o$ & $\Omega$ & $\omega$ & $\Theta$ \\
			\hline
			$\lg^kn$ & $n^\epsilon$ & tak & tak & nie & nie & nie \\
			\hline
			$n^k$ & $c^n$ & tak & tak & nie & nie & nie \\
			\hline
			$\sqrt{n}$ & $n^{\sin n}$ & nie & nie & nie & nie & nie \\
			\hline
			$2^n$ & $2^{n/2}$ & nie & nie & tak & tak & nie \\
			\hline
			$n^{\lg c}$ & $c^{\lg n}$ & tak & nie & tak & nie & tak \\
			\hline
			$\lg(n!)$ & $\lg(n^n)$ & tak & nie & tak & nie & tak \\
			\hline
		\end{tabular}
		\caption{Relacje między przykładowymi funkcjami} \label{tab:3-2}
	\end{center}
\end{table}

\problem{Porządkowanie ze względu na rząd wielkości funkcji} %3-3

\subproblem %3-3(a)
Poniższe uzasadnienia stanowią dowody $g_i=\Omega(g_{i+1})$ dla $i=1$, 2,~\dots,~29, gdzie $g_i$ są rozważanymi funkcjami. Zauważmy, że bycie w~relacji $\omega$ jest warunkiem wystarczającym do bycia w~relacji $\Omega$. W~wielu przypadkach korzystamy z~obserwacji, że jeśli $f(n)=g(n)h(n)$ i~$g(n)=\omega(1)$, to prawdą jest $f(n)=\omega(h(n))$. Ponadto, jeśli zachodzi $\lg f(n)=\omega(\lg g(n))$, to prawdziwe jest także $f(n)=\omega(g(n))$.
\begin{itemize}
\item $2^{2^{n+1}}=\Omega\bigl(2^{2^n}\bigr)$
	\[
		2^{2^{n+1}} = 2^{2^n}\cdot2^{2^n} = \omega\bigl(2^{2^n}\bigr), \quad\text{bo $2^{2^n} = \omega(1)$}.
	\]
\item $2^{2^n}=\Omega((n+1)!)$ \\
	Logarytmując obie funkcje i~wykorzystując wzór~(3.18), otrzymujemy
	\[
		\lg 2^{2^n} = 2^n \quad\text{oraz}\quad \lg((n+1)!) = \Theta((n+1)\lg(n+1)) = \Theta(n\lg n).
	\]
	Ponieważ $2^n=\omega(n^2)$ oraz $n^2=\omega(n\lg n)$, to mamy, że $2^n=\omega(n\lg n)$. Powracając do oryginalnych funkcji dostajemy $2^{2^n}=\omega((n+1)!)$, skąd wynika prawdziwość początkowej zależności.
\item $(n+1)!=\Omega(n!)$
	\[
		(n+1)! = (n+1)\cdot n! = \omega(n!), \quad\text{bo $n+1 = \omega(1)$}.
	\]
\item $n!=\Omega(e^n)$ \\
	Zbadajmy logarytmy obu funkcji. Zachodzi $\lg e^n=\Theta(n)$, a~z~wzoru~(3.18) mamy $\lg(n!)=\Theta(n\lg n)=\omega(n)$. Prawdą jest zatem, że $\lg(n!)=\omega(\lg e^n)$, a~stąd wynika początkowa zależność.
\item $e^n=\Omega(n\cdot2^n)$
	\[
		e^n = (e/2)^n2^n = \omega(n\cdot2^n), \quad\text{bo $(e/2)^n = \omega(n)$},
	\]
	ponieważ funkcje wykładnicze rosną szybciej niż wielomiany.
\item $n\cdot2^n=\Omega(2^n)$ \\
	Tożsamość zachodzi, bo $n=\omega(1)$.
\item $2^n=\Omega((3/2)^n)$
	\[
		2^n = (4/3)^n(3/2)^n = \omega((3/2)^n), \quad\text{bo $(4/3)^n = \omega(1)$}.
	\]
\item $(3/2)^n=\Omega\bigl(n^{\lg\lg n}\bigr)$ \\
	Logarytmując obie funkcje dostajemy
	\[
		\lg(3/2)^n = \Theta(n) \quad\text{oraz}\quad \lg n^{\lg\lg n} = \lg n\lg\lg n = o(\lg^2n)
	\]
	Wystarczy pokazać, że $n=\omega(\lg^2n)$. Podstawiając $n=2^h$ dostajemy nierówność $2^h>ch^2$ prawdziwą dla wszystkich $c>0$ oraz $n\ge5$, co oznacza, że początkowa zależność jest prawdziwa.
\item $n^{\lg\lg n}=\Omega\bigl((\lg n)^{\lg n}\bigr)$ \\
	Na mocy tożsamości~(3.15) zachodzi $n^{\lg\lg n}=(\lg n)^{\lg n}$.
\item $(\lg n)^{\lg n}=\Omega((\lg n)!)$ \\
	Badając logarytmy obu funkcji, dostajemy $\lg\bigl((\lg n)^{\lg n}\bigr)=\Theta(\lg n\lg\lg n)$ oraz $\lg((\lg n)!)=\Theta(\lg n\lg\lg n)$ (z wzoru~(3.18)), a~zatem zależność jest prawdziwa.
\item $(\lg n)!=\Omega(n^3)$ \\
	Korzystając z~logarytmu pierwszej funkcji obliczonego w~poprzednim uzasadnieniu oraz z~tego, że $\lg n^3=\Omega(\lg n)$, udowadniamy zależność, ponieważ $\lg\lg n=\omega(1)$.
\item $n^3=\Omega(n^2)$ \\
	Tożsamość zachodzi trywialnie, bo $n^3=n\cdot n^2=\omega(n^2)$ oraz $n=\omega(1)$.
\item $n^2=\Omega\bigl(4^{\lg n}\bigr)$ \\
	Funkcje są tożsame -- na podstawie wzoru~(3.15) mamy $4^{\lg n}=n^{\lg4}=n^2$.
\item $4^{\lg n}=\Omega(n\lg n)$ \\
	Ponieważ $4^{\lg n}=n^2$, a~$n=\omega(\lg n)$, to stąd tożsamość zachodzi.
\item $n\lg n=\Omega(\lg(n!))$ \\
	Prawdziwość tożsamości wynika z~wzoru~(3.18).
\item $\lg(n!)=\Omega(n)$ \\
	Z~wzoru~(3.18) mamy, że $\lg(n!)=\Theta(n\lg n)$, więc tożsamość jest prawdziwa, bo $\lg n=\omega(1)$.
\item $n=\Omega\bigl(2^{\lg n}\bigr)$ \\
	Funkcje są tożsame, bo $2^{\lg n}=n^{\lg2}=n$ na mocy wzoru~(3.15).
\item $2^{\lg n}=\Omega\Bigl(\!\sqrt{2}^{\lg n}\Bigr)$ \\
	Z~poprzedniego uzasadnienia mamy, że $2^{\lg n}=n$, a~$\sqrt{2}^{\lg n}=n^{\lg\sqrt{2}}=\sqrt{n}$ z~wzoru~(3.15), więc tożsamość zachodzi, ponieważ $\sqrt{n}=\omega(1)$.
\item $\sqrt{2}^{\lg n}=\Omega\Bigl(2^{\sqrt{2\lg n}}\Bigr)$ \\
	Rozważmy tożsamość $2^{\lg n}=n$ i~podnieśmy ją do potęgi $\sqrt{2/\!\lg n}$. Otrzymujemy $2^{\sqrt{2\lg n}}=n^{\sqrt{2/\!\lg n}}$, a~zatem $2^{\sqrt{2\lg n}}=\Theta\Bigl(n^{\sqrt{2/\!\lg n}}\Bigr)$. Ponieważ $\sqrt{2}^{\lg n}=\sqrt{n}$, to wystarczy pokazać, że $1/2=\Omega\bigl(\!\sqrt{2/\!\lg n}\bigr)$. Tożsamość oczywiście zachodzi, gdyż funkcja z~prawej strony jest malejąca i~dąży do 0~wraz ze wzrostem $n$.
\item $2^{\sqrt{2\lg n}}=\Omega(\lg^2 n)$ \\
	Biorąc logarytmy obu funkcji, dostajemy
	\[
		\lg2^{\sqrt{2\lg n}} = \sqrt{2\lg n} = \Theta\bigl(\lg^{1/2}n\bigr) \quad\text{oraz}\quad \lg\lg^2n = \Theta(\lg\lg n)
	\]
	Pozostaje zatem zbadać prawdziwość tożsamości $\lg^{1/2}n=\Omega(\lg\lg n)$. Biorąc $n=2^{4^h}$, sprowadzamy ją do postaci $2^h=\Omega(h)$, co oczywiście zachodzi, podobnie jak początkowa zależność.
\item $\lg^2n=\Omega(\ln n)$ \\
	Zależność jest prawdziwa, ponieważ $\ln n=\Theta(\lg n)$ oraz $\lg n=\omega(1)$.
\item $\ln n=\Omega\bigl(\!\sqrt{\lg n}\bigr)$ \\
	Wystarczy przyjąć $n=e^h$, aby otrzymać tożsamość $h=\Omega\bigl(\!\sqrt{h}\bigr)$, która zachodzi na mocy tego, że $\sqrt{h}=\omega(1)$.
\item $\sqrt{\lg n}=\Omega(\ln\ln n)$ \\
	$\Omega(\ln\ln n)$ jest równoważne $\Omega(\lg\lg n)$, zatem przyjmując $n=2^{4^h}$ dostajemy tożsamość $2^h=\Omega(h)$.
\item $\ln\ln n=\Omega\bigl(2^{\lg^*n}\bigr)$ \\
	Logarytmując funkcje, dostajemy
	\[
		\lg\ln\ln n~= \Theta\bigl(\lg^{(3)}n\bigr) \quad\text{oraz}\quad \lg\bigl(2^{\lg^*n}\bigr) = \lg^*n.
	\]
	By wykazać prawdziwość tożsamości, dokonajmy podstawienia
	\[
		n = 2^{2^{2^{\cdot^{\cdot^{\cdot^\epsilon}}}}}\vbox{\hbox{$\Big\}\scriptstyle k$}\kern0pt},
	\]
	gdzie $0<\epsilon\le1$ i~$k>3$. Wyliczając wartości obu funkcji dla takiego argumentu, otrzymujemy
	\[
		\lg^{(3)}(n) = 2^{2^{\cdot^{\cdot^{\cdot^\epsilon}}}}\vbox{\hbox{$\Big\}\scriptstyle k-3$}\kern0pt} \quad\text{oraz}\quad \lg^*(n)=k.
	\]
	Oczywistym jest, że pierwsza wartość jest większa od drugiej dla dostatecznie dużych $n$, zatem zależność zachodzi.
\item $2^{\lg^*n}=\Omega(\lg^*n)$ \\
	Po zlogarytmowaniu obu funkcji i~wykorzystaniu wzoru~(3.15), mamy
	\[
		\lg2^{\lg^*n} = \lg^*n \quad\text{oraz}\quad \lg\lg^*n.
	\]
	Po podstawieniu $h=\lg^*n$, sprowadzamy tożsamość do udowodnionej wcześniej $h=\Omega(\lg h)$, a~zatem początkowa zależność jest prawdziwa.
\item $\lg^*n=\Omega(\lg^*\lg n)$ \\
	Zależność zachodzi, gdyż $n=\omega(\lg n)$, a~$\lg^*x$ jest funkcją niemalejącą.
\item $\lg^*\lg n=\Omega(\lg\lg^*n)$ \\
	Tożsamość zachodzi na podstawie wyniku \zad{3.2-5}.
\item $\lg\lg^*n=\Omega\bigl(n^{1/\!\lg n}\bigr)$ \\
	Z własności logarytmów mamy, że $1/\!\lg n=\log_n2$, a~więc wykorzystując wzór~(3.15) dostajemy $n^{1/\!\lg n}=n^{\log_n2}=2^{\log_nn}=2=\Theta(1)$, a~stąd wynika prawdziwość zależności.
\item $n^{1/\!\lg n}=\Omega(1)$ \\
	Tożsamość zachodzi, gdyż z~poprzedniego uzasadnienia, $n^{1/\!\lg n}=\Theta(1)$.
\end{itemize}

Tabela~\ref{tab:3-3} przedstawia badane funkcje uporządkowane względem relacji $\Omega$ na podstawie powyższych uzasadnień. Funkcje znajdujące się w~tym samym polu należą do tej samej klasy równoważności.
\begin{table}[ht]
	\begin{center}
		\[
			\begin{array}{|lc|lc|lc|} \hline
				g_1 & 2^{2^{n+1}} & g_{11} & (\lg n)! & g_{21} & \lg^2n \\ \hline
				g_2 & 2^{2^n} & g_{12} & n^3 & g_{22} & \ln n \\ \hline
				g_3 & (n+1)! & g_{13} & 4^{\lg n} & g_{23} & \sqrt{\lg n} \\ \cline{1-2}\cline{5-6}
				g_4 & n! & g_{14} & n^2 & g_{24} & \ln\ln n \\ \hline
				g_5 & e^n & g_{15} & \lg (n!) & g_{25} & 2^{\lg^*n} \\ \cline{1-2}\cline{5-6}
				g_6 & n\cdot2^n & g_{16} & n\lg n~& g_{26} & \lg^*\lg n \\ \cline{1-4}
				g_7 & 2^n & g_{17} & 2^{\lg n} & g_{27} & \lg^*n \\ \cline{1-2}\cline{5-6}
				g_8 & (3/2)^n & g_{18} & n~& g_{28} & \lg\lg^*n \\ \hline
				g_9 & (\lg n)^{\lg n} & g_{19} & \bigl(\!\sqrt{2}\bigr)^{\lg n} & g_{29} & n^{1/\!\lg n} \\ \cline{3-4}
				g_{10} & n^{\lg\lg n} & g_{20} & 2^{\sqrt{2\lg n}} & g_{30} & 1~\\ \hline
			\end{array}
		\]
	\end{center}
	\caption{Uporządkowanie funkcji względem relacji $\Omega$} \label{tab:3-3}
\end{table}

\subproblem %3-3(b)
Przykładem funkcji nie będącej w~relacji $\Omega$ z~żadnym z~$g_1$,~\dots,~$g_{30}$ jest
\[
	f(n) =
	\begin{cases}
		2^{2^{n+2}}, & \text{dla $n$ parzystych,} \\
		0, & \text{dla $n$ nieparzystych}.
	\end{cases}
\]
Gdyby rozważać funkcję $f$ w~dziedzinie liczb parzystych, to byłaby ona w~relacji $\Omega$ z~$g_1$, ponieważ wartości pierwszej rosłyby o~wiele szybciej od drugiej. Z~kolei obcinając dziedzinę do liczb nieparzystych, $f$ byłoby na końcu listy funkcji w~uporządkowaniu z~poprzedniego punktu. Dlatego wraz ze wzrostem argumentu, $f$ jest ``większe'' od wszystkich funkcji $g_i$, albo od nich ``mniejsze'' w~zależności od parzystości $n$. To sprawia, że $f$ nie może być w~relacji $\Omega$ z~żadną funkcją $g_i$.

\problem{Własności notacji asymptotycznej} %3-4

\subproblem %3-4(a)
Fałsz. Niech np.\ $f(n)=n$ i~$g(n)=n^2$. Wtedy $f(n)=O(g(n))$, ale $g(n)\ne O(f(n))$.

\subproblem %3-4(b)
Fałsz. Jako kontrprzykład rozważmy $f(n)=n$ i~$g(n)=n^2$. Zachodzi wtedy
\[
	\min(f(n),g(n)) = f(n) \quad\text{oraz}\quad f(n)+g(n) = \Theta(g(n)) \ne \Theta(f(n)).
\]

\subproblem %3-4(c)
Prawda. Z~faktu, że $f(n)=O(g(n))$ wynika $f(n)\le cg(n)$ dla $n\ge n_0$ i~pewnych stałych $c$,~$n_0>0$. Otrzymujemy
\[
	\lg f(n) \le \lg c+\lg g(n) \le \lg g(n)+\lg g(n) = 2\lg g(n) = O(\lg g(n)).
\]
Ponieważ $c$ jest stałą, to przyjmijmy, że dobrano $n_0$ tak, by $c\le g(n)$ dla wszystkich $n\ge n_0$. Wtedy mamy $\lg c\le\lg g(n)$ i~stąd wynika druga nierówność.

\subproblem %3-4(d)
Fałsz. Dla funkcji $f(n)=2^n$ oraz $g(n)=2^{n+1}$ zachodzi $f(n)=O(g(n))$, ale $2^{f(n)}\ne O\bigl(2^{g(n)}\bigr)$ (z punktu~(a) problemu~3-3).

\subproblem %3-4(e)
Fałsz. Np.\ dla $f(n)=2^n$ mamy $f^2(n)=(2^n)^2=4^n$, skąd $f(n)\ne O\bigl(f^2(n)\bigr)$.

\subproblem %3-4(f)
Prawda. Z~definicji notacji $O$, jeśli $f(n)=O(g(n))$, to istnieją stałe $c$,~$n_0>0$, że dla każdego $n\ge n_0$ zachodzi $0\le f(n)\le cg(n)$. Dzieląc nierówność przez $c$ otrzymujemy $0\le f(n)/c\le g(n)$, przy czym $1/c>0$, a~więc $g(n)=\Omega(f(n))$.

\subproblem %3-4(g)
Fałsz. Niech np.\ $f(n)=2^n$ i~wtedy $f(n/2)=2^{n/2}={\bigl(\!\sqrt{2}\bigr)}^n$ oraz $f(n)\ne O(f(n/2))$.

\subproblem %3-4(h)
Prawda. Niech $h(n)=o(f(n))$. Wtedy, na podstawie definicji notacji $o$ mamy, że dla każdej stałej $c>0$ istnieje stała $n_0>0$, taka że
\[
	0 \le h(n) < cf(n)
\]
zachodzi dla wszystkich $n\ge n_0$. To znaczy, że
\[
	f(n) \le f(n)+o(f(n)) = f(n)+h(n) < (c+1)f(n).
\]
Ponieważ $c+1>1$, to można $f(n)+o(f(n))$ ograniczyć od góry przez $c_2f(n)$ wybierając np.\ $c_2=2$. Dolnym ograniczeniem tej sumy jest $f(n)$, więc ustalamy $c_1=1$. Stałe $c_1$, $c_2$,~$n_0$ spełniają założenia definicji notacji $\Theta$, skąd wnioskujemy, że $f(n)+o(f(n))=\Theta(f(n))$.

\bigskip
\note{Od tego momentu, w~rozwiązaniach zadań nie będziemy odwoływać się do definicji notacji asymptotycznych w~obliczaniu oszacowań czasów działania algorytmów, ale korzystając z~wyżej udowodnionej tożsamości, będziemy opuszczać składniki niższego rzędu w~sumach, których oszacowania chcemy otrzymać.}

\problem{Wariacje na temat notacji $O$ i~$\Omega$} %3-5

\subproblem %3-5(a)
Niech $c>0$ będzie pewną stałą. Jeśli $f(n)\le cg(n)$ w~pewnym skończonym zbiorze, to można wybrać największe takie $n$, dla którego ta nierówność jest prawdziwa. Oznaczmy je przez $n_0$. Mamy zatem $0\le cg(n)\le f(n)$ dla $n>n_0$, a~więc dla nieskończenie wielu liczb naturalnych, zatem $f(n)=\overset{\infty}{\Omega}(g(n))$. Jeśli zaś w~tym skończonym zbiorze zachodzi $f(n)\ge cg(n)$, to analogicznie można dowieść, że $f(n)=O(g(n))$.

W ostatnim przypadku, zachodzi zarówno $f(n)\ge cg(n)$ jak i~$f(n)\le cg(n)$ dla nieskończonej liczby argumentów, a~więc zgodnie z~definicją $\overset{\infty}{\Omega}$ prawdziwe jest $f(n)=\overset{\infty}{\Omega}(g(n))$.

Nie jest natomiast prawdą podobne twierdzenie, gdyby zastosować notację $\Omega$ zamiast $\overset{\infty}{\Omega}$; jeśli np.\ $f(n)=n$ oraz $g(n)=n^{\sin n+1}$, to $f(n)=\overset{\infty}{\Omega}(g(n))$, ale $f(n)\ne\Omega(g(n))$ i~$f(n)\ne O(g(n))$.

\subproblem %3-5(b)
\textbf{Zalety:}
\begin{itemize}
	\item wiadomo, że jeśli czas działania algorytmu nie jest $O(f(n))$, to jest $\overset{\infty}{\Omega}(f(n))$ (z~poprzedniego punktu),
	\item nie istnieją funkcje, których nie da się porównać z~innymi przy pomocy notacji $O$ i~$\overset{\infty}{\Omega}$.
\end{itemize}
\textbf{Wady:}
\begin{itemize}
	\item dowód, że $f(n)=\overset{\infty}{\Omega}(g(n))$ jest nieco trudniejszy do przeprowadzenia niż w~przypadku $f(n)=\Omega(g(n))$,
	\item nowa notacja w~wielu przypadkach nie uwypukla faktu, że pewna funkcja jest ``asymptotycznie większa'' od innej, gdyż mimo że przynależność do klasy złożoności jest spełniona, to wartości tej funkcji mogą być w~pewnych przedziałach znacząco mniejsze od wartości innej, należącej do tej samej klasy.
\end{itemize}

\subproblem %3-5(c)
Definicja $O'$ dopuszcza badanie funkcji, które nie są asymptotycznie nieujemne, np.\ prawdziwe są następujące tożsamości:
\begin{align*}
	-n^2 &= O'(n^2), \\
	-\frac{n^3}{2}+\frac{n^2}{6}-3n &= O'(n^4).
\end{align*}

Z~równości $f(n)=\Theta(g(n))$ wynika, że $f(n)\ge0$, a~więc $f$ jest funkcją nieujemną i~twierdzenie stosuje się bez zmian, czyli implikacja w~lewą stronę zachodzi.

Zbadajmy teraz drugą implikację. Wprost z~definicji, jeśli $f(n)=O'(g(n))$, to istnieją takie stałe $c$,~$n_0>0$, że zachodzi
\[
	\begin{cases}
		0 \le f(n) \le cg(n), & \text{dla $f(n)\ge0$,} \\
		0 \le -f(n) \le cg(n), & \text{dla $f(n)<0$,}
	\end{cases}
\]
dla wszystkich $n\ge n_0$. Załóżmy, że $f(n_1)<0$ dla pewnego $n_1\ge n_0$, ponieważ w przeciwnym przypadku mamy do czynienia z~funkcją nieujemną, dla której jest $f(n)=O(g(n))$. Teraz jednak warunek $0\le c_1g(n)\le f(n)$ nie zajdzie dla żadnej stałej $c_1>0$, co jest konieczne do tego, by $f(n)=\Omega(g(n))$ oraz $f(n)=\Theta(g(n))$. To nie przeszkadza jednak, aby implikacja zachodziła, ponieważ w~takim przypadku z~fałszu wyniknie fałsz.

Widać, że zastosowanie notacji $O'$ w miejsce $O$, nie pozbawia prawdziwości twierdzenia~3.1.

\subproblem %3-5(d)
\[
	\begin{split}
		\overset{\sim}{\Omega}(g(n)) &\stackrel{\text{\scriptsize def}}{=} \bigl\{\,f(n):\text{istnieją dodatnie stałe $c$,~$k$,~$n_0$ takie, że} \\
		&\qquad 0 \le cg(n)\lg^kn \le f(n) \text{ dla wszystkich $n \ge n_0$}\,\bigr\} \\
		\overset{\sim}{\Theta}(g(n)) &\stackrel{\text{\scriptsize def}}{=} \bigl\{\,f(n):\text{istnieją dodatnie stałe $c_1$,~$c_2$,~$k_1$,~$k_2$,~$n_0$ takie, że} \\
		&\qquad 0 \le c_1g(n)\lg^{k_1}n \le f(n) \le c_2g(n)\lg^{k_2}n \text{ dla wszystkich $n \ge n_0$}\,\bigr\}
	\end{split}
\]

Dowód twierdzenia~3.1 dla notacji $\overset{\sim}{O}$, $\overset{\sim}{\Omega}$ i~$\overset{\sim}{\Theta}$ przebiega analogicznie jak dowód jego oryginalnego odpowiednika, przeprowadzony w~\zad{3.1-5}.

Z definicji notacji $\overset{\sim}{\Theta}$ mamy, że $f(n)=\overset{\sim}{\Theta}(g(n))$ wtedy i~tylko wtedy, gdy istnieją takie stałe $c_1$, $c_2$, $k_1$, $k_2$,~$n_0>0$, że
\[
	0 \le c_1g(n)\lg^{k_1}n \le f(n) \le c_2g(n)\lg^{k_2}n
\]
zachodzi dla wszystkich $n\ge n_0$. Powyższe wyrażenie można zapisać w~równoważnej postaci
\[
	\begin{cases}
		0 \le c_1g(n)\lg^{k_1}n \le f(n), \\
		0 \le f(n) \le c_2g(n)\lg^{k_2}n.
	\end{cases}
\]
Pierwsza nierówność w~tej koniunkcji stanowi definicję $f(n)=\overset{\sim}{\Omega}(g(n))$, a~druga -- $f(n)=\overset{\sim}{O}(g(n))$.

Podobnie jak w~oryginalnym dowodzie, przeprowadzenie rozumowania w~odwrotnej kolejności wykazuje prawdziwość odwrotnej implikacji.

\problem{Funkcje iterowane} %3-6
W~każdym punkcie za dziedzinę funkcji $f_c^*$ przyjęto dziedzinę $f$. By wyznaczyć oszacowanie $f_c^*(n)$, należy znaleźć najmniejsze $i\ge0$, dla którego zachodzi $f^{(i)}(n)\le c$.

\subproblem %3-6(a)
$\lg^{(i)}(n)\le2$, logarytmując obustronnie otrzymujemy $\lg^{(i+1)}(n)\le1$, skąd wynika, że $i=\lg^*n-1$, a~zatem:
\[
	f_c^*(n) =
	\begin{cases}
		0, & \text{dla $0<n<1$}, \\
		\lg^*n-1, & \text{dla $n\ge1$}.
	\end{cases}
\]
Oszacowanie dokładne: $f_c^*(n)=\Theta(\lg^*n)$.

\subproblem %3-6(b)
Ponieważ $(n-1)^{(i)}\equiv n-i$, więc otrzymujemy:
\[
	f_c^*(n) =
	\begin{cases}
		0, & \text{dla $n<1$}, \\
		\lceil n\rceil-1, & \text{dla $n\ge1$}.
	\end{cases}
\]
Oszacowanie dokładne: $f_c^*(n)=\Theta(n)$.

\subproblem %3-6(c)
$(n/2)^{(i)}\equiv n/2^i$, a~stąd:
\[
	f_c^*(n) =
	\begin{cases}
		0, & \text{dla $n<1$}, \\
		\lceil\lg n\rceil, & \text{dla $n\ge1$}.
	\end{cases}
\]
Oszacowanie dokładne: $f_c^*(n)=\Theta(\lg n)$.

\subproblem %3-6(d)
Korzystając z~poprzedniego punktu, ale zmieniając stałą $c$, mamy:
\[
	f_c^*(n) =
	\begin{cases}
		0, & \text{dla $n<2$}, \\
		\lceil\lg n\rceil-1, & \text{dla $n\ge2$}.
	\end{cases}
\]
Oszacowanie dokładne: $f_c^*(n)=\Theta(\lg n)$.

\subproblem %3-6(e)
Ponieważ $\sqrt{n}\equiv n^{1/2}$, mamy $\bigl(\!\sqrt{n}\bigr)^{(i)}\equiv n^{1/2^i}$\!. Rozwiązaniem nierówności $n\le2^{2^i}$ ze względu na $i$, jest $i\ge\lg\lg n$, dostajemy więc następujący wynik:
\[
	f_c^*(n) =
	\begin{cases}
		0, & \text{dla $0\le n\le2$}, \\
		\lceil\lg\lg n\rceil, & \text{dla $n>2$}.
	\end{cases}
\]
Oszacowanie dokładne: $f_c^*(n)=\Theta(\lg\lg n)$.

\subproblem %3-6(f)
Bieżący punkt różni się od poprzedniego jedynie stałą $c$, jednak ta drobna zmiana mocno wpływa na postać funkcji $f_c^*$:
\[
	f_c^*(n) =
	\begin{cases}
		0, & \text{dla $0\le n\le1$}, \\
		\infty, & \text{dla $n>1$}.
	\end{cases}
\]
Nie można podać oszacowania dokładnego, bo $f_c^*(n)=\omega(g(n))$ dla każdej funkcji $g(n)$.

\subproblem %3-6(g)
Postępując analogicznie jak w~punkcie~(e), mamy $\bigl(n^{1/3}\bigr)^{(i)}\equiv n^{1/3^i}$, a~stąd:
\[
	f_c^*(n) =
	\begin{cases}
		0, & \text{dla $n\le2$}, \\
		\lceil\log_3\lg n\rceil, & \text{dla $n>2$}.
	\end{cases}
\]
Oszacowanie dokładne: $f_c^*(n)=\Theta(\lg\lg n)$.

\subproblem %3-6(h)
Postać funkcji $f(n)$ jest zbyt skomplikowana, aby bezpośrednio badać jej wersje iterowane, a~tym bardziej $f_c^*(n)$. Dlatego znajdziemy jej oszacowanie górne $g(n)$ i~to ono będzie przedmiotem naszej analizy.

Zauważmy, że $n/\!\lg n\equiv n\log_n\!2$. Ponieważ dla $n\le2$, mamy $f_c^*(n)=0$, to rozważmy $n>2$. W~takim przypadku zachodzi $0<\log_n\!2<1$, a~więc możemy przyjąc oszacowanie górne $g(n)=n$. Niestety, jakikolwiek argument, jaki podamy funkcji $g$ nie jest zmniejszany w~kolejnych iteracjach, dlatego nigdy nie osiągniemy wartości~2.

Można znaleźć nieco lepsze oszacowanie $f(n)$ zauważając, że $f_c^*(n)=1$ dla $3\le n\le4$. Przyjmijmy zatem, że $n>4$. Zachodzi wtedy $\log_n\!2<1/2$, a~zatem nowym oszacowaniem $f(n)$ będzie $g(n)=n/2$. Kolejne iteracje mają postać $g^{(i)}(n)=n/2^i$. Pozostaje obliczyć najmniejsze $i$, dla którego $n/2^i\le2$. Po kilku przekształceniach otrzymujemy $i\ge\lg n-1$, a~zatem prawdą jest, że $f_c^*(n)=O(\lg n)$.

\endinput

\chapter{Rekurencje}

\subchapter{Metoda podstawiania}

\exercise %4.1-1
Niech $c>0$ będzie stałą. Przyjmujemy następujące założenie:
\[
	T(\lceil n/2\rceil) \le c\lg(\lceil n/2\rceil).
\]
Na jego podstawie oraz z~wzoru~(3.3) otrzymujemy
\begin{align*}
	T(n) &\le c\lg(\lceil n/2\rceil)+1 \\
	&< c\lg(n/2+1)+1 \\
	&\le c\lg(3n/4)+1 \\
	&= c\lg n+c\lg(3/4)+1 \\
	&\le c\lg n,
\end{align*}
przy czym ostatnia nierówność wymaga, aby $c\ge\log_{4/3}2$. Ponadto w~wyprowadzeniu założono, że $n\ge4$.

Niech $T(1)=1$, jednak nie można dobrać stałej $c$ tak, aby $T(1)\le c\lg1=0$. Dla $n>2$ rekurencja nie zależy bezpośrednio od $T(1)$, więc przyjmijmy $T(2)=2$ za podstawę indukcji, ponieważ dla dowolnego $c\ge2$ zachodzi $T(2)\le c\lg 2=c$. Mamy więc $T(n)=O(\lg n)$.

\exercise %4.1-2
Przyjmijmy założenie, że
\[
	T(\lfloor n/2\rfloor) \ge c\lfloor n/2\rfloor\lg(\lfloor n/2\rfloor)
\]
dla pewnej dodatniej stałej~$c$. Korzystając z~wzoru~(3.3), mamy:
\begin{align*}
	T(n) &\ge 2c\lfloor n/2\rfloor\lg(\lfloor n/2\rfloor)+n \\
	&> 2c(n/2-1)\lg(n/4)+n \\
	&= 2c((n/2)\lg n-\lg n-n+2)+n \\
	&= cn\lg n-2c\lg n-2cn+4c+n \\
	&> cn\lg n-2c(n+\lg n)+n \\
	&\ge cn\lg n,
\end{align*}
Ostatni krok uzasadniamy, rozwiązując nierówność $-2c(n+\lg n)+n\ge0$ ze względu na $c$:
\[
	c \le \frac{n}{2(n+\lg n)} \le \frac{n}{2n} = \frac{1}{2},
\]
a~zatem wybierając dowolne $0<c\le1/2$, spełniamy ostatni krok wyprowadzenia. Można przyjąć $T(1)=1$ za podstawę indukcji, bo $T(1)\ge c\lg1=0$. Wykazaliśmy, że $T(n)=\Omega(n\lg n)$, a~zatem na mocy tw.~3.1 mamy $T(n)=\Theta(n\lg n)$.

\exercise %4.1-3
Przyjmijmy założenie, że $T(\lfloor n/2\rfloor)\le c\lfloor n/2\rfloor^2$ dla pewnej stałej $c>0$, czyli chcemy udowodnić, że $T(n)=O(n^2)$. Mamy zatem
\[
	T(n) \le 2c\lfloor n/2\rfloor^2+n \le 2cn^2\!/4 + n = cn^2\!/2+n \le cn^2,
\]
co jest prawdą, jeśli przyjmiemy $c\ge2$. Widać teraz, że z~mocniejszym założeniem warunek brzegowy $T(1)=1$ może stanowić podstawę indukcji, bo $T(1)\le c\cdot1^2=c$.

\exercise %4.1-4
Rekurencję~(4.2) można przedstawić w~alternatywny sposób:
\[
	T(n) =
	\begin{cases}
		d_1, & \text{jeśli $n=1$}, \\
		T(\lceil n/2\rceil)+T(\lfloor n/2\rfloor)+d_2n, & \text{jeśli $n>1$},
	\end{cases}
\]
dla pewnych stałych $d_1$,~$d_2>0$. Dla innych stałych $c_1$,~$c_2>0$ przyjmujemy założenia
\[
	T(\lfloor n/2\rfloor) \ge c_1\lfloor n/2\rfloor\lg(\lfloor n/2\rfloor) \quad\text{oraz}\quad T(\lceil n/2\rceil) \le c_2\lceil n/2\rceil\lg(\lceil n/2\rceil).
\]
W~przypadku dolnego oszacowania mamy:
\begin{align*}
	T(n) &\ge 2T(\lfloor n/2\rfloor)+d_2n \\
	&\ge 2c_1\lfloor n/2\rfloor\lg(\lfloor n/2\rfloor)+d_2n \\
	&> 2c_1(n/2-1)\lg(n/4)+d_2n \\
	&= c_1n\lg n-2c_1n-2c_1\!\lg(n/4)+d_2n \\
	&\ge c_1n\lg n,
\end{align*}
ponieważ zachodzi $\lfloor n/2\rfloor>n/2-1$ i~$\lfloor n/2\rfloor\ge n/4$ oraz można tak dobrać stałą $c_1$, aby prawdziwa była ostatnia nierówność powyższego wyprowadzenia. Po podstawieniu $c_1=d_2/4$ sprowadza się ona do $n\ge\lg(n/4)$, co jest prawdą dla wszystkich $n$ dodatnich. Podstawę indukcji stanowi $T(1)=d_1$, gdyż $T(1)\ge d_2/4\cdot 1\cdot\lg1=0$, a~zatem $T(n)=\Omega(n\lg n)$.

Wykorzystując nierówność $\lceil n/2\rceil<n/2+1$ z~wzoru~(3.3), dowodzimy górnego oszacowania:
\begin{align*}
	T(n) &\le 2T(\lceil n/2\rceil)+d_2n \\
	&\le 2c_2\lceil n/2\rceil\lg(\lceil n/2\rceil)+d_2n \\
	&< 2c_2(n/2+1)\lg\frac{n\lceil n/2\rceil}{n}+d_2n \\
	&\le c_2n\lg n+c_2(n+2)\lg\frac{\lceil n/2\rceil}{n}+2c_2\lg n+d_2n \\
	&\le c_2n\lg n.
\end{align*}
W~ostatniej nierówności skorzystano z~tego, że dla $n\ge2$ wyrażenie $\lg\frac{\lceil n/2\rceil}{n}$ jest ujemne, a~zatem dobierając odpowiednio duże $c_2$, można uzasadnić nierówność, gdyż funkcja liniowa rośnie szybciej od logarytmicznej. Dokładniej, okazuje się, że przyjęcie $c_2\ge10d_2$ wystarcza, aby spełnić nierówność dla $n\ge4$.

Ponieważ dla $n\ge4$ rekurencja nie zależy bezpośrednio od $T(1)$, to za podstawę indukcji należy przyjąć $T(2)=2d_1+2d_2$ oraz $T(3)=3d_1+5d_2$. Można sprawdzić, że otrzymane oszacowanie dla nich zachodzi, o~ile $d_1$ jest dostatecznie małe. W~przeciwnym przypadku górne oszacowanie może nie być wystarczające, jednak zwiększenie $c_2$ pozwala na dowolne ograniczenie rekurencji od góry w~zależności od wartości stałych $d_1$ i~$d_2$. Niezależnie od ich doboru oszacowaniem górnym rekurencji $T(n)$ jest $O(n\lg n)$, co na mocy wcześniejszego wyniku dolnego oszacowania implikuje $T(n)=\Theta(n\lg n)$.

\exercise %4.1-5
Wykorzystując założenie
\[
	T(\lfloor n/2\rfloor+17) \le c(\lfloor n/2\rfloor+17)\lg(\lfloor n/2\rfloor+17)
\]
dla pewnej stałej $c>0$, otrzymujemy:
\begin{align*}
	T(n) &\le 2c(\lfloor n/2\rfloor+17)\lg(\lfloor n/2\rfloor+17)+n \\
	&\le 2c(n/2+17)\lg(n/2+17)+n \\
	&\le c(n+34)\lg(11n/20)+n \\
	&< cn\lg n+cn\lg(11/20)+34c\lg n+n \\
	&\le cn\lg n,
\end{align*}
co zachodzi, o~ile $n/2+17\le 11n/20$, skąd $n\ge340$ oraz
\[
	cn\lg(11/20)+34c\lg n+n \le 0.
\]
Badając ostatnią nierówność, można dojść do rezultatu, że przyjęcie $c=47$ wystarcza, aby spełnić nierówność dla wszystkich $n\ge n_0$, gdzie $n_0=340$.

Zauważmy, że stosowanie równania $T(n)=2T(\lfloor n/2\rfloor+17)+n$ dla $n\le35$ nie ma sensu, bo wtedy $T(n)$ nie zależy od niższych wyrazów. Przyjmijmy zatem, że $T(n)=1$ dla wszystkich $n\le35$ i~niech stanowi to przypadek brzegowy rekurencji. Za podstawę indukcji musimy jednak przyjąć wszystkie $T(i)$, gdzie $187\le i\le339$. Wykorzystując program komputerowy, można wykazać, że dla $c=47$ wartości te spełniają oszacowanie, a~zatem $T(n)=O(n\lg n)$.

Analiza rekurencji dla każdej innej stałej w~miejscu~17 przebiega analogicznie, zmianie ulegają natomiast wartości $n_0$ i~$c$, jednak w~każdym takim przypadku rekurencja jest klasy $O(n\lg n)$.

\exercise %4.1-6
Przyjmijmy, że $n=2^m$, skąd $m=\lg n$. Rekurencja przyjmuje teraz postać
\[
	T(2^m) = 2T(2^{m/2})+1.
\]
Z~kolei podstawiając $S(m)$ za $T(2^m)$, dostajemy nową rekurencję
\[
	S(m) = 2S(m/2)+1,
\]
dla której udowodnimy rozwiązanie $O(\lg m)$.

Ponieważ nie dbamy o~to, czy $m/2$ jest całkowite, to możemy sprowadzić rekurencję do postaci $S(m)=S(\lfloor m/2\rfloor)+S(\lceil m/2\rceil)+1$, o~której wiemy, że jej rozwiązanie wynosi $O(m)$.

Aby uzyskać oszacowanie dokładne pozostaje udowodnić, że $S(m)=\Omega(m)$. Przyjmujemy zatem założenie, że $S(m/2)\ge c(m/2)$ dla $c>0$. Na jego podstawie otrzymujemy:
\[
	S(m) \ge 2cm/2+1 = cm+1 > cm,
\]
co zachodzi dla dowolnej wartości $c$. Przyjęcie $S(1)=1$ na podstawę indukcji wystarcza, bo warunek $S(1)\ge c$ spełnia każde $c\le1$, a~zatem $S(m)=\Omega(m)$.

Stosując tw.~3.1, mamy $S(m)=\Theta(m)$, więc wracając do oryginalnej rekurencji i~starej zmiennej, dostajemy $T(n)=T(2^m)=S(m)=\Theta(m)=\Theta(\lg n)$.

\subchapter{Metoda drzewa rekursji}

\exercise %4.2-1
Dokonajmy pewnego uproszczenia, opuszczając podłogę w~rekurencji $T(n)$ i~rozważając zależność $T'(n)=3T'(n/2)+n$.
\begin{figure}[ht]
	\begin{center}
		\includegraphics{fig04.1}
	\end{center}
	\caption{Drzewo rekursji $T'(n)=3T'(n/2)+n$} \label{fig:4.2-1}
\end{figure}
W~drzewie z~rys.~\ref{fig:4.2-1} jest $\lg n+1$ poziomów, a~na \twoparts{$i$}{tym} z~nich znajduje się $3^i$ węzłów, zatem jest $n^{\lg3}$ liści. Koszt węzła na poziomie $i$ wynosi $n/2^i$, skąd wynika, że łączny koszt wszystkich węzłów na \twoparts{$i$}{tej} głębokości jest równy $(3/2)^in$. Ponieważ liście wnoszą stały koszt, czyli $T'(1)=\Theta(1)$, to ostatni poziom ma wartość $\Theta(n^{\lg3})$. Na podstawie tych wartości rozwiązujemy rekurencję:
\begin{align*}
	T'(n) &= n+\frac{3}{2}n+\left(\frac{3}{2}\right)^2n+\cdots+\left(\frac{3}{2}\right)^{\lg n-1}n+\Theta(n^{\lg 3}) \\
	&= \sum_{i=0}^{\lg n-1}\left(\frac{3}{2}\right)^in+\Theta(n^{\lg 3}) \\
	&= \frac{\left(\frac{3}{2}\right)^{\lg n}-1}{\frac{3}{2}-1}\cdot n+\Theta(n^{\lg 3}) \\[1mm]
	&= 2n(n^{\lg 3-1}-1)+\Theta(n^{\lg 3}) \\
	&= O(n^{\lg 3}).
\end{align*}

Wykorzystując otrzymany wynik, przyjmujemy założenie, że $T(\lfloor n/2\rfloor)\le c(\lfloor n/2\rfloor)^{\lg 3}$ dla pewnej stałej $c>0$ i~dowodzimy oszacowania dla oryginalnej rekurencji metodą przez podstawianie:
\begin{align*}
	T(n) &\le 3c(\lfloor n/2\rfloor)^{\lg 3}+n \\
	&\le 3c(n/2)^{\lg 3}+n \\
	&= cn^{\lg 3}+n.
\end{align*}
Nie możemy jednak na podstawie tego wyniku wywnioskować szukanego oszacowania. Wzmocnijmy zatem nasze założenie, niech
\[
	T(\lfloor n/2\rfloor) \le c(\lfloor n/2\rfloor)^{\lg 3}-b\lfloor n/2\rfloor,
\]
dla nowej stałej $b\ge0$. Korzystając z~wzoru~(3.3), mamy teraz
\begin{align*}
	T(n) &\le 3c(\lfloor n/2\rfloor)^{\lg 3}-3b\lfloor n/2\rfloor+n \\
	&< 3c(n/2)^{\lg 3}-3b(n/2-1)+n \\
	&= cn^{\lg 3}-3bn/2+3b+n \\
	&\le cn^{\lg 3}-bn,
\end{align*}
co zachodzi dla $b\ge14$ oraz $n\ge7$. Przyjmujemy $T(1)=1$ na warunek brzegowy rekurencji oraz wszystkie $T(i)$, gdzie $1\le i\le6$, na podstawę indukcji, ponieważ z~braku dodatkowych ograniczeń na $c$ można dobrać dla niego odpowiednią wartość tak, aby udowodnione oszacowania zachodziły dla $T(i)$. To kończy dowód, a~więc górnym oszacowaniem rekurencji $T(n)$ jest $O(n^{\lg3})$.

\exercise %4.2-2
Drzewo rekursji $T(n)$ nie jest pełnym drzewem binarnym. Najkrótszą ścieżką od korzenia do liścia jest $n\to n/3\to n/9\to\dots\to n/3^i\to\dots\to1$. Liść tej gałęzi znajduje się na poziomie $i=\log_3n$. Ponieważ pełne drzewo binarne o~wysokości $\log_3n+1$ wnosi koszt nie większy niż drzewo rekurencji $T(n)$, to zachodzi $T(n)\ge cn\log_3n$ dla pewnego $c>0$, a~stąd $T(n)=\Omega(n\lg n)$.

\exercise %4.2-3
Dla uproszczenia pomijamy branie części całkowitych.
\begin{figure}[ht]
	\begin{center}
		\includegraphics{fig04.2}
	\end{center}
	\caption{Drzewo rekursji $T(n)=4T(n/2)+cn$} \label{fig:4.2-3}
\end{figure}
W~drzewie rekursji z~rys.~\ref{fig:4.2-3} na \twoparts{$i$}{tym} poziomie jest $4^i$ węzłów, z~których każdy wnosi koszt równy $cn/2^i$. Stąd kosztem całego poziomu jest $2^icn$. Współczynnik przy $n$ w~koszcie węzła maleje dwukrotnie wraz ze wzrostem poziomu, więc wysokością drzewa jest $\lg n$. Wnioskujemy zatem, że liczbą liści w~tym drzewie jest $4^{\lg n}=n^2$ i~że koszt ostatniego poziomu wynosi $\Theta(n^2)$. Sumując koszty z~każdego poziomu, otrzymujemy:
\begin{align*}
	T(n) &= cn+2cn+2^2cn+\cdots+2^{\lg n-1}cn+\Theta(n^2) \\
	&= \sum_{i=0}^{\lg n-1}2^icn+\Theta(n^2) \\
	&= (2^{\lg n}-1)cn+\Theta(n^2) \\
	&= (n-1)cn+\Theta(n^2) \\
	&= \Theta(n^2).
\end{align*}

Sprawdzamy otrzymany wynik, wykorzystując do tego celu metodę podstawiania. Badamy najpierw oszacowanie dolne $T(n)$, przyjmując założenie
\[
	T(n/2) \ge c_1(n/2)^2,
\]
dla pewnej stałej $c_1>0$. Stąd
\[
	T(n) \ge c_1n^2+cn \ge c_1n^2,
\]
bo $c>0$. Podstawą indukcji jest $T(1)=1$, które spełnia oszacowanie dla $c_1\le1$, a~więc prawdą jest, że $T(n)=\Omega(n^2)$.

By udowodnić dodatkowo, że $T(n)=O(n^2)$, możemy przyjąć takie samo założenie indukcyjne ale z~przeciwnym znakiem nierówności, jednak nie uzyskamy szukanego ograniczenia górnego na $T(n)$. Przyjmijmy zatem, że dla stałych $c_2>0$ oraz $c_3\ge0$ zachodzi mocniejszy warunek
\[
	T(n/2) \le c_2(n/2)^2-c_3(n/2).
\]
Dowodzimy:
\[
	T(n) \le 4(c_2n^2\!/4-c_3n/2)+cn = c_2n^2-2c_3n+cn \le c_2n^2-c_3n,
\]
co jest prawdą, jeśli przyjmiemy $c_3\ge c$. Aby móc potraktować $T(1)=1$ jako podstawę indukcji, musimy nałożyć na stałe dodatkowe ograniczenie, $c_3\le c_2-1$.

Dzięki metodzie podstawiania oraz z~tw.~3.1 wykazaliśmy, że $T(n)=\Theta(n^2)$.
\medskip %korekcja underfull vboxa

\exercise %4.2-4
Dla uproszczenia przyjmijmy, że $T(n)=cn$ w~przypadku, gdy $n\le a$, czyli dla dostatecznie małego $n$ rekurencja przyjmuje wartość stałą.
\begin{figure}[ht]
	\begin{center}
		\includegraphics{fig04.3}
	\end{center}
	\caption{Drzewo rekursji $T(n)=T(n-a)+T(a)+cn$} \label{fig:4.2-4}
\end{figure}
Wysokość drzewa rekursji $T(n)$ z~rys.~\ref{fig:4.2-4} wynosi $\lfloor n/a\rfloor$. Koszt wnoszony przez \twoparts{$i$}{ty} poziom (oprócz zerowego i~ostatniego) wynosi $c(n-a(i-1))$. Na ostatnim poziomie jest tylko jeden liść, który kosztuje $\Theta(1)$. Mamy:
\begin{align*}
	T(n) &= cn+\sum_{i=1}^{\lfloor n/a\rfloor-1}c(n-a(i-1))+\Theta(1) \\
	&= cn+c\sum_{i=0}^{\lfloor n/a\rfloor-2}(n-ai)+\Theta(1) \\
	&= cn+cn\sum_{i=0}^{\lfloor n/a\rfloor-2}1-ca\sum_{i=0}^{\lfloor n/a\rfloor-2}i+\Theta(1) \\
	&= cn+cn(\lfloor n/a\rfloor-1)-ca\frac{(\lfloor n/a\rfloor-2)(\lfloor n/a\rfloor-1)}{2}+\Theta(1) \\[2mm]
	&= \Theta(n^2).
\end{align*}

\exercise %4.2-5
Zauważmy, że drzewo rekurencji $T(n)$ na rys.~\ref{fig:4.2-5} dla parametru $\alpha$ jest symetryczne do drzewa $T(n)$ przy parametrze $1-\alpha$, przyjmijmy więc, że $0<\alpha\le1/2$.
\begin{figure}[ht]
	\begin{center}
		\includegraphics{fig04.4}
	\end{center}
	\caption{Drzewo rekursji $T(n)=T(\alpha n)+T((1-\alpha)n)+cn$, gdzie $0<\alpha\le1/2$} \label{fig:4.2-5}
\end{figure}

Wyznaczmy najpierw oszacowanie dolne rekurencji. Na \twoparts{$i$}{tym} poziomie drzewa najmniejszy koszt wnoszą węzły o~wartościach $c\alpha^in$, a~więc elementy ze skrajnie lewej gałęzi. Sprawdzając kiedy osiągną one wartość stałą $d>0$, wyznaczamy najgłębszy poziom o~komplecie węzłów. Niższe poziomy są coraz mniej liczne, zatem sumując koszt węzłów drzewa $T(n)$ od korzenia aż do tego poziomu, uzyskujemy oszacowanie dolne rekurencji. Na pewnej głębokości $h$ będzie $c\alpha^hn=d$, skąd otrzymujemy $h=\log_{1/\alpha}(cn/d)$. Ponieważ $\alpha$ jest stałe, to $h=\Theta(\lg n)$. Każdy poziom o~głębokości nieprzekraczającej $h$ wnosi koszt $cn$, więc oszacowaniem dolnym rekurencji jest $T(n)=\Omega(cn(h+1))=\Omega(n\lg n)$.

Badając teraz skrajnie prawą gałąź, której elementy wnoszą największy koszt wśród wszystkich poziomów, możemy dojść do oszacowania górnego dla $T(n)$. Na głębokości $H$ równej wysokości drzewa mamy $c(1-\alpha)^Hn=d$, skąd $H=\log_{1/(1-\alpha)}(cn/d)$. Także w~tym przypadku mamy $H=\Theta(\lg n)$, a~więc $T(n)=O(cn(H+1))=O(n\lg n)$, skąd asymptotycznie dokładnym rozwiązaniem rekurencji jest $\Theta(n\lg n)$.

\subchapter{Metoda rekurencji uniwersalnej}

\note{Zarówno w~wersji oryginalnej jak i~w~tłumaczeniu treści twierdzenia~4.1 znajduje się poważny brak, który uniemożliwia m.in.\ rozwiązanie \zad{4.4-3}. Dla stosowanej tam stałej\/ $c$ podany jest tylko warunek, aby była ona mniejsza od\/ $1$, podczas gdy poprawnym zakresem dla niej powinien być zbiór\/ $(0,1)$.}

\exercise %4.3-1

\subexercise
W~równaniu~(4.5) przyjmujemy $a=4$ i~$b=2$ oraz $f(n)=n$. Ponieważ $f(n)=O(n^{2-\epsilon})$ dla $0<\epsilon\le1$, to z~tw.~o~rekurencji uniwersalnej mamy $T(n)=\Theta(n^2)$.

\subexercise
Postępując analogicznie jak w~poprzednim punkcie, mamy te same wartości $a$ i~$b$, ale teraz $f(n)=n^2$. Z~tego, że $f(n)=\Theta(n^2)$ dostajemy $T(n)=\Theta(n^2\lg n)$.

\subexercise
Dla tych samych $a$ i~$b$ ale $f(n)=n^3$, mamy $f(n)=\Omega(n^{2+\epsilon})$ dla $0<\epsilon\le1$ oraz
\[
	4f(n/2) = 4n^3\!/8 = n^3\!/2 = f(n)/2 \le cf(n),
\]
o~ile $c\ge1/2$, a~zatem warunek regularności jest spełniony i~$T(n)=\Theta(n^3)$.

\exercise %4.3-2
Rozwiążmy $T(n)$, korzystając z~metody rekurencji uniwersalnej. Ponieważ $n^{\log_ba}=n^{\lg7}$ oraz $f(n)=n^2=O(n^{\lg7-\epsilon})$ dla $0<\epsilon\le\lg7-2$, to zachodzi $T(n)=\Theta(n^{\lg7})$.

Pozostaje teraz zbadanie nierówności $T'(n)<T(n)$ w~zależności od parametru $a$, bo w~rekurencji $T'(n)$ jest $n^{\log_ba}=n^{\log_4a}$ oraz $f(n)=n^2$. Załóżmy, że $f(n)=O(n^{\log_4a-\epsilon})$, co jest prawdą dla $a>16$ i~wtedy $T'(n)=\Theta(n^{\log_4a})$. Algorytm $A'$ jest efektywniejszy od algorytmu $A$, gdy $\log_4a<\lg7$, skąd $16<a<49$. Pozostałe przypadki tw.~4.1 można stosować, o~ile $a\le16$, ale wtedy $A'$ jest wolniejszy od $A$, zatem pomińmy ich sprawdzanie.

Największym całkowitym $a$, dla którego algorytm $A'$ jest bardziej efektywny od algorytmu $A$, jest~$a=48$.

\exercise %4.3-3
Ponieważ $a=1$ oraz $b=2$, to $n^{\log_ba}=n^0=1$ jest funkcją stałą. W~rekurencji tej $f(n)$ także jest stałe, a~więc $f(n)=\Theta(n^{\log_ba})$ i~$T(n)=\Theta(n^{\log_ba}\lg n)=\Theta(\lg n)$.

\exercise %4.3-4
Dla rekurencji $T(n)$ mamy $a=4$, $b=2$ oraz $f(n)=n^2\lg n$, a~więc $n^{\log_ba}=n^2$, ale nie istnieje takie $\epsilon>0$, że $f(n)=\Omega(n^{2+\epsilon})$. Nie można zatem zastosować w~rozwiązaniu twierdzenia o~rekurencji uniwersalnej, zatem znajdziemy asymptotyczne górne oszacowanie na $T(n)$, zgadując rozwiązanie, a~następnie dowodząc jego poprawności metodą podstawiania.

W~pierwszym wywołaniu rekurencja wnosi koszt równy $n^2\lg n$. Następnie 4~razy wywołujemy $T(n/2)$, co daje koszt
\[
	4T(n/2) = 4(n/2)^2\lg(n/2) = n^2\lg n-n^2.
\]
Kolejne poziomy wywołań kosztują
\begin{gather*}
	16T(n/4) = 16(n/4)^2\lg(n/4) = n^2\lg n-2n^2, \qquad\phantom{\text{itd.}} \\
	64T(n/8) = 64(n/8)^2\lg(n/8) = n^2\lg n-3n^2, \qquad\text{itd.}
\end{gather*}
Wnioskujemy z~otrzymanych wyników, że \twoparts{$i$}{ty} poziom wprowadza koszt równy $n^2\lg n-in^2$. Liście o~koszcie stałym znajdują się na poziomie $\lg n$ i~jest ich $4^{\lg n}=n^2$, więc dostajemy:
\begin{align*}
	T(n) &= \sum_{i=0}^{\lg n-1}(n^2\lg n-in^2)+\Theta(n^2) \\
	&= n^2\lg^2n-n^2\sum_{i=0}^{\lg n-1}i+\Theta(n^2) \\[1mm]
	&= n^2\lg^2n-\frac{n^2\lg n(\lg n-1)}{2}+\Theta(n^2) \\[1mm]
	&= O(n^2\lg^2n).
\end{align*}

Udowodnimy teraz metodą podstawiania, że otrzymane przypuszczenie jest istotnie oszacowaniem górnym dla $T(n)$. Przyjmijmy założenie
\[
	T(n/2) \le c(n/2)^2\lg^2(n/2),
\]
dla pewnej stałej $c>0$. Mamy teraz:
\begin{align*}
	T(n) &\le 4c(n/2)^2\lg^2(n/2)+n^2\lg n \\
	&= cn^2(\lg n-1)^2+n^2\lg n \\
	&= cn^2\lg^2n-2cn^2\lg n+cn^2+n^2\lg n \\
	&\le cn^2\lg^2n.
\end{align*}
Ostatnią nierówność dla wszystkich $n\ge2$ spełniamy, dobierając $c\ge1$. Przyjmujemy $T(1)=1$ za warunek brzegowy rekurencji, zaś $T(2)=8$ oraz $T(3)=4+9\lg3$ jako podstawę indukcji, ponieważ dla $n>3$ rekurencja nie zależy już bezpośrednio od $T(1)$ oraz otrzymane oszacowanie dla $T(2)$ i~$T(3)$ jest spełnione. Rozwiązaniem rekurencji $T(n)$ jest zatem $O(n^2\lg^2n)$.

\exercise %4.3-5
Przyjmując $a=1$, $b=2$ oraz $f(n)=n(2-\cos n)$, dostajemy, że $f(n)=\Omega(n^\epsilon)$ dla pewnego $0<\epsilon\le1$. Jednak dla $n=2\pi$ mamy
\[
	af(n/b) = 3\pi \quad\text{oraz}\quad f(n) = 2\pi,
\]
a~więc nierówność $af(n/b)\le cf(n)$ z~warunku regularności nie zachodzi dla żadnego $0<c<1$.

\subchapter{Dowód twierdzenia o~rekurencji uniwersalnej}

\exercise %4.4-1
Wykażemy metodą indukcji, że $n_j=\left\lceil n/b^j\right\rceil$. Z~definicji~(4.12) bezpośrednio wynika prawdziwość tego wzoru dla $j=0$. Przyjmijmy zatem, że dla $j>0$ zachodzi $n_{j-1}=\left\lceil n/b^{j-1}\right\rceil$. Wykorzystując tożsamość~(3.4), otrzymujemy
\[
	n_j = \lceil n_{j-1}/b\rceil = \left\lceil\left\lceil n/b^{j-1}\right\rceil\!/b\right\rceil = \left\lceil n/b^j\right\rceil,
\]
co należało pokazać.

\exercise %4.4-2
Zastąpmy drugi przypadek tw.~4.1 ogólniejszym warunkiem, tzn.\ jeśli $f(n)=\Theta(n^{\log_ba}\lg^kn)$ dla $k\ge0$, to zachodzi $T(n)=\Theta(n^{\log_ba}\lg^{k+1}n)$. Dla tak zmodyfikowanego twierdzenia należy przeprowadzić dowód analogicznie zmodyfikowanych lematów~4.3 i~4.4 (oznaczonych poniżej przez 4.3$'$ oraz 4.4$'$).

\begin{proof}[Dowód lematu~4.3\/$'$]
	Przy założeniu, że $f(n)=\Theta(n^{\log_ba}\lg^kn)$, otrzymujemy
	\[
		f(n/b^j)=\Theta\bigl((n/b^j)^{\log_ba}\lg^k(n/b^j)\bigr)
	\]
	i podstawiamy do wzoru~(4.7):
	\[
		g(n) = \Theta\biggl(\sum_{j=0}^{\log_bn-1}a^j\left(\frac{n}{b^j}\right)^{\log_ba}\lg^k\frac{n}{b^j}\biggr).
	\]
	Mamy dalej:
	\begin{align*}
		\sum_{j=0}^{\log_bn-1}a^j\left(\frac{n}{b^j}\right)^{\log_ba}\lg^k\frac{n}{b^j} &= n^{\log_ba}\sum_{j=0}^{\log_bn-1}\left(\frac{a}{b^{\log_ba}}\right)^j\lg^k\frac{n}{b^j} \\
		&= n^{\log_ba}\sum_{j=0}^{\log_bn-1}(\lg n-j\lg b)^k \\
		&= n^{\log_ba}\cdot\Theta(\lg^{k+1}n) \\
		&= \Theta(n^{\log_ba}\lg^{k+1}n).
	\end{align*}
	Skorzystano z~wzoru~(3.2), podstawiając $\lg n$ w~miejsce $n$, a~następnie zauważając, że
	\[
		\sum_{j=0}^{\log_bn-1}\Theta(\lg^kn) = \log_bn\cdot\Theta(\lg^kn) = \Theta(\lg^{k+1}n).
	\]
	Pokazano, że $g(n)=\Theta(n^{\log_ba}\lg^{k+1}n)$, a~więc lemat jest prawdziwy.
\end{proof}

\begin{proof}[Dowód lematu~4.4\/$'$]
	Wystarczy wykazać jedynie drugi przypadek, bo $f(n)=\Theta(n^{\log_ba}\lg^kn)$. Z~lematu~4.2:
	\[
		T(n) = \Theta(n^{\log_ba})+\Theta(n^{\log_ba}\lg^{k+1}n) = \Theta(n^{\log_ba}\lg^{k+1}n),
	\]
	co kończy dowód i~jednocześnie pokazuje prawdziwość głównego twierdzenia.
\end{proof}

\exercise %4.4-3
Warunek $af(n/b)\le cf(n)$ implikuje
\[
	f(n) \ge (a/c)f(n/b),
\]
a~zatem iterując powyższe, dostajemy
\[
	f(n) \ge (a/c)f(n/b) \ge (a/c)^2f\bigl(n/b^2\bigr) \ge \dots \ge (a/c)^if\bigl(n/b^i\bigr).
\]
Niech $i=\lceil\log_bn\rceil$, co daje
\[
	f(n) \ge (a/c)^{\lceil\log_bn\rceil}f\bigl(n/b^{\lceil\log_bn\rceil}\bigr).
\]
Na mocy nierówności~(3.3) zachodzi $\log_bn\le\lceil\log_b n\rceil<\log_bn+1$. Zauważmy, że przy zadanych ograniczeniach na $a$ i~$c$, $f$ nie może być funkcją nierosnącą, a~więc
\[
	f\bigl(n/b^{\lceil\log_bn\rceil}\bigr) > f(n/b^{\log_bn+1}) = f(1/b).
\]
Mamy następnie
\[
	f(n) > \frac{a^{\log_bn}}{c^{\log_bn+1}}f(n/b^{\log_bn+1}) = \frac{n^{\log_ba}}{cn^{\log_bc}}f(1/b) = \frac{f(1/b)}{c}\cdot\frac{n^{\log_ba}}{n^{\log_bc}}.
\]
Ponieważ $0<c<1$ oraz $b>1$, to $\log_bc<0$, przyjmijmy więc $\epsilon=-\log_bc$, skąd
\[
	f(n) > \frac{f(1/b)}{c}\cdot\frac{n^{\log_ba}}{n^{-\epsilon}} = \frac{f(1/b)}{c}\cdot n^{\log_ba+\epsilon},
\]
a~zatem $f(n)=\Omega(n^{\log_ba+\epsilon})$, co należało wykazać.

\problems

\problem{Przykłady rekurencji} %4-1
W~punktach~(a)\nobreakdash--(f) skorzystano z~twierdzenia o~rekurencji uniwersalnej.

\subproblem %4-1(a)
\[
	n^{\log_ba} = n^{\log_22} = n \quad\text{oraz}\quad f(n) = n^3 = \Omega(n^{1+\epsilon}) \quad\text{dla $0<\epsilon\le2$}
\]
Ponieważ warunek regularności jest spełniony:
\begin{align*}
	2f(n/2) &\le cf(n) \\
	2n^3\!/8 &\le cn^3 \\
	c &\ge 1/4,
\end{align*}
to stąd wnioskujemy, że $T(n)=\Theta(n^3)$.

\subproblem %4-1(b)
\[
	n^{\log_ba} = n^{\log_{10/9}1} = 1 \quad\text{oraz}\quad f(n) = n = \Omega(n^\epsilon) \quad\text{dla $0<\epsilon\le1$}
\]
Badamy warunek regularności:
\begin{align*}
	f(9n/10) &\le cf(n) \\
	9n/10 &\le cn \\
	c &\ge 9/10
\end{align*}
i~stwierdzamy, że $T(n)=\Theta(n)$.

\subproblem %4-1(c)
\[
	n^{\log_ba} = n^{\log_416} = n^2 \quad\text{oraz}\quad f(n) = n^2 = \Theta(n^2),
\]
a~stąd $T(n)=\Theta(n^2\lg n)$.

\subproblem %4-1(d)
\[
	n^{\log_ba} = n^{\log_37} \quad\text{oraz}\quad f(n) = n^2 = \Omega(n^{\log_37+\epsilon}) \quad\text{dla $0<\epsilon\le2-\log_37$}
\]
Warunek regularności zachodzi:
\begin{align*}
	7f(n/3) &\le cf(n) \\
	7n^2\!/9 &\le cn^2 \\
	c &\ge 7/9,
\end{align*}
a~zatem $T(n)=\Theta(n^2)$.

\subproblem %4-1(e)
\[
	n^{\log_ba} = n^{\lg7} \quad\text{oraz}\quad f(n) = n^2 = O(n^{\lg7-\epsilon}) \quad\text{dla $0<\epsilon\le\lg7-2$},
\]
a~stąd $T(n)=\Theta(n^{\lg7})$.

\subproblem %4-1(f)
\[
	n^{\log_ba} = n^{\log_42} = n^{1/2} \quad\text{oraz}\quad f(n) = \sqrt{n} = \Theta(n^{1/2}),
\]
a~stąd $T(n)=\Theta\bigl(\!\sqrt{n}\lg n\bigr)$.

\subproblem %4-1(g)
Zauważmy, że rekurencja rozwija się następująco:
\begin{align*}
	T(n) &= T(n-1)+n \\
	&= T(n-2)+(n-1)+n \\
	&\hspace{.5in}\vdots \\
	&= c+3+4+\cdots+n \\
	&= c+\sum_{i=3}^ni = c+\frac{n(n+1)}{2}-3,
\end{align*}
gdyż $T(2)$ jest pewną stałą $c$, a~stąd otrzymujemy, że $T(n)=\Theta(n^2)$.

\subproblem %4-1(h)
Niech $n=2^m$, skąd $m=\lg n$. Rekurencja przyjmuje teraz postać
\[
	T(2^m) = T(2^{m/2})+1.
\]
Podstawiając $S(m)$ za $T(2^m)$, otrzymujemy
\[
	S(m) = S(m/2)+1.
\]
Ponieważ rozwiązaniem ostatniej rekurencji jest $\Theta(\lg m)$ (co wykazano w~\zad{4.3-3}), to stąd mamy, że $T(n)=T(2^m)=S(m)=\Theta(\lg m)=\Theta(\lg\lg n)$.

\problem{Szukanie brakującej liczby całkowitej} %4-2
Liczby z~zakresu $0\twodots n$ reprezentowane są binarnie za pomocą $\lfloor\lg n\rfloor+1$ bitów. Można dla wygody przyjąć, że $n$ jest potęgą~2 pomniejszoną o~1. W~przeciwnym przypadku wystarczy zwiększyć $n$, aby było o~1 mniejsze od najbliższej potęgi~2 większej od $n$. Jednocześnie rozszerzamy tablicę $A$, umieszczając w~niej nowe liczby naturalne aż do nowej wartości $n$. Ta modyfikacja sprawi, że wszystkie liczby w~$A$ będą reprezentowane tą samą ilością bitów, a~rozmiar problemu urośnie co najwyżej dwukrotnie.

Badając najmniej znaczące bity liczb z~tablicy $A$, sprawdzamy ich parzystość. W~zakresie $0\twodots n$ jest $n/2$ liczb parzystych i~tyle samo nieparzystych. Jeśli wśród pobranych bitów jest więcej jedynek, to brakuje liczby parzystej, a~jeśli więcej zer, to brakująca liczba jest nieparzysta. W~zależności od przypadku odrzucamy $n/2$ liczb o~parzystości różnej od brakującej. Wsród pozostałych badamy następnie drugi najmniej znaczący bit i~analogicznie postępując, wyznaczamy $n/4$ kolejnych liczb do wyeliminowania. Po wykonaniu opisanych czynności aż do odrzucenia wszystkich liczb z~tablicy poznamy tę, której brakuje.

Czas działania zaprezentowanego tutaj algorytmu można zapisać w~postaci rekurencji $T(n)=T(n/2)+n$, której rozwiązaniem jest $O(n)$, co można łatwo pokazać przy użyciu twierdzenia~4.1. Widać teraz, że modyfikacja oryginalnego problemu (z~dowolnym $n$) wprowadziła tylko stały czynnik do czasu działania algorytmu.

\problem{Koszty przekazywania parametrów} %4-3

\subproblem %4-3(a)
Dla pierwszej strategii rekurencja przyjmuje postać z~\zad{2.3-5}, której rozwiązaniem jest $T(n)=\Theta(\lg n)$. Po przyjęciu $n=N$ dostajemy $T(N)=\Theta(\lg N)$.

W~przypadku drugiej strategii na każdym poziomie rekursji należy dodać składnik $\Theta(N)$ odpowiedzialny za przekazywanie tablicy do wywołań rekurencyjnych. Otrzymujemy zatem
\[
	T(n) = \begin{cases}
		\Theta(1), & \text{jeśli $n\le1$}, \\
		T(\lfloor n/2\rfloor)+\Theta(N), & \text{jeśli $n>1$}.
	\end{cases}
\]
Ponieważ $\Theta(N)$ jest stałe ze względu na rozmiar podproblemu $n$, to stąd rozwiązaniem powyższej rekurencji jest $T(n)=\Theta(N\lg n)$, a~więc $T(N)=\Theta(N\lg N)$.

W~ostatnim przypadku przekazujemy podtablicę o~rozmiarze równym rozmiarowi podproblemu, co prowadzi do następującej rekurencji:
\[
	T(n) = \begin{cases}
		\Theta(1), & \text{jeśli $n\le1$}, \\
		T(\lfloor n/2\rfloor)+\Theta(\lfloor n/2\rfloor), & \text{jeśli $n>1$},
	\end{cases}
\]
którą można rozwiązać przy użyciu twierdzenia o~rekurencji uniwersalnej po zignorowaniu podłóg. Otrzymujemy wynik $T(n)=\Theta(n)$, a~stąd $T(N)=\Theta(N)$.

\subproblem %4-3(b)
W~przypadku zwykłego przekazywania wskaźnika mamy oryginalną postać rekurencji, której rozwiązaniem dla $n=N$ jest $T(N)=\Theta(N\lg N)$.

Rekurencja dla drugiej strategii przedstawia się następująco:
\[
	T(n) = \begin{cases}
		\Theta(1), & \text{jeśli $n=1$}, \\
		2T(\lfloor n/2\rfloor)+\Theta(n)+2\Theta(N), & \text{jeśli $n>1$},
	\end{cases}
\]
ponieważ należy przekazać całą tablicę do obu wywołań rekurencyjnych. Na każdym poziomie dodajemy składnik rzędu $\Theta(N)$, zatem rozwiązaniem tej rekurencji jest iloczyn tego składnika i~rozwiązania rekurencji z~pierwszego przypadku, czyli $T(n)=\Theta(Nn\lg n)$, a~stąd mamy $T(N)=\Theta(N^2\lg N)$.

Ostatni przypadek wprowadza narzut w~postaci przekazywania podtablicy o~rozmiarze podproblemu do każdego wywołania rekurencyjnego, jednak czas ten jest pochłaniany przez składnik liniowy odpowiadający za czas przeznaczony na procedurę \proc{Merge}, przez co rozwiązaniem jest identyczny wynik jak w~pierwszej strategii, $T(N)=\Theta(N\lg N)$.

\problem{Więcej przykładów rekurencji} %4-4

\subproblem %4-4(a)
Wykorzystując twierdzenie o~rekurencji uniwersalnej, mamy
\[
	n^{\log_ba} = n^{\lg3} \quad\text{oraz}\quad f(n) = n\lg n = O(n^{\lg3-\epsilon}), \quad\text{dla $0<\epsilon<\lg3-1$,}
\]
a~stąd $T(n)=\Theta(n^{\lg3})$.

\subproblem %4-4(b)
Z~twierdzenia o~rekurencji uniwersalnej obliczamy
\[
	n^{\log_ba} = n^{\log_55} = n,
\]
jednak dla żadnego $\epsilon>0$ nie jest prawdą, że
\[
	f(n) = \frac{n}{\lg n} = O(n^{1-\epsilon}),
\]
ponieważ dla pewnej stałej $c>0$ i~dowolnie dużych $n$ musiałoby zachodzić
\[
	\frac{n^\epsilon}{\lg n} \le c,
\]
a~ponieważ $n^\epsilon=\omega(\lg n)$, to niezależnie od wyboru $\epsilon$ dla dużych wartości $n$ licznik będzie dowolnie większy od mianownika i~ułamka nie da się z~tego powodu ograniczyć stałą.

Skorzystajmy zatem z~innego sposobu na obliczenie $T(n)$, rozważając rodzinę rekurencji postaci
\[
	T_a(n) = \begin{cases}
		\Theta(1), & \text{jeśli $1\le n<a$}, \\
		aT_a(n/a)+n/\!\lg n, & \text{jeśli $n\ge a$},
	\end{cases}
\]
dla pewnego całkowitego $a\ge2$. Skorzystamy teraz z~techniki zamiany zmiennych, podstawiając $m=\log_an$, skąd $n=a^m$ i~otrzymując
\[
	T_a(a^m) = aT_a(a^{m-1})+\frac{a^m}{m\lg a}.
\]
Możemy teraz podstawić $S_a(m)=T_a(a^m)$, otrzymując nową rekurencję
\[
	S_a(m) = aS_a(m-1)+\frac{a^m}{m\lg a},
\]
którą rozwiązujemy następująco (po przyjęciu $S_a(0)=T_a(1)=d$ dla pewnej stałej $d>0$):
\begin{align*}
	S_a(m) &= \frac{1}{\lg a}\left(\frac{a^m}{m}+a\cdot\frac{a^{m-1}}{m-1}+a^2\cdot\frac{a^{m-2}}{m-2}+\cdots+a^{m-1}\cdot\frac{a^1}{1}+a^md\right) \\[1mm]
	&= \frac{1}{\lg a}\biggl(\sum_{k=1}^m\frac{a^m}{k}+a^md\biggr) \\[1mm]
	&= \frac{a^m(H_m+d)}{\lg a} \\[1mm]
	&= \Theta(a^m\lg m).
\end{align*}
Zamieniając z~powrotem $S_a(m)$ na $T_a(n)$, otrzymujemy rozwiązanie
\[
	T_a(n) = T_a(a^m) = S_a(m) = \Theta(a^m\lg m) = \Theta(n\lg\log_a n) = \Theta(n\lg\lg n).
\]

Stosując znalezione oszacowanie do rekurencji $T(n)\equiv T_5(n)$ z~treści zadania, dostajemy oczywiście $T(n)=\Theta(n\lg\lg n)$.

\subproblem %4-4(c)
Z~twierdzenia o~rekurencji uniwersalnej,
\[
	n^{\log_ba} = n^{\lg4} = n^2 \quad\text{oraz}\quad f(n) = n^{5/2} = \Omega(n^{2+\epsilon}), \quad\text{dla $0<\epsilon\le1/2$,}
\]
Badamy warunek regularności:
\begin{align*}
	4f(n/2) &\le cf(n) \\
	\frac{4n^{5/2}}{2^{5/2}} &\le cn^{5/2} \\
	c &\ge \frac{1}{\sqrt{2}}
\end{align*}
i~stwierdzamy, że $T(n)=\Theta(n^{5/2})$.

\subproblem %4-4(d)
Wykażemy, że stała~5 w~argumencie $T$ nie wpływa na postać rozwiązania -- rozważając rekurencję $T'(n)=3T'(n/3)+n/2$ i~rozwiązując ją za pomocą twierdzenia o~rekurencji uniwersalnej, dostajemy $T'(n)=\Theta(n\lg n)$. Udowodnimy teraz metodą podstawiania, że identyczny wynik jest rozwiązaniem rekurencji $T(n)$.

Wykorzystując założenie
\[
	T(n/3+5) \le c_1(n/3+5)\lg(n/3+5)
\]
dla pewnej stałej $c_1>0$, otrzymujemy:
\begin{align*}
	T(n) &\le 3c_1(n/3+5)\lg(n/3+5)+n/2 \\
	&\le c_1(n+15)\lg(2n/5)+n/2 \\
	&< c_1n\lg n+c_1n\lg(2/5)+15c_1\!\lg n+n/2 \\
	&\le c_1n\lg n
\end{align*}
co zachodzi, o~ile $n/3+5\le2n/5$, skąd $n\ge75$ oraz
\[
	c_1n\lg(2/5)+15c_1\!\lg n+n/2 \le 0.
\]
Przeprowadzając podobną analizę jak w~\zad{4.1-5} dostajemy, że wybór dowolnego $c_1\ge7$ spełnia nierówność dla $n\ge75$.

Analogicznie dla dolnego oszacowania przyjmując, że
\[
	T(n/3+5) \ge c_2(n/3+5)\lg(n/3+5),
\]
gdzie $c_2>0$ jest pewną stałą, dostajemy
\begin{align*}
	T(n) &\ge 3c_2(n/3+5)\lg(n/3+5)+n/2 \\
	&> 3c_2(n/3)\lg(n/3)+n/2 \\
	&= c_2n\lg n-c_2n\lg3+n/2 \\
	&\ge c_2n\lg n,
\end{align*}
gdzie ostatnia nierówność zachodzi, o~ile
\[
	-c_2n\lg3+n/2 \ge 0,
\]
co spełniamy poprzez przyjęcie $c_2\le0{,}3$.

Dowód asymptotycznie dokładnego oszacowania dla $T(n)$ kończymy, wybierając odpowiednie wartości dla podstaw obu indukcji. Zauważmy, że obliczanie wartości rekurencji z~wzoru $T(n)=3T(n/3+5)+n/2$ dla $n\le7$ nie ma sensu, bo wtedy $T(n)$ nie zależy od niższych wyrazów. Przyjmijmy zatem, że $T(n)=1$ dla wszystkich $n\le7$ będzie przypadkiem brzegowym rekurencji. Dowód górnego oszacowania zakładał $n\ge75$, więc podstawą obu indukcji uczyńmy $T(i)$, gdzie $30\le i\le 74$, od których bezpośrednio zależą wyrazy rekursji dla tego zakresu $n$. Oszacowania weryfikujemy programem komputerowym, uzasadniając poprawny dobór stałych $c_1$ i~$c_2$. A~zatem $T(n)=\Theta(n\lg n)$.

\subproblem %4-4(e)
Mamy do czynienia z~rekurencją $T_a(n)$ dla $a=2$, którą rozważaliśmy w~punkcie~(b). Zgodnie z~przedstawionym tam rozumowaniem wnioskujemy, że $T(n)=\Theta(n\lg\lg n)$.

\subproblem %4-4(f)
W~celu rozwiązania rekurencji wykorzystamy metodę drzewa rekursji do odgadnięcia rozwiązania, które następnie udowodnimy. Drzewo zostało przedstawione na rys.~\ref{fig:4-4f}.
\begin{figure}[ht]
	\begin{center}
		\includegraphics{fig04.5}
	\end{center}
	\caption{Drzewo rekursji $T(n)=T(n/2)+T(n/4)+T(n/8)+n$} \label{fig:4-4f}
\end{figure}

Zauważmy, że najszybciej maleją argumenty na skrajnie prawej gałęzi, więc na pewnym poziomie $h$ gałąź ta posiada liścia. Ponieważ koszt węzła z~\twoparts{$i$}{tego} poziomu tej gałęzi wynosi $n/8^i$ oraz $T(1)=\Theta(1)$, to $h=\Theta(\log_8n)$. Kosztem \twoparts{$i$}{tego} poziomu jest $(7/8)^in$, więc sumując je od korzenia aż do poziomu \twoparts{$h$}{tego}, otrzymujemy przybliżone oszacowanie rekurencji od dołu:
\[
	T(n) \ge \sum_{i=0}^{\log_8n}\left(\frac{7}{8}\right)^in = n\cdot\frac{1-(7/8)^{\log_8n+1}}{1-7/8} = 8n(1-7/8\cdot n^{\log_8(7/8)}) \ge 8n(1-7/8) = \Omega(n).
\]

Zbadajmy teraz górne oszacowanie rekurencji $T(n)$ poprzez dokonanie obserwacji, że najwolniej malejącymi węzłami drzewa są węzły ze skrajnie lewej gałęzi, które wnoszą koszt równy $n/2^i$ na \twoparts{$i$}{tym} poziomie, więc liść znajduje się na poziomie $H=\Theta(\lg n)$. Mamy
\[
	T(n) \le \sum_{i=0}^{\lg n}\left(\frac{7}{8}\right)^in < \sum_{i=0}^\infty\left(\frac{7}{8}\right)^in = \frac{n}{1-\frac{7}{8}} = O(n).
\]
co pozwala przypuszczać, że oszacowaniem dokładnym na $T(n)$ jest $\Theta(n)$.

Przeprowadźmy teraz dowód tego wyniku metodą przez podstawianie. Załóżmy, że
\begin{gather*}
	c_1(n/2) \le T(n/2) \le c_2(n/2), \\
	c_1(n/4) \le T(n/4) \le c_2(n/4), \\
	c_1(n/8) \le T(n/8) \le c_2(n/8),
\end{gather*}
gdzie $c_1$, $c_2>0$ to pewne stałe. Mamy zatem
\[
	T(n) \ge c_1n/2+c_1n/4+c_1n/8+n = 7c_1n/8+n \ge c_1n,
\]
przy czym $c_1\le8$. Górne oszacowanie dowodzi się analogicznie, zmieniając kierunek znaku nierówności i~zakładając, że $c_2\ge8$. Przyjęcie warunku brzegowego rekurencji $T(1)=1$ na przypadek bazowy indukcji wystarcza w~obu indukcjach. A~zatem udowodniliśmy, że dokładnym rozwiązaniem rekurencji jest $T(n)=\Theta(n)$.

\subproblem %4-4(g)
Rozwijając rekurencję przy założeniu, że $T(1)=1$, otrzymujemy
\[
	T(n) = \frac{1}{n}+\frac{1}{n-1}+\cdots+\frac{1}{2}+\frac{1}{1} = H_n,
\]
a~zatem $T(n)=\Theta(\lg n)$.

\subproblem %4-4(h)
Przy założeniu, że $T(1)=0$, zachodzi
\[
	T(n) = \lg n+\lg(n-1)+\cdots+\lg2+\lg 1 = \lg\biggl(\prod_{i=1}^ni\biggr) = \lg(n!)
\]
i~z~wzoru~(3.18) dostajemy $T(n)=\Theta(n\lg n)$.

\subproblem %4-4(i)
Przyjmijmy $T(2)=2$ i~rozważmy przypadek, gdy $n$ jest parzyste. Rozwijamy rekurencję, otrzymując
\[
	T(n) = 2\lg n+2\lg(n-2)+\cdots+2\lg4+2 = 2\lg\biggl(\prod_{i=1}^{n/2}2i\biggr) = 2\bigl(\lg((n/2)!)+n/2\bigr).
\]
Wykorzystując wzór~(3.18), dostajemy
\[
	T(n) = 2\Theta((n/2)\lg (n/2))+n = \Theta(n\lg n).
\]

Dla $n$ nieparzystego, po przyjęciu $T(1)=0$, sprowadzamy rekurencję do sumy
\[
	T(n) = 2\lg n+2\lg(n-2)+\cdots+2\lg3+0 = 2\lg\biggl(\prod_{i=1}^{(n-1)/2}(2i-1)\biggr),
\]
którą można ograniczyć od góry przez $2\lg\Bigl(\prod_{i=1}^{n/2}2i\Bigr)$, a~z~dołu przez $2\lg\Bigl(\prod_{i=1}^{(n-1)/2-1}2i\Bigr)$. Obie wartości można z~wzoru~(3.18) sprowadzić do postaci $\Theta(n\lg n)$, a~zatem również w~przypadku nieparzystego argumentu $T(n)$ jest klasy $\Theta(n\lg n)$.

\subproblem %4-4(j)
Po podzieleniu równania rekurencji $T(n)$ przez $n$ dostajemy
\[
	\frac{T(n)}{n} = \frac{T(\!\sqrt{n})}{\sqrt{n}}+1.
\]
Podstawmy teraz $S(n)=T(n)/n$, otrzymując nową rekurencję
\[
	S(n) = S(\!\sqrt{n})+1.
\]
Następnie potraktujmy $n$ jako $2^m$, skąd $m=\lg n$ i~podstawmy $R(m)=S(2^m)$:
\[
	R(m) = R(m/2)+1.
\]
Rozwiązanie ostatniej rekurencji zostało wyznaczone w~\zad{4.3-3} i~wynosi $R(m)=\Theta(\lg m)$. Powracamy do oryginalnej rekurencji $T(n)$, dostając ostatecznie
\[
	T(n) = nS(n) = nR(\lg n) = n\cdot\Theta(\lg\lg n) = \Theta(n\lg\lg n).
\]

\problem{Liczby Fibonacciego} %4-5

\subproblem %4-5(a)
Wprost z~definicji $\mathcal{F}(z)$ mamy:
\begin{align*}
	\mathcal{F}(z) &= \sum_{i=0}^\infty F_iz^i \\
	&= F_0+zF_1+\sum_{i=2}^\infty (F_{i-1}+F_{i-2})z^i \\
	&= z+\sum_{i=2}^\infty F_{i-1}z^i+\sum_{i=2}^\infty F_{i-2}z^i \\
	&= z+\sum_{i=1}^\infty F_iz^{i+1}+\sum_{i=0}^\infty F_iz^{i+2} \\[2mm]
	&= z+z\mathcal{F}(z)+z^2\mathcal{F}(z),
\end{align*}
a~zatem tożsamość zachodzi.

\subproblem %4-5(b)
Z~wzoru z~poprzedniego punktu wynika pierwsza równość:
\begin{align*}
	\mathcal{F}(z) &= z+z\mathcal{F}(z)+z^2\mathcal{F}(z) \\
	(1-z-z^2)\mathcal{F}(z) &= z \\
	\mathcal{F}(z) &= \frac{z}{1-z-z^2}. \tag{$*$}\label{eq:4-5b_1}
\end{align*}

Mianownik prawej strony~(\ref{eq:4-5b_1}) jest trójmianem kwadratowym, który można zapisać w~równoważnej postaci:
\[
	1-z-z^2 \equiv -(z+\phi)\bigl(z+\widehat\phi\bigr). \tag{$*$}\label{eq:4-5b_2}
\]
Trójmian~(\ref{eq:4-5b_2}) przyjmuje wartość~0 dla $z=-\phi$ lub $z=-\widehat\phi$. Ponieważ zachodzi ciekawa własność $\phi\cdot\widehat\phi=-1$, zatem można zapisać trójmian w~postaci $(1-\phi z)\bigl(1-\widehat\phi z\bigr)$, skąd wynika druga równość.

Ostatnią z~nich dowodzimy zauważając, że:
\[
	\frac{1}{1-\phi z}-\frac{1}{1-\widehat\phi z} = \frac{1-\widehat\phi z-1+\phi z}{(1-\phi z)\bigl(1-\widehat\phi z\bigr)} = \frac{z\bigl(\phi-\widehat\phi\bigr)}{(1-\phi z)\bigl(1-\widehat\phi z\bigr)} = \frac{z\sqrt{5}}{(1-\phi z)\bigl(1-\widehat\phi z\bigr)},
\]
a~stąd mamy
\[
	\frac{z}{(1-\phi z)\bigl(1-\widehat\phi z\bigr)} = \frac{1}{\sqrt{5}}\cdot\frac{z\sqrt{5}}{(1-\phi z)\bigl(1-\widehat\phi z\bigr)} = \frac{1}{\sqrt{5}}\left(\frac{1}{1-\phi z}-\frac{1}{1-\widehat\phi z}\right).
\]

\subproblem %4-5(c)
Tezę otrzymujemy natychmiast, jeśli w~definicji $\mathcal{F}(z)$ podstawimy $F_i=\bigl(\phi^i-\widehat\phi^i\bigr)/\sqrt{5}$, co wykazano w~\zad{3.2-6}.

\subproblem %4-5(d)
Ponieważ $\bigl|\widehat\phi\bigr|<1$, to prawdą jest, że $\bigl|\widehat\phi^i\bigr|<1$ dla $i>0$ oraz $\bigl|\widehat\phi^i\bigr|/\sqrt{5}<1/\sqrt{5}<1/2$. Mamy
\[
	F_i = \frac{\phi^i-\widehat\phi^i}{\sqrt{5}} = \frac{\phi^i}{\sqrt{5}}-\frac{\widehat\phi^i}{\sqrt{5}},
\]
skąd
\[
	\frac{\phi^i}{\sqrt{5}}-\frac{1}{2} < F_i < \frac{\phi^i}{\sqrt{5}}+\frac{1}{2},
\]
a~zatem $F_i$ jest równe liczbie całkowitej najbliższej $\phi^i/\sqrt{5}$.

\subproblem %4-5(e)
Tożsamość została udowodniona w~\zad{3.2-7}.

\problem{Testowanie układów VLSI} %4-6

\subproblem %4-6(a)
Niech $D$ i~$Z$ będą zbiorami, odpowiednio, układów dobrych i~układów złych. Podczas testowania każdy zły układ może twierdzić, że każdy inny układ ze zbioru $Z$ jest dobry, a~każdy układ ze zbioru $D$ jest zły. Z~kolei dobry układ będzie orzekał o~każdym układzie z~$Z$, że jest zły, natomiast o~każdym innym z~$D$, że jest dobry. Inaczej ujmując, zbiór $Z$ wskazuje, że sam jest zbiorem układów dobrych, a~zbiór $D$ złych i~symetrycznie dla zbioru $D$. W~ogólności, $Z$ może składać się z~podzbiorów, które o~sobie samych będą twierdzić, że składają się z~dobrych układów, a~pozostałe zbiory ze złych. Nie da się natomiast rozdzielić w~taki sposób zbioru $D$, ponieważ jego układy zawsze orzekają prawdziwie. Musi przez to być $|D|>|Z|$, wtedy bowiem niezależnie od utworzonego w~wyniku testowania podziału zbioru $Z$, zbiór $D$ będzie wyznaczony jednoznacznie.

\subproblem %4-6(b)
Potraktujmy wynik każdego testu dwóch układów jako parę $(r_1,r_2)\in\{{\scriptstyle\rm D},{\scriptstyle\rm Z}\}^2$. Pierwszy element pary jest stwierdzeniem pierwszego układu o~drugim, a~drugi element -- drugiego o~pierwszym, przy czym ${\scriptstyle\rm D}$ oznacza pozytywny wynik testu, a~${\scriptstyle\rm Z}$ -- negatywny. Możliwe jest uzyskanie jednego z~czterech wyników:
\begin{itemize}
	\item $({\scriptstyle\rm D},{\scriptstyle\rm D})$ -- implikuje, że oba układy są dobre albo oba są złe,
	\item $({\scriptstyle\rm D},{\scriptstyle\rm Z})$ -- zachodzi tylko wtedy, gdy pierwszy z~układów jest zły,
	\item $({\scriptstyle\rm Z},{\scriptstyle\rm D})$ -- zachodzi tylko wtedy, gdy drugi z~układów jest zły,
	\item $({\scriptstyle\rm Z},{\scriptstyle\rm Z})$ -- co najmniej jeden z~układów jest zły.
\end{itemize}

Podzielmy zbiór układów na pary i~przetestujmy wzajemnie układy w~każdej takiej parze, wykonując przy tym $\lfloor n/2\rfloor$ testów. Zauważmy, że otrzymując dla pewnej pary wynik różny niż $({\scriptstyle\rm D},{\scriptstyle\rm D})$, możemy ją odrzucić, gdyż co najmniej jeden układ ją tworzący jest zły, ale w~wśród nieodrzuconych układów nadal pozostanie więcej dobrych niż złych. Otrzymamy w~wyniku od jednej do $\lfloor n/2\rfloor$ par o~tej własności, że w~każdej z~nich oba układy są dobre albo oba są złe. Odrzucając zatem po jednym układzie z~każdej pozostawionej pary, nadal zachowujemy własność o~większej liczbie dobrych układów w~pozostawionym zbiorze układów, który ma rozmiar co najwyżej $\lfloor n/2\rfloor$.

Powyżej opisany proces przeprowadzamy rekurencyjnie na otrzymywanych podproblemach, dostając w~końcu zbiór jednoelementowy, który na mocy założenia zawiera dobry układ.

\subproblem %4-6(c)
Wykorzystując wynik poprzedniego punktu, dostajemy następującą rekurencję opisującą liczbę testów koniecznych do znalezienia jednego dobrego układu w~pesymistycznym przypadku (każda para zwraca wynik $({\scriptstyle\rm D},{\scriptstyle\rm D})$):
\[
	T(n) =
	\begin{cases}
		\Theta(1), & \text{jeśli $n=1$}, \\
		T(\lfloor n/2\rfloor) + \lfloor n/2\rfloor, & \text{jeśli $n>1$}.
	\end{cases}
\]
Ignorując podłogi i~stosując twierdzenie o~rekurencji uniwersalnej, dostajemy jej rozwiązanie: $T(n)=\Theta(n)$. Wynik ten jest prawdziwy także w~przypadku optymistycznym -- tylko jedna para zwraca $({\scriptstyle\rm D},{\scriptstyle\rm D})$, a~więc dobry układ znajdziemy w~jednym zejściu rekurencyjnym, jednak wcześniej należało wykonać $\Theta(n)$ testów.

Ponieważ w~poprzednim punkcie znaleźliśmy dobry układ $u$, to wykorzystajmy go do znalezienia kolejnych. Testujemy tenże układ z~pozostałymi $n-1$. Wynikiem testu $u$ z~pewnym innym układem $v$ nie może być $({\scriptstyle\rm D},{\scriptstyle\rm Z})$, a~więc możliwe są trzy sytuacje. Jeśli otrzymamy $({\scriptstyle\rm D},{\scriptstyle\rm D})$, to oznacza to, że oba układy są tak samo dobre, a~więc $v$ również jest dobry. Uzyskując wynik $({\scriptstyle\rm Z},{\scriptstyle\rm D})$, mamy natychmiast, że $v$ jest zły, podobnie w~wypadku, gdy wynikiem testu będzie $({\scriptstyle\rm Z},{\scriptstyle\rm Z})$ -- wtedy co najmniej jeden z~testowanych układów jest zły, ale nie może nim być $u$. Wynika stąd, że sprawdzając $u$ z~$n-1$ innymi układami, wykonamy $n-1$ testów, a~ponieważ stwierdziliśmy, że $u$ jest dobre za pomocą $\Theta(n)$ testów, to znalezienie wszystkich dobrych układów można wykonać, przeprowadzając również $\Theta(n)$ testów.

\problem{Tablice Monge'a} %4-7

\subproblem %4-7(a)
\noindent\emph{Dowód $\Rightarrow$.} Tablica Monge'a $A$ spełnia nierówność
\[
	A[i,j]+A[k,l] \le A[i,l]+A[k,j], \quad\text{dla $1\le i<k\le m$ oraz $1\le j<l\le n$}.
\]
W~szczególności zaś może być $k=i+1$ oraz $l=j+1$, zatem implikacja zachodzi.
\bigskip

\noindent\emph{Dowód $\Leftarrow$.} Dowodzimy przez indukcję względem liczby wierszy $m$. Zauważmy, że warunek tablicy Monge'a nie ma większego sensu dla tablic o~jednej kolumnie lub jednym wierszu, zatem przyjmijmy $m=2$ za podstawę indukcji. Wtedy $i=1$ oraz $k=2$ i~przyjmując $l=j+1$, mamy
\[
	A[1,j]+A[2,j+1] \le A[1,j+1]+A[2,j], \tag{$*$}\label{eq:4-7a}
\]
dla $1\le j<n$. Dodajmy teraz powyższą nierówność dla pewnego $j<n-1$ (przy założeniu, że $n>2$) do niej samej ale z~$l=j+1$ w~miejscu $j$. Po zredukowaniu zbędnych składników dostajemy
\[
	A[1,j]+A[2,j+2] \le A[1,j+2]+A[2,j].
\]
Do ostatniej nierówności można ponownie dodawać~(\ref{eq:4-7a}), podstawiając w~miejsce $j$ coraz większe $l<n$ i~otrzymując dowolną nierówność stanowiącą warunek tablicy Monge'a.

Niech teraz $m>2$. Dowód wykorzystuje ten sam pomysł z~pierwszego kroku indukcji z~tym, że teraz $i$ może przyjmować każdą dopuszczalną wartość. Przyjmujemy ponadto założenie indukcyjne, że tablica $A$ pozbawiona ostatniego wiersza stanowi tablicę Monge'a. Pozostaje zatem wykazać, że zachodzą wszystkie nierówności z~definicji tablicy Monge'a dla $k=m$. Mamy
\[
	A[i,j]+A[i+1,j+1] \le A[i,j+1]+A[i+1,j], \quad\text{dla $1\le i<m-1$ oraz $1\le j<n$}.
\]
Dodajemy tę nierówność do niej samej po uprzednim podstawieniu w~ostatniej $l=j+1<n$ w~miejsce $j$. Dzięki temu otrzymamy
\[
	A[i,j]+A[i+1,j+2] \le A[i,j+2]+A[i+1,j].
\]
Podobnie jak wcześniej możemy tak dodawać żądaną ilość razy, przyjmując coraz większe wartości $l<n$ zamiast $j$ i~dostając wszystkie wymagane nierówności.

Widać zatem, że implikacja jest prawdziwa dla tablic o~pewnej ustalonej liczbie kolumn. Dowód dla zmiennej liczby kolumn przeprowadza się analogicznie przez indukcję po $n\ge2$, pokazując tym samym, że twierdzenie zachodzi dla tablic o~dowolnych wymiarach.

\subproblem %4-7(b)
Korzystając z~poprzedniego punktu, można pokazać, że nierówność
\[
	A[1,2]+A[2,3] \le A[1,3]+A[2,2]
\]
jest fałszywa, co zaburza właściwość tablicy Monge'a. By przywrócić własność, można zamienić $A[1,3]$ na~24.

\subproblem %4-7(c)
Korzystając z~poniższej nierówności prawdziwej dla tablicy Monge'a $A$:
\[
	A[i,j]+A[i+1,j+r] \le A[i,j+r]+A[i+1,j], \tag{$*$}\label{eq:4-7c}
\]
dla $0<r\le n-j$, wnioskujemy w~następujący sposób. Znajdujemy w~pierwszym wierszu tablicy $A$ pierwsze minimum z~lewej strony, które oznaczymy przez $\mu$. Indeksem $\mu$ jest oczywiście $f(1)$. Stwierdzamy teraz, że dla $1\le j<f(1)$ spełnione jest~(\ref{eq:4-7c}), czyli
\[
	A[1,j]+A[2,f(1)] \le \mu+A[2,j].
\]
Z~drugiej strony $\mu<A[1,j]$ dla każdego $1\le j<f(1)$. Łącząc oba fakty, otrzymujemy $A[2,f(1)]<A[2,j]$, a~to oznacza, że pierwsze z~lewej strony minimum wiersza~2 występuje w~nim na indeksie nie mniejszym niż $f(1)$, skąd $f(1)\le f(2)$.

Dowód kolejnych nierówności przebiega analogicznie, skąd dostajemy tezę.

\subproblem %4-7(d)
Korzystając z~twierdzenia z~poprzedniego punktu, szukając minimum wiersza \twoparts{$i$}{tego}, będziemy sprawdzać indeksy minimów wierszy \twoparts{$(i-1)$}{szego} oraz \twoparts{$(i+1)$}{szego}. Szukane minimum znajduje się pomiędzy nimi. Oczywiście nie istnieje wiersz zerowy, dlatego przetwarzając pierwszy wiersz, nie szukamy minimum poprzedniego, ale przyjmujemy dla uproszczenia procedury, że $f(0)=1$. Podobny przypadek może się zdarzyć dla ostatniego nieparzystego wiersza, jeśli jest on ostatnim wierszem tablicy, wtedy wystarczy przyjąć wartość $f(m+1)=n$.

Po wyznaczeniu $f(i-1)$ oraz $f(i+1)$ sprawdzamy $f(i+1)-f(i-1)+1$ komórek wiersza $i$ w~poszukiwaniu jego minimum. W~ciągu całej procedury sprawdzimy w~pesymistycznym przypadku
\begin{align*}
	\sum_{\begin{subarray}{l}1\le i\le m\\2\,\nmid\,i\end{subarray}}\bigl(f(i+1)-f(i-1)+1\bigr) &= \sum_{k=0}^{\lceil m/2\rceil-1}\bigl(f(2k+2)-f(2k)+1\bigr) \\
	&= \lceil m/2\rceil+\sum_{k=0}^{\lceil m/2\rceil-1}\bigl(f(2k+2)-f(2k)) \\[1mm]
	&= \lceil m/2\rceil+f(2\lceil m/2\rceil)-f(0) \\[2mm]
	&= \lceil m/2\rceil+n-1
\end{align*}
komórek, która to liczba jest rzędu $O(m+n)$.

\subproblem %4-7(e)
Na podstawie oszacowania z~poprzedniego punktu oraz z~tego, że na ostatnim poziomie rekursji wyznaczenie minimum jednego wiersza tablicy wymaga sprawdzenia co najwyżej $O(n)$ komórek, formułujemy następującą rekurencję:
\[
	T(m,n) =
	\begin{cases}
		O(n), & \text{jeśli $m=1$}, \\
		T(\lfloor m/2\rfloor,n)+O(m+n), & \text{jeśli $m>1$}.
	\end{cases}
\]
Na \twoparts{$i$}{tym} poziomie rekurencja wprowadza koszt równy $O(m/2^i+n)$. Łatwo zauważyć, że jest $\lfloor\lg m\rfloor+1$ poziomów, a~zatem całkowity koszt wynosi
\begin{align*}
	T(n) &= \sum_{i=0}^{\lfloor\lg m\rfloor-1}O\biggl(\frac{m}{2^i}+n\biggr)+O(n) \\
	&= O\biggl(\sum_{i=0}^{\lfloor\lg m\rfloor-1}\frac{m}{2^i}\biggr)+O(n\lg m) \\
	&= O\biggl(m\sum_{i=0}^{\lfloor\lg m\rfloor-1}\frac{1}{2^i}\biggr)+O(n\lg m) \\
	&= O(m+n\lg m),
\end{align*}
ponieważ suma w~przedostatniej linijce jest ograniczona od góry przez stałą.

\endinput

\chapter{Analiza probabilistyczna i~algorytmy randomizowane}

\subchapter{Problem zatrudnienia sekretarki}

\exercise %5.1-1
Niech $\preceq$ będzie relacją określoną na zbiorze rang kandydatek, za pomocą której rozstrzygamy, która kandydatka z~dwóch testowanych jest lepsza. Na wejściu procedury \proc{Hire-Assistant} może pojawić się każda permutacja kandydatek, więc jesteśmy w~stanie rozstrzygać o~każdej parze kandydatek. Pozostaje zatem udowodnić, że $\preceq$ jest porządkiem częściowym.

Możemy bezpiecznie założyć, że relacja $\preceq$ jest zwrotna, jako że nie testujemy żadnej kandydatki z~nią samą. Jeśli $\preceq$ nie byłoby antysymetryczne, to w~zależności od kolejności pojawienia się na wejściu procedury pewnych dwóch kandydatek, za lepszą mogłaby zostać uznana którakolwiek z~tej pary, co przeczyłoby założeniu. Podobnie można wykazać, że $\preceq$ jest przechodnie, bowiem w~przeciwnym przypadku dla pewnych trzech kandydatek, o~tym, która z~nich jest najlepsza, decydowałaby ich permutacja wejściowa.

\exercise %5.1-2
Poniższy algorytm implementuje generator liczb losowych z~zakresu $a\twodots b$, korzystając jedynie z~pomocniczych wywołań $\proc{Random}(0,1)$.
\begin{codebox}
\Procname{$\proc{Random}(a,b)$}
\li	\While $a<b$
\li		\Do
			$\id{mid}\gets\lfloor(a+b)/2\rfloor$
\li			\If $\proc{Random}(0,1)=0$
\li				\Then $a\gets\id{mid}+\,1$
\li				\Else $b\gets\id{mid}$
				\End
		\End
\li	\Return $a$
\end{codebox}

Niech $n=b-a+1$ będzie długością zakresu generowania. W~każdym wywołaniu rekurencyjnym odrzucana jest połowa zakresu z~dokładnością do jednego elementu. Działanie procedury jest więc analogiczne do pesymistycznego przypadku wyszukiwania binarnego w~\onedash{$n$}{elementowej} tablicy, przez co średnio działa ona w~czasie opisanym przez rekurencję z~\refExercise{4.3-3}. Rozwiązaniem tej rekursji jest $T(n)=\Theta(\lg n)$, a~zatem oczekiwany czas działania procedury $\proc{Random}(a,b)$ w~zależności od $a$ i~$b$ wynosi $\Theta(\lg(b-a))$.

\exercise %5.1-3
Zauważmy, że prawdopodobieństwo uzyskania najpierw orła, a~potem reszki w~dwóch rzutach monetą jest takie samo, jak uzyskanie najpierw reszki, a~potem orła i~wynosi $p(1-p)$. Będziemy zatem rzucać monetą po dwa razy, aż do uzyskania różnych wyników. Jako wynik procedury przyjmiemy wynik pierwszego rzutu w~ostatniej parze rzutów.

Następujący algorytm implementuje powyższy opis:
\begin{codebox}
\Procname{\proc{Unbiased-Random}}
\li	\Repeat
		$x\gets\proc{Biased-Random}$
\li		$y\gets\proc{Biased-Random}$
\li	\Until $x\ne y$ \label{li:unbiased-random-repeat-end}
\li	\Return $x$
\end{codebox}

Załóżmy, że każda iteracja pętli \kw{repeat} odbywa się w~czasie stałym. Kolejne iteracje tworzą ciąg prób Bernoulliego, w~których sukcesem jest warunek z~wiersza~\ref{li:unbiased-random-repeat-end}, zachodzący z~prawdopodobieństwem $2p(1-p)$. Oczekiwana liczba prób aż do osiągnięcia sukcesu jest zadana wzorem~(C.31) i~wynosi $1/(2p(1-p))$. Stąd wnioskujemy, że oczekiwanym czasem działania algorytmu jest $\Theta(1/(p(1-p)))$.

\subchapter{Zmienne losowe wskaźnikowe}

\exercise %5.2-1
Zatrudnienie tylko jednej kandydatki jest równoważne przyjęciu pierwszej z~nich i~tylko jej. Zauważmy, że pierwszą kandydatkę przyjmujemy w~procedurze \proc{Hire-Assistant} w~każdym przypadku. Jeśli ma ona być jedyną zatrudnioną osobą, to powinna być najbardziej wykwalifikowaną w~zbiorze wszystkich kandydatek (czyli mieć największą wartość \id{rank}). Najlepsza kandydatka może znajdować się na każdym z~$n$ miejsc w~ciągu wejściowym, zatem prawdopodobieństwo tego, że będzie zajmować pierwszą pozycję, jest równe $1/n$.

By dokonać zatrudnienia wszystkich $n$ kandydatek, musimy przesłuchiwać je w~kolejności rosnących rang. Jest tylko jedna taka permutacja wejściowa, zatem prawdopodobieństwo tego zdarzenia wynosi $1/n!$.

\exercise %5.2-2
Zauważmy, że zarówno kandydatka z~pierwszej pozycji w~ciągu wejściowym, jak również ta o~najwyższej randze, są zatrudniane w~każdym przypadku. Jeśli procedura \proc{Hire-Assistant} ma dokonać dokładnie dwóch zatrudnień, to kandydatka z~numerem~1 powinna mieć rangę $i\le n-1$, a~wszystkie kandydatki o~rangach $i+1$, $i+2$,~\dots,~$n-1$ powinny występować w~ciągu po kandydatce z~rangą równą $n$.

Oznaczmy przez $E_i$ zdarzenie, że pierwsza kandydatka ma rangę równą $i$. Zachodzi oczywiście $\Pr(E_i)=1/n$ dla każdego $i=1$, 2,~\dots,~$n$. Przyjmijmy, że $j$ jest pozycją najlepszej kandydatki w~ciągu i~niech $F$ będzie zdarzeniem polegającym na tym, że kandydatki o~numerach 2, 3,~\dots,~$j-1$ mają rangi mniejsze od rangi kandydatki numer~1. Jeśli zachodzi $E_i$, to $F$ zachodzi tylko wtedy, gdy $i\ne n$, a~spośród $n-i$ kandydatek, których rangi są większe niż $i$, ta z~rangą równą $n$ przesłuchiwana jest najwcześniej. Stąd mamy $\Pr(F\mid E_i)=1/(n-i)$, o~ile $i\ne n$. Niech w~końcu $A$ oznacza zdarzenie, że w~procedurze \proc{Hire-Assistant} zatrudniane są dokładnie dwie osoby. Ponieważ zdarzenia $E_1$, $E_2$,~\dots,~$E_n$ są rozłączne, to zachodzi
\[
	A = F\cap(E_1\cup E_2\cup\dots\cup E_{n-1}) = (F\cap E_1)\cup(F\cap E_2)\cup\dots\cup(F\cap E_{n-1})
\]
oraz
\[
	\Pr(A) = \sum_{i=1}^{n-1}\Pr(F\cap E_i).
\]
Z~tożsamości~(C.14),
\[
	\Pr(F\cap E_i) = \Pr(F\mid E_i)\Pr(E_i) = \frac{1}{n-i}\cdot\frac{1}{n},
\]
a~zatem
\[
	\Pr(A) = \sum_{i=1}^{n-1}\frac{1}{n-i}\cdot\frac{1}{n} = \frac{1}{n}\sum_{i=1}^{n-1}\frac{1}{n-i} = \frac{1}{n}\sum_{i=1}^{n-1}\frac{1}{i} = \frac{H_{n-1}}{n}.
\]

\exercise %5.2-3
Obliczmy wartość oczekiwaną liczby oczek w~jednym rzucie kostką. Definiując zmienną losową $X_i$ jako liczbę oczek na \onedash{$i$}{tej} kostce ($i=1$, 2,~\dots,~$n$), obliczamy $\E(X_i)$, przyjmując, że zmienne $X_i$ mają rozkład jednostajny (prawdopodobieństwo każdego wyniku jest równe $1/6$):
\[
	\E(X_i) = \sum_xx\Pr(X_i=x) = \frac{1+2+3+4+5+6}{6} = 3{,}5.
\]
Niech zmienna losowa $X$ oznacza sumę oczek na $n$ kostkach. Mamy $X=X_1+X_2+\dots+X_n$, więc z~liniowości wartości oczekiwanej
\[
	\E(X) = \E\biggl(\sum_{i=1}^nX_i\biggr) = \sum_{i=1}^n\E(X_i) = 3{,}5n.
\]

\exercise %5.2-4
Niech $S_i$, dla $i=1$, 2,~\dots,~$n$, będzie zdarzeniem oznaczającym, że \onedash{$i$}{ta} osoba otrzymała swój kapelusz. Definiujemy teraz zmienne losowe $X_i=\I(S_i)$ oraz $X=X_1+X_2+\dots+X_n$, przy czym $X$ oznacza liczbę osób, którym zwrócono właściwe kapelusze. Mamy
\[
	\E(X) = \E\biggl(\sum_{i=1}^nX_i\biggr) = \sum_{i=1}^n\E(X_i) = \sum_{i=1}^n\Pr(X_i=1) = \sum_{i=1}^n\frac{1}{n} = 1,
\]
a~zatem swój kapelusz otrzyma średnio tylko jedna osoba.

\exercise %5.2-5
Dla wszystkich całkowitych $i$,~$j$ takich, że $1\le i<j\le n$, zdefiniujmy zdarzenia $S_{ij}$ -- w~tablicy $A$ występuje inwersja $\langle i,j\rangle$. Szanse na to, aby elementy na pozycjach $i$ oraz $j$ tworzyły inwersję, są równe $1/2$. Definiujemy zmienne losowe $X_{ij}=\I(S_{ij})$ oraz $X=\sum_{i=1}^{n-1}\sum_{j=i+1}^nX_{ij}$, przy czym zmienna $X$ oznacza łączną liczbę inwersji tablicy $A$. Jej wartością oczekiwaną jest
\begin{align*}
	\E(X) &= \E\biggl(\sum_{i=1}^{n-1}\sum_{j=i+1}^nX_{ij}\biggr) = \sum_{i=1}^{n-1}\sum_{j=i+1}^n\E(X_{ij}) = \sum_{i=1}^{n-1}\sum_{j=i+1}^n\Pr(X_{ij}=1) \\[1mm]
	&= \sum_{i=1}^{n-1}\sum_{j=i+1}^n\frac{1}{2} = \frac{1}{2}\sum_{i=1}^{n-1}(n-i) = \frac{1}{2}\sum_{i=1}^{n-1}i = \frac{n(n-1)}{4}.
\end{align*}

\subchapter{Algorytmy randomizowane}

\exercise %5.3-1
Oto zmodyfikowana procedura \proc{Randomize-In-Place}:
\begin{codebox}
\Procname{$\proc{Randomize-In-Place}'(A)$}
\li	$n\gets\id{length}[A]$
\li	zamień $A[1]\leftrightarrow A[\proc{Random}(1,n)]$
\li	\For $i\gets2$ \To $n$
\li		\Do zamień $A[i]\leftrightarrow A[\proc{Random}(i,n)]$
		\End
\end{codebox}

Treść niezmiennika pozostaje taka sama (z~wyjątkiem fragmentu, który podaje linie kodu zawierające ciało pętli). Modyfikacji wymaga jedynie dowód jego pierwszej własności.
\begin{description}
	\item[Inicjowanie:] Gdy $i=2$, niezmiennik pętli mówi, że dla każdej \onedash{1}{permutacji} fragment tablicy $A[1\twodots1]$ zawiera tę permutację z~prawdopodobieństwem $(n-1)!/n!=1/n$. Podtablica $A[1\twodots1]$ stanowi tylko jeden element $A[1]$, który z~prawdopodobieństwem $1/n$ jest pewnym ustalonym elementem spośród $n$ elementów tablicy. A~więc niezmiennik jest spełniony przed pierwszą iteracją.
\end{description}

\exercise %5.3-2
\note{Przedstawiony w~treści zadania algorytm jest podany niepoprawnie, ponieważ wynik wywołania\/ $\proc{Random}(i+1,n)$ jest niezdefiniowany, gdy\/ $i$ przyjmuje wartość\/ $n$. Pętla \kw{for} w~tej procedurze powinna iterować po wszystkich\/ $i$ od\/ $1$ do\/ $n-1$.}

\noindent Algorytm ten nie działa zgodnie z~zamierzeniem. Jako przykład weźmy dowolną tablicę o~$n=3$ elementach. Istnieje $n!-1=5$ permutacji tej tablicy różnych niż identycznościowa. Pętla \kw{for} w~pierwszej iteracji zamienia pierwszy element tablicy z~losowo wybranym z~pozostałych dwóch. W~drugiej iteracji może zostać wybrana tylko jedna wartość na drugi element. Za pomocą tej procedury jesteśmy więc w~stanie utworzyć tylko dwie permutacje wejściowej tablicy.

\exercise %5.3-3
Zauważmy, że kolejne wywołania generatora liczb losowych w~procedurze \proc{Permute-With-All} generują jeden z~$n^n$ możliwych ciągów pozycji tablicy, podczas gdy istnieje $n!$ możliwych wyników procedury. Załóżmy, że $n>2$ i~że procedura generuje każdą permutację z~jednakowym prawdopodobieństwem. A~zatem każdej permutacji na wyjściu odpowiada stała liczba $c$ ciągów indeksów, czyli $n^n=cn!$. W~tym wzorze $n-1$ dzieli prawą stronę, a~więc powinno dzielić także lewą. Ale to nie jest prawdą, gdyż ze~wzoru~(A.5) dla $x=n$ mamy
\[
    \sum_{k=0}^{n-1}n^k = \frac{n^n-1}{n-1},
\]
skąd dostajemy
\[
    n^n = (n-1)\sum_{k=0}^{n-1}n^k+1,
\]
czyli $n^n$ daje resztę 1 przy dzieleniu przez $n-1$. Na podstawie otrzymanej sprzeczności wnioskujemy, że procedura \proc{Permute-With-All} nie generuje permutacji losowych zgodnie z~rozkładem jednostajnym.

\exercise %5.3-4
Na początku działania procedury losowana jest liczba \id{offset}, o~jaką zostaną przesunięte elementy tablicy $A$ cyklicznie w~prawo. Element z~pozycji $i$ znajdzie się w~wyniku tego przesunięcia na pozycji $\id{dest}=(i+\id{offset})\bmod n$ w~tablicy $B$. Ponieważ istnieje $n$ możliwych wartości zmiennej \id{offset}, to szanse, że element $A[i]$ znajdzie się na pewnej ustalonej pozycji w~$B$, są równe $1/n$.

Ponieważ nie jest zmieniana wzajemna kolejność elementów, to nie każdą permutację można otrzymać w~wyniku działania tej procedury -- na przykład nie dostaniemy nigdy permutacji będącej odwróceniem tablicy wejściowej, o~ile jej rozmiar jest większy niż~2.

\exercise %5.3-5
Spróbujmy skonstruować tablicę $P$, w~której wszystkie elementy są różne. Na pierwszy element tej tablicy możemy wybrać jedną z~$n^3$ liczb, drugi element może przyjąć jedną z~$n^3-1$ pozostałych wartości, trzeci -- jedną z~$n^3-2$ pozostałych itd. Ogólnie, \onedash{$i$}{ty} z~kolei element tablicy $P$ może być jedną z~$n^3-i+1$ liczb pozostałych po poprzednich wyborach. A~zatem prawdopodobieństwo tego, że wszystkie elementy tablicy $P$ są różne, wynosi
\[
	\prod_{i=1}^n\frac{n^3-i+1}{n^3} = \prod_{i=0}^{n-1}\frac{n^3-i}{n^3} = \prod_{i=0}^{n-1}\biggl(1-\frac{i}{n^3}\biggr) > \prod_{i=0}^{n-1}\biggl(1-\frac{n}{n^3}\biggr) = \biggl(1-\frac{1}{n^2}\biggr)^n.
\]
Wykorzystując teraz fakt, że ciąg $e_n={(1-1/n)}^n$ jest rosnący, otrzymujemy, że
\[
	\biggl(1-\frac{1}{n^2}\biggr)^{n^2} \ge \biggl(1-\frac{1}{n}\biggr)^n
\]
i~po zastosowaniu pierwiastka \onedash{$n$}{tego} stopnia do obu stron nierówności otrzymujemy żądany wynik.

\exercise %5.3-6
Gdy dwa priorytety powtarzają się, czyli $P[i]=P[j]$ dla pewnych $i\ne j$, to deterministyczny algorytm sortujący szereguje odpowiadające im elementy $A[i]$ oraz $A[j]$ zawsze w~tej samej kolejności. W~skrajnym przypadku, jeśli wszystkie priorytety w~$P$ są identyczne, to generowana będzie tylko jedna permutacja tablicy $A$.

Rozwiązaniem problemu powtarzających się priorytetów jest użycie randomizowanego algorytmu sortującego wykorzystującego porównania. Za każdym razem, gdy porównywane elementy $x$ i~$y$ okażą się równe, algorytm ten będzie losowo -- z~równym prawdopodobieństwem -- wybierał, czy potraktować relację między nimi jako $x<y$, czy jako $x>y$. Dzięki temu każda wejściowa tablica priorytetów z~punktu widzenia algorytmu sortowania będzie zawierała liczby parami różne, których każda permutacja może pojawić się na wejściu z~jednakowym prawdopodobieństwem.

\subchapter{Analiza probabilistyczna i~dalsze zastosowania zmiennych losowych wskaźnikowych}

\exercise %5.4-1
Podobnie jak w~analizie paradoksu dnia urodzin przyjmiemy, że $n$ jest liczbą dni w~roku i~ponumerujemy osoby znajdujące się w~pokoju liczbami całkowitymi 1, 2,~\dots,~$k$. Dla $i=1$, 2,~\dots,~$k$ niech $A_i$ będzie zdarzeniem polegającym na tym, że osoba $i$ ma urodziny kiedy indziej niż ja. Wówczas
\[
	B_k = \bigcap_{i=1}^kA_i
\]
jest zdarzeniem, że żadna z~$k$ osób nie ma urodzin wtedy co ja. Dla każdego $i=1$, 2,~\dots,~$k$ zachodzi $\Pr(A_i)=1-1/n$. Przy założeniu, że zdarzenia $A_1$, $A_2$,~\dots,~$A_k$ są wzajemnie niezależne, mamy
\[
	\Pr(B_k) = \Pr\biggl(\bigcap_{i=1}^kA_i\biggr) = \prod_{i=1}^k\Pr(A_i) = \prod_{i=1}^k\biggl(1-\frac{1}{n}\biggr) = \biggl(1-\frac{1}{n}\biggr)^k.
\]
Prawdopodobieństwo zdarzenia, że wśród $k$ osób jest przynajmniej jedna, która ma urodziny tego samego dnia co ja, ma być mniejsze niż $1/2$, czyli
\[
	\biggl(1-\frac{1}{n}\biggr)^k \le \frac{1}{2}.
\]
Rozwiązując tę nierówność ze względu na $k$, otrzymujemy $k\ge\log_{1-1/n}(1/2)$ i~po przyjęciu $n=365$ mamy, że najmniejszym całkowitym $k$ spełniającym tę nierówność jest $k=253$.

W~rozwiązaniu drugiej części zadania pozostaniemy przy poprzednim znaczeniu symbolu $n$ i~numeracji osób kolejnymi liczbami całkowitymi. Ponadto przez $r$ oznaczymy dzień 3~maja. Niech teraz $B_k$ będzie zdarzeniem polegającym na tym, że wśród $k$ osób co najwyżej jedna ma urodziny w~dniu $r$ oraz $A_i$, dla $i=1$, 2,~\dots,~$k$, niech będzie zdarzeniem, że osoba $i$ ma urodziny w~inny dzień niż $r$. Podobnie jak w~pierwszej części zadania zachodzi $\Pr(A_1)=\Pr(A_2)=\dots=\Pr(A_k)=1-1/n$. Dla $i=1$, 2,~\dots,~$k$ zdefiniujmy jeszcze
\[
	C_i = \overline{A_i}\cap\bigcap_{\substack{j=1\\j\ne i}}^kA_j
\]
jako zdarzenie polegające na tym, że wśród $k$ osób tylko \onedash{$i$}{ta} obchodzi urodziny dnia $r$. Zachodzi
\[
	B_k = \bigcap_{i=1}^kA_i\cup\bigcup_{i=1}^kC_i.
\]
Obliczmy $\Pr(C_i)$, zakładając, że zdarzenia $A_1$, $A_2$,~\dots,~$A_k$ są wzajemnie niezależne:
\[
	\Pr(C_i) = (1-\Pr(A_i))\prod_{\substack{j=1\\j\ne i}}^k\Pr(A_j) = \frac{1}{n}\prod_{\substack{j=1\\j\ne i}}^k\biggl(1-\frac{1}{n}\biggr) = \frac{1}{n}\biggl(1-\frac{1}{n}\biggr)^{k-1}.
\]
Zdarzenia $\bigcap_{i=1}^kA_i$, $C_1$, $C_2$,~\dots,~$C_k$ wzajemnie się wykluczają, a~więc dostajemy
\begin{align*}
	\Pr(B_k) &= \Pr\biggl(\bigcap_{i=1}^kA_i\biggr)+\Pr\biggl(\bigcup_{i=1}^kC_i\biggr) \\
	&= \prod_{i=1}^k\Pr(A_i)+\sum_{i=1}^k\Pr(C_i) \\
	&= \biggl(1-\frac{1}{n}\biggr)^k+\frac{k}{n}\biggl(1-\frac{1}{n}\biggr)^{k-1} \\
	&= \biggl(1+\frac{k-1}{n}\biggr)\biggl(1-\frac{1}{n}\biggr)^{k-1}.
\end{align*}
To czego poszukujemy, to najmniejsze $k$ takie, że $\Pr(B_k)<1/2$. Można w~tym momencie przyjąć $n=365$ i~obliczać prawdopodobieństwa $B_k$ dla kolejnych naturalnych wartości $k$. W~wyniku takich obliczeń można ustalić, że szukaną wartością jest $k=613$.

\exercise %5.4-2
Niech $X$ oznacza liczbę potrzebnych rzutów, zanim w~pewnej urnie znajdą się dwie kule. Załóżmy, że po $k$ rzutach ($k=1$, 2,~\dots,~$b$) nie było urny z~więcej niż jedną kulą i~obliczmy szanse, że również po \onedash{$(k+1)$}{szym} rzucie nie będzie kolizji. Ponieważ jest zajętych $k$ urn, to \onedash{$(k+1)$}{sza} kula wpada do pustej urny z~prawdopodobieństwem równym
\[
	\Pr(X>k+1\mid X>k) = \frac{b-k}{b}.
\]
Zachodzi
\[
	\Pr(X>k+1) = \prod_{i=1}^k\Pr(X>i+1\mid X>i) = \frac{(b-1)(b-2)\dots(b-k)}{b^k}.
\]
Oczywiście $\Pr(X=1)=0$ i~$\Pr(X>1)=1$. Dla $k=1$, 2,~\dots,~$b$ mamy
\[
	\Pr(X=k+1) = \Pr(X>k)-\Pr(X>k-1) = \frac{(b-1)(b-2)\dots(b-k+1)k}{b^k} = \frac{b!\,k}{(b-k)!\,b^{k+1}}.
\]

Zajmijmy się teraz wartością oczekiwaną zmiennej losowej $X$:
\begin{align*}
	\E(X) &= \sum_{k=0}^b(k+1)\Pr(X=k+1) \\
	&= \frac{b!}{b^b}\sum_{k=0}^b\frac{b^{b-k}k}{(b-k)!\,b}(k+1) \\
	&= \frac{b!}{b^b}\sum_{k=0}^b\frac{b^k(b-k)}{k!\,b}(b-k+1) \\
	&= \frac{b!}{b^b}\biggl(\sum_{k=0}^b\frac{b^k}{k!}(b-k+1)-\sum_{k=0}^b\frac{b^kk}{k!\,b}(b-k+1)\biggr) \\
	&= \frac{b!}{b^b}\biggl(\sum_{k=0}^b\frac{b^k}{k!}(b-k+1)-\sum_{k=1}^b\frac{b^{k-1}}{(k-1)!}(b-k+1)\biggr) \\
	&= \frac{b!}{b^b}\biggl(\sum_{k=0}^b\frac{b^k}{k!}(b-k+1)-\sum_{k=0}^{b-1}\frac{b^k}{k!}(b-k)\biggr) \\
	&= \frac{b!}{b^b}\biggl(\sum_{k=0}^b\frac{b^k}{k!}(b-k+1)-\sum_{k=0}^b\frac{b^k}{k!}(b-k)\biggr) \\
	&= \frac{b!}{b^b}\sum_{k=0}^b\frac{b^k}{k!}.
\end{align*}
Ostatnie wyrażenie jest badane w~\cite{taocp1frag}, gdzie wyprowadzono jego oszacowanie $\sqrt{b\pi/2}+O(1)$, a~zatem $E(X)=\Theta(\!\sqrt{b})$.

\exercise %5.4-3
W~analizie paradoksu dnia urodzin niezależność urodzin wykorzystuje się jedynie we wzorze
\[
    \Pr(b_i=r\;\;\text{i}\;\;b_j=r) = \Pr(b_i=r)\Pr(b_j=r) = 1/n^2.
\]
Wystarczy zatem założenie, że zdarzenia te są parami niezależne.

\exercise %5.4-4
Niech $n$ będzie liczbą dni w~roku. Oznaczmy przez $P_1(k,n)$ prawdopodobieństwo tego, że wszystkie osoby z~\onedash{$k$}{osobowej} grupy mają urodziny w~różne dni, a~$P_2(k,n)$ niech będzie prawdopodobieństwem tego, że pewnego dnia w~roku urodziły się dokładnie dwie osoby z~tej grupy. Szanse na to, aby wśród tych $k$ osób co najmniej troje miało urodziny tego samego dnia, są równe
\[
	P(k,n) = 1-(P_1(k,n)+P_2(k,n)).
\]
W~klasycznym problemie dnia urodzin zostało wyznaczone
\[
	P_1(k,n) = \frac{n!}{(n-k)!\,n^k} = \frac{k!}{n^k}\binom{n}{k},
\]
pozostaje zatem obliczyć $P_2(k,n)$.

Spośród $n$ dni w~roku wybierzmy jeden, który będzie urodzinami pewnych dwóch osób z~naszej grupy. Pozostałe $k-2$ osób możemy rozdzielić między $n-1$ dni oznaczających ich urodziny. Ponieważ rozróżniamy osoby, to liczbę sposobów takiego wyboru należy pomnożyć przez liczbę ich permutacji $k!$ i~podzielić przez~2 z~racji tego, że zmiana kolejności dwóch osób wybranych do tego samego dnia w~roku nie jest odrębnym przypadkiem. Ale takich par może być więcej. Można tak naprawdę wybrać $i\le\lfloor k/2\rfloor$ różnych dni, którym przypiszemy pary osób wtedy urodzonych -- liczba możliwości wynosi $\binom{n}{i}$. Pozostałe osoby rozdzielamy między pozostałe dni w~roku tak, aby każdej przypadł inny dzień, co da się wykonać na $\binom{n-i}{k-2i}$ sposobów. Podobnie jak wcześniej rozróżnianie osób wprowadza czynnik $k!/2^i$ -- kolejność w~parze w~ramach tego samego dnia jest bowiem nieistotna. Ponieważ liczba możliwości przypisania $k$ osób do $n$ różnych dni wynosi $n^k$, to ostatecznie otrzymujemy wzór
\[
	P_2(k,n) = \frac{k!}{n^k}\sum_{i=1}^{\lfloor k/2\rfloor}\frac{1}{2^i}\binom{n}{i}\binom{n-i}{k-2i}.
\]

Ustalając wartość $n$ na 365, można teraz obliczyć prawdopodobieństwa $P(k,n)$ dla wszystkich $k=1$, 2,~\dots,~$n$, po czym wyznaczyć najmniejszą wartość $k$, dla której $P(k,n)\ge1/2$. Okazuje się, że rozwiązaniem jest $k=88$.

\exercise %5.4-5
Wszystkich możliwych \onedash{$k$}{słów} nad zbiorem \onedash{$n$}{elementowym} jest $n^k$. Aby \onedash{$k$}{słowo} było w~istocie \onedash{$k$}{permutacją}, na pierwszy z~jego elementów należy wybrać jeden z~$n$ elementów zbioru, na drugi element jeden z~$n-1$ dotychczas niewybranych itd. Istnieje zatem $n(n-1)\dots(n-k+1)$ możliwych \onedash{$k$}{permutacji}, więc prawdopodobieństwo, że dane \onedash{$k$}{słowo} będzie jedną z~nich wynosi
\[
	\frac{n(n-1)\dots(n-k+1)}{n^k} = \biggl(1-\frac{1}{n}\biggr)\biggl(1-\frac{2}{n}\biggr)\dots\biggl(1-\frac{k-1}{n}\biggr).
\]

Problem jest analogiczny do pytania o~prawdopodobieństwo zdarzenia, że wśród $k$ osób nie ma dwóch takich, które urodziły się tego samego dnia roku, gdzie $n$ jest liczbą dni w~roku.

\exercise %5.4-6
Obliczmy najpierw oczekiwaną liczbę pustych urn. Niech $S_i$, dla $i=1$, 2,~\dots,~$n$, będzie zdarzeniem, że \onedash{$i$}{ta} urna jest pusta po wykonaniu $n$ rzutów. Definiujemy zmienną losową $X_i=\I(S_i)$ oraz $X=\sum_{i=1}^nX_i$, która oznacza liczbę pustych urn. Wtedy
\[
	\E(X) = \E\biggl(\sum_{i=1}^nX_i\biggr) = \sum_{i=1}^n\E(X_i) = \sum_{i=1}^n\Pr(S_i).
\]

Rozważmy teraz \onedash{$i$}{tą} urnę i~potraktujmy każdy rzut kulą jako próbę Bernoulliego, gdzie sukcesem jest trafienie do tej urny. Mamy zatem $n$ niezależnych prób Bernoulliego, każda z~prawdopodobieństwem sukcesu $p=1/n$. Aby \onedash{$i$}{ta} urna pozostała pusta, nie możemy uzyskać żadnego sukcesu, a~zatem korzystając z~rozkładu dwumianowego, dostajemy
\[
	\Pr(S_i) = b(0;n,p) = \binom{n}{0}\biggl(\frac{1}{n}\biggr)^0\biggl(1-\frac{1}{n}\biggr)^n = \biggl(1-\frac{1}{n}\biggr)^n
\]
oraz
\[
	\E(X) = \sum_{i=1}^n\biggl(1-\frac{1}{n}\biggr)^n = n\biggl(1-\frac{1}{n}\biggr)^n.
\]

Wyznaczmy teraz oczekiwaną liczbę urn z~dokładnie jedną kulą. W~tym celu, podobnie jak poprzednio, dla $i=1$, 2,~\dots,~$n$ zdefiniujemy zdarzenie $S_i$, że \onedash{$i$}{ta} urna po wykonaniu $n$ rzutów zawiera dokładnie jedną kulę. Definicje zmiennych losowych $X_i$ oraz $X$ pozostają bez zmian i, tak jak poprzednio, zachodzi
\[
	\E(X) = \sum_{i=1}^n\Pr(S_i).
\]
Dla analogicznej serii prób Bernoulliego stwierdzamy, że aby \onedash{$i$}{ta} urna zawierała dokładnie jedną kulę, potrzebny jest 1 sukces i~$n-1$ porażek, więc
\[
	\Pr(S_i) = b(1;n,p) = \binom{n}{1}\biggl(\frac{1}{n}\biggr)^1\biggl(1-\frac{1}{n}\biggr)^{n-1} = \biggl(1-\frac{1}{n}\biggr)^{n-1}
\]
oraz
\[
	\E(X) = \sum_{i=1}^n\biggl(1-\frac{1}{n}\biggr)^{n-1} = n\biggl(1-\frac{1}{n}\biggr)^{n-1}.
\]

\exercise %5.4-7
W~zadaniu przyjmujemy, że $n>16$, ponieważ wtedy wyrażenie $\lg n-2\lg\lg n$ jest dodatnie. Ponadto dla uproszczenia rachunków nie dbamy o~to, aby niektóre liczby były całkowite.

Korzystając z~przedstawionego w~Podręczniku wyprowadzenia, mamy, że prawdopodobieństwo zdarzenia, że ciąg orłów długości co najmniej $\lg n-2\lg\lg n$ rozpoczyna się na pozycji $i$, jest równe
\[
	\Pr(A_{i,\,\lg n-2\lg\lg n}) = \frac{1}{2^{\lg n-2\lg\lg n}} = \frac{2^{2\lg\lg n}}{2^{\lg n}} = \frac{\lg^2n}{n},
\]
a~zatem prawdopodobieństwo, że ciąg orłów o~długości co najmniej $\lg n-2\lg\lg n$ nie rozpoczyna się na pozycji $i$, wynosi
\[
	1-\frac{\lg^2n}{n}.
\]

Podzielmy ciąg $n$ rzutów monetą na $n/(\lg n-2\lg\lg n)$ grup po $\lg n-2\lg\lg n$ kolejnych rzutów każda. Grupy te złożone są z~różnych i~wzajemnie niezależnych rzutów, a~zatem prawdopodobieństwo, że żadna z~nich nie będzie ciągiem orłów o~długości $\lg n-2\lg\lg n$, wynosi
\begin{align*}
	\biggl(1-\frac{\lg^2n}{n}\biggr)^{n/(\lg n-2\lg\lg n)} &\le \bigl(e^{-(\lg^2n)/n}\bigr)^{n/(\lg n-2\lg\lg n)} \\
	&= e^{-(\lg^2n)/(\lg n-2\lg\lg n)} \\
	&< e^{-\lg n} \\
	&= 1/n.
\end{align*}
Skorzystaliśmy tutaj z~nierówności~(3.11) oraz z~tego, że dla $n>16$ zachodzi
\[
	\frac{\lg^2n}{\lg n-2\lg\lg n} > \lg n.
\]

\problems

\problem{Zliczanie probabilistyczne} %5-1

\subproblem %5-1(a)
Zdefiniujmy $X_j$, dla $j=1$, 2,~\dots,~$n$, jako zmienną losową oznaczającą liczbę, o~jaką zwiększy się wartość reprezentowana przez licznik po \onedash{$j$}{tym} wykonaniu operacji \proc{Increment}. Ponadto niech zmienna losowa $X$ przyjmuje wartość reprezentowaną przez licznik po wykonaniu $n$ operacji \proc{Increment}. Zachodzi $X=\sum_{j=1}^nX_j$ oraz, ze względu na liniowość wartości oczekiwanej,
\[
	\E(X) = \E\biggl(\sum_{j=1}^nX_j\biggr) = \sum_{j=1}^n\E(X_j).
\]

Załóżmy teraz, że przed wykonaniem \onedash{$j$}{tej} operacji \proc{Increment} licznik przechowuje wartość $i$, co stanowi reprezentację $n_i$. Jeśli inkrementacja powiedzie się, co zdarzy się z~prawdopodobieństwem równym $1/(n_{i+1}-n_i)$, to wartość reprezentowana na liczniku zwiększy się o~$n_{i+1}-n_i$. Dla każdego $j=1$, 2,~\dots,~$n$ mamy zatem
\[
	\E(X_j) = 0\cdot\biggl(1-\frac{1}{n_{i+1}-n_i}\biggr)+(n_{i+1}-n_i)\cdot\biggl(\frac{1}{n_{i+1}-n_i}\biggr) = 1,
\]
a~więc
\[
	\E(X) = \sum_{j=1}^n\E(X_j) = n,
\]
co należało wykazać.

\subproblem %5-1(b)
Dla zmiennych losowych $X_j$ oraz $X$ zdefiniowanych w~poprzednim punkcie mamy
\[
	\Var(X) = \Var\biggl(\sum_{j=1}^nX_j\biggr) = \sum_{j=1}^n\Var(X_j),
\]
co zachodzi na mocy wzoru~(C.28), ponieważ zmienne $X_1$, $X_2$,~\dots,~$X_n$ są parami niezależne. Mamy $n_i=100i$, a~więc zwiększenie wartości reprezentowanej przez licznik o~$n_{i+1}-n_i=100$ odbędzie się z~prawdopodobieństwem $1/(n_{i+1}-n_i)=1/100$. Ze wzoru~(C.26) otrzymujemy, że dla każdego $j=1$, 2,~\dots,~$n$ zachodzi
\[
	\Var(X_j) = \E(X_j^2)-\E^2(X_j) = 0^2\cdot\biggl(1-\frac{1}{100}\biggr)+100^2\cdot\frac{1}{100}-1^2 = 99,
\]
a~stąd
\[
	\Var(X) = \sum_{j=1}^n\Var(X_j) = 99n.
\]

\problem{Wyszukiwanie w~nieposortowanej tablicy} %5-2

\subproblem %5-2(a)
Oto procedura implementująca opisaną strategię:
\begin{codebox}
\Procname{$\proc{Random-Search}(A,x)$}
\li	$n\gets\id{length}[A]$
\li	\For $k\gets1$ \To $n$
\li		\Do $B[k]\gets\const{false}$
		\End
\li	$\id{checked}\gets0$
\li	\While $\id{checked}<n$ \label{li:random-search-while-begin}
\li		\Do
			$i\gets\proc{Random}(1,n)$
\li			\If $A[i]=x$
\li				\Then \Return $i$
				\End
\li			\If $B[i]=\const{false}$
\li				\Then
					$B[i]\gets\const{true}$
\li					$\id{checked}\gets\id{checked}+\,1$
				\End
		\End \label{li:random-search-while-end}
\li	\Return \const{nil}
\end{codebox}
Algorytm korzysta z~pomocniczej tablicy wartości logicznych $B[1\twodots n]$, która na pozycji $i$ przechowuje informację o~tym, czy wybrana była już \onedash{$i$}{ta} pozycja tablicy $A$. Ponadto zmienna \id{checked} przechowuje liczbę testowanych dotychczas komórek. W~każdej iteracji pętli \kw{while} w~wierszach \twodashes{\ref{li:random-search-while-begin}}{\ref{li:random-search-while-end}} algorytm sprawdza losowo wybrany indeks tablicy $A$. W~przypadku odnalezienia $x$ natychmiast zwracana jest jego pozycja. Jeśli jednak element $x$ nie zostanie odnaleziony, a~bieżąca komórka tablicy $A$ nie była jeszcze wcześniej sprawdzana, to informacja ta zostaje odnotowana w~tablicy $B$, a~zmienna \id{checked} jest inkrementowana. Jeśli elementu $x$ nie ma w~tablicy $A$, to po sprawdzeniu wszystkich indeksów co najmniej raz, algorytm zwróci specjalną wartość \const{nil}.

\subproblem %5-2(b)
Niech $X$ będzie zmienną losową oznaczającą ilość wybranych indeksów tablicy $A$ zanim odnaleziono $x$. Szukanie $x$ realizowane przez procedurę \proc{Random-Search} jest serią prób Bernoulliego, każda z~prawdopodobieństwem sukcesu $p=1/n$. Stosując wzór~(C.31), otrzymujemy, że zostanie wybranych średnio $\E(X)=1/p=n$ indeksów tablicy $A$.

\subproblem %5-2(c)
Rozważmy ponownie zmienną losową $X$ i~analogiczną serię prób Bernoulliego do tej z~poprzedniego punktu. Jednak w~tym przypadku sukces następuje z~prawdopodobieństwem $p=k/n$, a~zatem średnią liczbą wybranych indeksów przed odnalezieniem $x$ jest $\E(X)=1/p=n/k$.

\subproblem %5-2(d)
Ten przypadek wyszukiwania można sprowadzić do problemu kolekcjonera kuponów. Pozycje tablicy $A$ reprezentują kupony, których skompletowanie (odpowiadające sprawdzeniu wszystkich pozycji tablicy) jest celem problemu. Zgodnie z~uzasadnieniem podanym w~Podręczniku, aby uzbierać pełny zestaw $n$ kuponów pojawiających się losowo, należy zdobyć ich około $n\ln n$.

\subproblem %5-2(e)
Procedura \proc{Deterministic-Search} jest identyczna z~algorytmem wyszukiwania liniowego opisanego w~\refExercise{2.1-3}. Z~rozwiązania \refExercise{2.2-3} wynika zatem, że czas tego algorytmu -- wyrażony jako liczba sprawdzanych indeksów tablicy -- wynosi w~średnim przypadku $(n+1)/2$, a~w~pesymistycznym $n$.

\subproblem %5-2(f)
Oznaczmy przez $X$ zmienną losową przyjmującą liczbę wybranych indeksów tablicy $A$ przed odnalezieniem $x$. Zdarzenie $X=i$ zachodzi wtedy i~tylko wtedy, gdy pierwsza z~lewej wartość $x$ zajmuje w~$A$ pozycję $i$. Pozostałe $k-1$ elementów o~wartości $x$ można rozmieścić w~obszarze $A[i+1\twodots n]$ na $\binom{n-i}{k-1}$ sposobów. Stąd $\Pr(X=i)=\binom{n-i}{k-1}/\binom{n}{k}$. Wartość oczekiwana $X$ wynosi zatem
\[
    \E(X) = \sum_{i=1}^{n-k+1}i\Pr(X=i) = \frac{1}{\binom{n}{k}}\sum_{i=1}^{n-k+1}i\binom{n-i}{k-1}.
\]

Pokażemy przez indukcję po $n$, że dla dowolnego $k=1$, 2,~\dots,~$n$ zachodzi $E(X)=\frac{n+1}{k+1}$, co na mocy wzoru~(C.8) jest równoważne z~udowodnieniem tożsamości
\[
    \binom{n+1}{k+1} = \sum_{i=1}^{n-k+1}i\binom{n-i}{k-1}.
\]

Jeśli $k=n$, to po lewej stronie powyższego wzoru mamy $\binom{n+1}{n+1}=1$, a~po prawej stronie $\sum_{i=1}^1i\binom{n-i}{n-1}=\binom{n-1}{n-1}=1$. A~więc w~tym przypadku wzór jest prawdziwy. Pokazaliśmy przy okazji, że spełniony jest pierwszy krok indukcji, gdy $n=1$.

W~drugim kroku zakładamy, że $n>1$ i~że dla każdego $k=1$, 2,~\dots,~$n$ zachodzi
\[
    \binom{n}{k} = \sum_{i=1}^{n-k+1}i\binom{n-1-i}{k-2}.
\]
Korzystając dwukrotnie z~\refExercise{C.1-7}, dla dowolnego $k=1$, 2,~\dots,~$n-1$ mamy
\begin{align*}
    \binom{n+1}{k+1} &= \binom{n}{k+1}+\binom{n}{k} \\
	&= \sum_{i=1}^{n-k}i\binom{n-1-i}{k-1}+\sum_{i=1}^{n-k+1}i\binom{n-1-i}{k-2} \\
	&= \sum_{i=1}^{n-k}i\biggl(\binom{n-1-i}{k-1}+\binom{n-1-i}{k-2}\biggr)+(n-k+1)\binom{k-2}{k-2} \\
	&= \sum_{i=1}^{n-k}i\binom{n-i}{k-1}+(n-k+1)\binom{k-1}{k-1} \\
	&= \sum_{i=1}^{n-k+1}i\binom{n-i}{k-1}.
\end{align*}
Wzór jest zatem prawdziwy dla wszystkich $n$ naturalnych i~wszystkich $k=1$, 2,~\dots,~$n$.

Pesymistyczny przypadek dla algorytmu \proc{Deterministic-Search} ma miejsce wtedy, gdy wszystkie egzemplarze $x$ zajmują w~tablicy $k$ końcowych pozycji. Algorytm sprawdzi wówczas $n-k$ komórek tablicy, zanim odnajdzie pierwsze wystąpienie $x$.

\subproblem %5-2(g)
Przypadek średni i~pesymistyczny są równoważne przy braku $x$ w~tablicy $A$, bowiem w~obu tych przypadkach algorytm przegląda całą tablicę, co zajmuje czas $n$.

\subproblem %5-2(h)
Załóżmy, że do permutowania tablicy używany jest algorytm \proc{Randomize-In-Place}, który generuje permutację losową zgodnie z~rozkładem jednostajnym, wykonując przy tym $n$ zamian elementów. Czas algorytmu \proc{Scramble-Search} jest wtedy sumą $n$ oraz liczby porównań wykonywanych podczas deterministycznego wyszukiwania liniowego. Wartości te -- w~zależności od przypadku -- zostały wyznaczone w~punktach (e), (f) i~(g).

\subproblem %5-2(i)
W~przypadku gdy tablica nie zawiera szukanego elementu, czasy działania algorytmów \proc{Deterministic-Search} i~\proc{Scramble-Search} są asymptotycznie mniejsze od czasu działania \proc{Random-Search}, a~w~pozostałych przypadkach są one tego samego rzędu (o~ile traktujemy $k$ jako stałą). Jest to wystarczający powód, aby odrzucić algorytm \proc{Random-Search} z~praktycznych zastosowań. Spośród pozostałych dwóch \proc{Deterministic-Search} jest bardziej efektywny, ponieważ nie wprowadza narzutu w~postaci permutowania losowego tablicy i~w~efekcie działa szybciej.

Okazuje się więc, że w~problemie wyszukiwania zastosowanie randomizacji nie jest szczególnie pomocne i~zwykłe wyszukiwanie liniowe jest algorytmem optymalnym.

\endinput


\setcounter{part}{1}
\part{Sortowanie i~statystyki pozycyjne}

\setcounter{chapter}{5}
\chapter{Heapsort -- sortowanie przez kopcowanie}

\subchapter{Kopce}

\exercise %6.1-1
Korzystamy z~faktu, że kopiec stanowi prawie pełne drzewo binarne, tzn.\ takie, w~którym wszystkie poziomy, być może z~wyjątkiem najniższego, zawierają komplet węzłów. Jeśli drzewo to ma wysokość $h$, to maksymalnie może mieć $2^{h+1}-1$ węzłów (gdy jest drzewem pełnym), a~minimalnie $(2^h-1)+1=2^h$ (gdy jego najniższy poziom składa się z~tylko jednego węzła).

\exercise %6.1-2
Niech $h$ oznacza wysokość kopca. Z~poprzedniego zadania mamy, że $2^h\le n<2^{h+1}$, skąd dostajemy $h\le\lg n<h+1$. Ponieważ $h$ jest całkowite, to $h=\lfloor\lg n\rfloor$.

\exercise %6.1-3
Wartość korzenia każdego poddrzewa w~kopcu typu max jest równa lub większa od wartości obu synów tego korzenia (o~ile istnieją). Dla dowolnego poddrzewa $T$ można łatwo dowieść przez indukcję względem jego wysokości, że wartości węzłów wchodzących w~skład ścieżek od liści w~górę $T$, tworzą ciągi niemalejące. Ponieważ wszystkie takie ścieżki kończą się w~korzeniu poddrzewa $T$, to musi on mieć największą wartość w~$T$.

\exercise %6.1-4
Analizując ścieżki od liści do korzenia kopca jak w~poprzednim zadaniu, stwierdzamy, że najmniejsza wartość w~każdej takiej ścieżce znajduje się w~jej pierwszym elemencie. Ponieważ warunek kopca typu max nie narzuca żadnego ograniczenia w~zbiorze liści, to każdy z~nich może stanowić najmniejszą wartość kopca.

\exercise %6.1-5
Powtarzając rozumowanie z~\refExercise{6.1-3} dla kopców typu min, wnioskujemy, że korzeń takiego kopca stanowi jego najmniejszy element, czyli zajmuje pierwszą pozycję w~posortowanej (niemalejąco) tablicy $A$. Dla każdego indeksu tablicy $i$ z~wyjątkiem pierwszego zachodzi $\proc{Parent}(i)<i$, a~więc $A[\proc{Parent}(i)]\le A[i]$. Własność kopca typu min jest zatem spełniona i~tablica posortowana $A$ stanowi taki kopiec.

\exercise %6.1-6
Potraktujmy ten ciąg jak tablicę $A$. Elementy na pozycjach $i=9$ oraz $\proc{Parent}(i)=4$ nie spełniają własności $A[\proc{Parent}(i)]\ge A[i]$, zatem tablica $A$ nie jest kopcem typu max.

\exercise %6.1-7
Element kopca na pozycji $i$ nie jest liściem wtedy i~tylko wtedy, gdy istnieje jego lewy syn. W~kopcu o~$n$ elementach wierzchołki wewnętrzne znajdują się zatem na pozycjach $i$ takich, że $\proc{Left}(i)\le n$. Warunek ten sprowadza się do nierówności $2i\le n$, skąd $i\le\lfloor n/2\rfloor$, bo $i$ jest całkowite. Pozostałe wierzchołki są liśćmi i~zajmują pozycje $\lfloor n/2\rfloor+1\twodots n$.

\subchapter{Przywracanie własności kopca}

\exercise %6.2-1
Rys.~\ref{fig:6.2-1} przedstawia działanie procedury $\proc{Max-Heapify}(A,3)$.
\begin{figure}[ht]
	\begin{center}
		\includegraphics{fig_6.2-1}
	\end{center}
	\caption{Działanie procedury $\proc{Max-Heapify}(A,3)$ dla tablicy $A=\langle$27,\!~17,\! 3,\! 16,\! 13,\! 10,\! 1,\! 5,\! 7,\! 12,\! 4,\! 8,\! 9,\!~0$\rangle$. {\sffamily\bfseries\doubledash{(a)}{(b)}} Drzewo binarne reprezentujące $A$, w~którym przywracana jest własność kopca typu max, odpowiednio, w~węzłach $i=3$ oraz $i=6$. {\sffamily\bfseries(c)} Wynikowy kopiec z~przywróconą własnością kopca.} \label{fig:6.2-1}
\end{figure}

\exercise %6.2-2
Poniższy pseudokod prezentuje procedurę przywracania własności kopca typu min. Ponieważ jedyną modyfikacją w~porównaniu z~procedurą \proc{Max-Heapify} jest zmiana znaków nierówności na przeciwne w~warunkach w~wierszach~\ref{li:min-heapify-check1} i~\ref{li:min-heapify-check2}, to czas działania tej procedury jest identyczny z~czasem działania \proc{Max-Heapify}, czyli $\Theta(\lg n)$.
\begin{codebox}
\Procname{$\proc{Min-Heapify}(A,i)$}
\li	$l\gets\proc{Left}(i)$
\li	$r\gets\proc{Right}(i)$
\li	\If $l\le\attrib{A}{heap-size}$ i~$A[l]<A[i]$ \label{li:min-heapify-check1}
\li		\Then $\id{smallest}\gets l$
\li		\Else $\id{smallest}\gets i$
		\End
\li	\If $r\le\attrib{A}{heap-size}$ i~$A[r]<A[\id{smallest}]$ \label{li:min-heapify-check2}
\li		\Then $\id{smallest}\gets r$
		\End
\li	\If $\id{smallest}\ne i$
\li		\Then
			zamień $A[i]\leftrightarrow A[\id{smallest}]$
\li			$\proc{Min-Heapify}(A,\id{smallest})$
		\End
\end{codebox}

\exercise %6.2-3
Jeśli element $A[i]$ jest większy niż jego synowie, to \id{largest} jest ustawiane na $i$ i~warunek z~wiersza~8 nie jest spełniony. Procedura zakończy więc działanie, nie dokonując żadnej zamiany elementów.

\exercise %6.2-4
Z~\refExercise{6.1-7} mamy, że element o~indeksie $i>\attrib{A}{heap-size}/2$ jest liściem kopca, czyli nie istnieją jego synowie. W~dwóch pierwszych wierszach procedury \proc{Max-Heapify} obliczone zostaną wartości przekraczające \attrib{A}{heap-size}, więc zmienna \id{largest} przyjmie wartość $i$. Warunek w~wierszu~8 będzie więc fałszywy i~natychmiast po jego sprawdzeniu procedura zakończy działanie.

\exercise %6.2-5
Iteracyjna wersja procedury \proc{Max-Heapify} została przedstawiona poniżej.
\begin{codebox}
\Procname{$\proc{Iterative-Max-Heapify}(A,i)$}
\li	\While \const{true}
\li		\Do
			$l\gets\proc{Left}(i)$ \label{li:iterative-max-heapify-begin}
\li			$r\gets\proc{Right}(i)$
\li			\If $l\le\attrib{A}{heap-size}$ i~$A[l]>A[i]$
\li				\Then $\id{largest}\gets l$
\li				\Else $\id{largest}\gets i$
				\End
\li			\If $r\le\attrib{A}{heap-size}$ i~$A[r]>A[\id{largest}]$
\li				\Then $\id{largest}\gets r$
				\End \label{li:iterative-max-heapify-end}
\li			\If $\id{largest}=i$ \label{li:iterative-max-heapify-cond}
\li				\Then \Return
				\End
\li			zamień $A[i]\leftrightarrow A[\id{largest}]$
\li			$i\gets\id{largest}$
		\End
\end{codebox}
Działania wykonywane w~wierszach \doubledash{\ref{li:iterative-max-heapify-begin}}{\ref{li:iterative-max-heapify-end}} są identyczne jak w~oryginalnej implementacji procedury. W~zależności od wyniku testu z~wiersza~\ref{li:iterative-max-heapify-cond} procedura kończy działanie albo zamienia elementy $A[i]$ i~$A[\id{largest}]$, po czym symuluje wywołanie rekurencyjne, aktualizując wartość zmiennej $i$ i~wykonując kolejną iterację pętli \kw{while}.

\exercise %6.2-6
Najgorszy przypadek dla procedury \proc{Max-Heapify} zachodzi wówczas, gdy zostanie ona wywołana dla korzenia kopca i~schodzi rekurencyjnie aż do jego ostatniego poziomu. Najdłuższa ścieżka od korzenia do liścia składa się z~$h=\lfloor\lg n\rfloor$ krawędzi (z~\refExercise{6.1-2}) i~tyle będzie wywołań rekurencyjnych procedury w~najgorszym przypadku. Koszt pracy wykonanej na każdym poziomie rekursji jest stały, a~więc procedura \proc{Max-Heapify} działa wtedy w~czasie $\Omega(\lg n)$. Przykładowym drzewem, dla którego procedura wykona opisane operacje, jest takie, w~którym korzeń ma wartość 0, a~każdy inny węzeł ma wartość 1.

\subchapter{Budowanie kopca}

\exercise %6.3-1
Ilustracja działania procedury \proc{Build-Max-Heap} dla tablicy $A$ znajduje się na rys.~\ref{fig:6.3-1}.
\begin{figure}[ht!]
	\begin{center}
		\includegraphics{fig_6.3-1}
	\end{center}
	\caption{Działanie procedury \proc{Build-Max-Heap} dla tablicy $A=\langle5,3,17,10,84,19,6,22,9\rangle$. {\sffamily\bfseries(a)} Tablica $A$ i~reprezentowane przez nią drzewo binarne przed pierwszym wywołaniem \proc{Max-Heapify} z~wiersza~3. {\sffamily\bfseries\doubledash{(b)}{(d)}} Drzewo przed każdym kolejnym wywołaniem \proc{Max-Heapify}. {\sffamily\bfseries(e)} Wynikowy kopiec typu max} \label{fig:6.3-1}
\end{figure}

\exercise %6.3-2
Wywołując $\proc{Max-Heapify}(A,i)$, zakładamy, że drzewa o~korzeniach w~$\proc{Left}(i)$ i~$\proc{Right}(i)$ (o~ile istnieją) są kopcami typu max. Jeżeli podczas budowy kopca procedura \proc{Max-Heapify} byłaby wywoływana dla węzłów o~rosnących indeksach, to nie moglibyśmy zagwarantować, że założenie to jest spełnione, dlatego przetwarzanie odbywa się w~kolejności malejących indeksów.

\exercise %6.3-3
Oznaczmy kopiec przez $T$, a~przez $n_h$ -- ilość węzłów kopca $T$ znajdujących się na wysokości $h$. Udowodnimy fakt przez indukcję względem $h$.

W~pierwszym kroku indukcji musimy pokazać, że $n_0\le\lceil n/2\rceil$. W~rzeczywistości udowodnimy, że $n_0=\lceil n/2\rceil$. Korzystając z~\refExercise{6.1-7}, mamy, że węzły znajdujące się w~$T$ na wysokości 0, czyli jego liście, zajmują pozycje $\lfloor n/2\rfloor+1\twodots n$. Jest ich zatem
\[
    n_0 = n-(\lfloor n/2\rfloor+1)+1 = n-\lfloor n/2\rfloor = \lceil n/2\rceil.
\]
A~więc przypadek bazowy indukcji jest spełniony.

Załóżmy teraz, że $h>0$ i~że twierdzenie jest spełnione dla węzłów na wysokości $h-1$. Ponadto niech $T'$ będzie kopcem powstałym z~$T$ po usunięciu z~niego wszystkich jego liści. Nowy kopiec ma zatem $n'=n-n_0$ węzłów. Ponieważ w~kroku bazowym pokazaliśmy, że $n_0=\lceil n/2\rceil$, to stąd $n'=n-\lceil n/2\rceil=\lfloor n/2\rfloor$. Węzły, które w~kopcu $T$ znajdują się na wysokości $h$, w~$T'$ zajmują wysokość $h-1$, więc jeśli oznaczymy przez $n_{h-1}'$ liczbę węzłów na wysokości $h-1$ w~kopcu $T'$, to wówczas będzie $n_h=n_{h-1}'$. Wykorzystując założenie indukcyjne, dostajemy
\[
    n_h = n_{h-1}' \le \lceil n'\!/2^h\rceil = \lceil\lfloor n/2\rfloor/2^h\rceil \le \lceil(n/2)/2^h\rceil = \lceil n/2^{h+1}\rceil,
\]
co kończy dowód.

\subchapter{Algorytm sortowania przez kopcowanie (heapsort)}

\exercise %6.4-1
Na rys.~\ref{fig:6.4-1} przedstawiono ilustrację działania sortowania przez kopcowanie dla tablicy $A$.
\begin{figure}[ht]
	\begin{center}
		\includegraphics{fig_6.4-1}
	\end{center}
	\caption{Działanie procedury \proc{Heapsort} dla tablicy $A=\langle5,13,2,25,7,17,20,8,4\rangle$. {\sffamily\bfseries(a)} Kopiec zaraz po jego zbudowaniu przez \proc{Build-Max-Heap}. {\sffamily\bfseries\doubledash{(b)}{(i)}} Kopiec i~elementy z~niego usunięte po każdym wywołaniu \proc{Max-Heapify} w~wierszu~5. {\sffamily\bfseries(j)} Wynikowa posortowana tablica.} \label{fig:6.4-1}
\end{figure}

\exercise %6.4-2
\begin{description}
	\item[Inicjowanie:] Przed pierwszą iteracją pętli mamy $i=\attrib{A}{length}=n$. Wówczas fragment $A[1\twodots i]$ jest całą tablicą $A$, która stanowi kopiec typu max, utworzony w~wyniku działania procedury \proc{Build-Max-Heap}, natomiast fragment $A[i+1\twodots n]$ jest pusty.
	\item[Utrzymanie:] Załóżmy, że niezmiennik jest prawdziwy przed wykonaniem kolejnej iteracji pętli. Podtablica $A[1\twodots i]$ tworzy więc kopiec typu max, którego korzeniem jest $A[1]$, czyli największy element w~tej podtablicy. Po wykonaniu wiersza~3 znajdzie się on na pozycji $i$. Dekrementacja \attrib{A}{heap-size} powoduje, że element $A[i]$ nie wchodzi teraz w~skład kopca, ale podtablica $A[i\twodots n]$ zawiera teraz $n-i+1$ największych elementów z~$A[1\twodots n]$ posortowanych niemalejąco, ponieważ element $A[i]$ jest równy lub mniejszy od wcześniej umieszczonych tam elementów. W~tym momencie korzeń może naruszać własność kopca typu max, dlatego w~wierszu~5 zostaje wywołana dla niego procedura \proc{Max-Heapify} przywracająca tę własność. Uaktualnienie $i$ powoduje odtworzenie niezmiennika.
	\item[Zakończenie:] Po zakończeniu działania pętli jest $i=1$, zatem podtablica $A[1\twodots i]$ składa się z~jednego elementu, który jest najmniejszym elementem tablicy $A$. Ponadto $n-1$ pozostałych elementów jest uszeregowanych w~kolejności niemalejącej w~podtablicy $A[2\twodots n]$. Stąd mamy, że tablica $A$ jest posortowana.
\end{description}

\exercise %6.4-3
Na podstawie analizy zamieszczonej w~Podręczniku czasem działania algorytmu heapsort dla tablicy o~rozmiarze $n$ jest $O(n\lg n)$. Z~\refExercise{6.4-5} mamy, że jest to w~rzeczywistości oszacowanie dokładne. A~zatem w~szczególności dla tablicy posortowanej rosnąco i~tablicy posortowanej malejąco heapsort działa w~czasie $\Theta(n\lg n)$.

\exercise %6.4-4
Przypadek pesymistyczny algorytmu heapsort ma miejsce wówczas, gdy każde wywołanie \proc{Max-Heapify} z~wiersza~5 schodzi rekurencyjnie aż do ostatniego poziomu drzewa. Na mocy wyniku z~\refExercise{6.2-6} oraz wzoru~(3.18) czasem działania algorytmu heapsort w~takim przypadku jest
\[
	T(n) = \Theta(n)+\sum_{i=2}^{n}\Omega(\lg i) = \Theta(n)+\Omega(\lg(n!)) = \Omega(n\lg n),
\]
przy czym składnik $\Theta(n)$ jest czasem spędzonym na budowaniu kopca z~tablicy \singledash{$n$}{elementowej}.

\exercise %6.4-5
Dokonamy analizy liczby wykonywanych instrukcji z~linii~9 procedury \proc{Max-Heapify} podczas działania algorytmu sortowania przez kopcowanie w~przypadku optymistycznym.

Załóżmy, że algorytm heapsort działa na tablicy $A$ o~rozmiarze $n=2^{h+1}-1$, gdzie $h$ jest dodatnią liczbą całkowitą. A~zatem kopiec zbudowany z~$A$ stanowi pełne drzewo binarne o~wysokości $h$. Rozważanie tylko takich kopców nie powoduje zmniejszenia ogólności analizy. Przez \singledash{$j$}{ty} etap działania algorytmu heapsort, gdzie $j=0$, 1,~\dots,~$h-1$, będziemy rozumieć działania wykonywane podczas iteracji pętli \kw{for} z~procedury \proc{Heapsort}, w~których $2^{h-j}\le i\le2^{h-j+1}-1$. Inaczej mówiąc, \singledash{$j$}{ty} etap pozbawia kopiec \singledash{$(h-j)$}{tego} poziomu.

\medskip
\noindent\textsf{\textbf{Lemat.}} \textit{Podczas\/ \singledash{$j$}{tego} etapu działania algorytmu heapsort na kopcu\/ $A$ o~rozmiarze\/ $n=2^{h+1}-1$,\/ $h\ge5$, którego wszystkie elementy są różne, liczba wykonanych zamian elementów w~linii~9 procedury \proc{Max-Heapify},\/ $m_j$, jest większa niż\/ $(h-j-5)2^{h-j-3}$.}
\begin{proof}
Niech $j=0$. Bez utraty ogólności załóżmy, że $\langle A[1],A[2],\dots,A[n]\rangle$ jest permutacją $\langle1,2,\dots,n\rangle$. Liczbę $k$ będziemy nazywać \textbf{dużą}, jeśli $k\ge(n+1)/2$. Niech $S$ będzie zbiorem indeksów dużych elementów w~kopcu $A$, które nie są liśćmi, czyli
\[
    S = \biggl\{\,i\in\Bigl\{1,2,\dots,\frac{n-1}{2}\Bigr\}:A[i]\ge\frac{n+1}{2}\,\biggr\}.
\]
Zauważmy, że wszystkie elementy, których pozycjami w~$A$ są indeksy ze zbioru $S$, zostaną usunięte z~kopca w~etapie $j=0$. A~zatem muszą wpierw znaleźć się w~korzeniu kopca za sprawą wykonania pewnej liczby zamian z~linii~9 procedury \proc{Max-Heapify}. Stąd $m_0$ spełnia nierówność
\[
    m_0 \ge \sum_{i\in S}d_i,
\]
gdzie $d_i$ oznacza głębokość węzła o~początkowej pozycji $i$ w~kopcu $A$.

Węzły o~indeksach ze zbioru $S$ tworzą w~kopcu $A$ poddrzewo $T$ o~korzeniu w~$A[1]$. Jest tak dlatego, że jeśli węzeł $A[i]$ jest duży, to $A[\proc{Parent}(i)]$ również jest duży, a~więc także wszystkie węzły na ścieżce od $A[i]$ do korzenia kopca, czyli $A[1]$. Jeśli zastąpimy każde puste poddrzewo w~$T$ pojedynczym węzłem, to dostaniemy regularne drzewo binarne, którego długość ścieżki wewnętrznej (patrz \refExercise{B.5-5}) wynosi $m_0$. W~zbiorze wszystkich drzew binarnych o~$|S|$ węzłach wewnętrznych najmniejsza możliwa długość ścieżki wewnętrznej jest osiągana dla pełnego drzewa binarnego (przy czym ostatni poziom tego drzewa może nie być wypełniony) i~wynosi $\sum_{k=1}^{|S|}\lfloor\lg k\rfloor$. Korzystając ze wzoru~(3.3) i~\refExercise{8.1-2}, mamy
\[
    m_0 \ge \sum_{k=1}^{|S|}\lfloor\lg k\rfloor > \sum_{k=1}^{|S|}(\lg k-1) = \sum_{k=1}^{|S|}\lg k-|S| \ge \frac{|S|}{2}\lg\frac{|S|}{2}-|S| = \frac{|S|}{2}\lg|S|-\frac{3}{2}|S|.
\]

Pokażemy teraz, że $|S|\ge2^{h-2}$. Rozważmy w~tym celu permutację $\pi$ elementów kopca $A$ na początku zerowego etapu w~kolejności ich odwiedzania podczas przechodzenia kopca metodą inorder. Jeśli $\pi(i)$, gdzie $i\ge2$, jest liściem kopca, to $\pi(i-1)$ nie może być liściem kopca. Jeśli w~dodatku $\pi(i)$ jest dużym liściem, to $\pi(i-1)$ jest dużym węzłem wewnętrznym. Stąd indeks elementu $\pi(i-1)$ należy do $S$. Mamy więc, że $l$ -- liczba dużych liści -- nie przekracza $|S|+1$, nawet jeśli $\pi(1)$ jest dużym liściem. Ponieważ liczba dużych elementów w~kopcu wynosi $(n+1)/2=2^h$, to otrzymujemy, że $|S|=2^h-l\ge2^h-(|S|+1)$, skąd $|S|\ge2^{h-1}-1/2\ge2^{h-2}$.

Powracając teraz do oszacowania na $m_0$, mamy
\[
    m_0 > \frac{|S|}{2}\lg|S|-\frac{3}{2}|S| = \frac{|S|}{2}(\lg|S|-3) \ge 2^{h-3}(h-5),
\]
czyli lemat jest prawdziwy, gdy $j=0$.

Na początku \singledash{$j$}{tego} etapu kopiec ma wysokość $h-j$, więc dowód lematu dla \singledash{$j$}{tego} etapu, gdzie $1\le j\le h-1$, sprowadza się do dowodu oszacowania $m_0$ dla kopca o~rozmiarze $n=2^{h-j+1}-1$.
\end{proof}

Załóżmy teraz, że $h\ge5$. Sumaryczną liczbę zamian elementów podczas sortowania $n$ liczb możemy, dzięki powyższemu lematowi, ograniczyć od dołu:
\[
    \sum_{j=0}^{h-1}m_j > \sum_{j=0}^{h-5}m_j > \sum_{j=0}^{h-5}(h-j-5)2^{h-j-3} = \sum_{j=0}^{h-5}j2^{j+2} = 4\sum_{j=0}^{h-5}j2^j.
\]
Ostatnią sumę obliczamy poprzez skorzystanie ze wzoru~(A.5):
\[
    \sum_{j=0}^{h-5}jx^j = x\cdot\frac{d}{dx}\biggl(\sum_{j=0}^{h-5}x^j\biggr) = x\cdot\frac{d}{dx}\biggl(\frac{x^{h-4}-1}{x-1}\biggr) = x\,\frac{(h-4)x^{h-5}(x-1)-(x^{h-4}-1)}{(x-1)^2}.
\]
Przyjmując teraz $x=2$ i~korzystając z~nierówności $2^h>n/2$ i~$h>\lg n-1$, mamy ostatecznie
\[
    4\sum_{j=0}^{h-5}j2^j = (h-4)2^{h-2}-2^{h-1}+8 > (h-6)2^{h-2} > \frac{1}{8}n\lg n-\frac{7}{8}n = \Omega(n\lg n).
\]
Otrzymany wynik stanowi oszacowanie czasu działania algorytmu heapsort w~przypadku optymistycznym.

\subchapter{Kolejki priorytetowe}

\exercise %6.5-1
Na rys.~\ref{fig:6.5-1} został przedstawiony kopiec wejściowy $A$ i~wynikowy kopiec otrzymany w~wyniku działania procedury \proc{Heap-Extract-Max}. W~wierszu~3 zmiennej \id{max} przypisywana jest maksymalna wartość kopca, czyli 15. Następnie korzeń otrzymuje wartość 1 i~rozmiar kopca jest pomniejszany o~1. Po przywróceniu własności kopca w~linii~6 procedura zwraca wartość \id{max}.
\begin{figure}[ht]
	\begin{center}
		\includegraphics{fig_6.5-1}
	\end{center}
	\caption{Działanie procedury \proc{Heap-Extract-Max} dla kopca $A=\langle15,13,9,5,12,8,7,4,0,6,2,1\rangle$. {\sffamily\bfseries(a)} Kopiec wejściowy $A$. {\sffamily\bfseries(b)} Kopiec $A$ po usunięciu maksymalnej wartości i~przywróceniu własności kopca naruszonej przez korzeń, któremu wcześniej przypisano wartość 1.} \label{fig:6.5-1}
\end{figure}

\exercise %6.5-2
Procedura \proc{Max-Heap-Insert} rozpoczyna działanie od dodania do kopca nowego elementu o~wartości $-\infty$. Wartość ta jest następnie odpowiednio modyfikowana i~element jest umieszczany w~odpowiednim miejscu w~kopcu dzięki wywołaniu \proc{Heap-Increase-Key}. Działanie procedury $\proc{Max-Heap-Insert}(A,10)$ zostało przedstawione na rys.~\ref{fig:6.5-2}.
\begin{figure}[ht]
	\begin{center}
		\includegraphics{fig_6.5-2}
	\end{center}
	\caption{Działanie procedury $\proc{Max-Heap-Insert}(A,10)$ dla kopca $A=\langle$15,\!~13,\! 9,\! 5,\! 12,\! 8,\! 7,\! 4,\! 0,\! 6,\! 2,\!~1$\rangle$. {\sffamily\bfseries(a)} Kopiec po dodaniu nowego elementu o~wartości początkowej $-\infty$. {\sffamily\bfseries(b)} Działa teraz procedura \proc{Heap-Increase-Key}. Na rysunku pokazano wartość zmiennej $i$ w~tej procedurze. Wartość nowego elementu została zwiększona i~wynosi teraz 10. {\sffamily\bfseries(c)} Po wykonaniu pierwszej iteracji pętli \kw{while} procedury \proc{Heap-Increase-Key} nowy element został zamieniony ze swoim ojcem. {\sffamily\bfseries(d)} Po drugiej iteracji pętli nastąpiła jeszcze jedna zamiana nowego elementu i~jego aktualnego ojca, dzięki czemu $A$ spełnia już własność kopca i~procedura kończy działanie.} \label{fig:6.5-2}
\end{figure}

\exercise %6.5-3
Zakładamy, że tablica $A$ stanowi kopiec typu min. Poniższe procedury stanowią implementację kolejki priorytetowej typu min i~działają analogicznie do odpowiadających im procedur dla kolejki priorytetowej typu max.
\begin{codebox}
\Procname{$\proc{Heap-Minimum}(A)$}
\li	\Return $A[1]$
\end{codebox}
\begin{codebox}
\Procname{$\proc{Heap-Extract-Min}(A)$}
\li	\If $\attrib{A}{heap-size}<1$
\li		\Then \Error ,,kopiec pusty''
		\End
\li	$\id{min}\gets A[1]$
\li	$A[1]\gets A[\attrib{A}{heap-size}]$
\li	$\attrib{A}{heap-size}\gets\attrib{A}{heap-size}-1$
\li	$\proc{Min-Heapify}(A,1)$
\li	\Return \id{min}
\end{codebox}
\begin{codebox}
\Procname{$\proc{Heap-Decrease-Key}(A,i,\id{key})$}
\li	\If $\id{key}>A[i]$
\li		\Then \Error ,,nowy klucz jest większy niż klucz aktualny''
		\End
\li	$A[i]\gets\id{key}$
\li	\While $i>1$ i~$A[\proc{Parent}(i)]>A[i]$
\li		\Do
			zamień $A[i]\leftrightarrow A[\proc{Parent}(i)]$
\li			$i\gets\proc{Parent}(i)$
		\End
\end{codebox}
\begin{codebox}
\Procname{$\proc{Min-Heap-Insert}(A,\id{key})$}
\li	$\attrib{A}{heap-size}\gets\attrib{A}{heap-size}+1$
\li	$A[\attrib{A}{heap-size}]\gets\infty$
\li	$\proc{Heap-Decrease-Key}(A,\attrib{A}{heap-size},\id{key})$
\end{codebox}

\exercise %6.5-4
Po wykonaniu wiersza~1 procedury \proc{Max-Heap-Insert} wartość $A[\attrib{A}{heap-size}]$ pozostaje niezdefiniowana i~może zawierać liczbę większą niż \id{key}. Wówczas jednak wywołanie \proc{Heap-Increase-Key} zakończy się z~błędem. Radzimy sobie z~tym problemem poprzez nadanie elementowi wartości $-\infty$.

\exercise %6.5-5
\begin{description}
	\item[Inicjowanie:] Przed wykonaniem wiersza~3 tablica $A[1\twodots\attrib{A}{heap-size}]$ jest kopcem typu max. Zwiększenie wartości $A[i]$ może naruszyć własność kopca tylko dla elementów $A[i]$ oraz $A[\proc{Parent}(i)]$.
	\item[Utrzymanie:] Dokonując zamiany elementów w~wierszu~5 w~bieżącej iteracji pętli, przywracamy własność kopca dla elementów $A[i]$ oraz $A[\proc{Parent}(i)]$. Jednak operacja ta może wygenerować nową parę elementów niespełniających własności kopca: $A[\proc{Parent}(i)]$ oraz $A[\proc{Parent}(\proc{Parent}(i))]$. Aktualizacja wartości $i$ powoduje zachowanie niezmiennika, albowiem nowa para elementów jest jedyną, która może naruszać własność kopca.
	\item[Zakończenie:] Pętla kończy działanie, gdy $i\le1$ lub $A[\proc{Parent}(i)]\ge A[i]$. W~pierwszym przypadku $\proc{Parent}(i)\le0$, co jest niepoprawną wartością dla indeksów tablicy $A$. W~drugim natomiast jedyna para, która mogłaby naruszać własność kopca, w~rzeczywistości ją spełnia. A~zatem po zakończeniu wykonywania pętli tablica $A[1\twodots\attrib{A}{heap-size}]$ stanowi kopiec typu max.
\end{description}
Z~prawdziwości niezmiennika pętli wynika, że procedura \proc{Heap-Increase-Key} poprawnie zwiększa wartość węzła $i$, pozostawiając kopiec typu max.

\exercise %6.5-6
Kolejkę FIFO implementujemy, wykorzystując do tego celu kolejkę priorytetową typu min. Przy inicjalizacji kolejki będziemy ustawiać wartość dodatkowej zmiennej \id{rank} na 1. Przed dodaniem nowego elementu do kolejki FIFO nadamy mu rangę, czyli powiążemy go z~aktualną wartością zmiennej \id{rank}, po czym wstawimy element wraz z~jego rangą do kolejki priorytetowej procedurą \proc{Min-Heap-Insert}. Warunek kolejki priorytetowej spełniany będzie tylko na podstawie wartości rang elementów. Po umieszczeniu obiektu w~kolejce wartość zmiennej \id{rank} zostanie zwiększona o~1. Z~kolei usuwanie elementów będzie odbywać się poprzez zwykłe wywołanie procedury \proc{Heap-Extract-Min}. Taka implementacja operacji na kolejce FIFO zapewnia, że w~danym momencie w~strukturze danych nie będzie dwóch różnych elementów z~tą samą rangą i~elementy pobierane będą w~odpowiedniej kolejności.

Realizacja stosu jest podobna, ale używamy do tego celu kolejki priorytetowej typu max, w~której porównań dokonujemy na rangach związanych z~elementami. Podczas wstawiania elementów na stos korzystamy z~procedury \proc{Max-Heap-Insert} i~inkrementujemy zmienną \id{rank} (zainicjalizowaną na 1 w~momencie utworzenia stosu). Usuwanie polega na odnalezieniu i~pobraniu elementu z~największą rangą, co realizowane jest za pomocą \proc{Heap-Extract-Max}. W~wyniku tego elementy pobierane są w~kolejności odwrotnej do tej, w~której były wstawiane.

\exercise %6.5-7
\note{Zmienimy nazwę operacji z~sugerowanej w~Podręczniku \proc{Heap-Delete} na \proc{Max-Heap-Delete}, aby odróżnić ją od analogicznej procedury dla kopca typu min.}

\noindent Przedstawiona poniżej procedura \proc{Max-Heap-Delete} zamienia element $A[i]$ w~\singledash{$n$}{elementowym} kopcu $A$ typu max z~jego liściem $A[\attrib{A}{heap-size}]$, po czym dekrementuje \attrib{A}{heap-size}. Po tym kroku własność kopca może być naruszona przez węzeł $A[i]$ na dwa sposoby. W~pierwszym przypadku mamy sytuację, w~której $A[i]<A[\proc{Left}(i)]$ lub $A[i]<A[\proc{Right}(i)]$ -- przywracamy więc własność kopca za pomocą wywołania \proc{Max-Heapify}. W~drugim przypadku $A[i]>A[\proc{Parent}(i)]$, więc w~celu odbudowy struktury kopca wystarczy wykonać podobne operacje, jak w~procedurze \proc{Heap-Increase-Key}. Ostatni krok procedury to zwrócenie elementu, który początkowo zajmował w~kopcu pozycję $i$.
\begin{codebox}
\Procname{$\proc{Max-Heap-Delete}(A,i)$}
\li	zamień $A[i]\leftrightarrow A[\attrib{A}{heap-size}]$
\li	$\attrib{A}{heap-size}\gets\attrib{A}{heap-size}-1$
\li	$\proc{Max-Heapify}(A,i)$ \label{li:max-heap-delete-heapify}
\li	\While $i>1$ i~$A[\proc{Parent}(i)]<A[i]$ \label{li:max-heap-delete-while-begin}
\li		\Do
			zamień $A[i]\leftrightarrow A[\proc{Parent}(i)]$
\li			$i\gets\proc{Parent}(i)$
		\End \label{li:max-heap-delete-while-end}
\li	\Return $A[\attrib{A}{heap-size}+1]$
\end{codebox}

Zarówno wywołanie z~wiersza~\ref{li:max-heap-delete-heapify}, jak i~pętla \kw{while} w~wierszach \doubledash{\ref{li:max-heap-delete-while-begin}}{\ref{li:max-heap-delete-while-end}} zajmuje czas $O(\lg n)$, a~więc czasem działania procedury \proc{Max-Heap-Delete} jest również $O(\lg n)$.

\exercise %6.5-8
W~algorytmie wykorzystamy kolejkę priorytetową typu min jako strukturę pomocniczą. Na początku do kolejki zostaną przeniesione elementy z~głowy każdej listy. Jest oczywiste, że wśród tych elementów znajduje się najmniejszy element listy wynikowej. Aby go uzyskać, wystarczy wywołać na kolejce operację \proc{Extract-Min}. Kolejnego elementu należy szukać wśród aktualnych węzłów kolejki lub w~głowie listy, do której początkowo należało usunięte przed chwilą minimum. Głowę tej listy, o~ile istnieje, przenosimy do kolejki. W~kolejnych iteracjach powtarzamy te operacje -- pobieramy najmniejszy element kolejki i~wstawiamy na listę wynikową, po czym uzupełniamy kolejkę głową listy, do której należał pobrany element, o~ile lista ta nie jest jeszcze pusta. Proces ten powtarzamy aż do opróżnienia kolejki, co następuje po przetworzeniu zawartości wszystkich list. Aby zachować kolejność niemalejącą na liście wynikowej, musimy wstawiać elementy na jej koniec, pamiętając wskaźnik do ogona tej listy.

Podczas działania algorytmu wykonamy $n$ razy operację wstawienia węzła do kolejki zawierającej co najwyżej $k$ elementów i~tyleż samo operacji \proc{Extract-Min}. Otrzymujemy zatem górne oszacowanie $O(n\lg k)$ na czas działania algorytmu, przy założeniu, że kolejka priorytetowa została zaimplementowana w~oparciu o~kopiec typu min.

\problems

\problem{Budowa kopca przez wstawianie} %6-1

\subproblem %6-1(a)
Procedury te nie zawsze generują identyczne kopce dla tej samej tablicy wejściowej. Jeśli na przykład rozważymy tablicę $A=\langle1,2,3\rangle$, to kopce budowane przez obie procedury różnią się, jak to widać na rys.~\ref{fig:6-1(a)}.
\begin{figure}[ht]
	\begin{center}
		\includegraphics{fig_6-1.a}
	\end{center}
	\caption{Porównanie kopców budowanych przez obie procedury dla tablicy $A=\langle1,2,3\rangle$. {\sffamily\bfseries(a)} Wynik działania \proc{Build-Max-Heap}. {\sffamily\bfseries(b)} Wynik działania \proc{Build-Max-Heap}$'$.} \label{fig:6-1(a)}
\end{figure}

\subproblem %6-1(b)
Najgorszym przypadkiem dla procedury \proc{Build-Max-Heap}$'$ jest tablica uporządkowana rosnąco. W~każdym z~$n-1$ wywołań \proc{Max-Heap-Insert} z~wiersza~3 nowy węzeł transportowany jest wówczas aż do korzenia kopca, co wymaga $\Theta(\lg i)$ operacji przy \singledash{$i$}{elementowym} kopcu. Stąd czas działania \proc{Build-Max-Heap}$'$ w~przypadku pesymistycznym wynosi
\[
	T(n) = \sum_{i=1}^{n-1}\Theta(\lg i) = \Theta(\lg(n!)) = \Theta(n\lg n).
\]

\problem{Analiza kopców rzędu $d$} %6-2

\subproblem %6-2(a)
\singledash{$d$}{kopiec} będziemy reprezentować w~tablicy w~następujący sposób. Podobnie jak w~reprezentacji tablicowej kopców binarnych tablica $A$ reprezentująca \singledash{$d$}{kopiec} będzie mieć atrybuty \attrib{A}{length} oraz \attrib{A}{heap-size}. Na pierwszej pozycji tablicy znajdzie się korzeń kopca, a~pozycje $2\twodots d+1$ będą zajmowane przez $d$ synów korzenia. Synowie pierwszego z~lewej syna korzenia zajmą pozycje $d+2\twodots2d+1$, synowie drugiego od lewej syna korzenia -- pozycje $2d+2\twodots3d+1$ itd. Ogólnie, mając dany indeks węzła $i$, można wyznaczyć indeks jego ojca, korzystając ze wzoru $\lceil(i-1)/d\rceil$. Uogólnienie procedury \proc{Parent} dla kopca rzędu~$d$ wygląda zatem następująco:
\begin{codebox}
\Procname{$\proc{Multiary-Parent}(d,i)$}
\zi	\Return $\lceil(i-1)/d\rceil$
\end{codebox}

Łatwo pokazać, że indeks \singledash{$k$}{tego} od lewej syna węzła o~indeksie $i$, gdzie $k=1$, 2,~\dots,~$d$, jest opisany wzorem $d(i-1)+k+1$. Poniższa procedura stanowi uogólnienie procedur \proc{Left} i~\proc{Right} dla kopca rzędu $d$ -- w~porównaniu do nich przyjmuje dodatkowy parametr $k$ oznaczający numer szukanego syna węzła $i$.
\begin{codebox}
\Procname{$\proc{Multiary-Child}(d,k,i)$}
\zi	\Return $d(i-1)+k+1$
\end{codebox}

Można sprawdzić, że zachodzi $\proc{Multiary-Parent}(d,\proc{Multiary-Child}(d,k,i))=i$ dla każdego $k=1$, 2,~\dots,~$d$.

\subproblem %6-2(b)
Uogólnimy rozumowanie z~\refExercise{6.1-1} na kopce rzędu $d$. Potraktujmy taki kopiec jak drzewo \singledash{$d$}{arne} o~wysokości $h$ i~$n$ węzłach. Na \singledash{$i$}{tym} poziomie tego drzewa, gdzie $i=0$, 1,~\dots,~$h-1$, znajduje się $d^i$ węzłów. Najniższy, \singledash{$h$}{ty} poziom, może zawierać od 1 do $d^h$ węzłów. Mamy zatem
\[
    \sum_{i=0}^{h-1}d^i+1 \le n \le \sum_{i=0}^hd^i.
\]
Na podstawie punktu~(c) problemu~\refProblem{3-1} sumę po lewej stronie można oszacować przez $\Theta(d^{h-1})$, a~sumę po prawej -- przez $\Theta(d^h)$. Oba te oszacowania dają w~wyniku $h=\Theta(\log_dn)$.

\subproblem %6-2(c)
Przedstawimy najpierw implementację procedury \proc{Max-Heapify} dla kopców rzędu $d$. Ogólny zarys jej działania pozostaje niezmieniony w~porównaniu z~oryginalną procedurą \proc{Max-Heapify}. Na każdym poziomie rekursji musimy wyznaczyć maksimum z~$d+1$ wartości -- bieżącego węzła i~jego $d$ synów. W~tym celu stosujemy pętlę przeglądającą wszystkich synów bieżącego węzła.
\begin{codebox}
\Procname{$\proc{Multiary-Max-Heapify}(A,d,i)$}
\li	$\id{largest}\gets i$
\li	$k\gets1$
\li	$\id{child}\gets\proc{Multiary-Child}(d,1,i)$
\li	\While $k\le d$ i~$\id{child}\le\attrib{A}{heap-size}$ \label{li:multiary-max-heapify-while-begin}
\li		\Do
			\If $A[\id{child}]>A[\id{largest}]$
\li				\Then $\id{largest}\gets\id{child}$
				\End
\li			$k\gets k+1$
\li			$\id{child}\gets\proc{Multiary-Child}(d,k,i)$ \label{li:multiary-max-heapify-child}
		\End \label{li:multiary-max-heapify-while-end}
\li	\If $\id{largest}\ne i$
\li		\Then
			zamień $A[i]\leftrightarrow A[\id{largest}]$
\li			$\proc{Multiary-Max-Heapify}(A,d,\id{largest})$
		\End
\end{codebox}
Zauważmy, że podczas wykonywania pętli \kw{while} w~wierszach \doubledash{\ref{li:multiary-max-heapify-while-begin}}{\ref{li:multiary-max-heapify-while-end}} zmienna $k$ może przyjąć wartość $d+1$ i~wówczas w~wierszu~\ref{li:multiary-max-heapify-child} zostaje wyznaczony indeks nieistniejącego, \singledash{$(d+1)$}{szego} syna węzła $i$. Jednak wartości tej nigdzie później nie wykorzystujemy, ponieważ następną operacją jest przerwanie pętli \kw{while}.

Na każdym poziomie rekursji (z~wyjątkiem być może ostatniego) wykonywanych jest $\Theta(d)$ operacji. Na mocy poprzedniego punktu mamy $\Theta(\log_dn)$ wywołań rekurencyjnych, a~zatem czasem działania powyższej procedury jest $\Theta(d\log_dn)$.

Procedura \proc{Multiary-Heap-Extract-Max}, która implementuje operację \proc{Extract-Max} dla \singledash{$d$}{kopca}, przyjmuje jako parametry kopiec $A$ oraz jego rząd $d$. Działa ona identyczne jak operacja \proc{Extract-Max} dla kopca binarnego, jednak w~wierszu~6 zamiast procedury \proc{Max-Heapify} wywołuje procedurę \proc{Multiary-Max-Heapify} przedstawioną powyżej. Czas działania tej operacji wynosi $\Theta(d\log_dn)$.

\subproblem %6-2(d)
Procedura \proc{Multiary-Max-Heap-Insert} implementująca operację \proc{Insert} dla \singledash{$d$}{kopca} przyjmuje na wejściu kopiec $A$, rząd kopca $d$ oraz wartość \id{key}, która będzie wstawiana do $A$. Jej działanie jest analogiczne do działania procedury wstawiania węzła do kopca binarnego. Jedyną różnicą jest wiersz~\ref{li:multiary-max-heap-insert-increase-key}, który zamiast \proc{Heap-Increase-Key} zawiera analogiczne wywołanie procedury \proc{Multiary-Heap-Increase-Key} zwiększającej wartość węzła w~kopcu rzędu $d$.
\begin{codebox}
\Procname{$\proc{Multiary-Max-Heap-Insert}(A,d,\id{key})$}
\li	$\attrib{A}{heap-size}\gets\attrib{A}{heap-size}+1$
\li	$A[\attrib{A}{heap-size}]\gets-\infty$
\li	$\proc{Multiary-Heap-Increase-Key}(A,d,\attrib{A}{heap-size},\id{key})$ \label{li:multiary-max-heap-insert-increase-key}
\end{codebox}

Czas działania operacji \proc{Insert} dla \singledash{$d$}{kopca} jest tego samego rzędu co czas działania wywołania z~wiersza~3. Implementacja wywoływanej procedury \proc{Multiary-Heap-Increase-Key} i~analiza jej czasu działania zostały opisane w~następnym punkcie.

\subproblem %6-2(e)
Implementacja tej operacji dla \singledash{$d$}{kopca} jest analogiczna do jej implementacji dla kopca binarnego. Jednak zamiast sprawdzania poprawności parametru $k$, do $A[i]$ przypisujemy natychmiast odpowiednią wartość.
\begin{codebox}
\Procname{$\proc{Multiary-Heap-Increase-Key}(A,d,i,k)$}
\li	$A[i]\gets\max(A[i],k)$ \label{li:multiary-heap-increase-key}
\li	\While $i>1$ i~$A[\proc{Multiary-Parent}(d,i)]<A[i]$
\li		\Do
			zamień $A[i]\leftrightarrow A[\proc{Multiary-Parent}(d,i)]$
\li			$i\gets\proc{Multiary-Parent}(d,i)$
		\End
\end{codebox}

Po wykonaniu wiersza~\ref{li:multiary-heap-increase-key} wartość węzła $i$ może być większa niż wartość jego ojca. W~najgorszym przypadku, jeśli węzeł $i$ jest liściem i~jego nowa wartość jest największą wartością w~kopcu, to zostanie on przetransportowany aż do korzenia, co zajmie czas proporcjonalny do wysokości kopca, czyli, na mocy punktu~(b), $\Theta(\log_dn)$.

\problem{Tablice Younga} %6-3

\subproblem %6-3(a)
Jedną z~tablic Younga zawierających podane elementy jest
\[
	\begin{pmatrix}
		2 & 3 & 14 & 16 \\
		4 & 8 & \infty & \infty \\
		5 & 12 & \infty & \infty \\
		9 & \infty & \infty & \infty
	\end{pmatrix}.
\]

\subproblem %6-3(b)
Załóżmy, że $Y[1,1]=\infty$ i~że tablica $Y$ nie jest pusta, tzn.\ $Y[i,j]\ne\infty$ dla pewnych $i$, $j$ takich, że $1\le i\le m$ oraz $1\le j\le n$. Ale z~własności tablicy Younga otrzymujemy, że $Y[1,1]\le Y[1,j]\le Y[i,j]$, co prowadzi do sprzeczności z~założeniem. A~więc tablica Younga $Y$, w~której $Y[1,1]=\infty$, jest pusta.

Dowód drugiej własności przebiega analogicznie. Przypuśćmy, że $Y[m,n]\ne\infty$ i~że tablica $Y$ nie jest pełna, tzn.\ $Y[i,j]=\infty$ dla pewnych $i$, $j$, gdzie $1\le i\le m$ oraz $1\le j\le n$. Wykorzystując własność tablicy Younga, dostajemy $Y[i,j]\le Y[i,n]\le Y[m,n]$, co jest sprzeczne z~założeniem. Tablica Younga $Y$, w~której $Y[m,n]\ne\infty$, jest pełna.

\subproblem %6-3(c)
Procedura ekstrakcji najmniejszego elementu tablicy Younga $Y$ o~rozmiarach $m\times n$ będzie opierać się o~pomysł z~\proc{Max-Heapify}. Najmniejszym elementem tablicy $Y$ jest $\mu=Y[1,1]$. Przetransportujemy go na ostatnią pozycję ostatniego wiersza tablicy, skąd będzie można bezpiecznie go usunąć przy jednoczesnym zachowaniu własności tablicy Younga. W~tym celu porównajmy $\mu$ z~elementem znajdującym się bezpośrednio na prawo i~elementem bezpośrednio w~dół od niego (o~ile istnieją). Mniejszy z~nich zamieniany jest następnie z~$\mu$, po czym procedura wywołuje się rekurencyjnie dla podtablicy Younga o~rozmiarach $(m-1)\times n$ albo $m\times(n-1)$, w~której $\mu$ stanowi pierwszy element pierwszej kolumny. Otrzymując w~wyniku tego postępowania tablicę o~rozmiarach $1\times1$, można usunąć jej jedyny element będący najmniejszym elementem początkowej tablicy Younga (zastępując go wartością $\infty$) i~zwrócić go jako wynik algorytmu.

Opisany sposób został zaimplementowany w~poniższym pseudokodzie. Aby pobrać minimum z~tablicy Younga $Y$ o~rozmiarach $m\times n$, należy wywołać $\proc{Young-Extract-Min}(Y,m,n,1,1)$.
\begin{codebox}
\Procname{$\proc{Young-Extract-Min}(Y,m,n,i,j)$}
\li	\If $\langle i,j\rangle=\langle m,n\rangle$
\li		\Then
			$\id{min}\gets Y[i,j]$
\li			$Y[i,j]\gets\infty$
\li			\Return \id{min}
		\End
\li	$\langle i',j'\rangle\gets\langle i,j+1\rangle$
\li	\If $i<m$
\li		\Then
			\If $j=n$ lub $Y[i+1,j]<Y[i,j+1]$
\li				\Then $\langle i',j'\rangle\gets\langle i+1,j\rangle$
				\End
		\End
\li	zamień $Y[i,j]\leftrightarrow Y[i',j']$
\li	\Return $\proc{Young-Extract-Min}(Y,m,n,i',j')$
\end{codebox}

Niech $T(p)$ będzie maksymalnym czasem działania powyższego algorytmu dla tablicy Younga $m\times n$, gdzie $p=m+n$ jest łączną liczbą jej kolumn i~wierszy. W~każdym wywołaniu rekurencyjnym zmniejszamy $p$ o~1, wykonując przy tym czas stały, skąd dostajemy
\[
	T(p) =
	\begin{cases}
		\Theta(1), & \text{jeśli $p=2$}, \\
		T(p-1)+\Theta(1), & \text{jeśli $p>2$}.
	\end{cases}
\]
Łatwo sprawdzić, że rozwiązaniem tej rekurencji jest $T(p)=O(p)=O(m+n)$.

\subproblem %6-3(d)
Podamy najpierw pomocniczą procedurę \proc{Youngify}, która działa analogicznie do \proc{Max-Heapify} i~ma na celu przywrócenie własności tablicy Younga $Y$ naruszoną przez $Y[i,j]$. Element ten wystarczy porównać z~jego sąsiadem znajdującym się powyżej lub sąsiadem znajdującym się po lewej stronie w~tablicy (o~ile istnieją). W~zależności od tego, który z~tych trzech elementów jest największy, dokonywana jest odpowiednia zamiana i~procedura wywoływana jest rekurencyjnie.
\begin{codebox}
\Procname{$\proc{Youngify}(Y,i,j)$}
\li	$\langle i',j'\rangle\gets\langle i,j\rangle$
\li	\If $i>1$ i~$Y[i-1,j]>Y[i',j']$
\li		\Then $\langle i',j'\rangle\gets\langle i-1,j\rangle$
		\End
\li	\If $j>1$ i~$Y[i,j-1]>Y[i',j']$
\li		\Then $\langle i',j'\rangle\gets\langle i,j-1\rangle$
		\End
\li	\If $\langle i',j'\rangle\ne\langle i,j\rangle$
\li		\Then
			zamień $Y[i,j]\leftrightarrow Y[i',j']$
\li			$\proc{Youngify}(Y,i',j')$
		\End
\end{codebox}

Ponieważ zakładamy, że tablica Younga $Y$ nie jest pełna, to na mocy punktu~(b) mamy $Y[m,n]=\infty$, czyli pozycja ta jest pusta i~można wstawić na nią nowy element. Wówczas jednak własność tablicy Younga może być naruszona, dlatego korzystamy z~procedury \proc{Youngify} w~celu przywrócenia tej własności.
\begin{codebox}
\Procname{$\proc{Young-Insert}(Y,m,n,\id{key})$}
\li	$Y[m,n]\gets\id{key}$
\li	$\proc{Youngify}(Y,m,n)$
\end{codebox}

Analiza poprawności i~czasu działania procedury \proc{Youngify} opiera się na analizie procedury \proc{Max-Heapify}. W~każdym kolejnym wywołaniu rekurencyjnym jedna z~liczb, $i$ lub $j$, jest mniejsza o~1. Koniec działania następuje w~najgorszym przypadku, gdy $i=j=1$, po wykonaniu $O(m+n)$ operacji. A~zatem czasem działania operacji \proc{Young-Insert} jest również $O(m+n)$.

\subproblem %6-3(e)
Niech $A$ będzie tablicą $n^2$ liczb, które należy posortować. Poniższy algorytm buduje tablicę Younga $n\times n$ z~liczb tablicy $A$, wykonując na każdej z~nich operację \proc{Young-Insert}. Następnie liczby te są pobierane w~kolejności niemalejącej dzięki $n^2$ wywołaniom \proc{Young-Extract-Min}.
\begin{codebox}
\Procname{$\proc{Young-Sort}(A)$}
\li	$n\gets\sqrt{\attrib{A}{length}}$
\li	\For $i\gets1$ \To $n^2$
\li		\Do $\proc{Young-Insert}(Y,n,n,A[i])$
		\End
\li	\For $i\gets1$ \To $n^2$
\li		\Do $A[i]\gets\proc{Young-Extract-Min}(Y,n,n,1,1)$
		\End
\end{codebox}

Czas działania obu wywoływanych procedur wynosi $O(n)$, zatem powyższy algorytm działa w~czasie $O(n^3)$. Jeśli mamy danych $m=n^2$ liczb, to jesteśmy w~stanie posortować je przy użyciu tego algorytmu w~czasie $O(m^{3/2})$. Jest to lepsza złożoność niż kwadratowa, ale gorsza od złożoności liniowo-logarytmicznej.

\subproblem %6-3(f)
Zbadajmy, jak szukana liczba $v$ ma się do ostatniego elementu pierwszego wiersza tablicy Younga $m\times n$. Jeśli wartości te są równe, to oczywiście można zakończyć poszukiwania z~rezultatem pozytywnym. W~przeciwnym przypadku, w~zależności od tego, która z~liczb jest większa, odrzucamy z~dalszych poszukiwań cały pierwszy wiersz lub całą ostatnią kolumnę i~kontynuujemy szukanie $v$ w~otrzymanej podtablicy, która stanowi tablicę Younga $(m-1)\times n$ albo $m\times(n-1)$. W~momencie uzyskania tablicy pustej wiadomo, że szukanej liczby nie ma w~początkowej tablicy.
\begin{codebox}
\Procname{$\proc{Young-Search}(Y,m,n,v)$}
\li	$i\gets1$
\li	$j\gets n$
\li	\While $i\le m$ i~$j\ge1$
\li		\Do
			\If $v=Y[i,j]$
\li				\Then \Return \const{true}
				\End
\li			\If $v>Y[i,j]$
\li				\Then $i\gets i+1$
\li				\Else $j\gets j-1$
				\End
		\End
\li	\Return \const{false}
\end{codebox}

W~każdym kroku pętli \kw{while} zmniejszamy o~1 liczbę kolumn lub liczbę wierszy rozważanej tablicy, wykonując przy tym stałą liczbę operacji -- jasne jest zatem, że czas działania algorytmu wynosi $O(m+n)$.

\endinput
\chapter{Quicksort -- sortowanie szybkie}

\subchapter{Opis algorytmu}

\exercise %7.1-1
\note{Rozwiązanie dotyczy przykładu z~tekstu oryginalnego.}

\noindent Rys.~\ref{fig:7.1-1} przedstawia działanie procedury \proc{Partition} dla tablicy~$A$.
\begin{figure}[ht]
	\begin{center}
		\includegraphics{fig07.1}
	\end{center}
	\caption{Działanie procedury \proc{Partition} dla tablicy $A=\langle13,19,9,5,12,8,7,4,21,2,6,11\rangle$. {\sffamily\bfseries(a)} Tablica wejściowa z~zaznaczonymi początkowymi wartościami zmiennych. {\sffamily\bfseries\twodashes{(b)}{(l)}} Kolejne iteracje pętli \kw{for} w~wierszach \twodashes{3}{6}. {\sffamily\bfseries(m)} Wynikowa tablica $A$ po wykonaniu zamiany z~wiersza~7.} \label{fig:7.1-1}
\end{figure}

\exercise %7.1-2
\note{Poprawną wartością dla\/ $q$ z~treści zadania powinno być\/ $\lfloor(p+r)/2\rfloor$.}

\noindent Zauważmy, że jeśli wszystkie elementy podtablicy $A[p\twodots r]$ mają taką samą wartość, to warunek z~wiersza~4 procedury \proc{Partition} jest spełniony w~każdej iteracji pętli \kw{for}. Oznacza to, że po wykonaniu tej pętli będzie $i=r-1$ i~procedura zwróci $q=r$.

Odpowiedniej modyfikacji procedury dokonujemy poprzez wprowadzenie licznika elementów równych elementowi rozdzielającemu w~badanej podtablicy. W~każdej iteracji pętli \kw{for} sprawdzamy dodatkowo, czy $A[j]=x$ i~jeśli tak, to licznik ten inkrementujemy. Jeśli na końcu procedury jego wartość jest równa $r-p+1$, czyli rozmiarowi podtablicy, to zwracamy $q=\lfloor(p+r)/2\rfloor$.

\exercise %7.1-3
Podczas przetwarzania podtablicy $A[p\twodots r]$ o~rozmiarze $n=r-p+1$ wykonywanych jest $n-1$ iteracji pętli \kw{for}, a~każda z~nich przeprowadza operacje zajmujące czas stały. Stąd czas działania procedury \proc{Partition} wynosi $\Theta(n)$.

\exercise %7.1-4
Wystarczy zamienić znak nierówności na przeciwny w~warunku z~wiersza~4 procedury \proc{Partition}.

\subchapter{Czas działania algorytmu quicksort}

\exercise %7.2-1
Niech $c_1$, $d_1>0$ będą stałymi. Korzystając z~założenia, że $T(n-1)\le c_1(n-1)^2$, dostajemy
\[
	T(n) = T(n-1)+\Theta(n) \le c_1(n-1)^2+d_1n = c_1n^2+(d_1-2c_1)n+c_1 \le c_1n^2.
\]
Ostatnia nierówność jest spełniona dla każdego $n\ge1$, jeśli np.\ $c_1=d_1=1$.

Weźmy teraz inne stałe $c_2$, $d_2>0$. Dolnego oszacowania na $T(n)$ dowodzimy analogicznie, wychodząc z~założenia, że $T(n-1)\ge c_2(n-1)^2$:
\[
	T(n) = T(n-1)+\Theta(n) \ge c_2(n-1)^2+d_2n = c_2n^2+(d_2-2c_2)n+c_2 \ge c_2n^2.
\]
Przyjmujemy $c_2=1/2$, $d_2=2$, dzięki czemu ostatnia nierówność zachodzi dla wszystkich $n\ge1$.

Przypadek brzegowy $n=1$ można przyjąć za podstawę obu powyższych indukcji dla podanych wartości stałych, co kończy dowód, że $T(n)=\Theta(n^2)$.

\exercise %7.2-2
Procedura \proc{Partition} w~takim przypadku zwraca wartość $q=r$ (\refExercise{7.1-2}), a~więc w~następnych wywołaniach rekurencyjnych procedury \proc{Quicksort} będą przetwarzane podtablice rozmiarów 0 i~$n-1$. Napotykając w~każdym wywołaniu rekurencyjnym przypadek pesymistyczny, algorytm będzie działać w~czasie opisanym przez rekurencję z~\refExercise{7.2-1}, której rozwiązaniem jest $\Theta(n^2)$.

\exercise %7.2-3
W~tym przypadku podczas każdego wywołania rekurencyjnego procedury \proc{Partition} elementem rozdzielającym jest najmniejszy element przetwarzanej podtablicy $A[p\twodots r]$. W~każdym wywołaniu jedyne elementy, które zostaną zamienione, to element rozdzielający i~$A[p]$, po czym zwrócona zostanie wartość $q=r$. Przypadek ten jest zatem analogiczny do rozważanego w~poprzednim zadaniu, a~więc czasem działania procedury \proc{Quicksort} dla tablicy posortowanej malejąco jest $\Theta(n^2)$.

\exercise %7.2-4
Prawie posortowany ciąg jest niemal najgorszym przypadkiem dla algorytmu quicksort. W~wielu wywołaniach rekurencyjnych na element rozdzielający wybierany jest bowiem największy element przetwarzanej podtablicy. Wykonywane są wówczas niezrównoważone podziały, przez co czas działania procedury \proc{Quicksort} zbliża się do kwadratowego.

Z~kolei dla sortowania przez wstawianie ciąg taki jest przypadkiem zbliżonym do optymistycznego, ponieważ czas działania procedury \proc{Insertion-Sort} jest tego samego rzędu, co ilość inwersji w~ciągu wejściowym (punkt~(c) problemu~\refProblem{2-4}). W~ciągu prawie posortowanym ilość inwersji jest niewielka, możemy zatem mówić o~co najwyżej liniowym czasie działania sortowania przez wstawianie dla tego przypadku.

\exercise %7.2-5
Rozważmy drzewo rekursji dla opisanego przypadku dokonywania podziałów. Najkrótsza gałąź w~tym drzewie składa się z~węzłów o~wartościach $\alpha^in$, gdzie $i$ jest poziomem węzła, zaś najdłuższa gałąź na \onedash{$i$}{tym} poziomie posiada węzeł o~wartości $(1-\alpha)^in$. Niech $h$ będzie głębokością liścia najkrótszej gałęzi. Mamy wtedy $\alpha^hn=1$, skąd
\[
	h = \log_\alpha\frac{1}{n} = -\frac{\lg n}{\lg\alpha}.
\]
Jeśli przez $H$ oznaczymy głębokość liścia na gałęzi najdłuższej, to mamy $(1-\alpha)^Hn=1$, a~zatem
\[
	H = \log_{1-\alpha}\frac{1}{n} = -\frac{\lg n}{\lg(1-\alpha)}.
\]

\exercise %7.2-6
Załóżmy, że procedura \proc{Partition} utworzyła podział w~stosunku $1-\beta$ do $\beta$, gdzie $0<\beta\le1/2$. Aby był to podział bardziej zrównoważony niż $1-\alpha$ do $\alpha$, to musi być $\alpha<\beta$. Zdarzenie to można modelować za pomocą ciągłego rozkładu jednostajnego dla przedziału $(\alpha,1/2]$ w~przestrzeni $(0,1/2]$. Wykorzystując definicję prawdopodobieństwa o~ciągłym rozkładzie jednostajnym, dostajemy
\[
	\Pr((\alpha,1/2]) = \frac{1/2-\alpha}{1/2-0} = 1-2\alpha.
\]

\subchapter{Randomizowana wersja algorytmu quicksort}

\exercise %7.3-1
Randomizacja algorytmu sprawia, że żadne jego dane wejściowe nie stanowią przypadku pesymistycznego. Można więc przyjąć, że dla każdych danych wejściowych algorytm zachowuje się jak w~przypadku średnim.

\exercise %7.3-2
W~przypadku pesymistycznym generator liczb losowych za każdym razem wybiera wartości, które tworzą najbardziej niezrównoważony podział podtablicy, czyli $p$ lub $r$. W~takiej sytuacji liczba wywołań generatora jest rzędu $\Theta(n)$.

W~przypadku optymistycznym generator za każdym razem zwraca liczbę bliską $(p+r)/2$. Tworzone podziały są zatem zrównoważone i~liczba wywołań generatora w~tym przypadku wynosi $\Theta(\lg n)$.

\subchapter{Analiza algorytmu quicksort}

\exercise %7.4-1
Niech $n\ge1$. Zgadujemy, że $T(q)\ge dq^2$ dla pewnej stałej $d>0$ i~wszystkich $q=0$, 1,~\dots,~$n-1$. Podstawiamy do wzoru na $T(n)$ i~na podstawie \refExercise{7.4-3} otrzymujemy:
\begin{align*}
	T(n) &\ge \max_{0\le q\le n-1}(dq^2+d(n-q-1)^2)+\Theta(n) \\
	&= d\cdot\!\!\!\max_{0\le q\le n-1}(q^2+(n-q-1)^2)+\Theta(n) \\
	&= dn^2-d(2n-1)+\Theta(n) \\
	&\ge dn^2.
\end{align*}
Ostatnia nierówność zostaje spełniona, jeśli przyjmiemy $d$ odpowiednio małe tak, aby składnik $\Theta(n)$ zdominował wyrażenie $d(2n-1)$. Przyjmijmy, że przypadkiem brzegowym rekurencji jest $T(0)=1$. Dla dowolnego $d$ zachodzi $T(0)\ge d\cdot0^2=0$, a~więc $n=0$ przyjmujemy na podstawę indukcji, co kończy dowód, że $T(n)=\Omega(n^2)$.

\exercise %7.4-2
Czas algorytmu quicksort w~przypadku optymistycznym jest opisany przez rekurencję
\[
    T(n) = \min_{0\le q\le n-1}(T(q)+T(n-q-1))+\Theta(n).
\]
Niech $c>0$ będzie stałą. Zgadujemy, że $T(q)\ge cq\lg q$ dla każdego $q=0$, 1,~\dots,~$n-1$ (przyjmujemy dla wygody, że $0\lg0=0$) i~podstawiamy:
\begin{align*}
    T(n) &\ge \min_{0\le q\le n-1}(cq\lg q+c(n-q-1)\lg(n-q-1))+\Theta(n) \\
	&= c\cdot\!\!\!\min_{0\le q\le n-1}(q\lg q+(n-q-1)\lg(n-q-1))+\Theta(n).
\end{align*}

Obliczymy teraz minimum wyrażenia $q\lg q+(n-q-1)\lg(n-q-1)$, gdy $0\le q\le n-1$. Jeśli $q=0$ lub $q=n-1$, to wyrażenie ma wartość $(n-1)\lg(n-1)$. W~pozostałych przypadkach potraktujmy je jako funkcję $f$ zmiennej $q$. Pochodne $f$ są postaci:
\begin{align*}
    \frac{df}{dq}(q) &= \frac{d}{dq}\biggl(\frac{q\ln q+(n-q-1)\ln(n-q-1)}{\ln2}\biggr) = \frac{\ln q-\ln(n-q-1)}{\ln2}, \\[1mm]
	\frac{d^2\!f}{dq^2}(q) &= \frac{d}{dq}\biggl(\frac{\ln q-\ln(n-q-1)}{\ln2}\biggr) = \frac{1}{\ln2}\biggl(\frac{1}{q}+\frac{1}{n-q-1}\biggr).
\end{align*}
Pierwsza pochodna zeruje się tylko wtedy, gdy $q=(n-1)/2$. Wyznaczamy drugą pochodną w~tym punkcie:
\[
    \frac{d^2\!f}{dq^2}\biggl(\frac{n-1}{2}\biggr) = \frac{1}{\ln2}\biggl(\frac{2}{n-1}+\frac{2}{2n-(n-1)-2}\biggr) = \frac{4}{\ln2\cdot(n-1)}.
\]
Liczba ta jest dodatnia, o~ile $n>1$. Mamy zatem, że minimum funkcji $f$ znajduje się w~punkcie $q=(n-1)/2$ i~wynosi $(n-1)\lg((n-1)/2)$. Wartość ta stanowi tym samym minimum szukanego wyrażenia.

Powracamy do oszacowania rekurencji $T(n)$, przyjmując $n\ge2$:
\begin{align*}
    T(n) &\ge c(n-1)\lg\frac{n-1}{2}+\Theta(n) \\
	&= c(n-1)\lg(n-1)-c(n-1)+\Theta(n) \\
	&= cn\lg(n-1)-c\lg(n-1)-c(n-1)+\Theta(n) \\
	&\ge cn\lg(n/2)-c\lg(n-1)-c(n-1)+\Theta(n) \\
	&= cn\lg n-c(2n+\lg(n-1)-1)+\Theta(n) \\
	&\ge cn\lg n.
\end{align*}
Ostatnia nierówność zachodzi, o~ile stała $c$ jest na tyle mała, że wyrażenie $c(2n+\lg(n-1)-1)$ jest zdominowane przez składnik $\Theta(n)$. Przyjmijmy, że przypadkiem brzegowym rekurencji jest $T(1)=1$. Na podstawę indukcji możemy więc przyjąć $n=1$, dla którego $T(n)$ spełnia rozważane oszacowanie, a~zatem $T(n)=\Omega(n\lg n)$.

\exercise %7.4-3
Potraktujmy wyrażenie jako funkcję $f(q)=q^2+(n-q-1)^2$, gdzie $0\le q\le n-1$. W~celu znalezienia maksimum globalnego tej funkcji obliczmy jej pierwszą i~drugą pochodną:
\begin{align*}
    \frac{df}{dq}(q) &= 2q-2(n-q-1), \\
	\frac{d^2\!f}{dq^2}(q) &= 4.
\end{align*}
Ponieważ druga pochodna jest dodatnia, to funkcja $f$ nie posiada maksimum lokalnego i~jej największej wartości należy szukać w~punktach brzegowych dziedziny. Mamy $f(0)=f(n-1)=(n-1)^2$, więc maksimum jest osiągane w~obu tych punktach.

\exercise %7.4-4
Przy wyznaczaniu dolnego oszacowania na oczekiwany czas działania algorytmu quicksort korzystamy z~analizy przedstawionej w~Podręczniku dla górnego oszacowania. Zauważmy, że lemat~7.1 pozostaje prawdziwy, gdyby zamiast notacji $O$ zastosować $\Omega$. Prowadząc rozumowanie analogicznie, dochodzimy w~końcu do wartości oczekiwanej zmiennej losowej $X$, którą następnie ograniczamy od dołu:
\[
	\E(X) = \sum_{i=1}^{n-1}\sum_{k=1}^{n-i}\frac{2}{k+1} \ge \sum_{i=1}^{n-1}\sum_{k=1}^{n-i}\frac{2}{2k} = \sum_{i=1}^{n-1}H_{n-i} = \sum_{i=1}^{n-1}H_i = \sum_{i=1}^{n-1}\Omega(\lg i) = \Omega(n\lg n).
\]
Skorzystaliśmy tutaj z~\refExercise{A.2-3} oraz z~punktu~(b) problemu~\refProblem{A-1} dla $s=1$, skąd otrzymaliśmy ostatnie dwie równości.

\exercise %7.4-5
W~rozważanej modyfikacji drzewo rekursji w~algorytmie quicksort ma około $\lg(n/k)$ poziomów, z~których każdy wnosi koszt $O(n)$. W~przypadku średnim czas wykonania tego kroku wynosi $O(n\lg(n/k))$. Liczba fragmentów o~mniej niż $k$ elementach, których nie uporządkowano w~pierwszej fazie, jest rzędu $O(n/k)$. Drugi krok polega na posortowaniu ich przez wstawianie i~zajmuje czas $O(n/k\cdot k^2)=O(nk)$. Stąd całkowitym oczekiwanym czasem działania algorytmu jest $O(nk+n\lg(n/k))$.

Teoretycznie parametr $k$ powinien być rzędu co najwyżej $O(\lg n)$ -- wtedy złożoność czasowa tego algorytmu nie przewyższa złożoności czasowej sortowania szybkiego. W~praktyce jednak $k$ powinno być tak dobrane, aby sortowanie przez wstawianie tablicy o~długości $k$ było wykonywane szybciej od sortowania takiej tablicy algorytmem quicksort.

\exercise %7.4-6
Niech $P$ będzie szukaną funkcją zmiennej $0<\alpha<1$. Zauważmy, że funkcja ta spełnia własność $P(\alpha)=P(1-\alpha)$, w~rozwiązaniu będziemy więc zakładać, że $\alpha\le1/2$. Niech $i<j<k$ będą indeksami elementów losowo wybranych z~$A$. Utworzenie w~najgorszym przypadku podziału $\alpha$ do $1-\alpha$ jest równoważne temu, że $\alpha n\le j\le(1-\alpha)n$. Możliwe są następujące przypadki ze względu na wartości przyjmowane przez pozostałe indeksy:
\begin{enumerate}
	\renewcommand{\labelenumi}{(\roman{enumi})}
	\item $i<\alpha n$, $k>(1-\alpha)n$, co zachodzi z~prawdopodobieństwem około $6\alpha^2(1-2\alpha)$;
	\item $i<\alpha n$, $\alpha n\le k\le(1-\alpha)n$, co zachodzi z~prawdopodobieństwem około $3\alpha(1-2\alpha)^2$;
	\item $\alpha n\le i\le(1-\alpha)n$, $k>(1-\alpha)n$, co zachodzi z~prawdopodobieństwem około $3\alpha(1-2\alpha)^2$;
	\item $\alpha n\le i\le(1-\alpha)n$, $\alpha n\le k\le(1-\alpha)n$, co zachodzi z~prawdopodobieństwem około $(1-2\alpha)^3$.
\end{enumerate}
Sumując prawdopodobieństwa z~wszystkich przypadków, otrzymujemy $P(\alpha)\approx4\alpha^3-6\alpha^2+1$.

\problems

\problem{Poprawność algorytmu podziału Hoare'a} %7-1

\subproblem %7-1(a)
Działanie procedury \proc{Hoare-Partition} dla przykładowej tablicy $A$ zostało przedstawione na rys.~\ref{fig:7-1a}.
\begin{figure}[ht]
	\begin{center}
		\includegraphics{fig07.2}
	\end{center}
	\caption{Działanie procedury \proc{Hoare-Partition} dla tablicy $A=\langle13,19,9,5,12,8,7,4,11,2,6,21\rangle$. Wszystkie elementy jasnoszare stanowią obszar złożony z~wartości nie większych niż $x$. Ciemnoszare elementy tworzą obszar złożony z~wartości nie mniejszych niż $x$. {\sffamily\bfseries(a)} Wejściowa tablica wraz z~początkowym ustawieniem zmiennych. {\sffamily\bfseries\twodashes{(b)}{(d)}} Tablica i~bieżące wartości zmiennych po wykonaniu, odpowiednio, jednej, dwóch i~trzech iteracji pętli \kw{while} w~wierszach \twodashes{4}{11}.} \label{fig:7-1a}
\end{figure}

\subproblem %7-1(b)
W~pierwszej iteracji pętli \kw{while} zmienna $j$ zatrzyma się na pewnym indeksie $q\ge p$, a~zmienna $i$ pozostanie na indeksie $p$, pod którym przechowywana jest wartość $x$. Jeśli $i=j$, to procedura kończy działanie, załóżmy więc, że $i<j$. Element $A[i]=A[p]=x$ zostanie zamieniony z~$A[q]$. W~kolejnych iteracjach pętli \kw{while} zmienna $i$ może dotrzeć najdalej na indeks $q$, natomiast $j$ nie będzie nigdy mniejsze niż $p$, bo $A[p]$ zawiera teraz wartość mniejszą bądź równą $x$. Ponieważ $q>p$, to w~pewnym momencie podczas działania pętli indeksy $i$ i~$j$ miną się, czyli będzie $i\ge j$, co spowoduje przerwanie pętli w~wierszu~11.

\subproblem %7-1(c)
\note{Zakładamy, że podtablica\/ $A[p\twodots r]$ składa się z~co najmniej dwóch elementów.}

\noindent Fakt udowodniony w~punkcie~(b) mówi o~tym, że $p\le j\le r$. Załóżmy, że $A[r]\le x$, bowiem w~przeciwnym przypadku $j<r$ już po pierwszej iteracji pętli \kw{while}. W~pierwszej iteracji element $A[p]$ jest zamieniany z~$A[r]$, po czym w~kolejnych iteracjach indeks $j$ jest zmniejszany i~również w~tym przypadku mamy $j<r$.

\subproblem %7-1(d)
Wynika to z~warunków stopu obu pętli \kw{repeat} i~testu z~wiersza~9. Każda para elementów tablicy, która tworzyła inwersję, jest odwracana, dzięki czemu elementy, które naruszały warunek posortowania, po zamianie będą znajdować się w~odpowiednich obszarach tablicy. Jak tylko indeksy $i$ i~$j$ miną się, każdy element z~podtablicy $A[p\twodots j]$ będzie równy bądź mniejszy od każdego elementu z~$A[j+1\twodots r]$.

\subproblem %7-1(e)
Jedyną różnicą w~porównaniu z~procedurą \proc{Quicksort} jest to, że element rozdzielający nie znajduje się w~$A[q]$ po wykonaniu \proc{Hoare-Partition}, musimy więc sortować rekurencyjnie fragment $A[p\twodots q]$ zamiast $A[p\twodots q-1]$. Pseudokod procedury został przedstawiony poniżej.
\begin{codebox}
\Procname{$\proc{Hoare-Quicksort}(A,p,r)$}
\li	\If $p<r$
\li		\Then
			$q\gets\proc{Hoare-Partition}(A,p,r)$
\li			$\proc{Hoare-Quicksort}(A,p,q)$
\li			$\proc{Hoare-Quicksort}(A,q+1,r)$
		\End
\end{codebox}

\problem{Alternatywna analiza algorytmu quicksort} %7-2

\subproblem %7-2(a)
W~procedurze \proc{Randomized-Partition} element $A[r]$ jest zamieniany z~elementem losowo wybranym z~tablicy $A$ o~rozmiarze $n$. Stąd szanse, że pewien ustalony element tablicy $A$ zostanie elementem rozdzielającym, wynoszą $1/n$. Wobec tego wartość oczekiwana zmiennej $X_i$, gdzie $i=1$, 2,~\dots,~$n$, wynosi $\E(X_i)=\Pr(X_i=1)=1/n$.

\subproblem %7-2(b)
Wywołanie procedury \proc{Randomized-Quicksort} dla tablicy wejściowej $A$ o~rozmiarze $n$ potrzebuje czasu $\Theta(n)$ na podział tablicy. Zostaje wyznaczony pewien indeks $1\le q\le n$, po czym procedura jest wywoływana rekurencyjnie dla podtablic rozmiarów $q-1$ i~$n-q$. Zmienna losowa $X_i$ zdefiniowana w~punkcie~(a) przyjmuje wartość 1, gdy $i=q$, a~w~pozostałych przypadkach przyjmuje wartość 0. Stąd czas potrzebny na posortowanie \onedash{$n$}{elementowej} tablicy wyraża się wzorem
\[
	T(n) = T(q-1)+T(n-q)+\Theta(n) = \sum_{i=1}^nX_i(T(i-1)+T(n-i)+\Theta(n)).
\]
Biorąc wartości oczekiwane skrajnych wyrażeń, otrzymujemy wzór~(7.5) na oczekiwany czas działania algorytmu quicksort.

\subproblem %7-2(c)
\note{Przyjmiemy, że we wzorze~(7.6) sumowanie przebiega od\/ $q=2$ do\/ $q=n-1$.}

\noindent Wykorzystując części~(a) i~(b) oraz liniowość wartości oczekiwanej, otrzymujemy
\begin{align*}
	\E(T(n)) &= \sum_{q=1}^n\E\bigl(X_q(T(q-1)+T(n-q)+\Theta(n))\bigr) \\
	&= \sum_{q=1}^n\E(X_q)\bigl(\E(T(q-1))+\E(T(n-q))+\E(\Theta(n))\bigr) \\
	&= \sum_{q=1}^n\frac{1}{n}\bigl(\E(T(q-1))+\E(T(n-q))+\Theta(n)\bigr) \\
	&= \frac{1}{n}\sum_{q=0}^{n-1}2\E(T(q))+\frac{1}{n}\sum_{q=1}^n\Theta(n) \\
	&= \frac{2}{n}\sum_{q=2}^{n-1}\E(T(q))+\frac{2}{n}\bigl(\E(T(0))+\E(T(1))\bigr)+\Theta(n) \\
	&= \frac{2}{n}\sum_{q=2}^{n-1}\E(T(q))+\Theta(n).
\end{align*}
Skorzystaliśmy z~faktu, że $\E(T(0))$ i~$\E(T(1))$ są stałymi i~mogą zostać wchłonięte przez $\Theta(n)$.

\subproblem %7-2(d)
Rozdzielamy sumę na dwie części:
\[
	\sum_{k=1}^{n-1}k\lg k = \sum_{k=1}^{\lceil n/2\rceil-1}k\lg k+\sum_{k=\lceil n/2\rceil}^{n-1}k\lg k,
\]
po czym zauważamy, że czynnik $\lg k$ w~pierwszej sumie po prawej stronie znaku równości możemy ograniczyć z~góry przez $\lg(n/2)=\lg n-1$, a~czynnik $\lg k$ w~drugiej sumie -- przez $\lg n$. Stąd
\begin{align*}
	\sum_{k=1}^{n-1}k\lg k &\le (\lg n-1)\sum_{k=1}^{\lceil n/2\rceil-1}k+\lg n\sum_{k=\lceil n/2\rceil}^{n-1}k \\
	&= \lg n\sum_{k=1}^{n-1}k-\sum_{k=1}^{\lceil n/2\rceil-1}k \\[2mm]
	&\le \frac{n(n-1)\lg n}{2}-\frac{n}{4}\biggl(\frac{n}{2}-1\biggr) \\
	&= \frac{n^2\lg n}{2}-\frac{n\lg n}{2}-\frac{n^2}{8}+\frac{n}{4} \\[1mm]
	&\le \frac{n^2\lg n}{2}-\frac{n^2}{8},
\end{align*}
przy czym ostatnia nierówność zachodzi, gdy $n\ge\sqrt{2}$.

\subproblem %7-2(e)
Załóżmy, że $n>2$ i~przyjmijmy założenie indukcyjne $\E(T(q))\le aq\lg q$ dla każdego $q=2$, 3,~\dots,~$n-1$ oraz pewnej stałej $a>0$. Z~punktu~(c) mamy
\[
	\E(T(n)) = \frac{2}{n}\sum_{q=2}^{n-1}\E(T(q))+\Theta(n) \le \frac{2}{n}\sum_{q=2}^{n-1}aq\lg q+\Theta(n) = \frac{2a}{n}\sum_{q=1}^{n-1}q\lg q+\Theta(n).
\]
Wykorzystując teraz wynik z~punktu~(d), dostajemy
\[
	\E(T(n)) \le \frac{2a}{n}\biggl(\frac{n^2\lg n}{2}-\frac{n^2}{8}\biggr)+\Theta(n) = an\lg n-\frac{an}{4}+\Theta(n) \le an\lg n,
\]
gdyż możemy wybrać $a$ wystarczająco duże, by wyrażenie $an/4$ dominowało nad składnikiem $\Theta(n)$.

Jeśli przyjmiemy, że $\E(T(2))=1$, to wówczas będziemy mieć $1\le a\cdot2\cdot\lg2=2a$, skąd $a\ge1/2$. A~zatem można przyjąć $n=2$ na podstawę indukcji i~dowód faktu, że $\E(T(n))=O(n\lg n)$ jest zakończony. Łącząc ten wynik z~oszacowaniem $\Omega(n\lg n)$ dla przypadku optymistycznego, które zostało wyznaczone w~\refExercise{7.4-2}, dostajemy, że oczekiwanym czasem działania algorytmu quicksort jest $\Theta(n\lg n)$.

\problem{Nieefektywne sortowanie} %7-3

\subproblem %7-3(a)
Udowodnimy poprawność algorytmu przez indukcję względem rozmiaru tablicy $n$. Łatwo sprawdzić, że algorytm działa poprawnie w~przypadku, gdy $n\le2$. Załóżmy więc, że $n>2$ i~że poprawnie sortowane są tablice o~rozmiarach mniejszych niż $n$. W~wyniku wykonania wiersza~6 na pozycjach $\lfloor n/3\rfloor+1\twodots n-\lfloor n/3\rfloor$ znajdują się elementy nie mniejsze od tych z~pozycji $1\twodots\lfloor n/3\rfloor$. A~zatem, po wykonaniu wiersza~7, $\lfloor n/3\rfloor$ największych elementów tablicy $A$ znajduje się w~obszarze $A[n-\lfloor n/3\rfloor+1\twodots n]$. Aby zakończyć sortowanie tablicy $A$, wystarczy uporządkować fragment $A[1\twodots n-\lfloor n/3\rfloor]$, co realizuje wiersz~8.

\subproblem %7-3(b)
Procedura jest wywoływana rekurencyjnie 3~razy, zaś każde wywołanie otrzymuje tablicę o~rozmiarze około $2/3$ rozmiaru oryginalnej tablicy. Ponadto praca poza wywołaniami rekurencyjnymi jest wykonywana w~czasie stałym. Stąd dostajemy równanie rekurencyjne
\[
	T(n) =
	\begin{cases}
		\Theta(1), & \text{jeśli $n\le2$}, \\
		3T(2n/3)+\Theta(1), & \text{jeśli $n>2$},
	\end{cases}
\]
które rozwiązujemy przy użyciu twierdzenia o~rekurencji uniwersalnej, otrzymując wynik $T(n)=\Theta(n^{\log_{3/2}3})\approx \Theta(n^{2{,}71})$.

\subproblem %7-3(c)
Pesymistyczny czas działania algorytmu \proc{Stooge-Sort} jest wyższego rzędu nie tylko od pesymistycznego czasu działania algorytmów sortowania przez scalanie, kopcowanie czy sortowania szybkiego, ale również od mniej efektywnego sortowania przez wstawianie. Metoda sortowania profesorów jest więc poprawna, ale bardzo powolna.

\problem{Głębokość stosu w~algorytmie quicksort} %7-4

\subproblem %7-4(a)
\proc{Quicksort}$'$ wykonuje te same operacje na tablicy $A$ co oryginalny algorytm quicksort. Różnica jest tylko w~przetwarzaniu podtablicy $A[q+1\twodots r]$. Zamiast drugiego wywołania rekurencyjnego procedura wykonuje przypisanie w~wierszu~5, po czym następuje kolejna iteracja pętli \kw{while}, co działa identycznie jak wywołanie $\proc{Quicksort}'(A,q+1,r)$, ale bez zwiększania stosu rekurencji. Poprawność algorytmu wynika zatem z~poprawności oryginalnego algorytmu quicksort.

\subproblem %7-4(b)
Stos rekurencji może urosnąć do rozmiaru $\Theta(n)$ w~sytuacji, gdy będzie $\Theta(n)$ wywołań rekurencyjnych \proc{Quicksort}$'$. Dzieje się tak wtedy, gdy na każdym poziomie rekursji procedura \proc{Partition} zwraca $q=r$. Do wywołania rekurencyjnego przekazywana jest wtedy podtablica o~1 mniejsza w~porównaniu z~początkową tablicą. Algorytm zachowuje się w~ten sposób, jeśli na wejście dostanie tablicę posortowaną niemalejąco.

\subproblem %7-4(c)
Wykorzystamy pomysł, aby wywoływać procedurę rekurencyjnie dla mniejszej podtablicy, natomiast większą przetwarzać w~bieżącym wywołaniu. Dzięki temu na kolejnym poziomie rekursji rozważany będzie problem mniejszy co najmniej o~połowę, więc w~najgorszym przypadku głębokość stosu rekurencji wyniesie $\Theta(\lg n)$. Tworzone podziały będą takie same jak przed dokonaniem usprawnienia, zatem oczekiwany czas działania algorytmu nie zmieni się. Zmodyfikowany kod procedury \proc{Quicksort}$'$ został przedstawiony poniżej.
\begin{codebox}
\Procname{$\proc{Quicksort}''(A,p,r)$}
\li	\While $p<r$
\li		\Do
			\Comment Dziel i~sortuj mniejszą podtablicę.
\li			$q\gets\proc{Partition}(A,p,r)$
\li			\If $q-p<r-q$
\li				\Then
					$\proc{Quicksort}''(A,p,q-1)$
\li					$p\gets q+1$
\li				\Else
					$\proc{Quicksort}''(A,q+1,r)$
\li					$r\gets q-1$
				\End
		\End
\end{codebox}

\problem{Podział względem mediany trzech wartości} %7-5

\subproblem %7-5(a)
Element rozdzielający $x$ znajdzie się na pozycji $i$ w~tablicy $A'$, jeżeli drugi z~wybranych elementów będzie na lewo od $i$ w~tej tablicy, a~trzeci wybrany element -- na prawo od $i$. Liczby możliwych pozycji, jakie mogą zająć drugi i~trzeci element, wynoszą, odpowiednio, $i-1$ i~$n-i$. Otrzymujemy zatem
\[
	p_i = \frac{(i-1)(n-i)}{\binom{n}{3}}.
\]

\subproblem %7-5(b)
W~zwykłej implementacji szanse wyboru mediany tablicy $A[1\twodots n]$ na element rozdzielający są równe $1/n$ (z~części~(a) problemu~\refProblem{7-2}), natomiast w~metodzie mediany trzech wartości wynoszą one $p_{\lfloor(n+1)/2\rfloor}$. Jeśli $n$ jest parzyste, to
\[
	p_{\lfloor(n+1)/2\rfloor} = p_{n/2} = \frac{\bigl(\frac{n}{2}-1\bigr)\bigl(n-\frac{n}{2}\bigr)}{\binom{n}{3}} = \frac{3}{2(n-1)}.
\]
Stosunek prawdopodobieństw z~obu metod dąży do
\[
    \lim_{n\to\infty}\frac{p_{\lfloor(n+1)/2\rfloor}}{1/n} = \lim_{n\to\infty}\frac{3n}{2(n-2)} = \frac{3}{2}.
\]
Dla $n$ nieparzystego mamy
\[
	p_{\lfloor(n+1)/2\rfloor} = p_{(n+1)/2} = \frac{\bigl(\frac{n+1}{2}-1\bigr)\bigl(n-\frac{n+1}{2}\bigr)}{\binom{n}{3}} = \frac{3(n-1)}{2n(n-2)},
\]
więc granica stosunku wynosi
\[
    \lim_{n\to\infty}\frac{p_{\lfloor(n+1)/2\rfloor}}{1/n} = \lim_{n\to\infty}\frac{3(n-1)}{2(n-2)} = \frac{3}{2}.
\]
A~zatem niezależnie od parzystości $n$ szanse na to, że mediana tablicy $A[1\twodots n]$ zostanie wybrana na element rozdzielający, są większe o~50\% w~metodzie mediany trzech wartości dla dostatecznie dużych $n$.

\subproblem %7-5(c)
W~tradycyjnym podejściu szansa na uzyskanie dobrego podziału jest bliska $1/3$. Szanse na dobry podział w~metodzie mediany trzech wartości wynoszą
\[
	\sum_{i=\lceil n/3\rceil}^{\lfloor 2n/3\rfloor}p_i \approx 1-2\sum_{i=1}^{n/3}\frac{(i-1)(n-i)}{\binom{n}{3}} = 1-\frac{12}{n(n-1)(n-2)}\sum_{i=1}^{n/3}(i-1)(n-i).
\]
Ostatnią sumę przybliżamy za pomocą całki $\int_1^{n/3}(x-1)(n-x)\,dx$, podstawiając $t=x-1$:
\begin{align*}
    \int_0^{n/3-1}t(n-t-1)\,dt &= \biggl[\frac{(n-1)t^2}{2}-\frac{t^3}{3}\biggr]_0^{n/3-1} \\[1mm]
	&= \frac{(n-1)(n/3-1)^2}{2}-\frac{(n/3-1)^3}{3} \\[1mm]
	&= \frac{(n-1)(n-3)^2}{18}-\frac{(n-3)^3}{81}.
\end{align*}
Wracając do oszacowania prawdopodobieństwa uzyskania dobrego podziału, otrzymujemy
\[
    \sum_{i=\lceil n/3\rceil}^{\lfloor 2n/3\rfloor}p_i \approx 1-\frac{2(n-3)^2}{3n(n-2)}+\frac{4(n-3)^3}{27n(n-1)(n-2)}.
\]

Dzięki zastosowaniu nowej strategii wyboru elementu rozdzielającego szanse na utworzenie dobrego podziału wzrastają o~czynnik
\[
	\lim_{n\to\infty}\frac{\sum_{i=\lceil n/3\rceil}^{\lfloor 2n/3\rfloor}p_i}{1/3} \approx \frac{1-2/3+4/27}{1/3} = \frac{13}{9}.
\]

\subproblem %7-5(d)
Nowy sposób wyboru elementu rozdzielającego zwiększa jedynie szanse na uzyskanie podziału zrównoważonego, co z~kolei obniża prawdopodobieństwo, że algorytm quicksort będzie działał w~czasie kwadratowym. Jednakże dolne oszacowanie na czas działania algorytmu pozostaje bez zmian i~wynosi nadal $\Omega(n\lg n)$ -- można sobie wyobrazić sytuację, w~której oryginalny sposób wyboru elementu rozdzielającego generuje za każdym razem najbardziej zrównoważony podział.

\problem{Rozmyte sortowanie przedziałów} %7-6

\subproblem %7-6(a)
Niech $A$ będzie tablicą wejściową, przy czym $A[i]=[a_i,b_i]$ dla $i=1$, 2,~\dots,~$n$. Zauważmy, że jeśli $[a_i,b_i]\cap[a_j,b_j]\ne\emptyset$, czyli przedziały $A[i]$ i~$A[j]$ nachodzą na siebie, to mogą wystąpić w~tablicy wynikowej w~dowolnej kolejności. Algorytm działa podobnie jak quicksort, ale wykorzystuje tę obserwację, znajdując zbiór przedziałów nachodzących na przedział stanowiący element rozdzielający i~pomijając wywołanie rekurencyjne dla podtablicy utworzonej przez ten zbiór przedziałów.

Procedura \proc{Fuzzy-Sort} implementuje rozmyte sortowanie przedziałów. Aby posortować całą tablicę $A$, należy wywołać $\proc{Fuzzy-Sort}(A,1,\id{length}[A])$.
\begin{codebox}
\Procname{$\proc{Fuzzy-Sort}(A,p,r)$}
\li	\If $p<r$
\li		\Then
			$\langle q_1,q_2\rangle\gets\proc{Fuzzy-Partition}(A,p,r)$
\li			$\proc{Fuzzy-Sort}(A,p,q_1-1)$ \label{li:fuzzy-sort-recursion1}
\li			$\proc{Fuzzy-Sort}(A,q_2+1,r)$ \label{li:fuzzy-sort-recursion2}
		\End
\end{codebox}

Pomocnicza procedura \proc{Fuzzy-Partition} dokonuje podziału tablicy $A[p\twodots r]$ na 3 podtablice: $A[q_1\twodots q_2]$, która nie musi być dalej sortowana, oraz $A[p\twodots q_1-1]$ i~$A[q_2+1\twodots r]$, które następnie sortowane są w~wywołaniach rekurencyjnych w~wierszach~\ref{li:fuzzy-sort-recursion1} i~\ref{li:fuzzy-sort-recursion2}. Pseudokod tej procedury pomocniczej został przedstawiony poniżej.
\begin{codebox}
\Procname{$\proc{Fuzzy-Partition}(A,p,r)$}
\li	zamień $A[r]\leftrightarrow A[\proc{Random}(p,r)]$
\li	$x\gets a_r$
\li $i\gets p-1$
\li	\For $j\gets p$ \To $r-1$
\li		\Do
			\If $a_j\le x$
\li				\Then
					$i\gets i+1$
\li					zamień $A[i]\leftrightarrow A[j]$
				\End
		\End
\li	zamień $A[i+1]\leftrightarrow A[r]$ \label{li:fuzzy-partition-swap}
\li	$q\gets i+1$ \label{li:fuzzy-partition-q-init}
\li	\For $k\gets i$ \Downto $p$
\li		\Do
			\If $b_k\ge x$
\li				\Then
					$q\gets q-1$
\li					zamień $A[q]\leftrightarrow A[k]$
				\End
		\End \label{li:fuzzy-partition-for-end}
\li	\Return $\langle q,i+1\rangle$
\end{codebox}
Lewy koniec przedziału stanowiącego element rozdzielający, wybrany losowo spośród wszystkich elementów tablicy wejściowej, zostaje przypisany do zmiennej $x$. Po wykonaniu wiersza~\ref{li:fuzzy-partition-swap} tablica $A$ jest podzielona na dwie podtablice według lewych końców przedziałów względem $x$. Ta część jest więc analogiczna do zwykłej procedury \proc{Partition}, w~wyniku czego dostajemy dwa obszary tablicy rozdzielone elementem $x$. Następnie, w~wierszach \twodashes{\ref{li:fuzzy-partition-q-init}}{\ref{li:fuzzy-partition-for-end}}, wszystkie przedziały z~podtablicy $A[p\twodots i]$, które nachodzą na element rozdzielający znajdujący się teraz w~$A[i+1]$, zostają przeniesione na koniec tej podtablicy. W~rezultacie z~przedziałów tych utworzony zostaje trzeci obszar tablicy, niewymagający dalszego sortowania. Na końcu zwracane są indeksy początku i~końca tego obszaru.

\subproblem %7-6(b)
Algorytm został oparty o~randomizowaną wersję quicksorta, więc jego czas działania dla tablicy \onedash{$n$}{elementowej} w~przypadku średnim wynosi $\Theta(n\lg n)$. Jeśli jednak wszystkie przedziały na siebie nachodzą, to lewa część podtablicy podzielonej w~wyniku działania procedury \proc{Fuzzy-Partition} będzie za każdym razem pusta. Na~każdym poziomie rekursji będzie więc sortowany tylko jeden obszar. Randomizacja zapewnia, że oczekiwaną pozycją elementu rozdzielającego jest środek podtablicy $A[p\twodots r]$ (patrz \refExercise{C.3-2}) i~w~kolejnym wywołaniu rekurencyjnym sortowany fragment jest o~połowę mniejszy. Oczekiwany czas działania algorytmu w~tym przypadku jest więc opisany przez rekurencję $T(n)=T(n/2)+\Theta(n)$, której rozwiązaniem jest $\Theta(n)$.

\endinput
\chapter{Sortowanie w~czasie liniowym}

\subchapter{Dolne ograniczenia dla problemu sortowania}

\exercise %8.1-1
Do liścia o~najmniejszej głębokości w~drzewie można dotrzeć, poruszając się po węzłach o~etykietach $1:2$, $2:3$, \dots, $(n-1):n$, czyli testując każde dwa kolejne elementy w~ciągu o~długości $n$. Najmniejszą głębokością liścia jest zatem $n-1$.

\exercise %8.1-2
Ogólna postać tej sumy została wyznaczona w~problemie~A-1(b) przy wykorzystaniu ograniczenia jej za pomocą całek. Po przyjęciu $s=1$ otrzymujemy rozwiązanie $\sum_{k=1}^n\lg k=\Theta(n\lg n)$.

\exercise %8.1-3
Jeśli sortowanie działa w~czasie liniowym dla $m$ permutacji wejściowych, to wysokość $h$ drzewa złożonego tylko z~liści odpowiadających tym permutacjom i~ich przodków jest liniowa.

Powtarzając rozumowanie przedstawione w~tw.~8.1, dostajemy nierówność $2^h\ge m$, a~stąd $h\ge\lg m$. Pozostaje więc sprawdzić jakiego rzędu jest $h$ dla poszczególnych wartości przyjmowanych przez $m$:
\begin{align*}
	m = n!/2: &\qquad \lg m = \lg(n!/2) = \lg(n!)-1 = \Omega(n\lg n), \\
	m = n!/n: &\qquad \lg m = \lg(n!/n) = \lg(n!)-\lg n = \Omega(n\lg n), \\
	m = n!/2^n: &\qquad \lg m = \lg(n!/2^n) = \lg(n!)-n = \Omega(n\lg n).
\end{align*}
Widać, że w~każdym testowanym przypadku $h\ge\lg m=\Omega(n\lg n)$, zatem w~żadnym z~nich sortowanie nie jest liniowe dla wszystkich $m$ wejść.

\exercise %8.1-4
Rozważmy drzewo decyzyjne dla takiego częściowo posortowanego ciągu. Mamy $n/k$ podciągów, każdy zawierający po $k$ posortowanych elementów, co daje $(k!)^{n/k}$ możliwych permutacji w~ciągu wejściowym, czyli tyle osiągalnych liści znajduje się w~drzewie decyzyjnym. Modyfikując tw.~8.1, dostajemy nierówność
\[
	2^h \le l \le (k!)^{n/k},
\]
co po zlogarytmowaniu obustronnie i~skorzystaniu z~\zad{3.1-8} daje
\[
	h \ge (n/k)\lg(k!) = \Omega(n\lg k).
\]

\subchapter{Sortowanie przez zliczanie}

\exercise %8.2-1
Rys.~\ref{fig:8.2-1} przedstawia początkowe kroki oraz wynik procedury \proc{Counting-Sort} podczas sortowania tablicy~$A$ po przyjęciu $k=6$.
\begin{figure}[ht]
	\begin{center}
		\includegraphics{fig08.1}
	\end{center}
	\caption{Symulacja działania procedury \proc{Counting-Sort}} \label{fig:8.2-1}
\end{figure}

\exercise %8.2-2
Elementy są przetwarzane od końca tablicy wejściowej, a~odpowiednia wartość w~tablicy $C$ jest za każdym razem dekrementowana. A~zatem równe sobie elementy będą umieszczane na coraz niższych pozycjach w~tablicy wyjściowej, co zachowuje ich początkową kolejność -- stąd sortowanie jest stabline.

\exercise %8.2-3
Po takiej modyfikacji przetwarzanie elementów zachodzi w~kolejności ich występowania w~tablicy wejściowej, a~ponieważ elementy równe sobie są  umieszczane na coraz niższych pozycjach tablicy wynikowej, to będą one w~odwrotnej kolejności niż początkowa, co zaburza stabilność algorytmu.

\exercise %8.2-4
Algorytm w~czasie $\Theta(n+k)$ zlicza elementy z~wejściowej tablicy, zbierając wyniki do pomocniczej tablicy. Następnie, zapytany o~liczbę elementów z~przedziału $[a\twodots b]$, zwraca liczbę elementów z~zakresu $[0\twodots b]$ pomniejszoną o~liczbę elementów z~$[0\twodots a-1]$. Dokładniej, jego pierwsza faza (preprocessing), jest równoważna pierwszym~7 linijkom procedury \proc{Counting-Sort}. Druga faza, czyli każde pytanie o~przedział $[a\twodots b]$ zwraca liczbę $C[\min(b,k)]-C[\max(a-1,0)]$.

\subchapter{Sortowanie pozycyjne}

\exercise %8.3-1
Przebieg działania procedury \proc{Radix-Sort} został przedstawiony na rys.~\ref{fig:8.3-1}.
\begin{figure}[ht]
	\begin{center}
		\includegraphics{fig08.2}
	\end{center}
	\caption{Symulacja działania procedury \proc{Radix-Sort}} \label{fig:8.3-1}
\end{figure}

\exercise %8.3-2
Algorytmami stabilnymi są sortowanie przez wstawianie i~sortowanie przez scalanie. Heapsort i~quicksort natomiast nie sortują stabilnie.

Aby dowolny algorytm sortowania za pomocą porównań uczynić stabilnym, można posłużyć się pomocniczą tablicą przechowującą początkowe indeksy elementów w~tablicy wejściowej i~przy każdym teście dającym odpowiedź, że elementy są sobie równe, porządkować je za pomocą ich indeksów. Wprowadza to dodatkową pamięć rzędu $O(n)$, ale nie wpływa na asymptotyczne oszacowanie na czas działania sortowania, ponieważ przeprowadzenie testu odbywa się nadal w~czasie stałym.

\exercise %8.3-3
Przeprowadźmy dowód przez indukcję względem liczby cyfr $d$ elementów wejściowych. Dla $d=1$ algorytm sprowadza się do wywołania sortowania stabilnego na tablicy wejściowej, więc poprawność algorytmu wynika z~poprawności tegoż sortowania.

Załóżmy teraz, że $d>1$ i~że \proc{Radix-Sort} po $d-1$ przebiegach zwróciło tablicę elementów, które obcięte do $d-1$ najmniej znaczących cyfr wyznaczają porządek rosnący. Teraz elementy sortowane są po \twoparts{$d$}{tych} najbardziej znaczących cyfrach. Niech $a$ i~$b$ będą pewnymi testowanymi cyframi podczas tego sortowania. Jeśli $a<b$ albo $a>b$, to niezależnie od pozostałych cyfr, elementy testowane są ustawiane w~odpowiednim porządku. Jeśli jednak $a=b$, to elementy nie zostaną zamienione miejscami, bo korzystamy z~założenia, że sortowanie jest stabilne. Elementy te pozostają jednak we właściwym porządku, bo jest on wyznaczony przez ich $d-1$ mniej znaczących cyfr, a~te zostały posortowane w~poprzednim przebiegu sortowania.

\exercise %8.3-4
Każdą liczbę z~zakresu od~0 do $n^2$ można traktować jako liczbę dwucyfrową w~systemie o~podstawie $n$. Wykorzystując lemat~8.4, wyznaczmy występujące tam zmienne. Liczby przyjmują $n^2$ możliwych wartości, więc $2^b-1=n^2$, skąd $b=\lg(n^2+1)$. Każda z~dwóch składowych cyfr może być równa jednej z~$n$ wartości, co daje $2^r-1=n$, więc $r=\lg(n+1)$. Zgodnie z~lematem, $n$ takich liczb można posortować w~czasie $\Theta((b/r)(n+2^r))=\Theta(2(n+n))=\Theta(n)$.

\exercise %8.3-5

\subchapter{Sortowanie kubełkowe}

\exercise %8.4-1
Na rys.~\ref{fig:8.4-1} zostało przedstawione działanie sortowania kubełkowego dla tablicy $A$.
\begin{figure}[ht]
	\begin{center}
		\includegraphics{fig08.3}
	\end{center}
	\caption{Symulacja działania procedury \proc{Bucket-Sort}} \label{fig:8.4-1}
\end{figure}

\exercise %8.4-2
Pesymistyczny czas działania algorytmu sortowania kubełkowego wynosi $O(n^2)$, gdyż może się zdarzyć, że wszystkie elementy trafią do tego samego kubełka.

Zamiast sortowania przez wstawianie można użyć algorytmu sortującego, który wykonuje $O(n\lg n)$ operacji w~przypadku pesymistycznym, np.\ sortowanie przez scalanie. Średni czas działania procedury \proc{Bucket-Sort} pozostaje liniowy, gdyż podstawiając do wzoru~(8.1) ograniczenie na czas działania sortowania przez scalanie, otrzymujemy
\[
	\E(T(n)) = \Theta(n)+\sum_{i=0}^{n-1}O(\E(n_i\lg n_i)).
\]
Ponieważ $\E(n_i\lg n_i)\le\E(n_i^2)$, to drugi składnik powyższej sumy jest rzędu $O(n)$, a~zatem $\E(T(n))=\Theta(n)$.

\exercise %8.4-3
Szanse nieuzyskania orła wynoszą $1/4$, co jest równe szansom uzyskania 2~orłów. Dokładnie jeden orzeł można uzyskać z~prawdopodobieństwem równym $1/2$. Mając te wartości, obliczamy:
\begin{align*}
	\E(X^2) &= 0^2\cdot\Pr(X^2=0^2)+1^2\cdot\Pr(X^2=1^2)+2^2\cdot\Pr(X^2=2^2) = 3/2, \\
	\E^2(X) &= \bigl(0\cdot\Pr(X=0)+1\cdot\Pr(X=1)+2\cdot\Pr(X=2)\bigr)^2 = 1.
\end{align*}

\exercise %8.4-4
\exercise %8.4-5

\problems

\problem{Dolne ograniczenia na średni czas działania sortowania za pomocą porównań} %8-1

\subproblem %8-1(a)
Podczas sortowania żadne dwie różne permutacje wejściowe nie prowadzą do tego samego liścia w~drzewie decyzyjnym -- jest zatem co najmniej $n!$ liści. Ponieważ algorytm $A$ jest deterministyczny, to dla pewnej permutacji wejściowej osiąga zawsze ten sam liść, a~więc w~drzewie decyzyjnym jest co najwyżej $n!$ osiągalnych liści. Wynika stąd, że algorytm $A$ może dotrzeć do dokładnie $n!$ liści. Ponieważ każda permutacja ma szanse pojawić się na wejściu z~równym prawdopodobieństwem, to szanse na osiągnięcie każdego liścia wynoszą $1/n!$.

Można traktować drzewo $T_A$ jako składające się tylko z~osiągalnych liści oraz ich przodków, ponieważ pozostałe węzły nigdy nie będą rozważane podczas działania algorytmu $A$.

\subproblem %8-1(b)
W~$LT$ i~$RT$ długość każdej ścieżki od korzenia do liścia jest o~1 mniejsza niż od korzenia do tego liścia w~drzewie $T$. Ponieważ lewe i~prawe poddrzewo $T$ mają w~sumie $k$ liści, to zachodzi wzór $D(T)=D(LT)+D(RT)+k$.

\subproblem %8-1(c)
Jeśli $T$ jest drzewem decyzyjnym o~$k$ liściach oraz $i$ jest liczbą liści w~$LT$, to $RT$ posiada $k-i$ liści. Rozważmy minimum po wszystkich wartościach $i$ i~skorzystajmy z~poprzedniego punktu. Otrzymujemy
\[
	d(k) = \min_T(D(T)) = \min_{1\le i\le k-1}(d(i)+d(k-i)+k).
\]

\subproblem %8-1(d)
Niech $f(i)=i\lg i+(k-i)\lg(k-i)$, gdzie $1\le i\le k-1$ i~wyznaczmy minimum tej funkcji obliczając jej pochodne:
\begin{align*}
	\frac{df}{di}(i) &= \frac{df}{di}\left(\frac{i\ln i+(k-i)\ln(k-i)}{\ln2}\right) = \frac{\ln i+1-\ln(k-i)-1}{\ln2} = \frac{\ln i-\ln(k-i)}{\ln2}, \\[2mm]
	\frac{d^2\!f}{di^2}(i) &= \frac{1}{\ln2}\left(\frac{1}{i}-\frac{1}{k-i}\right).
\end{align*}
Pierwsza pochodna zeruje się dla $i=k/2$, a~w~tym punkcie druga pochodna jest dodatnia, czyli $f$ przyjmuje tam minimum wynoszące $k\lg k-k$.

Załóżmy teraz, że $d(i)\ge i\lg i$ dla $1\le i\le k-1$ i~wyznaczmy $d(k)$, korzystając z~powyższego wyniku:
\begin{align*}
	d(k) &= \min_{1\le i\le k-1}(d(i)+d(k-i)+k) \\
	&\ge \min_{1\le i\le k-1}(i\lg i+(k-i)\lg(k-i))+k \\
	&\ge k\lg k-k+k \\
	&= k\lg k.
\end{align*}
Podstawa indukcji zachodzi trywialnie, bo $d(1)=0\ge 1\lg1=0$, a~zatem $d(k)=\Omega(k\lg k)$.

\subproblem %8-1(e)
Ponieważ drzewo $T_A$ posiada $n!$ liści, to wykorzystując definicję $d(k)$ oraz wynik poprzedniej części, mamy
\[
	D(T_A) \ge d(n!) = \Omega(n!\lg(n!)).
\]

Zgodnie z~częścią~(a) prawdopodobieństwo osiągnięcia przez algorytm każdego z~$n!$ liści drzewa jest równe $1/n!$, a~korzystając z~tego, że $D(T)$ stanowi sumę długości wszystkich ścieżek drzewa decyzyjnego, a~długości te są proporcjonalne do czasu działania algorytmu, to oczekiwany czas sortowania $n$ elementów za pomocą porównań wynosi
\[
	\frac{D(T_A)}{n!} = \frac{\Omega(n!\lg(n!))}{n!} = \Omega(\lg(n!)) = \Omega(n\lg n).
\]

\subproblem %8-1(f)

\problem{Sortowanie w~miejscu w~czasie liniowym} %8-2

\subproblem %8-2(a)
Sortowanie przez zliczanie.

\subproblem %8-2(b)
Sortowanie kubełkowe.

\subproblem %8-2(c)
Sortowanie przez wstawianie.

\subproblem %8-2(d)
Można użyć sortowania przez zliczanie, ponieważ działa w~czasie $O(n)$ i~sortuje stabilnie. Używając go do sortowania pozycyjnego, jesteśmy więc w~stanie uzyskać czas $O(bn)$ dla \twoparts{$b$}{bitowych} liczb.

\subproblem %8-2(e)
Użyjemy pomocniczej tablicy $P[1\twodots k]$ przechowującej liczbę elementów umieszczonych na właściwej pozycji w~tablicy.
\begin{codebox}
\Procname{$\proc{k-value-Sort}(A,k)$}
\li	\For $i\gets1$ \To $k$
\li		\Do $P[i]\gets0$
		\End
\li	$i\gets1$
\li	\While $i\ge1$
\li		\Do
			$i\gets i-P[A[i]]$
\li			$P[A[i]]\gets0$
\li			$j\gets C[A[i]]$
\li			zamień $A[j]\leftrightarrow A[i]$
\li			$C[A[j]]\gets C[A[j]]-1$
\li			$P[A[j]]\gets P[A[j]]+1$
		\End
\li	\Return $A$
\end{codebox}

Można wykazać, że algorytm używa czasu $O(n)$ oraz $O(k)$ pomocniczej pamięci, sortując tablicę $A$ stabilnie.

\problem{Sortowanie obiektów zmiennej długości} %8-3

\subproblem %8-3(a)
Sortowanie odbywa się w~dwóch fazach. Najpierw porządkujemy liczby według liczby cyfr -- im większa ilość cyfr składająca się na liczbę, tym liczba ta jest większa. Następnie sortujemy pozycyjnie liczby o~tej samej liczbie cyfr. W~tablicy wejściowej może znaleźć się od jednej do $n$ liczb, zatem pierwsza faza zajmuje czas $O(n)$, a~druga faza -- czas proporcjonalny do ilości cyfr we wszystkich liczbach łącznie, czyli również $O(n)$.

\subproblem %8-3(b)

\problem{Dzbanki} %8-4

\subproblem %8-4(a)
Abstrahując od modelu dzbanków, w~tym problemie mamy dwie tablice $n$ różnych liczb całkowitych, z~których jedna tablica jest permutacją drugiej. Należy przestawić elementy w~tablicach tak, aby reprezentowały one tę samą permutację, przy czym porównania między elementami w~obrębie jednej tablicy są zabronione.

Można opisać prosty algorytm działający w~czasie $\Theta(n^2)$. Testowanie każdej liczby z~pierwszej tablicy z~każdą liczbą z~drugiej tablicy daje nam pełną wiedzę o~elementach, dzięki której można utworzyć żądaną permutację.

\subproblem %8-4(b)
Problem jest analogiczny do problemu sortowania za pomocą porównań. Każde porównanie elementów decyduje o~wynikowej permutacji, których jest $n!$. Ponieważ każda z~nich może pojawić się na wyjściu algorytmu, to rozważamy drzewo decyzyjne o~$n!$ liściach podobne jak podczas analizy czasu działania sortowania za pomocą porównań w~przypadku pesymistycznym, jednak drzewo w~tym przypadku jest \twoparts{3}{arne}, rozróżniamy bowiem dodatkowo przypadek równości testowanych elementów. Wykorzystując podobne rozumowanie jak w~lemacie~8.1, dostajemy nierówność $3^h\ge n!$, skąd
\[
	h \ge \log_3(n!) = \lg(n!)/\!\lg3 = \Omega(n\lg n),
\]
co stanowi dolne oszacowanie na liczbę porównań w~tym problemie.

\subproblem %8-4(c)

\problem{Sortowanie względem średnich} %8-5

\subproblem %8-5(a)
Zgodnie z~definicją tablica $A[1\twodots n]$ jest \twoparts{1}{posortowana}, jeśli dla każdego $i=1$, 2,~\dots,~$n-1$ zachodzi $A[i]\le A[i+1]$, a~to jest równoważne temu, że tablica $A$ jest posortowana niemalejąco.

\subproblem %8-5(b)
Jedną z~takich permutacji jest $\langle2,1,5,3,7,4,8,6,10,9\rangle$.

\subproblem %8-5(c)
Udowodnijmy implikację w~prawą stronę. Załóżmy, że tablica $A$ jest \twoparts{$k$}{posortowana}. Wtedy dla każdego $i=1$, 2,~\dots,~$n-k$,
\begin{align*}
	\frac{\sum_{j=i}^{i+k-1}A[j]}{k} &\le \frac{\sum_{j=i+1}^{i+k}A[j]}{k} \\[1mm]
	\sum_{j=i}^{i+k-1}A[j] &\le \sum_{j=i+1}^{i+k}A[j] \\[1mm]
	A[i] &\le A[i+k].
\end{align*}

Przeprowadzenie rozumowania w~odwrotnej kolejności stanowi dowód implikacji przeciwnej.

\subproblem %8-5(d)
Zgodnie z~poprzednim punktem musimy spowodować, że wszystkie elementy z~wyjątkiem oddalonych od siebie o~mniej niż $k$ jednostek w~tablicy powinny być uporządkowane. Stosujemy zatem algorytm quicksort, który zatrzymuje się, kiedy otrzyma w~wywołaniu rekurencyjnym tablicę o~rozmiarze krótszym niż $k$. Analiza czasu działania sortowania jest identyczna jak wyznaczanie czasu działania zajmowanego przez quicksort w~\zad{7.4-5} -- tam również quicksort zatrzymuje się na tablicach o~rozmiarach mniejszych niż $k$.

\subproblem %8-5(e)
Podążając za wskazówką z~treści zadania, wykorzystujemy wynik z~\zad{6.5-8} zauważając, że mamy do scalenia $k$ posortowanych list w~jedną listę o~długości $n$. A~zatem czas działania tej operacji wynosi $O(n\lg k)$.

\subproblem %8-5(f)

\problem{Dolna granica dla scalania posortowanych list} %8-6

\subproblem %8-6(a)
Mając $2n$ różnych liczb, możemy wybrać $n$ z~nich do pierwszej listy, pozostałe pozostawiając w~drugiej liście. Ponieważ po takim podziale istnieje tylko jedna możliwa para posortowanych list, to liczba sposobów, na jakie można je utworzyć wynosi $\binom{2n}{n}$.

\subproblem %8-6(b)
\subproblem %8-6(c)
\subproblem %8-6(d)

\endinput

\chapter{Mediany i~statystyki pozycyjne}

\subchapter{Minimum i~maksimum}

\exercise %9.1-1
Wyznaczmy najpierw najmniejszą spośród $n$ liczb w~następujący sposób. W~każdym etapie testujemy liczby parami, odrzucając te, które były większe w~swoich parach. Procedurę wykonujemy na pozostawionych liczbach, aż do uzyskania jednej liczby -- będącej oczywiście minimum początkowego zbioru. Przyjmujemy, że w~razie nieparzystej liczby elementów na danym etapie, element bez pary nie jest testowany z~żadnym i~przechodzi do kolejnego etapu. Ponieważ po każdym etapie z~$k$ liczb zostaje $\lceil k/2\rceil$, to będzie $\lceil\lg n\rceil$ etapów i~tyle razy testowi będzie poddawany element najmniejszy. Zauważmy, że każdy test odrzuca jedną liczbę, wykonamy zatem dokładnie $n-1$ testów.

Zastanówmy się teraz, która z~odrzuconych liczb może być drugą najmniejszą w~zbiorze. Ponieważ zostanie ona odrzucona tylko w~teście z~elementem najmniejszym, to problem sprowadza się do wyznaczenia minimum zbioru tych liczb, które były testowane z~elementem najmniejszym. Na mocy wcześniejszej obserwacji mamy, że zbiór ten składa się z~$\lceil\lg n\rceil$ elementów, więc wystarczy $\lceil\lg n\rceil-1$ porównań do wyznaczenia jego minimum.

Ostatecznie dostajemy, że drugą najmniejszą spośród $n$ liczb można wyznaczyć, wykonując $n+\lceil\lg n\rceil-2$ porównań.

\exercise %9.1-2
Zadanie rozwiążemy prostszą metodą niż sugeruje nam to wskazówka.

Jeśli $n$ jest parzyste, to zgodnie z~podaną w~podręczniku informacją, wykonywanych jest $3n/2-2$ porównań. Ale dla parzystego $n$ prawdą jest, że $3n/2=\lceil3n/2\rceil$, więc wzór na liczbę koniecznych porównań przyjmuje postać $\lceil3n/2\rceil-2$. Niech teraz $n$ będzie liczbą nieparzystą, czyli $n=2k+1$ dla pewnego całkowitego $k$. Chcemy wykazać, że potrzebnych jest $\lceil3n/2\rceil-2$ porównań, czyli $\lceil3k+3/2\rceil-2=3k+2-2=3k$. Ale wynik ten zgadza się z~opisanym w~podręczniku dolnym oszacowaniem na liczbę porównań dla nieparzystego $n$, bo $3\lfloor n/2\rfloor=3\lfloor k+1/2\rfloor=3k$.

\subchapter{Wybór w~oczekiwanym czasie liniowym}

\exercise %9.2-1
Zakładamy, że parametr $i$ jest liczbą całkowitą spełniającą warunek $1\le i\le r-p+1$. Wywołanie z~wiersza~3 procedury \proc{Randomized-Partition} zwraca liczbę całkowitą $q$ taką, że $p\le q\le r$. Dla liczby całkowitej $k$ wyznaczonej w~kolejnym wierszu zachodzi więc $1\le k\le r-p+1$. Wywołanie rekurencyjne w~wierszu~8 nastąpi dla tablicy długości~0, jeśli $i<k$ i~$q=p$, ale wtedy $k=1$ i~taki przypadek nie ma prawa zajść. Podobnie w~wierszu~9 funkcja zostanie wywołana rekurencynie dla pustej tablicy, o~ile $i>k$ i~$q=r$, lecz wtedy $k=r-p+1$.

\exercise %9.2-2

\exercise %9.2-3
\begin{codebox}
\Procname{$\proc{Iterative-Randomized-Select}(A,p,r,i)$}
\li	\While $p<r$
\li		\Do
			$q\gets\proc{Randomized-Partition}(A,p,r)$
\li			$k\gets q-p+1$
\li			\If $i=k$
\li				\Then \Return $A[q]$
				\End
\li			\If $i<k$
\li				\Then $r\gets q-1$
\li				\Else
					$p\gets q+1$
\li					$i\gets i-k$
				\End
		\End
\li	\Return $A[p]$
\end{codebox}

\exercise %9.2-4
W~przypadku szukania elementu najmniejszego pesymistyczny przypadek dzielenia podtablicy występuje, gdy na element rozdzielający wybierany jest za każdym razem jej największy element. Kolejne wywołania rekurencyjne zmniejszają obszar poszukiwań o~1, w~wyniku czego efektem ubocznym jest posortowanie tablicy w~czasie kwadratowym. Ciąg tak wyznaczonych podziałów dla tablicy $A$ przedstawia rys.~\ref{fig:9.2-4}.

\begin{figure}[ht]
	\begin{center}
		\includegraphics{fig09.1}
	\end{center}
	\caption{Ciąg pesymistycznych podziałów tablicy $A$ podczas szukania elementu najmniejszego.} \label{fig:9.2-4}
\end{figure}

\subchapter{Wybór w~pesymistycznym czasie liniowym}

\exercise %9.3-1
Dokonajmy analogicznej analizy algorytmu \proc{Select} w~przypadku, gdy podział następuje na grupy \onedash{7}{elementowe}. W~co najmniej $\lceil n/7\rceil$ grupach są po 4 elementy większe od $x$, oprócz jednej grupy o~mniej niż 7 elementach, jeśli $n$ nie jest podzielne przez~7, i~jednej grupy zawierającej sam element $x$. Odliczając te dwie grupy, wnioskujemy, że liczba elementów większych od $x$ wynosi co najmniej
\[
	4\left(\left\lceil\frac{1}{2}\left\lceil\frac{n}{7}\right\rceil\right\rceil-2\right) \ge \frac{2n}{7}-8.
\]
Podobnie wykazuje się, że liczba elementów mniejszych od $x$ wynosi co najmniej $2n/7-8$. Stąd wywołanie rekurencyjne nastąpi dla zbioru \onedash{$(5n/7+8)$}{elementowego}. Rekurencja przyjmuje więc postać
\[
	T(n) \le \begin{cases}
		\Theta(1), & \text{jeśli $n\le d$}, \\
		T(\lceil n/7\rceil)+T(5n/7+8)+O(n), & \text{jeśli $n>d$},
	\end{cases}
\]
przy czym $d$ jest pewną stałą, którą wyznaczymy później.

Wykażemy metodą przez podstawienie, że $T(n)=O(n)$. Zachodzi oczywiście $T(n)\le cn$ dla pewnej stałej $c>0$ oraz wszystkich $n\le d$. Niech teraz $n>d$ i~załóżmy, że $T(k)\le ck$ dla pewnej stałej $c>0$ i~wszystkich $k<n$. Dla pewnej stałej $a>0$ zachodzi
\begin{align*}
	T(n) &\le c\lceil n/7\rceil+c(5n/7+8)+an \\
	&\le cn/7+c+5cn/7+8c+an \\
	&= 6cn/7+9c+an \\
	&= cn+(-cn/7+9c+an) \\
	&\le cn,
\end{align*}
o~ile składnik $-cn/7+9c+an$ jest niedodatni. Dla $n>63$ warunek ten zachodzi, gdy $c\ge7a(n/(n-63))$. Jeśli z~kolei $n\ge126$, to $n/(n-63)\le2$ i~dostajemy, że $c\ge14a$. Można zatem przyjąć za $d$ wartość 126.

Pokażemy teraz, że jeśli podział będzie dokonywany na grupy \onedash{3}{elementowe}, to czas działania tak zmodyfikowanego algorytmu \proc{Select} jest wyższy od liniowego. Rozważmy przypadek, w~którym jest dokładnie $\left\lceil\frac{1}{2}\left\lceil\frac{n}{3}\right\rceil\right\rceil$ grup o~medianach większych lub równych $x$, a~ostatnia, niepełna grupa, zawiera dokładnie 2 elementy większe niż $x$. Stąd liczba elementów większych od $x$ wynosi
\[
	2\left(\left\lceil\frac{1}{2}\left\lceil\frac{n}{3}\right\rceil\right\rceil-1\right)+1 = 2\left\lceil\frac{n}{6}\right\rceil-1.
\]
Procedura może zostać zatem wywołana rekurencyjne dla co najmniej $n-(2\left\lceil n/6\right\rceil-1)\ge n-(2(n/6+1)-1)=2n/3-1$ elementów nieprzekraczających $x$. Zauważmy też, że sortowanie elementów w~kroku 2 algorytmu \proc{Select} zabiera czas nie tylko $O(n)$, ale dokładnie $\Theta(n)$. Rekurencja opisująca czas działania algorytmu w~tym przypadku ma postać
\[
	T(n) \ge \begin{cases}
		\Theta(1), & \text{jeśli $n\le d$}, \\
		T(\lceil n/3\rceil)+T(2n/3-1)+\Theta(n), & \text{jeśli $n>d$},
	\end{cases}
\]
dla pewnej stałej $d>0$. Można wykazać, stosując metodę przez podstawienie, że rozwiązaniem powyższej rekurencji jest $T(n)=\Omega(n\lg n)$, w~związku z~czym algorytm po takiej modyfikacji nie działa w~czasie liniowym.

\exercise %9.3-2
Z~analizy algorytmu \proc{Select} wynika, że liczba elementów większych od $x$ wynosi co najmniej $3n/10-6$ i~podobnie dla liczby elementów mniejszych od $x$. Mamy teraz
\[
	\frac{3n}{10}-6-\left\lceil\frac{n}{4}\right\rceil \ge \frac{3n}{10}-6-\left(\frac{n}{4}+1\right) = \frac{n-140}{20} \ge 0,
\]
ponieważ mieliśmy założenie, że $n\ge140$. A~zatem obie te wartości są nie mniejsze niż $\lceil n/4\rceil$.

\exercise %9.3-3
Można wymusić dokonywanie najbardziej zrównoważonych podziałów przez procedurę \proc{Partition} dzięki zastosowaniu algorytmu \proc{Select} do wyboru mediany sortowanej podtablicy na element rozdzielający. Ponieważ pesymistyczny czas algorytmu \proc{Select} wynosi $O(n)$, to pesymistyczny czas algorytmu quicksort sprowadza się po takiej modyfikacji do rekurencji $T(n)\le2T(n/2)+\Theta(n)$, której rozwiązaniem jest $O(n\lg n)$.

\exercise %9.3-4

\exercise %9.3-5
Problem rozwiązujemy analogicznie jak w~\refExercise{9.3-3} ale dla algorytmu wyboru. Algorytm jest podobny do \proc{Randomized-Select}, w~wierszu~3 wywołujemy jednak procedurę \proc{Partition}, wewnątrz której elementem rozdzielającym czynimy medianę przetwarzanej podtablicy wyznaczoną za pomocą ``czarnej skrzynki''. W~każdym wywołaniu rekurencyjnym będzie dokonywany najbardziej zrównoważony podział podtablicy, więc czas działania algorytmu w~przypadku pesymistycznym jest postaci
\[
	T(n) \le
	\begin{cases}
		O(1), & \text{jeśli $n=1$}, \\
		T(\lfloor n/2\rfloor)+O(n), & \text{jeśli $n>1$}.
	\end{cases}
\]
Rozwiązaniem powyższej rekurencji jest $T(n)=O(n)$.

\exercise %9.3-6
Załóżmy, że badany zbiór przechowujemy w~tablicy \onedash{$n$}{elementowej}. Podzielmy ją na $\lfloor n/k\rfloor$ podtablic i~posortujmy każdą z~nich. Wszystkie te podtablice składają się z~$k$ elementów (z~dokładnością do~1), więc ich sortowanie algorytmem quicksort zajmuje łącznie czas $\lfloor n/k\rfloor\cdot O(k\lg k)=O(n\lg k)$. %Zauważmy, że najmniejszy kwantyl rzędu $k$ początkowej tablicy jest największym elemente

\exercise %9.3-7
Znajdujemy elementy $i_1=\lfloor n/2\rfloor-\lfloor k/2\rfloor$ oraz $i_2=\lfloor n/2\rfloor+\lfloor k/2\rfloor$ wejściowej tablicy w~porządku rosnącym za pomocą dwóch wywołań algorytmu \proc{Select} w~czasie $O(n)$. Szukane elementy są większe od \onedash{$i_1$}{tego} i~mniejsze od \onedash{$i_2$}{tego} -- wyznaczamy je przeglądając liniowo tablicę i~testując czy znajdują się w~wyznaczonym przedziale.

\exercise %9.3-8
Porównajmy mediany każdej z~tablic. Jeśli są one równe, to wśród wszystkich $2n$ elementów $n$ jest większych od tych median -- $n/2$ z~tablicy $X$ i~$n/2$ z~tablicy $Y$. Załóżmy teraz, że mediany są różne i~bez utraty ogólności, niech mniejsza będzie mediana tablicy $X$. Pomijając teraz elementy $X$ mniejsze od jej mediany i~elementy $Y$ większe od jej mediany, sprowadzamy problem do identycznego ale o~połowę mniejszego, ponieważ wiadomo, że poszukiwana mediana znajduje się wśród $n$ pozostawionych elementów.
\begin{codebox}
\Procname{$\proc{Two-Array-Median}(X,p_X,r_X,Y,p_Y,r_Y)$}
\li	\If $p_X=r_X$
\li		\Then \Return $\min(X[p_X],Y[p_Y])$
		\End
\li	$q_X\gets\lfloor(p_X+r_X)/2\rfloor$
\li	$q_Y\gets\lfloor(p_Y+r_Y)/2\rfloor$
\li	\If $X[q_X]=Y[q_Y]$
\li		\Then \Return $X[q_X]$
		\End
\li	\If $X[q_X]>Y[q_Y]$
\li		\Then \Return $\proc{Two-Array-Median}(X,p_X,q_X-1,Y,q_Y+1,r_Y)$
\li		\Else \Return $\proc{Two-Array-Median}(X,q_X+1,r_X,Y,p_Y,q_Y-1)$
		\End
\end{codebox}

Czas działania algorytmu jest opisany przez rekurencję z~\refExercise{2.3-5}, ponieważ jego działanie jest analogiczne do wyszukiwania binarnego -- na każdym poziomie rekursji po wykonaniu jednego porównania odrzucana jest połowa problemu.

\exercise %9.3-9

\problems

\problem{Sortowanie $i$ największych elementów} %9-1

\subproblem %9-1(a)
Liczby można posortować algorytmem sortowania przez scalanie, który w~najgorszym przypadku potrzebuje czasu $\Theta(n\lg n)$. Następnie zwracamy $i$ największych liczb poprzez proste przejście po prawym końcu tablicy, co zajmuje czas $\Theta(i)$, więc całkowity czas algorytmu w~najgorszym przypadku wynosi $\Theta(n\lg n+i)$.

\subproblem %9-1(b)
Zbudowanie kopca typu max dla kolejki priorytetowej wymaga czasu $\Theta(n)$. Wykonanie $i$ operacji \proc{Extract-Max} zajmuje czas
\[
	\sum_{k=1}^i\Theta(\lg(n-k)) = \sum_{k=n-i}^n\Theta(\lg k) = \Theta(n\lg n-(n-i-1)\lg n) = \Theta(i\lg n)
\]
na podstawie punktu~(b) problemu~\refProblem{A-1}, a~zatem całkowitym czasem algorytmu w~przypadku pesymistycznym jest $\Theta(n+i\lg n)$.

\subproblem %9-1(c)
Aby osiągnąć najlepszy czas w~przypadku pesymistycznym, użyjemy procedury \proc{Select} do znalezienia \onedash{$i$}{tej} statystyki pozycyjnej. Sortowanie $i$ największych liczb można wykonać algorytmem sortowania przez scalanie, które w~najgorszym przypadku zajmuje czas $\Theta(i\lg i)$, skąd całkowity czas działania algorytmu wynosi $\Theta(n+i\lg i)$, co czyni go najbardziej efektywnym spośród wszystkich rozważanych algorytmów w~obecnym problemie.

\problem{Mediana ważona} %9-2

\subproblem %9-2(a)
Dla tak przyjętych wag elementów mamy
\[
	\sum_{x_i<x_k}w_i = \frac{k-1}{n} \quad\text{oraz}\quad \sum_{x_i>x_k}w_i = \frac{n-k}{n}.
\]
Jeśli ograniczymy obie sumy od góry przez $1/2$, to dostaniemy, że medianą ważoną jest element $x_k$, gdzie $n/2\le k\le n/2+1$, czyli $k=\lfloor(n+1)/2\rfloor$ lub $k=\lceil(n+1)/2\rceil$, a~to jest warunek na to, aby $x_k$ było zwykłą medianą.

\subproblem %9-2(b)
Po posortowaniu elementów wyznaczenie mediany ważonej odbywa się w~prosty sposób -- wystarczy sumować wagi elementów, poruszając się w~ich porządku rosnącym i~zwrócić pierwszy element, dla którego suma wag osiągnie lub przekroczy $1/2$. Złożoność tej procedury zależy od efektywności sortowania, które w~pesymistycznym czasie zajmuje $O(n\lg n)$ dla $n$ elementów.

\subproblem %9-2(c)
\subproblem %9-2(d)
\subproblem %9-2(e)

\problem{Małe statystyki pozycyjne} %9-3

\subproblem %9-3(a)
\note{W~tłumaczeniu występuje błąd w~sformułowaniu rekurencji\/ $U_i(n)$. Wartość\/ $T(n)$ powinna być zwracana nie dla\/ $i\le n/2$, lecz dla\/ $i\ge n/2$.}

\noindent Postępując za podaną wskazówką, grupujemy najpierw elementy w~pary i~wyznaczamy mniejsze w~każdej parze. Zauważmy, że \onedash{$i$}{ty} w~kolejności element całego zbioru, gdzie $i<n/2$, znajduje się wśród elementów mniejszych w~parach lub wśród 

\subproblem %9-3(b)
Załóżmy, że $n/2^k\le i<n/2^{k-1}$ i~rozwińmy rekurencję $U_i(n)$ w~przypadku, gdy $i<n/2$:
\[
	U_i(n) = \left\lfloor\frac{n}{2}\right\rfloor+T(2i)+\left\lfloor\frac{1}{2}\left\lceil\frac{n}{2}\right\rceil\right\rfloor+T(2i)+\dots+\left\lfloor\frac{1}{2}\left\lceil\frac{n}{2^{k-1}}\right\rceil\right\rfloor+T(2i)+T\Bigl(\left\lceil\frac{n}{2^k}\right\rceil\Bigr).
\]
Na mocy tego, że $k\ge\lg(n/i)$ oraz $T(\lceil n/2^k\rceil)\le T(n/2^k+1)\le T(i+1)\le T(2i)$, mamy
\begin{align*}
	U_i(n) &\le \frac{n}{2}+\frac{n}{4}+\dots+\frac{n}{2^k}+(k-1)T(2i)+T\Bigl(\left\lceil\frac{n}{2^k}\right\rceil\Bigr) \\
	&< n\sum_{j=1}^\infty\frac{1}{2^j}+kT(2i) \\
	&= n+O(T(2i)\lg(n/i)).
\end{align*}

\subproblem %9-3(c)
Ponieważ $i$ jest stałą, to $T(2i)$ również można potraktować jako wartość stałą. Wykorzystując poprzedni punkt, mamy
\[
	U_i(n) = n+O(T(2i)\lg(n/i)) = n+O(\lg n-\lg i) = n+O(\lg n),
\]
co należało dowieść.

\subproblem %9-3(d)
Dla $k>2$ oszacowanie wynika natychmiast z~części~(b) po podstawieniu $i=n/k$, bo wtedy oczywiście $i<n/2$.

Jeśli $k=2$, to rozwiązaniem rekurencji $U_i(n)$ jest $T(n)$, czyli $O(n)$. Zauważmy, że teza w~tym przypadku sprowadza się do $U_i(n)=n+O(T(n))$ i~rzeczywiście zachodzi, ponieważ $n+O(T(n))=O(n)$.

\endinput


%\setcounter{part}{2}
%\part{Struktury danych}

%\setcounter{chapter}{9}
%\chapter{Elementarne struktury danych}

\makeatletter
\def\input@path{{chapter10/}}
\makeatother

\subchapter{Stosy i~kolejki}

\exercise %10.1-1
Ciąg operacji na stosie $S$ został przedstawiony na rys.\ \ref{fig:10.1-1}.
\begin{figure}[ht]
    \begin{center}
		\includegraphics{fig_10.1-1}
	\end{center}
	\caption{Operacje wstawiania i~usuwania elementów na stosie $S$.
{\sffamily\bfseries(a)} Pusty stos $S$ reprezentowany jako tablica $S[1\twodots6]$.
{\sffamily\bfseries\doubledash{(b)}{(d)}} Stos $S$ po wykonaniu na nim kolejnych operacji \proc{Push}.
{\sffamily\bfseries(e)} Po usunięciu elementu ze stosu $S$ zmieniana jest jedynie pozycja atrybutu \attrib{S}{top}, natomiast sam element pozostaje w~tablicy.
{\sffamily\bfseries(f)} Wstawienie nowego elementu na stos $S$ nadpisuje stary element, który został usunięty w~poprzednim kroku.
{\sffamily\bfseries(g)} Stos $S$ po wykonaniu ostatniej operacji \proc{Pop}.} \label{fig:10.1-1}
\end{figure}

\exercise %10.1-2
Z~tablicą $A$ związujemy atrybuty \attrib{A}{left-top} i~\attrib{A}{right-top}, które będą wskazywać na pozycje wierzchołków, odpowiednio, pierwszego i~drugiego stosu.
Pierwszy stos będzie składał się z~elementów $A[1\twodots\attrib{A}{left-top}]$, a~drugi -- z~elementów $A[\attrib{A}{right-top}\twodots n]$.
Początkowo $\attrib{A}{left-top}=0$ i~$\attrib{A}{right-top}=n+1$.
Dodawanie i~usuwanie elementów w~pierwszym stosie działa identycznie jak dla pojedynczego stosu znajdującego się w~tablicy $A$, którego wierzchołek zajmuje pozycję \attrib{A}{left-top}.
Drugi stos zachowuje się symetrycznie do pierwszego -- podczas dodawania do niego nowego elementu atrybut \attrib{A}{right-top} będzie dekrementowany, a~podczas usuwania -- inkrementowany.
Operacje dodawania i~usuwania elementów działają oczywiście w~czasie stałym.
Jeśli $\attrib{A}{left-top}=0$, to pierwszy stos jest pusty, a~jeśli $\attrib{A}{right-top}=n+1$, to drugi stos jest pusty.
Przepełnienie ma miejsce tylko wtedy, gdy $\attrib{A}{left-top}=\attrib{A}{right-top}-1$ i~usiłujemy dodać nowy element do któregokolwiek stosu, czyli wówczas, gdy łączna liczba elementów na obu stosach wynosi $n$.

\exercise %10.1-3
Ciąg operacji na kolejce $Q$ ilustruje rys.\ \ref{fig:10.1-3}.

\begin{figure}[ht]
	\begin{center}
		\includegraphics{fig_10.1-3}
	\end{center}
	\caption{Operacje wstawiania i~usuwania elementów na kolejce $Q$.
{\sffamily\bfseries(a)} Pusta kolejka $Q$ reprezentowana jako tablica $Q[1\twodots6]$.
{\sffamily\bfseries\doubledash{(b)}{(d)}} Kolejka $Q$ po wykonaniu na niej kolejnych operacji \proc{Enqueue}.
{\sffamily\bfseries(e)} Po usunięciu elementu z~kolejki $Q$ zmieniana jest jedynie pozycja atrybutu \attrib{Q}{head}, natomiast sam element pozostaje w~tablicy.
{\sffamily\bfseries(f)} Wstawienie nowego elementu do kolejki $Q$ nadpisuje stary element, który został usunięty w~poprzednim kroku.
{\sffamily\bfseries(g)} Kolejka $Q$ po wykonaniu ostatniej operacji \proc{Dequeue}.} \label{fig:10.1-3}
\end{figure}

\exercise %10.1-4
Poniżej znajduje się pseudokod operacji analogicznej do \proc{Stack-Empty}, ale testującej pustość kolejki.
Wykorzystujemy ją podczas wykrywania błędów niedomiaru.

\begin{codebox}
\Procname{$\proc{Queue-Empty}(Q)$}
\li	\If $\attrib{Q}{head}=\attrib{Q}{tail}$
\li		\Then \Return \const{true}
\li		\Else \Return \const{false}
\End
\end{codebox}

Następujące wiersze należy dodać na początek procedury \proc{Enqueue}:
\begin{codebox}
\zi	\If $\attrib{Q}{head}=\attrib{Q}{tail}+1$
\zi		\Then \Error ,,nadmiar''
		\End
\zi	\If $\attrib{Q}{head}=1$ i~$\attrib{Q}{tail}=\attrib{Q}{length}$
\zi		\Then \Error ,,nadmiar''
		\End
\end{codebox}
Z~kolei poniższy fragment kodu umieszczamy na początku procedury \proc{Dequeue}:
\begin{codebox}
\zi	\If $\proc{Queue-Empty}(Q)$
\zi		\Then \Error ,,niedomiar''
		\End
\end{codebox}

\exercise %10.1-5
Kolejkę dwustronną implementujemy przy użyciu tablicy $D[1\twodots n]$.
Podobnie jak w~zwykłej kolejce atrybut \attrib{D}{head} wskazuje na początek kolejki, natomiast atrybut \attrib{D}{tail} wyznacza następną wolną pozycję, na którą można wstawić nowy element.
Procedury \proc{Head-Enqueue} oraz \proc{Head-Dequeue} mają na celu, odpowiednio, dodanie nowego elementu na początek kolejki i~usunięcie elementu z~początku kolejki.
Z~kolei procedury \proc{Tail-Enqueue} oraz \proc{Tail-Dequeue} implementują dodawanie i~usuwanie elementów na końcu kolejki.
Dla skrócenia zapisu pominięto sprawdzanie błędów niedomiaru i~przepełnienia.

\begin{codebox}
\Procname{$\proc{Head-Enqueue}(D,x)$}
\li	\If $\attrib{D}{head}=1$
\li		\Then $\attrib{D}{head}\gets\attrib{D}{length}$
\li		\Else $\attrib{D}{head}\gets\attrib{D}{head}-1$
		\End
\li	$D[\attrib{D}{head}]\gets x$
\end{codebox}

\begin{codebox}
\Procname{$\proc{Head-Dequeue}(D)$}
\li	$x\gets D[\attrib{D}{head}]$
\li	\If $\attrib{D}{head}=\attrib{D}{length}$
\li		\Then $\attrib{D}{head}\gets1$
\li		\Else $\attrib{D}{head}\gets\attrib{D}{head}+1$
		\End
\li	\Return $x$
\end{codebox}

\begin{codebox}
\Procname{$\proc{Tail-Enqueue}(D,x)$}
\li	$D[\attrib{D}{tail}]\gets x$
\li	\If $\attrib{D}{tail}=\attrib{D}{length}$
\li		\Then $\attrib{D}{tail}\gets1$
\li		\Else $\attrib{D}{tail}\gets\attrib{D}{tail}+1$
		\End
\end{codebox}

\begin{codebox}
\Procname{$\proc{Tail-Dequeue}(D,x)$}
\li	\If $\attrib{D}{tail}=1$
\li		\Then $\attrib{D}{tail}\gets\attrib{D}{length}$
\li		\Else $\attrib{D}{tail}\gets\attrib{D}{tail}-1$
		\End
\li	\Return $D[\attrib{D}{tail}]$
\end{codebox}

Wszystkie powyższe operacje działają w~czasie $\Theta(1)$.

\exercise %10.1-6
Wszystkie elementy kolejki będą trzymane na jednym stosie -- drugi stos będzie pełnił funkcję pomocniczą.
Dodawanie nowego elementu do kolejki oraz sprawdzanie czy kolejka jest pusta, nie różnią się od analogicznych operacji wykonywanych na pierwszym stosie.
Usuwanie elementu z~kolejki jest już operacją nieco bardziej skomplikowaną, gdyż należy pobrać element znajdujący się najgłębiej na tym stosie.
Ściągamy wpierw z~niego wszystkie elementy za pomocą operacji \proc{Pop} i~umieszczamy kolejno na drugim stosie, wywołując ciąg operacji \proc{Push}.
W~rezultacie drugi stos będzie zawierał wszystkie elementy kolejki w~odwrotnej kolejności.
Teraz pobieramy i~zapamiętujemy element ze szczytu drugiego stosu, gdyż za chwilę zwrócimy go jako wynik procedury.
Wcześniej trzeba bowiem przenieść pozostałą zawartość drugiego stosu z~powrotem do pierwszego, co przy okazji przywróci początkową kolejność elementów.

Łatwo sprawdzić, że testowanie pustości kolejki oraz dodawanie do niej nowego elementu, są wykonywane w~czasie stałym, natomiast usuwanie wymaga czasu proporcjonalnego do liczby elementów kolejki.

\exercise %10.1-7
Rozwiązanie jest analogiczne do rozwiązania poprzedniego zadania.
Sprawdzanie czy stos jest pusty, jak również dodawanie nowego elementu, to identyczne operacje wywoływane na pierwszej kolejce.
Usuwanie elementu ze stosu odbywa się poprzez pobranie wszystkich elementów z~wyjątkiem ostatniego z~pierwszej kolejki i~dodanie tych elementów do drugiej.
Ostatni z~nich także usuwamy z~kolejki, zapamiętawszy go w~celu późniejszego zwrócenia jako wyniku operacji.
Ostatnim krokiem jest przeniesienie reszty elementów z~powrotem do pierwszej kolejki.

Podobnie jak w~poprzednim zadaniu operacja odpowiedzialna za sprawdzenie czy stos jest pusty oraz dodawanie elementu działają w~czasie stałym, a~operacja usuwania -- w~czasie liniowym względem liczby elementów na stosie.

\subchapter{Listy}

\exercise %10.2-1
Operację \proc{Insert}, dodającą nowy element $x$ na początek listy jednokierunkowej $L$, można zaimplementować w~czasie stałym -- wystarczy ustawić pole \attrib{x}{next} na głowę listy $L$ (albo na \const{nil}, jeśli lista $L$ jest pusta) i~uaktualnić wskaźnik \attrib{L}{head}.
Operacja \proc{Delete}, czyli usuwanie z~listy jednokierunkowej elementu wskazywanego przez $x$, wymaga jednak czasu wyższego niż stały.
Jest tak dlatego, że jedynym sposobem na dotarcie do elementu poprzedzającego $x$ na liście $L$ w~celu aktualizacji jego pola \id{next}, jest przejście tej listy od głowy aż do tegoż elementu.
Czynność ta w~najgorszym przypadku zajmuje czas proporcjonalny do liczby elementów listy $L$.

Poniżej zamieszczamy implementacje obu tych operacji -- będziemy z~nich korzystać w~późniejszych zadaniach.
\begin{codebox}
\Procname{$\proc{Singly-Linked-List-Insert}(L,x)$}
\li	$\attrib{x}{next}\gets\attrib{L}{head}$
\li	$\attrib{L}{head}\gets x$
\end{codebox}

\begin{codebox}
\Procname{$\proc{Singly-Linked-List-Delete}(L,x)$}
\li	\If $x=\attrib{L}{head}$
\li		\Then $\attrib{L}{head}\gets\attribb{L}{head}{next}$
\li		\Else $y\gets\attrib{L}{head}$
\li			\While $\attrib{y}{next}\ne x$
\li				\Do $y\gets\attrib{y}{next}$
				\End
\li			$\attrib{y}{next}\gets\attrib{x}{next}$
		\End
\end{codebox}

\exercise %10.2-2
Wszystkie elementy implementowanego stosu będziemy trzymać na liście jednokierunkowej $L$ w~kolejności od szczytu w~głowie listy do dna stosu w~ogonie.
Dzięki takiemu rozwiązaniu operacje dodawania i~usuwania elementów będą działać w~czasie $\Theta(1)$.
Pseudokody operacji \proc{Push} i~\proc{Pop} dla opisanej implementacji stosu prezentujemy poniżej.
\begin{codebox}
\Procname{$\proc{Singly-Linked-List-Push}(L,k)$}
\li	$\attrib{x}{key}\gets k$
\li $\proc{Singly-Linked-List-Insert}(L,x)$
\end{codebox}

\begin{codebox}
\Procname{$\proc{Singly-Linked-List-Pop}(L)$}
\li	\If $\attrib{L}{head}=\const{nil}$
\li		\Then \Error ,,niedomiar''
		\End
\li	$x\gets\attrib{L}{head}$
\li	$\attrib{L}{head}\gets\attrib{x}{next}$
\li	\Return \attrib{x}{key}
\end{codebox}

\exercise %10.2-3
Elementy kolejki będziemy przechowywać na liście jednokierunkowej $L$ w~kolejności od głowy kolejki do jej ogona.
Aby jednak operacja dodawania była wykonalna w~czasie $\Theta(1)$, wymagane jest związanie z~listą $L$ dodatkowego atrybutu \attrib{L}{tail}, który będzie wskazywał na ogon listy $L$ albo na \const{nil}, jeżeli lista $L$ jest pusta.
Implementacje operacji \proc{Enqueue} i~\proc{Dequeue} dla listowej reprezentacji kolejki znajdują się poniżej.
\begin{codebox}
\Procname{$\proc{Singly-Linked-List-Enqueue}(L,k)$}
\li	$\attrib{x}{next}\gets\const{nil}$
\li	$\attrib{x}{key}\gets k$
\li	\If $\attrib{L}{tail}\ne\const{nil}$
\li		\Then $\attribb{L}{tail}{next}\gets x$
\li		\Else $\attrib{L}{head}\gets x$
		\End
\li	$\attrib{L}{tail}\gets x$
\end{codebox}

\begin{codebox}
\Procname{$\proc{Singly-Linked-List-Dequeue}(L)$}
\li	\If $\attrib{L}{head}=\const{nil}$
\li		\Then \Error ,,niedomiar''
		\End
\li	$x\gets\attrib{L}{head}$
\li	$\attrib{L}{head}\gets\attrib{x}{next}$
\li	\If $\attrib{L}{tail}=x$
\li		\Then $\attrib{L}{tail}\gets\const{nil}$
		\End
\li	\Return \attrib{x}{key}
\end{codebox}

\exercise %10.2-4
Zauważmy, że pole \attribb{L}{nil}{key} jest niewykorzystywane w~implementacji listy z~wartownikami.
Możemy zatem na początku działania procedury \proc{List-Search}$'$ nadać mu wartość $k$.
Pętla \kw{while} w~tej procedurze nie musi wówczas sprawdzać warunku, czy $x\ne\attrib{L}{nil}$, ponieważ jeśli na liście $L$ nie ma elementu o~kluczu $k$, to pętla zatrzyma się na wartowniku \attrib{L}{nil} i~zostanie on zwrócony jako wynik procedury.

\exercise %10.2-5
Operacja wstawiania nowego elementu na jednokierunkową listę cykliczną $L$ umieszcza go jako następnik głowy listy $L$.
Podczas operacji usuwania znajdowany jest poprzednik $y$ usuwanego elementu $x$, po czym uaktualnione zostaje \attrib{y}{next}.
Jeśli $x$ jest głową listy, to atrybut \attrib{L}{head} zostaje ustawiony na następnik $x$ albo na \const{nil}, w~przypadku gdy $x$ stanowi jedyny element listy $L$.
Wreszcie operacja wyszukiwania przechodzi całą listę $L$, począwszy od jej głowy, i~zatrzymuje się w~momencie odnalezienia elementu o~szukanym kluczu bądź w~chwili ponownego dotarcia do głowy listy $L$.

Poniższe procedury stanowią implementacje tych trzech operacji słownikowych.
\begin{codebox}
\Procname{$\proc{Circular-List-Insert}(L,x)$}
\li	\If $\attrib{L}{head}=\const{nil}$
\li		\Then $\attrib{L}{head}\gets\attrib{x}{next}\gets x$
\li		\Else $\attrib{x}{next}\gets\attribb{L}{head}{next}$
\li			$\attribb{L}{head}{next}\gets x$
		\End
\end{codebox}

\begin{codebox}
\Procname{$\proc{Circular-List-Delete}(L,x)$}
\li	$y\gets\attrib{L}{head}$
\li	\While $\attrib{y}{next}\ne x$
\li		\Do $y\gets\attrib{y}{next}$
		\End
\li	$\attrib{y}{next}\gets\attrib{x}{next}$
\li	\If $\attrib{L}{head}=x$
\li		\Then \If $\attrib{x}{next}=x$
\li				\Then $\attrib{L}{head}\gets\const{nil}$
\li				\Else $\attrib{L}{head}\gets\attrib{x}{next}$
				\End
		\End
\end{codebox}

\begin{codebox}
\Procname{$\proc{Circular-List-Search}(L,k)$}
\li	\If $\attrib{L}{head}=\const{nil}$
\li		\Then \Return \const{nil}
		\End
\li	\If $\attribb{L}{head}{key}=k$
\li		\Then \Return \attrib{L}{head}
		\End
\li	$x\gets\attribb{L}{head}{next}$
\li	\While $x\ne\attrib{L}{head}$
\li		\Do \If $\attrib{x}{key}=k$
\li				\Then \Return $x$
				\End
\li			$x\gets\attrib{x}{next}$
		\End
\li	\Return \const{nil}
\end{codebox}

Łatwo sprawdzić, że procedura implementująca operację \proc{Insert} działa w~czasie stałym, natomiast pozostałe dwie procedury dla listy o~$n$ elementach w~pesymistycznym przypadku potrzebują czasu $\Theta(n)$.

\exercise %10.2-6
Zbiory można zaimplementować jako jednokierunkowe listy cykliczne.
Zbiory $S_1$ i~$S_2$ są rozłączne, więc w~reprezentacji sumy $S_1\cup S_2$ nie pojawią się powtarzające się elementy.
Operacja \proc{Union} może więc tylko ,,sklejać'' listy reprezentujące zbiory $S_1$ i~$S_2$.
Podczas działania tej operacji następnikiem głowy pierwszej listy staje się następnik głowy drugiej listy, a~następnikiem głowy drugiej staje się początkowy następnik głowy pierwszej.
Działania te są wykonywane tylko wtedy, gdy obie listy są niepuste.
Wykonując stałą liczbę kroków, otrzymujemy w~wyniku tej operacji jednokierunkową listę cykliczną (której głową może być dowolny jej element) reprezentującą zbiór $S_1\cup S_2$.

\exercise %10.2-7
W~algorytmie wykorzystamy pomocniczą listę jednokierunkową $L'$, na którą będziemy wstawiać kolejno usuwane elementy z~głowy listy $L$.
Łatwo zauważyć, że po przeprowadzeniu tego ciągu operacji przeniesione elementy będą znajdować się w~$L'$ w~porządku odwrotnym względem początkowego ustawienia na liście $L$.
Nadanie atrybutowi \attrib{L}{head} wartości \attrib{L'}{head} wystarczy, aby przenieść całą zawartość listy $L'$ do listy $L$.
\begin{codebox}
\Procname{$\proc{Singly-Linked-List-Reverse}(L)$}
\li	$\attrib{L'}{head}\gets\const{nil}$
\li	\While $\attrib{L}{head}\ne\const{nil}$
\li		\Do $x\gets\attrib{L}{head}$
\li			$\proc{Singly-Linked-List-Delete}(L,\attrib{L}{head})$
\li			$\proc{Singly-Linked-List-Insert}(L',x)$
		\End
\li	$\attrib{L}{head}\gets\attrib{L'}{head}$
\end{codebox}

Procedury \proc{Singly-Linked-List-Insert} i~\proc{Singly-Linked-List-Delete} zostały przedstawione w~\refExercise{10.2-1}.
Ich wywołania w~powyższym algorytmie zajmują czas stały, stąd czasem działania algorytmu jest $\Theta(n)$.

\exercise %10.2-8
Zgodnie z~opisem w~treści zadania przyjmijmy, że każdy element listy zamiast wskaźników na poprzednik i~następnik posiada atrybut \id{np}, będący alternatywą wykluczającą tych dwóch wskaźników.
Załóżmy, że znamy adres elementu $x$ na liście $L$ i~elementu $y$ będącego poprzednikiem $x$ w~$L$.
Niech $z$ będzie następnikiem $x$ na tej liście.
Wówczas $\attrib{x}{np}=z\func{xor}y$, zatem na podstawie własności operacji \func{xor}, aby dostać się do $z$, wystarczy obliczyć wartość
\[
    z = z\func{xor}0 = z\func{xor}{}(y\func{xor}y) = (z\func{xor}y)\func{xor}y = \attrib{x}{np}\func{xor}y.
\]
Podobnie ma się rzecz podczas wyznaczania $y$ na podstawie $x$ i~$z$ -- wówczas $y=\attrib{x}{np}\func{xor}z$.
Zauważmy, że jeśli element $x$ jest ogonem listy, to $\attrib{x}{np}=\const{nil}\func{xor}y=0\func{xor}y=y$, skąd mamy $z=\attrib{x}{np}\func{xor}y=0=\const{nil}$ i~analogicznie dla głowy listy, z~faktu, że $\attrib{x}{np}=z\func{xor}0=z$ wynika $y=\const{nil}$.
Podobnie jak w~zwykłych listach przyjmujemy, że atrybut \attrib{L}{head} przechowuje wskaźnik do głowy listy $L$.
Ponadto z~listą związujemy atrybut \attrib{L}{tail}, który będzie wskazywał na jej ogon -- jest on konieczny do tego, aby operacja odwracania kolejności elementów na liście działała w~czasie stałym.

Poniżej przedstawiono procedury implementujące operacje \proc{Search}, \proc{Insert} i~\proc{Delete} dla takiej listy.
\begin{codebox}
\Procname{$\proc{Xor-Linked-List-Search}(L,k)$}
\li	$x\gets\attrib{L}{head}$
\li	$y\gets\const{nil}$
\li	\While $x\ne\const{nil}$ i~$\attrib{x}{key}\ne k$
\li		\Do $z\gets\attrib{x}{np}\func{xor}y$
\li			$y\gets x$
\li			$x\gets z$
		\End
\li	\Return $x$
\end{codebox}

\begin{codebox}
\Procname{$\proc{Xor-Linked-List-Insert}(L,x)$}
\li	$\attrib{x}{np}\gets\attrib{L}{head}$
\li	\If $\attrib{L}{head}\ne\const{nil}$
\li		\Then $\attribb{L}{head}{np}\gets\attribb{L}{head}{np}\func{xor}x$
		\End
\li	$\attrib{L}{head}\gets x$
\li	\If $\attrib{L}{tail}=\const{nil}$
\li		\Then $\attrib{L}{tail}\gets x$
		\End
\end{codebox}

\begin{codebox}
\Procname{$\proc{Xor-Linked-List-Delete}(L,x)$}
\li	$x'\gets\attrib{L}{head}$
\li	$y\gets\const{nil}$
\li	\While $x'\ne x$
\li		\Do $z\gets\attrib{x'}{np}\func{xor}y$
\li			$y\gets x'$
\li			$x'\gets z$
		\End
\li	$z\gets\attrib{x}{np}\func{xor}y$
\li	\If $x=\attrib{L}{head}$
\li		\Then $\attrib{L}{head}\gets z$
\li		\Else $\attrib{y}{np}\gets\attrib{y}{np}\func{xor}x\func{xor}z$
		\End
\li	\If $x=\attrib{L}{tail}$
\li		\Then $\attrib{L}{tail}\gets y$
\li		\Else $\attrib{z}{np}\gets\attrib{z}{np}\func{xor}x\func{xor}y$
		\End
\end{codebox}

Procedury \proc{Xor-Linked-List-Search} i~\proc{Xor-Linked-List-Insert} są analogiczne do swoich odpowiedników dla zwykłej listy dwukierunkowej i~pesymistyczny czas ich działania wynosi odpowiednio $\Theta(n)$ oraz $\Theta(1)$.
W~procedurze \proc{Xor-Linked-List-Delete} należy odszukać na liście poprzednika i~następnika elementu $x$, aby uaktualnić ich pola \id{np}, co w~pesymistycznym przypadku zajmuje czas $\Theta(n)$.
Odwracanie kolejności elementów listy w~opisanej tu implementacji można zrealizować poprzez zamianę wartości jej atrybutów $head$ i~$tail$.

\subchapter{Reprezentowanie struktur wskaźnikowych za pomocą tablic}

\exercise %10.3-1
\note{Oryginalna treść drugiej części zadania żąda, aby podany ciąg zilustrować jako listę dwukierunkową w~reprezentacji jednotablicowej.}

\noindent Rys.\ \ref{fig:10.3-1} przedstawia przykładowe rozmieszczenie elementów ciągu jako listy dwukierunkowej w~reprezentacji wielotablicowej i~jednotablicowej.
\begin{figure}[ht]
	\centering \begin{tikzpicture}[
	cell/.append style = {light grayed, minimum size=4mm},
	med cell/.style = {column #1/.style={nodes={med grayed}}},
	index node/.append style = {node distance=6mm and 2mm},
	outer/.append style = {node distance=10mm and 2mm}
]

\node[outer] (pic a) {
\begin{tikzpicture}
	\matrix[
		array,
		med cell/.list = {1,3,6,8}
	] (arr) {
		& 10 & &  5 &  9 & &    & & 7 & 4 \\
		& 13 & &  8 & 19 & & 11 & & 5 & 4 \\
		&    & & 10 &  4 & &  9 & & 5 & 2 \\
	};
	\draw (arr-3-2.south west) + (1mm, 1mm) -- +(3mm, 3mm);
	\draw (arr-1-7.south west) + (1mm, 1mm) -- +(3mm, 3mm);
	\draw[very thick] (arr-3-1.south west) rectangle (arr-1-10.north east);
	\foreach \x in {2, ..., 10} {
		\draw[very thick] (arr-3-\x.south west) -- (arr-1-\x.north west);
	}
	\foreach \x in {1, ..., 10} {
		\node[index node, above=of arr-1-\x] {\x};
	}
	\node[index node, left=of arr-1-1.west] {$\id{next}$};
	\node[index node, left=of arr-2-1.west] {$\id{key}$};
	\node[index node, left=of arr-3-1.west] {$\id{prev}$};
	\node[cell, above left=3mm and 4mm of arr-1-1] (L) {2};
	\node[left=of L.west] {$L$};
	
	\draw[arrow] (L) -| (arr-1-2.110);
	\draw[arrow] (arr-1-2.70) -- +(0,5mm) -| (arr-1-10.70);
	\draw[arrow] (arr-1-10.110) -- +(0,4mm) -| (arr-1-4.110);
	\draw[arrow] (arr-1-4.70) -- +(0,3mm) -| (arr-1-5.110);
	\draw[arrow] (arr-1-5.70) -- +(0,3mm) -| (arr-1-9.70);
	\draw[arrow] (arr-1-9.110) -- +(0,2mm) -| (arr-1-7);
	
	\draw[arrow] (arr-3-7.south) -- +(0,-2mm) -| (arr-3-9.250);
	\draw[arrow] (arr-3-9.290) -- +(0,-3mm) -| (arr-3-5.290);
	\draw[arrow] (arr-3-5.250) -- +(0,-3mm) -| (arr-3-4.290);
	\draw[arrow] (arr-3-4.250) -- +(0,-4mm) -| (arr-3-10.250);
	\draw[arrow] (arr-3-10.290) -- +(0,-5mm) -| (arr-3-2);
	
	\node[subpicture label, left=10mm of L] {(a)};
\end{tikzpicture}
};

\node[outer, below=of pic a] (pic b) {
\begin{tikzpicture}
	\matrix[
		array,
		med cell/.list = {4,5,6,13,14,15,19,20,21,22,23,24}
	] (arr) {5 & 10 & 16 & & & & 13 & 25 & & 11 & & 1 & & & & 19 & 1 & 28 & & & & & & & 4 & 28 & 7 & 8 & 16 & 25 \\};
	\draw (arr-1-9.south west) + (1mm, 1mm) -- +(3mm, 3mm);
	\draw (arr-1-11.south west) + (1mm, 1mm) -- +(3mm, 3mm);
	\draw[very thick] (arr-1-1.south west) rectangle (arr-1-30.north east);
	\foreach \x in {4, 7, ..., 28} {
		\draw[very thick] (arr-1-\x.south west) -- (arr-1-\x.north west);
	}
	\foreach \x in {1, ..., 30} {
		\node[index node, above=of arr-1-\x] {\x};
	}
	\node[left=of arr-1-1.west] {$A$};
	\node[cell, above left=3mm and 4mm of arr-1-1] (L) {7};
	\node[left=of L.west] {$L$};
	
	\draw[arrow] (L) -| (arr-1-7);
	\draw[arrow] (arr-1-8.north) -- +(0,2mm) -| (arr-1-25);
	\draw[arrow] (arr-1-26.north) -- +(0,2mm) -| (arr-1-28);
	\draw[arrow] (arr-1-29.north) -- +(0,3mm) -| (arr-1-16);
	\draw[arrow] (arr-1-17.north) -- +(0,4mm) -| (arr-1-1);
	\draw[arrow] (arr-1-2.north) -- +(0,3mm) -| (arr-1-10);
	
	\draw[arrow] (arr-1-12.south) -- +(0,-3mm) -| (arr-1-1);
	\draw[arrow] (arr-1-3.south) -- +(0,-4mm) -| (arr-1-16);
	\draw[arrow] (arr-1-18.south) -- +(0,-4mm) -| (arr-1-28);
	\draw[arrow] (arr-1-30.south) -- +(0,-3mm) -| (arr-1-25);
	\draw[arrow] (arr-1-27.south) -- +(0,-2mm) -| (arr-1-7);

	\draw (arr-1-16.south) + (1mm, -1mm) -- +(2mm, -7mm) node[index node, anchor=50] {$\id{key}$};
	\draw (arr-1-17.south) + (0mm, -1mm) -- +(0, -10mm) node[index node, anchor=north] {$\id{next}$};
	\draw (arr-1-18.south) + (-1mm, -1mm) -- +(-2mm, -7mm) node[index node, anchor=140] {$\id{prev}$};
	
	\node[subpicture label, left=10mm of L] {(b)};
\end{tikzpicture}
};

\end{tikzpicture}

	\caption{Ciąg $\langle13,4,8,19,5,11\rangle$ w~postaci listy dwukierunkowej w~reprezentacji tablicowej.
{\sffamily\bfseries(a)} Lista reprezentowana za pomocą tablic \id{key}, \id{next} oraz \id{prev}.
{\sffamily\bfseries(b)} Ta sama lista reprezentowana w~pojedynczej tablicy $A$.} \label{fig:10.3-1}
\end{figure}

\exercise %10.3-2
Załóżmy, że właściwa lista oraz lista wolnych pozycji znajdują się w~pojedynczej tablicy $A$.
Poniższe procedury są adaptacjami operacji przydzielania i~zwalniania pamięci dla list dwukierunkowych w~reprezentacji jednotablicowej.
Wykorzystujemy tutaj fakt, że polu \id{next} odpowiada przesunięcie 1 względem początku fragmentu tablicy $A$ przechowującego dany element listy.
\begin{codebox}
\Procname{$\proc{Single-Array-Allocate-Object}()$}
\li	\If $\id{free}=\const{nil}$
\li		\Then \Error ,,brak pamięci''
		\End
\li	$x\gets\id{free}$
\li	$\id{free}\gets A[x+1]$
\li	\Return $x$
\end{codebox}

\begin{codebox}
\Procname{$\proc{Single-Array-Free-Object}(x)$}
\li	$A[x+1]\gets\id{free}$
\li	$\id{free}\gets x$
\end{codebox}

\exercise %10.3-3
Lista wolnych pozycji jest listą jednokierunkową -- nie są więc wykorzystywane w~niej pola \id{prev}.

\exercise %10.3-4
Korzystając ze wskazówki z~treści zadania, zaimplementujemy listę wolnych pozycji jako stos.
Będzie on zajmował wszystkie pozycje na prawo od tej wskazywanej przez \id{free}, na której znajdzie się jego wierzchołek.
Stos jest pusty wtedy i~tylko wtedy, gdy $\id{free}=\const{nil}$.
Przydzielanie pamięci polega na wywołaniu na stosie wolnych pozycji operacji \proc{Pop}, a~więc procedura \proc{Compact-List-Allocate-Object} jest identyczna z~oryginalną \proc{Allocate-Object}.
Z~kolei zwalnianie pamięci to wywołanie na tym stosie operacji \proc{Push}.
Wcześniej jednak należy zamienić zwalniany element z~tym, który znajduje się bezpośrednio na lewo od wierzchołka stosu.
Poniższy pseudokod stanowi implementację tej operacji.
\begin{codebox}
\Procname{$\proc{Compact-List-Free-Object}(x)$}
\li	\If $\id{free}=\const{nil}$
\li		\Then $y\gets\attrib{key}{length}$
\li		\Else $y\gets\id{free}-1$
		\End
\li	\If $\id{next}[x]\ne\const{nil}$
\li		\Then $\id{prev}[\id{next}[x]]\gets\id{prev}[x]$
		\End
\li	\If $\id{prev}[x]\ne\const{nil}$
\li		\Then $\id{next}[\id{prev}[x]]\gets\id{next}[x]$
		\End
\li	\If $x\ne y$
\li		\Then
			\If $\id{next}[y]\ne\const{nil}$
\li				\Then $\id{prev}[\id{next}[y]]\gets x$
				\End
\li			\If $\id{prev}[y]\ne\const{nil}$
\li				\Then $\id{next}[\id{prev}[y]]\gets x$
				\End
		\End
\li	skopiuj wartości pól \id{key}, \id{next} i~\id{prev} elementu $y$ do pól elementu $x$
\li	\If $L=y$
\li		\Then $L\gets x$
		\End
\li	$\proc{Free-Object}(y)$
\end{codebox}

\exercise %10.3-5
Ogólna idea procedury \proc{Compactify-List} wygląda następująco.
Podczas przechodzenia po liście $L$ wyznaczane są te elementy, które zajmują pozycje na prawo od $m$.
Wszystkie one muszą być zamienione z~elementami listy $F$ znajdującymi się na pozycjach do $m$ włącznie.
Aby zachować czas $\Theta(m)$, procedura nie może przechodzić po liście $F$, która składa się z~$n-m$ elementów.
Zamiast tego będzie ona przeszukiwać liniowo tablice implementujące listy i~wyznaczać rekordy należące do listy wolnych pozycji.
Jednym ze sposobów rozróżniania elementów list $L$ i~$F$ jest wykorzystanie wskaźników \id{prev}.
Na początku działania procedury do pól \id{prev} wszystkich elementów listy $L$ zostanie wpisana specjalna wartość, nie będąca poprawnym indeksem tablicy, np.\ $-1$.
Tuż przed zakończeniem działania wskaźnikom \id{prev} zostaną nadane właściwe wartości, ale wpierw pola te posłużą do identyfikacji elementów listy.
\begin{codebox}
\Procname{$\proc{Compactify-List}(L,F)$}
\li	$n\gets\attrib{key}{length}$
\li	wpisz $-1$ do pól \id{prev} elementów listy $L$ i~wyznacz liczbę jej elementów $m$ \label{li:compactify-list-preprocess}
\li $x\gets L$
\li	$x'\gets\const{nil}$
\li	$y\gets1$
\li	\While $x\ne\const{nil}$ \label{li:compactify-list-while-begin}
\li		\Do
			\If $x\le m$
\li				\Then
					$x'\gets x$
\li					$x\gets\id{next}[x]$
\li				\Else
					\While $\id{prev}[y]=-1$ \label{li:compactify-list-while2-begin}
\li						\Do $y\gets y+1$
						\End \label{li:compactify-list-while2-end}
\li					zamień wartości pól \id{key}, \id{next} i~\id{prev} elementu $x$ z~polami elementu $y$ \label{li:compactify-list-swap}
\li					\If $\id{next}[x]\ne\const{nil}$
\li						\Then $\id{prev}[\id{next}[x]]\gets x$ \label{li:compactify-list-update-prev-next}
						\End
\li					\If $\id{prev}[x]\ne\const{nil}$
\li						\Then $\id{next}[\id{prev}[x]]\gets x$ \label{li:compactify-list-update-next-prev}
\li						\Else $F\gets x$ \label{li:compactify-free-list-head-update}
						\End
\li					\If $x'\ne\const{nil}$
\li						\Then $\id{next}[x']\gets y$ \label{li:compactify-list-update-predecessor-next}
\li						\Else $L\gets y$ \label{li:compactify-list-head-update}
						\End
\li					$x'\gets y$
\li					$x\gets\id{next}[y]$
				\End
		\End \label{li:compactify-list-while-end}
\li	przywróć poprawne wartości w~polach \id{prev} elementów listy $L$ \label{li:compactify-list-postprocess}
\end{codebox}

Procedura przechodzi przez listę $L$ trzy razy.
Najpierw w~wierszu \ref{li:compactify-list-preprocess} tablice przechowujące listy $L$ i~$F$ są przygotowywane do właściwego przetwarzania poprzez ustawienie pól \id{prev} wszystkich elementów listy $L$ na $-1$.
W~tym samym kroku wyznaczana jest wartość $m$ -- rozmiar listy $L$.

Kolejne przejście po liście to właściwe kompaktowanie.
Jeśli dany element $x$ listy $L$ zajmuje w~tablicach pozycję wyższą niż $m$, to zostaje zamieniony z~elementem $y$ listy $F$ znajdującym się najbardziej na lewo.
Pętla \kw{while} w~wierszach \doubledash{\ref{li:compactify-list-while2-begin}}{\ref{li:compactify-list-while2-end}} wyznacza $y$ poprzez liniowe przeglądanie tablicy \id{prev}, po czym w~wierszu \ref{li:compactify-list-swap} elementy $x$ i~$y$ zamieniają swoje pozycje.
Należy jeszcze uaktualnić pola \id{prev} i~\id{next} elementów sąsiednich na liście $F$ (wiersze \ref{li:compactify-list-update-prev-next} i~\ref{li:compactify-list-update-next-prev}), pole \id{next} poprzednika $x$ na liście $L$ (wiersz \ref{li:compactify-list-update-predecessor-next}), jak również wskaźniki $L$ i~$F$, w~przypadku gdy wśród zamienianych elementów była głowa którejś z~list (wiersze \ref{li:compactify-free-list-head-update} oraz \ref{li:compactify-list-head-update}).

Ostatnim krokiem procedury jest ponowne przejście przez listę $L$ w~wierszu \ref{li:compactify-list-postprocess} i~przywrócenie odpowiednich wartości w~polach \id{prev} wszystkich jej elementów.

Po wykonaniu procedury elementy z~listy wolnych pozycji nie zawsze będą umieszczone w~tablicach według ich kolejności na tej liście.
Z~tego powodu na liście przetworzonej procedurą \proc{Compactify-List} nie możemy korzystać z~procedur \proc{Compact-List-Allocate-Object} i~\proc{Compact-List-Free-Object} z~\refExercise{10.3-4}, które zakładają, że reprezentacja tablicowa listy wolnych pozycji stanowi stos.

\subchapter{Reprezentowanie drzew (ukorzenionych)}

\exercise %10.4-1
Drzewo binarne, którego reprezentacją są podane tablice, przedstawiono na rys.\ \ref{fig:10.4-1}.
Węzły o~kluczach 14 i~15 nie należą do drzewa.
\begin{figure}[!ht]
	\centering \begin{tikzpicture}[
	every node/.style = {tree node, anchor=center},
	every label/.style = {index node, draw=none, fill=none}
]
\node[label=6] {18}
	child {node[label=1] {12}
		child {node[label=7] {7}}
		child {node[label=3] {4}
			child {node[label=10] {5}}
			child[missing]
		}
	}
	child {node[label=4] {10}
		child {node[label=5] {2}}
		child {node[label=9] {21}}
	};

\end{tikzpicture}

	\caption{Drzewo binarne o~korzeniu o~indeksie 6 reprezentowane przez tablice \id{key}, \id{left} i~\id{right}.} \label{fig:10.4-1}
\end{figure}

\exercise %10.4-2
Szukany algorytm został opisany w~Podręczniku w~podrozdziale 12.1 jako procedura \proc{Inorder-Tree-Walk}.
Czas działania tego algorytmu dla drzewa o~$n$ węzłach wynosi $\Theta(n)$ -- mówi o~tym tw.\ 12.1 z~Podręcznika.

\exercise %10.4-3
Przedstawiona poniżej procedura stanowi nierekurencyjną implementację algorytmu przechodzenia drzewa metodą preorder (patrz podrozdział 12.1).
Do emulowania rekursji wykorzystywany jest stos.
Każdy węzeł drzewa jest dokładnie raz wstawiany na stos i~dokładnie raz z~niego usuwany, stąd czas działania tej procedury dla drzewa o~$n$ węzłach wynosi $\Theta(n)$.
\begin{codebox}
\Procname{$\proc{Iterative-Preorder-Tree-Walk}(T)$}
\li	\If $\attrib{T}{root}=\const{nil}$
\li		\Then \Return
		\End
\li	$\proc{Push}(S,\attrib{T}{root})$
\li	\While $\proc{Stack-Empty}(S)=\const{false}$
\li		\Do $x\gets\proc{Pop}(S)$
\li			wypisz \attrib{x}{key}
\li			\If $\attrib{x}{right}\ne\const{nil}$
\li				\Then $\proc{Push}(S,\attrib{x}{right})$
				\End
\li			\If $\attrib{x}{left}\ne\const{nil}$
\li				\Then $\proc{Push}(S,\attrib{x}{left})$
				\End
		\End
\end{codebox}

\exercise %10.4-4
Nasza procedura będzie przyjmować węzeł $x$ drzewa w~reprezentacji ,,na lewo syn, na prawo brat'' i~wypisywać wszystkie klucze z~poddrzewa o~korzeniu w~$x$.
Jeśli $x\ne\const{nil}$, to zostanie wypisany klucz węzła $x$ i~procedura zostanie wywołana rekurencyjnie najpierw dla najbardziej lewego syna $x$, a~następnie dla kolejnego brata $x$.
\begin{codebox}
\Procname{$\proc{Tree-Walk}(x)$}
\li	\If $x\ne\const{nil}$
\li		\Then wypisz \attrib{x}{key}
\li			$\proc{Tree-Walk}(\attrib{x}{left-child})$
\li			$\proc{Tree-Walk}(\attrib{x}{right-sibling})$
		\End
\end{codebox}
Aby wypisać wszystkie klucze drzewa $T$ w~reprezentacji ,,na lewo syn, na prawo brat'', należy wywołać $\proc{Tree-Walk}(\attrib{T}{root})$.

Każdy węzeł drzewa jest wypisywany w~dokładnie jednym wywołaniu rekurencyjnym.
Procedura wywoływana jest także dla wszystkich pustych najbardziej lewych synów i~dla wszystkich pustych braci znajdujących się w~drzewie.
Stąd wnioskujemy, że procedura wypisze wszystkie $n$ kluczy drzewa w~czasie $\Theta(n)$.

\exercise %10.4-5
W~algorytmie będziemy symulować przechodzenie drzewa w~porządku inorder, używając trzech wskaźników -- \id{curr} będący aktualnie odwiedzanym węzłem oraz \id{prev} i~\id{next} -- odpowiednio, poprzednio przetwarzanym i~kolejnym do przetworzenia węzłem.
Odwiedzenie danego węzła $x$ będzie realizowane przez pomocniczą procedurę $\proc{Stackless-Inorder-Visit}(x)$.
Polega ono na wypisaniu klucza $x$ oraz wyznacza kolejny węzeł do przetworzenia.
\begin{codebox}
\Procname{$\proc{Stackless-Inorder-Visit}(x)$}
\li	wypisz \attrib{x}{key}
\li	\If $\attrib{x}{right}\ne\const{nil}$
\li		\Then \Return \attrib{x}{right}
\li		\Else \Return \attrib{x}{p}
		\End
\end{codebox}

Algorytm zapisujemy w~postaci pseudokodu:
\begin{codebox}
\Procname{$\proc{Stackless-Inorder-Tree-Walk}(T)$}
\li	$\id{prev}\gets\const{nil}$
\li	$\id{curr}\gets\attrib{T}{root}$
\li	\While $\id{curr}\ne\const{nil}$
\li		\Do \If $\id{prev}=\attrib{curr}{p}$
\li				\Then \If $\attrib{curr}{left}\ne\const{nil}$
\li						\Then $\id{next}\gets\attrib{curr}{left}$ \label{li:stackless-inorder-tree-walk-go-left}
\li						\Else $\id{next}\gets\proc{Stackless-Inorder-Visit}(\id{curr})$ \label{li:stackless-inorder-tree-walk-visit1}
						\End
\li				\ElseIf $\id{prev}=\attrib{curr}{left}$
\li					\Then $\id{next}\gets\proc{Stackless-Inorder-Visit}(\id{curr})$ \label{li:stackless-inorder-tree-walk-visit2}
\li				\ElseNoIf $\id{next}\gets\attrib{curr}{p}$ \label{li:stackless-inorder-tree-walk-go-up}
				\End
\li			$\id{prev}\gets\id{curr}$ \label{li:stackless-inorder-tree-walk-update-prev}
\li			$\id{curr}\gets\id{next}$ \label{li:stackless-inorder-tree-walk-update-curr}
		\End
\end{codebox}
Kolejne węzły, przez które przechodzimy w~algorytmie, wyznaczane są na podstawie wzajemnej relacji między węzłami \id{prev} i~\id{curr}.
Gdy \id{prev} jest ojcem \id{curr}, to aktualnie schodzimy w~dół drzewa po lewych synach (wiersz \ref{li:stackless-inorder-tree-walk-go-left}).
W~przypadku osiągnięcia liścia, odwiedzamy go, po czym przechodzimy do jego prawego poddrzewa albo zawracamy w~kierunku korzenia (wiersz \ref{li:stackless-inorder-tree-walk-visit1}).
Gdy \id{prev} jest lewym synem \id{curr}, to wracamy w~górę drzewa, odwiedzając napotykane węzły i~przechodząc ich prawe poddrzewa (wiersz \ref{li:stackless-inorder-tree-walk-visit2}).
Jeśli wreszcie \id{prev} jest prawym synem węzła \id{curr}, to wracamy w~górę drzewa z~właśnie odwiedzonego prawego poddrzewa (wiersz \ref{li:stackless-inorder-tree-walk-go-up}).
W~wierszach \ref{li:stackless-inorder-tree-walk-update-prev} i~\ref{li:stackless-inorder-tree-walk-update-curr} następuje aktualizacja wskaźników \id{prev} i~\id{curr}, po czym algorytm kontynuuje swoje działanie, aż $\id{curr}=\const{nil}$.

Każdą krawędzią drzewa przejdziemy dokładnie 2 razy -- poruszając się najpierw w~dół, a~później w~górę drzewa.
Odwiedzenie węzła $x$ następuje w~dwóch przypadkach -- gdy docieramy do niego od jego ojca i~$x$ nie ma lewego syna lub wtedy, gdy wracamy w~górę drzewa z~lewego syna $x$.
Algorytm odwiedzi zatem każdy węzeł dokładnie raz, działając w~czasie $\Theta(n)$.

\exercise %10.4-6
\note{W~oryginalnej treści zadania szukana jest taka reprezentacja drzewa, która umożliwi wyznaczanie i~uzyskiwanie dostępu do ojca danego węzła \textbf{lub} wszystkich jego synów w~czasie proporcjonalnym do liczby synów.}

\noindent Opiszemy modyfikacje, jakie należy wprowadzić w~reprezentacji ,,na lewo syn, na prawo brat'', aby spełnić wymaganie z~treści zadania.
Ponieważ nie wymagamy dostępu do ojca danego węzła w~stałym czasie, to możemy zrezygnować z~atrybutu $p$.
Zauważmy ponadto, że w~reprezentacji ,,na lewo syn, na prawo brat'', jeśli węzeł $x$ jest korzeniem drzewa albo najbardziej na prawo położonym synem swojego ojca, to $\attrib{x}{right-sibling}=\const{nil}$.
Wykorzystamy ten wskaźnik w~węźle $x$, pokazując nim na ojca $x$ (albo \const{nil}, jeśli $x$ jest korzeniem).
Aby móc jednoznacznie określać, czy węzeł wskazywany przez ten atrybut jest bratem, czy ojcem $x$, wykorzystamy dodatkowe pole -- zmienną boolowską.
Nazwijmy następująco atrybuty każdego węzła $x$ w~nowej reprezentacji:
\begin{itemize}
	\item \attrib{x}{left-child} -- wskazuje na najbardziej lewego syna $x$ albo \const{nil}, jeśli $x$ jest liściem (identyczny z~\attrib{x}{left-child} z~reprezentacji ,,na lewo syn, na prawo brat'');
	\item \attrib{x}{next} -- wskazuje na kolejnego brata $x$ albo na ojca $x$, jeśli $x$ jest korzeniem drzewa lub najbardziej na prawo wysuniętym synem swojego ojca;
	\item \attrib{x}{last-sibling} -- zmienna boolowska przyjmująca wartość \const{true}, jeśli węzeł $x$ jest korzeniem lub najbardziej na prawo wysuniętym synem swojego ojca i~\const{false} w~przeciwnym przypadku.
\end{itemize}

Dzięki tak zdefiniowanym atrybutom, dla danego węzła $x$ możemy wyznaczyć wszystkich jego synów poprzez przejście do węzła \attrib{x}{left-child}, a~następnie poruszając się po wskaźnikach \id{next} kolejnych synów $x$.
Każdy napotykany węzeł $y$ pokazuje wskaźnikiem \attrib{y}{next} na swojego brata po prawej stronie, o~ile $\attrib{y}{last-sibling}=\const{false}$.
W~momencie dotarcia do węzła $y$, dla którego $\attrib{y}{last-sibling}=\const{true}$, odwiedzimy wszystkich synów węzła $x$.
Podobnie, aby ustalić ojca $x$, przechodzimy po wskaźnikach \id{next} braci $x$ znajdujących się po jego prawej stronie, aż do napotkania węzła $z$, dla którego $\attrib{z}{last-sibling}=\const{true}$.
Warunek ten oznacza, że węzeł $\attrib{z}{next}$, o~ile istnieje, jest ojcem zarówno $z$, jak i~$x$.
Jeśli z~kolei $\attrib{z}{next}=\const{nil}$, to $x$ jest korzeniem drzewa.

Podana implementacja pozwala na wyznaczenie wszystkich synów danego węzła w~czasie proporcjonalnym do ich liczby oraz jego ojca w~czasie proporcjonalnym do liczby braci danego węzła.


\problems

\problem{Porównanie list} %10-1
Tabela \ref{tab:10-1} zawiera pesymistyczne czasy poszczególnych operacji słownikowych dla danych czterech typów list.
Przyjmujemy, że operacje wykonywane są na listach o~rozmiarach $n$.

\begin{table}[ht]
	\begin{center}
		\[
			\begin{array}{l|c|c|c|c}
				& \text{Nieposortowana} & \text{Posortowana} & \text{Nieposortowana} & \text{Posortowana} \\
				& \text{jedno-} & \text{jedno-} & \text{dwu-} & \text{dwu-} \\
				& \text{kierunkowa} & \text{kierunkowa} & \text{kierunkowa} & \text{kierunkowa} \\
				\hline
				\proc{Search}(L,k) & \Theta(n) & \Theta(n) & \Theta(n) & \Theta(n) \\
				\hline
				\proc{Insert}(L,x) & \Theta(1) & \Theta(n) & \Theta(1) & \Theta(n) \\
				\hline
				\proc{Delete}(L,x) & \Theta(n) & \Theta(n) & \Theta(1) & \Theta(1) \\
				\hline
				\proc{Successor}(L,x) & \Theta(n) & \Theta(1) & \Theta(n) & \Theta(1) \\
				\hline
				\proc{Predecessor}(L,x) & \Theta(n) & \Theta(n) & \Theta(n) & \Theta(1) \\
				\hline
				\proc{Minimum}(L) & \Theta(n) & \Theta(1) & \Theta(n) & \Theta(1) \\
				\hline
				\proc{Maximum}(L) & \Theta(n) & \Theta(n) & \Theta(n) & \Theta(n)
			\end{array}
		\]
	\end{center}
	\caption{Porównanie pesymistycznych złożoności operacji słownikowych dla różnych typów list.} \label{tab:10-1}
\end{table}
Jeśli w~implementacjach list będziemy dodatkowo utrzymywać atrybut \id{tail} wskazujący na ogon listy, to operację \proc{Maximum} dla list posortowanych możemy wykonywać w~czasie stałym.

\problem{Listowa reprezentacja kopców złączalnych} %10-2

\subproblem %10-2(a)
Kopiec zaimplementujemy jako posortowaną listę jednokierunkową.
Operacja \proc{Make-Heap} tworzy pustą listę, co zajmuje oczywiście czas stały.
Dodanie elementu do kopca polega na dodaniu go do listy.
Aby zachować jej uporządkowanie, musimy odnaleźć miejsce, które zajmie nowy element, co w~pesymistycznym przypadku wymaga czasu $\Theta(n)$.
Dzięki uporządkowaniu listy można zaimplementować operacje \proc{Minimum} i~\proc{Extract-Min} działające w~czasie $\Theta(1)$.
Z~kolei stosując algorytm opisany w~\refExercise{6.5-8} w~wersji dla list jednokierunkowych, a~następnie przechodząc po scalonej liście w~celu usunięcia powtarzających się elementów, jesteśmy w~stanie zaimplementować operację \proc{Union} w~czasie $\Theta(n)$, gdzie $n$ jest liczbą elementów na wynikowej liście.

\subproblem %10-2(b)
W~tym przypadku także wystarczy nam lista jednokierunkowa.
Zarówno utworzenie pustego kopca, jak i~dodanie do niego nowego elementu odbywa się w~czasie stałym, ale odszukanie oraz usunięcie minimalnego elementu wiąże się z~przeszukaniem całej listy, co zajmuje czas liniowy względem rozmiaru listy.
Możemy jednak wprowadzić usprawnienie, dzięki któremu operacja \proc{Minimum} będzie działać w~czasie stałym.
Będziemy mianowicie przechowywać minimalny element listy w~jej głowie.
Podczas dodawania nowego elementu wystarczy porównać ten element z~głową listy (o~ile istnieje) i~jeśli stanowi on nowe minimum, umieścić go w~głowie listy albo, w~przeciwnym przypadku, tuż za nią.
Operacja \proc{Extract-Min} po usunięciu głowy nadal jednak musi przeszukać całą listę w~celu odszukania aktualnego minimum i~umieszczenia go w~głowie listy.

Podczas operacji \proc{Union}, musimy pamiętać o~tym, że łączone listy nie zawsze reprezentują rozłączne zbiory, a~także o~tym, że w~głowie wynikowej listy powinien znaleźć się najmniejszy element z~obu łączonych list.
Efektywna implementacja spełniająca oba warunki będzie polegać na wywołaniu operacji \proc{Union} z~punktu (a) po uprzednim posortowaniu obu list, co zrealizujemy za pomocą algorytmu sortowania przez scalanie w~wersji dla list jednokierunkowych.
Do zaimplementowania pomocniczej procedury \proc{Merge} możemy użyć algorytmu przedstawionego w~\refExercise{6.5-8}.
Ostatecznie łączenie list jesteśmy w~stanie zaimplementować w~czasie $\Theta(n\lg n)$, gdzie $n$ jest rozmiarem wynikowej listy.

\subproblem %10-2(c)
Przypadek jest identyczny z~poprzednim, ale tym razem nie jest wymagane wykrywanie powtórzeń podczas operacji \proc{Union}.
Łączenie list możemy więc zaimplementować jako ich konkatenację, to znaczy ustawienie głowy jednej listy jako elementu następującego po ogonie drugiej.
To, w~jakiej kolejności skleimy listy, będzie zależeć od tego, której głowa przechowuje mniejszy element, co ma na celu spełnienie usprawnienia opisanego w~poprzednim punkcie.
Operację łączenia możemy zrealizować w~czasie stałym, jeśli dla listy w~tej reprezentacji kopca będziemy pamiętać wskaźnik na jej ogon.

\bigskip
\noindent Zestawienie czasów działania poszczególnych operacji w~przypadkach pesymistycznych dla omawianych reprezentacji listowych kopców złączalnych przedstawiono w~tabeli \ref{tab:10-2}.

\begin{table}[ht]
	\begin{center}
		\[
			\begin{array}{l|c|c|c}
				& \text{Listy posortowane} & \text{Listy nieposortowane} & \text{Listy nieposortowane,} \\
				&  &  & \text{rozłączne zbiory w~\proc{Union}} \\
				\hline
				\proc{Make-Heap} & \Theta(1) & \Theta(1) & \Theta(1) \\
				\hline
				\proc{Insert} & \Theta(n) & \Theta(1) & \Theta(1) \\
				\hline
				\proc{Minimum} & \Theta(1) & \Theta(1) & \Theta(1) \\
				\hline
				\proc{Extract-Min} & \Theta(1) & \Theta(n) & \Theta(n) \\
				\hline
				\proc{Union} & \Theta(n) & \Theta(n\lg n) & \Theta(1)
			\end{array}
		\]
	\end{center}
	\caption{Porównanie pesymistycznych czasów operacji słownikowych dla reprezentacji listowych kopców złączalnych.
Dla operacji \proc{Union} $n$ oznacza rozmiar zbioru po połączeniu.} \label{tab:10-2}
\end{table}

\problem{Wyszukiwanie na posortowanej liście zajmującej spójny obszar pamięci (liście upakowanej)} %10-3
\note{W~pseudokodzie procedury \proc{Compact-List-Search} jest błąd -- w~linijce 4 zamiast testowania, czy\/ $\id{key}[j]<k$, powinno być sprawdzenie, czy\/ $\id{key}[j]\le k$.
Ponadto w~wywołaniach procedur \proc{Compact-List-Search} i~\proc{Compact-List-Search}$'$ na listach ich argumentów brakuje\/ $n$ jako drugiego parametru.}

\subproblem %10-3(a)
W~żadnej z~procedur nigdy nie zostanie wykonany skok na losowo wybraną pozycję bliżej głowy listy ani na pozycję zawierającą element o~kluczu większym niż $k$ -- gwarantują to warunki w~wierszach 4 obu algorytmów.
Ponadto w~każdej iteracji w~pętlach \kw{while} obu procedur następuje przesunięcie indeksu $i$ o~jedną pozycję do przodu.
Skoki te wykonywane są, dopóki indeks $i$ nie dotrze na pozycję elementu o~kluczu większym lub równym $k$, bądź nie osiągnie końca listy $L$.
A~zatem oba wywołania zwrócą ten sam wynik.

Zauważmy, że podczas ustalonej iteracji pętli \kw{while} w~procedurze \proc{Compact-List-Search} indeks $i$ nie jest bliżej głowy listy $L$ niż ten sam indeks podczas tej samej iteracji pętli \kw{for} w~procedurze \proc{Compact-List-Search}$'$.
Pętla pierwszego algorytmu nie zakończy się więc później niż pętla drugiego algorytmu, która to wykona dokładnie $t$ iteracji.
A~zatem łączna liczba iteracji pętli \kw{for} i~\kw{while} w~procedurze \proc{Compact-List-Search}$'$ wynosi co najmniej $t$.

\subproblem %10-3(b)
W~wywołaniu \proc{Compact-List-Search}$'(L,n,k,t)$ zostanie wykonanych nie więcej niż $t$ iteracji pętli \kw{for}, czyli etap ten wymaga czasu $O(t)$.
Na podstawie definicji zmiennej losowej $X_t$ otrzymujemy z~kolei, że średnia liczba wykonanych iteracji pętli \kw{while} wynosi $\E(X_t)$.
A~zatem oczekiwanym czasem działania procedury \proc{Compact-List-Search}$'(L,n,k,t)$ jest $O(t+\E(X_t))$.

\subproblem %10-3(c)
Oznaczmy przez $s$ indeks klucza $k$ na liście $L$, a~przez $j_1$, $j_2$, \dots, $j_t$ -- ciąg liczb całkowitych wyznaczonych przez $t$ wywołań $\proc{Random}(1,n)$.
Ponadto niech $\pi$ będzie funkcją przypisującą indeksowi elementu na liście $L$ jego rzeczywistą pozycję na tej liście wyznaczoną na podstawie łańcucha wskaźników \id{next}.
Po $t$ iteracjach pętli \kw{for} odległość między szukanym kluczem a~elementem o~indeksie $i$ (mierzona długością łańcucha wskaźników \id{next}) będzie większa lub równa $r$, jeśli dla każdego $q=1$, 2, \dots, $t$ spełnione będzie $\pi(j_q)\le\pi(s)-r$ lub $\pi(j_q)>\pi(s)$.
Mamy więc
\[
	\Pr(X_t\ge r) = \prod_{q=1}^t\Pr(\pi(j_q)\le\pi(s)-r\;\;\text{lub}\;\;\pi(j_q)>\pi(s)) = \prod_{q=1}^t\frac{n-r}{n} = \biggl(1-\frac{r}{n}\biggr)^t.
\]
Korzystając z~tożsamości (C.24) i~z~powyższego oszacowania, otrzymujemy
\[
	\E(X_t) = \sum_{r=1}^\infty\Pr(X_t\ge r) = \sum_{r=1}^n\Pr(X_t\ge r) = \sum_{r=1}^n\biggl(1-\frac{r}{n}\biggr)^t.
\]

\subproblem %10-3(d)
Sumę ograniczamy całką, otrzymując:
\[
	\sum_{r=0}^{n-1}r^t \le \int_0^nx^t\,dx = \biggl[\frac{x^{t+1}}{t+1}\biggr]_0^n = \frac{n^{t+1}}{t+1}.
\]

\subproblem %10-3(e)
Na podstawie rozwiązań punktów (c) i~(d) mamy:
\[
	\E(X_t) = \sum_{r=1}^n\biggl(1-\frac{r}{n}\biggr)^t = \sum_{r=1}^{n-1}\biggl(\frac{n-r}{n}\biggr)^t = \sum_{r=1}^{n-1}\biggl(\frac{r}{n}\biggr)^t = \frac{\sum_{r=0}^{n-1}r^t}{n^t} \le \frac{\frac{n^{t+1}}{t+1}}{n^t} = \frac{n}{t+1}.
\]

\subproblem %10-3(f)
Na mocy punktów (b) i~(e) dostajemy, że oczekiwany czas działania procedury \proc{Compact-List-Search}$'(L,n,k,t)$ wynosi $O(t+\E(X_t))=O(t+n/(t+1))=O(t+n/t)$.

\subproblem %10-3(g)
Oznaczmy przez $p$ pozycję listy $L$, na którą został wykonany ostatni skok w~wierszu 5 procedury \proc{Compact-List-Search}.
Zauważmy, że w~procedurze \proc{Compact-List-Search} po wykonaniu tego skoku zostanie wykonanych co najwyżej $t$ operacji $i\gets\id{next}[i]$.
Z~kolei pozycja na liście $L$, będąca celem ostatniego skoku z~wiersza 5 w~\proc{Compact-List-Search}$'$, nie może znajdować się bliżej głowy listy niż pozycja $p$.
Na tej podstawie wnioskujemy, że liczba wykonanych operacji $i\gets\id{next}[i]$ w~pętli \kw{while} algorytmu \proc{Compact-List-Search}$'$ również nie przekroczy $t$.
Oznacza to, że etap, który na podstawie poprzedniego punktu zajmuje czas $O(n/t)$, będzie w~rzeczywistości działać w~czasie $O(t)$.
Stąd $O(n/t)=O(t)$, czyli $t=O(\!\sqrt{n})$, a~więc czas działania procedury \proc{Compact-List-Search} wynosi $O(\!\sqrt{n})$.

Wynik ten jest prawdziwy także w~przypadku, gdy elementu o~kluczu $k$ nie ma na liście.
Wystarczy bowiem zdefiniować zmienną losową $X_t$ jako odległość na liście (mierzoną długością łańcucha wskaźników \id{next}) od pozycji $i$ do pozycji elementu o~kluczu większym niż $k$ albo do pozycji bezpośrednio za ogonem listy, w~przypadku, gdy taki element nie istnieje.
Analizę w~punktach \doubledash{(b)}{(g)} dla nowej definicji zmiennej $X_t$ przeprowadza się analogicznie.

\subproblem %10-3(h)
Jeśli dopuścimy, aby elementy na liście powtarzały się, to może dojść do sytuacji, w~której procedura próbuje wykonać skok na pozycję $j$ bliższą szukanemu elementowi niż pozycja $i$, ale nie wykonuje go, gdyż stwierdza w~linii 4, że $\id{key}[i]=\id{key}[j]$.
Jeśli każdy losowy skok będzie eliminowany na tej podstawie, to działanie procedury sprowadzi się do działania zwykłego algorytmu wyszukiwania na posortowanej liście.


%\chapter{Tablice z~haszowaniem}

\input{chapter11/sc11.1}
\subchapter{Tablice z~haszowaniem}

\exercise %11.2-1
Dla kluczy $k$, $l$ ($k\ne l$), definiujemy zmienną losową $X_{kl}=\I(h(k)=h(l))$.
Przy założeniu o~prostym równomiernym haszowaniu mamy $\Pr(h(k)=h(l))=1/m$ i~na podstawie lematu 5.1 otrzymujemy $\E(X_{kl})=1/m$.
Niech $X$ będzie zmienną losową oznaczającą liczbę kolizji w~tablicy $T$, czyli
\[
    X = \sum_{k=1}^{n-1}\sum_{l=k+1}^nX_{kl}.
\]
Oczekiwana liczba kolizji wynosi
\begin{align*}
	\E(X) &= \E\biggl(\sum_{k=1}^{n-1}\sum_{l=k+1}^nX_{kl}\biggr) = \sum_{k=1}^{n-1}\sum_{l=k+1}^n\E(X_{kl}) \\[1mm]
	&= \sum_{k=1}^{n-1}\sum_{l=k+1}^n\frac{1}{m} = \frac{1}{m}\sum_{k=1}^{n-1}(n-k) = \frac{1}{m}\sum_{k=1}^{n-1}k = \frac{n(n-1)}{2m}.
\end{align*}

\exercise %11.2-2
Ciąg wstawień elementów do tablicy z~haszowaniem został zobrazowany na rys.\ \ref{fig:11.2-2}.
\medskip
\begin{figure}[ht!]
	\centering \begin{tikzpicture}[
	array/.append style = {nodes={light grayed, inner sep=2pt, anchor=center, minimum width=7mm}},
	double cell/.style = {array, nodes={inner sep=1pt, minimum size=4.5mm}, node distance=5mm},
	med cell/.style = {row #1/.style={nodes={med grayed}}},
	outer/.append style = {node distance=10mm and 5.5mm}
]

\newcommand\arrayindexes{%
	\foreach[count=\i] \x in {0, ..., 8} {
		\node[index node, left=1mm of arr-\i-1] {\x};
	}
	\node[above=2mm of arr-1-1.north] {$T$};
}

\node[outer] (pic a) {
\begin{tikzpicture}
	\matrix[
		array,
		med cell/.list = {1,2,3,4,5,7,8,9}
	] (arr) { \\ \\ \\ \\ \\ \\ \\ \\ \\};
	\arrayindexes
	\foreach \x in {1,2,3,4,5,7,8,9} {
		\draw (arr-\x-1.south west) + (2mm, 1mm) -- +(5mm, 4mm);
	}
	
	\matrix[double cell, right=of arr-6-1] (cell 6) {5 & \\};
	\draw[arrow] (arr-6-1.center) -- (cell 6-1-1);
	\draw (cell 6-1-2.south west) + (1mm, 1mm) -- +(3.5mm, 3.5mm);
\end{tikzpicture}
};

\node[outer, right=of pic a] (pic b) {
\begin{tikzpicture}
	\matrix[
		array,
		med cell/.list = {1,3,4,5,7,8,9}
	] (arr) { \\ \\ \\ \\ \\ \\ \\ \\ \\};
	\arrayindexes
	\foreach \x in {1,3,4,5,7,8,9} {
		\draw (arr-\x-1.south west) + (2mm, 1mm) -- +(5mm, 4mm);
	}
	
	\foreach \x/\val in {2/28, 6/5} {
		\matrix[double cell, right=of arr-\x-1, ampersand replacement=\&] (cell \x) {\val \& \\};
		\draw[arrow] (arr-\x-1.center) -- (cell \x-1-1);
	}
	\foreach \x in {2,6} {
		\draw (cell \x-1-2.south west) + (1mm, 1mm) -- +(3.5mm, 3.5mm);
	}
\end{tikzpicture}
};

\node[outer, right=of pic b] (pic c) {
\begin{tikzpicture}
	\matrix[
		array,
		med cell/.list = {1,3,4,5,7,8,9}
	] (arr) { \\ \\ \\ \\ \\ \\ \\ \\ \\};
	\arrayindexes
	\foreach \x in {1,3,4,5,7,8,9} {
		\draw (arr-\x-1.south west) + (2mm, 1mm) -- +(5mm, 4mm);
	}
	
	\foreach \x/\val in {2/19, 6/5} {
		\matrix[double cell, right=of arr-\x-1, ampersand replacement=\&] (cell \x) {\val \& \\};
		\draw[arrow] (arr-\x-1.center) -- (cell \x-1-1);
	}
	\foreach \x/\val in {2/28} {
		\matrix[double cell, right=of cell \x-1-2, ampersand replacement=\&] (cell \x a) {\val \& \\};
		\draw[arrow] (cell \x-1-2.center) -- (cell \x a-1-1);
	}
	\foreach \x in {2a,6} {
		\draw (cell \x-1-2.south west) + (1mm, 1mm) -- +(3.5mm, 3.5mm);
	}
\end{tikzpicture}
};

\node[outer, right=of pic c] (pic d) {
\begin{tikzpicture}
	\matrix[
		array,
		med cell/.list = {1,3,4,5,8,9}
	] (arr) { \\ \\ \\ \\ \\ \\ \\ \\ \\};
	\arrayindexes
	\foreach \x in {1,3,4,5,8,9} {
		\draw (arr-\x-1.south west) + (2mm, 1mm) -- +(5mm, 4mm);
	}
	
	\foreach \x/\val in {2/19, 6/5, 7/15} {
		\matrix[double cell, right=of arr-\x-1, ampersand replacement=\&] (cell \x) {\val \& \\};
		\draw[arrow] (arr-\x-1.center) -- (cell \x-1-1);
	}
	\foreach \x/\val in {2/28} {
		\matrix[double cell, right=of cell \x-1-2, ampersand replacement=\&] (cell \x a) {\val \& \\};
		\draw[arrow] (cell \x-1-2.center) -- (cell \x a-1-1);
	}
	\foreach \x in {2a,6,7} {
		\draw (cell \x-1-2.south west) + (1mm, 1mm) -- +(3.5mm, 3.5mm);
	}
\end{tikzpicture}
};

\node[outer, below=of pic a.south west, anchor=north west] (pic e) {
\begin{tikzpicture}
	\matrix[
		array,
		med cell/.list = {1,4,5,8,9}
	] (arr) { \\ \\ \\ \\ \\ \\ \\ \\ \\};
	\arrayindexes
	\foreach \x in {1,4,5,8,9} {
		\draw (arr-\x-1.south west) + (2mm, 1mm) -- +(5mm, 4mm);
	}
	
	\foreach \x/\val in {2/19, 3/20, 6/5, 7/15} {
		\matrix[double cell, right=of arr-\x-1, ampersand replacement=\&] (cell \x) {\val \& \\};
		\draw[arrow] (arr-\x-1.center) -- (cell \x-1-1);
	}
	\foreach \x/\val in {2/28} {
		\matrix[double cell, right=of cell \x-1-2, ampersand replacement=\&] (cell \x a) {\val \& \\};
		\draw[arrow] (cell \x-1-2.center) -- (cell \x a-1-1);
	}
	\foreach \x in {2a,3,6,7} {
		\draw (cell \x-1-2.south west) + (1mm, 1mm) -- +(3.5mm, 3.5mm);
	}
\end{tikzpicture}
};

\node[outer, right=14mm of pic e] (pic f) {
\begin{tikzpicture}
	\matrix[
		array,
		med cell/.list = {1,4,5,8,9}
	] (arr) { \\ \\ \\ \\ \\ \\ \\ \\ \\};
	\arrayindexes
	\foreach \x in {1,4,5,8,9} {
		\draw (arr-\x-1.south west) + (2mm, 1mm) -- +(5mm, 4mm);
	}

	\foreach \x/\val in {2/19, 3/20, 6/5, 7/33} {
		\matrix[double cell, right=of arr-\x-1, ampersand replacement=\&] (cell \x) {\val \& \\};
		\draw[arrow] (arr-\x-1.center) -- (cell \x-1-1);
	}
	\foreach \x/\val in {2/28, 7/15} {
		\matrix[double cell, right=of cell \x-1-2, ampersand replacement=\&] (cell \x a) {\val \& \\};
		\draw[arrow] (cell \x-1-2.center) -- (cell \x a-1-1);
	}
	\foreach \x in {2a,3,6,7a} {
		\draw (cell \x-1-2.south west) + (1mm, 1mm) -- +(3.5mm, 3.5mm);
	}
\end{tikzpicture}
};

\node[outer, below=of pic d.south west, anchor=north west] (pic g) {
\begin{tikzpicture}
	\matrix[
		array,
		med cell/.list = {1,5,8,9}
	] (arr) { \\ \\ \\ \\ \\ \\ \\ \\ \\};
	\arrayindexes
	\foreach \x in {1,5,8,9} {
		\draw (arr-\x-1.south west) + (2mm, 1mm) -- +(5mm, 4mm);
	}

	\foreach \x/\val in {2/19, 3/20, 4/12, 6/5, 7/33} {
		\matrix[double cell, right=of arr-\x-1, ampersand replacement=\&] (cell \x) {\val \& \\};
		\draw[arrow] (arr-\x-1.center) -- (cell \x-1-1);
	}
	\foreach \x/\val in {2/28, 7/15} {
		\matrix[double cell, right=of cell \x-1-2, ampersand replacement=\&] (cell \x a) {\val \& \\};
		\draw[arrow] (cell \x-1-2.center) -- (cell \x a-1-1);
	}
	\foreach \x in {2a,3,4,6,7a} {
		\draw (cell \x-1-2.south west) + (1mm, 1mm) -- +(3.5mm, 3.5mm);
	}
\end{tikzpicture}
};

\node[outer, below=of pic e.south, anchor=north west] (pic h) {
\begin{tikzpicture}
	\matrix[
		array,
		med cell/.list = {1,5,8}
	] (arr) { \\ \\ \\ \\ \\ \\ \\ \\ \\};
	\arrayindexes
	\foreach \x in {1,5,8} {
		\draw (arr-\x-1.south west) + (2mm, 1mm) -- +(5mm, 4mm);
	}

	\foreach \x/\val in {2/19, 3/20, 4/12, 6/5, 7/33, 9/17} {
		\matrix[double cell, right=of arr-\x-1, ampersand replacement=\&] (cell \x) {\val \& \\};
		\draw[arrow] (arr-\x-1.center) -- (cell \x-1-1);
	}
	\foreach \x/\val in {2/28, 7/15} {
		\matrix[double cell, right=of cell \x-1-2, ampersand replacement=\&] (cell \x a) {\val \& \\};
		\draw[arrow] (cell \x-1-2.center) -- (cell \x a-1-1);
	}
	\foreach \x in {2a,3,4,6,7a,9} {
		\draw (cell \x-1-2.south west) + (1mm, 1mm) -- +(3.5mm, 3.5mm);
	}
\end{tikzpicture}
};

\node[outer, right=14mm of pic h] (pic i) {
\begin{tikzpicture}
	\matrix[
		array,
		med cell/.list = {1,5,8}
	] (arr) { \\ \\ \\ \\ \\ \\ \\ \\ \\};
	\arrayindexes
	\foreach \x in {1,5,8} {
		\draw (arr-\x-1.south west) + (2mm, 1mm) -- +(5mm, 4mm);
	}

	\foreach \x/\val in {2/10, 3/20, 4/12, 6/5, 7/33, 9/17} {
		\matrix[double cell, right=of arr-\x-1, ampersand replacement=\&] (cell \x) {\val \& \\};
		\draw[arrow] (arr-\x-1.center) -- (cell \x-1-1);
	}
	\foreach \x/\val in {2/19, 2a/28, 7/15} {
		\matrix[double cell, right=of cell \x-1-2, ampersand replacement=\&] (cell \x a) {\val \& \\};
		\draw[arrow] (cell \x-1-2.center) -- (cell \x a-1-1);
	}
	\foreach \x in {2aa,3,4,6,7a,9} {
		\draw (cell \x-1-2.south west) + (1mm, 1mm) -- +(3.5mm, 3.5mm);
	}
\end{tikzpicture}
};

\foreach \x in {a, ..., i} {
	\node[subpicture label, below=2mm of pic \x] {(\x)};
}

\end{tikzpicture}

	\caption{Ilustracja wstawiania do tablicy z~haszowaniem $T$ elementów o~kluczach 5, 28, 19, 15, 20, 33, 12, 17, 10.
Do rozwiązywania kolizji używana jest metoda łańcuchowa.} \label{fig:11.2-2}
\end{figure}

\exercise %11.2-3
Zakładamy, że listy są dwukierunkowe i~wartości funkcji haszującej można wyznaczać w~czasie stałym.
Na podstawie wyników z~problemu \refProblem{10-1} usuwanie elementu z~tablicy z~haszowaniem będzie działać w~czasie $\Theta(1)$.
Aby wstawić nowy element do tablicy, należy umieścić go na liście wskazanej przez funkcję haszującą, znajdując wpierw odpowiednie miejsce na tej liście, aby dodanie elementu nie zaburzyło jej uporządkowania.
W~pesymistycznym przypadku dla tablicy o~$n$ elementach zajmie to czas $\Theta(n)$, podobnie jak wyszukiwanie elementu, gdyż operacja ta także polega na liniowym przejrzeniu jednej z~list.

\exercise %11.2-4
\note{Treść zadania podaje opis tablicy z~haszowaniem jako struktury heterogenicznej, czyli przechowującej obiekty różnego typu w~zależności od zajętości pozycji danego obiektu. W~rozwiązaniu podajemy zaś implementację tej struktury w~postaci tablicy homogenicznej.}

\noindent Każda pozycja tablicy $T$ w~takiej reprezentacji będzie zawierać znacznik przechowujący informację o~tym, czy pozycja ta jest wolna.
Na zajętych pozycjach oprócz elementu będzie przechowywany wskaźnik (który oznaczymy przez \id{next}) do innej pozycji tablicy $T$ lub wskaźnik pusty, dzięki czemu komórki te będą mogły formować listy jednokierunkowe.
Wolne pozycje natomiast tworzyć będą listę dwukierunkową $F$, do zrealizowania której wykorzystane zostaną dwa wskaźniki na każdej wolnej pozycji -- \id{prev} na poprzednik i~\id{next} na następnik.
Do zapamiętania, gdzie znajduje się głowa listy $F$, zostanie użyty dodatkowy wskaźnik będący atrybutem tablicy $T$.
Wskaźnik ten oraz \id{prev} i~\id{next} są tutaj tak naprawdę liczbami całkowitymi określającymi indeksy tablicy $T$, przy czym pusty wskaźnik (odpowiednik \const{nil}) będziemy reprezentować wartością $-1$.

W~celu zachowania homogeniczności tablicy $T$ musimy sprawić, by pozycje wolne i~zajęte składały się z~tego samego zestawu atrybutów.
Niech więc pole odpowiedzialne za przechowywanie elementu będzie wskaźnikiem na ten element.
Dopuścimy obecność tego wskaźnika również na wolnych pozycjach -- przyjmie on wtedy wartość \const{nil}.
Z~kolei pole \id{prev} możemy wykorzystać w~dwóch celach -- do przechowywania indeksu poprzedniej pozycji na liście $F$, jak również jako znacznik określający zajętość aktualnej pozycji.
Jeśli tablica $T$ ma długość $m$, to na wolnych pozycjach pole \id{prev} może przyjąć wartości od 0 do $m-1$ oznaczające indeksy tablicy $T$ oraz wartość $-1$ jako reprezentację wskaźnika pustego.
Przyjmiemy natomiast, że na zajętych pozycjach pole to będzie miało wartość $m$ i~będzie to warunek rozstrzygający o~zajętości danej pozycji.

Aby dodać do tablicy $T[0\twodots m-1]$ element $x$ o~kluczu $k$, wpierw sprawdzamy, czy pozycja $h(k)$ w~$T$ jest wolna.
Jeśli tak, to usuwamy ją z~listy $F$, umieszczamy na niej element $x$, ustawiamy \attrib{T[h(k)]}{prev} na $m$, a~\attrib{T[h(k)]}{next} -- na $-1$.
W~przeciwnym razie na pozycji $h(k)$ znajduje się już inny element $y$ o~kluczu $l$.
Wówczas usuwamy głowę listy $F$, przenosimy na nią element $y$ wraz ze wskaźnikiem \attrib{T[h(k)]}{next} i~ustawiamy znacznik \id{prev} tej pozycji na $m$.
Następnie do komórki $h(k)$ wstawiamy element $x$.

Pozostaje uaktualnić wskaźnik \attrib{T[h(k)]}{next}.
W~tym celu oznaczmy przez $j$ indeks nowej pozycji elementu $y$ i~rozważmy dwie sytuacje.
W~pierwszej z~nich na podstawie funkcji haszującej elementowi $y$ odpowiada komórka $h(k)$, tzn.\ $h(l)=h(k)$.
Wówczas wskaźnikiem \attrib{T[h(k)]}{next} należy pokazać na pozycję $j$, gdyż $x$ i~$y$ powinny należeć do tej samej listy zajętych pozycji.
W~drugim przypadku element $y$ znajdował się na liście zajętych pozycji rozpoczynającej się na indeksie $h(l)\ne h(k)$.
Musimy zatem przejść po tej liście i~zmodyfikować ją tak, aby wskaźnik \id{next} poprzednika elementu $y$ pokazywał teraz na pozycję $j$.
Element $x$ będzie wówczas jedyny na swojej liście -- wystarczy więc wpisać do \attrib{T[h(k)]}{next} wartość $-1$.

Ponieważ traktujemy zajęte pozycje tablicy $T$ jak węzły list jednokierunkowych, to usuwanie elementu $x$ o~kluczu $k$ polega na usunięciu pozycji, która go zawiera, z~listy o~głowie w~$h(k)$.
Należy pamiętać jeszcze o~tym, aby wstawić zwolnioną pozycję na początek listy $F$ i~przestawić znacznik \id{prev} na $-1$.
Z~kolei wyszukiwanie elementu o~kluczu $k$ sprowadza się do przeglądnięcia listy o~głowie w~$h(k)$ i~zwrócenia wskaźnika do takiego elementu (o~ile istnieje) albo wartości \const{nil}.

Zauważmy, że przechowywanie list zajętych pozycji bezpośrednio w~tablicy nie zmienia czasów działania operacji słownikowych w~porównaniu z~zastosowaniem metody łańcuchowej, w~której listy te znajdują się poza tablicą.
Ponadto w~opisanej reprezentacji mamy zawsze $\alpha\le1$.
Operacje \proc{Insert}, \proc{Delete} i~\proc{Search} działają więc tutaj w~oczekiwanym czasie $\Theta(1+\alpha)=\Theta(1)$.
Aby go uzyskać, musieliśmy założyć, że lista wolnych pozycji $F$ jest listą dwukierunkową -- w~przeciwnym przypadku bowiem nie można byłoby usuwać z~niej w~czasie stałym.

\exercise %11.2-5
Funkcja haszująca przyjmuje $m$ różnych wartości, a~elementów zbioru $U$ jest więcej niż $nm$.
Z~tego powodu istnieje taka wartość, która została przyporządkowana więcej niż $n$ elementom ze zbioru $U$.
Fakt ten wynika z~nieco zmodyfikowanej wersji \textbf{zasady szufladkowej Dirichleta} \cite{pigeonholeprinciple} zwanej także \textbf{zasadą gołębnika}.

\input{chapter11/sc11.3}
\subchapter{Adresowanie otwarte}

\exercise %11.4-1
\note{Funkcja\/ $h'$ nie jest pierwotną funkcją haszującą, tylko pomocniczą funkcją haszującą.
Zakładamy, że pierwszą pomocniczą funkcją haszującą w~haszowaniu dwukrotnym jest\/ $h_1=h'$.}

\noindent Tabela \ref{tab:11-1} przedstawia pozycje tablicy, które są obliczane dla poszczególnych kluczy przy zastosowaniu różnych sposobów obliczania ciągów kontrolnych.

\begin{table}[!ht]
	\centering
		\[
			\begin{array}{c|c|c|c}
				& \text{adresowanie} & \text{adresowanie} & \text{haszowanie} \\
				& \text{liniowe} & \text{kwadratowe} & \text{dwukrotne} \\
				\hline
				10 & 10 & 10 & 10 \\
				\hline
				22 & 0 & 0 & 0 \\
				\hline
				31 & 9 & 9 & 9 \\
				\hline
				4 & 4 & 4 & 4 \\
				\hline
				15 & 4,5 & 4,8 & 4,10,5 \\
				\hline
				28 & 6 & 6 & 6 \\
				\hline
				17 & 6,7 & 6,10,9,3 & 6,3 \\
				\hline
				88 & 0,1 & 0,4,3,8,8,3,4,0,2 & 0,9,7 \\
				\hline
				59 & 4,5,6,7,8 & 4,8,7 & 4,3,2
			\end{array}
		\]
	\caption{Pozycje obliczane dla podanego ciągu kluczy w~różnych metodach adresowania otwartego.
Dany klucz trafia ostatecznie na pierwszą wolną pozycję ze swojego ciągu kontrolnego.} \label{tab:11-1}
\end{table}

\exercise %11.4-2
Pseudokod procedury \proc{Hash-Delete} przedstawiono poniżej:
\begin{codebox}
\Procname{$\proc{Hash-Delete}(T,k)$}
\li	$i\gets0$
\li	\Repeat
		$j\gets h(k,i)$
\li		\If $T[j]=k$
\li			\Then
				$T[j]\gets\const{deleted}$
\li				\Return
			\End
\li		$i\gets i+1$
\li	\Until $T[j]=\const{nil}$ lub $i=m$
\end{codebox}
W~procedurze \proc{Hash-Insert} wystarczy zamienić warunek z~wiersza 3 na następujący:
\begin{codebox}
\setcounter{codelinenumber}{2}
\li	\If $T[j]=\const{nil}$ lub $T[j]=\const{deleted}$
\end{codebox}

\exercise %11.4-3
Dzięki skorzystaniu z~tw.\ 31.20 otrzymujemy, że rzędem grupy generowanej przez $h_2(k)$ jest $m/d$, co oznacza, że w~$\mathbb{Z}_m$ wyrażenie $ih_2(k)$ dla $i=0$, 1, \dots, $m-1$ przyjmuje $m/d$ różnych wartości.
Podobną własność wykazuje także suma $h_1(k)+ih_2(k)$.
A~zatem stosując funkcję haszującą $h$, podczas wyszukiwania klucza $k$, które zakończy się porażką, zostanie sprawdzonych $m/d$ różnych komórek tablicy, co stanowi $(1/d)$-tą część tej tablicy.

\exercise %11.4-4
Wykorzystując tw.\ 11.8, otrzymujemy, że dla współczynnika zapełnienia $\alpha=1/2$ oczekiwana liczba porównań nie przekracza $2\ln2\approx1{,}386$.
Dla $\alpha=3/4$ oszacowanie to wynosi $(4/3)\ln4\approx1{,}848$, a~dla $\alpha=7/8$ jest ono równe $(8/7)\ln8\approx2{,}377$.

\exercise %11.4-5
Z~tw.\ 11.6 i~11.8 dostajemy, że szukane $0<\alpha<1$ spełnia równanie
\[
	\frac{1}{1-\alpha} = \frac{2}{\alpha}\ln\frac{1}{1-\alpha}.
\]
Niech $\beta=1/(1-\alpha)$.
Wówczas $\alpha=1-1/\beta$ i~powyższy wzór przyjmuje postać
\[
	\beta = \frac{2\beta}{\beta-1}\ln\beta,
\]
co jest równoważne
\[
	\beta-2\ln\beta = 1.
\]
Mamy dalej:
\begin{align*}
	e^{\beta-2\ln\beta} &= e, \\
	\beta^{-2}e^\beta &= e, \\
	\beta e^{-\beta/2} &= e^{-1/2}, \\
	(-\beta/2)e^{-\beta/2} &= -e^{-1/2}\!/2.
\end{align*}
Skorzystamy teraz z~\textbf{funkcji $W$ Lamberta} \cite{lambertwfunction} zdefiniowanej jako wielowartościowe odwzorowanie odwrotne do funkcji $f(x)=xe^x$.
Innymi słowy, dla każdej liczby rzeczywistej $x\ge-1/e$ spełniony jest wzór
\[
	x = W(x)e^{W(x)}.
\]
Nasze równanie sprowadza się zatem do
\[
	W(-e^{-1/2}\!/2) = -\beta/2,
\]
skąd
\[
	\beta = -2W(-e^{-1/2}\!/2).
\]
Dla argumentu $-e^{-1/2}\!/2$ odwzorowanie $W$ przyjmuje dwie wartości.
Jedną z~nich jest oczywiście $-1/2$.
Ale wówczas $\beta=1$ i~$\alpha=0$, co jest sprzeczne z~założeniem.
Drugą wartość $W(-e^{-1/2}\!/2)$ można uzyskać, stosując metody numeryczne -- wynosi ona w~przybliżeniu $-1{,}756$, skąd $\beta\approx3{,}513$ i~$\alpha\approx0{,}715$.

\subchapter{Haszowanie doskonałe}

\exercise %11.5-1
\note{W~treści zadania zamiast haszowania uniwersalnego powinno być haszowanie równomierne.}

\noindent Rozważmy prawdopodobieństwo $q(n,m)$, że wystąpi przynajmniej jedna kolizja.
Jest $\binom{n}{2}$ par kluczy, które mogą tworzyć kolizję z~prawdopodobieństwem $1/m$ każda.
Mamy więc
\[
	q(n,m) = \binom{n}{2}\frac{1}{m} = \frac{n(n-1)}{2m}
\]
i~teraz, dzięki skorzystaniu ze wzoru (3.11), otrzymujemy
\[
	p(n,m) = 1-q(n,m) = 1-\frac{n(n-1)}{2m} \le e^{-n(n-1)/2m}.
\]

W~drugiej części zadania rozważymy funkcję $f(x)=e^{-x(x-1)/2m}$ zmiennej rzeczywistej $x$, traktując $m$ jak stałą dodatnią.
Jak łatwo zauważyć, granicą tej funkcji w~$\infty$ jest 0.
Wyznaczając jej pierwszą i~drugą pochodną, mamy
\[
	\frac{df}{dx}(x) = e^{-x(x-1)/2m}\,\frac{1-2x}{2m}
\]
oraz
\[
	\frac{d^2\!f}{dx^2}(x) = e^{-x(x-1)/2m}\biggl(\frac{1-2x}{2m}\biggr)^2+e^{-x(x-1)/2m}\cdot\frac{-1}{m} = e^{-x(x-1)/2m}\,\frac{4x^2-4x+1-4m}{4m^2}.
\]
Czynnik $e^{-x(x-1)/2m}$ jest dodatni niezależnie od wartości $x$ i~$m$.
Jeśli $x>\sqrt{m}$, to natychmiast widać, że pierwsza pochodna funkcji $f$ jest ujemna.
Do tego samego wniosku dochodzimy dla drugiej pochodnej, ograniczając jej drugi czynnik:
\[
	\frac{4x^2-4x+1-4m}{4m^2} < \frac{4m-4\sqrt{m}+1-4m}{4m^2} = \frac{-4\sqrt{m}+1}{4m^2} < 0.
\]
A~zatem dla $x>\sqrt{m}$ funkcja $f$ jest malejąca i~wklęsła.
Oznacza to, że prawdopodobieństwo $p(n,m)$, które wynosi co najwyżej $f(n)$, maleje gwałtownie do zera, gdy $n$ przekracza $\sqrt{m}$.


\problems

\problem{Szacowanie najdłuższego ciągu odwołań do tablicy z~haszowaniem} %11-1
\note{W~punkcie (b) należy wykazać, że prawdopodobieństwo opisanego tam zdarzenia wynosi\/ $O(1/n^2)$.
W~treści (także oryginalnej) brakuje założenia o~równomiernym haszowaniu wymaganego do pokazania tego oszacowania.
W~paragrafie między częścią (b) a~(c) powinno być\/ $\Pr(X_i>2\lg n)=O(1/n^2)$.
W~punkcie (c) należy wykazać, że\/ $\Pr(X>2\lg n)=O(1/n)$.
Ponadto użyto błędnych liter do oznaczenia punktów (c) i~(d).}
\bignegskip

\subproblem %11-1(a)
Niech $X_i$ będzie zmienną losową oznaczającą liczbę odwołań do tablicy wykonywanych podczas wstawiania \singledash{$i$}{tego} elementu.
Operacja ta jest poprzedzona wyszukiwaniem tego elementu w~tablicy z~negatywnym skutkiem.
Dzięki założeniu o~równomiernym haszowaniu możemy napisać $\Pr(X_i>k)=\Pr(X_i\ge k+1)\le\alpha^k$, co wynika z~oszacowania wykazanego w~dowodzie tw.\ 11.6.
Ponieważ $n\le m/2$, to $\alpha\le1/2$, a~stąd $\Pr(X_i>k)\le(1/2)^k=2^{-k}$.

\subproblem %11-1(b)
Wynik otrzymujemy natychmiast po podstawieniu $k=2\lg n$ do oszacowania z~poprzedniego punktu.

\subproblem %11-1(c)
Oznaczmy przez $A$ zdarzenie, że $X>2\lg n$, a~przez $A_i$, dla $i=1$, 2, \dots, $n$, zdarzenie, że $X_i>2\lg n$.
Zauważmy, że $A=\bigcup_{i=1}^nA_i$.
Wykorzystując nierówność Boole'a (\refExercise{C.2-1}) oraz wynik z~punktu (b), w~którym pokazaliśmy, że $\Pr(A_i)=O(1/n^2)$, otrzymujemy
\[
	\Pr(A) = \Pr\biggl(\bigcup_{i=1}^nA_i\biggr) \le \sum_{i=1}^n\Pr(A_i) = n\cdot O(1/n^2) = O(1/n).
\]

\subproblem %11-1(d)
Na podstawie oszacowania z~poprzedniej części otrzymujemy
\begin{align*}
	\E(X) &= \sum_{k=1}^nk\Pr(X=k) \\
	&= \sum_{k=1}^{\lfloor2\lg n\rfloor}k\Pr(X=k)+\sum_{k=\lfloor2\lg n\rfloor+1}^nk\Pr(X=k) \\
	&\le \sum_{k=1}^{\lfloor2\lg n\rfloor}\lfloor2\lg n\rfloor\Pr(X=k)+\sum_{k=\lfloor2\lg n\rfloor+1}^nn\Pr(X=k) \\
	&= \lfloor2\lg n\rfloor\Pr(X\le2\lg n)+n\Pr(X>2\lg n) \\
	&\le \lfloor2\lg n\rfloor\cdot1+n\cdot O(1/n) \\
	&= \lfloor2\lg n\rfloor+O(1) \\
	&= O(\lg n).
\end{align*}

\problem{Długość listy w~metodzie łańcuchowej} %11-2

\subproblem %11-2(a)
Potraktujmy odwzorowywanie kluczy jako ciąg prób Bernoulliego, w~których sukcesem jest odwzorowanie klucza na ustaloną pozycję tablicy.
Prawdopodobieństwo sukcesu każdej takiej próby jest równe $1/n$.
Z~rozkładu dwumianowego wynika, że uzyskanie dokładnie $k$ sukcesów w~tej serii prób jest równe
\[
	Q_k = b(k;n,1/n) = \binom{n}{k}\biggl(\frac{1}{n}\biggr)^k\biggl(1-\frac{1}{n}\biggr)^{n-k}.
\]

\subproblem %11-2(b)
Niech $A_i$, dla $i=1$, 2, \dots, $n$, oznacza zdarzenie, że liczba elementów odwzorowanych na $i$\nbhyphen tą pozycję tablicy wynosi $k$.
Wówczas zdarzenie $A$, że $M=k$, spełnia inkluzję $A\subseteq\bigcup_{i=1}^nA_i$.
Z~własności prawdopodobieństwa i~z~punktu (a) mamy
\[
	P_k = \Pr(A) \le \Pr\biggl(\bigcup_{i=1}^nA_i\biggr) \le \sum_{i=1}^n\Pr(A_i) = \sum_{i=1}^nQ_k = nQ_k.
\]

\subproblem %11-2(c)
Załóżmy, że $k>0$.
Z~definicji współczynnika dwumianowego:
\[
	\binom{n}{k} = \frac{n!}{k!(n-k)!} = \frac{(n-k+1)(n-k+2)\dots n}{k!}.
\]
Każdy z~czynników w~liczniku ograniczamy od góry przez $n$.
Z~drugiej strony, na podstawie wzoru Stirlinga, mamy $k!=\sqrt{2\pi k}\,(k/e)^k(1+\Theta(1/k))>(k/e)^k$.
Stąd
\[
	\binom{n}{k} < \frac{n^k}{(k/e)^k} = \biggl(\frac{ne}{k}\biggr)^k.
\]
Wykorzystując tę nierówność, mamy
\[
	Q_k = \biggl(\frac{1}{n}\biggr)^k\biggl(1-\frac{1}{n}\biggr)^{n-k}\binom{n}{k} < \frac{(n-1)^{n-k}}{n^n}\biggl(\frac{ne}{k}\biggr)^k = \biggl(\frac{n-1}{n}\biggr)^{n-k}\biggl(\frac{e}{k}\biggr)^k < \frac{e^k}{k^k}.
\]

\subproblem %11-2(d)
\note{W~drugiej części zadania należy wywnioskować, że\/ $P_k<1/n^2$ dla\/ $k\ge k_0=c\lg n/\!\lg\lg n$.}

\noindent Badając wyrażenie $\lg Q_{k_0}$, wyznaczymy odpowiednie $c$ tak, aby zachodziło $Q_{k_0}<1/n^3$, gdzie $k_0=c\lg n/\!\lg\lg n$.
Z~poprzedniego punktu mamy, że $Q_{k_0}<e^{k_0}\!/{k_0}^{k_0}$, a~więc
\begin{align*}
	\lg Q_{k_0} &< k_0\lg e-k_0\lg k_0 \\
	&= \frac{c\lg n}{\lg\lg n}\biggl(\lg e-\lg\frac{c\lg n}{\lg\lg n}\biggr) \\[1mm]
	&= \frac{c\lg n}{\lg\lg n}(\lg e-\lg c-\lg\lg n+\lg\lg\lg n).
\end{align*}
Nierówność $Q_{k_0}<1/n^3$ jest równoważna $\lg Q_{k_0}<-3\lg n$, zatem na podstawie powyższego:
\[
	\frac{c\lg n}{\lg\lg n}(\lg e-\lg c-\lg\lg n+\lg\lg\lg n) < -3\lg n.
\]
Po podzieleniu obu stron tej nierówności przez $-\lg n$ dostajemy
\[
	c\biggl(1-\frac{\lg\lg\lg n}{\lg\lg n}+\frac{\lg(c/e)}{\lg\lg n}\biggr) > 3.
\]
Można pokazać, że $\lg\lg\lg n/\lg\lg n<1$.
Zatem dla $c\ge e$ wyrażenie w~nawiasie jest dodatnie i~wybierając wystarczająco duże $c$, możemy spełnić nierówność.

Ustalmy $c$ tak, aby $k_0>e$.
Wówczas $e/k_0<1$ i~dla $k\ge k_0$ wyrażenie $(e/k)^k$ maleje wraz ze wzrostem $k$.
Łącząc tę obserwację z~wynikami z~poprzednich części, otrzymujemy, że
\[
	P_k\le nQ_k < ne^k\!/k^k \le ne^{k_0}\!/k_0^{k_0} < n\cdot1/n^3 = 1/n^2
\]
dla dowolnego $k\ge k_0$.

\subproblem %11-2(e)
Niech $k_0=c\lg n/\!\lg\lg n$.
Wówczas
\begin{align*}
	\E(M) &= \sum_{k=0}^nkP_k = \sum_{k=0}^{k_0}kP_k+\sum_{k=k_0+1}^nkP_k \\
	&\le k_0\sum_{k=0}^{k_0}P_k+n\sum_{k=k_0+1}^nP_k = k_0\Pr(M\le k_0)+n\Pr(M>k_0).
\end{align*}

W~celu uzasadnienia górnego oszacowania dla $\E(M)$, wykorzystamy fakt, który udowodniliśmy w~punkcie (d):
\begin{align*}
	\E(M) &\le k_0\Pr(M\le k_0)+n\Pr(M>k_0) = k_0\cdot1+n\sum_{k=k_0+1}^nP_k \\
	&< k_0+n\sum_{k=k_0+1}^n1/n^2 \le k_0+n\cdot n\cdot1/n^2 = k_0+1 = O(\lg n/\!\lg\lg n).
\end{align*}

\problem{Adresowanie kwadratowe} %11-3
\note{Krok 3 opisanego algorytmu wyszukiwania powinien mieć następującą treść:
\begin{enumerate}
	\setcounter{enumi}{2}
	\item Wykonaj\/ $j\gets j+1$.
Jeśli\/ $j=m$, to tablica jest pełna, więc zakończ wyszukiwanie.
W~przeciwnym przypadku wykonaj\/ $i\gets(i+j)\bmod m$, a~następnie wróć do kroku 2.
\end{enumerate}
}
\bignegskip

\subproblem %11-3(a)
Dla klucza $k$ algorytm ten, o~ile wcześniej nie zostanie przerwany, odwołuje się kolejno do pozycji $h(k)$, $(h(k)+1)\bmod m$, $(h(k)+1+2)\bmod m$, \dots, $\bigl(h(k)+\sum_{j=1}^{m-1}j\bigr)\bmod m$.
Stąd mamy, że \singledash{$i$}{tą} sprawdzaną pozycją tablicy ($i=0$, 1, \dots, $m-1$) jest
\[
	\biggl(h(k)+\sum_{j=0}^ij\biggr)\bmod m = \biggl(h(k)+\frac{i(i+1)}{2}\biggr)\bmod m = \biggl(h(k)+\frac{i}{2}+\frac{i^2}{2}\biggr)\bmod m.
\]
Jest to zatem przykład adresowania kwadratowego, w~którym $c_1=c_2=1/2$.

\subproblem %11-3(b)
Niech $h'(k,i)=(h(k)+i/2+i^2\!/2)\bmod m$.
Udowodnimy, że dla dowolnego klucza $k$ wyrazy ciągu $\langle h'(k,0),h'(k,1),\dots,h'(k,m-1)\rangle$ są parami różne.

Załóżmy nie-wprost, że istnieje klucz $k$ oraz liczby całkowite $i$, $j$ takie, że $0\le i<j<m$ oraz $h'(k,i)=h'(k,j)$.
Wówczas
\[
	h(k)+i(i+1)/2 \equiv h(k)+j(j+1)/2 \pmod m,
\]
co daje
\[
	j(j+1)/2-i(i+1)/2 \equiv 0 \pmod m
\]
i~po przekształceniu lewej strony dostajemy
\[
	(j-i)(j+i+1)/2 \equiv 0 \pmod m.
\]
Kongruencja ta oznacza, że istnieje całkowite $r$, dla którego zachodzi $(j-i)(j+i+1)=2rm$.
Przy założeniu, że $m$ jest potęgą 2, $m=2^p$, sprowadza się to do postaci $(j-i)(j+i+1)=r2^{p+1}$.
Nietrudno zauważyć, że tylko jeden z~czynników, $j-i$ albo $j+i+1$, jest parzysty, zatem $2^{p+1}$ dzieli tylko jeden z~nich.
Nie może nim być $j-i$, gdyż $j-i<m<2^{p+1}$.
Ale czynnik $j+i+1$ również nie dzieli się przez $2^{p+1}$, bo $j+i+1\le(m-1)+(m-2)+1=2m-2<2^{p+1}$.
Otrzymana sprzeczność prowadzi do wniosku, że $h'(k,i)\ne h'(k,j)$.

Udowodnione stwierdzenie implikuje to, że dla dowolnego klucza $k$ algorytm wyszukiwania, używający opisanej tu funkcji haszującej $h$, w~pesymistycznym przypadku sprawdzi każdą pozycję tablicy w~celu odnalezienia $k$.

\problem{Haszowanie \singledash{$k$}{uniwersalne} i~uwierzytelnianie} %11-4
\note{Poprawki wprowadzone w~angielskiej treści tego problemu okazały się na tyle znaczące, że problem został napisany od nowa.
Poniżej prezentujemy polskie tłumaczenie jego nowej wersji.}

\noindent Niech $\mathcal{H}$ będzie rodziną funkcji haszujących, które odwzorowują uniwersum kluczy $U$ w~zbiór $\{0,1,\dots,m-1\}$.
Powiemy, że $\mathcal{H}$ jest \textbf{\singledash{$k$}{uniwersalna}}, jeśli dla każdego ustalonego ciągu $k$ różnych kluczy $\langle x^{(1)},x^{(2)},\dots,x^{(k)}\rangle$ oraz funkcji $h$ wybranej losowo z~$\mathcal{H}$ ciąg $\langle h(x^{(1)}),h(x^{(2)}),\dots,h(x^{(k)})\rangle$ jest z~jednakowym prawdopodobieństwem równy dowolnemu spośród $m^k$ ciągów $k$ elementów ze zbioru $\{0,1,\dots,m-1\}$.
\begin{description}
	\setlength\labelsep{11pt}
	\item[{\sffamily\bfseries(a)}] Wykaż, że jeśli rodzina funkcji haszujących $\mathcal{H}$ jest \singledash{2}{uniwersalna}, to jest uniwersalna.
	\item[{\sffamily\bfseries(b)}] Załóżmy, że $U$ jest zbiorem \singledash{$n$}{tek} o~wartościach z~$\mathbb{Z}_p=\{0,1,\dots,p-1\}$, gdzie $p$ jest liczbą pierwszą.
Rozważmy element $x=\langle x_0,x_1,\dots,x_{n-1}\rangle\in U$.
Dla każdej \singledash{$n$}{tki} $a=\langle a_0,a_1,\dots,a_{n-1}\rangle\in U$ definiujemy funkcję haszującą $h_a$ jako
	\[
		h_a(x) = \biggl(\sum_{j=0}^{n-1}a_jx_j\biggr)\bmod p.
	\]
	Udowodnij, że rodzina $\mathcal{H}=\{h_a\}$ jest uniwersalna, ale nie \singledash{2}{uniwersalna}.
(\!\emph{Wskazówka:} Znajdź klucz, dla którego wszystkie funkcje z~$\mathcal{H}$ przyjmują tę samą wartość.)
	\item[{\sffamily\bfseries(c)}] Załóżmy, że zmodyfikowaliśmy nieco rodzinę $\mathcal{H}$ z~punktu (b): dla każdego $a\in U$ i~każdego $b\in\mathbb{Z}_p$ definiujemy
	\[
		h_{a,b}'(x) = \biggl(\sum_{j=0}^{n-1}a_jx_j+b\biggr)\bmod p
	\]
	oraz $\mathcal{H}'=\{h_{a,b}'\}$.
Udowodnij, że rodzina $\mathcal{H}'$ jest \singledash{2}{uniwersalna}.
(\!\emph{Wskazówka:} Rozważ ustalone $x\in U$ i~$y\in U$ spełniające $x_i\ne y_i$ dla pewnego $i$.
Co dzieje się z~$h_{a,b}'(x)$ i~$h_{a,b}'(y)$, gdy $a_i$ i~$b$ przyjmują poszczególne wartości z~$\mathbb{Z}_p$?)
	\item[{\sffamily\bfseries(d)}] Przypuśćmy, że Alicja i~Bob uzgodnili w~sekrecie funkcję haszującą $h$ z~\singledash{2}{uniwersalnej} rodziny funkcji haszujących $\mathcal{H}$.
Każda funkcja $h\in\mathcal{H}$ odwzorowuje uniwersum kluczy $U$ w~$\mathbb{Z}_p$, gdzie $p$ jest liczbą pierwszą.
Następnie Alicja przesyła do Boba przez Internet komunikat $m\in U$.
Alicja uwierzytelnia komunikat, przesyłając dodatkowo znacznik $t=h(m)$, a~Bob sprawdza, czy para $\langle m,t\rangle$, którą otrzymuje, faktycznie spełnia $t=h(m)$.
Załóżmy, że przeciwnik przechwytuje przesyłaną parę $\langle m,t\rangle$ i~próbuje oszukać Boba, zamieniając ją na inną parę $\langle m',t'\rangle$.
Wykaż, że prawdopodobieństwo, iż przeciwnikowi uda się oszukać Boba i~że zaakceptuje on parę $\langle m',t'\rangle$, wynosi co najwyżej $1/p$, niezależnie od tego, jak wielką mocą obliczeniową przeciwnik dysponuje, i~nawet wówczas, gdy przeciwnik zna rodzinę funkcji haszujących $\mathcal{H}$.
\end{description}

\bigskip
\note{Poniżej znajduje się rozwiązanie nowej wersji problemu.}
\vspace{-2ex}

\subproblem %11-4(a)
Niech $\mathcal{H}$ będzie \singledash{2}{uniwersalną} rodziną funkcji haszujących oraz niech $\langle x,y\rangle$ będzie parą różnych kluczy z~$U$.
Wówczas, dla losowo wybranej funkcji haszującej $h_a\in\mathcal{H}$, para $\langle h(x),h(y)\rangle$ jest z~jednakowym prawdopodobieństwem dowolną spośród $m^2$ par o~elementach ze zbioru $\{0,1,\dots,m-1\}$.
A~zatem kolizja, czyli zdarzenie, że $h(x)=h(y)$, wystąpi z~prawdopodobieństwem $1/m$.
Rodzina $\mathcal{H}$ jest więc uniwersalna.

\subproblem %11-4(b)
Aby zbadać \singledash{2}{uniwersalność}, wykorzystamy wskazówkę.
Dla $x=\langle0,0,\dots,0\rangle$ wszystkie funkcje haszujące z~$\mathcal{H}$ dają w~wyniku 0, więc dla dowolnej funkcji $h\in\mathcal{H}$ i~dowolnej pary $\langle x,y\rangle$ różnych kluczy z~$U$ nigdy nie otrzymamy pary $\langle h_a(x),h_a(y)\rangle$, której pierwszym elementem jest liczba różna od zera.
To wyklucza \singledash{2}{uniwersalność} rodziny $\mathcal{H}$.

Udowodnimy teraz, że rodzina $\mathcal{H}$ jest uniwersalna.
Wybierzmy w~tym celu dowolną parę $\langle x,y\rangle$ różnych kluczy z~$U$ i~pewną funkcję $h_a$ z~$\mathcal{H}$.
Bez utraty ogólności przyjmijmy, że $x_0\ne y_0$.
Kolizja $h_a(x)=h_a(y)$ wystąpi tylko wtedy, gdy suma $\sum_{j=0}^{n-1}a_jx_j$ będzie dawać taką samą resztę z~dzielenia przez $p$, co suma $\sum_{j=0}^{n-1}a_jy_j$, lub równoważnie, kiedy spełniony będzie poniższy wzór:
\[
	\sum_{j=0}^{n-1}a_j(x_j-y_j) \equiv 0 \pmod p.
\]
Niech $S=\sum_{j=1}^{n-1}a_j(x_j-y_j)$.
Wówczas wzór przyjmuje postać
\[
	a_0(x_0-y_0) \equiv -S \pmod p,
\]
którą potraktujemy jak modularne równanie liniowe zmiennej $a_0$.
Ponieważ $x_0\ne y_0$, a~$p$ jest liczbą pierwszą, to $\gcd(x_0-y_0,p)=1$ i~na podstawie wniosku 31.25 $a_0$ jest wyznaczone jednoznacznie modulo $p$.
A~zatem dla ustalonych $a_1$, $a_2$, \dots, $a_{n-1}$ istnieje dokładnie jedno $a_0$, dla którego funkcja $h_a$, gdzie $a=\langle a_0,a_1,\dots,a_{n-1}\rangle$, generuje kolizję między $x$ a~$y$.
To oznacza, że dokładnie $p^{n-1}$ spośród $p^n$ funkcji w~$\mathcal{H}$ doprowadzi do kolizji.
Prawdopodobieństwo tego zdarzenia, przy założeniu, że funkcja $h_a$ jest wybrana losowo z~$\mathcal{H}$, wynosi $1/p$, co kończy dowód.

\subproblem %11-4(c)
Ustalmy parę $\langle x,y\rangle$ różnych kluczy z~$U$ i~wybierzmy pewną funkcję $h_{a,b}'$ z~$\mathcal{H}'$.
Bez utraty ogólności przyjmijmy, że $x_0\ne y_0$.
Wprowadźmy oznaczenia $\alpha=h_{a,b}'(x)$, $\beta=h_{a,b}'(y)$ oraz $X=\sum_{j=1}^{n-1}a_jx_j$, $Y=\sum_{j=1}^{n-1}a_jy_j$.
Zachodzi $\alpha=(a_0x_0+X+b)\bmod p$ oraz $\beta=(a_0y_0+Y+b)\bmod p$.
Zauważmy, że aby wygenerować każdą możliwą parę $\langle\alpha,\beta\rangle$, wystarczy abyśmy byli w~stanie wygenerować dowolne $\alpha-\beta$ i~dowolne $\beta$.
Mamy $\alpha-\beta=(a_0(x_0-y_0)+X-Y)\bmod p$, skąd
\[
	a_0(x_0-y_0) \equiv \alpha-\beta-X+Y \pmod p.
\]
Ustalmy dowolną wartość wyrażenia $\alpha-\beta$ i~potraktujmy powyższy wzór jak modularne równanie liniowe zmiennej $a_0$.
Oczywiście $\gcd(x_0-y_0,p)=1$, więc na podstawie wniosku 31.25 $a_0$ jest wyznaczone jednoznacznie modulo $p$.
Istnieje zatem jednoznaczna odpowiedniość między wartością $a_0$ a~wartością $\alpha-\beta$.
Mając ustalone $a_0$ i~dobierając różne wartości dla $b$, możemy z~kolei wygenerować każde $\beta$.
Jest dokładnie $p^2$ możliwych par $\langle\alpha,\beta\rangle$ i~tyleż samo możliwości wyboru $a_0$ i~$b$.
Stąd wnioskujemy, że każda para $\langle\alpha,\beta\rangle$ jest jednoznacznie generowana przez odpowiednie $a_0$ i~$b$.
Istnieje zatem $p^{n-1}$ funkcji $h_{a,b}'\in\mathcal{H}'$, które generują zadaną parę $\langle\alpha,\beta\rangle$.
Wnioskujemy stąd, że uzyskanie każdej takiej pary jest jednakowo prawdopodobne, gdy funkcja $h_{a,b}'$ jest wybrana losowo z~$\mathcal{H}'$, a~to oznacza, że rodzina $\mathcal{H}'$ jest \singledash{2}{uniwersalna}.

\subproblem %11-4(d)
Ponieważ rodzina $\mathcal{H}$ jest \singledash{2}{uniwersalna}, to dla każdej pary kluczy $\langle m,m'\rangle$, w~której $m\ne m'$, uzyskanie dowolnej pary wartości $\langle h(m),h(m')\rangle$ jest jednakowo prawdopodobne, gdy funkcja $h$ zostanie wybrana losowo z~$\mathcal{H}$.
W~szczególności każda z~$p$ par postaci $\langle t,h(m')\rangle$ ma jednakowe szanse wystąpienia.
Dlatego przeciwnik, nawet jeśli dysponuje pełną wiedzą na temat rodziny $\mathcal{H}$ i~przechwyci parę $\langle m,t\rangle$, to nie zyskuje żadnej informacji o~wartości $h(m')$, którą powinien przesłać jako $t'$ celem oszukania Boba.
Przeciwnik może więc tylko zgadywać, a~szansa, że wybierze właściwą spośród $p$ wartości, jest równa $1/p$.


\endinput

%\chapter{Drzewa wyszukiwań binarnych}

\subchapter{Co to jest drzewo wyszukiwań binarnych?}

\exercise %12.1-1
\exercise %12.1-2
\exercise %12.1-3
\exercise %12.1-4
\exercise %12.1-5

\subchapter{Wyszukiwanie w~drzewie wyszukiwań binarnych}

\exercise %12.2-1
\exercise %12.2-2
\exercise %12.2-3
\exercise %12.2-4
\exercise %12.2-5
\exercise %12.2-6
\exercise %12.2-7
\exercise %12.2-8
\exercise %12.2-9

\subchapter{Wstawianie i~usuwanie}

\exercise %12.3-1
\exercise %12.3-2
\exercise %12.3-3
\exercise %12.3-4
\exercise %12.3-5
\exercise %12.3-6

\subchapter{Losowo skonstruowane drzewa wyszukiwań binarnych}

\exercise %12.4-1
\exercise %12.4-2
\exercise %12.4-3
\exercise %12.4-4
\exercise %12.4-5

\problems

\problem{Drzewa wyszukiwań binarnych z~powtarzającymi się kluczami} %12-1

\subproblem %12-1(a)
\subproblem %12-1(b)
\subproblem %12-1(c)
\subproblem %12-1(d)

\problem{Drzewa pozycyjne} %12-2

\problem{Średnia głębokość węzła w~losowo zbudowanym drzewie wyszukiwań binarnych} %12-3

\subproblem %12-3(a)
\subproblem %12-3(b)
\subproblem %12-3(c)
\subproblem %12-3(d)
\subproblem %12-3(e)
\subproblem %12-3(f)

\problem{Zliczanie różnych drzew binarnych} %12-4

\subproblem %12-4(a)
\subproblem %12-4(b)
\subproblem %12-4(c)
\subproblem %12-4(d)

\endinput

%\chapter{Drzewa \onedash{czerwono}{czarne}}

\subchapter{Własności drzew \onedash{czerwono}{czarnych}}

\exercise %13.1-1
\exercise %13.1-2
\exercise %13.1-3
\exercise %13.1-4
\exercise %13.1-5
\exercise %13.1-6
\exercise %13.1-7

\subchapter{Operacje rotacji}

\exercise %13.2-1
\exercise %13.2-2
\exercise %13.2-3
\exercise %13.2-4
\exercise %13.2-5

\subchapter{Operacje wstawiania}

\exercise %13.3-1
\exercise %13.3-2
\exercise %13.3-3
\exercise %13.3-4
\exercise %13.3-5
\exercise %13.3-6

\subchapter{Operacja usuwania}

\exercise %13.4-1
\exercise %13.4-2
\exercise %13.4-3
\exercise %13.4-4
\exercise %13.4-5
\exercise %13.4-6
\exercise %13.4-7

\problems

\problem{Zbiory dynamiczne z~historią} %13-1

\subproblem %13-1(a)
\subproblem %13-1(b)
\subproblem %13-1(c)
\subproblem %13-1(d)
\subproblem %13-1(e)

\problem{Złączanie drzew \onedash{czerwono}{czarnych}} %13-2

\subproblem %13-2(a)
\subproblem %13-2(b)
\subproblem %13-2(c)
\subproblem %13-2(d)
\subproblem %13-2(e)
\subproblem %13-2(f)

\problem{Drzewa AVL} %13-3

\subproblem %13-3(a)
\subproblem %13-3(b)
\subproblem %13-3(c)
\subproblem %13-3(d)

\problem{Drzepce, czyli drzewa ,,treaps''} %13-4

\subproblem %13-4(a)
\subproblem %13-4(b)
\subproblem %13-4(c)
\subproblem %13-4(d)
\subproblem %13-4(e)
\subproblem %13-4(f)
\subproblem %13-4(g)
\subproblem %13-4(h)
\subproblem %13-4(i)
\subproblem %13-4(j)

\endinput

%\chapter{Wzbogacanie struktur danych}

\subchapter{Dynamiczne statystyki pozycyjne}

\exercise %14.1-1
\exercise %14.1-2
\exercise %14.1-3
\exercise %14.1-4
\exercise %14.1-5
\exercise %14.1-6
\exercise %14.1-7
\exercise %14.1-8

\subchapter{Jak wzbogacać strukturę danych}

\exercise %14.2-1
\exercise %14.2-2
\exercise %14.2-3
\exercise %14.2-4
\exercise %14.2-5

\subchapter{Drzewa przedziałowe}

\exercise %14.3-1
\exercise %14.3-2
\exercise %14.3-3
\exercise %14.3-4
\exercise %14.3-5
\exercise %14.3-6
\exercise %14.3-7

\problems

\problem{Punkt o~największej liczbie przecięć} %14-1

\subproblem %14-1(a)
\subproblem %14-1(b)

\problem{Permutacja Józefa} %14-2

\subproblem %14-2(a)
\subproblem %14-2(b)

\endinput


%\setcounter{part}{3}
%\part{Zaawansowane metody konstruowania i~analizowania algorytmów}

%\setcounter{chapter}{14}
%\chapter{Programowanie dynamiczne}

\makeatletter
\def\input@path{{chapter15/}}
\makeatother

\subchapter{Planowanie czynności na liniach montażowych}

\exercise %15.1-1
Poniższa procedura wypisuje wszystkie stanowiska w~kolejności rosnącej -- stanowisko o~numerze $n$ wypisywane jest po stanowiskach o~niższych numerach, co realizowane jest poprzez zastosowanie rekursji.
\begin{codebox}
\Procname{$\proc{Print-Stations}'(l,l^*\!,n)$}
\li	\If $n\ge1$
\li	\Then $\proc{Print-Stations}'(l,l_{l^*\!}[n],n-1)$
\li		wypisz ,,linia '' $l^*\!$ ,,{}, stanowisko '' $n$
	\End
\end{codebox}
Do wywołania rekurencyjnego przekazywany jest numer wykorzystanej linii na stanowisku $n-1$, który odczytywany jest z~$l_{l^*\!}[n]$.
Imitujemy dzięki temu zachowanie pętli z~oryginalnej procedury.

\exercise %15.1-2
W~pierwszym kroku indukcyjnym wzór oczywiście zachodzi:
\[
	r_1(n) = r_2(n) = 1 = 2^{n-n}.
\]
Niech teraz $j=1$, 2, \dots, $n-1$ i~przyjmijmy, że $r_1(j+1)=r_2(j+1)=2^{n-(j+1)}$.
Stąd
\[
	r_1(j) = r_2(j) = r_1(j+1)+r_2(j+1) = 2^{n-(j+1)}+2^{n-(j+1)} = 2^{n-(j+1)+1} = 2^{n-j},
\]
a~więc wzór jest prawdziwy w~ogólnym przypadku.

\exercise %15.1-3
Wykorzystując wynik poprzedniego zadania oraz wzór A.5, mamy:
\[
	\sum_{i=1}^2\sum_{j=1}^nr_i(j) = 2\sum_{j=1}^n2^{n-j} = 2\sum_{j=0}^{n-1}2^j = 2\cdot\frac{2^n-1}{2-1} = 2^{n+1}-2.
\]

\exercise %15.1-4
Możemy zrezygnować z~większości komórek tablic $f_i$.
Wystarczy zauważyć, że jedynymi wartościami z~tablic $f_i$ potrzebnymi do obliczenia $f_i[j]$ są $f_i[j-1]$.
Tablice $f_i$ nie są też wykorzystywane podczas wypisywania optymalnego rozwiązania w~procedurze \proc{Print-Stations}.
Można więc w~procedurze \proc{Fastest-Way} zaalokować jedynie $f_i[1\twodots2]$, czyli w~sumie 4 komórki.
Na pozycji $f_i[1]$ przechowamy wartości dla poprzedniego stanowiska i~wykorzystamy je do obliczenia wartości dla obecnego stanowiska, co zapiszemy w~$f_i[2]$.
Na początku każdej iteracji pętli \kw{for} $f_i[2]$ będzie przepisywane do $f_i[1]$.
Dzięki tej modyfikacji tablice $f_i$ i~$l_i$ zawierają łącznie $4+(2n-2)=2n+2$ pozycji.

\exercise %15.1-5

\subchapter{Mnożenie ciągu macierzy}

\exercise %15.2-1
Posługując się algorytmami \proc{Matrix-Chain-Order} i~\proc{Print-Optimal-Parens}, dostajemy, że dla ciągu macierzy $\langle A_1,\dots,A_6\rangle$ o~zadanych rozmiarach, optymalnym nawiasowaniem jest $((A_1A_2)((A_3A_4)(A_5A_6)))$.
Mnożąc macierze zgodnie z~tym nawiasowaniem, wykonamy 2010 mnożeń skalarnych.

\exercise %15.2-2
\exercise %15.2-3
\exercise %15.2-4
\exercise %15.2-5

\subchapter{Podstawy programowania dynamicznego}

\exercise %15.3-1
\exercise %15.3-2
\exercise %15.3-3
\exercise %15.3-4
\exercise %15.3-5

\subchapter{Najdłuższy wspólny podciąg}

\exercise %15.4-1
Dla zadanych ciągów NWP wyznaczonym przez procedury \proc{LCS-Length} i~\proc{Print-LCS} jest $\langle1,0,0,1,1,0\rangle$.
Poza nim istnieje jeszcze 7 innych NWP tych ciągów:
\begin{gather*}
	\langle0,0,1,0,1,0\rangle, \langle0,0,1,0,1,1\rangle, \langle0,0,1,1,0,1\rangle, \langle0,1,0,1,0,1\rangle, \\
	\langle1,0,1,0,1,0\rangle, \langle1,0,1,0,1,1\rangle, \langle1,0,1,1,0,1\rangle.
\end{gather*}

\exercise %15.4-2
Poniższy pseudokod stanowi implementację zmodyfikowanej wersji procedury \proc{Print-LCS}, która wypisuje NWP ciągów $X$, $Y$ bez korzystania z~tablicy $b$.
\begin{codebox}
\Procname{$\proc{Print-LCS}'(c,X,Y,i,j)$}
\li	\If $i=0$ lub $j=0$
\li		\Then \Return
		\End
\li	\If $x_i=y_j$
\li		\Then $\proc{Print-LCS}'(c,X,Y,i-1,j-1)$
\li			wypisz $x_i$
\li		\ElseIf $c[i,j]=c[i-1,j]$
\li			\Then $\proc{Print-LCS}'(c,X,Y,i-1,j)$
\li		\ElseNoIf $\proc{Print-LCS}'(c,X,Y,i,j-1)$
		\End
\end{codebox}

\exercise %15.4-3
W~naszej implementacji będziemy obliczać kolejne wartości rekurencyjnie bezpośrednio ze wzoru (15.14), ale z~wykorzystaniem tablicy $c[1\twodots m,1\twodots n]$ i~mechanizmu spamiętywania.
Aby zaznaczyć, że dane pole tablicy $c$ nie zostało jeszcze obliczone, użyjemy specjalnej wartości $\infty$.
Pierwsza z~procedur inicjalizuje tablicę $c$, po czym wywołuje właściwy algorytm odpowiedzialny za obliczenie wartości $c[m,n]$.
\begin{codebox}
\Procname{$\proc{Memoized-LCS-Length}(X,Y)$}
\li	$m\gets\attrib{X}{length}$
\li	$n\gets\attrib{Y}{length}$
\li	\For $i\gets0$ \To $m$
\li		\Do \For $j\gets0$ \To $n$
\li				\Do $c[i,j]\gets\infty$ \label{li:memoized-lcs-length-init}
				\End
		\End
\li	\Return $\proc{Lookup-LCS}(c,X,Y,m,n)$
\end{codebox}
\begin{codebox}
\Procname{$\proc{Lookup-LCS}(c,X,Y,i,j)$}
\li	\If $c[i,j]<\infty$
\li		\Then \Return $c[i,j]$
		\End
\li	\If $i=0$ lub $j=0$
\li		\Then $c[i,j]\gets0$
\li		\ElseIf $x_i=y_j$
\li			\Then $c[i,j]\gets\proc{Lookup-LCS}(c,X,Y,i-1,j-1)+1$
\li		\ElseNoIf $c[i,j]\gets\max(\proc{Lookup-LCS}(c,X,Y,i,j-1),\proc{Lookup-LCS}(c,X,Y,i-1,j))$
		\End
\li	\Return $c[i,j]$
\end{codebox}

Każde z~$(m+1)(n+1)$ pól tablicy $c$ zostaje zainicjalizowane w~wierszu \ref{li:memoized-lcs-length-init}, a~następnie zmodyfikowane przez jedno wywołanie procedury \proc{Lookup-LCS}.
Wywołania procedury \proc{Lookup-LCS} możemy podzielić na dwa typy:
\begin{enumerate}
	\item wywołania, w~których $c[i,j]=\infty$, oraz
	\item wywołania, w~których $c[i,j]<\infty$.
\end{enumerate}
Wywołań pierwszego typu jest dokładnie $\Theta(mn)$, jedno na każde pole tablicy $c$.
Wszystkie wywołania drugiego typu pojawiają się jako rekurencyjne wywołania w~wywołaniach pierwszego typu.
Kiedy w~danym wywołaniu \proc{Lookup-LCS} pojawiają się wywołania rekurencyjne, jest ich $\Theta(1)$, dlatego łącznie wywołań drugiego typu jest $\Theta(mn)$.
Każde wywołanie drugiego typu zabiera czas $\Theta(1)$, a~każde wywołanie pierwszego typu jest wykonywane w~czasie $O(1)$ plus czas spędzany na obliczenia rekurencyjne.
Dlatego łączny czas wykonania algorytmu \proc{Memoized-LCS-Length} wynosi $\Theta(mn)$.

\exercise %15.4-4
Zauważmy, że do obliczenia $c[i,j]$ w~procedurze \proc{LCS-Length} wykorzystywane są wartości $c[i-1,j]$, $c[i,j-1]$ lub $c[i-1,j-1]$.
Możemy więc utrzymywać tylko 2 wiersze tablicy $c$ -- wiersz aktualnie obliczany oraz wiersz bezpośrednio go poprzedzający.
Po każdej iteracji poprzedni wiersz będzie nadpisywany wartościami z~bieżącego wiersza, aby przygotować tablicę do kolejnej iteracji.
Aby tablica $c$ była rozmiaru $2\cdot\min(m,n)$, przed jej utworzeniem należy jeszcze sprawdzić, czy $n\le m$.
Jeśli nie, to wystarczy uruchomić procedurę z~ciągami $X$ i~$Y$ zamienionymi miejscami.

Można jednak jeszcze bardziej ograniczyć zapotrzebowanie procedury na pamięć.
Dwuwymiarowa tablica $c$ może zostać zamieniona na jednowymiarową tablicę $C[1\twodots n]$, która reprezentować będzie aktualnie obliczany wiersz tablicy $c$ w~trakcie działania procedury.
W~momencie obliczania $c[i,j]$ reprezentowanego przez $C[j]$, komórka $C[j]$ przechowuje wartość $c[i-1,j]$ z~poprzedniego wiersza.
Jeśli w~osobnej zmiennej $p$ zapamiętamy $c[i-1,j-1]$, czyli wartość $C[j-1]$ przed jej aktualizacją, to będziemy dysponować wszystkimi danymi potrzebnymi do obliczenia $C[j]$.
Jeśli $x_i=y_j$, to $C[j]$ ustawione zostanie na $p+1$, co odpowiada wierszowi 10 z~oryginalnej procedury \proc{LCS-Length}.
W~przeciwnym przypadku do $C[j]$ wpisane zostanie $\max(C[j],C[j-1])$, co odpowiada wierszom \doubledash{12}{16}.
Ostatnim krokiem w~bieżącej iteracji pętli jest aktualizacja zmiennej $p$ na wartość znajdującą się w~$C[j]$ przed modyfikacją tej komórki.
Dzięki tak wprowadzonym zmianom w~procedurze wykorzystujemy $\min(m,n)+\Theta(1)$ pamięci.

\exercise %15.4-5
W~problemie najdłuższego niemalejącego podciągu (w~skrócie: NNP) dla danego ciągu $X=\langle x_1,x_2,\dots,x_n\rangle$ szukany jest podciąg ciągu $X$ o~największej możliwej długości, którego wyrazy ustawione są w~porządku niemalejącym.
Przez $c[i]$ oznaczmy długość najdłuższego niemalejącego podciągu ciągu $X$, w~którym ostatnim wyrazem jest $x_i$.
NNP ciągu $X$ zakończony wyrazem $x_i$ składa się z~NNP prefiksu $X_{i-1}$ z~dołączonym na końcu $x_i$, o~ile tylko NNP prefiksu $X_{i-1}$ zakończony jest wyrazem $x_j\le x_i$.
Jeśli nie istnieje takie $1\le j<i$, to NNP ciągu $X$ jest $\langle x_i\rangle$.
Problem ma zatem własność optymalnej podstruktury.
Aby znaleźć długość NNP danego ciągu, należy wpierw znaleźć rozwiązania podproblemów, czyli długości NNP prefiksów tego ciągu.
Otrzymujemy zależność rekurencyjną:
\[
	c[i] = \begin{cases}
		1, & \text{jeśli $x_j>x_i$ dla każdego $1\le j<i$}, \\
		\displaystyle\max_{\substack{1\le j<i\\x_j\le x_i}}(c[j])+1, & \text{w~przeciwnym przypadku}.
	\end{cases}
\]
NNP ciągu $X$ ma wówczas długość $\max_{1\le i\le n}c[i]$.

Obliczanie $c[n]$ bezpośrednio z~definicji doprowadziłoby do algorytmu o~czasie wykładniczym z~racji wielokrotnego rozwiązywania tych samych podproblemów.
Dzięki wykorzystaniu tablicowania wyników w~programowaniu dynamicznym, możemy podać algorytm działający w~czasie $O(n^2)$.
\begin{codebox}
\Procname{$\proc{LIS-Length}(X)$}
\li	$n\gets\attrib{X}{length}$
\li	$\id{longest}\gets0$
\li	$b^*\!\gets0$
\li	\For $i\gets1$ \To $n$ \label{li:lis-length-computing-c-begin}
\li		\Do $c[i]\gets1$
\li			$b[i]\gets0$
\li			\For $j\gets1$ \To $i-1$
\li				\Do \If $x_j\le x_i$ i~$c[j]+1>c[i]$
\li						\Then $c[i]\gets c[j]+1$
\li							$b[i]\gets j$
						\End
				\End
\li			\If $c[i]>\id{longest}$
\li				\Then $\id{longest}\gets c[i]$
\li					$b^*\!\gets i$
				\End
		\End \label{li:lis-length-computing-c-end}
\li	\Return $\id{longest}$, $b$ i~$b^*$
\end{codebox}
W~pętli \kw{for} w~wierszach \doubledash{\ref{li:lis-length-computing-c-begin}}{\ref{li:lis-length-computing-c-end}} wyznaczane są kolejne pola tablicy $c$, a~także aktualizowana jest zmienna \id{longest} przechowująca maksimum aktualnie znanego NNP wejściowego ciągu $X$.
W~tablicy $b[1\twodots n]$ zapamiętywane są dodatkowe informacje służące później do odtworzenia znalezionego NNP.
Dokładniej, po zakończeniu działania procedury, w~$b[i]$ znajduje się ta wartość $j$, dla której $c[i]=c[j]+1$, albo $b[i]=0$, jeżeli takie $j$ nie istnieje.
Ponadto, $b^*\!$ przechowuje takie $i$, dla którego $\id{longest}=c[i]$, albo 0, jeżeli ciąg $X$ jest pusty.

Znaleziony podciąg można otrzymać (w~odwrotnej kolejności) poprzez wypisanie $x_{b^*\!}$, a~następnie $x_{b[b^*\!]}$, $x_{b[b[b^*\!]]}$ itd., dopóki indeks nie jest zerem.
Następująca rekurencyjna procedura pozwala wypisać NNP we właściwej kolejności.
\begin{codebox}
\Procname{$\proc{Print-LIS}(b,X,i)$}
\li	\If $i>0$
\li		\Then $\proc{Print-LIS}(b,X,b[i])$
\li			wypisz $x_i$
		\End
\end{codebox}
Dysponując tablicą $b$ i~wartością $b^*\!$ wyznaczonymi dla danego ciągu $X$, możemy wypisać NNP ciągu $X$, wywołując $\proc{Print-LIS}(b,X,b^*\!)$.

\exercise %15.4-6
Dla ciągu wejściowego $X=\langle x_1,x_2,\dots,x_n\rangle$ niech $Y=\langle y_1,y_2,\dots,y_m\rangle$ będzie ciągiem takim, że $y_i$ jest najmniejszym elementem w~$X$ kończącym pewien podciąg niemalejący ciągu $X$ o~długości $i$.
Kluczowa jest własność ciągu $Y$ podana we wskazówce, czyli to, że jest on niemalejący.
Podamy krótki dowód tego faktu.
Załóżmy, że istnieją indeksy $i$, $j$, takie, że $i<j$ oraz $y_i>y_j$.
Oznacza to, że najmniejszy element ciągu $X$ będący ostatnim elementem niemalejącego podciągu $X$ o~długości $i$ jest większy niż najmniejszy element ciągu $X$ kończący pewien niemalejący podciąg $X$ o~długości $j$.
Moglibyśmy jednak usunąć $j-i$ elementów z~podciągu o~długości $j$ zakończonego na $y_j$, otrzymując niemalejący podciąg ciągu $X$ o~długości $i$, w~którym ostatni element jest mniejszy niż $y_i$.
Sprzeczność.

Zamiast zapamiętywania ciągu $Y$, w~algorytmie będziemy korzystać z~tablicy $a[0\twodots n]$, w~której na \singledash{$j$}{tej} pozycji, gdzie $j=1$, 2, \dots, $m$, będziemy przechowywać indeks $i$, taki że $y_j=x_i$.
Na pozycjach $j=0$ oraz $j=m+1$, $m+2$, \dots, $n$, $a[j]=0$.
Przechowywanie indeksów elementów pozwoli nam na zbudowanie tablicy $b$ identycznej, jak w~procedurze \proc{LIS-Length} z~poprzedniego zadania, dzięki której będzie można odtworzyć znaleziony NNP.

Załóżmy, że w~trakcie przeglądania ciągu $X$ od lewej do prawej, aktualnie przetwarzanym elementem jest $x_i$.
Niech $k$ będzie takim indeksem w~tablicy $a$, że $x_{a[j]}\le x_i$ dla każdego $j=1$, 2, \dots, $k-1$, oraz $x_{a[j]}>x_i$ dla każdego $j=k$, $k+1$, \dots, $n$.
Element $x_i$ stanowi zatem najmniejszy napotkany do tej pory element kończący pewien niemalejący podciąg ciągu $X$ o~długości $k$ -- na pozycję $a[k]$ można więc wpisać $i$.
Dzięki uporządkowaniu elementów ciągu $X$ o~indeksach z~tablicy $a$, liczbę $k$ dla każdego elementu $x_i$ jesteśmy w~stanie wyszukiwać w~czasie logarytmicznym, stosując wyszukiwanie binarne.

\begin{codebox}
\Procname{$\proc{LIS-Length}'(X)$}
\li	$n\gets\attrib{X}{length}$
\li	\For $i\gets0$ \To $n$ \label{li:lis-length'-a-init-begin}
\li		\Do $a[i]\gets0$
		\End \label{li:lis-length'-a-init-end}
\li	$\id{longest}\gets0$
\li	\For $i\gets1$ \To $n$ \label{li:lis-length'-for-begin}
\li		\Do $\id{low}\gets1$ \label{li:lis-length'-binary-search-begin}
\li			$\id{high}\gets\id{longest}$
\li			\While $\id{low}\le\id{high}$
\li				\Do $\id{mid}\gets\lfloor(\id{low}+\id{high})/2\rfloor$
\li					\If $x_i<x_{a[\id{mid}]}$
\li						\Then $\id{high}\gets\id{mid}-1$
\li						\Else $\id{low}\gets\id{mid}+1$
						\End
				\End \label{li:lis-length'-binary-search-end}
\li			$a[\id{low}]\gets i$
\li			$b[i]\gets a[\id{low}-1]$
\li			\If $\id{low}>\id{longest}$
\li				\Then $\id{longest}\gets\id{low}$
				\End
		\End \label{li:lis-length'-for-end}
\li	\Return $\id{longest}$, $b$ i~$a[\id{longest}]$
\end{codebox}
W~wierszach \doubledash{\ref{li:lis-length'-a-init-begin}}{\ref{li:lis-length'-a-init-end}} tablica $a$ inicjalizowana jest zerami, aby zaznaczyć, że żaden z~elementów ciągu $Y$ nie jest jeszcze znany.
W~zmiennej \id{longest} będzie znajdować się długość najdłuższego znanego niemalejącego podciągu $X$.
Elementy ciągu $X$ przeglądane są następnie z~lewej do prawej.
W~wierszach \doubledash{\ref{li:lis-length'-binary-search-begin}}{\ref{li:lis-length'-binary-search-end}} dla danego elementu $x_i$ odbywa się wyszukiwanie odpowiedniego indeksu tablicy $a$, gdzie należy umieścić $i$, aby zachować uporządkowanie elementów ciągu $X$ indukowane przez fragment $a[1\twodots\id{longest}]$.
Po aktualizacji odpowiedniej komórki tablicy $a$, wpisywany jest do $b[i]$ indeks poprzedniego elementu w~ciągu $Y$, a~następnie aktualizowana wartość zmiennej \id{longest}.
Procedura zwraca ten sam zestaw wyników, co \proc{LIS-Length} z~poprzedniego zadania.
Znaleziony NNP można więc wypisać, wywołując procedurę \proc{Print-LIS} tak, jak to opisano w~tamtym rozwiązaniu.

Każda iteracja pętli \kw{for} w~liniach \doubledash{\ref{li:lis-length'-for-begin}}{\ref{li:lis-length'-for-end}} przeprowadza wyszukiwanie binarne w~tablicy o~co najwyżej $n$ elementach.
Poza pętlą inicjalizującą tablicę $a$ pozostałe operacje wykonywane w~procedurze zajmują czas co najwyżej stały.
Stąd czasem działania algorytmu jest $O(n\lg n)$.

\subchapter{Optymalne drzewa wyszukiwań binarnych}

\exercise %15.5-1
\exercise %15.5-2
\exercise %15.5-3
\exercise %15.5-4


\problems

\problem{Bitoniczny problem komiwojażera} %15-1
Podążając za wskazówką z~treści problemu, będziemy przeglądać punkty wejściowe od lewej do prawej, po uprzednim posortowaniu ich po współrzędnych $x$.
Tak uporządkowane punkty oznaczymy przez $p_1$, $p_2$, \dots, $p_n$, a~więc $p_1$ jest punktem wysuniętym najbardziej na lewo, a~$p_n$ punktem wysuniętym najbardziej na prawo.

Dla $1\le i\le j\le n$ rozważmy zbiory $B_{i,j}$ ścieżek zawierających każdy punkt $p_1$, $p_2$, \dots, $p_j$ dokładnie raz, z~wyjątkiem przypadku $i=j$, w~którym $p_j$ występuje dokładnie 2 razy, i~mających postać $\langle p_i,\dots,p_1,\dots,p_j\rangle$, przy czym podścieżka $\langle p_i,\dots,p_1\rangle$ (złożona z~tylko jednego punktu, gdy $i=1$) biegnie po punktach o~malejących indeksach, a~podścieżka $\langle p_1,\dots,p_j\rangle$ (złożona z~tylko jednego punktu, gdy $j=1$) -- po punktach o~rosnących indeksach.
Przez $|pp'|$ oznaczymy odległość euklidesową między punktami $p$ i~$p'$, a~przez $b[i,j]$ -- długość mierzoną odległością euklidesową najkrótszej ścieżki ze zbioru $B_{i,j}$.
Zauważmy, że zbiory $B_{j,j}$ składają się ze ścieżek bitonicznych między punktami $p_1$, $p_2$, \dots, $p_j$.
Rozwiązanie problemu stanowi więc wartość $b[n,n]$.

Niech $\beta_{i,j}$ będzie najkrótszą ścieżką w~$B_{i,j}$.
Oczywiście $b[1,1]=0$, przyjmijmy więc, że $j>1$.
Gdy $i=1$ lub $i<j-1$, bezpośrednio przed $p_j$ na $\beta_{i,j}$ znajduje się punkt $p_{j-1}$.
Podścieżka $\langle p_i,\dots,p_{j-1}\rangle$ musi być najkrótszą ścieżką z~$B_{i,j-1}$; inaczej moglibyśmy ją ,,wyciąć'' i~,,wkleić'' w~jej miejsce podścieżkę z~tego zbioru o~mniejszej długości, uzyskując ścieżkę krótszą niż $\beta_{i,j}$.
Stąd długość $b[i,j]$ ścieżki $\beta_{i,j}$ wynosi $b[i,j-1]+|p_{j-1}p_j|$.
Jeśli z~kolei $i>1$ oraz zachodzi $i=j-1$ lub $i=j$, to punkt $p_j$ jest bezpośrednio poprzedzony pewnym punktem $p_k$, gdzie $k<i$.
Tutaj także ma miejsce optymalna podstruktura -- podścieżka $\langle p_i,\dots,p_k\rangle$ jest odwróconą najkrótszą ścieżką z~$B_{k,i}$, o~czym można się przekonać, stosując metodę ,,wytnij i~wklej''.
W~tym przypadku ścieżka $\beta_{i,j}$ ma długość $b[i,j]=\min_{1\le k<i}(b[k,i]+|p_kp_j|)$.
Na podstawie tej analizy dostajemy następującą zależność rekurencyjną:
\[
	b[i,j] = \begin{cases}
		0, & \text{jeśli $i=j=1$}, \\
		b[i,j-1]+|p_{j-1}p_j|, & \text{jeśli $i=1<j$ lub $i<j-1$}, \\
		\displaystyle\min_{1\le k<i}(b[k,i]+|p_kp_j|), & \text{w~pozostałych przypadkach}.
	\end{cases}
\]

Aby zrekonstruować rozwiązanie, dla każdych $1\le i<j\le n$ obliczymy $r[i,j]$, czyli indeks punktu bezpośrednio poprzedzającego $p_j$ na ścieżce $\beta_{i,j}$.
Poniższy pseudokod wyznacza wartości w~tablicach $b$ i~$r$, wykorzystując programowanie dynamiczne.
\begin{codebox}
\Procname{$\proc{Bitonic-TSP}(p)$}
\li	$n\gets\attrib{p}{length}$
\li	posortuj ciąg punktów wejściowych $p$ rosnąco względem ich współrzędnych $x$ \label{li:bitonic-tsp-sorting}
\li	$b[1,1]\gets0$
\li	\For $j\gets2$ \To $n$
\li		\Do \For $i\gets1$ \To $j$
\li				\Do \If $i=1$ lub $i<j-1$
\li						\Then $b[i,j]\gets b[i,j-1]+|p_{j-1}p_j|$
\li							$r[i,j]\gets j-1$
\li						\Else $b[i,j]\gets\infty$ \label{li:bitonic-tsp-min-begin}
\li							\For $k\gets1$ \To $i-1$
\li								\Do $q\gets b[k,i]+|p_kp_j|$
\li									\If $q<b[i,j]$
\li										\Then $b[i,j]\gets q$
\li											$r[i,j]\gets k$
										\End
								\End
						\End \label{li:bitonic-tsp-min-end}
				\End
		\End
\li	\Return $b$ i~$r$
\end{codebox}

Dla posortowanego ciągu punktów wejściowych $p$ znalezione rozwiązanie wypiszemy, zaczynając od $p_n$, następnie podamy punkty na podścieżce zawierającej $p_{n-1}$, aż do $p_1$, po czym pozostałe punkty z~podścieżki biegnącej w~prawo aż do $p_n$.
\begin{codebox}
\Procname{$\proc{Print-Bitonic-Path}(p,r)$}
\li	$n\gets\attrib{p}{length}$
\li	wypisz $p_n$
\li	wypisz $p_{n-1}$
\li	$\proc{Print-Path}(p,r,n-1,n)$
\end{codebox}
\begin{codebox}
\Procname{$\proc{Print-Path}(p,r,i,j)$}
\li	\If $i<j$ i~($i>1$ lub $r[i,j]>1$)
\li		\Then $\proc{Print-Path}(p,r,i,r[i,j])$
\li			wypisz $p_{r[i,j]}$
		\End
\li	\If $i>j$ i~($j>1$ lub $r[j,i]>1$)
\li		\Then wypisz $p_{r[j,i]}$
\li			$\proc{Print-Path}(p,r,r[j,i],j)$
		\End
\end{codebox}
Wywołanie $\proc{Print-Path}(p,r,i,j)$ wypisuje ścieżkę o~minimalnej długości między punktami $p_i$ i~$p_j$.
W~zależności od wyniku porównania $i$ z~$j$ wypisane zostają punkty z~podścieżki biegnącej w~lewo albo, w~kolejności odwrotnej do ich napotykania, punkty z~podścieżki biegnącej w~prawo.
Procedura stosuje dodatkowe warunki, aby nie dopuścić do podwójnego wypisania punktu $p_1$.

Dla każdego $j>1$ co najwyżej 2 pozycje $b[i,j]$ obliczane są w~algorytmie \proc{Bitonic-TSP} w~wierszach \doubledash{\ref{li:bitonic-tsp-min-begin}}{\ref{li:bitonic-tsp-min-end}} przy użyciu $O(n)$ iteracji, a~każda inna pozycja tablicy $b$ wyznaczona zostaje w~czasie stałym.
Ponieważ sortowanie punktów w~wierszu \ref{li:bitonic-tsp-sorting} można wykonać w~czasie $O(n\lg n)$, to czasem działania tego algorytmu jest $\Theta(n^2)$.
Wypisanie znalezionej ścieżki bitonicznej za pomocą procedury \proc{Print-Bitonic-Path} odbywa się w~czasie $\Theta(n)$, gdyż każdy punkt wypisywany jest dokładnie raz, a~na każdy z~nich przypada stała liczba wykonywanych operacji.

\problem{Estetyczny wydruk} %15-2

\problem{Odległość redakcyjna} %15-3

\subproblem %15-3(a)
\subproblem %15-3(b)

\problem{Planowanie bankietu w~firmie} %15-4

\problem{Algorytm Viterbiego} %15-5

\subproblem %15-5(a)
\subproblem %15-5(b)

\problem{Przesuwanie pionka} %15-6
Ponumerujmy kolejnymi liczbami całkowitymi od 1 do $n$ wiersze z~dołu do góry, oraz kolumny z~lewej do prawej.
Każde pole szachownicy utożsamimy z~jego położeniem na szachownicy w~postaci pary $\langle i,j\rangle$, gdzie $i$ to numer wiersza tego pola, a~$j$ -- numer jego kolumny.
Zdefiniujmy $g[i,j]$ dla $1\le i$, $j\le n$ jako największą liczbę złotych możliwą do uzyskania, przechodząc pionkiem z~pewnego pola na dolnym brzegu szachownicy na pole $\langle i,j\rangle$, wykonując przy tym tylko dopuszczalne ruchy.
Oczywiście $g[1,j]=0$ dla $j=1$, 2, \dots, $n$.
Maksymalny zysk w~polu $\langle i,j\rangle$ możemy osiągnąć, przechodząc na nie z~pola $\langle i-1,j-1\rangle$ (o~ile $j>1$), $\langle i-1,j\rangle$ albo $\langle i-1,j+1\rangle$ (o~ile $j<n$).
Każda możliwość do sumarycznego zysku dodaje częściowy zysk związany z~przejściem na pole $\langle i,j\rangle$, zdefiniowanym przez funkcję $p$.
Dla $i=2$, 3, \dots, $n$ zachodzi następująca zależność:
\[
	g[i,j] = \max\begin{cases}
		g[i-1,j-1]+p(\langle i-1,j-1\rangle,\langle i,j\rangle), & \text{jeśli $j>1$,} \\
		g[i-1,j]+p(\langle i-1,j\rangle,\langle i,j\rangle), & \\
		g[i-1,j+1]+p(\langle i-1,j+1\rangle,\langle i,j\rangle), & \text{jeśli $j<n$.}
	\end{cases}
\]

W~naszym algorytmie zastosujemy programowanie dynamiczne w~celu obliczenia wartości w~tablicy $g$.
Ponadto w~tablicy $m$ będziemy przechowywać dane ułatwiające rekonstrukcję znalezionej ścieżki pionka.
Pozycja $m[i,j]$ będzie zawierać numer kolumny pola z~wiersza $i-1$, z~którego prowadzi optymalna ścieżka do pola $\langle i,j\rangle$.
Ponadto w~zmiennej $m^*\!$ po zakończeniu działania algorytmu znajdzie się numer kolumny pola na górnym brzegu szachownicy, na którym kończy się optymalna ścieżka pionka.
\begin{codebox}
\Procname{$\proc{Checkerboard}(n,p)$}
\li	\For $j\gets1$ \To $n$
\li		\Do $g[1,j]\gets0$
		\End
\li	\For $i\gets2$ \To $n$
\li		\Do \For $j\gets1$ \To $n$
\li				\Do $g[i,j]\gets g[i-1,j]+p(\langle i-1,j\rangle,\langle i,j\rangle)$
\li					$m[i,j]\gets j$
\li					\If $j>1$ i~$g[i-1,j-1]+p(\langle i-1,j-1\rangle,\langle i,j\rangle)>g[i,j]$
\li						\Then $g[i,j]\gets g[i-1,j-1]+p(\langle i-1,j-1\rangle,\langle i,j\rangle)$
\li							$m[i,j]\gets j-1$
						\End
\li					\If $j<n$ i~$g[i-1,j+1]+p(\langle i-1,j+1\rangle,\langle i,j\rangle)>g[i,j]$
\li						\Then $g[i,j]\gets g[i-1,j+1]+p(\langle i-1,j+1\rangle,\langle i,j\rangle)$
\li							$m[i,j]\gets j+1$
						\End
				\End
		\End
\li	$\id{result}\gets-\infty$
\li	$m^*\!\gets1$
\li	\For $j\gets1$ \To $n$
\li		\Do \If $g[n,j]>\id{result}$
\li				\Then $\id{result}\gets g[n,j]$
\li					$m^*\!\gets j$
				\End
		\End
\li	\Return $\id{result}$, $m$ i~$m^*\!$
\end{codebox}
Wykorzystując zwrócone wartości przez powyższy algorytm, możemy wypisać optymalną ścieżkę jako ciąg odwiedzonych pól, używając w~tym celu wywołania $\proc{Print-Moves}(m,n,m^*\!)$.
Pseudokod tej procedury przedstawiono poniżej.
\begin{codebox}
\Procname{$\proc{Print-Moves}(m,i,j)$}
\li	\If $i>1$
\li		\Then $\proc{Print-Moves}(m,i-1,m[i,j])$
		\End
\li	wypisz ,,$\langle$'' $i$ ,,{}, '' $j$ ,,$\rangle$''
\end{codebox}

Algorytm \proc{Checkerboard} wypełnia tablicę $g$, dla każdej z~$n^2$ jej komórek wykonując operacje w~czasie ograniczonym przez stałą.
Stąd czasem działania algorytmu jest $\Theta(n^2)$.

\problem{Planowanie prac} %15-7


%\chapter{Algorytmy zachłanne}

\makeatletter
\def\input@path{{chapter16/}}
\makeatother

\subchapter{Problem wyboru zajęć}
\note{W~procedurze \proc{Greedy-Activity-Selector} w~linii 1 zmienna\/ $n$ powinna przyjąć wartość\/ $\attrib{s}{length}-2$.
Wynika to z~tego, że tablica\/ $s$ składa się tak naprawdę z~\/$n+2$ elementów -- oprócz czasów rozpoczęcia \/$n$ rzeczywistych zajęć przechowuje także czasy fikcyjnych zajęć\/ $a_0$ i~\/$a_{n+1}$.}
\bignegskip

\exercise %16.1-1
W~poniższym algorytmie przyjmujemy, że zajęcia zostały uporządkowane niemalejąco według czasów zakończenia.
\begin{codebox}
\Procname{$\proc{Dynamic-Activity-Selector}(s,f)$}
\li	$n\gets\attrib{s}{length}-2$
\li	\For $l\gets2$ \To $n+2$ \label{li:dynamic-activity-selector-main-loop-begin}
\li		\Do \For $i\gets0$ \To $n-l+2$
\li				\Do $j\gets i+l-1$
\li					$c[i,j]\gets0$
\li					$A_{i,j}\gets\emptyset$
\li					\For $k\gets i+1$ \To $j-1$
\li						\Do $q\gets c[i,k]+c[k,j]+1$
\li							\If $f_i\le s_k<f_k\le s_j$ i~$q>c[i,j]$
\li								\Then $c[i,j]\gets q$
\li									$A_{i,j}\gets A_{i,k}\cup\{a_k\}\cup A_{k,j}$
								\End
						\End
				\End
		\End \label{li:dynamic-activity-selector-main-loop-end}
\li	\Return $A_{0,n+1}$
\end{codebox}
Algorytm oblicza wartości $c[i,j]$ oraz konstruuje zbiory $A_{i,j}$ dla $0\le i<j\le n+1$.
W~pierwszej iteracji pętli głównej (wiersze \doubledash{\ref{li:dynamic-activity-selector-main-loop-begin}}{\ref{li:dynamic-activity-selector-main-loop-end}}) dla $i=0$, 1, \dots, $n$ za pomocą równania (16.3) wyznaczane są wartości $c[i,i+1]$, a~na podstawie wzoru (16.2) zbiory $A_{i,i+1}$, czyli rozwiązywane są podproblemy składające się z~dokładnie dwóch zajęć.
W~kolejnej iteracji zostają wyznaczone rozwiązania podproblemów z~dokładnie trzema zajęciami itd.
Zwracanym wynikiem jest najliczniejszy zbiór zajęć stanowiący rozwiązanie podproblemu $S_{0,n+1}=S$.

Struktura pętli zaprezentowanego algorytmu przypomina tę z~procedury \proc{Matrix-Chain-Order} z~podrozdziału 15.2.
Można więc zastosować tu podobną analizę efektywności jak w~przypadku tamtej procedury i~dojść do oszacowania $\Theta(n^3)$ na czas działania naszego algorytmu -- znacznie wyższego od złożoności czasowej rozwiązania zachłannego.

\exercise %16.1-2
Załóżmy, że zajęcia $a_1$, $a_2$, \dots, $a_n$, wraz z~dwoma fikcyjnymi $a_0$ i~$a_{n+1}$, uporządkowane są według czasów rozpoczęcia, tzn.
\[
	s_0 \le s_1 \le s_2 \le \dots \le s_n \le s_{n+1}.
\]
Możemy wówczas udowodnić następujące twierdzenie analogiczne do tw.\ 16.1:

\bigskip
\noindent\textsf{\textbf{Twierdzenie 16.1$'$.}} \textit{Rozważmy niepusty podproblem\/ $S_{i,j}$ i~niech\/ $a_m$ będą zajęciami w~\/$S_{i,j}$ rozpoczynającymi się najpóźniej:
\[
	s_m = \max\bigl\{\,s_k:a_k\in S_{i,j}\,\bigr\}.
\]
Wtedy:
\begin{enumerate}
	\item Wśród najliczniejszych pozdbiorów parami zgodnych zajęć z~\/$S_{i,j}$ istnieje taki, który zawiera\/ $a_m$.
	\item Podproblem\/ $S_{m,j}$ jest pusty, tak więc po wybraniu\/ $a_m$ jedynym niepustym podproblemem może być tylko\/ $S_{i,m}$.
\end{enumerate}
}

Twierdzenie to mówi nam, że w~optymalnym rozwiązaniu podproblemu $S_{i,j}$, po wyborze zajęcia $a_m\in S_{i,j}$ rozpoczynającego się najpóźniej, do rozwiązania pozostanie tylko jeden podproblem, podczas gdy drugi z~nich jest pusty.
O~wyborze zajęć o~najpóźniejszym czasie rozpoczęcia można więc myśleć jak o~wyborze ,,zachłannym'', gdyż pozostawia on jak najwięcej możliwości wybrania zgodnych zajęć, czyli -- podobnie jak w~oryginalnej strategii -- maksymalizuje on ilość czasu do zagospodarowania.
Algorytm realizujący strategię wyboru zajęć o~najpóźniejszym starcie może rozwiązywać każdy podproblem zstępująco przy pomocy rekurencji lub iteracyjnie, analogicznie do oryginalnych procedur zaprezentowanych w~Podręczniku.
Poniższy pseudokod jest implementacją wariantu iteracyjnego.
\begin{codebox}
\Procname{$\proc{Greedy-Activity-Selector}'(s,f)$}
\li	$n\gets\attrib{s}{length}-2$
\li	$A\gets\{a_n\}$
\li	$i\gets n$
\li	\For $m\gets n-1$ \Downto 1
\li		\Do \If $f_m\le s_i$
\li				\Then $A\gets A\cup\{a_m\}$
\li					$i\gets m$
				\End
		\End
\li	\Return $A$
\end{codebox}

\exercise %16.1-3
Założymy, że liczba dostępnych sal wykładowych jest odpowiednio duża, aby pomieścić wszystkie zajęcia.

Wykorzystamy pomysł z~rozwiązania \refExercise{14.3-7}.
Zdarzenia polegające na rozpoczęciu lub zakończeniu zajęć będziemy przeglądać od najwcześniejszego do najpóźniejszego, przy czym w~danej chwili rozpoczęcia będą rozważane przed zakończeniami.
Równocześnie będziemy utrzymywać stos $F$ złożony z~wolnych sal wykładowych oraz zbiór $B$ sal zajmowanych w~danej chwili.
Początkowo zbiór $B$ jest pusty.
Jeśli kolejnym napotkanym zdarzeniem jest rozpoczęcie pewnych zajęć $a_i$, to przypiszemy je do nowej wolnej sali poprzez pobranie jej ze stosu $F$ i~umieszczenie w~zbiorze $B$ wraz z~informacją, że jest ona zajmowana przez zajęcia $a_i$.
Gdy z~kolei napotkamy zakończenie pewnych zajęć $a_j$, to w~zbiorze $B$ należy odszukać jedyną salę zajmowaną przez $a_j$, a~następnie umieścić ją na stosie $F$.

Zastosowanie stosu do przechowywania wolnych sal sprawia, że użyjemy ich tylko tyle, ile potrzeba.
Załóżmy, że w~trakcie działania algorytmu największym rozmiarem zbioru $B$ jest $m$ i~niech $a_k$ będą pierwszymi zajęciami przypisanymi do \singledash{$m$}{tej} sali.
Powodem, dla którego zajęcia $a_k$ zajęły tę salę, jest to, że pozostałe $m-1$ sal było zajętych w~chwili $s_k$, co oznacza, że w~chwili tej odbywa się jednocześnie $m$ zajęć.
Stąd w~każdym przydziale zajęć do sal musi być użytych co najmniej $m$ sal, więc ten zawracany przez nasz algorytm jest optymalny.

Czas działania algorytmu jest sumą czasu spędzonego na sortowaniu rozpoczęć i~zakończeń zajęć oraz czasu spędzonego na ich przeglądaniu i~konstruowaniu rozwiązania.
Do efektywnego zaimplementowania zbioru dynamicznego $B$ można zastosować np.\ drzewa czerwono-czarne.
Jesteśmy więc w~stanie podać implementację tego algorytmu wymagającą czasu $\Theta(n\lg n)$, gdzie $n$ jest ilością zajęć na wejściu.

\exercise %16.1-4
\note{Oryginalna treść zadania wymaga też podania kontrprzykładu dla strategii polegającej na wyborze zajęć o~najwcześniejszym czasie rozpoczęcia spośród zajęć zgodnych z~dotychczas wybranymi.}

\noindent Oto zbiory zajęć stanowiących kontrprzykłady dla strategii wyboru zajęć, odpowiednio, o~najkrótszym czasie trwania oraz o~najwcześniejszym czasie rozpoczęcia:
\begin{center}
	\begin{tabular}{cccc}
		$i$ & 1 & 2 & 3 \\ \hline
		$s_i$ & 5 & 1 & 7 \\
		$f_i$ & 8 & 6 & 13
	\end{tabular}
	\hskip3cm
	\begin{tabular}{cccc}
		$i$ & 1 & 2 & 3 \\ \hline
		$s_i$ & 1 & 3 & 5 \\
		$f_i$ & 6 & 4 & 12
	\end{tabular}
\end{center}
W~obu sytuacjach zastosowanie danej strategii skutkuje uzyskaniem zbioru $\{a_1\}$, podczas gdy zbiór $\{a_2,a_3\}$ złożony z~parami zgodnych zajęć jest liczniejszy.

Dla strategii opartej na wyborze zajęcia kolidującego z~najmniejszą liczbą pozostałych zajęć do rozpatrzenia można podać następujący kontrprzykład:
\begin{center}
	\begin{tabular}{cccccccccccc}
		$i$ & 1 & 2 & 3 & 4 & 5 & 6 & 7 & 8 & 9 & 10 & 11 \\ \hline
		$s_i$ & 1 & 2 & 2 & 2 & 3 & 4 & 5 & 6 & 6 & 6 & 7 \\
		$f_i$ & 3 & 4 & 4 & 4 & 5 & 6 & 7 & 8 & 8 & 8 & 9
	\end{tabular}
\end{center}
Istnieje tylko jeden zbiór 4 parami zgodnych zajęć, $\{a_1,a_5,a_7,a_{11}\}$, natomiast w~obranej strategii w~pierwszym kroku wybierane są zajęcia $a_6$, co prowadzi do rozwiązania będącego zbiorem o~mniej niż 4 elementach.

\subchapter{Podstawy strategii zachłannej}

\exercise %16.2-1
\exercise %16.2-2
Niech $a_1$, $a_2$, \dots, $a_n$ będą przedmiotami, które złodziej będzie wkładał do plecaka.
Przedmioty te ważą, odpowiednio $w_1$, $w_2$, \dots, $w_n$ i~są warte, odpowiednio, $v_1$, $v_2$, \dots, $v_n$.
Niech $S_{i,j}$, dla $0\le i\le n$, $0\le j\le W$, będzie podproblemem polegającym na wybraniu do plecaka przedmiotów spośród $a_1$, \dots, $a_i$ o~sumarycznej wadze nieprzekraczającej $j$ i~możliwie największej sumarycznej wartości.
Podczas rozwiązywania podproblemu $S_{i,j}$, gdzie $i\ge1$, musimy zdecydować, czy włożyć do plecaka przedmiot $a_i$ (o~ile $w_i\le j$).
Jeśli tak, to do sumarycznej wartości konstruowanego plecaka należy dodać wartość $v_i$ wprowadzaną przez ten przedmiot, a~następnie rozwiązać podproblem $S_{i-1,j-w_i}$.
W~przeciwnym przypadku podproblem, jaki zostaje do rozpatrzenia to $S_{i-1,j}$, a~sumaryczna wartość nie zostaje powiększona.
Zdefiniujmy $K[i,j]$ jako największą sumaryczną wartość przedmiotów wchodzących w~skład rozwiązania podproblemu $S_{i,j}$.
Z~naszego rozumowania wynika zależność
\[
	K[i,j] = \begin{cases}
		0, & \text{jeśli $i=0$}, \\
		K[i-1,j], & \text{jeśli $i\ge1$, $w_i>j$}, \\
		\max(K[i-1,j],K[i-1,j-w_i]+v_i), & \text{jeśli $i\ge1$, $w_i\le j$}.
	\end{cases}
\]

Następujący algorytm oparty na programowaniu dynamicznym wylicza kolejne wartości w~tablicy $K$:
\begin{codebox}
\Procname{$\proc{0-1-Knapsack}(w,v,W)$}
\li	$n\gets\attrib{w}{length}$
\li	\For $j\gets0$ \To $W$
\li		\Do $K[0,j]\gets0$
		\End
\li	\For $i\gets1$ \To $n$
\li		\Do \For $j\gets0$ \To $W$
\li				\Do $K[i,j]\gets K[i-1,j]$
\li					\If $w_i\le j$ i~$K[i-1,j-w_i]+v_i>K[i,j]$
\li						\Then $K[i,j]\gets K[i-1,j-w_i]+v_i$
						\End
				\End
		\End
\li	\Return $K$
\end{codebox}
Nietrudno przekonać się, że wypełnienie całej tablicy $K$ wymaga czasu $\Theta(nW)$ i~pamięci tego samego rzędu.

Do wypisania optymalnego zbioru przedmiotów, jakie należy umieścić w~plecaku, służy poniższa procedura.
W~trakcie przeglądania tablicy $K$, w~zależności od podjętej decyzji w~algorytmie \proc{Knapsack}, wypisywany jest odpowiedni przedmiot.
\begin{codebox}
\Procname{$\proc{Print-Knapsack}(K,w,i,j)$}
\li	\If $i\ge1$
\li		\Then \If $K[i,j]=K[i-1,j]$
\li				\Then $\proc{Print-Knapsack}(K,w,i-1,j)$
\li				\Else $\proc{Print-Knapsack}(K,w,i-1,j-w_i)$
\li					wypisz $a_i$
				\End
		\End
\end{codebox}
W~celu wypisania rozwiązania całego problemu, czyli $S_{n,W}$, procedura powinna zostać wywołana jako $\proc{Print-Knapsack}(K,w,n,W)$, co zajmuje czas $\Theta(n)$.

\exercise %16.2-3
Załóżmy, że przedmioty $a_1$, $a_2$, \dots, $a_n$ posortowane są według niemalejących wag, czyli $w_1\le w_2\le\dots\le w_n$.
Z~założenia mamy też, że kolejne wartości tych przedmiotów tworzą ciąg nierosnący: $v_1\ge v_2\ge\dots\ge v_n$.
Przez $A_W$ oznaczmy optymalny podzbiór przedmiotów będący rozwiązaniem dyskretnego problemu plecakowego, gdzie $W$ jest pojemnością plecaka.

Niech $1\le i<j\le n$.
Udowodnimy, że jeśli $a_j\in A_W$, to $a_i\in A_W$, co jest równoważne temu, że istnieje $0\le k\le n$, że $A_W=\{a_1,a_2,\dots,a_k\}$.
Ustalmy $i$, $j$, takie że $1\le i<j\le n$ i~załóżmy nie-wprost, że $a_j\in A_W$, ale $a_i\not\in A_W$.
Rozważmy zbiór $A_W'=A_W\setminus\{a_j\}\cup\{a_i\}$.
Ponieważ $w_i\le w_j$, to elementy $A_W'$ mieszczą się w~plecaku o~pojemności $W$, toteż $A_W'$ może stanowić rozwiązanie problemu.
Jednak suma wartości przedmiotów wchodzących w~skład $A_W'$ jest większa od sumarycznej wartości przedmiotów z~$A_W$ o~$v_i-v_j\ge0$.
Oznacza to, że $A_W'$ jest rozwiązaniem rozważanego problemu plecakowego o~niemniejszej wartości, co stoi w~sprzeczności z~definicją $A_W$.

Udowodniona obserwacja pozwala nam na skonstruowanie rozwiązania za pomocą algorytmu zachłannego.
Wystarczy przeglądać przedmioty od najlżejszych do najcięższych (czyli równoważnie, od najbardziej do najmniej wartościowych) i~włączać aktualny przedmiot do rozwiązania, o~ile tylko wraz z~poprzednio wybranymi przedmiotami nie przekracza pojemności plecaka.

\exercise %16.2-4
\exercise %16.2-5
Oznaczmy przez $X$ zbiór punktów wejściowych, a~przez $k$ prostą, na której leżą punkty z~$X$.
Niech $x\in X$ będzie takim punktem, który wszystkie pozostałe punkty z~$X$ ma po tej samej stronie na prostej $k$.
Odcinek jednostkowy leżący na prostej $k$ zawierający $x$ można umieścić w~taki sposób, aby jeden z~jego końców pokrywał się z~$x$, a~drugi sięgał w~kierunku innych punktów z~$X$.
Żadne inne ustawienie tego odcinka nie prowadzi do lepszego rozwiązania.
Jest tak dlatego, że w~takiej pozycji odcinek pokrywa największą możliwą liczbę innych punktów z~$X$ -- wszystkie te, których odległość od $x$ nie przekracza 1.
Wybór tego położenia jest więc zachłanny i~pozostawia on tylko jeden podproblem do rozwiązania -- taki, w~którym danymi wejściowymi jest podzbiór $X$ składający się z~punktów odległych od $x$ o~więcej niż 1.
Algorytm zachłanny rozwiązujący ten problem może więc znajdować punkt $x$ na prostej $k$, a~następnie usuwać wszystkie punkty z~$X$ odległe od $x$ o~nie więcej niż 1, po czym rozwiązywać pozostały podproblem.

\exercise %16.2-6
\exercise %16.2-7
Posortujmy zbiór $A$ i~zbiór $B$ niemalejąco.
Niech $1\le i<j\le n$.
Zachodzi wtedy $a_i\le a_j$ oraz $b_i\le b_j$, skąd
\[
	\frac{a_i^{b_i}a_j^{b_j}}{a_i^{b_j}a_j^{b_i}} = \frac{a_j^{b_j-b_i}}{a_i^{b_j-b_i}} = \biggl(\frac{a_j}{a_i}\biggr)^{b_j-b_i} \ge 1.
\]
Ostatnia nierówność zachodzi, ponieważ $a_j/a_i\ge1$ i~$b_j-b_i\ge0$.
Wynika z~tego, że w~iloczynie będącym zyskiem z~uporządkowania bardziej opłaca się mieć czynniki będące potęgami $a^b$, w~których podstawa $a$ jest tą samą statystyką pozycyjną w~zbiorze $A$, co wykładnik $b$ w~zbiorze $B$.
Jednym z~uporządkowań maksymalizujących zysk jest więc porządek niemalejący obu zbiorów, lub porządek nierosnący obu zbiorów.
Algorytm rozwiązujący ten problem może zatem jedynie sortować niemalejąco oba zbiory, co wymaga czasu $\Theta(n\lg n)$.

\subchapter{Kody Huffmana}

\exercise %16.3-1
Drzewo $T$, które nie jest regularne, zawiera co najmniej jeden węzeł $x$ o~stopniu 1.
Jeśli węzeł ten zostanie ,,wycięty'' z~$T$, to głębokość każdego potomka węzła $x$ zmniejszy się o~1 i~powstałe w~ten sposób drzewo będzie mieć koszt niższy od $T$.
Można ten koszt obniżać, pozbywając się w~ten sposób wszystkich węzłów stopnia 1 i~uzyskując drzewo regularne.

\exercise %16.3-2
\exercise %16.3-3
\exercise %16.3-4
\exercise %16.3-5
\exercise %16.3-6
\exercise %16.3-7
\exercise %16.3-8

\subchapter{Teoretyczne podstawy strategii zachłannych}

\exercise %16.4-1
Rodzina $\mathcal{I}_k$ jest dziedziczna, bo każdy podzbiór zbioru o~mocy co najwyżej $k$ też jest mocy co najwyżej $k$.
Jeśli teraz wybierzemy $A$, $B\in\mathcal{I}_k$ takie, że $|A|<|B|$, to dla każdego $x\in B\setminus A$, $A\cup\{x\}\in\mathcal{I}_k$, bo $|A\cup\{x\}|\le|B|\le k$.
A~zatem $M=\langle S,\mathcal{I}_k\rangle$ spełnia własność wymiany, co kończy dowód, że $M$ jest matroidem.

\exercise %16.4-2
Oczywiście zbiór $S$ jest skończony.
Dziedziczność rodziny $\mathcal{I}$ wynika z~faktu, że każdy podzbiór zbioru kolumn liniowo niezależnych także składa się z~kolumn liniowo niezależnych.

W~dowodzie własności wymiany wykorzystamy pojęcie rzędu macierzy, czyli mocy największego zbioru jej kolumn liniowo niezależnych (patrz podrozdział 28.1).
Załóżmy, że $A$, $B\in\mathcal{I}$ i~$|A|<|B|$.
Kolumny w~zbiorze $A$ są niezależne, więc jeśli zbudujemy z~nich macierz $T_A$, to $\mathrm{rank}(T_A)=|A|$, i~podobnie z~macierzą $T_B$ złożoną z~kolumn ze zbioru $B$, $\mathrm{rank}(T_B)=|B|$.
Ponieważ $|A|<|B|$, to $\mathrm{rank}(T_A)<\mathrm{rank}(T_B)$.

Załóżmy teraz nie-wprost, że każda kolumna w~$B$ jest liniową kombinacją kolumn w~$A$.
Oznacza to, że każdy wektor będący liniową kombinacją kolumn z~$B$ jest także liniową kombinacją kolumn z~$A$.
Zatem przestrzeń liniowa generowana przez wektory będące kolumnami macierzy $T_B$ stanowi podprzestrzeń przestrzeni liniowej generowanej przez kolumny macierzy $T_A$.
Z~własności podprzestrzeni, jej baza (czyli zbiór wektorów liniowo niezależnych generujący tę podprzestrzeń) ma moc nie większą niż baza obejmującej ją przestrzeni; zachodzi zatem $\mathrm{rank}(T_B)\le\mathrm{rank}(T_A)$, co jest sprzeczne z~poprzednią nierównością.

\exercise %16.4-3
\exercise %16.4-4
Niech $B\in\mathcal{I}$.
Z~definicji rodziny $\mathcal{I}$ mamy, że do zbioru $B$ należy po co najwyżej jednym elemencie z~każdego zbioru $S_1$, $S_2$, \dots, $S_k$.
Oczywiście każdy jego podzbiór $A$ też spełnia taką własność, dlatego $A\in\mathcal{I}$.

Wybierzmy teraz $A,B\in\mathcal{I}$ takie, że $|A|<|B|$, a~także $x\in B\setminus A$.
Istnieje więc $1\le j\le k$, dla którego $x\in S_j$.
Wynika stąd, że $A\cap S_j=\emptyset$ oraz $|(A\cup\{x\})\cap S_j|=1$.
Ponadto dla każdego $i\ne j$, $(A\cup\{x\})\cap S_i=A\cap S_i$, dlatego $|(A\cup\{x\})\cap S_i|\le1$ na mocy przynależności zbioru $A$ do rodziny $\mathcal{I}$.
To kończy dowód, że para $\langle S,\mathcal{I}\rangle$ jest matroidem.

\exercise %16.4-5
Niech $M=\langle S,\mathcal{I}\rangle$ będzie matroidem ważonym z~funkcją wagi $w$ i~niech $W$ będzie liczbą większą niż waga któregokolwiek elementu zbioru $S$.
Dla każdego $x\in S$ zdefiniujmy nową dodatnią funkcję wagi $w'(x)=W-w(x)$.
Wówczas dla dowolnego $A\in\mathcal{I}$,
\[
	w'(A) = \sum_{x\in A}w'(x) = \sum_{x\in A}(W-w(x)) = W|A|-w(A).
\]
Niech $A\in\mathcal{I}$ będzie maksymalnym zbiorem, dla którego $w(A)$ jest maksymalne, zaś $A'\in\mathcal{I}$ -- innym zbiorem maksymalnym.
Z~tw.\ 16.6 mamy, że $|A|=|A'|$.
Zachodzi
\[
	w'(A)-w'(A') = W|A|-w(A)-(W|A'|-w(A')) = w(A')-w(A) \le 0,
\]
skąd $w'(A)\le w'(A')$.
Zatem każdy maksymalny zbiór matroidu $M$ maksymalizuje wagę $w$ wtedy i~tylko wtedy, gdy minimalizuje wagę $w'$.

\subchapter{Problem szeregowania zadań}

\exercise %16.5-1
Dla tak zmodyfikowanych kar algorytm zachłanny wybierze kolejno zadania $a_7$, $a_6$, $a_5$, $a_4$ i~$a_3$, a~odrzuci $a_2$ i~$a_1$.
W~otrzymanym optymalnym uszeregowaniu $\langle a_5,a_4,a_6,a_3,a_7,a_2,a_1\rangle$ suma kar wynosi $w_2+w_1=30$.

\exercise %16.5-2


\problems

\problem{Wydawanie reszty} %16-1

\subproblem %16-1(a)
Optymalna reszta o~wartości $n$ składa się z~monety o~pewnym nominale $d$ (o~ile $d\le n$) plus optymalnej reszty dla podproblemu wydawania reszty o~wartości $n-d$.
Ma tutaj zastosowanie standardowe rozumowanie ,,wytnij i~wklej''.
Załóżmy, że dysponujemy rozwiązaniem dla reszty o~wartości $n$, w~którym nie jest uwzględniona moneta o~największym nominale nieprzekraczającym $n$.
Moglibyśmy jednak dołączyć taką monetę do rozwiązania, zastępując nią kilka innych o~niższych nominałach:
\begin{itemize}
	\item $50\ \mathrm{gr}=20\ \mathrm{gr}+20\ \mathrm{gr}+10\ \mathrm{gr}$,
	\item $20\ \mathrm{gr}=10\ \mathrm{gr}+10\ \mathrm{gr}$,
	\item $10\ \mathrm{gr}=5\ \mathrm{gr}+5\ \mathrm{gr}$,
	\item $5\ \mathrm{gr}=2\ \mathrm{gr}+2\ \mathrm{gr}+1\ \mathrm{gr}$,
	\item $2\ \mathrm{gr}=1\ \mathrm{gr}+1\ \mathrm{gr}$.
\end{itemize}
Zamiana każdego takiego zestawu monet w~rozwiązaniu pojedynczą monetą o~sumarycznym nominale pozwala zredukować liczbę użytych monet i~daje lepsze rozwiązanie.
Problem ma zatem zarówno własność optymalnej podstruktury, jak i~własność wyboru zachłannego.

Można więc użyć algorytmu zachłannego, w~którym dla reszty o~wartości $n$ do rozwiązania dodawany będzie największy nominał $d$ nieprzekraczający $n$, po czym algorytm zostanie wywołany dla reszty o~wartości $n-d$, o~ile $n-d>0$.
Łatwo jednak przekształcić ten algorytm w~taki, który działa w~czasie $O(1)$.
Wystarczy zauważyć, że dany nominał będzie wybierany najdłużej jak to tylko możliwe i~do wydania reszty o~wartości $n$ użytych zostanie kolejno:
\begin{itemize}
	\item $\lfloor n/50\rfloor$ monet \singledash{50}{groszowych}, pozostawiając resztę o~wartości $n_{50}=n\bmod50$,
	\item $\lfloor n_{50}/20\rfloor$ monet \singledash{20}{groszowych}, pozostawiając resztę o~wartości $n_{20}=n_{50}\bmod20$,
	\item $\lfloor n_{20}/10\rfloor$ monet \singledash{10}{groszowych}, pozostawiając resztę o~wartości $n_{10}=n_{20}\bmod10$,
	\item $\lfloor n_{10}/5\rfloor$ monet \singledash{5}{groszowych}, pozostawiając resztę o~wartości $n_5=n_{10}\bmod5$,
	\item $\lfloor n_5/2\rfloor$ monet \singledash{2}{groszowych}, pozostawiając resztę o~wartości $n_2=n_5\bmod2$,
	\item $n_2$ monet \singledash{1}{groszowych}.
\end{itemize}

\subproblem %16-1(b)
Do uzasadnienia poprawności algorytmu zachłannego w~tym przypadku wystarczy użyć argumentacji z~punktu (a) zmodyfikowanej jedynie w~części mówiącej o~wzajemnych powiązaniach dostępnych nominałów.
Dla każdego $c>1$ oraz $i=1$, 2, \dots, $k$ zachodzi $c^i=c\cdot c^{i-1}$, toteż własność wyboru zachłannego w~tym przypadku jest zachowana.

Algorytm zachłanny będzie przeglądał nominały w~kolejności malejącej.
Załóżmy, że aktualnym przetwarzanym nominałem jest $c^i$, a~pozostałą resztą do wydania jest $m$.
Do rozwiązania zostanie wówczas dołączone $\lfloor m/c^i\rfloor$ monet o~nominale $c^i$, a~w~następnym kroku wydawana będzie reszta $m\bmod c^i$.
Do znalezienia optymalnego rozwiązania algorytm potrzebuje czasu $\Theta(k)$.

\subproblem %16-1(c)
Jeśli do dyspozycji będą nominały 1~gr, 3~gr i~4~gr, to np.\ reszta o~wartości 6~gr przez opisany algorytm zachłanny zostanie wydana za pomocą 3 monet: $6\ \mathrm{gr}=4\ \mathrm{gr}+1\ \mathrm{gr}+1\ \mathrm{gr}$, podczas gdy w~optymalnym rozwiązaniu wystarczą 2 monety: $6\ \mathrm{gr}=3\ \mathrm{gr}+3\ \mathrm{gr}$.

\subproblem %16-1(d)
Oznaczmy dopuszczalne nominały przez $d_1$, $d_2$, \dots, $d_k$, wśród których jednym z~nich jest 1~gr.
W~tablicy $c[0\twodots n]$ na pozycji $c[j]$ będziemy przechowywać minimalną liczbę monet konieczną do wydania reszty o~wartości $j$.
Jako przypadek bazowy przyjmujemy $c[0]=0$.
Na podstawie optymalnej podstruktury mamy, że jeśli w~optymalnym rozwiązaniu dla $j>0$ użyta została moneta o~nominale $d_i$, to $c[j]=1+c[j-d_i]$.
Zachodzi więc zależność rekurencyjna:
\[
	c[j] = \max\begin{cases}
		0, & \text{jeśli $j=0$,} \\
		\displaystyle\min_{1\le i\le k}(1+c[j-d_i]), & \text{jeśli $j>1$.}
	\end{cases}
\]
Wartości tablicy $c$ można wyliczać dla rosnących indeksów.
Po zakończeniu wypełniania tablicy $c[n]$ będzie równe minimalnej liczbie monet w~optymalnym rozwiązaniu problemu.

Poniższa procedura realizuje opisany plan, zapisując dodatkowo w~tablicy $\id{denom}[1\twodots n]$ na pozycji $\id{denom}[j]$ nominał $d_i$ minimalizujący $c[j]$.
\begin{codebox}
\Procname{$\proc{Make-Change}(n,d)$}
\li	$c[0]\gets0$
\li	\For $j\gets1$ \To $n$
\li		\Do $c[j]\gets\infty$
\li			\For $i\gets1$ \To $k$
\li				\Do \If $j\ge d_i$ i~$1+c[j-d_i]<c[j]$
\li						\Then $c[j]\gets1+c[j-d_i]$
\li							$\id{denom}[j]\gets d_i$
						\End
				\End
		\End
\li	\Return $c$ i~\id{denom}
\end{codebox}
Tablica \id{denom} pozwala wypisać monety używane w~optymalnym rozwiązaniu.
\begin{codebox}
\Procname{$\proc{Print-Change}(\id{denom},n)$}
\li	\While $n>0$
\li		\Then wypisz $\id{denom}[n]$
\li			$n\gets n-\id{denom}[n]$
		\End
\end{codebox}

\problem{Szeregowanie o~minimalnym średnim czasie wykonania zadań} %16-2

\subproblem %16-2(a)
Rozważmy dowolne uszeregowanie zadań stanowiące rozwiązanie przedstawionego problemu.
Załóżmy, że dla pewnych zadań $a_i$, $a_j$ w~tym uszeregowaniu, tuż po zakończeniu wykonywania zadania $a_i$ w~chwili $c_i$ komputer przerywa pracę na jakiś czas -- czyli kolejne zadanie $a_j$ rozpoczęte zostaje w~chwili $b_j$ późniejszej niż $c_i$.
Jednakże przez wyeliminowanie tej przerwy moglibyśmy przesunąć rozpoczęcie wykonywania wszystkich zadań występujących po $a_i$ o~$c_i-b_j$ jednostek czasu wcześniej.
W~rezultacie wykonywanie tych zadań zakończyłoby się wcześniej o~tyle samo jednostek czasu i~otrzymane w~ten sposób uszeregowanie miałoby niższy średni czas zakończenia.
Wnioskujemy stąd, że w~optymalnym uporządkowaniu zadania wykonywane są bezpośrednio po sobie.

Przyjmijmy teraz, że w~uszeregowaniu nie ma przerw między zadaniami i~że istnieją zadania $a_i$, $a_j$, dla których $p_i>p_j$ oraz $a_i$ wykonywane jest bezpośrednio przed $a_j$.
Suma ich czasów zakończenia wnoszona do średniego czasu zakończenia rozważanego uszeregowania wynosi $C=c_i+c_j$.
Oznaczmy przez $c_i'$, $c_j'$, odpowiednio, czasy zakończenia zadań $a_i$, $a_j$, gdyby zostały one zamienione miejscami w~tym uporządkowaniu.
Z~założenia $c_i'=c_j$ oraz $c_j'<c_i$, a~więc suma nowych czasów zakończenia wynosi $C'=c_i'+c_j'<c_j+c_i=C$, podczas gdy czasy zakończenia pozostałych zadań nie ulegają zmianie.
Wynika stąd, że w~optymalnym uszeregowaniu zadania powinny występować według ich czasów wykonywania, od najkrótszego do najdłuższego.

Na podstawie powyższej analizy wynika, że w~celu rozwiązania postawionego problemu wystarczy posortować zadania ze zbioru $S$ według ich czasów wykonywania, co daje algorytm działający w~czasie $O(n\lg n)$.

\subproblem %16-2(b)
Ponieważ wykonanie każdego zadania można dowolnie przerywać, to o~zadaniach można myśleć jak o~obiektach mających wagę równą liczbie jednostek czasu potrzebnych do ich ukończenia.
Wykonywanie zadania o~wadze $p_i$ przez $t$ jednostek czasu ($0<t\le p_i$) zmniejszałoby jego wagę o~$t$, symbolizując skrócenie pozostałego czasu do jego ukończenia.
Podobnie jak w~punkcie (a), dążymy do minimalizacji średniego czasu zakończenia, więc opłaca się priorytetyzować zadania gotowe do wykonania o~jak najniższych wagach.
Stwierdzenie to uzasadniamy jak w~rozwiązaniu powyżej -- zamiana zadań w~optymalnym rozwiązaniu tak, aby wpierw wykonane było to o~dłuższym czasie wykonywania, może tylko powiększyć sumaryczny czas zakończenia.

Do pamiętania zadań gotowych do wykonania wykorzystamy kolejkę priorytetową $Q$ typu min porządkującą zadania względem ich wag.
Załóżmy, że ciąg $\langle a_1,a_2,\dots,a_n\rangle$ stanowi uszeregowanie zadań według ich czasów pojawienia się.
Algorytm będzie kolejno przeglądał zadania z~tego ciągu i~aktualizował kolejkę $Q$.
Dla zadania $a_i$ o~czasie pojawienia się $r_i$, po dodaniu go do kolejki $Q$, czas między chwilą $r_i$ a~$r_{i+1}-1$ włącznie (dla wygody można przyjąć $r_{n+1}=\infty$) zostanie maksymalnie wykorzystany na wykonanie zadań aktualnie znajdujących się w~kolejce.
Częściowe wykonanie danego zadania polega na zmniejszeniu jego wagi, co można zrealizować operacją \proc{Decrease-Key}.
Na każde zadanie pojawiające się do wykonania, co najwyżej jedno zadanie wykona się częściowo.
Z~kolei wykonawszy zadanie w~całości, zwyczajnie usuniemy go z~kolejki.
W~trakcie działania algorytmu każde zadanie dokładnie raz zostanie wstawione i~dokładnie raz usunięte z~kolejki $Q$, więc w~kolejce znajdzie się maksymalnie $n$ elementów.
W~najgorszym przypadku algorytm wykona po $n$ operacji \proc{Insert}, \proc{Extract-Min} i~\proc{Decrease-Key}, zatem jego czasem działania algorytmu jest $O(n\lg n)$.

\problem{Podgrafy acykliczne} %16-3

\subproblem %16-3(a)
Twierdzenie to pokrywa się z~tw.\ 16.5 udowodnionym w~Podręczniku.

\subproblem %16-3(b)
\note{Kluczowe jest doprecyzowanie, że\/ $M$ jest macierzą określoną nad ciałem\/ $\mathbb{Z}_2$ liczb całkowitych modulo\/ $2$.}

\noindent Niech $A$ będzie zbiorem liniowo niezależnych kolumn macierzy $M$.
Załóżmy, że odpowiadający mu zbiór krawędzi $E_A\subseteq E$ zawiera cykl $\langle v_0,v_1,v_2,\dots,v_{k-1},v_0\rangle$ o~długości $k>1$, tzn.\ $\{v_i,v_{(i+1)\bmod k}\}\in E_A$ dla $i=0$, 1, \dots, $k-1$.
A~zatem dla każdego $i=0$, 1, \dots, $k-1$, $M_{v_i,e}=1$ wtedy i~tylko wtedy, gdy $e=\bigl\{v_{(i-1)\bmod k},v_i\bigr\}$ lub $e=\bigl\{v_i,v_{(i+1)\bmod k}\bigr\}$.
Stąd suma kolumn ze zbioru $A$ stanowi wektor składający się z~dwójek na każdej pozycji, czyli wektor zerowy w~$\mathbb{Z}_2$.
Kolumny te są więc liniowo zależne, co stoi w~sprzeczności z~założeniem.

Dla dowodu w~drugą stronę przyjmijmy, że $E_A\subseteq E$ jest zbiorem $k>1$ krawędzi, które nie tworzą cyklu i~niech $A=\{a_1,a_2,\dots,a_k\}$ będzie podzbiorem kolumn macierzy $M$ odpowiadających zbiorowi $E_A$.
Gdyby zbiór $A$ zawierał kolumny liniowo zależne, to istniałyby takie współczynniki $c_1$, $c_2$, \dots, $c_k\in\mathbb{Z}_2$, nie wszystkie równe 0, że $\sum_{i=1}^kc_ia_i$ byłoby wektorem zerowym w~$\mathbb{Z}_2$.
Ponieważ działania wykonywane są w~arytmetyce modulo 2, to równoważnie istniałby niepusty podzbiór kolumn $A'\subseteq A$, których suma byłaby wektorem złożonym z~samych liczb parzystych.
Oznacza to, że w~grafie $G_{A'}=\langle V,E_{A'}\rangle$, gdzie $E_{A'}\subseteq E$ jest zbiorem krawędzi odpowiadających kolumnom z~$A'$, każdy wierzchołek byłby parzystego stopnia.
Każda składowa grafu $G_{A'}$ byłaby więc tzw.\ \textbf{grafem eulerowskim}, czyli miała cykl Eulera, o~czym mówi problem \refProblem{22-3} zaadaptowany do grafów nieskierowanych.
Otrzymana sprzeczność dowodzi, że zbiór $A$ nie może składać się z~kolumn liniowo zależnych.

Niech $S$ będzie zbiorem kolumn macierzy $M$, a~$\mathcal{I}'$ rodziną podzbiorów zbioru $S$ taką, że $A\in\mathcal{I}'$ wtedy i~tylko wtedy, gdy kolumny należące do $A$ są liniowo niezależne.
Każdej kolumnie macierzy $M$ odpowiada inna krawędź grafu $G=\langle V,E\rangle$ i~podobnie w~drugą stronę.
Dla pewnego $A\subseteq S$, niech $E_A$ oznacza zbiór krawędzi grafu $G$ odpowiadających kolumnom z~$A$.
Na mocy dowodu z~poprzednich paragrafów istnieje jednoznaczne odwzorowanie rodziny $\mathcal{I}'$ na rodzinę $\mathcal{I}=\{\,E_A\subseteq E:A\in\mathcal{I}'\,\}$.
W~\refExercise{16.4-2} pokazaliśmy, że $\langle S,\mathcal{I}'\rangle$ jest matroidem, a~zatem $\langle E,\mathcal{I}\rangle$ także jest matroidem.

\subproblem %16-3(c)
Dla grafu $G=\langle V,E\rangle$ rozważmy matroid ważony $M=\langle E,\mathcal{I}\rangle$ zdefiniowany jak w~punkcie (a) wraz z~funkcją wagi $w$.
Do znalezienia acyklicznego podzbioru krawędzi grafu $G$ o~największej wadze można użyć algorytmu opartego na \proc{Greedy}, w~którym używana jest efektywna metoda weryfikowania, czy dany zbiór krawędzi $A$ nie zawiera cyklu, czyli czy $A\in\mathcal{I}$.
Wymaganie to spełnia algorytm Kruskala z~podrozdziału 23.2, z~tą różnicą, że ma on na celu minimalizację sumarycznej wagi krawędzi.
Aby wartość ta była maksymalizowana, wystarczy zastosować w~tym algorytmie funkcję wagową zmodyfikowaną metodą z~\refExercise{16.4-5}.

\subproblem %16-3(d)
Na rys.\ \ref{fig:16-3c} zamieszczono przykładowy graf skierowany $G=\langle V,E\rangle$, dla którego zdefiniowana w~treści problemu para $\langle E,\mathcal{I}\rangle$ nie jest matroidem.
Rozważmy zbiory $A=\{e_1,e_4\}$, $B=\{e_1,e_2,e_3\}$.
Oczywiście $A$, $B\in\mathcal{I}$, ale zarówno w~$A\cup\{e_2\}$, jak i~w~$A\cup\{e_3\}$ znajduje się cykl, czyli zbiory te nie należą do $\mathcal{I}$.
Para $\langle E,\mathcal{I}\rangle$ nie spełnia więc własności wymiany, przez co nie stanowi matroidu.
\begin{figure}[!ht]
	\centering \begin{tikzpicture}

\node[tree node] (a) {};
\node[tree node, above right=10mm and 7mm of a] (b) {};
\node[tree node, right=16mm of a] (c) {};
\path (a) edge[arrow] node[index node, auto] {$e_1$} (b) edge[arrow, bend left=25] node[index node, auto] {$e_3$} (c);
\path (b) edge[arrow] node[index node, auto] {$e_2$} (c);
\path (c) edge[arrow, bend left=25] node[index node, auto] {$e_4$} (a);

\end{tikzpicture}

	\caption{Przykładowy graf skierowany, dla którego struktura $\langle E,\mathcal{I}\rangle$ nie jest matroidem.} \label{fig:16-3c}
\end{figure}

\subproblem %16-3(e)
Załóżmy, że graf $G$ zawiera cykl $p$.
Rozważmy podgraf $G'=\langle V',E'\rangle$, w~którym $V'$ składa się z~wierzchołków wchodzących w~skład cyklu $p$, a~$E'$ -- z~krawędzi tworzących cykl $p$.
Oczywiście $G'$ jest grafem eulerowskim, czyli zawiera cykl Eulera.
Na podstawie punktu (a) problemu \refProblem{22-3} stopień wejściowy każdego wierzchołka $v\in V'$ jest równy jego stopniowi wyjściowemu.
Oznacza to, że dla każdego wiersza macierzy $M$ odpowiadającego wierzchołkowi z~$V'$ wartości w~kolumnach odpowiadającym krawędziom z~$E'$ sumują się do zera.
Zatem dla każdego grafu skierowanego $G$ zawierającego cykl istnieje niepusta podmacierz macierzy $M$ o~wszystkich wierszach sumujących się do zera.
Płynie z~tego wniosek, że kolumny macierzy $M$ są liniowo zależne lub, równoważnie, jeżeli kolumny macierzy $M$ są liniowo niezależne, to odpowiadający im graf skierowany nie zawiera cyklu.

\subproblem %16-3(f)
W~punkcie (e) udowodniliśmy wynikanie tylko w~jedną stronę.
Implikacja przeciwna jest fałszywa; istnieją bowiem grafy skierowane niezawierające cyklu, których macierz incydencji zawiera kolumny liniowo zależne.
Jednym z~nich jest graf z~rys.\ \ref{fig:16-3c} pozbawiony krawędzi $e_4$.
Struktura zdefiniowana w~punkcie (d) dla takiego grafu jest matroidem.
Z~kolei w~macierzy incydencji tego grafu
\[
	\bordermatrix{~ & e_1 & e_2 & e_3 \cr
		v_1 & -1 & 0 & -1 \cr
		v_2 & 1 & -1 & 0 \cr
		v_3 & 0 & 1 & 1}
\]
ostatnia kolumna jest sumą pozostałych, przez co zbiór kolumn tej macierzy nie tworzy matroidu.
Powodem tego jest brak własności, którą pokazaliśmy w~punkcie (b) dla grafów nieskierowanych, czyli brak dokładnej odpowiedniości między liniowo niezależnym zbiorem krawędzi macierzy incydencji grafu skierowanego a~acyklicznym zbiorem krawędzi w~tym grafie.

\problem{Inne metody szeregowania} %16-4

\subproblem %16-4(a)
Suma kar za niedotrzymanie terminu w~tym problemie zależy wyłącznie od zbioru $L$ zadań spóźnionych.
Wartość tę można próbować zmniejszyć poprzez usunięcie z~$L$ pewnego zadania, czyniąc go terminowym.
Alternatywnie można próbować wymienić pewne zadanie z~$L$ z~innym zadaniem terminowym.
Dzięki zachłannej naturze problemu, każde zmniejszenie kosztu rozwiązania jedną z~opisanych modyfikacji, zbliża nas do optymalnego zbioru $L_0$, dla którego suma kar za spóźnione zadania jest minimalna.

Rozważmy rozwiązanie zwrócone przez opisany algorytm.
Niech $a_j$ będzie dowolnym zadaniem w~zbiorze $L$ tego rozwiązania zajmującym przedział $k>d_j$.
Wynika stąd, że w~trakcie działania algorytmu $d_j$ zadań o~wyższych karach zdążyło już zająć przedziały 1, 2, \dots, $d_j$ i~wszystkie z~nich są zadaniami terminowymi.
Zamiana miejscami zadania $a_j$ z~innym takim zadaniem terminowym $a_i$ może tylko powiększyć koszt rozwiązania, bowiem $k>d_i$.
Gdyby zachodziło $k\le d_i$, to zadanie $a_i$, które jest przetwarzane przed $a_j$, byłoby umieszczone w~najpóźniejszym pustym przedziale przed terminem $d_i$, a~skoro przedział $k$ był pusty, zanim przetwarzane było zadanie $a_j$, to przedział ten zostałby wybrany dla $a_i$.
A~zatem nie sposób zmniejszyć kosztu tego rozwiązania, skąd wynika, że jest ono optymalne.

\subproblem %16-4(b)
\note{Rozwiązanie zostanie podane w~wersji 0.7.}


%\chapter{Analiza kosztu zamortyzowanego}

\makeatletter
\def\input@path{{chapter17/}}
\makeatother

\subchapter{Metoda kosztu sumarycznego}

\exercise %17.1-1
Nawet jeśli ograniczymy ilość wkładanych elementów na stos w~operacji \proc{Multipush} przez $n$, to nie jesteśmy w~stanie osiągnąć zamortyzowanego kosztu $O(1)$.
Ciąg $n$ operacji mógłby się bowiem składać z~występujących na przemian \proc{Multipush} umieszczającej na stosie $n$ elementów i~\proc{Multipop}, która wszystkie te elementy usuwa ze stosu.
Koszt takiego ciągu operacji wynosi $O(n^2)$ i~oszacowaniem na zamortyzowany koszt operacji jest $O(n^2)/n=O(n)$.

\exercise %17.1-2
\exercise %17.1-3

\subchapter{Metoda księgowania}

\exercise %17.2-1
Operację wykonywania kopii bezpieczeństwa będziemy nazywać \proc{Backup}.
W~naszym rozwiązaniu metodą księgowania operacjom przypiszemy następujące koszty zamortyzowane:
\begin{flushleft}
	\begin{tabular}{lr}
		\proc{Push} & 2, \\
		\proc{Pop} & 1, \\
		\proc{Backup} & 0.
	\end{tabular}
\end{flushleft}
Wykażemy, że przy takich wartościach jesteśmy w~stanie spełnić nierówność (17.1) dla dowolnego ciągu $n$ operacji.
Możemy wyobrazić sobie, że za każde wstawienie elementu na stos płacimy 2~zł, czyli 1~zł za rzeczywisty koszt samej operacji plus 1~zł kredytu na poczet przyszłych operacji.
Po wywołaniu dokładnie $k$ operacji \proc{Push} i~\proc{Pop} łączna wartość kredytu na stosie wynosi dokładnie $k$~zł, co wystarcza na zapłacenie za wykonanie kopii bezpieczeństwa, gdy operacja \proc{Backup} jest wywoływana na stosie o~co najwyżej $k$ elementach.

Zamortyzowanym kosztem każdej operacji jest więc $O(1)$, co oznacza, że łączny koszt ciągu $n$ operacji można oszacować przez $O(n)$.

\exercise %17.2-2
Każdej operacji przypiszemy koszt zamortyzowany $\widehat{c_i}=3$.
Z~\refExercise{17.1-3} wiemy, że $\sum_{i=1}^nc_i<3n$, dlatego kredyt w~żadnym momencie nie jest ujemny, bo
\[
	\sum_{i=1}^n\widehat{c_i} = 3n > \sum_{i=1}^nc_i.
\]
Koszt zamortyzowany każdej operacji jest więc rzędu $O(1)$.

\exercise %17.2-3
W~naszej implementacji licznika z~wektorem bitów $A$ powiążemy atrybut \id{highest}, w~którym będziemy pamiętać najwyższy rząd bitu ustawionego na 1 aktualnie w~liczniku.
Dokładniej, $\attrib{A}{highest}=k$, gdzie $k\ge0$, jeśli wartość licznika znajduje się w~przedziale $[2^k,2^{k+1})$, oraz $\attrib{A}{highest}=\const{nil}$, jeśli wartością licznika jest 0.
Procedura \proc{Increment} powinna dodatkowo tuż przed zakończeniem uaktualnić wartość pola \attrib{A}{highest}.
Pseudokod procedury zerującej licznik przedstawiono poniżej.
\begin{codebox}
\Procname{$\proc{Reset}(A)$}
\li	\For $i\gets0$ \To \attrib{A}{highest}
\li		\Do $A[i]\gets0$
		\End
\li	$\attrib{A}{highest}\gets\const{nil}$
\end{codebox}

Podobnie jak w~analizie kosztu zamortyzowanego przeprowadzonej w~Podręczniku, za każde ustawienie bitu na 1 w~procedurze \proc{Increment} zapłacimy 1~zł, po czym ,,położymy'' na tym bicie 1~zł jako kredyt.
Dodatkowo, jeśli zwiększy się wartość pola \attrib{A}{highest}, to kolejne 1~zł ,,położymy'' na nowej jedynce najwyższego rzędu.
Ten dodatkowy kredyt wykorzystany będzie do spłacenia kosztu zerowania licznika procedurą \proc{Reset}.

Możemy zatem powiązać koszt zamortyzowany równy 3 z~każdym wywołaniem \proc{Increment} oraz zerowy koszt zamortyzowany z~każdym wywołaniem \proc{Reset}.
Wnioskujemy stąd, że dowolny ciąg $n$ operacji zwiększenia i~zerowania licznika można wykonać w~czasie $O(n)$.

\subchapter{Metoda potencjału}

\exercise %17.3-1
By zachować takie same koszty zamortyzowane operacji, musimy zapewnić, że przyrosty potencjału względem funkcji $\Phi'$ są identyczne jak przyrosty potencjału względem $\Phi$.
A~zatem dla każdego $i=1$, 2, \dots, $n$ musi zachodzić
\[
	\Phi'(D_i)-\Phi'(D_{i-1}) = \Phi(D_i)-\Phi(D_{i-1}).
\]
Mamy $\Phi'(D_1)-\Phi'(D_0)=\Phi(D_1)-\Phi(D_0)$, skąd $\Phi'(D_1)=\Phi(D_1)-\Phi(D_0)$.
Następnie mamy $\Phi'(D_2)=\Phi(D_2)-\Phi(D_1)+\Phi'(D_1)$ i~wstawiając uprzednio wyznaczone $\Phi'(D_1)$, dostajemy $\Phi'(D_2)=\Phi(D_2)-\Phi(D_0)$.
Postępując w~ten sposób dla kolejnych $i$, skonstruujemy funkcję $\Phi'(D_i)=\Phi(D_i)-\Phi(D_0)$ dla każdego $i=1$, 2, \dots, $n$.

\exercise %17.3-2
W~naszej analizie skorzystamy z~następującej funkcji potencjału:
\[
	\Phi(D_i) = \begin{cases}
		0, & \text{jeśli $i=0$}, \\
		k+2, & \text{jeśli $i=2^k$, $k=1$, 2, \dots, $\lfloor\lg n\rfloor$}, \\
		\Phi(D_{i-1})+2, & \text{w~przeciwnym przypadku}.
	\end{cases}
\]
Gdy $i=1$ lub gdy $i$ nie jest potęgą 2, to $\widehat{c_i}=c_i+\Phi(D_i)-\Phi(D_{i-1})=1+2=3$.
Przyjmijmy teraz, że $i=2^k$ dla pewnego $k=1$, 2, \dots, $\lfloor\lg n\rfloor$.
Mamy
\begin{align*}
	\widehat{c_{2^k}} &= c_{2^k}+\Phi(D_{2^k})-\Phi(D_{2^k-1}) \\
	&= 2^k+k+2-\bigl(\Phi(D_{2^{k-1}})+2(2^k-1-2^{k-1})\bigr) \\
	&= 2^k+k+2-(k+1+2^k-2) \\
	&= 3.
\end{align*}
Dzięki wybraniu szczególnej funkcji $\Phi$, jako koszt zamortyzowany każdej operacji mogliśmy przyjąć 3, podobnie jak w~rozwiązaniu \refExercise{17.2-2}.
Oczywiście $\Phi(D_i)\ge\Phi(D_0)$ dla każdego $i\ge1$, zatem łączny koszt zamortyzowany $n$ operacji stanowi górne ograniczenie ich faktycznego kosztu.
Stąd pesymistyczny koszt ciągu $n$ operacji wynosi $O(n)$.

\exercise %17.3-3
\exercise %17.3-4
\exercise %17.3-5
\exercise %17.3-6
\exercise %17.3-7

\subchapter{Tablice dynamiczne}

\exercise %17.4-1
\exercise %17.4-2
Gdy $\alpha_{i-1}\ge1/2$, to jedynymi sytuacjami, kiedy współczynnik zapełnienia zostaje obniżony poniżej $1/4$, jest wykonanie operacji \proc{Table-Delete} na tablicy dynamicznej, w~której $\id{size}_{i-1}\le2$ oraz $\id{num}_{i-1}=1$.
Oczywiście zamortyzowany koszt operacji usuwania w~takich przypadkach jest ograniczony przez stałą.

Przy założeniu, że $\id{size}_{i-1}>2$, faktyczny koszt operacji wynosi $c_i=1$.
Jeśli $\alpha_i\ge1/2$, to
\begin{align*}
	\widehat{c_i} &= c_i+\Phi_i-\Phi_{i-1} \\
	&= 1+(2\cdot\id{num}_i-\id{size}_i)-(2\cdot\id{num}_{i-1}-\id{size}_{i-1}) \\
	&= 1+2\cdot\id{num}_i-\id{size}_i-2\cdot\id{num}_i-2+\id{size}_i \\
	&= -1.
\end{align*}
Jeśli natomiast $\alpha_i<1/2$, to
\begin{align*}
	\widehat{c_i} &= c_i+\Phi_i-\Phi_{i-1} \\
	&= 1+(\id{size}_i/2-\id{num}_i)-(2\cdot\id{num}_{i-1}-\id{size}_{i-1}) \\
	&= 1+\id{size}_i/2-\id{num}_i-2\cdot\id{num}_i-2+\id{size}_i \\
	&= (3/2)\id{size}_i-3\cdot\id{num}_i-1 \\[1mm]
	&= (3/2)\id{size}_i-3\alpha_i\id{size}_i-1 \\[1mm]
	&< (3/2)\id{size}_i-(3/2)\id{size}_i-1 \\
	&= -1.
\end{align*}
A~zatem we wszystkich przypadkach $\widehat{c_i}$ jest ograniczone od góry przez stałą.

\exercise %17.4-3
W~naszej analizie skorzystamy z~oznaczeń identycznych do tych z~Podręcznika.
Potrzebna nam będzie następująca obserwacja:

\medskip
\noindent\textsf{\textbf{Lemat.}} \textit{Dla dowolnych nieujemnych liczb rzeczywistych\/ $x$, $y$ zachodzi
\[
	|x-y|\le x+y.
\]}
\begin{proof}
Jeśli $x\ge y$, to $|x-y|=x-y\le x+y$, a~jeśli $x<y$, to $|x-y|=-x+y\le x+y$.
\end{proof}

Jeśli wywołanie operacji \proc{Delete} nie powoduje zmniejszenia rozmiaru tablicy, to faktyczny koszt operacji wynosi $c_i=1$.
Wówczas $\id{num}_i=\id{num}_{i-1}-1$, $\id{size}_i=\id{size}_{i-1}$ i~na podstawie powyższego lematu mamy
\begin{align*}
	\widehat{c_i} &= c_i+\Phi_i-\Phi_{i-1} \\
	&= 1+|2\cdot\id{num}_i-\id{size}_i|-|2\cdot\id{num}_{i-1}-\id{size}_{i-1}| \\
	&= 1+|2\cdot\id{num}_i-\id{size}_i|-|2\cdot\id{num}_i-2-\id{size}_i| \\
	&\le 1+(2\cdot\id{num}_i+\id{size}_i)-(2\cdot\id{num}_i+2+\id{size}_i) \\
	&= -1.
\end{align*}
W~sytuacji, w~której tablica zostaje zmniejszona faktyczny koszt wynosi $c_i=\id{num}_i+1$, a~także zachodzą równości $\id{size}_i/2=\id{size}_{i-1}/3=\id{num}_{i-1}=\id{num}_i+1$.
Koszt zamortyzowany w~tym przypadku wynosi więc
\begin{align*}
	\widehat{c_i} &= c_i+\Phi_i-\Phi_{i-1} \\
	&= (\id{num}_i+1)+|2\cdot\id{num}_i-\id{size}_i|-|2\cdot\id{num}_{i-1}-\id{size}_{i-1}| \\
	&= (\id{num}_i+1)+|2\cdot\id{num}_i-2\cdot\id{num}_i-2|-|2\cdot\id{num}_i+2-3\cdot\id{num}_i-3| \\
	&= (\id{num}_i+1)+|{-}2|-|{-}\id{num}_i-1| \\
	&= 2.
\end{align*}


\problems

\problem{Licznik binarny z~odwróconymi bitami} %17-1

\subproblem %17-1(a)
Poniższa procedura oblicza kolejne wartości funkcji $\mathrm{rev}_k$ i~zamienia miejscami elementy wejściowej tablicy, o~ile nie zostały one zamienione już wcześniej.
\begin{codebox}
\Procname{$\proc{Bit-Reversal-Permutation}(A)$}
\li	$n\gets\attrib{A}{length}$
\li	$k\gets\lg n$
\li	\For $i\gets1$ \To $n-2$
\li		\Do $j\gets\mathrm{rev}_k(i)$
\li			\If $i<j$
\li				\Then zamień $A[i]\leftrightarrow A[j]$
				\End
		\End
\end{codebox}
Ponieważ $\mathrm{rev}_k(0)=0$ oraz $\mathrm{rev}_k(2^k-1)=2^k-1$, to elementy $A[0]$ oraz $A[n-1]$ pozostaną na swoich miejscach, dlatego pętla \kw{for} iteruje od $i=1$ do $i=n-2$.
Nietrudno zauważyć, że czas działania tej procedury ograniczony jest przez $O(nk)$.

\subproblem %17-1(b)
Kolejna wartość licznika powstaje w~wyniku zinkrementowania bieżącej wartości, ale na odwróconych bitach, czyli przy potraktowaniu skrajnie lewego bitu jako najmniej znaczącego, a~skrajnie prawego -- jako najbardziej znaczącego.
Opiszemy implementację procedury \proc{Increment} z~podrozdziału 17.1 przy pomocy operacji bitowych, a~następnie dostosujemy ją do działania na liczniku z~odwróconymi bitami.

W~pętli \kw{while} procedury \proc{Increment} wszystkie mniej znaczące jedynki od najmniej znaczącego zera są zamieniane na zera.
Można zasymulować te czynności przez utrzymywanie tzw.\ \textbf{maski bitowej} służącej do szybkiego odczytu danego bitu licznika i~jego modyfikacji.
W~$i$\nbhyphen tej iteracji, gdzie $i=0$, 1, \dots, $k-1$, maska będzie mieć wartość $2^i$ otrzymaną przez wykonywanie operacji przesunięcia bitowego w~lewo na początkowej wartości $2^0=1$.
Aby odczytać $i$\nbhyphen ty najmniej znaczący bit licznika, należy wykonać operację bitową AND na liczniku i~aktualnej masce.
Podobnie, aby wyzerować ten bit, należy wykonać bitowe XOR\@.
Zamiana znalezionego zera na jedynkę realizowana jest za pomocą bitowego OR\@.
Przystosowanie opisanych kroków do działania na odwrotnej kolejności bitów wymaga zastosowania maski zainicjalizowanej na $2^{k-1}$ i~przesuwania jej bitowo w~prawo.

W~pseudokodzie poniżej \func{SHL} oznacza operację przesunięcia bitowego w~lewo, a~\func{SHR} -- operację przesunięcia bitowego w~prawo.
\begin{codebox}
\Procname{$\proc{Bit-Reversed-Increment}(a,k)$}
\li	$m\gets1\func{SHL}{}(k-1)$
\li	\While $a\func{AND}m\ne0$
\li		\Do $a\gets a\func{XOR}m$
\li			$m\gets m\func{SHR}1$
		\End
\li	\Return $a\func{OR}m$
\end{codebox}
W~celu wyznaczenia permutacji odwracającej bity utrzymujemy dwa $k$\nbhyphen bitowe liczniki binarne -- zwykły oraz z~odwróconymi bitami -- i~zamieniamy elementy w~tablicy $A$ znajdujące się na indeksach równych tym licznikom, upewniając się, że zamiany te nie powtarzają się.
\begin{codebox}
\Procname{$\proc{Bit-Reversal-Permutation}'(A)$}
\li	$n\gets\attrib{A}{length}$
\li	$k\gets\lg n$
\li	$j\gets0$
\li	\For $i\gets1$ \To $n-2$
\li		\Do $j\gets\proc{Bit-Reversed-Increment}(j,k)$
\li			\If $i<j$
\li				\Then zamień $A[i]\leftrightarrow A[j]$
				\End
		\End
\end{codebox}

Procedura \proc{Bit-Reversed-Increment} działa w~tym samym czasie co \proc{Increment}, na której jest oparta, zatem $n$ kolejnych jej wywołań wymaga czasu $O(n)$.
Jest to też czas, jaki potrzebuje procedura $\proc{Bit-Reversal-Permutation}'$ wywołana na tablicy $A[0\twodots n-1]$.

\subproblem %17-1(c)
Przy takim założeniu inicjalizacja maski w~procedurze \proc{Bit-Reversed-Increment} wymaga czasu $\Theta(k)$.
Sprawia to, że $n$ kolejnych wywołań tej procedury zajmuje czas $O(nk)$.
Bez przygotowania maski nie jesteśmy jednak w~stanie przeglądać bitów w~kolejności od lewej do prawej i~wyznaczenie permutacji odwracającej bity w~czasie $O(n)$ jest w~tym przypadku niemożliwe.



%\setcounter{part}{4}
%\part{Złożone struktury danych}

%\setcounter{chapter}{17}
%\chapter{B-drzewa}

\makeatletter
\def\input@path{{chapter18/}}
\makeatother

\subchapter{Definicja B-drzewa}

\exercise %18.1-1
\exercise %18.1-2
\exercise %18.1-3
\exercise %18.1-4
\exercise %18.1-5

\subchapter{Podstawowe operacje na B-drzewach}

\exercise %18.2-1
\exercise %18.2-2
\exercise %18.2-3
\exercise %18.2-4
\exercise %18.2-5
\exercise %18.2-6
\exercise %18.2-7

\input{sc18.3}

\problems

\problem{Stosy w~pamięci zewnętrznej} %18-1

\subproblem %18-1(a)
Każda operacja na stosie w~tej implementacji wymaga stałej liczby dostępów do dysku oraz $\Theta(m)$ czasu procesora.
Zatem ciąg $n$ operacji na stosie wymaga $\Theta(n)$ dostępów do dysku i~$\Theta(mn)$ czasu.

\subproblem %18-1(b)
W~ciągu $n$ operacji \proc{Push}, mniej więcej co $m$\nbhyphen ta operacja wiąże się ze zmianą strony.
Wynika stąd, że liczba dostępów do dysku wynosi sumarycznie $\Theta(n/m)$.
Całkowity czas procesora jest sumą czasu spędzonego na samym wstawianiu na stos oraz czasu potrzebnego do wczytywania i~zapisywania strony na dysku, wynosi więc $\Theta(n+m\cdot n/m)=\Theta(n)$.

\subproblem %18-1(c)
Najgorszy przypadek dla ciągu dowolnych operacji ma miejsce, gdy dostęp do dysku jest wymagany przez jak największą liczbę operacji w~tym ciągu.
Sytuacja taka zachodzi, gdy początkowo wskaźnik $p$ znajduje się na ostatnim słowie strony i~przeplatane są ze sobą dwie operacje \proc{Push} i~dwie operacje \proc{Pop}.
Pierwsze wywołanie \proc{Push} przestawia wskaźnik $p$ na kolejną stronę, ale jej wczytanie i~zapis poprzedniej strony są wykonywane podczas drugiego wywołania \proc{Push}.
Następujące po nich dwa wywołania \proc{Pop} powodują wczytanie poprzedniej strony i~przestawienie wskaźnika $p$ na wyjściową pozycję.
W~ciągu $n$ operacji w~najgorszym przypadku wykonywanych jest zatem $\Theta(n)$ operacji dostępu do dysku, a~całkowity czas procesora wynosi $\Theta(mn)$.

\subproblem %18-1(d)
W~pamięci wewnętrznej będziemy zawsze trzymać aktualną stronę, czyli tę, na którą pokazuje wskaźnik $p$ i~taką, która była wykorzystywana tuż przed przeniesieniem $p$ na aktualną stronę.
Wczytanie do pamięci wewnętrznej nowej strony będzie poprzedzone zapisem na dysk starej strony.
Każde wczytanie zastępuje w~pamięci wewnętrznej stronę, która nie była wykorzystywana od co najmniej $m$ operacji.
A~zatem zmiana strony w~pamięci wewnętrznej będzie się odbywać w~najgorszym przypadku co $m$ operacji.
Tym samym analiza efektywności tej implementacji stosu sprowadza się do analizy z~punktu (b) -- w~ciagu $n$ operacji wykonanych będzie $O(n/m)$ dostępów do dysku, a~całkowity czas wyniesie $O(n)$.
Oznacza to, że zamortyzowana liczba dostępów do dysku wynosi $O(1/m)$, a~zamortyzowany czas jest stały.

\problem{Sklejanie i~rozbijanie 2-3-4 drzew} %18-2

\subproblem %18-2(a)
Przez wprowadzenie atrybutu \id{height} atrybut \id{leaf} jest teraz zbędny, dlatego pominiemy go w~implementacji 2-3-4 drzew.
Operacje tworzenia pustego drzewa, wyszukiwania, wstawiania i~usuwania dla 2-3-4 drzew opierają się w~dużej mierze na analogicznych operacjach dla B-drzew, z~kilkoma modyfikacjami.
Po pierwsze w~implementacjach tych operacji rezygnujemy ze stosowania zmiennej $t$, zastępując ją przez minimalny stopień 2-3-4 drzew, czyli 2.
Ponadto każde odwołanie do pola \id{leaf} zamieniamy na odpowiednie odwołanie do pola \id{height}.
Warunek \attrib{x}{leaf} zastępujemy przez $\attrib{x}{height}=0$, a~warunek $\attrib{x}{leaf}=\const{false}$ -- przez $\attrib{x}{height}>0$.
W~wersji procedury \proc{B-Tree-Create} dla 2-3-4 drzew inicjalizacja pola \id{leaf} na \const{true} jest zastąpiona przez inicjalizację pola \id{height} na 0.
W~procedurze rozbijania węzła w~2-3-4 drzewie kopiowanie pola \id{leaf} z~węzła $y$ do $z$ jest zastąpione przez kopiowanie pola \id{height} między tymi węzłami.
Wreszcie, w~procedurze wstawiania do 2-3-4 drzewa, jeśli w~wyniku rozbicia korzenia $r$ powstanie nowy korzeń, to jego pole \id{height} jest inicjalizowane na $\attrib{r}{height}+1$.

Utrzymywanie i~aktualizowanie atrybutu \id{height} nie wpływa na asymptotyczny czas działania poszczególnych operacji, ponieważ w~B-drzewach -- a~więc w~szczególności w~2-3-4 drzewach -- węzły są zawsze tworzone i~usuwane na głębokości 0, dzięki czemu wysokość żadnego poddrzewa nie ulega wtedy zmianie.

\subproblem %18-2(b)
Ogólna koncepcja operacji sklejania polega na wstawieniu klucza $k$ na odpowiednią wysokość w~wyższym drzewie, a~następnie podpięciu niższego drzewa do węzła, do którego trafiło $k$.
Dzięki temu wynikowe drzewo będzie poprawnym 2-3-4 drzewem -- jego wszystkie liście znajdą się na tej samej głębokości.
Do realizacji tego zadania będzie nam potrzebna zmodyfikowana procedura wstawiania klucza do 2-3-4 drzewa, umiejscawiająca klucz na zadanej wysokości $h>0$ w~tym drzewie.
W~porównaniu do zwykłego wstawiania do 2-3-4 drzewa opisanego w~punkcie (a), w~nowej procedurze wysokość węzła $x$ będzie porównywana nie z~0, ale z~$h$, a~oprócz kluczy węzła $x$ będą przesuwane też odpowiednie wskaźniki do synów $x$.
Procedura dodatkowo zwróci jako wynik zmodyfikowany węzeł $x$.

Podczas sklejania osobno obsłużymy sytuację, gdy jedno z~drzew $T'$ lub $T''$ jest puste.
Wystarczy wówczas wstawić klucz $k$ do drugiego drzewa na wysokość 0, czyli do jednego z~jego liści, wykorzystując w~tym celu zwykłą procedurę wstawiania do 2-3-4 drzewa.

Załóżmy teraz, że oba drzewa są niepuste i~oznaczmy przez $h'$ wysokość drzewa $T'$, a~przez $h''$ -- wysokość drzewa $T''$.
Jeżeli $h'<h''$, to wstawiamy klucz $k$ do drzewa $T''$, do jego węzła $x$ znajdującego się na wysokości $h'+1$, wywołując opisaną powyżej zmodyfikowaną procedurę wstawiania.
Ponieważ klucz $k$ jest mniejszy od każdego klucza w~drzewie $T''$, to w~węźle $x$ zajął on pierwszą pozycję, czyli \attribxx{x}{key_1}.
Wskaźnik \attribxx{x}{c_1} ustawiamy następnie na korzeń drzewa $T'$, efektywnie czyniąc to drzewo jednym z~poddrzew węzła $x$.
Wynikowym drzewem jest tu $T''$.

Przypadek, gdy $h'>h''$, jest symetryczny do poprzedniego.
Klucz $k$ ląduje tym razem w~węźle $x$ drzewa $T'$ na wysokości $h''+1$, a~dokładniej w~\attribxx{x}{key_{\attrib{x}{n}}} (po uprzedniej inkrementacji pola \attrib{x}{n}).
Przyłączenie drzewa $T''$ polega na aktualizacji wskaźnika \attribxx{x}{c_{\attrib{x}{n}+1}} na jego korzeń.
Zwracane jest w~wyniku drzewo $T'$.

W~sytuacji, w~której wysokości obu drzew są równe, pozostaje nam utworzenie i~zwrócenie nowego drzewa $T$ z~korzeniem zawierającym jedynie klucz $k$ oraz $T'$, $T''$ w~roli jego poddrzew.

Cały powyższy opis formalizuje następujący pseudokod.
Wykorzystujemy w~nim wywołania do procedur \proc{2-3-4-Tree-Create}, \proc{2-3-4-Tree-Insert} i~\proc{2-3-4-Tree-Insert-At} -- odpowiednio, tworzenia nowego 2-3-4 drzewa, zwykłego wstawiania klucza do 2-3-4 drzewa oraz wstawiania klucza do 2-3-4 drzewa na zadaną wysokość.
\begin{codebox}
    \Procname{$\proc{2-3-4-Tree-Join}(T',T'',k)$}
    \li \If $\attrib{\attrib{T'}{root}}{n}=0$
    \li     \Then $\proc{2-3-4-Tree-Insert}(T'',k)$
    \li         \Return $T''$
            \End
    \li \If $\attrib{\attrib{T''}{root}}{n}=0$
    \li     \Then $\proc{2-3-4-Tree-Insert}(T',k)$
    \li         \Return $T'$
            \End
    \li $h'\gets\attrib{\attrib{T'}{root}}{height}$
    \li $h''\gets\attrib{\attrib{T''}{root}}{height}$
    \li \If $h'<h''$
    \li     \Then $x\gets\proc{2-3-4-Tree-Insert-At}(T'',k,h'+1)$
    \li         $\attribxx{x}{c_1}\gets\attrib{T'}{root}$
    \li         \Return $T''$
            \End
    \li \If $h'>h''$
    \li     \Then $x\gets\proc{2-3-4-Tree-Insert-At}(T',k,h''+1)$
    \li         $\attribxx{x}{c_{\attrib{x}{n}+1}}\gets\attrib{T''}{root}$
    \li         \Return $T'$
            \End
    \li $\proc{2-3-4-Tree-Create}(T)$
    \li $\proc{2-3-4-Tree-Insert}(T,k)$
    \li $\attrib{\attrib{T}{root}}{height}\gets h'+1$
    \li $\attribxx{\attrib{T}{root}}{c_1}\gets\attrib{T'}{root}$
    \li $\attribxx{\attrib{T}{root}}{c_2}\gets\attrib{T''}{root}$
    \li \Return $T$
\end{codebox}

Nietrudno zauważyć, że algorytm wykonuje co najwyżej $|h'-h''+1|$ wywołań rekurencyjnych w~procedurach \proc{2-3-4-Tree-Insert} lub \proc{2-3-4-Tree-Insert-At}, zaś pozostałe operacje wnoszą koszt stały.
Stąd algorytm działa w~czasie $O(1+|h'-h''|)$.

\subproblem %18-2(c)
Przypatrzmy się działaniu operacji wyszukiwania klucza $k$ w~drzewie $T$, która oczywiście podąża ścieżką $p$.
Pracując na węźle $x$, procedura przegląda wszystkie klucze $k'$ węzła $x$ nieprzekraczające $k$.
W~przypadku nieodnalezienia $k$ w~$x$ procedura wywoła się rekurencyjne dla syna $x$ o~kluczach większych niż każdy dotychczas przejrzany $k'$.
Oznacza to, że procedura przejrzy we wszystkich odwiedzonych węzłach klucze $k_1'$, $k_2'$, \dots, $k_m'$, $k_{m+1}'=k$, w~porządku rosnącym.
Mamy więc $k_1'<k_2'<\dots<k_m'<k$.

Wybierzmy teraz dowolne $i=1$, 2, \dots, $m+1$.
Jeśli $x$ jest węzłem wewnętrznym, w~którym $k_i'=\attribxx{x}{key_j}$ dla pewnego $j$, to wszystkie klucze z~poddrzewa o~korzeniu w~\attribxx{x}{c_j} są mniejsze od $k_i'$, a~dodatkowo, gdy $i>1$, to są one też większe od $k_{i-1}'$.
Jeśli przez $T_{i-1}'$ oznaczymy wspomniane poddrzewo, a~w~przypadku, gdy $k_i'$ jest w~liściu -- drzewo puste -- to otrzymamy tezę.

Dla każdego $i=1$, 2, \dots, $m$, jeśli $T_{i-1}'$ jest poddrzewem węzła $x$, to $T_i'$ jest poddrzewem $x$ albo któregoś z~jego potomków, zatem wysokość $T_{i-1}'$ wynosi co najmniej tyle, ile wysokość $T_i'$.

Możemy rozumować analogicznie, aby pokazać, że ścieżka $p$ rozbija zbiór $S''$ na zbiór drzew $\{T_0''$, $T_1''$, \dots, $T_m''\}$ i~zbiór kluczy $\{k_1''$, $k_2''$, \dots, $k_m''\}$ takie, że $y>k_i''>z$ dla $i=1$, 2, \dots, $m$ oraz każdej pary kluczy $y\in T_{i-1}''$ i~$z\in T_i''$.
Podobnie jak w~rozbiciu zbioru $S'$, indeksy $i$ porządkują drzewa $T_i''$ nierosnąco według wysokości.

\subproblem %18-2(d)
Danymi wejściowymi operacji rozbijania dla 2-3-4 drzew jest 2-3-4 drzewo $T$ oraz klucz $k$ znajdujący się w~$T$.

Niech $S'$, $S''$ będą zbiorami kluczy $T$, odpowiednio, mniejszych od $k$ i~większych od $k$.
Do zbudowania drzew $T'$ i~$T''$ reprezentujących zbiory $S'$ i~$S''$ potrzebne nam będą rozbicia tych zbiorów opisane w~części (c).
Do ich wyznaczenia wykorzystamy procedury pomocnicze \proc{2-3-4-Tree-Left-Partition} i~\proc{2-3-4-Tree-Right-Partition}.
Poniżej podajemy pseudokod pierwszej z~nich służącej do znalezienia rozbicia zbioru $S'$.
\begin{codebox}
    \Procname{$\proc{2-3-4-Tree-Left-Partition}(T,k)$}
    \li $h\gets\attrib{\attrib{T}{root}}{height}$
    \li utwórz tablice $L[0\twodots 3h+2]$ i~$K[1\twodots 3h+2]$
    \li $m\gets0$
    \li $x\gets\attrib{T}{root}$
    \li \While \const{true}
    \li     \Do $i\gets1$
    \li     \While $i\le\attrib{x}{n}$ i~$k\ge\attribxx{x}{key_i}$
    \li         \Do $\proc{2-3-4-Tree-Create}(T^*)$
    \li             \If $\attrib{x}{height}>0$
    \li                 \Then $\attrib{T^*}{root}\gets\attribxx{x}{c_i}$
                        \End
    \li             $L[m]\gets T^*$
    \li             \If $k=\attribxx{x}{key_i}$
    \li                 \Then \Return $\langle L[0\twodots m], K[1\twodots m]\rangle$
                        \End
    \li             $m\gets m+1$
    \li             $K[m]=\attribxx{x}{key_i}$
    \li             $i\gets i+1$
                \End
    \li         $x\gets\attribxx{x}{c_i}$
            \End
\end{codebox}
Główna idea przyświecająca tej procedurze opiera się na opisie z~rozwiązania punktu (c).
Procedura przechodzi po ścieżce $p$ od korzenia $T$ do klucza $k$, który z~założenia znajduje się w~$T$, i~na każdym poziomie drzewa wyznacza elementy zbioru drzew i~elementy zbioru kluczy, wchodzące w~skład rozbicia $S'$.
Zbiory te reprezentowane są w~algorytmie przez tablice $L$ i~$K$.
Zauważmy, że na każdym z~$h+1$ poziomów drzewa $T$ co najwyżej 3 drzewa mogą zostać umieszczone w~tablicy $L$, oraz co najwyżej 3 klucze w~tablicy $K$ (z~wyjątkiem poziomu liści drzewa, który generuje co najwyżej 2 klucze).
Stąd biorą się ograniczenia na rozmiary tych tablic użyte przy ich inicjalizacji.
W~momencie odnalezienia klucza $k$ -- co jest gwarantowane na podstawie założenia -- procedura natychmiast kończy działanie, zwracając fragmenty tablic z~wyznaczonym rozbiciem.

Jak nietrudno się domyślić, procedura \proc{2-3-4-Tree-Right-Partition} jest symetryczna do powyższej.
Jedynymi różnicami w~porównaniu z~\proc{2-3-4-Tree-Left-Partition} jest odwrotna kolejność przeglądania kluczy węzłów na ścieżce $p$ oraz dodawanie do zbioru $L$ poddrzew sąsiadujących z~aktualnym kluczem od prawej strony.

W~głównej operacji rozbijania \proc{2-3-4-Tree-Split} drzewo $T'$ zostanie zbudowane przez wykonywanie sklejania kolejnych par drzew i~kluczy z~rozbicia zbioru $S'$, a~drzewo $T''$ -- symetrycznie przy użyciu rozbicia zbioru $S''$.
Moglibyśmy sklejać kolejne drzewa, zaczynając od tych najwyższych, ale taka strategia pochłania asymptotycznie wyższy czas, niż jesteśmy w~stanie osiągnąć przez odwrócenie kolejności sklejania.
\begin{codebox}
    \Procname{$\proc{2-3-4-Tree-Split}(T,k)$}
    \li $\langle L',K'\rangle\gets\proc{2-3-4-Tree-Left-Partition}(T,k)$ \label{li:2-3-4-tree-split-left-partition}
    \li $\langle L'',K''\rangle\gets\proc{2-3-4-Tree-Right-Partition}(T,k)$ \label{li:2-3-4-tree-split-right-partition}
    \li \For $i\gets\attrib{K'}{length}$ \Downto 1 \label{li:2-3-4-tree-split-left-tree-create-begin}
    \li     \Do $L'[i-1]\gets\proc{2-3-4-Tree-Join}(L'[i-1],L'[i],K'[i])$
            \End \label{li:2-3-4-tree-split-left-tree-create-end}
    \li \For $i\gets\attrib{K''}{length}$ \Downto 1 \label{li:2-3-4-tree-split-right-tree-create-begin}
    \li     \Do $L''[i-1]\gets\proc{2-3-4-Tree-Join}(L''[i],L''[i-1],K'[i])$
            \End \label{li:2-3-4-tree-split-right-tree-create-end}
    \li \Return $\langle L'[0],L''[0]\rangle$
\end{codebox}

Przeanalizujmy czas działania procedury \proc{2-3-4-Tree-Split} dla drzewa $T$ o~wysokości $h$ i~$n$ kluczach.
Oczywiście czas potrzebny na wyznaczenie rozbić zbiorów $S'$ i~$S''$ w~wierszach \ref{li:2-3-4-tree-split-left-partition} i~\ref{li:2-3-4-tree-split-right-partition} wynosi tyle samo, co czas (dwukrotnego) odszukania w~$T$ klucza $k$, czyli $O(h)$.
Niech $m'=\attrib{L'}{length}$ i~$m''=\attrib{L''}{length}$.
Jak zauważyliśmy wcześniej, $m'\le3h+3$ i~$m''\le3h+3$.
Oznaczmy przez $h_i'$ dla $i=0$, 1, \dots, $m'-1$ wysokość drzewa $T_i'$ znajdującego się tuż po linii \ref{li:2-3-4-tree-split-right-partition} w~$L'[i]$ i~podobnie, przez $h_i''$ dla $i=0$, 1, \dots, $m''-1$ wysokość drzewa $T_i''$ znajdującego się wtedy w~$L''[i]$.
Na mocy obserwacji z~punktu (c) mamy, że oba te ciągi wysokości są nierosnące, ponadto $h_m'-h_0'\le h$ i~$h_m''-h_0''\le h$.
Czas działania utworzenia $T'$ w~wierszach \doubledash{\ref{li:2-3-4-tree-split-left-tree-create-begin}}{\ref{li:2-3-4-tree-split-left-tree-create-end}} wynosi zatem
\[
    \sum_{i=1}^mO(1+|h_i'-h_{i-1}'|) = O\biggl(\sum_{i=1}^m(1+(h_i'-h_{i-1}'))\biggr) = O(m+h_m'-h_0') = O(h),
\]
co z~tw.\ 12.1 daje $O(\lg n)$.
Oczywiście analogicznie otrzymuje się identyczne oszacowanie na czas budowy drzewa $T''$ w~liniach \doubledash{\ref{li:2-3-4-tree-split-right-tree-create-begin}}{\ref{li:2-3-4-tree-split-right-tree-create-end}}.
A~zatem czas działania operacji rozbijania 2-3-4 drzewa wynosi $O(\lg n)$.


%\chapter{Kopce dwumianowe}

\makeatletter
\def\input@path{{chapter19/}}
\makeatother

\subchapter{Drzewa i~kopce dwumianowe}

\exercise %19.1-1
\exercise %19.1-2
\exercise %19.1-3

\subchapter{Operacje na kopcach dwumianowych}

\exercise %19.2-1
\exercise %19.2-2
\exercise %19.2-3
\exercise %19.2-4
\exercise %19.2-5
\exercise %19.2-6
\exercise %19.2-7
\exercise %19.2-8
\exercise %19.2-9
\exercise %19.2-10


\problems

\problem{2-3-4 kopce} %19-1

\subproblem %19-1(a)
\subproblem %19-1(b)
\subproblem %19-1(c)
\subproblem %19-1(d)
\subproblem %19-1(e)
\subproblem %19-1(f)

\problem{Algorytm znajdowania minimalnego drzewa rozpinającego z~użyciem kopców dwumianowych} %19-2


%\chapter{Kopce Fibonacciego}

\makeatletter
\def\input@path{{chapter20/}}
\makeatother

\subchapter{Struktura kopców Fibonacciego}

\subchapter{Operacje kopca złączalnego}

\exercise %20.2-1
\exercise %20.2-2
\exercise %20.2-3
\exercise %20.2-4
\exercise %20.2-5

\subchapter{Zmniejszanie wartości klucza i~usuwanie węzła}

\exercise %20.3-1
\exercise %20.3-2

\subchapter{Oszacowanie maksymalnego stopnia}

\exercise %20.4-1
\exercise %20.4-2


\problems

\problem{Alternatywna implementacja operacji usuwania węzła} %20-1

\subproblem %20-1(a)
\subproblem %20-1(b)
\subproblem %20-1(c)
\subproblem %20-1(d)

\problem{Inne operacje na kopcach Fibonacciego} %20-2

\subproblem %20-2(a)
\subproblem %20-2(b)


%\chapter{Struktury danych dla zbiorów rozłącznych}

\makeatletter
\def\input@path{{chapter21/}}
\makeatother

\subchapter{Operacje na zbiorach rozłącznych}

\exercise %21.1-1
\exercise %21.1-2
\exercise %21.1-3

\subchapter{Listowa reprezentacja zbiorów rozłącznych}

\exercise %21.2-1
\exercise %21.2-2
\exercise %21.2-3
\exercise %21.2-4
\exercise %21.2-5

\subchapter{Lasy zbiorów rozłącznych}

\exercise %21.3-1
\exercise %21.3-2
\exercise %21.3-3
\exercise %21.3-4

\subchapter{Analiza metody łączenia według rangi z~kompresją ścieżki}

\exercise %21.4-1
\exercise %21.4-2
\exercise %21.4-3
\exercise %21.4-4
\exercise %21.4-5
\exercise %21.4-6


\problems

\problem{Minimum ,,off-line''} %21-1

\subproblem %21-1(a)
\subproblem %21-1(b)
\subproblem %21-1(c)

\problem{Wyznaczanie głębokości} %21-2

\subproblem %21-2(a)
\subproblem %21-2(b)
\subproblem %21-2(c)
\subproblem %21-2(d)
\subproblem %21-2(e)

\problem{Algorytm Tarjana znajdowania najniższych wspólnych przodków ,,off-line''} %21-3

\subproblem %21-3(a)
\subproblem %21-3(b)
\subproblem %21-3(c)
\subproblem %21-3(d)



%\setcounter{part}{5}
%\part{Algorytmy grafowe}

%\setcounter{chapter}{21}
%\chapter{Podstawowe algorytmy grafowe}

\makeatletter
\def\input@path{{chapter22/}}
\makeatother

\subchapter{Reprezentacja grafów}

\exercise %22.1-1
\exercise %22.1-2
\exercise %22.1-3
\exercise %22.1-4
\exercise %22.1-5
\exercise %22.1-6
\exercise %22.1-7
\exercise %22.1-8

\subchapter{Przeszukiwanie wszerz}

\exercise %22.2-1
\exercise %22.2-2
\exercise %22.2-3
\exercise %22.2-4
\exercise %22.2-5
\exercise %22.2-6
\exercise %22.2-7
\exercise %22.2-8

\subchapter{Przeszukiwanie w~głąb}

\exercise %22.3-1
\exercise %22.3-2
\exercise %22.3-3
\exercise %22.3-4
\exercise %22.3-5
\exercise %22.3-6
\exercise %22.3-7
\exercise %22.3-8
\exercise %22.3-9
\exercise %22.3-10
\exercise %22.3-11
\exercise %22.3-12

\subchapter{Sortowanie topologiczne}

\exercise %22.4-1
\exercise %22.4-2
\exercise %22.4-3
\exercise %22.4-4
\exercise %22.4-5

\subchapter{Silnie spójne składowe}

\exercise %22.5-1
\exercise %22.5-2
\exercise %22.5-3
\exercise %22.5-4
\exercise %22.5-5
\exercise %22.5-6
\exercise %22.5-7


\problems

\problem{Klasyfikowanie krawędzi za pomocą przeszukiwania wszerz} %22-1

\subproblem %22-1(a)
\subproblem %22-1(b)

\problem{Punkty artykulacji, mosty i~dwuspójne składowe} %22-2

\subproblem %22-2(a)
\subproblem %22-2(b)
\subproblem %22-2(c)
\subproblem %22-2(d)
\subproblem %22-2(e)
\subproblem %22-2(f)
\subproblem %22-2(g)
\subproblem %22-2(h)

\problem{Cykl Eulera} %22-3

\subproblem %22-3(a)
\subproblem %22-3(b)

\problem{Osiągalność} %22-4


%\chapter{Minimalne drzewa rozpinające}

\makeatletter
\def\input@path{{chapter23/}}
\makeatother

\subchapter{Rozrastanie się minimalnego drzewa rozpinającego}

\exercise %23.1-1
\exercise %23.1-2
\exercise %23.1-3
\exercise %23.1-4
\exercise %23.1-5
\exercise %23.1-6
\exercise %23.1-7
\exercise %23.1-8
\exercise %23.1-9
\exercise %23.1-10
\exercise %23.1-11

\subchapter{Algorytmy Kruskala i~Prima}

\exercise %23.2-1
\exercise %23.2-2
\exercise %23.2-3
\exercise %23.2-4
\exercise %23.2-5
\exercise %23.2-6
\exercise %23.2-7
\exercise %23.2-8


\problems

\problem{Drugie w~kolejności minimalne drzewo rozpinające} %23-1

\subproblem %23-1(a)
\subproblem %23-1(b)
\subproblem %23-1(c)
\subproblem %23-1(d)

\problem{Minimalne drzewa rozpinające w~grafach rzadkich} %23-2

\subproblem %23-2(a)
\subproblem %23-2(b)
\subproblem %23-2(c)
\subproblem %23-2(d)
\subproblem %23-2(e)
\subproblem %23-2(f)

\problem{Zatorowe drzewo rozpinające} %23-3

\subproblem %23-3(a)
\subproblem %23-3(b)
\subproblem %23-3(c)

\problem{Alternatywne algorytmy obliczania minimalnego drzewa rozpinającego} %23-4

\subproblem %23-4(a)
\subproblem %23-4(b)
\subproblem %23-4(c)


%\chapter{Najkrótsze ścieżki z~jednym źródłem}

\makeatletter
\def\input@path{{chapter24/}}
\makeatother

\subchapter{Algorytm Bellmana-Forda}

\exercise %24.1-1
\exercise %24.1-2
\exercise %24.1-3
\exercise %24.1-4
\exercise %24.1-5
\exercise %24.1-6

\subchapter{Najkrótsze ścieżki z~jednym źródłem w~acyklicznych grafach skierowanych}

\exercise %24.2-1
\exercise %24.2-2
\exercise %24.2-3
\exercise %24.2-4

\subchapter{Algorytm Dijkstry}

\exercise %24.3-1
\exercise %24.3-2
\exercise %24.3-3
\exercise %24.3-4
\exercise %24.3-5
\exercise %24.3-6
\exercise %24.3-7
\exercise %24.3-8

\subchapter{Ograniczenia różnicowe i~najkrótsze ścieżki}

\exercise %24.4-1
\exercise %24.4-2
\exercise %24.4-3
\exercise %24.4-4
\exercise %24.4-5
\exercise %24.4-6
\exercise %24.4-7
\exercise %24.4-8
\exercise %24.4-9
\exercise %24.4-10
\exercise %24.4-11
\exercise %24.4-12

\subchapter{Dowody własności najkrótszych ścieżek}

\exercise %24.5-1
\exercise %24.5-2
\exercise %24.5-3
\exercise %24.5-4
\exercise %24.5-5
\exercise %24.5-6
\exercise %24.5-7
\exercise %24.5-8


\problems

\problem{Poprawka Yena do algorytmu Bellmana-Forda} %24-1

\subproblem %24-1(a)
\subproblem %24-1(b)
\subproblem %24-1(c)

\problem{Zagnieżdżanie kostek} %24-2

\subproblem %24-2(a)
\subproblem %24-2(b)
\subproblem %24-2(c)

\problem{Arbitraż} %24-3

\subproblem %24-3(a)
\subproblem %24-3(b)

\problem{Algorytm skalujący Gabowa dla problemu najkrótszych ścieżek z~jednym źródłem} %24-4

\subproblem %24-4(a)
\subproblem %24-4(b)
\subproblem %24-4(c)
\subproblem %24-4(d)
\subproblem %24-4(e)
\subproblem %24-4(f)

\problem{Algorytm Karpa wyznaczania cyklu o~minimalnej średniej wadze} %24-5

\subproblem %24-5(a)
\subproblem %24-5(b)
\subproblem %24-5(c)
\subproblem %24-5(d)
\subproblem %24-5(e)
\subproblem %24-5(f)
\subproblem %24-5(g)

\problem{Bitoniczne najkrótsze ścieżki} %24-6


%\chapter{Najkrótsze ścieżki między wszystkimi parami wierzchołków}

\makeatletter
\def\input@path{{chapter25/}}
\makeatother

\subchapter{Najkrótsze ścieżki i~mnożenie macierzy}

\exercise %25.1-1
\exercise %25.1-2
\exercise %25.1-3
\exercise %25.1-4
\exercise %25.1-5
\exercise %25.1-6
\exercise %25.1-7
\exercise %25.1-8
\exercise %25.1-9
\exercise %25.1-10

\subchapter{Algorytm Floyda-Warshalla}

\exercise %25.2-1
\exercise %25.2-2
\exercise %25.2-3
\exercise %25.2-4
\exercise %25.2-5
\exercise %25.2-6
\exercise %25.2-7
\exercise %25.2-8
\exercise %25.2-9

\subchapter{Algorytm Johnsona dla grafów rzadkich}

\exercise %25.3-1
\exercise %25.3-2
\exercise %25.3-3
\exercise %25.3-4
\exercise %25.3-5
\exercise %25.3-6


\problems

\problem{Domknięcie przechodnie grafu dynamicznego} %25-1

\subproblem %25-1(a)
\subproblem %25-1(b)
\subproblem %25-1(c)

\problem{Najkrótsze ścieżki w~grafach $\epsilon$-gęstych} %25-2

\subproblem %25-2(a)
\subproblem %25-2(b)
\subproblem %25-2(c)
\subproblem %25-2(d)


%\chapter{Maksymalny przepływ}

\makeatletter
\def\input@path{{chapter26/}}
\makeatother

\subchapter{Sieci przepływowe}

\exercise %26.1-1
\exercise %26.1-2
\exercise %26.1-3
\exercise %26.1-4
\exercise %26.1-5
\exercise %26.1-6
\exercise %26.1-7
\exercise %26.1-8
\exercise %26.1-9

\subchapter{Metoda Forda-Fulkersona}

\exercise %26.2-1
\exercise %26.2-2
\exercise %26.2-3
\exercise %26.2-4
\exercise %26.2-5
\exercise %26.2-6
\exercise %26.2-7
\exercise %26.2-8
\exercise %26.2-9
\exercise %26.2-10

\subchapter{Najliczniejsze skojarzenia w~grafach dwudzielnych}

\exercise %26.3-1
\exercise %26.3-2
\exercise %26.3-3
\exercise %26.3-4
\exercise %26.3-5

\subchapter{Algorytmy typu ,,prześlij-przemianuj''}

\exercise %26.4-1
\exercise %26.4-2
\exercise %26.4-3
\exercise %26.4-4
\exercise %26.4-5
\exercise %26.4-6
\exercise %26.4-7
\exercise %26.4-8
\exercise %26.4-9

\subchapter{Algorytm ,,przemianuj i~przesuń na początek''}

\exercise %26.5-1
\exercise %26.5-2
\exercise %26.5-3
\exercise %26.5-4
\exercise %26.5-5


\problems

\problem{Problem ucieczki} %26-1

\subproblem %26-1(a)
\subproblem %26-1(b)

\problem{Minimalne pokrycie ścieżkowe} %26-2

\subproblem %26-2(a)
\subproblem %26-2(b)

\problem{Eksperymenty kosmiczne} %26-3

\subproblem %26-3(a)
\subproblem %26-3(b)
\subproblem %26-3(c)

\problem{Aktualizowanie maksymalnego przepływu} %26-4

\subproblem %26-4(a)
\subproblem %26-4(b)

\problem{Obliczanie maksymalnego przepływu metodą skalowania} %26-5

\subproblem %26-5(a)
\subproblem %26-5(b)
\subproblem %26-5(c)
\subproblem %26-5(d)
\subproblem %26-5(e)
\subproblem %26-5(f)

\problem{Maksymalny przepływ z~ujemnymi przepustowościami} %26-6

\subproblem %26-6(a)
\subproblem %26-6(b)
\subproblem %26-6(c)
\subproblem %26-6(d)

\problem{Algorytm Hopcrofta-Karpa obliczania najliczniejszego skojarzenia w~grafie dwudzielnym} %26-7

\subproblem %26-7(a)
\subproblem %26-7(b)
\subproblem %26-7(c)
\subproblem %26-7(d)
\subproblem %26-7(e)
\subproblem %26-7(f)
\subproblem %26-7(g)



% \setcounter{part}{6}
% \part{Wybrane zagadnienia}

%\setcounter{chapter}{26}
%\chapter{Sieci sortujące}

\makeatletter
\def\input@path{{chapter27/}}
\makeatother

\subchapter{Sieci porównujące}

\exercise %27.1-1
\exercise %27.1-2
\exercise %27.1-3
\exercise %27.1-4
\exercise %27.1-5
\exercise %27.1-6
\exercise %27.1-7
\exercise %27.1-8

\subchapter{Zasada zero-jedynkowa}

\exercise %27.2-1
\exercise %27.2-2
\exercise %27.2-3
\exercise %27.2-4
\exercise %27.2-5

\subchapter{Bitoniczna sieć sortująca}

\exercise %27.3-1
\exercise %27.3-2
\exercise %27.3-3
\exercise %27.3-4
\exercise %27.3-5
\exercise %27.3-6

\subchapter{Sieć scalająca}

\exercise %27.4-1
\exercise %27.4-2
\exercise %27.4-3
\exercise %27.4-4
\exercise %27.4-5

\subchapter{Sieć sortująca}

\exercise %27.5-1
\exercise %27.5-2
\exercise %27.5-3
\exercise %27.5-4
\exercise %27.5-5


\problems

\problem{Transpozycyjne sieci sortujące} %27-1

\subproblem %27-1(a)
\subproblem %27-1(b)
\subproblem %27-1(c)

\problem{Sieć scalająca odd-even Batchera} %27-2

\subproblem %27-2(a)
\subproblem %27-2(b)
\subproblem %27-2(c)
\subproblem %27-2(d)

\problem{Sieci permutacyjne} %27-3

\subproblem %27-3(a)
\subproblem %27-3(b)
\subproblem %27-3(c)
\subproblem %27-3(d)
\subproblem %27-3(e)


%\chapter{Operacje na macierzach}

\makeatletter
\def\input@path{{chapter28/}}
\makeatother

\subchapter{Własności macierzy}

\exercise %28.1-1
\exercise %28.1-2
\exercise %28.1-3
\exercise %28.1-4
\exercise %28.1-5
\exercise %28.1-6
\exercise %28.1-7
\exercise %28.1-8
\exercise %28.1-9
\exercise %28.1-10
\exercise %28.1-11

\subchapter{Algorytm Strassena mnożenia macierzy}

\exercise %28.2-1
\exercise %28.2-2
\exercise %28.2-3
\exercise %28.2-4
\exercise %28.2-5
\exercise %28.2-6

\subchapter{Rozwiązywanie układów równań liniowych}

\exercise %28.3-1
\exercise %28.3-2
\exercise %28.3-3
\exercise %28.3-4
\exercise %28.3-5
\exercise %28.3-6
\exercise %28.3-7

\subchapter{Odwracanie macierzy}

\exercise %28.4-1
\exercise %28.4-2
\exercise %28.4-3
\exercise %28.4-4
\exercise %28.4-5
\exercise %28.4-6

\subchapter{Symetryczne macierze dodatnio określone i~metoda najmniejszych kwadratów}

\exercise %28.5-1
\exercise %28.5-2
\exercise %28.5-3
\exercise %28.5-4
\exercise %28.5-5
\exercise %28.5-6
\exercise %28.5-7


\problems

\problem{Trójdiagonalne układy równań liniowych} %28-1

\subproblem %28-1(a)
\subproblem %28-1(b)
\subproblem %28-1(c)
\subproblem %28-1(d)
\subproblem %28-1(e)

\problem{Krzywe sklejane} %28-2

\subproblem %28-2(a)
\subproblem %28-2(b)
\subproblem %28-2(c)
\subproblem %28-2(d)
\subproblem %28-2(e)
\subproblem %28-2(f)


%\chapter{Programowanie liniowe}

\makeatletter
\def\input@path{{chapter29/}}
\makeatother

\subchapter{Postać standardowa i~uzupełnieniowa}

\exercise %29.1-1
\exercise %29.1-2
\exercise %29.1-3
\exercise %29.1-4
\exercise %29.1-5
\exercise %29.1-6
\exercise %29.1-7
\exercise %29.1-8
\exercise %29.1-9

\subchapter{Formułowanie problemów w~postaci programów liniowych}

\exercise %29.2-1
\exercise %29.2-2
\exercise %29.2-3
\exercise %29.2-4
\exercise %29.2-5
\exercise %29.2-6
\exercise %29.2-7

\subchapter{Algorytm sympleks}

\exercise %29.3-1
\exercise %29.3-2
\exercise %29.3-3
\exercise %29.3-4
\exercise %29.3-5
\exercise %29.3-6

\subchapter{Dualność}

\exercise %29.4-1
\exercise %29.4-2
\exercise %29.4-3
\exercise %29.4-4
\exercise %29.4-5
\exercise %29.4-6

\subchapter{Początkowe bazowe rozwiązanie dopuszczalne}

\exercise %29.5-1
\exercise %29.5-2
\exercise %29.5-3
\exercise %29.5-4
\exercise %29.5-5
\exercise %29.5-6
\exercise %29.5-7


\problems

\problem{Dopuszczalność układu nierówności liniowych} %29-1

\subproblem %29-1(a)
\subproblem %29-1(b)

\problem{Uzupełnienie dualne} %29-2

\subproblem %29-2(a)
\subproblem %29-2(b)
\subproblem %29-2(c)

\problem{Programowanie liniowe całkowitoliczbowe} %29-3

\subproblem %29-3(a)
\subproblem %29-3(b)
\subproblem %29-3(c)

\problem{Lemat Farkasa} %29-4


%\chapter{Wielomiany i~FFT}

\makeatletter
\def\input@path{{chapter30/}}
\makeatother

\subchapter{Reprezentacja wielomianów}

\exercise %30.1-1
\exercise %30.1-2
\exercise %30.1-3
\exercise %30.1-4
\exercise %30.1-5
\exercise %30.1-6
\exercise %30.1-7

\subchapter{DFT i~FFT}

\exercise %30.2-1
\exercise %30.2-2
\exercise %30.2-3
\exercise %30.2-4
\exercise %30.2-5
\exercise %30.2-6
\exercise %30.2-7
\exercise %30.2-8

\subchapter{Efektywne implementacje FFT}

\exercise %30.3-1
\exercise %30.3-2
\exercise %30.3-3
\exercise %30.3-4


\problems

\problem{Mnożenie metodą ,,dziel i~zwyciężaj''} %30-1

\subproblem %30-1(a)
\subproblem %30-1(b)
\subproblem %30-1(c)

\problem{Macierze Toeplitza} %30-2

\subproblem %30-2(a)
\subproblem %30-2(b)
\subproblem %30-2(c)
\subproblem %30-2(d)

\problem{Wielowymiarowe szybkie przekształcenie Fouriera} %30-3

\subproblem %30-3(a)
\subproblem %30-3(b)
\subproblem %30-3(c)

\problem{Ewaluacja wszystkich pochodnych wielomianu w~punkcie} %30-4

\subproblem %30-4(a)
\subproblem %30-4(b)
\subproblem %30-4(c)
\subproblem %30-4(d)

\problem{Ewaluacja wielomianu w~wielu punktach} %30-5

\subproblem %30-5(a)
\subproblem %30-5(b)
\subproblem %30-5(c)
\subproblem %30-5(d)

\problem{FFT w~arytmetyce modularnej} %30-6

\subproblem %30-6(a)
\subproblem %30-6(b)
\subproblem %30-6(c)
\subproblem %30-6(d)


%\chapter{Algorytmy teorioliczbowe}

\makeatletter
\def\input@path{{chapter31/}}
\makeatother

\subchapter{Podstawowe pojęcia teorii liczb}

\exercise %31.1-1
\exercise %31.1-2
\exercise %31.1-3
\exercise %31.1-4
\exercise %31.1-5
\exercise %31.1-6
\exercise %31.1-7
\exercise %31.1-8
\exercise %31.1-9
\exercise %31.1-10
\exercise %31.1-11
\exercise %31.1-12

\subchapter{Największy wspólny dzielnik}

\exercise %31.2-1
\exercise %31.2-2
\exercise %31.2-3
\exercise %31.2-4
\exercise %31.2-5
\exercise %31.2-6
\exercise %31.2-7
\exercise %31.2-8
\exercise %31.2-9

\subchapter{Arytmetyka modularna}

\exercise %31.3-1
\exercise %31.3-2
\exercise %31.3-3
\exercise %31.3-4
\exercise %31.3-5

\subchapter{Rozwiązywanie modularnych równań liniowych}

\exercise %31.4-1
\exercise %31.4-2
\exercise %31.4-3
\exercise %31.4-4

\subchapter{Chińskie twierdzenie o~resztach}

\exercise %31.5-1
\exercise %31.5-2
\exercise %31.5-3
\exercise %31.5-4

\subchapter{Potęgi elementu}

\exercise %31.6-1
\exercise %31.6-2
\exercise %31.6-3

\subchapter{System kryptograficzny z~kluczem publicznym RSA}

\exercise %31.7-1
\exercise %31.7-2
\exercise %31.7-3

\subchapter{Sprawdzanie, czy dana liczba jest pierwsza}

\exercise %31.8-1
\exercise %31.8-2
\exercise %31.8-3

\subchapter{Rozkład na czynniki}

\exercise %31.9-1
\exercise %31.9-2
\exercise %31.9-3
\exercise %31.9-4


\problems

\problem{Binarny algorytm gcd} %31-1

\subproblem %31-1(a)
\subproblem %31-1(b)
\subproblem %31-1(c)
\subproblem %31-1(d)

\problem{Analiza operacji na bitach w~algorytmie Euklidesa} %31-2

\subproblem %31-2(a)
\subproblem %31-2(b)
\subproblem %31-2(c)

\problem{Trzy algorytmy dla liczb Fibonacciego} %31-3

\subproblem %31-3(a)
\subproblem %31-3(b)
\subproblem %31-3(c)
\subproblem %31-3(d)

\problem{Reszty kwadratowe} %31-4

\subproblem %31-4(a)
\subproblem %31-4(b)
\subproblem %31-4(c)
\subproblem %31-4(d)


%\chapter{Wyszukiwanie wzorca}

\makeatletter
\def\input@path{{chapter32/}}
\makeatother

\subchapter{Algorytm ,,naiwny'' wyszukiwania wzorca}

\exercise %32.1-1
\exercise %32.1-2
\exercise %32.1-3
\exercise %32.1-4

\subchapter{Algorytm Rabina-Karpa}

\exercise %32.2-1
\exercise %32.2-2
\exercise %32.2-3
\exercise %32.2-4

\subchapter{Wyszukiwanie wzorca z~wykorzystaniem automatów skończonych}

\exercise %32.3-1
\exercise %32.3-2
\exercise %32.3-3
\exercise %32.3-4
\exercise %32.3-5

\subchapter{Algorytm Knutha-Morrisa-Pratta}

\exercise %32.4-1
\exercise %32.4-2
\exercise %32.4-3
\exercise %32.4-4
\exercise %32.4-5
\exercise %32.4-6


\problems

\problem{Wyszukiwanie wzorca z~wykorzystaniem współczynników powtórzeń} %32-1

\subproblem %32-1(a)
\subproblem %32-1(b)
\subproblem %32-1(c)


%\chapter{Geometria obliczeniowa}

\makeatletter
\def\input@path{{chapter33/}}
\makeatother

\subchapter{Własności odcinków}

\exercise %33.1-1
\exercise %33.1-2
\exercise %33.1-3
\exercise %33.1-4
\exercise %33.1-5
\exercise %33.1-6
\exercise %33.1-7
\exercise %33.1-8

\subchapter{Sprawdzanie, czy jakakolwiek para odcinków się przecina}

\exercise %33.2-1
\exercise %33.2-2
\exercise %33.2-3
\exercise %33.2-4
\exercise %33.2-5
\exercise %33.2-6
\exercise %33.2-7
\exercise %33.2-8
\exercise %33.2-9

\subchapter{Znajdowanie otoczki wypukłej}

\exercise %33.3-1
\exercise %33.3-2
\exercise %33.3-3
\exercise %33.3-4
\exercise %33.3-5
\exercise %33.3-6

\subchapter{Znajdowanie pary najmniej odległych punktów}

\exercise %33.4-1
\exercise %33.4-2
\exercise %33.4-3
\exercise %33.4-4
\exercise %33.4-5


\problems

\problem{Warstwy wypukłe} %33-1

\subproblem %33-1(a)
\subproblem %33-1(b)

\problem{Warstwy maksimów} %33-2

\subproblem %33-2(a)
\subproblem %33-2(b)
\subproblem %33-2(c)
\subproblem %33-2(d)

\problem{Pogromcy duchów} %33-3

\subproblem %33-3(a)
\subproblem %33-3(b)

\problem{Gra w~bierki} %33-4

\subproblem %33-4(a)
\subproblem %33-4(b)

\problem{Rozkłady skoncentrowane} %33-5

\subproblem %33-5(a)
\subproblem %33-5(b)


% \setcounter{chapter}{33}
% \newcommand{\Pclass}{\text{P}}
\newcommand{\NPclass}{\text{NP}}
\newcommand{\coNPclass}{\text{co-NP}}
\DeclareGraphicsRule{.1}{mps}{*}{}

\chapter{\singledash{\NPclass}{zupełność}}

% ROZDZIAŁ 34 -- rozdział oderwany od głównego planu, sprawdzić po przerobieniu rozdziałów z~grafami
% 34.1-4 -- zostaje do publikacji rozdziału 16
% 34.2-4 --
% 34.2-11 -- zostaje do publikacji rozdziału 23
% 34.3-4 --
% 34.3-5 --
% 34.3-8
% 34.4-1
% 34.4-7 -- zostaje do publikacji części VI
% 34.5-2
% 34.5-4
% 34.5-5
% 34.5-6
% 34.5-7
% 34.5-8
% 34-1(a)
% 34-1(b)
% 34-1(c)
% 34-1(d)
% 34-2(a)
% 34-2(b)
% 34-2(c)
% 34-2(d)
% 34-3(b) -- dokończyć
% 34-3(c) -- czy można tak dowieść NP-trudności?
% 34-3(d)
% 34-3(e)
% 34-3(f)
% 34-4(a)
% 34-4(b)
% 34-4(c)
% 34-4(d)

\subchapter{Czas wielomianowy}

\exercise %34.1-1
Fakt, że $\text{LONGEST-PATH}\in\Pclass$ oznacza, że problem ten można rozwiązać w~czasie wielomianowym, a~zatem istnieje taka stała $c$, że problem rozwiązuje pewien algorytm o~złożoności $O(n^c)$, gdzie $n$ jest długością standardowego kodowania egzemplarza tego problemu.
Wykonajmy ten algorytm dla kolejnych wartości $k$, zaczynając od 0 i~po każdym wykonaniu sprawdzając odpowiedź.
Jeśli dla pewnego $k=k_0$ odpowiedzią było ,,tak'', a~dla $k=k_0+1$ ,,nie'', to wiadomo, że w~grafie $G$ istnieje ścieżka prosta o~długości $k_0$ między wierzchołkami $u$ i~$v$, ale nie istnieje taka ścieżka o~długości $k_0+1$.
Ponieważ $k_0$ jest ograniczone przez $|V|-1$, to algorytm wywołamy co najwyżej $|V|$ razy.
Na mocy tego, że $|V|=O(\langle G\rangle)=O(n)$ mamy, że złożonością przedstawionej tu procedury rozwiązywania problemu LONGEST-PATH-LENGTH jest $O(n^{c+1})$, a~więc złożoność wielomianowa.

\exercise %34.1-2
Problem znajdowania najdłuższego cyklu prostego w~grafie nieskierowanym to relacja
\[
	\text{LONGEST-CYCLE-LENGTH}\subseteq I\times S, \quad \text{gdzie}
\]
\begin{align*}
	I &= \bigl\{\,G:G=\langle V,E\rangle\text{ jest grafem nieskierowanym}\,\bigr\}, \\
	S &= \bigl\{\,c\in V^*\!:c\text{ jest najdłuższym cyklem prostym w~$G$}\,\bigr\}.
\end{align*}
Związany z~nim problem decyzyjny jest funkcją
\[
	\text{LONGEST-CYCLE}\colon I_D\to\{0,1\}, \quad \text{gdzie}
\]
\[
	I_D = \bigl\{\,\langle G,c\rangle:G=\langle V,E\rangle\text{ jest grafem nieskierowanym i~}c\in V^*\,\bigr\}
\]
oraz $\text{LONGEST-CYCLE}(G,c)=1$ wtedy i~tylko wtedy, gdy $c$ stanowi najdłuższy cykl prosty w~grafie $G$.
Językiem odpowiadającym problemowi LONGEST-CYCLE jest
\[
	\begin{split}
		L = \bigl\{\,x\in\{0,1\}^*\!&:x\text{ jest standardowym kodowaniem $i\in I_D$,} \\
		&\qquad \text{dla którego }\text{LONGEST-CYCLE}(i)=1\,\bigr\}.
	\end{split}
\]

\exercise %34.1-3
Ponumerujmy wierzchołki grafu skierowanego $G=\langle V,E\rangle$ liczbami całkowitymi, zaczynając od 1.
Graf taki można reprezentować w~postaci ciągu $x\in\{0,1\}^*$ takiego, że $|x|=|V|^2$ i~na \singledash{$i$}{tej} jego pozycji znajduje się 1 wtedy i~tylko wtedy, gdy wierzchołek o~numerze $\lfloor(i-1)/|V|\rfloor+1$ ma sąsiada o~numerze $(i-1)\bmod|V|+1$.
Ciąg $x$ jest więc macierzą sąsiedztwa grafu $G$ zapisaną kolejnymi wierszami.

W~reprezentacji listowej grafu $G$ można dla każdego kolejnego wierzchołka po zapisaniu liczby jego sąsiadów, wypisywać ich numery w~kolejności rosnącej.
Wszystkie liczby będą reprezentowane binarnie na pewnej ustalonej liczbie bitów, otrzymamy więc w~rezultacie ciąg $y\in\{0,1\}^*$.

Aby pokazać, że reprezentacje te są wielomianowo równoważne, musimy znaleźć funkcje obliczalne w~czasie wielomianowym, dokonujące przekształceń jednej reprezentacji na drugą.
Z~reprezentacji macierzowej $x$ budujemy reprezentację listową, najpierw znajdując liczbę wierzchołków grafu poprzez wyciągnięcie pierwiastka kwadratowego z~$|x|$, a~następnie wyznaczamy dla każdego wierzchołka liczbę jego sąsiadów na podstawie liczby jedynek w~odpowiednim podciągu $x$.
Po zapisaniu tej liczby w~postaci binarnej wypisywane są reprezentacje binarne numerów jego sąsiadów w~kolejności rosnącej.

Z~kolei posiadając reprezentację listową, po wyznaczeniu liczby wierzchołków grafu, dla każdego sąsiada kolejnego wierzchołka wypisujemy 1 na odpowiedniej pozycji wynikowego ciągu, natomiast 0, jeśli odpowiednie wierzchołki nie są sąsiednie.

Nietrudno sprawdzić, że dla obu funkcji istnieją obliczające je algorytmy wielomianowe, a~zatem obie reprezentacje są wielomianowo równoważne.

\exercise %34.1-4
\note{Rozwiązanie ukaże się po publikacji rozdziału 16.}

\exercise %34.1-5
Załóżmy, że czasy działania wszystkich wywoływanych podprogramów ograniczone są przez $O(n^c)$, gdzie $c$ jest pewną stałą.
Algorytm wykonujący co najwyżej $k$ takich wywołań działa w~czasie $O(kn^c)$, a~co najwyżej $O(n^d)$ dla innej stałej $d$ -- w~czasie $O(n^{c+d})$.
Przyjęliśmy jednak, że wywołania podprogramów nie zmieniają danych wejściowych dla kolejnych wywołań i~za każdym razem ich długością jest $n$.

Algorytm wykonujący co najwyżej $O(n^c)$ operacji może z~danych o~rozmiarze $n$ utworzyć wynik o~rozmiarze $O(n^c)$, który może zostać przekazany jako dane wejściowe do następnego algorytmu.
W~takiej sytuacji po $k$ wywołaniach rozmiar danych będzie wynosił $O\bigl(n^{c^k}\bigr)$ i~czas działania tego ciągu wywołań będzie tego samego rzędu, a~więc co najwyżej wielomianowy.

Załóżmy jednak, że na początku działania algorytmu ustalamy $O(n^d)$ jako liczbę wywołań podprogramów o~złożoności $O(n^c)$ każdy.
Po wszystkich wywołaniach rozmiar danych może wzrosnąć do $O\Bigl(n^{c^{n^d}}\Bigr)$ i~taki też będzie czas działania całego algorytmu.
Ponieważ $n^{c^{n^d}}\!\!=\omega(c^n)$, to możemy otrzymać w~tym przypadku algorytm wykładniczy.

\exercise %34.1-6
Ponieważ $L_1$, $L_2\in\Pclass$, to istnieją algorytmy, odpowiednio, $A_1$, $A_2$, rozstrzygające te języki w~czasie wielomianowym.
Aby rozstrzygać język $L=L_1\cup L_2$, wystarczy dla dowolnego $x\in L$ zwracać 1, jeśli $A_1(x)=1$ lub $A_2(x)=1$, a~w~przeciwnym przypadku zwracać 0.
Oczywiście wykonanie obu algorytmów wielomianowych i~wykonanie dwóch takich testów zajmuje nadal czas wielomianowy, więc $L\in\Pclass$.

Analogicznie rozstrzygamy język $L=L_1\cap L_2$, zwracając1 wtedy i~tylko wtedy, gdy $A_1(x)=1$ i~$A_2(x)=1$, gdzie $x\in L$.

W~przypadku języka $L=\overline{L_1}$, dla $x\in L$ zwracamy 1 wtedy i~tylko wtedy, gdy $A_1(x)=0$.

Język $L=L_1\cdot L_2$ będący konkatenacją $L_1$ i~$L_2$ rozstrzygamy, biorąc $x\in L$ i~zwracając 1 wtedy i~tylko wtedy, gdy dla pewnego $m=0$, 1, \dots, $|x|$ zachodzi $A_1(x_1)=1$ oraz $A_2(x_2)=1$, przy czym $x_1$ jest \singledash{$m$}{symbolowym} prefiksem słowa $x$, a~$x_2$ -- jego \singledash{$(|x|-m)$}{symbolowym} sufiksem.

Aby rozstrzygać domknięcie $L=L_1^*$, należy dla $x\in L$ zwracać 1 wtedy i~tylko wtedy, gdy dla pewnego $k=0$, 1, \dots, $|x|$ prawdą jest, że $x\in L_1^k$.
Sprawdzenie tej przynależności jest wykonalne w~czasie wielomianowym, co wynika z~uogólnienia dowodu z~poprzedniego paragrafu na iloczyn dowolnej skończonej ilości języków.

Nietrudno zobaczyć, że w~każdym opisanym przypadku jesteśmy w~stanie rozstrzygać język $L$ w~czasie wielomianowym, czyli $L\in\Pclass$ dla każdego $L$ zdefiniowanego powyżej.

\subchapter{Weryfikacja w~czasie wielomianowym}

\exercise %34.2-1
Dwa grafy $G=\langle V,E\rangle$ i~$G'=\langle V',E'\rangle$ są izomorficzne, jeśli istnieje bijekcja $f\colon V\to V'$ taka, że $\langle u,v\rangle\in E$ wtedy i~tylko wtedy, gdy $\langle f(u),f(v)\rangle\in E'$.
Jeśli potrafimy zweryfikować w~czasie wielomianowym, że dana bijekcja spełnia ten warunek, to $\text{GRAPH-ISOMORPHISM}\in\NPclass$.
Jest to oczywiście możliwe -- dla reprezentacji macierzowej grafu potrafimy znaleźć algorytm o~złożoności $O(n^2)$, gdzie $n$ jest długością kodowania $G$.
Algorytm ten sprawdza, czy każdy element macierzy sąsiedztwa grafu $G$ jest równy odpowiadającemu, na podstawie odwzorowania $f$, elementowi macierzy sąsiedztwa grafu $G'$.

\exercise %34.2-2
Z~punktu (b) problemu \refProblem{B-1} mamy, że żaden graf dwudzielny nie posiada cyklu o~nieparzystej długości, więc w~grafach dwudzielnych o~nieparzystej liczbie wierzchołków nie istnieje cykl Hamiltona.

\exercise %34.2-3
Dla kolejnej krawędzi grafu sprawdzamy, czy jej usunięcie spowoduje zniszczenie cyklu Hamiltona, poprzez uruchomienie algorytmu sprawdzania czy graf pozbawiony takiej krawędzi nadal jest hamiltonowski.
Bezpieczne krawędzie usuwamy z~grafu, natomiast pozostałe zachowujemy.
Postępowanie kontynuujemy aż do pozostawienia samego cyklu Hamiltona, który można teraz wypisać.
Cały algorytm jest wielomianowy przy założeniu, że $\text{HAM-CYCLE}\in\Pclass$.

\exercise %34.2-4
Niech $L_1,L_2\in\NPclass$.
Oznacza to, że istnieje wielomianowy algorytm weryfikacji $A_1$ i~pewna stała $c_1$, że dla każdego $x\in L_1$ istnieje takie świadectwo $y$, gdzie $|y|=O(|x|^{c_1})$, że $A_1(x,y)=1$ oraz analogicznie dla algorytmu weryfikacji $A_2$ i~stałej $c_2$ dla języka $L_2$.
Rozważmy język $L=L_1\cup L_2$ oraz dowolne $x\in L$.
Oznacza to, że $x\in L_1$ lub $x\in L_2$.
Algorytm weryfikacji $L$ dla takiego słowa i~jego świadectwa $y$ zwraca 1 wtedy i~tylko wtedy, gdy $A_1(x,y)=1$ lub $A_2(x,y)=1$.
Jest to algorytm wielomianowy, skąd mamy $L\in\NPclass$.

Analogicznie, jeśli $L=L_1\cap L_2$, to dla dowolnego $x\in L$ wynika, że $x\in L_1$ oraz $x\in L_2$, a~więc dla tego $x$ i~jego świadectwa $y$ zwracamy 1 wtedy i~tylko wtedy, gdy $A_1(x,y)=1$ i~$A_2(x,y)=1$.
Taki algorytm rozstrzyga $L$ w~czasie wielomianowym.

Język $L=L_1\cdot L_2$ rozstrzygamy, biorąc $x\in L$ i~jego świadectwo, i~zwracając 1 wtedy i~tylko wtedy, gdy dla pewnego $m=0$, 1, \dots, $|x|$ istnieją $y_1$, $y_2$ oraz zachodzi $A_1(x_1,y_1)=1$ i~$A_2(x_2,y_2)$, gdzie $x_1$ jest \singledash{$m$}{symbolowym} prefiksem $x$, $x_2$ jest \singledash{$(|x|-m)$}{symbolowym} sufiksem $x$, a~$y_1$ i~$y_2$ odpowiednio, jego świadectwami.

Aby rozstrzygać domknięcie $L=L_1^*$, należy dla $x\in L$ zwracać 1 wtedy i~tylko wtedy, gdy dla pewnego $k=0$, 1, \dots, $|x|$ prawdą jest, że $x\in L_1^k$.
Sprawdzenie tej przynależności jest wykonalne w~czasie wielomianowym, co wynika z~uogólnienia dowodu z~poprzedniego paragrafu na iloczyn dowolnej skończonej ilości języków.

W~każdym z~powyżej rozważanych przypadków opisaliśmy algorytm weryfikujący $L$, więc każdy język $L$ definiowany w~powyższych paragrafach spełnia $L\in\NPclass$.

Jeśli z~kolei $L=\overline{L_1}$, to nie możemy stwierdzić, czy $L\in\NPclass$, ponieważ dla pewnego $x\in L$ może nie istnieć świadectwo $y$ lub istnieć, ale nie być ograniczone przez wielomian względem $|x|$.
Zaistniała trudność uniemożliwia dowód, że $\NPclass=\coNPclass$.

\exercise %34.2-5
Dla każdego języka $L$ z~klasy \NPclass\ znany jest algorytm weryfikacji działający w~czasie wielomianowym.
Przy braku znajomości świadectwa w~rozstrzyganiu czy $x\in L$ jesteśmy zmuszeni sprawdzić wszystkie możliwe ciągi $y\in\{0,1\}^*$ o~długości $O(n^c)$, gdzie $n=|x|$, a~$c$ jest pewną stałą.
Wszystkich takich ciągów jest $2^{O(n^c)}$, a~korzystając z~tego, że algorytm weryfikacji działa w~czasie $O(n^k)$ dla pewnej stałej $k\ge c$ otrzymujemy, że czasem działania algorytmu rozstrzygającego język $L$ jest $O(n^k)\cdot2^{O(n^c)}=2^{O(n^k)+O(n^c)}=2^{O(n^k)}$.

\exercise %34.2-6
Dla świadectwa będącego ścieżką z~$u$ do $v$ w~grafie $G$ możemy zweryfikować w~czasie wielomianowym, czy jest to ścieżka Hamiltona, przeglądając kolejne wierzchołki i~sprawdzając, czy są sąsiednie i~czy ścieżka zawiera każdy wierzchołek grafu $G$ dokładnie raz.
Jest to możliwe w~czasie kwadratowym przy reprezentacji macierzowej grafu.
Stąd, $\text{HAM-PATH}\in\NPclass$.

\exercise %34.2-7
Acykliczny graf skierowany jest dagiem, który możemy posortować topologicznie algorytmem \proc{Topological-Sort}, a~następnie sprawdzić, czy każde dwa kolejne wierzchołki w~otrzymanym uporządkowaniu topologicznym są sąsiednie w~tym grafie.
Dla dagu $G=\langle V,E\rangle$ algorytm znajdowania ścieżki Hamiltona działa więc w~czasie $O(V+E)$.

\exercise %34.2-8
Równoważnie należy dowieść, że dopełnienie języka TAUTOLOGY należy do klasy \NPclass.
Jest to język złożony z~takich formuł, które nie są tautologiami -- dla każdej formuły należącej do tego języka istnieje zatem pewne wartościowanie, które jej nie spełnia.
Dostając pewne świadectwo, będące wartościowaniem zmiennych logicznych użytych w~formule, możemy zweryfikować czy formuła ta nie jest dla niego spełniona w~czasie wielomianowym, ewaluując jej wartość dla tego wartościowania.

\exercise %34.2-9
Wiemy, że $\Pclass\subseteq\NPclass$.
Stąd dla dowolnego $L\in\Pclass$ zachodzi $L\in\NPclass$, czyli $\overline{L}\in\coNPclass$.
Z~drugiej strony mamy, że klasa \Pclass\ jest zamknięta na operację dopełnienia (\refExercise{34.1-6}), skąd $\overline{L}\in\Pclass$.
A~zatem dla dowolnego $\overline{L}$, jeśli $\overline{L}\in\Pclass$, to $\overline{L}\in\coNPclass$, co daje $\Pclass\subseteq\coNPclass$.

\exercise %34.2-10
Wiemy, że $\Pclass\subseteq\NPclass\cap\coNPclass$.
Jeśli $\NPclass\ne\coNPclass$, to klasa $\NPclass\setminus\coNPclass$ jest niepusta, czyli istnieją języki należące do tej klasy, ale nienależące do \Pclass, czyli $\Pclass\ne\NPclass$.

\exercise %34.2-11
\note{Rozwiązanie ukaże się po publikacji rozdziału 23.}

\subchapter{\singledash{\NPclass}{zupełność} i~redukowalność}

\exercise %34.3-1
Oznaczmy bramki jak na rys.\ \ref{fig:34.3-1}.
Widać, że do spełnienia układu konieczne jest, aby wszystkie wejścia bramki $g_7$ miały wartości 1.
Bramka $g_5$ zwraca 1 tylko wtedy, gdy na wejściu bramki $g_3$ będzie wartość 0 i~$g_2$ zwraca 1.
Jednakże wejście do $g_3$ jest rozgałęzione i~podawane również na wejście bramki $g_4$, która otrzymując 0 na którymkolwiek wejściu, zwróci 0.
Nie jest więc możliwa sytuacja, w~której bramka $g_7$ nie dostaje na którymkolwiek wejściu wartości 0, w~wyniku czego dla żadnego wartościowania układ nie zwróci 1, co oznacza, że nie jest spełnialny.
\begin{figure}[ht]
	\begin{center}
		\includegraphics{chapter34/fig34.1}
	\end{center}
	\caption{Niespełnialny układ logiczny.} \label{fig:34.3-1}
\end{figure}

\exercise %34.3-2
Z~założenia mamy, że istnieją funkcje redukcji $f_1$ i~$f_2$ przekształcające, odpowiednio, $L_1$ na $L_2$ i~$L_2$ na $L_3$.
Ponieważ złożenie funkcji obliczalnych w~czasie wielomianowym jest również funkcją obliczalną w~czasie wielomianowym, to $f_2\circ f_1$ można potraktować jako funkcję redukcji $L_1$ na $L_3$, co dowodzi przechodniości relacji $\le_\Pclass$.

\exercise %34.3-3
Równoważność jest symetryczna, udowodnimy więc tylko implikację w~jedną stronę.

Niech $f$ będzie funkcją redukcji przekształcającą $L$ na $\overline{L}$.
Oznacza to, że $x\in L$ wtedy i~tylko wtedy, gdy $f(x)\in\overline{L}$.
Równoważnie, jeśli $x\notin L$, czyli $x\in\overline{L}$, to zachodzi $f(x)\notin\overline{L}$, czyli $f(x)\in L$, a~to jest definicja funkcji redukcji przekształcającej $\overline{L}$ na $L$, skąd mamy $\overline{L}\le_\Pclass L$.

\exercise %34.3-4
Aby użyć wartościowania spełniającego, należy zweryfikować, że spełnia ono dany układ logiczny $C$ poprzez zasymulowanie działania tego układu dla wejściowych wartości.
Ponieważ układ $C$ można zamodelować jako acykliczny graf skierowany, to kolejne wartości można obliczać zgodnie z~uporządkowaniem topologicznym jego wierzchołków.
Jesteśmy zatem w~stanie w~czasie wielomianowym sprawdzać wyjście układu $C$ dla danego wartościowania, a~zatem weryfikować świadectwo.
Ponieważ wartościowanie spełniające jednoznacznie determinuje wartości na wszystkich przewodach układu $C$, to wystarcza ono jako świadectwo.

Modyfikacja tego dowodu lematu jest o~tyle trudniejsza, że musieliśmy zauważyć istnienie modelu układu jako grafu, po czym opisać przebieg weryfikacji w~tym modelu.

\exercise %34.3-5
Obszar roboczy musi być spójny, bo stanowi wejście do układu $M$, a~nigdzie nie przechowujemy informacji w~kolejnych konfiguracjach o~miejscu rezydowania obszaru roboczego w~pamięci -- zakładamy więc, że jest to spójny blok o~ustalonym rozmiarze.
Można jednak pozwolić na jego rozproszenie w~pamięci, pamiętając adresy poszczególnych bloków.
Dzięki tym adresom przechowywanym w~części zawierającej dane o~stanie maszyny symulujemy ciągłość obszaru roboczego.
Ponieważ jego rozmiar jest wielomianowy, to użyjemy dodatkowo wielomianowej ilości adresów -- nie zmieni to zatem rzędu wielkości pamięci zajmowanej przez każdą konfigurację.

\exercise %34.3-6
Niech $L\in\Pclass$ nie będzie językiem pustym lub $\{0,1\}^*$.
Wtedy dla dowolnego języka $L'\in\Pclass$ istnieje funkcja redukcji $f$ przeprowadzająca $L'$ w~$L$, ponieważ jesteśmy w~stanie w~czasie wielomianowym stwierdzić, czy $x\in L'$ i~na tej podstawie zwrócić słowo $f(x)$ z~$L$ albo z~$\overline{L}$.
Dowolny taki język $L$ jest więc zupełny w~\Pclass\ ze względu na redukcję w~czasie wielomianowym.

Jeśli $L=\emptyset$, to nie można odwzorować żadnego słowa z~$L'$ na słowo z~$L$ -- funkcja redukcji zatem nie istnieje.
Dla języka $L=\{0,1\}^*$ sytuacja jest symetryczna -- żadnego słowa z~$\overline{L'}$ nie da się odwzorować na słowo z~$\overline{L}$.
Te dwa języki są jedynymi w~klasie \Pclass, które nie są w~niej zupełne ze względu na redukcję w~czasie wielomianowym.

\exercise %34.3-7
Udowodnimy implikację tylko w~jedną stronę, gdyż druga z~nich jest symetryczna.

Załóżmy, że język $L$ jest \singledash{\NPclass}{zupełny}.
Na mocy definicji \singledash{\NPclass}{zupełności} $L\in\NPclass$, więc $\overline{L}\in\coNPclass$.
Pozostaje udowodnić drugi punkt z~definicji.
Wiemy, że $L'\le_\Pclass L$ dla każdego języka $L'\in\NPclass$.
Istnieje zatem funkcja redukcji $f$, która zwaca element z~$L$ dla każdego elementu należącego do $L'$ i~tylko takiego.
Oznacza to, że dla każdego słowa z~$\overline{L'}$, funkcja $f$ zwraca słowo z~$\overline{L}$, a~ponieważ $\overline{L'}\in\coNPclass$ jest dowolnym językiem, to $\overline{L'}\le_\Pclass\overline{L}$.
Język $\overline{L}$ jest zatem zupełny w~\coNPclass.

\exercise %34.3-8

\subchapter{Dowodzenie \singledash{\NPclass}{zupełności}}

\exercise %34.4-1

\exercise %34.4-2
Szukana formuła \singledash{3}{CNF} ma postać
\[
	\phi''' = \psi_0\wedge\psi_1\wedge\psi_2\wedge\psi_3\wedge\psi_4\wedge\psi_5\wedge\psi_6,
\]
dla następujących klauzul $\psi_i$:
\begin{align*}
	\psi_0 &= (y_1\vee p\vee q)\wedge(y_1\vee p\vee\neg q)\wedge(y_1\vee\neg p\vee q)\wedge(y_1\vee\neg p\vee\neg q), \\
	\psi_1 &= (\neg y_1\vee\neg y_2\vee\neg x_2)\wedge(\neg y_1\vee y_2\vee\neg x_2)\wedge(\neg y_1\vee y_2\vee x_2)\wedge(y_1 \vee\neg y_2\vee x_2), \\
	\psi_2 &= (y_2\vee y_3\vee\neg y_4)\wedge(y_2\vee\neg y_3\vee y_4)\wedge(y_2\vee\neg y_3\vee\neg y_4)\wedge(\neg y_2\vee y_3\vee y_4), \\
	\psi_3 &= (y_3\vee x_1\vee x_2)\wedge(y_3\vee x_1\vee\neg x_2)\wedge(y_3\vee\neg x_1\vee\neg x_2)\wedge(\neg y_3\vee\neg x_1\vee x_2), \\
	\psi_4 &= (y_4\vee y_5\vee p)\wedge(y_4\vee y_5\vee\neg p)\wedge(\neg y_4\vee\neg y_5\vee p)\wedge(\neg y_4\vee\neg y_5\vee\neg p), \\
	\psi_5 &= (y_5\vee y_6\vee\neg x_4)\wedge(y_5\vee\neg y_6\vee x_4)\wedge(y_5\vee\neg y_6\vee\neg x_4)\wedge(\neg y_5\vee y_6\vee x_4), \\
	\psi_6 &= (y_6\vee x_1\vee\neg x_3)\wedge(y_6\vee\neg x_1\vee x_3)\wedge(\neg y_6\vee x_1\vee x_3)\wedge(\neg y_6\vee\neg x_1\vee\neg x_3).
\end{align*}

\exercise %34.4-3
Główny problem tego rozumowania polega na tym, że dla formuły o~$n$ zmiennych tablica ich wartości składa się z~$2^n$ wierszy, ponieważ odpowiada rozważeniu wszystkich \singledash{$n$}{elementowych} wektorów bitowych.
Ponieważ może się zdarzyć, że dla każdego takiego wartościowania formuła przyjmie wartość 0, to budowa równoważnej formuły \singledash{3}{DNF} zajmie czas wykładniczy.

\exercise %34.4-4
Z~\refExercise{34.3-7} wiemy, że równoważnie wystarczy udowodnić, iż dopełnienie języka TAUTOLOGY jest \singledash{\NPclass}{zupełne}.
Ponieważ w~\refExercise{34.2-8} wykazaliśmy, że należy do klasy \NPclass, to wystarczy pokazać, że jest to język \singledash{\NPclass}{trudny}.

Dana formuła nie jest tautologią, jeśli istnieje wartościowanie, dla którego nie jest spełniona.
Możemy więc pokazać, że $\text{SAT}\le_\Pclass\overline{\text{TAUTOLOGY}}$, definiując funkcję redukcji, która dla danej formuły $\phi$ zwraca dowolną formułę nie będącą tautologią, jeśli $\phi$ jest spełnialne i~dowolną tautologię w~przeciwnym przypadku.
Oczywiście funkcja ta jest obliczalna w~czasie wielomianowym, co dowodzi \singledash{\NPclass}{trudności} języka.

Na mocy powyższych wniosków stwierdzamy, że język TAUTOLOGY jest \singledash{\coNPclass}{zupełny}.

\exercise %34.4-5
Formuła $\phi$ jest w~dysjunkcyjnej postaci normalnej, jeśli
\[
	\phi = \phi_1\vee\phi_2\vee\cdots\vee\phi_n,
\]
a~$\phi_i$ dla dowolnego $i=1$, 2, \dots, $n$ jest koniunkcją pewnej ilości literałów.
Formuła $\phi$ jest spełnialna, jeśli spełnialna jest pewna klauzula $\phi_i$, a~to z~kolei można zweryfikować w~czasie wielomianowym, sprawdzając, czy nie składa się ona z~literałów komplementarnych, czyli $x$ i~$\neg x$.
Jeśli nie ma pary takich literałów, to możliwe jest przyjęcie wartościowania, w~którym każdy literał przyjmuje wartość 1, co spełnia klauzulę $\phi_i$.

\exercise %34.4-6
Załóżmy, że $\phi$ jest formułą zawierającą zmienne $x_1$, $x_2$, \dots, $x_n$, dla której chcemy znaleźć wartościowanie spełniające.
Przyjmijmy, że $\phi$ jest spełnialna -- w~przeciwnym przypadku wartościowanie spełniające nie istnieje.

Dla kolejnej zmiennej $x_i$ będziemy w~pętli sprawdzać, czy formuła $\phi$, dla której przyjęto $x_i=1$, jest spełnialna.
Jeśli tak, to zmienna $x_i$ przyjmuje wartość 1 w~pewnym wartościowaniu spełniającym dla $\phi$, więc zastępujemy każde wystąpienie tej zmiennej w~$\phi$ przez tautologię $(x_i\vee\neg x_i)$.
W~przeciwnym przypadku musi być $x_i=0$, zatem wystąpienia $x_i$ w~$\phi$ zamieniamy w~formułę niespełnialną $(x_i\wedge\neg x_i)$.
Wykonując taki test dla każdej zmiennej, dostajemy jedno ze spełniających wartościowań dla $\phi$.
Po wykonaniu pętli otrzymana formuła będzie składać się z~dwukrotnie większej liczby literałów, co $\phi$.

Ponieważ dysponujemy wielomianowym algorytmem rozstrzygania spełnialności formuł, a~wejście zwiększy się tylko dwukrotnie, to opisany algorytm jest również wielomianowy.

\exercise %34.4-7
\note{Rozwiązanie ukaże się po publikacji części VI.}

\subchapter{Problemy \singledash{\NPclass}{zupełne}}

\exercise %34.5-1
Rozpocznijmy od formalnego sformułowania problemu izomorfizmu podgrafu za pomocą następującego języka formalnego 
\[
	\text{SUBGRAPH-ISOMORPHISM} = \bigl\{\,\langle G_1,G_2\rangle:\text{graf $G_1$ jest izomorficzny z~podgrafem grafu $G_2$}\,\bigr\}
\]
oraz przyjęcia oznaczeń $G_1=\langle V_1,E_1\rangle$ i~$G_2=\langle V_2,E_2\rangle$.

W~pierwszej części dowodu wykażemy, że $\text{SUBGRAPH-ISOMORPHISM}\in\NPclass$, przy czym jako świadectwa użyjemy bijekcji $f$ przekształcającej zbiór $V_1$ w~pewien podzbiór $V'\subseteq V_2$.
Świadectwo potrafimy zweryfikować w~czasie wielomianowym, sprawdzając, czy dla wszystkich $u$, $v\in V_1$ zachodzi $\langle u,v\rangle\in E_1$ wtedy i~tylko wtedy, gdy $\langle f(u),f(v)\rangle\in E_2$.

Wystarczy jeszcze udowodnić \singledash{\NPclass}{trudność} problemu izomorfizmu podgrafu.
Wykorzystamy w~tym celu redukcję problemu \text{CLIQUE}.
Dla egzemplarza $\langle G,k\rangle$ problemu kliki wystarczy zwrócić $\langle K_k,G\rangle$ jako egzemplarz problemu izomorfizmu podgrafu, gdzie $K_k$ oznacza graf pełny o~$k$ wierzchołkach.
Jeśli $G$ ma klikę rozmiaru $k$, to jako podgraf $G$ jest ona oczywiście izomorficzna z~$K_k$.
W~drugą stronę, jeśli $K_k$ jest izomorficzne z~pewnym podgrafem $G$, to tym podgrafem musi być graf pełny o~$k$ wierzchołkach, czyli klika rozmiaru $k$.

\exercise %34.5-2
\exercise %34.5-3
Problem \singledash{zero}{jedynkowego} programowania całkowitoliczbowego jest szczególnym przypadkiem dla problemu programowania liniowego całkowitoliczbowego.
Ponieważ pierwszy z~nich jest \singledash{\NPclass}{trudny}, to drugi z~nich również.
Wystarczy jeszcze pokazać, że ogólny problem jest w~\NPclass.
Jako świadectwa użyjemy wektora $x$, który można zweryfikować poprzez obliczenie wektora $Ax$ i~porównanie go z~wektorem $b$, co z~łatwością można wykonać w~czasie wielomianowym.

\exercise %34.5-4
\exercise %34.5-5
\exercise %34.5-6
\exercise %34.5-7
% Zajmijmy się decyzyjną wersją problemu.
% W~tym celu oprócz grafu mamy liczbę całkowitą $k$ i~pytamy, czy w~grafie istnieje cykl prosty o~długości $k$, ale nie istnieje taki cykl o~długości $k+1$.
% Problem jest w~\NPclass\ -- mając świadectwo w~postaci ciągu wierzchołków grafu, potrafimy w~czasie wielomianowym sprawdzić, czy każde dwa sąsiednie i~ostatni z~pierwszym są sąsiednie.

\exercise %34.5-8

\problems

\problem{Zbiór niezależny} %34-1

\subproblem %34-1(a)
\subproblem %34-1(b)
\subproblem %34-1(c)
\subproblem %34-1(d)

\problem{Bonnie i~Clyde} %34-2

\subproblem %34-2(a)
\subproblem %34-2(b)
\subproblem %34-2(c)
\subproblem %34-2(d)

\problem{Kolorowanie grafu} %34-3

\subproblem %34-3(a)
Przechodzimy wszerz graf $G$, rozpoczynając od dowolnego jego wierzchołka jako źródła.
Wierzchołki kolorujemy zgodnie z~parzystością ich odległości od źródła.
Jeśli sąsiad przetwarzanego wierzchołka ma już przypisany kolor, którym miał być pokolorowany ten wierzchołek, to znaczy, że \singledash{2}{kolorowanie} grafu nie istnieje.
Algorytm działa w~czasie $O(V+E)$.

\subproblem %34-3(b)
\[
	\text{COLOR} = \bigl\{\,\langle G,k\rangle:\text{$k\ge0$ jest liczbą całkowitą oraz graf $G$ jest \singledash{$k$}{kolorowalny}}\,\bigr\}
\]

Pierwsza implikacja jest oczywista -- jeśli pewne \singledash{$k$}{kolorowanie} grafu $G$ jesteśmy w~stanie otrzymać w~czasie wielomianowym, to jednocześnie w~czasie wielomianowym rozwiązujemy problem COLOR.

% Na odwrót, jeśli problem decyzyjny jest w~\Pclass, to \singledash{$k$}{kolorowanie} grafu możemy znaleźć w~następujący sposób.
% Budujemy graf $G$, rozpoczynając od pustego zbioru wierzchołków i~krawędzi, w~kolejnej iteracji dodając kolejny wierzchołek grafu $G$ i~krawędzie incydentne z~nim i~z~wierzchołkami dodanymi wcześniej.
% W~każdym kroku pamiętamy też bieżące $k$, na początku równe 0.
% Po dodaniu nowego wierzchołka ustalamy jego kolor, sprawdzając, czy bieżący graf jest \singledash{$k$}{kolorowalny}.

\subproblem %34-3(c)
Oba problemy są w~\NPclass, dla świadectwa będącego listą kolorów wierzchołków sprawdzamy, czy kolorowanie to jest poprawne, tzn.\ czy każde dwa sąsiednie wierzchołki mają różne kolory.

Na mocy tego, że problem \singledash{3}{COLOR} jest szczególnym przypadkiem problemu COLOR, to z~\singledash{\NPclass}{trudności} ostatniego wynika \singledash{\NPclass}{trudność} pierwszego.

\subproblem %34-3(d)
\subproblem %34-3(e)
\subproblem %34-3(f)

\problem{Szeregowanie zadań z~premiami za dotrzymanie terminu} %34-4

\subproblem %34-4(a)
\subproblem %34-4(b)
\subproblem %34-4(c)
\subproblem %34-4(d)

\endinput

%\chapter{Algorytmy aproksymacyjne}

\makeatletter
\def\input@path{{chapter35/}}
\makeatother

\subchapter{Problem pokrycia wierzchołkowego}

\exercise %35.1-1
\exercise %35.1-2
\exercise %35.1-3
\exercise %35.1-4
\exercise %35.1-5

\subchapter{Problem komiwojażera}

\exercise %35.2-1
\exercise %35.2-2
\exercise %35.2-3
\exercise %35.2-4
\exercise %35.2-5

\subchapter{Problem pokrycia zbioru}

\exercise %35.3-1
\exercise %35.3-2
\exercise %35.3-3
\exercise %35.3-4
\exercise %35.3-5

\subchapter{Randomizacja i~programowanie liniowe}

\exercise %35.4-1
\exercise %35.4-2
\exercise %35.4-3
\exercise %35.4-4

\subchapter{Problem sumy podzbioru}

\exercise %35.5-1
\exercise %35.5-2
\exercise %35.5-3
\exercise %35.5-4


\problems

\problem{Problem pakowania} %35-1

\subproblem %35-1(a)
\subproblem %35-1(b)
\subproblem %35-1(c)
\subproblem %35-1(d)
\subproblem %35-1(e)
\subproblem %35-1(f)

\problem{Aproksymacja rozmiaru maksymalnej kliki} %35-2

\subproblem %35-2(a)
\subproblem %35-2(b)

\problem{Problem ważonego pokrycia zbioru} %35-3

\problem{Największe skojarzenie} %35-4

\subproblem %35-4(a)
\subproblem %35-4(b)
\subproblem %35-4(c)
\subproblem %35-4(d)
\subproblem %35-4(e)
\subproblem %35-4(f)

\problem{Szeregowanie zadań dla wielu procesorów} %35-5

\subproblem %35-5(a)
\subproblem %35-5(b)
\subproblem %35-5(c)
\subproblem %35-5(d)



\setcounter{part}{7}
\part{Dodatek: Podstawy matematyczne}

\appendix
\makeatletter
	\@appendixtrue
\makeatother

\chapter{Sumy}

\subchapter{Wzory i~własności dotyczące sum}

\exercise %A.1-1
\[
	\sum_{k=1}^n(2k-1) = 2\sum_{k=1}^nk-\sum_{k=1}^n1 = \frac{2n(n+1)}{2}-n = n^2.
\]

\exercise %A.1-2
Korzystając z~oszacowania na $H_n$ (wzór~(A.7)), mamy
\begin{align*}
	\sum_{k=1}^n\frac{1}{2k-1} &= H_{2n-1}-\sum_{k=1}^{n-1}\frac{1}{2k} \\
	&= H_{2n-1}-\frac{1}{2}H_{n-1} \\
	&= \ln (2n-1)+O(1)-\left(\frac{1}{2}\ln(n-1)+O(1)\right) \\
	&= \ln (2n-2)+\underbrace{\ln\frac{2n-1}{2n-2}}_{O(1)}-\ln\sqrt{n-1}+O(1) \\
	&= \ln\frac{2n-2}{\sqrt{n-1}}+O(1) \\[1mm]
	&= \ln\left(2\sqrt{n-1}\right)+O(1) \\
	&= \ln 2+\ln\sqrt{n}+\underbrace{\ln\frac{\sqrt{n-1}}{\sqrt{n}}}_{O(1)}+\,O(1) \\
	&= \ln\sqrt{n}+O(1).
\end{align*}

\exercise %A.1-3
\[
	\sum_{k=0}^\infty k^2x^k = x\frac{d}{dx}\sum_{k=0}^\infty kx^k = \frac{x(x+1)}{(1-x)^3}.
\]
W~ostatniej równości wykorzystano wzór~(A.8).

\exercise %A.1-4
\begin{align*}
	\sum_{k=0}^\infty\frac{k-1}{2^k} &= \sum_{k=0}^\infty\frac{k}{2^k}-\sum_{k=0}^\infty\frac{1}{2^k} \\
	&= \frac{\frac{1}{2}}{\left(1-\frac{1}{2}\right)^2}+\sum_{k=0}^\infty\left(\frac{1}{2}\right)^k \\
	&= 2-\frac{1}{1-\frac{1}{2}} \\
	&= 0.
\end{align*}
Skorzystano z~wzoru~(A.8), a~następnie~(A.6), w~obu przypadkach biorąc $x=1/2$.

\exercise %A.1-5
\begin{align*}
	\sum_{k=1}^\infty(2k+1)x^{2k} &= 2\sum_{k=0}^\infty kx^{2k}+\sum_{k=0}^\infty x^{2k}-1 \\
	&= \frac{2x^2}{(1-x^2)^2}+\frac{1}{1-x^2}-1 \\[1mm]
	&= \frac{x^2(3-x^2)}{(1-x^2)^2}.
\end{align*}
W~drugiej równości zastosowano wzory~(A.6) i~(A.8). Otrzymany wynik jest wartością sumy, gdy $|x^2|<1$. W~pozostałych przypadkach suma jest rozbieżna do $\infty$.

\exercise %A.1-6
Zgodnie z~definicją notacji $O$, dla dowolnej funkcji $F$ zachodzi $F(n) = O\bigl(\sum_{k=1}^nf_k(n)\bigr)$ wtedy i~tylko wtedy, gdy istnieją stałe $n_0$, $c>0$ takie, że dla każdego $n\ge n_0$ prawdą jest
\[
	0\le F(n)\le c\sum_{k=1}^n f_k(n).
\]

Niech $(F_i(n))_{1\le i\le n}$ będzie ciągiem funkcji, który spełnia układ nierówności
\[
	\begin{cases}
		0 \le F_1(n) \le cf_1(n) \\
		0 \le F_2(n) \le cf_1(n)+cf_2(n) \\
		\phantom{0 \le F_2} \vdots \\
		0 \le F_n(n) \le \sum_{k=1}^ncf_k(n), \\
	\end{cases} \tag{$*$}\label{eq:A.1-6}
\]
dla pewnych dodatnich stałych $n_0$ i~$c$ oraz dla każdego $n\ge n_0$. Wprost z~definicji notacji $O$ mamy
\begin{align*}
	F_1(n) &= O(f_1(n)), \\
	F_2(n)-F_1(n) &= O(f_2(n)), \\
	& \,\,\,\vdots \\
	F_n(n)-F_{n-1}(n) &= O(f_n(n)).
\end{align*}
Dodając powyższe równania stronami, otrzymujemy
\[
	F_1(n)+(F_2(n)-F_1(n))+\cdots+(F_n(n)-F_{n-1}(n)) = \sum_{k=1}^nO(f_k(n)).
\]
Wartość po lewej stronie jest równa $F_n(n)$. Ale z~ostatniej nierówności układu~(\ref{eq:A.1-6}) wynika $F_n(n)=O\bigl(\sum_{k=1}^nf_k(n)\bigr)$, a~stąd wynika tożsamość
\[
	\sum_{k=1}^nO(f_k(n)) = O\biggl(\sum_{k=1}^nf_k(n)\biggr).
\]

\exercise %A.1-7
\[
	\prod_{k=1}^n2\cdot 4^k = 2^n\cdot 4^{1+2+\cdots+n} = 2^n\cdot 4^{\frac{n(n+1)}{2}} = 2^{n(n+2)}.
\]

\exercise %A.1-8
\begin{align*}
	\prod_{k=2}^n\left(1-\frac{1}{k^2}\right) &= \prod_{k=2}^n\frac{k^2-1}{k^2} \\[2mm]
	&= \frac{\prod_{k=2}^n(k^2-1)}{\prod_{k=2}^nk^2} \\[2mm]
	&= \frac{\prod_{k=2}^n(k-1)\cdot\prod_{k=2}^n(k+1)}{\prod_{k=1}^nk\cdot\prod_{k=1}^nk} \\[2mm]
	&= \frac{\prod_{k=1}^{n-1}k\cdot\prod_{k=3}^{n+1}k}{(n!)^2} \\[2mm]
	&= \frac{(n-1)!\cdot\frac{(n+1)!}{2}}{(n!)^2} \\[2mm]
	&= \frac{n(n+1)((n-1)!)^2}{2n^2((n-1)!)^2} \\[2mm]
	&= \frac{n+1}{2n}.
\end{align*}

\subchapter{Szacowanie sum}

\exercise %A.2-1
Wykorzystując własności szeregu teleskopowego, dostajemy:
\begin{align*}
	\sum_{k=1}^n\frac{1}{k^2} &\le 1+\sum_{k=2}^n\frac{1}{k(k-1)} \\
	&= 1+\sum_{k=2}^n\left(\frac{1}{k-1}-\frac{1}{k}\right) \\
	&= 1+1-\frac{1}{n} \\
	&< 2,
\end{align*}
a~zatem badana suma jest ograniczona przez stałą. Okazuje się, że dla coraz większych $n$, jej wartość zbliża się do liczby $\pi^2\!/6$.

\exercise %A.2-2
Zauważmy, że gdy $n$ osiąga wartość będącą potęgą 2, to zwiększa się o~1 liczba sumowanych składników. Funkcja $F(n)=\sum_{k=0}^{\lfloor\lg n\rfloor}\left\lceil n/2^k\right\rceil$ jest rosnąca w~dziedzinie liczb naturalnych, więc dla pewnego $m$ całkowitego, $F(n)<F(2^m)$ dla wszystkich $n<2^m$. Oszacowaniem górnym tejże funkcji będzie zatem oszacowanie górne jej wartości przyjmowanych dla potęg 2. Dla $n=2^m$ zachodzi
\[
	F(n) = F(2^m) = \sum_{k=0}^m\left\lceil\frac{2^m}{2^k}\right\rceil = 2^m+2^{m-1}+\cdots+2^0 = 2^{m+1}-1 = 2n-1,
\]
z~czego wynika, że $F(n)=O(n)$ i~asymptotycznym górnym ograniczeniem sumy jest $O(n)$.

\exercise %A.2-3
Załóżmy, że $H_n$ rozwinęliśmy do sumy ułamków od $1/1$ do $1/n$. Ustawmy wyrazy tej sumy malejąco i~podzielmy je w~grupy w~taki sposób, by rozmiary kolejnych grup (być może z~wyjątkiem ostatniej) były kolejnymi potęgami~2. Umieszczamy pierwszy składnik w~grupie 1 o~rozmiarze~1, dwa kolejne w~grupie~2 o~rozmiarze~2, cztery kolejne w~grupie~3 o~rozmiarze~4, itd.\ aż do grupy~$s$, jak to widać poniżej:
\[
	\underbrace{\frac{1}{1}}_{\text{\scriptsize grupa 1}}\!\!+\;\;\underbrace{\frac{1}{2}+\frac{1}{3}}_{\text{\scriptsize grupa 2}}\;\;+\;\;\underbrace{\frac{1}{4}+\frac{1}{5}+\frac{1}{6}+\frac{1}{7}}_{\text{\scriptsize grupa 3}}\;\;+\cdots+\;\;\underbrace{\cdots +\frac{1}{n}}_{\text{\scriptsize grupa $s$}}.
\]
Wszystkie składniki w~grupie~2 przyjmują wartości od $1/3$ do~$1/2$, zatem suma składników tejże grupy mieści się pomiędzy $2\cdot1/3=2/3$ a~$2\cdot1/2=1$. Wszystkie składniki w~grupie~3 przyjmują wartości od $1/7$ do~$1/4$, więc ich suma jest między $4/7$ a~1. W~ogólności, każdy z~$2^{k-1}$ składników w~grupie~$k$ jest zawarty w~przedziale $\bigl(2^{-k},2^{1-k}\,\bigr]$, a~stąd suma składników w~każdej grupie znajduje się w~przedziale $(1/2,1\,]$. W~końcu, jeśli umieściliśmy składnik $1/n$ w~grupie $s = \lfloor\lg n\rfloor+1$, to zachodzi $s/2<H_n\le s$. Zatem $\lim_{n\to\infty}H_n=\infty$ i~istotnie
\[
	\frac{\lfloor\lg n\rfloor+1}{2} < H_n \le \lfloor\lg n\rfloor+1, \quad\text{więc $H_n=\Theta(\lg n)$}.
\]

\exercise %A.2-4
Funkcja $f(k)=k^3$ jest monotonicznie rosnąca, zatem rozdzielamy sumę za pomocą nierówności~(A.11):
\[
	\int_0^nx^3\,dx \le \sum_{k=1}^nk^3 \le \int_1^{n+1}x^3\,dx.
\]
Znajdujemy oszacowania obu całek:
\begin{gather*}
	\int_0^nx^3\,dx = \frac{n^4}{4} = \Theta(n^4), \\[2mm]
	\int_1^{n+1}x^3\,dx = \frac{(n+1)^4-1}{4} = \Theta(n^4),
\end{gather*}
a~zatem
\[
	\sum_{k=0}^nk^3 = \Theta(n^4).
\]

\exercise %A.2-5
Zastosowanie nierówności~(A.12) do $\sum_{k=1}^n(1/k)$ doprowadza do wyrażenia niezdefiniowanego
\[
	\sum_{k=1}^n\frac{1}{k} \le \int_0^n\frac{dx}{x} = \ln n-\ln0,
\]
ponieważ wartości $\ln0$ nie można wyrazić liczbą rzeczywistą. Potraktowanie sumy $\sum_{k=1}^n(1/k)$ jako $1+\sum_{k=2}^n(1/k)$ pozwala na zastosowanie wzoru~(A.12) do drugiego składnika wprowadzając jedynie stałą różnicę.

\problems

\problem{Szacowanie sum} %A-1
Wszystkie następujące funkcje: $f(k)=k^r$, $g(k)=\lg^sk$, $h(k)=k^r\lg^sk$, są niemalejące dla każdych stałych $r\ge0$ i~$s\ge0$. Pozwala to na wykorzystanie nierówności~(A.11) we wszystkich poniższych punktach.

\subproblem %A-1(a)
Z~wzoru~(A.11) dostajemy
\[
	\int_0^nx^r\,dx \le \sum_{k=1}^nk^r \le \int_1^{n+1}x^r\,dx.
\]
Wartości obu całek wynoszą
\begin{gather*}
	\int_0^nx^r\,dx = \frac{n^{r+1}}{r+1} = \Theta(n^{r+1}), \\[2mm]
	\int_1^{n+1}x^r\,dx = \frac{(n+1)^{r+1}-1}{r+1} = \Theta(n^{r+1}),
\end{gather*}
zatem oszacowaniem sumy jest
\[
	\sum_{k=1}^nk^r = \Theta(n^{r+1}).
\]

\subproblem %A-1(b)
Stosujemy wzór~(A.11) do sumy $\sum_{k=2}^n\lg^sk$, która jest równoważna sumie badanej:
\[
	\int_1^n\lg^sx\,dx \le \sum_{k=2}^n\lg^sk \le \int_2^{n+1}\lg^sx\,dx.
\]
Obliczmy całkę nieoznaczoną $I(x) = \int\lg^sx\,dx$ metodą całkowania przez części. Niech $u(x)=\lg^sx$ oraz $dv(x)/dx=dx$. Pomijając stałe całkowania otrzymujemy
\[
	\frac{du(x)}{dx} = \frac{s\lg^{s-1}x\,dx}{x\ln 2} = s\lg e\cdot\frac{\lg^{s-1}x\,dx}{x} \quad\text{oraz}\quad v(x) = x,
\]
a~następnie obliczamy
\begin{align*}
	I(x) &= u(x)v(x)-\int\frac{du(x)}{dx}v(x), \\
	I(x) &= x\lg^sx-\int s\lg e\lg^{s-1}x\,dx, \\
	I(x) &= x\lg^sx-\frac{s\lg e}{\lg x}\cdot I(x), \\
	I(x) &= \frac{x\lg^{s+1}x}{\lg x+s\lg e}.
\end{align*}
Wykorzystujemy otrzymany wynik do obliczenia całek ograniczających badaną sumę:
\begin{gather*}
	{\int_1^n\lg^sx\,dx} = \frac{n\lg^{s+1}n}{\lg n+s\lg e} = \Theta(n\lg^sn), \\[2mm]
	{\int_2^{n+1}\lg^sx\,dx} = \frac{(n+1)\lg^{s+1}(n+1)}{\lg(n+1)+s\lg e}-\dfrac{2}{1+s\lg e} = \Theta(n\lg^sn),
\end{gather*}
ponieważ wyrażenie $s\lg e$ jest stałe, a~zatem oszacowaniem sumy jest
\[
	\sum_{k=1}^n\lg^sk = \Theta(n\lg^sn).
\]

\subproblem %A-1(c)
Po zastosowaniu wzoru~(A.11) do sumy $\sum_{k=2}^nk^r\lg^sk$, dostajemy
\[
	\int_1^nx^r\lg^sx\,dx \le \sum_{k=2}^nk^r\lg^sk \le \int_2^{n+1}x^r\lg^sx\,dx.
\]
Analogicznie jak w~poprzednim punkcie, obliczmy całkę $I(x) = \int x^r\lg^sx\,dx$ przez części. Niech $u(x)=\lg^sx$ oraz $dv(x)/dx=x^r\,dx$, a~stąd (po zignorowaniu stałych całkowania)
\[
	\frac{du(x)}{dx} = s\lg e\cdot\frac{\lg^{s-1}x\,dx}{x} \quad\text{oraz}\quad v(x) = \frac{x^{r+1}}{r+1},
\]
a~więc mamy
\begin{align*}
	I(x) &= u(x)v(x)-\int\frac{du(x)}{dx}v(x), \\
	I(x) &= \frac{x^{r+1}\lg^sx}{r+1}-\frac{s\lg e}{(r+1)\lg x}\cdot I(x), \\
	I(x) &= \frac{x^{r+1}\lg^{s+1}x}{(r+1)\lg x+s\lg e}.
\end{align*}
Całki ograniczające badaną sumę wynoszą zatem
\begin{gather*}
	{\int_1^nx^r\lg^sx\,dx} = \frac{n^{r+1}\lg^{s+1}n}{(r+1)\lg n+s\lg e} = \Theta(n^{r+1}\lg^sn), \\[2mm]
	{\int_2^{n+1}x^r\lg^sx\,dx} = \frac{(n+1)^{r+1}\lg^{s+1}(n+1)}{(r+1)\lg(n+1)+s\lg e}-\dfrac{2^{r+1}}{r+1+s\lg e} = \Theta(n^{r+1}\lg^sn).
\end{gather*}
Stąd wnioskujemy, że
\[
	\sum_{k=1}^nk^r\lg^sk = \Theta(n^{r+1}\lg^sn).
\]

Przyjmując w~powyższym oszacowaniu odpowiednio $s=0$ i~$r=0$, otrzymamy sumy i~ich oszacowania z~punktów (a) i~(b).

\endinput

\chapter{Zbiory i~nie tylko}

\subchapter{Zbiory}

\exercise %B.1-1
\begin{figure}[ht]
	\begin{center}
		\includegraphics{figb.1}
	\end{center}
	\caption{Diagramy Venna ilustrujące pierwsze prawo rozdzielności.}
\end{figure}

\exercise %B.1-2
Przeprowadzimy dowód dla dopełnienia przecięcia zbiorów przez indukcję względem $n$. Dla $n=1$ dowód jest trywialny, a~dla $n=2$ wzór jest podany w~podręczniku. Załóżmy więc, że $n>2$ i~że twierdzenie zachodzi dla rodziny $n-1$ zbiorów. Mamy
\begin{align*}
	\overline{A_1\cap A_2\cap\cdots\cap A_{n-1}\cap A_n} &= \overline{(A_1\cap A_2\cap\cdots\cap A_{n-1})\cap A_n} \\
	&= \overline{A_1\cap A_2\cap\cdots\cap A_{n-1}}\cup\overline{A_n} \\
	&= \overline{A_1}\cup\overline{A_2}\cup\cdots\cup\overline{A_{n-1}}\cup\overline{A_n}.
\end{align*}
W~drugiej równości skorzystano z~prawa de~Morgana dla dopełnienia przecięcia dwóch zbiorów, a~w~trzeciej -- z~założenia indukcyjnego.

Dla dopełnienia sumy zbiorów dowód przebiega analogicznie -- wystarczy tylko w~powyższym wyprowadzeniu zamienić ze~sobą symbole sumy i~przecięcia zbiorów.

\exercise %B.1-3
Udowodnimy zasadę włączania i~wyłączania przez indukcję względem liczby zbiorów $n$. Dla $n=1$ dowód jest trywialny, a~dla $n=2$ wynika z~wzoru~(B.3). Na mocy tego wzoru dla $n>2$ mamy
\begin{align*}
    |A_1\cup A_2\cup\dots\cup A_n| &= \bigl|(A_1\cup A_2\cup\dots\cup A_{n-1})\cup A_n\bigr| \\
	&= |A_1\cup A_2\cup\dots\cup A_{n-1}|+|A_n|-\bigl|(A_1\cup A_2\cup\dots\cup A_{n-1})\cap A_n\bigr| \\
	&= |A_1\cup A_2\cup\dots\cup A_{n-1}|+|A_n|-\bigl|(A_1\cap A_n)\cup\dots\cup(A_{n-1}\cap A_n)\bigr|.
\end{align*}
Stosujemy teraz założenie indukcyjne do pierwszego i~ostatniego składnika sumy, w~wyniku czego otrzymujemy
\begin{align*}
	|A_1\cup A_2\cup\dots\cup A_{n-1}| &= \sum_{1\le i_1<n}|A_{i_1}|-\sum_{1\le i_1<i_2<n}|A_{i_1}\cap A_{i_2}|+\sum_{1\le i_1<i_2<i_3<n}|A_{i_1}\cap A_{i_2}\cap A_{i_3}| \\[1mm]
	&\quad {}-\dots+(-1)^{n-2}|A_1\cap A_2\cap\dots\cap A_{n-1}|
\end{align*}
oraz
\begin{align*}
	\bigl|(A_1\cap A_n)\cup\dots\cup(A_{n-1}\cap A_n)\bigr| &= \sum_{1\le i_1<n}|A_{i_1}\cap A_n|-\sum_{1\le i_1<i_2<n}|A_{i_1}\cap A_{i_2}\cap A_n|\\[1mm]
	&\quad {}+\dots+(-1)^{n-1}|A_1\cap A_2\cap\dots\cap A_{n-1}\cap A_n|.
\end{align*}
Wystarczy wstawić otrzymane wyrażenia do początkowego wzoru, który przyjmuje postać
\begin{align*}
	|A_1\cup A_2\cup\dots\cup A_n| &= \sum_{1\le i_1\le n}|A_{i_1}|-\sum_{1\le i_1<i_2\le n}|A_{i_1}\cap A_{i_2}|+\sum_{1\le i_1<i_2<i_3\le n}|A_{i_1}\cap A_{i_2}\cap A_{i_3}| \\[1mm]
	&\quad {}-\dots+(-1)^{n-1}|A_1\cap A_2\cap\dots\cap A_n|,
\end{align*}
a~zatem zasada zachodzi dla dowolnej skończonej ilości zbiorów.

\exercise %B.1-4
\note{Poniższe rozwiązanie dowodzi przeliczalności zbioru nieparzystych liczb naturalnych, czego dotyczy oryginalna treść zadania. Tłumaczenie pyta natomiast o~przeliczalność zbioru wszystkich liczb nieparzystych.}

\noindent Aby wykazać ten fakt, należy znaleźć wzajemnie jednoznaczne odwzorowanie ze zbioru $\mathbb{N}$ w~zbiór $\bigl\{\,2k+1:k\in\mathbb{N}\,\bigr\}$. Zauważmy, że sposób odwzorowania wynika wprost z~zapisu tego ostatniego zbioru -- danej liczbie naturalnej $k$ wystarczy przyporządkować liczbę $2k+1$. Wnioskujemy zatem, że zbiór nieparzystych liczb naturalnych jest przeliczalny.

\exercise %B.1-5
Dowód przez indukcję względem liczby elementów $S$. Jeśli $|S|=0$, czyli $S=\emptyset$, to $|2^S|=2^{|S|}=1$, bo $S$ ma tylko jeden podzbiór -- zbiór pusty. Załóżmy teraz, że $|S|>0$ oraz że $|2^S|=2^{|S|}$. Niech $p\not\in S$ i~rozważmy zbiór $S'=S\cup\{p\}$. Zauważmy, że podzbiory zbioru $S'$ można podzielić na takie, które zawierają $p$ i~na takie, które nie zawierają $p$. Tych ostatnich jest $|2^S|=\bigl|2^{S'\setminus\{p\}}\bigr|=2^{|S'\setminus\{p\}|}$ (z założenia indukcyjnego). Okazuje się, że podzbiorów zawierających $p$ jest tyle samo, ponieważ każdy powstaje przez zsumowanie singletonu $\{p\}$ z~pewnym z~$2^{|S'\setminus\{p\}|}$ podzbiorów niezawierających $p$. Mamy zatem
\[
	\bigl|2^{S'}\bigr|=2\cdot2^{|S'\setminus\{p\}|} = 2^{|S'\setminus\{p\}|+1} = 2^{|S'|}.
\]
Na mocy indukcji twierdzenie jest spełnione dla dowolnego zbioru skończonego.

\exercise %B.1-6
\[
	\langle a_1,a_2,\dots,a_n\rangle =
	\begin{cases}
		\emptyset, & \text{jeśli $n=0$}, \\
		\{a_1\}, & \text{jeśli $n=1$}, \\
		\{a_1,\{a_1,a_2\}\}, & \text{jeśli $n=2$}, \\
		\langle a_1,\langle a_2,\dots,a_n\rangle\rangle, & \text{jeśli $n\ge3$}.
	\end{cases}
\]

\subchapter{Relacje}

\exercise %B.2-1
Porządek częściowy jest relacją zwrotną, antysymetryczną i~przechodnią. Zauważmy, że relacja $\subseteq$ w~$2^\mathbb{Z}$ posiada każdą z~tych cech. Dla zbioru $A\in2^\mathbb{Z}$ zachodzi $A\subseteq A$ (zwrotność). Dla zbiorów $A$, $B\in2^\mathbb{Z}$, jeśli spełniają one $A\subseteq B$ i~$B\subseteq A$, to zachodzi $A=B$ (antysymetria). Wreszcie, dla zbiorów $A$, $B$, $C\in2^\mathbb{Z}$, jeżeli $A\subseteq B$ i~$B\subseteq C$, to $A\subseteq C$ (przechodniość). Jednak w~$2^\mathbb{Z}$ porządek $\subseteq$ nie jest liniowy, bo np.\ $\{0,1\}\not\subseteq\{1,2\}$ i~$\{1,2\}\not\subseteq\{0,1\}$.

\exercise %B.2-2
Oznaczmy przez $R_n$ dla $n\in\mathbb{N}\setminus\{0\}$ relację ``przystaje modulo $n$'':
\[
	R_n = \bigl\{\,\langle a,b\rangle\in\mathbb{Z}\times\mathbb{Z}\;:\;a\equiv b\!\!\!\pmod{n}\,\bigr\}.
\]

Dla dowolnego $a\in\mathbb{Z}$ mamy $a\equiv a\pmod{n}$, bo $a-a=0$, więc relacja $R_n$ jest zwrotna. Dla dowolnych $a$, $b\in\mathbb{Z}$, jeśli istnieje $q\in\mathbb{Z}$, że $a-b=qn$, to $b-a=-qn$, a~zatem z~faktu, że $a\equiv b\pmod{n}$ wynika, że $b\equiv a\pmod{n}$, co dowodzi symetrii $R_n$. Dla dowodu przechodniości wybierzmy dowolne $a$, $b$, $c\in\mathbb{Z}$ i~załóżmy, że zachodzi $a\equiv b\pmod{n}$ oraz $b\equiv c\pmod{n}$. Oznacza to, że istnieją $q$, $r\in\mathbb{Z}$, że $a-b=qn$ oraz $b-c=rn$. Stąd $a-c=a-b+b-c=qn+rn=(q+r)n$, a~zatem $a\equiv c\pmod{n}$.

Na mocy powyższych faktów $R_n$ jest relacją równoważności i~dzieli zbiór $\mathbb{Z}$ na $n$ klas abstrakcji; \compound{$i$}{ta} klasa, gdzie $1\le i\le n$, jest zbiorem takich liczb całkowitych, które przy dzieleniu przez $n$ dają resztę $i-1$.

\exercise %B.2-3
\subexercise
Relacja $R=\bigl\{\langle a,a\rangle,\langle b,b\rangle,\langle c,c\rangle,\langle a,b\rangle,\langle b,a\rangle,\langle a,c\rangle,\langle c,a\rangle\bigr\}$ określona w~zbiorze $\{a,b,c\}$.

\subexercise
Relacja $\le$ określona w~zbiorze liczb rzeczywistych.

\subexercise
Relacja $T=\bigl\{\langle a,a\rangle,\langle b,b\rangle,\langle a,b\rangle,\langle b,a\rangle\bigr\}$ określona w~zbiorze $\{a,b,c\}$.

\exercise %B.2-4
Jeśli $R$ jest relacją równoważności, to dla każdego $s\in S$ zachodzi $s\in[s]$. Na mocy antysymetrii $R$, jeśli zachodzi $s'\,R\,s$ oraz $s\,R\,s'$, to $s=s'$, a~więc nie istnieją takie elementy $s'$, że $s'\in[s]\setminus\{s\}$, zatem klasy abstrakcji są singletonami.

\exercise %B.2-5
W~definicji symetrii i~przechodniości relacji mamy implikacje. Aby były one prawdziwe, nie muszą być spełnione ich poprzedniki. Zwrotność wymaga natomiast, aby każdy element ze zbioru, w~którym określamy relację, był ze sobą w~relacji. Istnieją zatem relacje symetryczne i~przechodnie, ale nie zwrotne. Jednym z~przykładów jest relacja z~punktu~(c) w~\refExercise{B.2-3}.

\subchapter{Funkcje}

\exercise %B.3-1
\subexercise
Zbiór wartości funkcji $f\colon A\to B$, czyli obraz jej dziedziny, jest zdefiniowany następująco:
\[
	f(A) = \bigl\{\,b\in B:b=f(a)\text{ dla pewnego $a\in A$}\,\bigr\}.
\]
Z~tego, że $f$ jest injekcją, mamy, że $|A|=|f(A)|$. Z~kolei $|f(A)|\le|B|$, bo w~$B$ mogą być takie elementy $b$, dla których nie istnieje $a\in A$ takie, że $b=f(a)$. Stąd $|A|\le|B|$.

\subexercise
Dla surjekcji $f\colon A\to B$ zachodzi $f(A)=B$, więc $|f(A)|=|B|$. Dla pewnych elementów $a_1$,~$a_2\in A$ może zachodzić $f(a_1)=f(a_2)$, mamy zatem $|A|\ge|f(A)|$, a~stąd $|A|\ge|B|$.
\bigskip

Zauważmy, że z~powyższych faktów wynika, że jeśli $f\colon A\to B$ jest bijekcją, to $|A|=|B|$.

\exercise %B.3-2
Funkcja $f$ o~dziedzinie i~przeciwdziedzinie $\mathbb{N}$ nie jest bijekcją, gdyż dla żadnego $x\in\mathbb{N}$ nie zachodzi $f(x)=0$. Jeśli zamiast $\mathbb{N}$ rozważamy $\mathbb{Z}$, to $f$ jest bijekcją -- każda liczba całkowita jest wartością $f$ w~pewnej jednoznacznie wyznaczonej liczbie całkowitej.

\exercise %B.3-3
Niech $R$ będzie relacją binarną w~zbiorze $A$. Relację $R^{-1}$ w~zbiorze $A$ nazywamy relacją odwrotną do $R$, jeżeli dla dowolnych $a$,~$b\in A$, $a\,R^{-1}\,b$ wtedy i~tylko wtedy, gdy $b\,R\,a$. Łatwo zauważyć, że jeśli $R$ jest bijekcją, to $R^{-1}$ jest jej funkcją odwrotną.

\exercise %B.3-4
Ponieważ każda bijekcja posiada funkcję odwrotną, która także jest bijekcją, to znajdziemy funkcję $F\colon\mathbb{Z}\times\mathbb{Z}\to\mathbb{Z}$ będącą odwrotnością szukanego odwzorowania. Wyznaczenie $F$ jest równoważne znalezieniu sposobu ponumerowania liczbami całkowitymi każdej pary o~elementach całkowitych tak, aby żadne dwie pary nie miały tego samego numeru i~żeby każda liczba całkowita była wykorzystana jako numer pewnej pary. Opiszemy teraz konstrukcję jednej z~takich funkcji.

Dokonajmy pewnego uproszczenia -- zamiast numerować pary liczbami całkowitymi, ograniczmy się do liczb naturalnych. Niech $g\colon\mathbb{Z}\times\mathbb{Z}\to\mathbb{N}$ oraz $h\colon\mathbb{N}\to\mathbb{Z}$ będą pewnymi bijekcjami oraz niech $F=h\circ g$. Łatwo wykazać, że $h(n)=(-1)^n\lceil n/2\rceil$ jest bijekcją, pozostaje więc jeszcze znaleźć funkcję $g$.

Rozważmy numerację par o~elementach całkowitych przedstawioną na rys.~\ref{fig:B.3-4} w~formie spirali.
\begin{figure}[ht]
	\begin{center}
		\includegraphics{figb.2}
	\end{center}
	\caption{Bijekcja ze zbioru $\mathbb{Z}\times\mathbb{Z}$ w~zbiór $\mathbb{N}$. Poszczególne liczby naturalne oznaczają wartości tej bijekcji dla punktów o~współrzędnych całkowitych w~układzie kartezjańskim.} \label{fig:B.3-4}
\end{figure}
Ponieważ każdemu punktowi o~współrzędnych całkowitych przypisywana jest unikalna liczba naturalna, to możemy tę spiralę potraktować jako opis funkcji~$g$. Przyjmijmy wpierw oznaczenia: $d=\max(|x|,|y|)$ oraz $D=(2d-1)^2-1$. Nieformalnie liczby te oznaczają, odpowiednio, numer ``okrążenia'' punktu $\langle0,0\rangle$ pokonywanego przez spiralę w~momencie przechodzenia przez punkt $\langle x,y\rangle$ oraz największą wartość przyjmowaną przez spiralę podczas pokonywania poprzedniego ``okrążenia''. Można teraz przyjąć następującą definicję funkcji~$g$:
\[
	g(x,y) =
	\begin{cases}
		0, & \text{jeśli $d=|x|=|y|=0$}, \\
		D+d+y, & \text{jeśli $d\ne|y|$ i~$d=x$}, \\
		D+3d-x, & \text{jeśli $d\ne0$ i~$d=y$}, \\
		D+5d-y, & \text{jeśli $d\ne|y|$ i~$d=-x$}, \\
		D+7d+x, & \text{jeśli $d\ne0$ i~$d=-y$}.
	\end{cases}
\]

Zaprezentowana tutaj spirala przypomina znaną w~literaturze \emph{spiralę Ulama}, opisaną w~\cite{ulamspiral} i~wykorzystywaną do znajdowania pewnych własności liczb pierwszych.

\subchapter{Grafy}

\exercise %B.4-1
Jeśli będziemy reprezentować zbiór pracowników przez zbiór wierzchołków $V$, a~dla każdych $u$, $v\in V$ relację ``pracownik $u$ podał rękę pracownikowi $v$'' przez zbiór krawędzi $E$, to otrzymamy graf nieskierowany $G=\langle V,E\rangle$. Sumując stopnie wszystkich wierzchołków tego grafu, otrzymamy podwojoną liczbę krawędzi, gdyż każdą krawędź policzymy dwa razy (każda krawędź jest incydentna z~dokładnie dwoma wierzchołkami). Mamy więc
\[
	\sum_{v\in V}\deg(v) = 2|E|.
\]

\exercise %B.4-2
Ścieżka z~wierzchołka $u$ do wierzchołka $v$ w~dowolnym grafie jest ciągiem wierzchołków kolejno odwiedzanych na tej ścieżce, $\langle v_0,v_1,\dots,v_n\rangle$, przy czym $v_0=u$ i~$v_n=v$. Jeśli ścieżka jest prosta, to wyrazy tego ciągu nie powtarzają się. W~każdej ścieżce można wyeliminować pewne spójne podciągi, otrzymując w~wyniku ścieżkę prostą. Jeśli podciągiem ścieżki z~$u$ do $v$ jest ciąg $\langle v_i,v_{i+1},\dots,v_{i+k},v_i\rangle$, to eliminując jego podciąg $\langle v_i,v_{i+1},\dots,v_{i+k}\rangle$, odrzucimy jedno powtórzenie $v_i$, a~tym samym podcykl ścieżki, który sprawia, że nie jest ona prosta. Po eliminacji wszystkich takich podcykli otrzymamy ścieżkę prostą. Oznacza to, że każda ścieżka zawiera ścieżkę prostą reprezentowaną przez ciąg wierzchołków pozostałych z~początkowego ciągu po eliminacji podcykli. Procedura wyznaczania ścieżki prostej jest poprawna, gdyż jeśli z~$v_i$ istnieje krawędź do $v_{i+k+1}$, to możemy wcześniej przejść tą krawędzią, zapobiegając ponownemu odwiedzeniu $v_i$.

Dowód dla cykli przeprowadzamy analogicznie z~$v_n=u$, pamiętając jednak, by nie eliminować ostatniego powtórzenia $u$, które jest wymagane do tego, by ścieżka stanowiła cykl.

\exercise %B.4-3
Z~twierdzenia~B.2 mamy, że graf $G=\langle V,E\rangle$ będący drzewem jest spójny i~acykliczny oraz że ma $|E|=|V|-1$ krawędzi. Gdy dodamy do $E$ nową krawędź, to $G$ nie będzie już drzewem, ale nadal będzie spójny -- może być zatem $|E|\ge|V|-1$. Z~kolei, gdy usuniemy z~$E$ jakąkolwiek krawędź, to rozspójnimy $G$, przez co nie może zachodzić $|E|<|V|-1$.

\exercise %B.4-4
Każdy wierzchołek grafu skierowanego lub nieskierowanego jest osiągalny z~samego siebie, ponieważ istnieje ścieżka o~długości równej 1 zawierająca tylko ten wierzchołek, zatem relacja osiągalności jest zwrotna.

Dla dowolnych wierzchołków $u$, $v$ i~$w$ grafu skierowanego lub nieskierowanego z~faktu, że $u\leadsto v$ i~$v\leadsto w$ wynika, że $u\leadsto w$. Istnieje bowiem ścieżka z~$u$ do $w$ będąca konkatenacją ciągów reprezentujących ścieżki z~$u$ do $v$ i~z~$v$ do $w$ (z~pominięciem powtórzenia $v$ między nimi).

Relacja osiągalności jest symetryczna jedynie w~grafach nieskierowanych, gdyż dla dowolnych wierzchołków $u$ i~$v$, jeśli $u\leadsto v$, to $v\leadsto u$. Ścieżka z~$v$ do $u$ powstaje przez lustrzane odbicie ścieżki z~$u$ do $v$; powstały ciąg reprezentuje poprawną ścieżkę, bo każdą krawędzią można poruszać się w~obie strony. W~grafie skierowanym krawędzie są jednokierunkowe, więc symetria nie zachodzi.

\exercise %B.4-5
\begin{figure}[ht]
	\begin{center}
		\includegraphics{figb.3}
	\end{center}
	\caption{{\sffamily\bfseries(a)} Wersja nieskierowana grafu skierowanego z~rysunku~B.2(a). {\sffamily\bfseries(b)} Wersja skierowana grafu nieskierowanego z~rysunku~B.2(b).} \label{fig:B.4-5}
\end{figure}

\exercise %B.4-6
Hipergraf $H=\langle V_H,E_H\rangle$ można reprezentować jako graf dwudzielny $G=\langle V_1\cup V_2,E\rangle$, w~którym $V_1=V_H$ oraz $V_2=E_H$. Krawędź $\langle u,v\rangle\in V_1\times V_2$ w~grafie $G$ istnieje wtedy i~tylko wtedy, gdy hiperkrawędź $v$ jest incydentna z~$u$ (hiperkrawędzie mogą być incydentne z~więcej niż dwoma wierzchołkami). Graf $G$ rzeczywiście jest dwudzielny, ponieważ nie~istnieją krawędzie pomiędzy elementami $V_1$ ani pomiędzy elementami $V_2$.

\subchapter{Drzewa}

\exercise %B.5-1
\begin{figure}[ht]
	\begin{center}
		\includegraphics{figb.4}
	\end{center}
	\caption{{\sffamily\bfseries(a)} Drzewa wolne złożone z~3 wierzchołków $A$, $B$ i~$C$. {\sffamily\bfseries(b)} Drzewa ukorzenione o~węzłach $A$, $B$ i~$C$, w~których $A$ jest korzeniem. {\sffamily\bfseries(c)} Drzewa uporządkowane o~węzłach $A$, $B$ i~$C$, w~których $A$ jest korzeniem. {\sffamily\bfseries(d)} Drzewa binarne o~węzłach $A$, $B$ i~$C$, w~których $A$ jest korzeniem.} \label{fig:B.5-1}
\end{figure}

\exercise %B.5-2
Przypuśćmy, że twierdzenie jest fałszywe, czyli że wersja nieskierowana grafu $G$ nie tworzy drzewa, a~więc posiada cykl, w~szczególności cykl prosty (\refExercise{B.4-2}). Niech $\langle v_1,v_2,\dots,v_k,v_1\rangle$ będzie takim cyklem. Graf $G$ jest acykliczny, zatem dla pewnego $1\le l\le k$ istnieją krawędzie $\langle v_l,v_{l+1}\rangle$, $\langle v_{l+2},v_{l+1}\rangle\in E$, przy czym $v_{k+1}$ utożsamiamy z~$v_1$, a~$v_{k+2}$ z~$v_2$. Wiemy z~założenia, że $v_0\leadsto v_l$ oraz $v_0\leadsto v_{l+2}$, zatem istnieją dwie różne ścieżki z~$v_0$ do $v_{l+1}$:
\[
	\langle v_0,\dots,v_l,v_{l+1}\rangle \quad\text{oraz}\quad \langle v_0,\dots,v_{l+2},v_{l+1}\rangle.
\]
Otrzymana sprzeczność prowadzi do wniosku, że wersja nieskierowana grafu $G$ istotnie stanowi drzewo.

\exercise %B.5-3
W~drzewie o~jednym węźle jest jeden liść i~brak węzłów wewnętrznych, więc przypadek bazowy zachodzi. Zauważmy, że wszystkie krawędzie pomiędzy węzłami stopnia~1, a~ich synami można ściągnąć, nie powodując zmian w~liczbie węzłów stopnia~2. W~rzeczywistości operacja ściągnięcia po wszystkich takich krawędziach pozbawia drzewo wszystkich węzłów stopnia~1, które teraz jest drzewem regularnym. Dalej będziemy zatem rozważać tylko takie drzewa.

\medskip
\noindent\textsf{\textbf{Lemat.}} \textit{Niepuste regularne drzewo binarne ma nieparzystą liczbę węzłów.}
\begin{proof}
Niech $T=\langle V,E\rangle$ będzie niepustym regularnym drzewem binarnym, a~$L\subseteq V$ -- zbiorem liści tego drzewa. Obliczmy sumę wszystkich stopni węzłów $T$ (w sensie grafowym, czyli uwzględniając ojca węzła):
\[
	\sum_{v\in V}\deg(v) = \sum_{v\in L}1+\sum_{v\in V\setminus L}\!\!\!3\;-1=3|V|-2|L|-1.
\]
Z~lematu o~podawaniu rąk (\refExercise{B.4-1}) mamy, że $\sum_{v\in V}\deg(v) = 2|E|$, a~stąd
\[
	|V| = \frac{2|E|+2|L|+1}{3}.
\]
Licznik ułamka jest nieparzysty, zatem liczba węzłów $T$ także jest nieparzysta.
\end{proof}

Korzystając z~powyższego lematu, założymy, że twierdzenie jest prawdziwe dla drzewa o~$2k-1$ węzłach ($k\ge1$) i~wykażemy jego prawdziwość dla drzewa o~$2k+1$ węzłach. Mamy więc, że liczba węzłów $w$ stopnia 2 w~regularnym drzewie binarnym o~$2k-1$ węzłach jest o~1 mniejsza od liczby jego liści $l$. Wybierając dowolny liść i~czyniąc z~niego węzeł wewnętrzny, poprzez dołączenie do niego dwóch synów, tworzymy regularne drzewo binarne o~$2k+1$ węzłach. W~nowym drzewie mamy teraz $w'=w+1$ węzłów stopnia 2 oraz $l'=(l-1)+2=l+1$ liści, więc z~założenia indukcyjnego, że $w=l-1$ dostajemy $w'=l'-1$, a~zatem twierdzenie jest prawdziwe.

\exercise %B.5-4
Udowodnimy nierówność $h\ge\lfloor\lg n\rfloor$ przez indukcję względem $h$. Dla $h=0$ drzewo posiada tylko jeden węzeł, zatem $n=1$ i~nierówność zachodzi. Załóżmy zatem, że $h\ge1$ oraz że nierówność jest spełniona dla drzewa $T$ o~wysokości $h-1$ i~posiadającego $m$ węzłów. Na głębokości $h-1$ znajduje się pewna ilość liści, powiedzmy $l$. Utworzymy teraz nowe drzewo $T'$ o~wysokości $h$, dodając pewną ilość nowych węzłów, które będą potomkami niektórych liści drzewa $T$. Oznaczmy przez $n$ liczbę węzłów drzewa $T'$. Ponieważ $l\le2^{h-1}$, to $n\le m+2l\le m+2^h$, bo każdy liść drzewa $T$ może stać się ojcem dla co najwyżej dwóch nowych węzłów. Korzystając z~założenia indukcyjnego, mamy $h-1\ge\lfloor\lg m\rfloor$, a~ponieważ dla dowolnego $h\ge1$ zachodzi $h-1=\bigl\lfloor\lg(2^h-1)\bigr\rfloor$, to musi być $m\le 2^h-1$. Wynika stąd ograniczenie na ilość węzłów drzewa $T'$:
\[
	n \le m+2^h \le 2^h-1+2^h = 2^{h+1}-1.
\]
Logarytmując i~biorąc podłogi obu stron nierówności, dostajemy
\[
	\lfloor\lg n\rfloor \le \bigl\lfloor\lg(2^{h+1}-1)\bigr\rfloor = h,
\]
a~zatem twierdzenie jest prawdziwe dla wszystkich drzew binarnych, bo każde drzewo o~wysokości $h$ można otrzymać z~pewnego drzewa o~wysokości $h-1$ poprzez opisaną wyżej operację dołączania nowych węzłów.

\exercise %B.5-5
Dowód przez indukcję względem $n$. Gdy $n=0$, to drzewo jest puste, więc $i=e=0$, zatem baza zachodzi. Załóżmy więc, że $n>0$ i~że równanie $e=i+2(n-1)$ jest spełnione przez drzewo regularne o~$n-1$ węzłach wewnętrznych. Zauważmy, że aby zwiększyć liczbę węzłów wewnętrznych o~1, należy z~dowolnego liścia na pewnej głębokości $d$ utworzyć węzeł wewnętrzny poprzez dołączenie do niego dwóch nowych liści.

Zbadajmy co się dzieje z~długościami ścieżek wewnętrznej i~zewnętrznej po takiej modyfikacji. Oznaczmy przez $e'$ oraz $i'$, odpowiednio, długość nowej ścieżki zewnętrznej i~długość nowej ścieżki wewnętrznej. Zachodzi $e'=e-d+(d+1)+(d+1)=i+2(n-1)+d+2$ oraz $i'=i+d$, a~zatem $e'=i'+2n$ i~twierdzenie jest spełnione, gdyż teraz w~drzewie jest $n$ węzłów wewnętrznych.

\exercise %B.5-6
Niech $h$ będzie wysokością drzewa binarnego $T$. Zauważmy, że możemy zwiększyć sumę wag liści drzewa $T$ poprzez dołączenie do każdego węzła o~stopniu~1 nowego węzła będącego jego potomkiem, a~jednocześnie będącego nowym liściem drzewa $T$. W~wyniku tej operacji utworzymy z~$T$ pewne regularne drzewo binarne o~wysokości $h$. Można następnie zauważyć, że liść na głębokości $d$ wnosi do sumy składnik równy $2^{-d}=2\cdot2^{-(d+1)}$, zatem uczynienie z~takiego liścia węzła wewnętrznego, poprzez dołączenie do niego nowych węzłów, nie spowoduje zmian w~sumie wag liści. Powtarzając tę czynność dla każdego liścia o~głębokości mniejszej niż $h$, utworzymy z~$T$ pełne drzewo binarne $T'$. Mamy w~nim $2^h$ liści, wszystkie na głębokości $h$, a~zatem
\[
	\sum_{x}w(x) = 2^h\cdot2^{-h} = 1,
\]
gdzie sumujemy po wszystkich liściach $x$ z~$T'$. Ponieważ suma wag liści przyjmuje maksymalną wartość w~pełnym drzewie binarnym, to na mocy powyższego wyniku w~dowolnym drzewie nie przekroczy ona 1.

\exercise %B.5-7
\note{Zarówno w~oryginalnym tekście zadania jak i~w~jego tłumaczeniu występuje błąd. Twierdzenie nie zachodzi bowiem dla drzew o~jednym liściu, dlatego poniższy dowód dotyczy drzew binarnych posiadających co najmniej dwa liście.}

\noindent Niech $T$ będzie drzewem binarnym o~$L\ge2$ liściach oraz niech $LT$ i~$RT$ będą, odpowiednio, jego lewym i~prawym poddrzewem. Niech ponadto $L_1$ i~$L_2$ stanowią, odpowiednio, liczbę liści $LT$ i~liczbę liści $RT$. Bez straty ogólności załóżmy, że $L_1\le L_2$. Jeśli $2L_1\ge L_2$, to $LT$ i~$RT$ są szukanymi poddrzewami. W~przeciwnym przypadku zachodzi $L_2>2L/3$, powtarzamy zatem powyższe rozumowanie rekurencyjnie dla drzewa $RT$. W~ciągu tej procedury w~pewnym momencie będziemy rozważać drzewo $T'$, którego większe poddrzewo $RT'$ będzie mieć nie więcej niż $2L/3$ liści. Ale ponieważ $T'$ ma więcej niż $2L/3$ liści, to liczba liści $RT'$ jest większa niż $L/3$, zatem $RT'$ jest szukanym poddrzewem.

\problems

\problem{Kolorowanie grafów} %B-1

\subproblem %B-1(a)
Zamiast dowolnych drzew rozważmy bez straty ogólności drzewa ukorzenione. Krawędzie są incydentne z~węzłami z~sąsiednich poziomów, a~więc można pokolorować drzewo w~taki sposób, żeby węzły miały kolor równy parzystości swojej głębokości.

\subproblem %B-1(b)
Przyjmijmy bez straty ogólności, że będziemy rozważać tylko grafy spójne, gdyż stwierdzenia te pozostaną prawdziwe, jeżeli będą zachodzić osobno dla każdej składowej.
\bigskip

$1.\Rightarrow 2.\!\!:$ W~grafie dwudzielnym krawędzie łączą wierzchołki między dwoma rozłącznymi zbiorami, zatem można pokolorować wierzchołki dwoma barwami w~zależności od ich przynależności do danego zbioru, uzyskując prawidłowe \compound{2}{kolorowanie}.
\bigskip

$2.\Rightarrow 3.\!\!:$ Niech $G$ będzie grafem \compound{2}{kolorowalnym} i~niech zawiera pewien cykl nieparzysty $\langle v_1,v_2,\dots,v_{2k+1},v_1\rangle$ dla pewnego $k\ge1$. Bez utraty ogólności niech $c(v_1)=0$ i~wtedy musi być $c(v_{2i})=1$ oraz $c(v_{2i+1})=0$, gdzie $1\le i\le k$. Zauważmy jednak, że dwa sąsiednie wierzchołki mają ten sam kolor, $c(v_{2k+1})=c(v_1)=0$, co przeczy założeniu o~tym, że $G$ jest \compound{2}{kolorowalny}. Wnioskujemy zatem, że $G$ nie zawiera cyklu o~długości nieparzystej.
\bigskip

$3.\Rightarrow 1.\!\!:$ Wybierzmy pewien wierzchołek $v\in V$ i~uczyńmy elementami zbioru $V_1$ wszystkie wierzchołki o~odległości parzystej od $v$ (w~sensie najkrótszej ścieżki) wraz z~nim samym. Pozostałe niech tworzą zbiór $V_2$. Ponieważ $G$ nie zawiera cyklu nieparzystego, to żaden jego wierzchołek nie sąsiaduje z~innym wierzchołkiem ze swojego zbioru, a~to oznacza, że graf $G=\langle V_1\cup V_2,E\rangle$ jest dwudzielny.

\subproblem %B-1(c)
Dowód przeprowadzimy indukcyjnie ze względu na liczbę wierzchołków w~grafie $G$. Jeśli graf ma jeden wierzchołek, to oczywiście wystarcza jeden kolor. Załóżmy więc, że $|V|\ge2$. Wybierzmy dowolny wierzchołek $v\in V$ i~rozważmy graf $G'=\bigl\langle V\setminus\{v\},E\bigr\rangle$. Na mocy założenia indukcyjnego da się go pokolorować $d+1$ barwami, gdzie $d$ to maksimum stopni wierzchołków w~$G'$. Zauważmy, że $v$ ma co najwyżej $d$ sąsiadów. Wśród $d+1$ kolorów użytych w~kolorowaniu grafu $G'$ jest więc kolor nieprzypisany żadnemu sąsiadowi wierzchołka $v$. Wybierając ten kolor dla $v$, poprawnie kolorujemy graf $G$ $d+1$ kolorami, więc dowód jest zakończony.

\subproblem %B-1(d)
W~optymalnym kolorowaniu grafu (czyli takim, które wykorzystuje możliwie najmniej kolorów), jeżeli $k$ barw wystarcza do pokolorowania grafu $G$, to dla każdej pary różnych barw muszą istnieć wierzchołki sąsiednie o~takich barwach. W~przeciwnym przypadku istniałyby dwa różne kolory, które nie miałyby sąsiednich wierzchołków, a~zatem można byłoby potraktować je jako jeden kolor, co przeczyłoby minimalności $k$. Wszystkich możliwych par wierzchołków o~różnych kolorach spośród $k$ jest zatem nie mniej niż
\[
	\binom{k}{2} = \frac{k(k-1)}{2} \ge |E|,
\]
skąd $k=O\bigl(\!\sqrt{|E|}\bigr)$ i~twierdzenie zachodzi na mocy tego, że $|E|=O(|V|)$.

\problem{Grafy znajomości} %B-2

\subproblem %B-2(a)
\textsf{\textbf{Twierdzenie.}} \textit{W~prostym grafie nieskierowanym\/ $G=\langle V,E\rangle$, w~którym\/ $|V|\ge2$, istnieją dwa wierzchołki o~tym samym stopniu.}
\begin{proof}
W~grafie $G$ o~$n\ge2$ wierzchołkach możliwymi stopniami wierzchołków są liczby 0, 1,~\dots,~$n-1$. Jeśli jednak pewien wierzchołek ma stopień równy~0, to żaden z~pozostałych nie ma stopnia $n-1$. Oznacza to, że jest $n$ wierzchołków, ale tylko co najwyżej $n-1$ liczb mogących jednocześnie być stopniami wierzchołków w~$G$, zatem istnieją pewne dwa wierzchołki o~tym samym stopniu.
\end{proof}

\subproblem %B-2(b)
\textsf{\textbf{Twierdzenie.}} \textit{Graf pełny\/ $K_3$ jest podgrafem dowolnego prostego grafu nieskierowanego\/ $G=\langle V,E\rangle$, w~którym\/ $|V|=6$, lub jego dopełnienia\/ $\overline{G}$.}
\begin{proof}
Wybierzmy pewne $v\in V$. Istnieją wtedy w~$V$ trzy inne wierzchołki $v_1$, $v_2$,~$v_3$ wszystkie sąsiednie z~$v$ albo wszystkie niesąsiednie z~$v$. Ponieważ przypadki te są symetryczne, rozważmy pierwszy z~nich. Jeśli nie istnieje wśród $v_1$, $v_2$,~$v_3$ para wierzchołków sąsiednich, to twierdzenie zachodzi. Załóżmy więc, że istnieje krawędź między pewnymi dwoma. Wtedy jednak tworzą one wraz z~$v$ graf $K_3$, a~więc również w~tym przypadku twierdzenie jest prawdziwe.
\end{proof}

Problem rozważany w~tym punkcie jest związany z~\emph{liczbami Ramseya} $R(q_1,q_2,\dots,q_k)$; powyższe twierdzenie stanowi dowód, że $R(3,3)\le6$.

\subproblem %B-2(c)
\textsf{\textbf{Twierdzenie.}} \textit{Zbiór wierzchołków\/ $V$ dowolnego prostego grafu nieskierowanego\/ $G=\langle V,E\rangle$, można podzielić na dwa rozłączne zbiory tak, aby co najmniej połowa wierzchołków sąsiednich z~wierzchołkiem\/ $v\in V$ nie należała do zbioru, do którego należy\/ $v$.}
\begin{proof}
Przypiszmy każdemu wierzchołkowi $v\in V$ wagę $d(v)$ równą różnicy liczby wierzchołków ze zbioru, do którego należy $v$ sąsiednich z~nim i~liczby wierzchołków sąsiednich z~$v$ spoza jego zbioru. Dowód sprowadza się do pokazania, że $d(v)\ge0$ dla każdego $v\in V$\!.

Zdefiniujmy teraz liczbę $\sigma=\sum_{v\in V}d(v)$ i~zastanówmy się jak można ją zmaksymalizować, wyznaczając podziały $V$ na dwa rozłączne podzbiory $V_1$ i~$V_2$. Załóżmy, że dokonaliśmy już pewnego takiego podziału i~wybierzmy pewien wierzchołek $v$ należący do zbioru $V_1$. Dowód w~przypadku gdy $v\in V_2$ przebiega symetrycznie. Zauważmy, że przenosząc wierzchołek $v$ do $V_2$ zmieniamy $d(v)$ na liczbę przeciwną. Ponadto dla każdego sąsiedniego do $v$ wierzchołka $v_1\in V_1$ jego waga $d(v_1)$ wzrasta o~2, a~dla każdego $v_2\in V_2$ sąsiedniego z~$v$ $d(v_2)$ maleje o~2. Operacja przeniesienia zwiększa zatem wartość $\sigma$ o~$-4d(v)$. Widać więc, że aby zmaksymalizować $\sigma$, należy przenosić wierzchołki o~ujemnych wagach. Ponieważ $\sigma$ nie może rosnąć w~nieskończoność (jest ograniczone od góry przez $2|E|$ w~grafach dwudzielnych), to po skończonej liczbie przenosin w~grafie $G$ wszystkie wierzchołki będą mieć wagi nieujemne, czego należało dowieść.
\end{proof}

\subproblem %B-2(d)
\textsf{\textbf{Twierdzenie} (Dirac)\textbf{.}} \textit{Jeśli dla każdego wierzchołka\/ $v$ prostego grafu nieskierowanego\/ $G=\langle V,E\rangle$, w~którym\/ $|V|=n\ge3$, zachodzi\/ $\deg(v)\ge n/2$, to\/ $G$ jest hamiltonowski.}

\medskip
\noindent Zanim zajmiemy się dowodem twierdzenia, udowodnimy następujący lemat.

\medskip
\noindent\textsf{\textbf{Lemat} (Ore)\textbf{.}} \textit{Jeśli w~prostym grafie nieskierowanym\/ $G=\langle V,E\rangle$, gdzie\/ $|V|=n\ge3$, zachodzi nierówność\/ $\deg(u)+\deg(v)\ge n$ dla każdej pary niesąsiednich wierzchołków\/ $u$ i~\/$v$, to\/ $G$ jest hamiltonowski.}
\begin{proof}
Przypuśćmy, że lemat jest fałszywy, czyli dla pewnego $n$ istnieje kontrprzykład -- graf $G$, który spełnia założenie lematu, ale nie jest hamiltonowski. Spośród wszystkich takich grafów rozpatrzmy ten, dla którego $|E|$ jest maksymalne. Jest to podgraf pełnego grafu hamiltonowskiego $K_n$. Dodanie do $G$ krawędzi z~grafu $K_n$ daje w~wyniku graf, który nadal spełnia założenie lematu i~który ma więcej niż $|E|$ krawędzi, a~więc ze względu na wybór grafu $G$, tak powstały graf będzie miał cykl Hamiltona. To znaczy, że $G$ musi mieć przynajmniej drogę Hamiltona określoną przez pewien ciąg wierzchołków $\langle v_1,v_2,\dots,v_n\rangle$. Ponieważ $G$ nie ma cyklu Hamiltona, to nie istnieje krawędź łącząca $v_1$ z~$v_n$. Z~kolei z~założenia wiemy, że $\deg(v_1)+\deg(v_n)\ge n$.

Można teraz zdefiniować podzbiory $S_1$ i~$S_n$ zbioru $\{2,3,\dots,n\}$ takie, że
\[
	S_1 = \bigl\{\,i:\{v_1,v_i\}\in E\,\bigr\} \quad\text{oraz}\quad S_n = \bigl\{\,i:\{v_{i-1},v_n\}\in E\,\bigr\}.
\]
Wtedy $|S_1|=\deg(v_1)$ i~$|S_n|=\deg(v_n)$. Ponieważ $|S_1|+|S_n|\ge n$ i~zbiór $S_1\cup S_n$ ma co najwyżej $n-1$ elementów, to zbiór $S_1\cap S_n$ musi być niepusty. Istnieje więc $i$, dla którego istnieją krawędzie $\{v_1,v_i\}$ oraz $\{v_{i-1},v_n\}$. Wtedy droga $\langle v_1,\dots,v_{i-1},v_n,v_{n-1},\dots,v_i,v_1\rangle$ jest cyklem Hamiltona w~grafie $G$. Sprzeczność -- lemat jest prawdziwy.
\end{proof}

Można teraz udowodnić główne twierdzenie.
\begin{proof}
Jeśli dla każdego $v\in V$ zachodzi $\deg(v)\ge n/2$, to $\deg(u)+\deg(v)\ge n$ dla każdych $u$,~$v\in V$ niezależnie od tego, czy są sąsiednie, czy nie, a~więc $G$ spełnia założenia powyższego lematu, czyli jest hamiltonowski.
\end{proof}

\problem{Podziały drzew} %B-3

\subproblem %B-3(a)
Niech $T=\langle V,E\rangle$ będzie drzewem binarnym, w~którym $|V|=n\ge2$. Przez \emph{krawędź dzielącą} będziemy rozumieć krawędź, po usunięciu której zbiór wierzchołków drzewa $T$ dzieli się na zbiory $A$ i~$B$ takie, że $|A|\le3n/4$ oraz $|B|\le3n/4$. Udowodnimy przez indukcję względem $n$, że w~każdym takim drzewie istnieje krawędź dzieląca.

Dla $n=2$ twierdzenie zachodzi, ponieważ w~drzewie $T$ istnieje tylko jedna krawędź, po usunięciu której dostajemy zbiory jednoelementowe. Niech zatem $n>2$ i~załóżmy, że w~drzewie o~$n-1$ wierzchołkach istnieje krawędź dzieląca $e\in E$ taka, że po podziale każdy ze zbiorów $A$ i~$B$ ma co najwyżej $3(n-1)/4$ elementów. Przyjmijmy bez utraty ogólności, że $|A|\le|B|$, co oznacza, że $|A|\le(n-1)/2$. Utwórzmy teraz nowe drzewo $T'$, dodając do $V$ nowy wierzchołek $v'$ oraz nową krawędź $\{v',v\}$ do $E$ dla pewnego $v\in V$. Niech teraz $A'$ oraz $B'$ będą zbiorami wierzchołków w~nowym drzewie utworzonymi w~wyniku podziału krawędzią $e$. Jeśli $v\in A$, to $A'=A\cup\{v'\}$ oraz $B'=B$. Oczywiście $|B'|<3n/4$, zbadajmy zatem $A'$:
\[
	|A'| = |A|+1 \le \frac{n-1}{2}+1 = \frac{n+1}{2} \le \frac{3n}{4},
\]
co jest prawdą, o~ile $n\ge2$, zatem w~tym przypadku twierdzenie zachodzi.

Niech teraz $v\in B$. Stąd $A'=A$ i~$B'=B\cup \{v'\}$, ale z~założenia $|B'|=|B|+1\le(3n+1)/4$, a~zatem $|B'|$ może przekroczyć $3n/4$, co oznacza, że musimy znaleźć inną krawędź dzielącą dla drzewa $T'$ w~przypadku, gdy $|B|=(3n-3)/4$, przy czym $n\ge5$.

Rozważmy drzewo $T'$ przedstawione na rys.~\ref{fig:B-3a}.
\begin{figure}[ht]
	\begin{center}
		\includegraphics{figb.5}
	\end{center}
	\caption{Drzewo $T'$ z~drugiego przypadku dowodu.} \label{fig:B-3a}
\end{figure}
Niech $u_1\in B$ oraz $e=\{u_1,u_2\}$. Oprócz $u_1$ do zbioru $B$ należą wierzchołki ze zbiorów $V_1$ i~$V_2$, a~do zbioru $A$ -- wierzchołek $u_2$ oraz wierzchołki ze zbiorów $V_3$ i~$V_4$. Załóżmy bez straty ogólności, że $|V_1|\le|V_2|$. Zbiór $V_2$ jest niepusty, gdyż $|B'|\ge4$, istnieje zatem krawędź $e'=\{u_1,w\}$, gdzie $w\in V_2$. Pokażemy, że jest to krawędź dzieląca drzewa $T'$. Rozważmy w~tym celu zbiory $A''$ i~$B''$, na które krawędź $e'$ dzieli zbiór $V\cup\{v'\}$. Mamy
\[
	|B''| = |V_2| \le |B| = \frac{3n-3}{4} < \frac{3n}{4}
\]
oraz
\[
	|A''| = \bigl|A\cup\{u_1\}\cup V_1\bigr| \le (n-1-|B|)+1+\frac{|B|}{2} = n-\frac{|B|}{2} = \frac{5n+4}{8}.
\]
Skorzystaliśmy z~tego, że $|A|+|B|=n-1$ oraz $|V_1|\le|B|/2$. Nierówność $|A''|\le3n/4$ zachodzi, o~ile $n\ge4$, więc istotnie $e'$ jest krawędzią dzielącą drzewa $T'$.

Rozpatrzyliśmy wszystkie przypadki, zatem na mocy indukcji twierdzenie zachodzi dla każdego drzewa binarnego $T$.

\subproblem %B-3(b)
Stała $3/4$ jest wystarczająca do dokonywania zrównoważonych podziałów, jak to wykazaliśmy w~punkcie~(a). Przykład drzewa binarnego z~rys.~\ref{fig:B-3b} pokazuje, że nie można rozważać mniejszej stałej niż $3/4$. Usuwając dowolną krawędź tego drzewa, dzielimy zbiór jego wierzchołków na podzbiory, z~których jeden ma trzy elementy.
\begin{figure}[ht]
	\begin{center}
		\includegraphics{figb.6}
	\end{center}
	\caption{Drzewo binarne, w~którym najbardziej zrównoważony podział tworzy podzbiór zawierający 3 wierzchołki.} \label{fig:B-3b}
\end{figure}

\subproblem %B-3(c)
Rozważmy następującą procedurę podziału zbioru wierzchołków. Na początku wynikowe zbiory $A$ i~$B$ są puste. Usuwając jedną krawędź, możemy podzielić \compound{$n$}{elementowy} zbiór~wierzchołków drzewa na dwa podzbiory, z~których większy będzie składać się z~co najwyżej $3n/4$ wierzchołków, co wynika na podstawie punktu~(a). Mniejszy podzbiór sumujemy z~jednym ze zbiorów wynikowych, natomiast większy z~nich będzie podlegał dalszemu podziałowi. Podczas działania procedury pilnujemy, aby liczby elementów $A$ i~$B$ nie przekroczyły $\lceil n/2\rceil$. Procedurę podziału zakończymy w~momencie, gdy jeden z~tych zbiorów będzie zawierał $\lceil n/2\rceil$ elementów, gdyż drugi zbiór zawiera wtedy $\lfloor n/2\rfloor$ elementów, czyli jest to podział, jakiego wymagano.

Zauważmy, że maksymalną liczbę podziałów dla zadanego drzewa wykonamy w~przypadku, gdy po~każdym kroku zostanie do podziału zbiór stanowiący $3/4$ zbioru z~poprzedniego kroku. Niech $k$ będzie taką maksymalną liczbą podziałów drzewa o~$n$ wierzchołkach. Zachodzi wtedy
$(3/4)^kn = 1,$
ponieważ singletonu nie trzeba dalej dzielić. Stąd $k=\log_{4/3}n$, a~zatem należy usunąć co najwyżej $k=O(\lg n)$ krawędzi.

\endinput

\chapter{Zliczanie i prawdopodobieństwo}

\section{Zliczanie}

\subsection{} %C.1-1
Załóżmy, że nie rozważamy słowa pustego i że $1\le k\le n$. Pierwsze $k$-podsłowo zajmuje w $n$-słowie pozycje $1$, $2$,~$\dots$,~$k$, drugie -- $2$, $3,$~$\dots$,~$k+1$ itd. Ostatnie $k$-podsłowo leży na pozycjach $n-k+1$, $n-k+2$,~$\dots$,~$n$. Jest zatem
\[
	n-k+1
\]
$k$-podsłów $n$-słowa.

By obliczyć łączną ilość podsłów $n$-słowa, należy zsumować liczby $k$-podsłów po wszystkich $1\le k\le n$, co daje
\[
	\sum_{k=1}^n(n-k+1) = \sum_{i=1}^ni = \frac{n(n+1)}{2}.
\]

\subsection{} %C.1-2
Niech $X=\{0,1,\dots,2^n-1\}$ i $Y=\{0,1,\dots,2^m-1\}$ będą zbiorami liczb, odpowiednio, $n$-bitowych i $m$-bitowych. Zauważmy, że funkcji logicznych o $n$ wejściach i $m$ wyjściach będzie tyle samo, co funkcji $f\colon X\to Y$.

Zagadnienie sprowadza się zatem do pytania o liczbę wszystkich ciągów $y_1,\dots,y_{2^n}$ o wyrazach ze zbioru $2^m$-elementowego $Y$. Każdy wyraz $y_i$ możemy wybrać na $2^m$ sposobów, co daje $(2^m)^{2^n} = 2^{m2^n}$ możliwości wyboru ciągu $y_1,\dots,y_{2^n}$. Jest zatem $2^{m2^n}$ funkcji logicznych o $n$ wejściach i $m$ wyjściach, a~stąd $2^{2^n}$ funkcji logicznych o $n$ wejściach i 1 wyjściu.

\subsection{} %C.1-3
Niech $S_n$ oznacza szukaną liczbę sposobów ustawienia $n$ osób przy stole. Jeden ze sposobów jest nierozróżnialny z $n-1$ innymi, dzięki temu, że stół jest okrągły, a osoby mogą przesuwać się miejscami nie zmieniając kolejności wzajemnego ustawienia. Ponadto, jest $n!$ możliwych permutacji osób, zatem $nS_n$ jest równe $n!$. Mamy zatem
\[
	S_n = \frac{n!}{n} = (n-1)!.
\]

\subsection{} %C.1-4
By wybrać trzy liczby ze zbioru $\{1,2,\dots,100\}$, które w sumie dadzą liczbę parzystą, można postąpić na dwa sposoby:
\begin{itemize}
	\item wybrać 3 liczby parzyste,
	\item wybrać 2 liczby nieparzyste i 1 liczbę parzystą.
\end{itemize}
W pierwszym przypadku możemy to zrobić na $\binom{50}{3}$ sposobów, a w drugim na $\binom{50}{2}\binom{50}{1}$ sposobów. Łączna liczba możliwości wyboru takich liczb wynosi zatem
\[
	\binom{50}{3}+\binom{50}{2}\binom{50}{1} = 80850.
\]

\subsection{} %C.1-5
\[
	\binom{n}{k} = \frac{n!}{k!\,(n-k)!} = \frac{n}{k}\cdot\frac{(n-1)!}{(k-1)!\,(n-k)!} = \frac{n}{k}\binom{n-1}{k-1}.
\]

\subsection{} %C.1-6
\[
	\binom{n}{k} = \frac{n!}{k!\,(n-k)!} = \frac{n}{n-k}\cdot\frac{(n-1)!}{k!\,(n-k-1)!} = \frac{n}{n-k}\binom{n-1}{k}.
\]

\subsection{} %C.1-7
Załóżmy, że wybieramy pewien $k$-podzbiór z $n$-elementowego zbioru $S$, co można zrobić na $\binom{n}{k}$ sposobów. Wyróżnijmy pewien element z $S$. Jeśli nie został on wybrany w $k$-podzbiorze, to istnieje $\binom{n-1}{k}$ możliwości wyboru $k$ elementów spośród $n-1$ pozostałych ze zbioru $S$. Jeżeli jednak wyróżniony element należy do wybranego podzbioru, to z $n-1$ pozostałych elementów należy wybrać jeszcze $k-1$, co można wykonać na $\binom{n-1}{k-1}$ sposobów. Otrzymujemy zatem
\[
	\binom{n}{k} = \binom{n-1}{k}+\binom{n-1}{k-1}.
\]

\subsection{} %C.1-8
Kilka początkowych wierszy trójkąta Pascala:
\[
	\begin{array}{ccccccccccccc}
		&&&&&& 1 \\
		&&&&& 1 && 1 \\
		&&&& 1 && 2 && 1 \\
		&&& 1 && 3 && 3 && 1 \\
		&& 1 && 4 && 6 && 4 && 1 \\
		& 1 && 5 && 10 && 10 && 5 && 1 \\
		1 && 6 && 15 && 20 && 15 && 6 && 1
	\end{array}
\]
W pierwszym wierszu mamy tylko jeden element, $\binom{0}{0}=1$. Drugi wiersz zawiera $\binom{1}{0}=1$ i $\binom{1}{1}=1$. Kolejne wiersze mają jedynki na końcach, lewym i prawym, zaś elementy wewnętrzne powstają przez zsumowanie dwóch liczb z poprzedniego wiersza znajdujących się bezpośrednio nad wyliczanym elementem.

\subsection{} %C.1-9
Z tożsamości (A.1) mamy
\[
	\sum_{i=1}^ni = \frac{n(n+1)}{2},
\]
a z definicji współczynnika dwumianowego
\[
	\binom{n+1}{2} = \frac{(n+1)!}{2!\,(n-1)!} = \frac{n(n+1)}{2}.
\]
Prawe strony powyższych równań są identyczne, czego należało dowieść.

\subsection{} %C.1-10
Potraktujmy współczynniki dwumianowe jako funkcję $b_n(k)=\binom{n}{k}$ dla $0\le k\le n$ i sprawdźmy, dla jakich $k$ wartość $b_n(k)$ jest największa.
\begin{eqnarray*}
	b_n(k+1) &>& b_n(k) \\
	\binom{n}{k+1} &>& \binom{n}{k} \\
	\frac{n!}{(k+1)!\,(n-k-1)!} &>& \frac{n!}{k!\,(n-k)!} \\
	\frac{(n-k)!}{(n-k-1)!} &>& \frac{(k+1)!}{k!} \\
	n-k &>& k-1 \\
	k &<& \frac{n-1}{2},
\end{eqnarray*}
a zatem $b_n(k)$ jest funkcją rosnącą o ile $k<(n-1)/2$, z największą wartością osiąganą dla $k=\lfloor(n+1)/2\rfloor$. W zależności od parzystości $n$, liczba ta jest równa $\lfloor n/2\rfloor$ lub $\lceil n/2\rceil$.

\subsection{} %C.1-11
Dla $n\ge0$, $j\ge0$, $k\ge0$ takich, że $j+k\le n$ dowodzimy
\begin{eqnarray*}
	\binom{n}{j+k} &\le& \binom{n}{j}\binom{n-j}{k} \\
	\frac{n!}{(j+k)!\,(n-j-k)!} &\le& \frac{n!}{j!\,(n-j)!}\cdot\frac{(n-j)!}{k!\,(n-j-k)!} \\
	\frac{1}{(j+k)!} &\le& \frac{1}{j!\,k!} \\
	j!\,k! &\le& (j+k)! \\
	j!\,k! &\le& j!\cdot\prod_{i=1}^k(j+i) \\
	\prod_{i=1}^ki &\le& \prod_{i=1}^k(j+i).
\end{eqnarray*}
Iloczyn $k$ liczb całkowitych od $1$ do $k$ jest oczywiście niewiększy od iloczynu $k$ liczb całkowitych od $j+1$ do $j+k$, zatem ostatnia nierówność jest prawdziwa. Równość zachodzi dla przypadków, gdy $j=0$ lub $j=1$, $k=0$.

Lewą stronę nierówności można zinterpretować jako liczbę możliwych wyborów $j+k$ przedmiotów spośród zbioru $n$-elementowego, prawą zaś jako liczbę możliwych sposobów wyboru najpierw $j$ przedmiotów spośród $n$, a następnie $k$ przedmiotów spośród $n-j$ pozostawionych po pierwszym wyborze.

Załóżmy, że $A=\{a_1,a_2,\dots,a_{j+k}\}$ jest zbiorem wybranych elementów. Jest tylko 1 sposób wyboru zadanego zbioru $A$ przy pierwszej strategii i o wiele więcej, jeśli zastosuje się drugie podejście. Można mianowicie dowolnie podzielić elementy z $A$ na $j$ takich, które będą wybierane w pierwszym kroku i $k$ takich, które wybierzemy w drugim kroku.

\subsection{} %C.1-12
Przypadek dla $k=0$ sprawdzamy w pierwszym kroku indukcyjnym i stwierdzamy, że zachodzi. Przyjmijmy, że $0\le k<n/2$ i załóżmy, że prawdą jest
\[
	\binom{n}{k} \le \frac{n^n}{k^k(n-k)^{n-k}}.
\]
Mamy teraz z założenia indukcyjnego
\[
	\binom{n}{k+1} = \frac{n-k}{k+1}\binom{n}{k} \le \frac{n^n}{(k+1)k^k(n-k)^{n-k-1}}.
\]
Zbadajmy następującą nierówność:
\begin{eqnarray*}
	\frac{n^n}{(k+1)k^k(n-k)^{n-k}} &\le& \frac{n^n}{(k+1)^{k+1}(n-k-1)^{n-k-1}} \\\\
	(k+1)^k(n-k-1)^{n-k-1} &\le& k^k(n-k)^{n-k-1} \\\\
	\left(\frac{k+1}{k}\right)^k &\le& \left(\frac{n-k}{n-k-1}\right)^{n-k-1} \\\\
	\left(1+\frac{1}{k}\right)^k &\le& \left(1+\frac{1}{n-k-1}\right)^{n-k-1}.
\end{eqnarray*}
Ciąg $e_n=\left(1+\frac{1}{n}\right)^n$ jest rosnący, a stąd dostajemy
\begin{eqnarray*}
	k &\le& n-k-1 \\
	k &\le& n/2.
\end{eqnarray*}
Twierdzenie zachodzi zatem dla wszystkich $k\le n/2$. Z wzoru (C.3) mamy, że $\binom{n}{k}=\binom{n}{n-k}$ i gdy $k>n/2$ sprowadzamy dowód twierdzenia do pokazania, że
\[
	\binom{n}{n-k} \le \frac{n^n}{k^k(n-k)^{n-k}},
\]
jako że $0\le n-k<n/2$. Wyczerpuje to wszystkie przypadki, a zatem twierdzenie zachodzi dla każdego $0\le k\le n$.

\subsection{} %C.1-13
Wykorzystując wzór Stirlinga mamy
\begin{eqnarray*}
	\binom{2n}{n} &=& \frac{(2n)!}{(n!)^2} \\
	&=& \frac{\sqrt{4\pi n}\left(\frac{2n}{e}\right)^{2n}\bigl(1+O(1/n)\bigr)}{2\pi n\left(\frac{n}{e}\right)^{2n}\bigl(1+O(1/n)\bigr)^2} \\
	&=& \frac{2^{2n}\sqrt{\pi n}}{\pi n\bigl(1+O(1/n)\bigr)} \\
	&=& \frac{2^{2n}}{\sqrt{\pi n}}\bigl(1+O(1/n)\bigr).
\end{eqnarray*}
W wyprowadzeniu oszacowania skorzystano z tożsamości
\[
	\frac{1}{\bigl(1+O(1/n)\bigr)} \equiv \bigl(1+O(1/n)\bigr).
\]

\subsection{} %C.1-14
Niech $g$ i $h$ będą surjekcjami w $(0,1)$ i niech będą różniczkowalne w~tym przedziale. Dla $h(x)=-x\lg x$ mamy
\[
	\frac{dh(g(x))}{dx} = -\frac{dg(x)}{dx}\bigl(\lg g(x)+\lg e\bigr).
\]
Liczymy pierwszą pochodną funkcji entropii $H(\lambda)=h(\lambda)+h(1-\lambda)$,
\begin{eqnarray*}
	\frac{dH(\lambda)}{d\lambda} &=& \frac{dh(\lambda)}{d\lambda}+\frac{dh(1-\lambda)}{d\lambda} \\
	&=& -\lg\lambda-\lg e+\lg(1-\lambda)+\lg e \\
	&=& \lg(1-\lambda)-\lg\lambda.
\end{eqnarray*}
Zbadajmy gdzie $H$ posiada ekstremum przyrównując pierwszą pochodną do 0:
\begin{eqnarray*}
	\frac{dH(\lambda)}{d\lambda} &=& 0 \\
	\lg(1-\lambda) &=& \lg\lambda \\
	1-\lambda &=& \lambda \\
	\lambda &=& 1/2.
\end{eqnarray*}
Należy jeszcze zbadać znak drugiej pochodnej w punkcie $\lambda=1/2$.
\[
	\frac{d^2H(\lambda)}{d\lambda^2} = -\frac{\lg e}{\lambda(1-\lambda)},
\]
a zatem $\frac{d^2H(1/2)}{d\lambda^2}<0$, więc w punkcie $\lambda=1/2$ binarna funkcja entropii $H$ osiąga maksimum wynoszące $H(1/2)=1$.

\subsection{} %C.1-15
Dla $n=0$ równość jest prawdziwa, więc rozważmy sumę od 1 do $n$,
\begin{eqnarray*}
	\sum_{k=1}^n\binom{n}{k}k &=& \sum_{k=1}^n\binom{n-1}{k-1}n \\
	&=& n\sum_{k=0}^{n-1}\binom{n-1}{k} \\
	&=& n2^{n-1}.
\end{eqnarray*}
Ostatnia równość zachodzi z wzoru (C.4) dla $x=y=1$.

\section{Prawdopodobieństwo}

\subsection{} %C.2-1
Utwórzmy skończoną lub przeliczalną rodzinę zdarzeń,
\begin{eqnarray*}
	C_1 &=& A_1 \\
	C_2 &=& A_2\setminus A_1 \\
	C_3 &=& A_3\setminus (A_1\cup A_2) \\
	C_4 &=& A_4\setminus (A_1\cup A_2\cup A_3) \\
	& \vdots \\
	C_k &=& A_k\setminus \bigcup_{i=1}^{k-1}A_i \\
	& \vdots
\end{eqnarray*}
Korzystając z tożsamości
\[
	\bigcup_iA_i = \bigcup_iC_i
\]
oraz z tego, że zdarzenia $C_1$, $C_2$, $\dots$ wzajemnie się wykluczają, otrzymujemy
\[
	\Pr\biggl(\bigcup_iA_i\biggr) = \Pr\biggl(\bigcup_iC_i\biggr) = \sum_i\Pr(C_i).
\]
Ponieważ $\Pr(C_i)\le\Pr(A_i)$ dla każdego $i$, to prawdą jest, że
\[
	\Pr\biggl(\bigcup_iA_i\biggr) \le \sum_i\Pr(A_i).
\]

\subsection{} %C.2-2
Zdefiniujmy $3$-słowo nad alfabetem $\{{\scriptstyle\rm O},{\scriptstyle\rm R}\}$ w następujący sposób. Pierwszy symbol tego słowa oznacza wynik rzutu monetą profesora Rosencrantza, drugi symbol to wynik rzutu pierwszą monetą profesora Guildensterna, a~trzeci to wynik rzutu jego drugą monetą, przy czym $\scriptstyle\rm O$ oznacza wyrzucenie orła, a $\scriptstyle\rm R$ -- wyrzucenie reszki. Tworzymy przestrzeń zdarzeń elementarnych
\[
	S = \{{\scriptstyle\rm OOO},{\scriptstyle\rm OOR},{\scriptstyle\rm ORO},{\scriptstyle\rm ORR},{\scriptstyle\rm ROO},{\scriptstyle\rm ROR},{\scriptstyle\rm RRO},{\scriptstyle\rm RRR}\}.
\]
Każde z tych zdarzeń zachodzi z prawdopodobieństwem równym $1/8$, w szczególności zdarzenie $\scriptstyle\rm ORR$, oznaczające wyrzucenie przez profesora Rosencrantza większej ilości orłów od rywala.

\subsection{} %C.2-3
Jeśli kolejno wyciągane karty mają mieć rosnące numery, to pierwsza z nich musi mieć numer od 1 do 8, druga -- od 2 do 9, a trzecia -- od 3 do 10. Oznaczmy zdarzenia: \\
\begin{tabular}{rcl}
	$A$ &--& numer drugiej karty jest większy od numeru pierwszej karty, \\
	$B$ &--& numer trzeciej karty jest większy od numeru drugiej karty.
\end{tabular}
\\
Jeśli numer drugiej karty wynosi $k$, to liczba zdarzeń sprzyjających $A$ wynosi $k-1$, a sprzyjających $B$ -- $10-k$.

Mamy obliczyć $\Pr(A\cap B)$. Korzystając z reguły iloczynu dostajemy, że liczba zdarzeń sprzyjających $A\cap B$ wynosi $\sum_{k=2}^9(k-1)(10-k)=120$. Liczba możliwych sposobów wyboru trzech kart spośród dziesięciu wynosi $\frac{10!}{7!}=720$ (liczba wszystkich 3-permutacji zbioru 10-elementowego), dostajemy zatem wynik $\Pr(A\cap B)=\frac{120}{720}=\frac{1}{6}$.

\subsection{} %C.2-4
Ponieważ $a<b$, to $a/b<1$. Rozważmy część ułamkową binarnego rozwinięcia ilorazu $a/b$, które jest nieskończonym ciągiem zer i jedynek.

Będziemy rzucać monetą i tworzyć nowy ciąg zer i jedynek, w zależności od wyniku rzutu, dla orła przyjmując 1, a dla reszki 0. Rzucamy monetą dopóki tworzony przez nas ciąg jest równy pewnemu prefiksowi rozwinięcia binarnego $a/b$. W momencie gdy natrafimy na pierwszą różnicę, otrzymany ciąg traktujemy jako część ułamkową rozwinięcia binarnego pewnej liczby. Jeśli liczba ta jest mniejsza od $a/b$, to zwracamy orła, w przeciwnym przypadku -- reszkę.

Oczekiwana liczba rzutów monetą potrzebnych do wyznaczenia wyniku jest oczekiwaną liczbą rzutów aż do pierwszej różnicy w porównaniu z rozwinięciem binarnym $a/b$. Jeśli przyjmiemy, że sukcesem jest wynik rzutu monetą niepasujący do bieżącego elementu rozwinięcia $a/b$, to jego prawdopodobieństwo wynosi $p=1/2$. Liczba rzutów $n$ aż do pierwszego sukcesu jest zmienną losową $X$ o~rozkładzie geometrycznym, dla której
\[
	\Pr(X=n) = (1-p)^{n-1}p.
\]
Z wzoru (C.31) otrzymujemy
\[
	\mathrm{E}(X) = 1/p = 2.
\]
Widać zatem, że oczekiwana liczba rzutów monetą w opisanej procedurze jest $O(1)$.

\subsection{} %C.2-5
Korzystając z tożsamości $(A\cap B)\cup(\overline{A}\cap B)=B$ oraz z tego, że zdarzenia $A\cap B$ i $\overline{A}\cap B$ wykluczają się, otrzymujemy
\[
	\Pr(A|B)+\Pr(\overline{A}|B) = \frac{\Pr(A\cap B)}{\Pr(B)}+\frac{\Pr(\overline{A}\cap B)}{\Pr(B)} = \frac{\Pr(B)}{\Pr(B)} = 1.
\]

\subsection{} %C.2-6
Dowód przez indukcję względem liczby zdarzeń.

Dla $n=1$ dowód jest trywialny, załóżmy zatem, że $n\ge1$. Otrzymujemy
\begin{eqnarray*}
	\Pr\biggl(\bigcap_{i=1}^{n+1}A_i\biggr) &=& \Pr\biggl(A_{n+1}\cap\bigcap_{i=1}^nA_i\biggr) \\
	&=& \Pr\biggl(\bigcap_{i=1}^nA_i\biggr)\Pr\biggl(A_{n+1}\biggm|\bigcap_{i=1}^nA_i\biggr) \\
	&=& \Pr(A_1)\Pr(A_2|A_1)\Pr(A_3|A_1\cap A_2)\ldots\Pr\biggl(A_{n+1}\biggm|\bigcap_{i=1}^nA_i\biggr).
\end{eqnarray*}
Druga równość wynika z definicji prawdopodobieństwa warunkowego, a trzecia -- z założenia indukcyjnego.

\subsection{} %C.2-7
\subsection{} %C.2-8
\subsection{} %C.2-9
Rozważmy zdarzenia, $A$ -- wybraliśmy zasłonę, za którą jest nagroda i zdarzenie do niego przeciwne $B$. Mamy $\Pr(A)=1/3$ i $\Pr(B)=2/3$. Obliczmy prawdopodobieństwo wygranej $W$ w zależności od podjętej decyzji po podniesieniu przez prowadzącego jednej z zasłon,
\[
	\Pr(W) = \Pr(W|A)\Pr(A)+\Pr(W|B)\Pr(B).
\]
W pierwszej strategii decydujemy się na pozostanie przy aktualnym wyborze, zatem ponieważ nie zmieniamy wybranej zasłony, mamy $\Pr(W|A)=1$ i~$\Pr(W|B)=0$, a więc wygramy z prawdopodobieństwem $\Pr(W)=1/3$. Jeśli teraz rozważymy drugą strategię, w której zmienimy zasłonę po ujawnieniu jednej z przegrywających, to będziemy mieć $\Pr(W|A)=0$ i $\Pr(W|B)=1$, czyli prawdopodobieństwo wygranej wynosi $\Pr(W)=2/3$. Widać zatem, że powinniśmy zdecydować się na zmianę zasłony.

\subsection{} %C.2-10
Niech $A_X$, $A_Y$ i $A_Z$ będą prawdopodobieństwami wyjścia na wolność, odpowiednio, więźnia $X$, $Y$ i $Z$. Przed rozmową ze strażnikiem, prawdopodobieństwo, że $X$ będzie wolny, wynosi $\Pr(A_X)=1/3$. Jeśli $X$ dostał informację, że $Y$ zostanie ścięty, to aby zobaczyć, czy zmienia to szanse $X$ na wolność, obliczmy $\Pr(A_X|\overline{A_Y})$. Z wzoru Bayesa (C.21), mamy
\[
	\Pr(A_X|\overline{A_Y}) = \frac{\Pr(\overline{A_Y}|A_X)\Pr(A_X)}{\Pr(\overline{A_Y})}.
\]
Z kolei wiadomo, że
\begin{eqnarray*}
	\Pr(\overline{A_Y}) &=& \Pr(\overline{A_Y}|A_X)\Pr(A_X)+\Pr(\overline{A_Y}|A_Y)\Pr(A_Y)+\Pr(\overline{A_Y}|A_Z)\Pr(A_Z) \\
	&=& \Pr(\overline{A_Y}\cap A_X)+\Pr(\overline{A_Y}\cap A_Y)+\Pr(\overline{A_Y}\cap A_Z) \\
	&=& 1/3+0+1/3 = 2/3,
\end{eqnarray*}
a zatem
\[
	\Pr(A_X|\overline{A_Y}) = \frac{1\cdot (1/3)}{2/3} = \frac{1}{2}>\frac{1}{3}.
\]
Wynika stąd, że szanse więźnia $X$ na wyjście na wolność zwiększyły się po rozmowie ze strażnikiem i teraz wynoszą 1/2.

\section{Dyskretne zmienne losowe}

\subsection{} %C.3-1
Niech $X$ będzie zmienną losową oznaczającą sumę oczek na obu kostkach. Mamy
\begin{eqnarray*}
	\mathrm{E}(X) &=& \sum_{x=2}^{12}x\Pr(X=x) \\
	&=& 2\cdot\frac{1}{36}+3\cdot\frac{2}{36}+4\cdot\frac{3}{36}+5\cdot\frac{4}{36}+6\cdot\frac{5}{36}+7\cdot\frac{6}{36} \\
	&& {}+8\cdot\frac{5}{36}+9\cdot\frac{4}{36}+10\cdot\frac{3}{36}+11\cdot\frac{2}{36}+12\cdot\frac{1}{36} \\\\
	&=& 7.
\end{eqnarray*}
Niech teraz $Y$ będzie zmienną losową oznaczającą większą z liczb oczek na obu kostkach. Zachodzi
\begin{eqnarray*}
	\mathrm{E}(Y) &=& \sum_{y=1}^{6}y\Pr(Y=y) \\
	&=& 1\cdot\frac{1}{36}+2\cdot\frac{3}{36}+3\cdot\frac{5}{36}+4\cdot\frac{7}{36}+5\cdot\frac{9}{36}+6\cdot\frac{11}{36} \\\\
	&\approx& 4{,}47.
\end{eqnarray*}

\subsection{} %C.3-2
Niech $X$ będzie zmienną losową przyjmującą wartość indeksu największego elementu tablicy $A$. Zauważmy, że jeśli tablica zawiera losową permutację $n$ liczb, to $\Pr(X=x)=1/n$ dla każdego $x=1$, $2$,~$\dots$,~$n$, a zatem
\[
	\mathrm{E}(X) = \sum_{x=1}^nx\Pr(X=x) = \frac{1}{n}\sum_{x=1}^nx = \frac{1}{n}\cdot\frac{n(n+1)}{2} = \frac{n+1}{2}.
\]
Wynik jest identyczny dla każdego elementu tablicy, w szczególności także dla elementu najmniejszego.

\subsection{} %C.3-3
Zdefiniujmy zmienną losową $X$ przyjmującą wielkość wygranej w opisanej grze. Mamy obliczyć
\[
	\mathrm{E}(X) = -\Pr(A_0)+\Pr(A_1)+2\Pr(A_2)+3\Pr(A_3),
\]
przy czym $A_i$ dla $i=0$, $1$, $2$, $3$, oznacza zdarzenie, że obstawiona przez gracza liczba oczek pojawiła się na dokładnie $i$ kostkach. Prawdopodobieństwa tych zdarzeń wynoszą
\[
\begin{array}{rcccr}
	\Pr(A_0) &=& \dfrac{5^3}{6^3} &=& \dfrac{125}{216}, \\\\
	\Pr(A_1) &=& \dfrac{3\cdot 5^2}{6^3} &=& \dfrac{75}{216}, \\\\
	\Pr(A_2) &=& \dfrac{3\cdot 5^1}{6^3} &=& \dfrac{15}{216}, \\\\
	\Pr(A_3) &=& \dfrac{1}{6^3} &=& \dfrac{1}{216}.
\end{array}
\]
Dostajemy zatem
\[
	\mathrm{E}(X) = -\dfrac{17}{216} \approx -0{,}0787,
\]
a więc gracz straci w tej grze średnio prawie 8 gr.

\subsection{} %C.3-4
Załóżmy, że $\max(X,Y)=X$. Z tego, że $Y\ge0$ wynika $X+Y\ge X$, a stąd $\mathrm{E}(X+Y)\ge\mathrm{E}(X)=\mathrm{E}(\max(X,Y))$. Z liniowości wartości oczekiwanej mamy $\mathrm{E}(X+Y)=\mathrm{E}(X)+\mathrm{E}(Y)$, a zatem $\mathrm{E}(X)+\mathrm{E}(Y)\ge\mathrm{E}(\max(X,Y))$. Analogicznie dowodzi się przypadek dla $\max(X,Y)=Y$.

\subsection{} %C.3-5
Zgodnie z definicją niezależnych zmiennych losowych $X$,~$Y$, mamy
\[
	\Pr(X=x\;\;\hbox{i}\;\;Y=y) = \Pr(X=x)\Pr(Y=y).
\]
Jeśli $X$ przyjmuje pewną wartość $x$, to zmienna losowa $f(X)$ przyjmuje wartość $f(x)$. Analogicznie dla $Y$, jeśli $Y=y$, to $g(Y)=g(y)$. Powyższe równanie przyjmuje zatem postać
\[
	\Pr\bigl(f(X)=f(x)\;\;\hbox{i}\;\;g(Y)=g(y)\bigr) = \Pr\bigl(f(X)=f(x)\bigr)\Pr\bigl(g(Y)=g(y)\bigr),
\]
a stąd wnioskujemy, że $f(X)$ i $g(Y)$ są zmiennymi losowymi niezależnymi.

\subsection{} %C.3-6
Niech $A$ będzie pewnym zdarzeniem, a $I(A)$ -- zmienną losową wskaźnikową zdarzenia $A$ zdefiniowaną następująco:
\[
	I(A) =
	\begin{cases}
		0, & \hbox{jeśli }A\hbox{ zachodzi,} \\
		1, & \hbox{jeśli }A\hbox{ nie zachodzi.}
	\end{cases}
\]
Dla każdego $t>0$ prawdziwe są nierówności
\[
	X\ge X\cdot I(X\ge t)\ge t\cdot I(X\ge t).
\]
Pierwsza z nich zachodzi w oczywisty sposób, ponieważ $X\ge0$ oraz $I(A)\le1$ dla każdego zdarzenia $A$. Druga nierówność przyjmuje postać
\[
	X\cdot I(X\ge t)\ge t\cdot I(X\ge t)\;\Leftrightarrow\;
	\begin{cases}
		0\ge0, & \hbox{dla }X<t, \\
		X\ge t, & \hbox{dla }X\ge t,
	\end{cases}
\]
a więc również zachodzi. Biorąc wartości oczekiwane powyższych zmiennych losowych i korzystając z elementarnych własności wartości oczekiwanej, otrzymujemy
\[
	\mathrm{E}(X) \ge \mathrm{E}(X\cdot I(X\ge t)) \ge \mathrm{E}(t\cdot I(X\ge t)) = t\,\mathrm{E}(I(X\ge t)) = t\,\Pr(X\ge t),
\]
a stąd
\[
	\Pr(X\ge t) \le \mathrm{E}(X)/t.
\]

\subsection{} %C.3-7
Zauważmy, że
\begin{eqnarray*}
	\Pr(X\ge t) &=& \sum_{\{\,s\in S:X(s)\ge t\,\}}\Pr(s), \\
	\Pr(X'\ge t) &=& \sum_{\{\,s\in S:X'(s)\ge t\,\}}\Pr(s).
\end{eqnarray*}
Dowodzimy, że
\[
	\sum_{\{\,s\in S:X(s)\ge t\,\}}\Pr(s)\quad \ge \quad\sum_{\{\,s\in S:X'(s)\ge t\,\}}\Pr(s).
\]
Z założenia wynika, że jeśli $X'(s)\ge t$, to $X(s)\ge t$. Niech $S'\subseteq S$ będzie takim zbiorem (być może pustym), że $X'(s')<t$ i $X(s')\ge t$ dla każdego $s'\in S'$. Wtedy suma po lewej stronie powyższej nierówności zawiera o~$|S'|$ więcej składników niż suma po prawej stronie, a z tego, że $\Pr(s)\ge0$ dla dowolnego $s\in S$ mamy, że nierówność zachodzi.

\subsection{} %C.3-8
Zauważmy, że
\[
	\mathrm{Var}(X) = \mathrm{E}\bigl((X-\mathrm{E}(X))^2\bigr) \ge 0,
\]
ponieważ liczymy wartość oczekiwaną nieujemnej zmiennej losowej. Z wzoru (C.26) otrzymujemy
\[
	0 \le \mathrm{Var}(X) = \mathrm{E}(X^2)-\mathrm{E}^2(X),
\]
a więc $\mathrm{E}(X^2)\ge\mathrm{E}^2(X)$.

\subsection{} %C.3-9
Ponieważ zmienna losowa $X$ przyjmuje wartości ze zbioru $\{0,1\}$, to dla pewnego $0\le p\le1$ zachodzi
\begin{eqnarray*}
	\Pr(X=0) &=& p, \\
	\Pr(X=1) &=& 1-p.
\end{eqnarray*}
Wartością oczekiwaną $X$ jest $\mathrm{E}(X)=0\cdot p+1\cdot(1-p)=1-p$. Zauważmy ponadto, że $\mathrm{E}(X^2)=\mathrm{E}(X)$ i obliczmy wariancję zmiennej losowej $X$,
\begin{eqnarray*}
	\mathrm{Var}(X) &=& \mathrm{E}(X^2)-\mathrm{E}^2(X) \\
	&=& (1-p)-(1-p)^2 \\
	&=& p(1-p) \\
	&=& \mathrm{E}(X)(1-\mathrm{E}(X)) \\
	&=& \mathrm{E}(X)\mathrm{E}(1-X),
\end{eqnarray*}
przy czym ostatnia równość zachodzi dzięki liniowości wartości oczekiwanej.

\subsection{} %C.3-10
\begin{eqnarray*}
	\mathrm{Var}(aX) &=& \mathrm{E}(a^2X^2)-\mathrm{E}^2(aX) \\
	&=& a^2\mathrm{E}(X^2)-(a\mathrm{E}(X))^2 \\
	&=& a^2\bigl(\mathrm{E}(X^2)-\mathrm{E}(X)\bigr) \\
	&=& a^2\mathrm{Var}(X).
\end{eqnarray*}

\section{Rozkłady: geometryczny i dwumianowy}
\subsection{} %C.4-1
Rodzina zdarzeń elementarnych $S$ zawiera zdarzenia, które wymagają $k$ prób zanim nastąpi pierwszy sukces dla każdego $k=1$, $2$,~$\dots$, mamy zatem
\begin{eqnarray*}
	\Pr(S) &=& \sum_{k=1}^\infty q^{k-1}p \\
	&=& p\cdot\sum_{k=0}^\infty (1-p)^k \\
	&=& \frac{p}{1-(1-p)} \\\\
	&=& 1.
\end{eqnarray*}
Wykorzystano wzór (A.6) przy założeniu, że $p>0$.

\subsection{} %C.4-2
Niech sukces oznacza uzyskanie w rzucie sześcioma monetami trzech orłów i trzech reszek, a porażka -- każdy inny wynik. Możemy wybrać dowolne trzy monety spośród sześciu, na których będzie orzeł, mamy zatem $\binom{6}{3}=20$ sposobów osiągnięcia sukcesu. Jest $2^6=64$ wszystkich możliwych wyników, zatem prawdopodobieństwo sukcesu wynosi $p=20/64=5/16$. Z wzoru (C.31) otrzymujemy, że musimy wykonać średnio $1/p=3{,}2$ rzutów.

\subsection{} %C.4-3
Z definicji rodziny rozkładów dwumianowych,
\begin{eqnarray*}
	b(k;n,p) &=& \binom{n}{k}p^k(1-p)^{n-k}, \\
	b(n-k;n,q) &=& \binom{n}{n-k}q^{n-k}(1-q)^k.
\end{eqnarray*}
Ponieważ $\binom{n}{n-k}=\binom{n}{k}$ (z wzoru (C.3)) oraz $q=1-p$, otrzymujemy $b(k;n,p)=b(n-k;n,q)$.

\subsection{} %C.4-4
Ponieważ rozkład dwumianowy przyjmuje maksimum dla pewnego $k$ całkowitego z przedziału $np-q\le k\le(n+1)p$, to dobrym przybliżeniem wartości maksymalnej jest wartość dla $k=np$, które oczywiście leży w tym przedziale.
\begin{eqnarray*}
	b(np;n,p) &=& \binom{n}{np}p^{np}(1-p)^{n-np} \\
	&=& \frac{n!}{(np)!\,(n-np)!}\,p^{np}(1-p)^{n-np} \\
	&=& \frac{n!}{(np)!\,(nq)!}\,p^{np}q^{nq}.
\end{eqnarray*}
Wykorzystując wzór Stirlinga do przybliżenia silni, możemy uprościć pierwszy czynnik następująco:
\begin{eqnarray*}
	\frac{n!}{(np)!\,(nq)!} &\approx& \frac{\sqrt{2\pi n}\left(\frac{n}{e}\right)^n}{\sqrt{2\pi np}\left(\frac{np}{e}\right)^{np}\sqrt{2\pi nq}\left(\frac{nq}{e}\right)^{nq}} \\\\
	&=& \frac{\left(\frac{n}{e}\right)^n\left(\frac{e}{np}\right)^{np}\left(\frac{e}{nq}\right)^{nq}}{\sqrt{2\pi npq}} \\\\
	&=& \frac{1}{p^{np}q^{nq}\sqrt{2\pi npq}}.
\end{eqnarray*}
Stąd dostajemy przybliżenie
\[
	b(np;p,n) \approx \frac{1}{\sqrt{2\pi npq}}
\]
na maksymalną wartość rozkładu dwumianowego $b(k;n,p)$.

\subsection{} %C.4-5
Niech $X$ będzie zmienną losową przyjmującą liczbę sukcesów w tym doświadczeniu losowym. Wtedy
\[
	\Pr(X=0) = b\Bigl(0;n,\frac{1}{n}\Bigr) = \left(1-\frac{1}{n}\right)^n,
\]
co dąży do $1/e$ wraz ze wzrostem $n$. Wynika to stąd, że ciąg $e_n=\left(1+\frac{k}{n}\right)^n$ ma dla dowolnej stałej $k$, granicę równą $e^k$ dla $n$ dążącego do $\infty$.
Analogicznie,
\[
	\Pr(X=1) = b\Bigl(1;n,\frac{1}{n}\Bigr) = \frac{\left(1-\frac{1}{n}\right)^n}{1-\frac{1}{n}}
\]
dąży do $1/e$, ponieważ mianownik zbliża się do 1 wraz ze wzrostem $n$.

\subsection{} %C.4-6
Obliczymy prawdopodobieństwo uzyskania przez profesorów równej liczby orłów na dwa sposoby. W pierwszym z nich, niech $X$ i $Y$ będą zmiennymi losowymi oznaczającymi liczby orłów uzyskane kolejno przez obu profesorów. Prawdopodobieństwo uzyskania $k$ orłów ($0\le k\le n$) przez każdego z nich jest rozkładem dwumianowym, $\Pr(X=k)=\Pr(Y=k)=b(k;n,1/2)$. Zdarzenia $X=k$ i $Y=k$ są niezależne, zatem prawdopodobieństwo uzyskania przez obu profesorów równej ilości orłów wynosi
\begin{eqnarray*}
	\sum_{k=0}^n\Pr(X=k\;\;\hbox{i}\;\;Y=k) &=& \sum_{k=0}^n\Pr(X=k)\Pr(Y=k) \\
	&=& \sum_{k=0}^n\binom{n}{k}\left(\frac{1}{2}\right)^n\binom{n}{k}\left(\frac{1}{2}\right)^n \\
	&=& \frac{\sum_{k=0}^n\binom{n}{k}^2}{4^n}.
\end{eqnarray*}

W drugim sposobie potraktujmy wynik każdego doświadczenia jako $2n$-elementowy ciąg wyników taki, że początkowych $n$ wyrazów oznacza wyniki uzyskane przez profesora Rosencrantza, a $n$ końcowych -- wyniki profesora Guildensterna. Na każdej pozycji znajduje się jedna z dwóch wartości, orzeł lub reszka, zatem wszystkich możliwych ciągów jest $2^{2n}=4^n$. Niech sukcesem dla profesora Rosencrantza będzie uzyskanie orła, a dla profesora Guildensterna -- uzyskanie reszki. Zauważmy, że wyrzucenie równej liczby orłów przez obu profesorów jest równoważne z osiągnięciem przez nich w sumie $n$ sukcesów. Liczba sposobów, na jakie można to zrobić, jest liczbą możliwości wyboru spośród $2n$ pozycji ciągu, $n$ odpowiedzialnych za sukces, która to wynosi $\binom{2n}{n}$, a zatem szukane prawdopodobieństwo jest równe
\[
	\frac{\binom{2n}{n}}{4^n}.
\]

Przyrównując do siebie wyniki z obu sposobów, dostajemy tożsamość
\[
	\sum_{k=0}^n\binom{n}{k}^2 = \binom{2n}{n}.
\]

\subsection{} %C.4-7
Wykorzystując nierówność
\[
	\binom{n}{\lambda n} \le 2^{nH(\lambda)},
\]
otrzymujemy
\[
	b\Bigl(k;n,\frac{1}{2}\Bigr) = \binom{n}{k}\left(\frac{1}{2}\right)^n = \frac{\binom{n}{k}}{2^n} \le \frac{2^{nH(n/k)}}{2^n} = 2^{nH(n/k)-n}.
\]

%poczyniono zalozenie ze p_i>0 i p>0
\subsection{} %C.4-8
% Na wstępie zauważmy, że $p\ge p_i$ dla każdego $i=1,2,\dots,n$ implikuje
% \[
% 	\frac{1}{1-p_i}\le\frac{1}{1-p}\quad\hbox{oraz}\quad\frac{p_i}{1-p_i}\le\frac{p}{1-p}.
% \]
% Ponieważ przy potęgowaniu obu stron nierówności oraz przy mnożeniu ich stronami (obie strony są nieujemne) nie zmienia się znak nierówności, to możemy napisać
% \[
% 	\left(\frac{p_i}{1-p_i}\right)^i\left(\frac{1}{1-p_i}\right)^n \le \left(\frac{p}{1-p}\right)^i\left(\frac{1}{1-p}\right)^n,
% \]
% a stąd
% \[
% 	\binom{n}{i}p_i^i(1-p_i)^{n-i} \le \binom{n}{i}p^i(1-p)^{n-i}.
% \]
% Sumując powyższe nierówności po wszystkich $i$, mamy
% \[
% 	\sum_{i=0}^{k-1}\binom{n}{i}p_i^i(1-p_i)^{n-i} \le \sum_{i=0}^{k-1}\binom{n}{i}p^i(1-p)^{n-i},
% \]
% co kończy dowód.

\subsection{} %C.4-9

\section{Krańce rozkładu dwumianowego}

\subsection{} %C.5-1
Niech $X$ i $Y$ będą zmiennymi losowymi przyjmującymi liczby uzyskanych orłów, kolejno w obu zdarzeniach. Mamy zatem
\begin{eqnarray*}
	\Pr(X=0) &=& b(0;n,1/2) = \binom{n}{0}\left(\frac{1}{2}\right)^0\left(\frac{1}{2}\right)^n = \left(\frac{1}{2}\right)^n, \\
	\Pr(Y<n) &=& b(n;4n,1/2) \\
	&<& \frac{\frac{n}{2}}{\frac{4n}{2}-n}\,b(n;4n,1/2) \\
	&<& \binom{4n}{n}\left(\frac{1}{2}\right)^n\left(\frac{1}{2}\right)^{3n} \\
	&=& \frac{(4n)!}{(3n)!\cdot n!\cdot 2^{4n}} \\
	&=& \frac{(3n+1)(3n+2)\ldots(4n-1)(4n)}{n!}\cdot\left(\frac{1}{2}\right)^{4n}.
\end{eqnarray*}
Przy obliczeniu ostatniego prawdopodobieństwa wykorzystano twierdzenie C.4. Zauważmy, że w powyższym ułamku licznik ma $n$ czynników, zatem można potraktować ułamek jako
\[
	\left(\frac{3n+1}{1}\right)\left(\frac{3n+2}{2}\right)\ldots\left(\frac{4n-1}{n-1}\right)\left(\frac{4n}{n}\right).
\]
Wszystkie czynniki w powyższym iloczynie są ograniczone od góry przez 4, zatem dostajemy ograniczenie
\[
	\Pr(B) < 4^n\left(\frac{1}{2}\right)^{4n} = \left(\frac{1}{4}\right)^n.
\]
Stąd mamy, że $\Pr(A)>\Pr(B)$, a zatem uzyskanie mniej niż $n$ orłów w $4n$ rzutach monetą jest mniej prawdopodobne od nieuzyskania żadnego orła w $n$ rzutach monetą.

\subsection{} %C.5-2
\noindent\emph{Dowód wniosku C.6.}

\noindent Ponieważ uzyskanie co najwyżej $k$ sukcesów w $n$ próbach jest równoważne usyskaniu co najmniej $n-k$ porażek, to zachodzi $\Pr(X\le k)=\Pr(Y\ge n-k)$, gdzie $Y$ jest zmienną losową oznaczającą liczbę porażek. Z tw. C.2 mamy
\begin{eqnarray*}
	\Pr(X\le k) = \Pr(Y\ge n-k) &=& \sum_{i=n-k}^nb(i;n,q) \\
	&<& \binom{n}{n-k}q^{n-k} \\
	&=& \binom{n}{k}(1-p)^{n-k}
\end{eqnarray*}
na podstawie tego, że $q=1-p$ i tożsamości C.3.\\

\noindent\emph{Dowód wniosku C.7.}

\noindent Podobnie, przy oznaczeniach w poprzedniego dowodu, mamy $\Pr(X>k)=\Pr(Y<n-k)$, a zatem z tw. C.4:
\begin{eqnarray*}
	\Pr(X>k) = \Pr(Y<n-k) &=& \sum_{i=0}^{n-k-1}b(i;n,q) \\
	&<& \frac{(n-k)p}{nq-(n-k)}\,b(n-k;n,q) \\
	&=& \frac{(n-k)p}{k-np}\binom{n}{n-k}q^{n-k}(1-q)^k \\
	&=& \frac{(n-k)p}{k-np}\binom{n}{k}(1-p)^{n-k}p^k \\
	&=& \frac{(n-k)p}{k-np}\,b(k;n,p).
\end{eqnarray*}

\subsection{} %C.5-3
Niech $p=\frac{a}{a+1}$, skąd mamy, że $a=\frac{p}{1-p}$. Zachodzi
\[
	\sum_{i=0}^{k-1}\binom{n}{i}a^i = \sum_{i=0}^{k-1}\binom{n}{i}\left(\frac{p}{1-p}\right)^i = \frac{\sum_{i=0}^{k-1}b(i;n,p)}{(1-p)^n},
\]
zatem z tw. C.4 otrzymujemy:
\begin{eqnarray*}
	\sum_{i=0}^{k-1}\binom{n}{i}a^i &<& \frac{\frac{k}{a+1}}{\left(\frac{1}{a+1}\right)^n\left(\frac{na}{a+1}-k\right)}\,b\Bigl(k;n,\frac{a}{a+1}\Bigr) \\
	&=& (a+1)^n\frac{k}{na-k(a+1)}\,b\Bigl(k;n,\frac{a}{a+1}\Bigr).
\end{eqnarray*}

\subsection{} %C.5-4
\subsection{} %C.5-5
Przeprowadzimy dowód analogicznie, jak przebiegał dowód tw. C.6. Dla dowolnego $\alpha>0$ mamy
\[
	\Pr(\mu-X\ge r) = \Pr\bigl(e^{\alpha(\mu-X)}\ge e^{\alpha r}\bigr)
\]
i z nierówności Markowa dostajemy
\[
	\Pr(\mu-X\ge r) = \mathrm{E}\bigl(e^{\alpha(\mu-X)}\bigr)e^{-\alpha r}.
\]
Niech $X_i$ dla $i=1$, $2$,~$\dots$,~$n$ będzie zmienną losową przyjmującą 1, jeśli wynikiem $i$-tej próby Bernoulliego jest sukces i 0 w przeciwnym przypadku. Wtedy
\[
	X = \sum_{i=1}^nX_i
\]
oraz
\[
	\mu-X = \sum_{i=1}^n(p_i-X_i),
\]
otrzymujemy zatem
\[
	\mathrm{E}\bigl(e^{\alpha(\mu-X)}\bigr) = \mathrm{E}\biggl(\prod_{i=1}^ne^{\alpha(p_i-X_i)}\biggr) = \prod_{i=1}^n\mathrm{E}\bigl(e^{\alpha(p_i-X_i)}\bigr),
\]
co wynika z wzajemnej niezależności zmiennych losowych $X_i$, a co za tym idzie także zmiennych losowych $e^{\alpha(p_i-X_i)}$. Z definicji wartości oczekiwanej mamy
\begin{eqnarray*}
	\mathrm{E}\bigl(e^{\alpha(p_i-X_i)}\bigr) &=& e^{\alpha(p_i-1)}p_i+e^{\alpha(p_i-0)}q_i \\
	&=& p_ie^{-\alpha q_i}+q_ie^{\alpha p_i} \\
	&\le& q_ie^{\alpha}+1 \\
	&\le& \exp(q_ie^\alpha),
\end{eqnarray*}
skąd zachodzi
\[
	\mathrm{E}\bigl(e^{\alpha(\mu-X)}\bigr) \le \exp\biggl(\prod_{i=1}^nq_ie^\alpha\biggr) = \exp((n-\mu)e^\alpha),
\]
bo $\mu=\sum_{i=1}^np_i$, a więc $\sum_{i=1}^nq_i=n-\mu$. Wracając do oszacowania prawdopodobieństwa dostajemy
\[
	\Pr(\mu-X\ge r) \le \exp((n-\mu)e^\alpha-\alpha r).
\]
Ponieważ we wzorze (C.45) wybranie $\alpha=\ln(r/\mu)$ minimalizuje prawą stronę tego wzoru, to w naszym przypadku po prawej stronie zamiast $\mu$ mamy $n-\mu$, a więc przyjmujemy $\alpha=\ln(r/(n-\mu))$. Dostajemy ostatecznie
\begin{eqnarray*}
	\Pr(\mu-X\ge r) &\le& \exp\left((n-\mu)e^{\ln\frac{r}{n-\mu}}-r\ln\frac{r}{n-\mu}\right) \\
	&=& \exp\left(r-r\ln\frac{r}{n-\mu}\right) \\
	&=& \frac{e^r}{\left(\frac{r}{n-\mu}\right)^r} \\
	&=& \left(\frac{(n-\mu)e}{r}\right)^r.
\end{eqnarray*}

\subsection{} %C.5-6
\subsection{} %C.5-7
Potraktujmy wyrażenie jako funkcję $f$ zmiennej $\alpha$:
\[
	f(\alpha) = \exp(\mu e^\alpha-\alpha r).
\]
W celu wyznaczenia jej minimum, obliczmy pierwszą i drugą pochodną:
\begin{eqnarray*}
	\frac{df(\alpha)}{d\alpha} &=& (\mu e^\alpha-r)\exp(\mu e^\alpha-\alpha r), \\
	\frac{d^2f(\alpha)}{d\alpha^2} &=& \left(\mu e^\alpha+(\mu e^\alpha-r)^2\right)\exp(\mu e^\alpha-\alpha r).
\end{eqnarray*}
Przyrównując pierwszą pochodną do 0, otrzymujemy, że w punkcie $\alpha_0=\ln(r/\mu)$ może istnieć ekstremum $f$. Po obliczeniu wartości drugiej pochodnej w tym punkcie dostajemy
\[
	\frac{d^2f(\alpha_0)}{d\alpha^2} = r\exp(r-r\ln(r/\mu)) > 0,
\]
ponieważ $\exp(x)$ jest rosnące oraz $r>\mu\ge0$, a zatem w punkcie $\ln(r/\mu)$ istnieje minimum funkcji $f$.

\problems

\subsection{} %C-1

\subsubsection{} %C-1(a)
Każda kula trafia do jednej z $b$ urn. Jest $b$ sposobów umieszczenia pierwszej kuli, na każdy z nich przypada $b$ sposobów umieszczenia drugiej kuli, itd. Jest zatem $b^n$ sposobów rozmieszczenia $n$ różnych kul w $b$ różnych urnach.

\subsubsection{} %C-1(b) %%%zmienic na bardziej podobne do hinta
Rozważmy ciąg $(b+n-1)$-wyrazowy w zbiorze kul i urn opisujący pewne rozmieszczenie. Do każdego z jego wyrazów możemy przypisać jedną z $n$ kul lub jedną z $b-1$ urn. Jeśli urna $B$ znajduje się na pozycji $i$ w tym ciągu, a~kolejna najbliższa urna w ciągu jest na pozycji $j>i$, to kule na pozycjach $i+1$, $i+2$,~$\dots$,~$j-1$ znajdują się w urnie $B$ w bieżącym rozmieszczeniu. Początkowe wyrazy ciągu, będące kulami należą do brakującej $b$-tej urny. Ostatnia urna w~ciągu zawiera kule znajdujące się za nią, aż do końca ciągu.

Wszystkich takich ciągów jest $(b+n-1)!$, ale dowolna permutacja $b-1$ par (urna, zbiór zawieranych kul) daje identyczne rozmieszczenie, zatem istnieje $\frac{(b+n-1)!}{(b-1)!}$ różnych rozmieszczeń kul w urnach.

\subsubsection{} %C-1(c)
Sytuacja jest podobna jak w punkcie (b) z tą różnicą, że nie rozróżniamy kul między sobą, a więc również każda permutacja $n$ kul między sobą opisuje ten sam sposób rozmieszczenia kul w urnach. Mamy zatem $\frac{(b+n-1)!}{n!\,(b-1)!} = \binom{b+n-1}{n}$ możliwości rozmieszczenia kul.

\subsubsection{} %C-1(d)
Zakładając, że $n\le b$, wybieramy spośród $b$ urn $n$ takich, które będą zawierać po jednej kuli. Jest $\binom{b}{n}$ sposobów ich wyboru.

\subsubsection{} %C-1(e)
Zakładamy, że $n\ge b$. Najpierw umieszczamy $b$ kul, po jednej w każdej urnie tak, aby żadna urna nie była pusta. Pozostałe $n-b$ kul możemy umieścić w $b$ urnach, zgodnie z punktem (c), na $\binom{b+(n-b)-1}{n-b}=\binom{n-1}{n-b}=\binom{n-1}{b-1}$ sposobów.


\backmatter

\let\chapter=\mychapter

\bibliographystyle{plplain}
\bibliography{cormensol}
\nocite{*}

\end{document}
