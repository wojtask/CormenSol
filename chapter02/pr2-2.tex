\problem{Poprawność sortowania bąbelkowego} %2-2

\subproblem %2-2(a)
Należy jeszcze pokazać, że tablica $A'$ stanowi permutację tablicy $A$.

\subproblem %2-2(b)
Niezmiennik wewnętrznej pętli \kw{for}:
\begin{quote}
Przed każdą iteracją pętli \kw{for} w~wierszach 2\nbendash4 najmniejszym elementem podtablicy $A[j\twodots\id{length}[A]]$ jest $A[j]$.
\end{quote}
\begin{description}
	\item[Inicjowanie:] Przed pierwszą iteracją $j=\attrib{A}{length}$, więc $A[j\twodots\attrib{A}{length}]$ zawiera tylko jeden element, który oczywiście jest najmniejszy w~tej podtablicy i~jest nim $A[j]$.
	\item[Utrzymanie:] Załóżmy, że $A[j]$ jest najmniejszym elementem w~$A[j\twodots\attrib{A}{length}]$.
Jeżeli $A[j-1]$ jest większe od $A[j]$, to $A[j]$ jest zamieniane z~$A[j-1]$ w~wierszu 4, więc w~tym momencie podtablica $A[j-1\twodots\attrib{A}{length}]$ posiada swój najmniejszy element w~$A[j-1]$.
Uaktualnienie $j$ powoduje odtworzenie niezmiennika.
W~przeciwnym przypadku zamiana nie następuje, przez co $A[j-1]$ stanowi najmniejszy element $A[j-1\twodots\attrib{A}{length}]$ i~aktualizacja $j$ także pozwala spełnić niezmiennik.
	\item[Zakończenie:] Po zakończeniu wykonywania pętli zachodzi $j=i$, a~więc $A[i]$ jest najmniejszym elementem podtablicy $A[i\twodots\attrib{A}{length}]$.
\end{description}

\subproblem %2-2(c)
Niezmiennik zewnętrznej pętli \kw{for}:
\begin{quote}
Przed każdą iteracją pętli \kw{for} w~wierszach 1\nbendash4 podtablica $A[1\twodots i-1]$ zawiera $i-1$ najmniejszych elementów tablicy $A$ uporządkowanych niemalejąco.
\end{quote}
\begin{description}
	\item[Inicjowanie:] Przed pierwszą iteracją $i=1$, czyli podtablica $A[1\twodots i-1]$ jest pusta, a~więc jest trywialnie posortowana.
	\item[Utrzymanie:] Z~założenia o~uporządkowaniu podtablicy $A[1\twodots i-1]$ wynika, że $A[i-1]$ stanowi jej największy element.
Wewnętrzna pętla \kw{for} wyszukuje najmniejszy element podtablicy $A[i\twodots\attrib{A}{length}]$ i~umieszcza go na pozycji $i$ (dowód w~poprzednim punkcie).
Z~założenia mamy też, że w~podtablicy $A[i\twodots\attrib{A}{length}]$ nie ma mniejszych elementów od $A[i-1]$, a~zatem w~szczególności zachodzi $A[i-1]\le A[i]$.
Stąd wnioskujemy, że podtablica $A[1\twodots i]$ zawiera $i$ najmniejszych elementów tablicy $A$ posortowanych niemalejąco i~po aktualizacji $i$ niezmiennik zostaje odtworzony.
	\item[Zakończenie:] Na końcu mamy $i=\attrib{A}{length}+1$.
Podtablica $A[1\twodots i-1]$ jest całą tablicą $A$ posortowaną niemalejąco, a~zatem algorytm sortuje poprawnie.
\end{description}

\subproblem %2-2(d)
Niech $n=\attrib{A}{length}$.
Dla wszystkich przypadków danych wejściowych pętla \kw{for} z~wierszy 2\nbendash4 wykonuje $n-i$ iteracji dla każdego $i=1$, 2, \dots, $n$.
Pesymistyczny czas działania sortowania bąbelkowego wynosi zatem
\[
	T(n) = \sum_{i=1}^n(n-i) = \sum_{i=0}^{n-1}i = \frac{n(n-1)}{2} = \Theta(n^2),
\]
jest on więc równy pesymistycznemu czasowi sortowania przez wstawianie.
