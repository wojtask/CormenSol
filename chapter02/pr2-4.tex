\problem{Inwersje} %2-4

\subproblem %2-4(a)
$\langle1,5\rangle$, $\langle2,5\rangle$, $\langle3,4\rangle$, $\langle3,5\rangle$, $\langle4,5\rangle$

\subproblem %2-4(b)
Dla zbioru $\{1,2,\dots,n\}$ tablicą o~największej możliwej liczbie inwersji jest tablica posortowana malejąco, czyli $\langle n,n-1,\dots,1\rangle$.
Inwersją jest wtedy każda para $\langle i,j\rangle$, gdzie $1\le i<j\le n$.
Łącznie w~tej tablicy jest zatem $\binom{n}{2}=n(n-1)/2$ inwersji.

\subproblem %2-4(c)
Dla danej tablicy $A[1\twodots n]$ przez $\mathrm{inv}(j)$ oznaczmy liczbę inwersji postaci $\langle i,j\rangle$.
W~procedurze \proc{Insertion-Sort} pętla \kw{while} w~wierszach 5\nbendash7 przenosi element znajdujący się początkowo na pozycji $j$ o~$\mathrm{inv}(j)$ pozycji w~lewo, dzięki czemu wyeliminowanych zostaje $\mathrm{inv}(j)$ inwersji.
Pętla \kw{while} działa dla każdego $j=2$, 3, \dots, $n$, wykonując łącznie $I=\sum_{j=2}^n\mathrm{inv}(j)$ iteracji, czyli dokładnie tyle, ile inwersji było w~wejściowej tablicy.
Z~tej analizy płynie wniosek, że czas działania sortowania przez wstawianie wynosi $\Theta(n+I)$.

\subproblem %2-4(d)
Niech $a$ i~$b$ będą dwoma różnymi elementami tablicy $A$.
Załóżmy, że podczas sortowania przez scalanie w~procedurze \proc{Merge} w~pewnym momencie $L[i]=a$ oraz $R[j]=b$.
Jeśli warunek z~wiersza 13 procedury \proc{Merge} zachodzi, to znaczy, że $a$ i~$b$ nie tworzą inwersji.
W~przeciwnym przypadku $a>b$, a~ponieważ scalane podtablice są posortowane, to $b$ jest mniejsze od każdego dotychczas nieprzetworzonego elementu podtablicy $L$.
Liczba elementów $A$ należących do $L$ wynosi $n_1$, zatem w~momencie przetwarzania elementu $b$ jest w~niej $n_1-i+1$ elementów nieprzetworzonych, a~więc tyle inwersji tworzy z~nimi $b$.
Od tego momentu element ten będzie z~nimi w~jednej podtablicy, więc nie policzymy żadnej inwersji dwukrotnie.

W~ten sposób, modyfikując algorytm sortowania przez scalanie, wyznaczamy liczbę inwersji $n$\nbhyphen elementowej tablicy $A$ w~czasie $\Theta(n\lg n)$, czego efektem ubocznym jest jej posortowanie.
W~procedurze \proc{Merge-Sort} wystarczy początkowo wyzerować licznik inwersji i~następnie sumować częściowe wyniki zwracane z~wywołań rekurencyjnych, a~w~\proc{Merge} -- doliczać odpowiednią liczbę inwersji tworzonych przez element $b$.
Poniższe pseudokody implementują to podejście.

\begin{codebox}
\Procname{$\proc{Count-Inversions}(A,p,r)$}
\li	$\id{inversions}\gets0$
\li	\If $p<r$
\li		\Then $q\gets\lfloor(p+r)/2\rfloor$
\li			$\id{inversions}\gets\id{inversions}+\proc{Count-Inversions}(A,p,q)$
\li			$\id{inversions}\gets\id{inversions}+\proc{Count-Inversions}(A,q+1,r)$
\li			$\id{inversions}\gets\id{inversions}+\proc{Merge-Inversions}(A,p,q,r)$
		\End
\li	\Return \id{inversions}
\end{codebox}

\begin{codebox}
\Procname{$\proc{Merge-Inversions}(A,p,q,r)$}
\li	$n_1\gets q-p+1$
\li	$n_2\gets r-q$
\li	utwórz tablice $L[1\twodots n_1+1]$ i~$R[1\twodots n_2+1]$
\li	\For $i\gets1$ \To $n_1$
\li		\Do $L[i]\gets A[p+i-1]$
		\End
\li	\For $j\gets1$ \To $n_2$
\li		\Do $R[j]\gets A[q+j]$
		\End
\li	$L[n_1+1]\gets R[n_2+1]\gets\infty$
\li	$i\gets1$
\li $j\gets1$
\li	$\id{inversions}\gets0$
\li	\For $k\gets p$ \To $r$
\li		\Do \If $L[i]\le R[j]$
\li				\Then $A[k]\gets L[i]$
\li					$i\gets i+1$
\li				\Else $A[k]\gets R[j]$
\li					$j\gets j+1$
\li					$\id{inversions}\gets\id{inversions}+n_1-i+1$
				\End
		\End
\li	\Return \id{inversions}
\end{codebox}
Aby wyznaczyć liczbę inwersji tablicy $A[1\twodots n]$, należy wywołać $\proc{Count-Inversions}(A,1,n)$.
