\problem{Poprawność schematu Hornera} %2-3

\subproblem %2-3(a)
Pętla \kw{while} w~wierszach 3\nbendash5 wykonuje $n+1$ iteracji, więc czas działania tego fragmentu kodu wynosi $\Theta(n)$.

\subproblem %2-3(b)
Następujący fragment kodu oblicza wartość $P(x)$ dla zadanych współczynników $a_0$, $a_1$, \dots, $a_n$ wielomianu $P$ i~wartości $x$:
\begin{codebox}
\li	$y\gets0$
\li	\For $i\gets0$ \To $n$
\li		\Do $s\gets a_i$
\li			\For $j\gets1$ \To $i$ \label{li:naive-polynomial-evaluation-for-begin}
\li				\Do $s\gets s\cdot x$
				\End \label{li:naive-polynomial-evaluation-for-end}
\li			$y\gets y+s$
		\End
\end{codebox}

Pętla \kw{for} w~wierszach \ref{li:naive-polynomial-evaluation-for-begin}\nbendash\ref{li:naive-polynomial-evaluation-for-end} wykonuje się $i$ razy dla każdego $i=0$, 1, \dots, $n$, czyli łącznie $\sum_{i=0}^ni=n(n+1)/2$ razy.
Czas działania powyższego kodu wynosi zatem $\Theta(n^2)$.
Jest to więc mniej efektywny sposób obliczania wartości wielomianu od schematu Hornera.

\subproblem %2-3(c)
Dowodzimy w~trzech krokach:
\begin{description}
	\item[Inicjowanie:] Przed pierwszą iteracją $i=n$, więc
	\[
	    y = \sum_{k=0}^{n-(i+1)}a_{k+i+1}x^k = \sum_{k=0}^{-1}a_{k+n+1}x^k = 0,
	\]
	co zgadza się z~początkową wartością $y$.
	\item[Utrzymanie:] Podczas każdej następnej iteracji $y$ przyjmuje wartość $a_i+xy$.
Przy założeniu, że niezmiennik jest spełniony przed bieżącą iteracją, mamy
	\[
		y = a_i+\sum_{k=0}^{n-(i+1)}a_{k+i+1}x^{k+1} = a_ix^0+\sum_{k=1}^{n-i}a_{k+i}x^k = \sum_{k=0}^{n-i}a_{k+i}x^k
	\]
	i~po aktualizacji $i$ niezmiennik zostaje odtworzony.
	\item[Zakończenie:] Tuż po zakończeniu pętli mamy $i=-1$, więc
	\[
		y = \sum_{k=0}^{n-(i+1)}a_{k+i+1}x^k = \sum_{k=0}^na_kx^k = P(x),
	\]
	zatem algorytm poprawnie oblicza wynik.
\end{description}

\subproblem %2-3(d)
Algorytm zwraca poprawny wynik, gdyż ustawia prawidłowe początkowe wartości, $y=0$ oraz $i=n$, a~poprawność pętli \kw{while} została wykazana w~poprzednim punkcie.
Procedura posiada własność stopu, ponieważ zmienna $i$ jest zmniejszana w~kolejnych iteracjach pętli, zatem po skończonej liczbie iteracji i~po skończonej liczbie kroków algorytmu będzie zachodzić $i=0$, co jest warunkiem zakończenia pętli.
Algorytm działa więc poprawnie.
