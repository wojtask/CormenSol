\subchapter{Analiza algorytmów}

\exercise %2.2-1
Najbardziej znaczącym składnikiem wyrażenia jest $n^3\!/1000$, a~więc funkcja jest rzędu $\Theta(n^3)$.

\exercise %2.2-2
Poniższy algorytm implementuje sortowanie przez wybieranie:
\begin{codebox}
\Procname{$\proc{Selection-Sort}(A)$}
\li	$n\gets\attrib{A}{length}$
\li	\For $j\gets1$ \To $n-1$ \label{li:selection-sort-for-begin}
\li		\Do $\id{min}\gets j$
\li			\For $i\gets j+1$ \To $n$
\li				\Do \If $A[i]<A[\id{min}]$
\li					\Then $\id{min}\gets i$
					\End
				\End
\li			zamień $A[\id{min}]\leftrightarrow A[j]$
		\End \label{li:selection-sort-for-end}
\end{codebox}
Zewnętrzna pętla algorytmu zachowuje następujący niezmiennik:
\begin{quote}
Na początku każdej iteracji pętli \kw{for} w~wierszach \ref{li:selection-sort-for-begin}\nbendash\ref{li:selection-sort-for-end} podtablica $A[1\twodots j-1]$ jest posortowana niemalejąco i~zawiera $j-1$ najmniejszych elementów znajdujących się początkowo w~tablicy $A$.
\end{quote}

Nie trzeba wykonywać $n$ iteracji pętli \kw{for} z~wierszy \ref{li:selection-sort-for-begin}\nbendash\ref{li:selection-sort-for-end}, gdyż po jej zakończeniu (po $n-1$ iteracjach) fragment $A[1\twodots n-1]$ zawiera $n-1$ najmniejszych elementów tablicy $A$ w~porządku niemalejącym, zatem element $A[n]$ jest większy lub równy względem każdego elementu z~podtablicy $A[1\twodots n-1]$, a~to oznacza, że cała tablica pozostaje posortowana niemalejąco.

Przeprowadzanych jest $n-1$ iteracji zewnętrznej pętli \kw{for}, a~wewnętrzna pętla \kw{for} iteruje po wszystkich elementach aktualnie nieposortowanego fragmentu tablicy, szukając jego minimalnego elementu.
Łącznie wykonywanych jest więc
\[
	\sum_{j=1}^{n-1}(n-j) = \sum_{i=1}^{n-1}j = \frac{n(n-1)}{2}
\]
iteracji wewnętrznej pętli \kw{for}, zatem zarówno pesymistyczny, jak i~optymistyczny czas działania algorytmu wynosi $\Theta(n^2)$.

\exercise %2.2-3
Z~\refExercise{C.3-2} mamy, że w~średnim przypadku należy sprawdzić $(n+1)/2$ elementów tablicy, zatem średni czas działania algorytmu wyszukiwania liniowego wynosi $\Theta(n)$.
W~przypadku pesymistycznym procedura sprawdza wszystkie $n$ elementów, nie znajdując szukanego, a~więc otrzymujemy ten sam wynik.

\exercise %2.2-4
Na początku działania algorytmu można wykrywać egzemplarze danych wejściowych, które stanowią dla niego przypadek optymistyczny i~wykonywać określone kroki dla takiego przypadku, ewentualnie zwracając uprzednio przygotowane wyniki.
Na przykład w~sortowaniu przez wstawianie, zanim uruchomiona zostanie właściwa procedura sortowania, w~czasie linowym względem długości tablicy wejściowej można weryfikować, czy jest już ona uporządkowana niemalejąco.
W~przypadku pozytywnym algorytm mógłby od razu zakończyć działanie.
