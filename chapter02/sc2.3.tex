\subchapter{Projektowanie algorytmów}

\exercise %2.3-1
Na rys.\ \ref{fig:2.3-1} przedstawiono działanie algorytmu sortowania przez scalanie dla tablicy $A$.
\begin{figure}[!ht]
	\centering \begin{tikzpicture}[
	every node/.style = {draw, semithick, light grayed, minimum size=6mm, inner sep=4pt, text height=1.5ex},
	level/.style = {level distance=15mm, sibling distance=100mm/2^#1, child anchor=north, edge from parent/.style = {arrow, <-, very thick, shorten <=4mm-#1mm, shorten >=4mm-#1mm}},
	index node/.append style = {draw=none, fill=none}
]

\def\valueskip{\hspace{2ex}}

\node (root) {3 \valueskip 9 \valueskip 26 \valueskip 38 \valueskip 41 \valueskip 49 \valueskip 52 \valueskip 57}
	child {node (0) {3 \valueskip 26 \valueskip 41 \valueskip 52}
		child {node (00) {3 \valueskip 41}
			child {node (000) {3}}
			child {node (001) {41}}
		}
		child {node (01) {26 \valueskip 52}
			child {node (010) {52}}
			child {node (011) {26}}
		}
	}
	child {node (1) {9 \valueskip 38 \valueskip 49 \valueskip 57}
		child {node (10) {38 \valueskip 57}
			child {node (100) {38}}
			child {node (101) {57}}		
		}
		child {node (11) {9 \valueskip 49}
			child {node (110) {9}}
			child {node (111) {49}}		
		}
	};

\path (root.north) node[index node, above=3mm] {Ciąg posortowany};
\foreach \child/\parent in {0/root, 00/0, 10/1, 000/00, 010/01, 100/10, 110/11} {
	\path (\parent) -- (\parent |- \child) node[index node, midway] {Scal};
}
\path (011.south) -- (100.south) node[index node, midway, below=3mm] {Ciąg początkowy};
\end{tikzpicture}

	\caption{Działanie algorytmu \proc{Merge-Sort} dla tablicy $A=\langle3,41,52,26,38,57,9,49\rangle$.} \label{fig:2.3-1}
\end{figure}

\exercise %2.3-2
Poniżej przedstawiono implementację procedury \proc{Merge}, w~której nie wykorzystuje się wartowników.
\begin{codebox}
\Procname{$\proc{Merge}'(A,p,q,r)$}
\li	$n_1\gets q-p+1$
\li	$n_2\gets r-q$
\li	utwórz tablice $L[1\twodots n_1]$ i~$R[1\twodots n_2]$
\li	\For $i\gets1$ \To $n_1$
\li		\Do $L[i]\gets A[p+i-1]$
		\End
\li	\For $j\gets1$ \To $n_2$
\li		\Do $R[j]\gets A[q+j]$
		\End
\li	$i\gets j\gets1$
\li $k\gets p$
\li \While $i\le n_1$ i~$j\le n_2$ \label{li:merge'-while-begin}
\li		\Do \If $L[i]\le R[j]$
\li				\Then $A[k]\gets L[i]$
\li					$i\gets i+1$
\li				\Else $A[k]\gets R[j]$
\li					$j\gets j+1$
				\End
\li				$k\gets k+1$
		\End \label{li:merge'-while-end}
\li \While $i\le n_1$
\li		\Do $A[k]\gets L[i]$
\li			$i\gets i+1$
\li			$k\gets k+1$
		\End
\li \While $j\le n_2$
\li		\Do $A[k]\gets R[j]$
\li			$j\gets j+1$
\li			$k\gets k+1$
		\End
\end{codebox}
Po zakończeniu wykonywania pętli \kw{while} z~wierszy \doubledash{\ref{li:merge'-while-begin}}{\ref{li:merge'-while-end}} wszystkie elementy z~co najmniej jednej tablicy $L$ lub $R$ zostały przekopiowane do $A$, więc w~kolejnych dwóch pętlach \kw{while} przeprowadzane jest kopiowanie pozostałej części $L$ lub $R$ na koniec $A$.
Tylko co najwyżej jedna z~tych pętli wykona swój kod.

\exercise %2.3-3
Przeprowadzimy dowód przez indukcję względem $k$.
Dla $k=1$ mamy $n=2$ i~$T(n)=2=2\lg2$, więc przypadek bazowy zachodzi.
Załóżmy teraz, że $k>1$, czyli $n>2$ i~że zachodzi $T(n/2)=(n/2)\lg(n/2)$.
Mamy
\[
	T(n) = 2T(n/2)+n = 2(n/2)\lg(n/2)+n = n(\lg n-1)+n = n\lg n,
\]
co dowodzi rozwiązania rekurencji dla $n$ będącego potęgą 2.

\exercise %2.3-4
Niech $T(n)$ będzie czasem potrzebnym na posortowanie tablicy $A[1\twodots n]$.
Wstawienie elementu $A[n]$ w~posortowaną podtablicę $A[1\twodots n-1]$ odbywa się w~najgorszym przypadku w~czasie $\Theta(n)$, otrzymujemy więc następujące równanie rekurencyjne:
\[
	T(n) =
	\begin{cases}
		\Theta(1), & \text{jeśli $n=1$}, \\
		T(n-1)+\Theta(n), & \text{jeśli $n>1$}.
	\end{cases}
\]

\exercise %2.3-5
Algorytm wyszukiwania binarnego przyjmuje na wejściu posortowaną niemalejąco tablicę $A$ oraz szukaną wartość $v$.
Podczas wyszukiwania $v$ w~podtablicy $A[\id{low}\twodots\id{high}]$, $v$ jest porównywane z~$A[\lfloor(\id{low}+\id{high})/2\rfloor]$, czyli elementem środkowym tego fragmentu i~na podstawie wyniku tego porównania eliminuje z~dalszych rozważań odpowiednią połowę tej podtablicy.

Poniżej przedstawiono wersję rekurencyjną oraz iteracyjną algorytmu wyszukiwania binarnego.
W~przypadku odnalezienia wartości $v$ w~tablicy $A$ zwracany jest taki indeks $i$, że $A[i]=v$.
Jeśli elementu $v$ nie ma w~tablicy, to wynikiem procedury jest specjalna wartość \const{nil}.
Wersja rekurencyjna przyjmuje dodatkowo parametry $\id{low}$ i~$\id{high}$ będące indeksami początku i~końca przetwarzanego fragmentu tablicy $A$.
Aby wyszukiwać w~całej tablicy $A$, używając procedury rekurencyjnej, należy wywołać ją z~parametrami $\id{low}=1$ i~$\id{high}=\attrib{A}{length}$.

\begin{codebox}
\Procname{$\proc{Recursive-Binary-Search}(A,v,\id{low},\id{high})$}
\li	\If $\id{low}>\id{high}$
\li		\Then \Return \const{nil}
		\End
\li	$\id{mid}\gets\lfloor(\id{low}+\id{high})/2\rfloor$
\li	\If $v=A[\id{mid}]$
\li		\Then \Return \id{mid}
		\End
\li	\If $v<A[\id{mid}]$
\li		\Then \Return $\proc{Recursive-Binary-Search}(A,v,\id{low},\id{mid}-1)$
\li		\Else \Return $\proc{Recursive-Binary-Search}(A,v,\id{mid}+1,\id{high})$
		\End
\end{codebox}

\begin{codebox}
\Procname{$\proc{Iterative-Binary-Search}(A,v)$}
\li	$\id{low}\gets1$
\li	$\id{high}\gets\attrib{A}{length}$
\li	\While $\id{low}\le\id{high}$
\li		\Do $\id{mid}\gets\lfloor(\id{low}+\id{high})/2\rfloor$
\li			\If $v=A[\id{mid}]$
\li				\Then \Return \id{mid}
				\End
\li			\If $v<A[\id{mid}]$
\li				\Then $\id{high}\gets\id{mid}-1$
\li				\Else $\id{low}\gets\id{mid}+1$
				\End
		\End
\li	\Return \const{nil}
\end{codebox}

W~obu wersjach algorytmu \proc{Binary-Search} $v$ jest przyrównywane do środkowego elementu fragmentu $A[\id{low}\twodots\id{high}]$, odrzucana jest (w~przybliżeniu) połowa tej podtablicy i~$v$ jest następnie poszukiwane w~drugiej połowie.
Procedury kończą swe działanie, gdy odnajdą $v$ albo gdy zakres poszukiwań okaże się pusty (czyli $\id{low}>\id{high}$), co oznacza, że elementu $v$ nie ma w~$A$.
Niech $n$ będzie rozmiarem tablicy $A$.
Rekurencja opisująca czas działania algorytmu w~przypadku pesymistycznym ma postać
\[
	T(n) =
	\begin{cases}
		\Theta(1), & \text{jeśli $n\le1$}, \\
		T(n/2)+\Theta(1), & \text{jeśli $n>1$}.
	\end{cases}
\]
Jej rozwiązaniem (z~\refExercise{4.3-3}) jest $T(n)=\Theta(\lg n)$.

\exercise %2.3-6
Wyszukując binarnie, można odnaleźć pozycję tablicy, na którą należy umieścić kolejny element z~nieposortowanego fragmentu, jednak wstawienie go wymaga przesunięcia pewnej części tablicy o~jedną pozycję w~prawo, co w~najgorszym przypadku zajmuje czas $\Theta(n)$.
Nie można zatem obniżyć czasu działania sortowania przez wstawianie poprzez zastosowanie wyszukiwania binarnego.

\exercise %2.3-7
Będziemy traktować zbiór $S$ jak tablicę $S[1\twodots n]$.
Dla każdego elementu $S[i]$ można wyszukiwać inny element w~tablicy $S$, który po zsumowaniu z~$S[i]$ daje $x$.
Będziemy wyszukiwać binarnie po uprzednim posortowaniu $S$ (procedura wyszukiwania binarnego została opisana w~\refExercise{2.3-5}).
Dla $S[i]$ szukamy zatem elementu o~wartości $x-S[i]$ w~podtablicy $S[i+1\twodots n]$.
Podtablica ta jest pusta dla $i=n$, dlatego wyszukiwanie dla ostatniego elementu pomijamy.
W~zależności od wyniku wyszukiwania zwracana jest wartość logiczna \const{true} lub \const{false}.
Algorytm zapisujemy w~postaci pseudokodu:
\begin{codebox}
\Procname{$\proc{Sum-Search}(S,x)$}
\li	$n\gets\attrib{S}{length}$
\li	$\proc{Merge-Sort}(S,1,n)$ \label{li:sum-search-sort}
\li	\For $i\gets1$ \To $n-1$
\li		\Do \If $\proc{Recursive-Binary-Search}(S,x-S[i],i+1,n)\ne\const{nil}$ \label{li:sum-search-bs}
\li				\Then \Return \const{true}
				\End
		\End
\li	\Return \const{false}
\end{codebox}

Sortowanie $S$ w~wierszu \ref{li:sum-search-sort} działa w~czasie $\Theta(n\lg n)$, a~procedura \proc{Recursive-Binary-Search} jest wywoływana w~wierszu \ref{li:sum-search-bs} dla tablic o~rozmiarach, kolejno, 1, 2, \dots, $n-1$.
Wywołania te łącznie zajmują czas
\[
	\sum_{i=1}^{n-1}\Theta(\lg i) = \Theta\biggl(\sum_{i=1}^n\lg i-\lg n\biggr) = \Theta\biggl(\lg\biggl(\prod_{i=1}^ni\biggr)-\lg n\biggr) = \Theta(\lg(n!)-\lg n) = \Theta(n\lg n),
\]
ponieważ $\lg(n!)=\Theta(n\lg n)$ (ze wzoru (3.18)).
Zatem pesymistyczny czas algorytmu \proc{Sum-Search} wynosi $\Theta(n\lg n)$.
