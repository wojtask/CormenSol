\subchapter{Sortowanie przez wstawianie}

\exercise %2.1-1
Rys.\ \ref{fig:2.1-1} przedstawia działanie algorytmu \proc{Insertion-Sort} dla tablicy $A$.
\begin{figure}[!ht]
	\centering \begin{tikzpicture}[
	row 1/.style = {nodes={index node, draw=none, fill=none}},
	light cell/.style = {row 2 column #1/.style={nodes={fill=black!20}}},
	black cell/.style = {row 2 column #1/.style={nodes={fill=black, text=white}}},
	outer/.append style = {node distance=10mm and 13mm},
	light edge/.style = {semithick, draw=black!20, bend right=90, ->, >={stealth'[bend]}},
	black edge/.style = {semithick, ->, >={stealth'[bend]}}
]

\node[outer] (pic a) {
\begin{tikzpicture}
	\matrix[
		array,
		light cell/.list = {1},
		black cell/.list = {2}		
	] (arr) {
		 1 &  2 &  3 &  4 &  5 &  6 \\		
		31 & 41 & 59 & 26 & 41 & 58 \\
	};
	\draw (arr-2-2) edge[black edge, loop below] ();
	\node[subpicture label, left=5mm of arr-2-1.west, anchor=east] {(a)};
\end{tikzpicture}
};

\node[outer, right=of pic a.north east, anchor=north west] (pic b) {
\begin{tikzpicture}
	\matrix[
		array,
		light cell/.list = {2},
		black cell/.list = {3}
	] (arr) {
		 1 &  2 &  3 &  4 &  5 &  6 \\
		31 & 41 & 59 & 26 & 41 & 58 \\
	};
	\draw (arr-2-3) edge[black edge, loop below] ();
	\node[subpicture label, left=5mm of arr-2-1.west, anchor=east] {(b)};
\end{tikzpicture}
};

\node[outer, right=of pic b.north east, anchor=north west] (pic c) {
\begin{tikzpicture}
	\matrix[
		array,
		light cell/.list = {1,2,3},
		black cell/.list = {4}
	] (arr) {
		 1 &  2 &  3 &  4 &  5 &  6 \\		
		31 & 41 & 59 & 26 & 41 & 58 \\
	};
	\draw (arr-2-1.280) edge[light edge] (arr-2-2.260);
	\draw (arr-2-2.280) edge[light edge] (arr-2-3.260);
	\draw (arr-2-3.280) edge[light edge] (arr-2-4.260);
	\draw (arr-2-4.300) edge[black edge, bend left=50] (arr-2-1.240);
	\node[subpicture label, left=5mm of arr-2-1.west, anchor=east] {(c)};
\end{tikzpicture}
};

\node[outer, below=of pic a.south west, anchor=north west] (pic d) {
\begin{tikzpicture}
	\matrix[
		array,
		light cell/.list = {3,4},
		black cell/.list = {5}
	] (arr) {
		 1 &  2 &  3 &  4 &  5 &  6 \\	
		26 & 31 & 41 & 59 & 41 & 58 \\
	};
	\draw (arr-2-4.280) edge[light edge] (arr-2-5.260);
	\draw (arr-2-5.300) edge[black edge, bend left=90] (arr-2-4.240);
	\node[subpicture label, left=5mm of arr-2-1.west, anchor=east] {(d)};
\end{tikzpicture}
};

\node[outer, right=of pic d.north east, anchor=north west] (pic e) {
\begin{tikzpicture}
	\matrix[
		array,
		light cell/.list = {4,5},
		black cell/.list = {6}
	] (arr) {
		 1 &  2 &  3 &  4 &  5 &  6 \\
		26 & 31 & 41 & 41 & 59 & 58 \\
	};
	\draw (arr-2-5.280) edge[light edge] (arr-2-6.260);
	\draw (arr-2-6.300) edge[black edge, bend left=90] (arr-2-5.240);
	\node[subpicture label, left=5mm of arr-2-1.west, anchor=east] {(e)};
\end{tikzpicture}
};

\node[outer, right=of pic e.north east, anchor=north west] (pic f) {
\begin{tikzpicture}
	\matrix[array] (arr) {
		 1 &  2 &  3 &  4 &  5 &  6 \\	
		26 & 31 & 41 & 41 & 58 & 59 \\
	};
	\node[subpicture label, left=5mm of arr-2-1.west, anchor=east] {(f)};
\end{tikzpicture}
};

\end{tikzpicture}

	\caption{Działanie algorytmu \proc{Insertion-Sort} dla tablicy $A=\langle31,41,59,26,41,58\rangle$.
{\sffamily\bfseries\doubledash{(a)}{(e)}} Iteracje pętli \kw{for} w~wierszach \doubledash{1}{8}.
{\sffamily\bfseries(f)} Wynikowa posortowana tablica.} \label{fig:2.1-1}
\end{figure}

\exercise %2.1-2
Aby sortować w~porządku nierosnącym, wystarczy w~warunku pętli \kw{while} w~linii 5 algorytmu \proc{Insertion-Sort} zmienić znak drugiej nierówności na przeciwny:
\begin{codebox}
\setcounter{codelinenumber}{4}
\li	\While $i>0$ i~$A[i]<\id{key}$
\end{codebox}

\exercise %2.1-3
Przedstawiony opis prowadzi do następującego algorytmu wyszukiwania liniowego:
\begin{codebox}
\Procname{$\proc{Linear-Search}(A,v)$}
\li	$i\gets1$
\li	\While $i\le\attrib{A}{length}$ i~$A[i]\ne v$ \label{li:linear-search-while-begin}
\li		\Do $i\gets i+1$
		\End \label{li:linear-search-while-end}
\li	\If $i\le\attrib{A}{length}$
\li		\Then \Return $i$
		\End
\li	\Return \const{nil}
\end{codebox}

Udowodnimy dla powyższej procedury niezmiennik pętli:
\begin{quote}
Na początku każdej iteracji pętli \kw{while} w~wierszach \doubledash{\ref{li:linear-search-while-begin}}{\ref{li:linear-search-while-end}} fragment tablicy $A[1\twodots i-1]$ nie zawiera elementu $v$.
\end{quote}
\begin{description}
	\item[Inicjowanie:] Przed pierwszą iteracją $i=1$, więc fragment $A[1\twodots i-1]$ jest pusty.
	\item[Utrzymanie:] Załóżmy, że podtablica $A[1\twodots i-1]$ nie zawiera elementu $v$.
W~warunku pętli \kw{while} sprawdzamy, czy $A[i]$ jest różne od $v$.
Jeśli tak, to $i$ jest zwiększane o~1, więc niezmiennik jest zachowany.
W~przeciwnym przypadku (odnaleziono $v$) przerywamy pętlę.
	\item[Zakończenie:] Pętla kończy swe działanie, kiedy zostanie odnaleziony indeks $i$ taki, że $A[i]=v$ albo $i=\attrib{A}{length}+1$.
Pierwszy przypadek oznacza odnalezienie pierwszego wystąpienia $v$ w~tablicy $A$, a~drugi -- że przejrzeliśmy całą tablicę, nie znajdując $v$ ($A[1\twodots i-1]$ jest teraz całą tablicą $A$).
\end{description}

\exercise %2.1-4
W~tym problemie rozważać będziemy tylko liczby całkowite nieujemne, które są reprezentowane przez tablice bitów w~kolejności od najmniej do najbardziej znaczącego.

\bigskip
\noindent\textbf{Dane wejściowe:} \singledash{$n$}{elementowe} tablice $A$ i~$B$ zawierające reprezentacje binarne \singledash{$n$}{bitowych} liczb całkowitych nieujemnych $a$ i~$b$.\\
\textbf{Wynik:} \singledash{$(n+1)$}{elementowa} tablica $C$ zawierająca reprezentację binarną \singledash{$(n+1)$}{bitowej} liczby całkowitej $c$ takiej, że $c=a+b$.

\begin{codebox}
\Procname{$\proc{Binary-Add}(A,B)$}
\li	$n\gets\attrib{A}{length}$
\li	\For $i\gets1$ \To $n+1$
\li		\Do $C[i]\gets0$
		\End
\li	\For $i\gets1$ \To $n$ \label{li:binary-add-for-begin}
\li		\Do
			$\id{sum}\gets A[i]+B[i]+C[i]$
\li			$C[i]\gets\id{sum}\bmod2$
\li			$C[i+1]\gets\lfloor\id{sum}/2\rfloor$
		\End \label{li:binary-add-for-end}
\li	\Return $C$
\end{codebox}

Po utworzeniu tablicy $C$ i~wypełnieniu jej zerami, w~pętli \kw{for} w~wierszach \doubledash{\ref{li:binary-add-for-begin}}{\ref{li:binary-add-for-end}} procedura dodaje poszczególne bity liczb $a$ i~$b$ w~kolejności od najmniej do najbardziej znaczącego.
W~rzeczywistości przeprowadzane jest dodawanie modulo 2, a~bit przeniesienia jest zapamiętywany na kolejnej pozycji w~tablicy wynikowej i~uczestniczy w~kolejnej iteracji pętli (zakładając, że obecna iteracja nie jest ostatnią).
