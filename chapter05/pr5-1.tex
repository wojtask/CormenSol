\problem{Zliczanie probabilistyczne} %5-1

\subproblem %5-1(a)
Zdefiniujmy $X_j$, dla $j=1$, 2, \dots, $n$, jako zmienną losową oznaczającą liczbę, o~jaką zwiększy się wartość reprezentowana przez licznik po \singledash{$j$}{tym} wykonaniu operacji \proc{Increment}.
Ponadto niech zmienna losowa $X$ przyjmuje wartość reprezentowaną przez licznik po wykonaniu $n$ operacji \proc{Increment}.
Zachodzi $X=\sum_{j=1}^nX_j$ oraz, ze względu na liniowość wartości oczekiwanej,
\[
	\E(X) = \E\biggl(\sum_{j=1}^nX_j\biggr) = \sum_{j=1}^n\E(X_j).
\]

Załóżmy teraz, że przed wykonaniem \singledash{$j$}{tej} operacji \proc{Increment} licznik przechowuje wartość $i$, co stanowi reprezentację $n_i$.
Jeśli inkrementacja powiedzie się, co zdarzy się z~prawdopodobieństwem równym $1/(n_{i+1}-n_i)$, to wartość reprezentowana na liczniku zwiększy się o~$n_{i+1}-n_i$.
Dla każdego $j=1$, 2, \dots, $n$ mamy zatem
\[
	\E(X_j) = 0\cdot\biggl(1-\frac{1}{n_{i+1}-n_i}\biggr)+(n_{i+1}-n_i)\cdot\frac{1}{n_{i+1}-n_i} = 1,
\]
a~więc
\[
	\E(X) = \sum_{j=1}^n\E(X_j) = n,
\]
co należało wykazać.

\subproblem %5-1(b)
Dla zmiennych losowych $X_j$ oraz $X$ zdefiniowanych w~poprzednim punkcie mamy
\[
	\Var(X) = \Var\biggl(\sum_{j=1}^nX_j\biggr) = \sum_{j=1}^n\Var(X_j),
\]
co zachodzi na mocy wzoru (C.28), ponieważ zmienne $X_1$, $X_2$, \dots, $X_n$ są parami niezależne.
Mamy $n_i=100i$, a~więc zwiększenie wartości reprezentowanej przez licznik o~$n_{i+1}-n_i=100$ odbędzie się z~prawdopodobieństwem $1/(n_{i+1}-n_i)=1/100$.
Ze wzoru (C.26) otrzymujemy, że dla każdego $j=1$, 2, \dots, $n$ zachodzi
\[
	\Var(X_j) = \E(X_j^2)-\E^2(X_j) = 0^2\cdot\biggl(1-\frac{1}{100}\biggr)+100^2\cdot\frac{1}{100}-1^2 = 99,
\]
a~stąd
\[
	\Var(X) = \sum_{j=1}^n\Var(X_j) = 99n.
\]
