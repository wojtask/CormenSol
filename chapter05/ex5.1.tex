\subchapter{Problem zatrudnienia sekretarki}

\exercise %5.1-1
Niech $\preceq$ będzie relacją określoną na zbiorze rang kandydatek, za pomocą której rozstrzygamy, która kandydatka z~dwóch testowanych jest lepsza.
Na wejściu procedury \proc{Hire-Assistant} może pojawić się każda permutacja kandydatek, więc jesteśmy w~stanie rozstrzygać o~każdej parze kandydatek.
Pozostaje zatem udowodnić, że $\preceq$ jest porządkiem częściowym.

Możemy bezpiecznie założyć, że relacja $\preceq$ jest zwrotna, jako że nie testujemy żadnej kandydatki z~nią samą.
Jeśli $\preceq$ nie byłoby antysymetryczne, to w~zależności od kolejności pojawienia się na wejściu procedury pewnych dwóch kandydatek, za lepszą mogłaby zostać uznana którakolwiek z~tej pary, co przeczyłoby założeniu.
Podobnie można wykazać, że $\preceq$ jest przechodnie, bowiem w~przeciwnym przypadku dla pewnych trzech kandydatek, o~tym, która z~nich jest najlepsza, decydowałaby ich permutacja wejściowa.

\exercise %5.1-2
Poniższy algorytm implementuje generator liczb losowych z~zakresu $a\twodots b$, korzystając jedynie z~pomocniczych wywołań $\proc{Random}(0,1)$.
\begin{codebox}
\Procname{$\proc{Random}(a,b)$}
\li	\While $a<b$
\li		\Do $\id{mid}\gets\lfloor(a+b)/2\rfloor$
\li			\If $\proc{Random}(0,1)=0$
\li				\Then $a\gets\id{mid}+1$
\li				\Else $b\gets\id{mid}$
				\End
		\End
\li	\Return $a$
\end{codebox}

Niech $n=b-a+1$ będzie długością zakresu generowania.
W~każdym wywołaniu rekurencyjnym odrzucana jest połowa zakresu z~dokładnością do jednego elementu.
Działanie procedury jest więc analogiczne do pesymistycznego przypadku wyszukiwania binarnego w~$n$\nbhyphen elementowej tablicy, przez co średnio działa ona w~czasie opisanym przez rekurencję z~\refExercise{4.3-3}.
Rozwiązaniem tej rekursji jest $T(n)=\Theta(\lg n)$, a~zatem oczekiwany czas działania procedury $\proc{Random}(a,b)$ w~zależności od $a$ i~$b$ wynosi $\Theta(\lg(b-a))$.

\exercise %5.1-3
Zauważmy, że prawdopodobieństwo uzyskania najpierw orła, a~potem reszki w~dwóch rzutach monetą jest takie samo, jak uzyskanie najpierw reszki, a~potem orła i~wynosi $p(1-p)$.
Będziemy zatem rzucać monetą po dwa razy, aż do uzyskania różnych wyników.
Jako wynik procedury przyjmiemy wynik pierwszego rzutu w~ostatniej parze rzutów.

Następujący algorytm implementuje powyższy opis:
\begin{codebox}
\Procname{\proc{Unbiased-Random}}
\li	\Repeat $x\gets\proc{Biased-Random}$
\li		$y\gets\proc{Biased-Random}$
\li	\Until $x\ne y$ \label{li:unbiased-random-repeat-end}
\li	\Return $x$
\end{codebox}

Załóżmy, że każda iteracja pętli \kw{repeat} odbywa się w~czasie stałym.
Kolejne iteracje tworzą ciąg prób Bernoulliego, w~których sukcesem jest warunek z~wiersza \ref{li:unbiased-random-repeat-end}, zachodzący z~prawdopodobieństwem $2p(1-p)$.
Oczekiwana liczba prób aż do osiągnięcia sukcesu jest zadana wzorem (C.31) i~wynosi $1/(2p(1-p))$.
Stąd wnioskujemy, że oczekiwanym czasem działania algorytmu jest $\Theta(1/(p(1-p)))$.
