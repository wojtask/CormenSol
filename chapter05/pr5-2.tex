\problem{Wyszukiwanie w~nieposortowanej tablicy} %5-2

\subproblem %5-2(a)
Oto procedura implementująca opisaną strategię:
\begin{codebox}
\Procname{$\proc{Random-Search}(A,x)$}
\li	$n\gets\attrib{A}{length}$
\li	\For $k\gets1$ \To $n$
\li		\Do $B[k]\gets\const{false}$
		\End
\li	$\id{checked}\gets0$
\li	\While $\id{checked}<n$ \label{li:random-search-while-begin}
\li		\Do
			$i\gets\proc{Random}(1,n)$
\li			\If $A[i]=x$
\li				\Then \Return $i$
				\End
\li			\If $B[i]=\const{false}$
\li				\Then
					$B[i]\gets\const{true}$
\li					$\id{checked}\gets\id{checked}+1$
				\End
		\End \label{li:random-search-while-end}
\li	\Return \const{nil}
\end{codebox}
Algorytm korzysta z~pomocniczej tablicy wartości logicznych $B[1\twodots n]$, która na pozycji $i$ przechowuje informację o~tym, czy wybrana była już \singledash{$i$}{ta} pozycja tablicy $A$.
Ponadto zmienna \id{checked} przechowuje liczbę testowanych dotychczas komórek.
W~każdej iteracji pętli \kw{while} w~wierszach \doubledash{\ref{li:random-search-while-begin}}{\ref{li:random-search-while-end}} algorytm sprawdza losowo wybrany indeks tablicy $A$.
W~przypadku odnalezienia $x$ natychmiast zwracana jest jego pozycja.
Jeśli jednak element $x$ nie zostanie odnaleziony, a~bieżąca komórka tablicy $A$ nie była jeszcze wcześniej sprawdzana, to informacja ta zostaje odnotowana w~tablicy $B$, a~zmienna \id{checked} jest inkrementowana.
Jeśli elementu $x$ nie ma w~tablicy $A$, to po sprawdzeniu wszystkich indeksów co najmniej raz, algorytm zwróci specjalną wartość \const{nil}.

\subproblem %5-2(b)
Niech $X$ będzie zmienną losową oznaczającą ilość wybranych indeksów tablicy $A$ zanim odnaleziono $x$.
Szukanie $x$ realizowane przez procedurę \proc{Random-Search} jest serią prób Bernoulliego, każda z~prawdopodobieństwem sukcesu $p=1/n$.
Stosując wzór (C.31), otrzymujemy, że zostanie wybranych średnio $\E(X)=1/p=n$ indeksów tablicy $A$.

\subproblem %5-2(c)
Rozważmy ponownie zmienną losową $X$ i~analogiczną serię prób Bernoulliego do tej z~poprzedniego punktu.
Jednak w~tym przypadku sukces następuje z~prawdopodobieństwem $p=k/n$, a~zatem średnią liczbą wybranych indeksów przed odnalezieniem $x$ jest $\E(X)=1/p=n/k$.

\subproblem %5-2(d)
Ten przypadek wyszukiwania można sprowadzić do problemu kolekcjonera kuponów.
Pozycje tablicy $A$ reprezentują kupony, których skompletowanie (odpowiadające sprawdzeniu wszystkich pozycji tablicy) jest celem problemu.
Zgodnie z~uzasadnieniem podanym w~Podręczniku, aby uzbierać pełny zestaw $n$ kuponów pojawiających się losowo, należy zdobyć ich około $n\ln n$.

\subproblem %5-2(e)
Procedura \proc{Deterministic-Search} jest identyczna z~algorytmem wyszukiwania liniowego opisanego w~\refExercise{2.1-3}.
Z~rozwiązania \refExercise{2.2-3} wynika zatem, że czas tego algorytmu -- wyrażony jako liczba sprawdzanych indeksów tablicy -- wynosi w~średnim przypadku $(n+1)/2$, a~w~pesymistycznym $n$.

\subproblem %5-2(f)
Oznaczmy przez $X$ zmienną losową przyjmującą liczbę wybranych indeksów tablicy $A$ przed odnalezieniem $x$.
Zdarzenie $X=i$ zachodzi wtedy i~tylko wtedy, gdy pierwsza z~lewej wartość $x$ zajmuje w~$A$ pozycję $i$.
Pozostałe $k-1$ elementów o~wartości $x$ można rozmieścić w~obszarze $A[i+1\twodots n]$ na $\binom{n-i}{k-1}$ sposobów.
Stąd $\Pr(X=i)=\binom{n-i}{k-1}/\binom{n}{k}$.
Wartość oczekiwana $X$ wynosi zatem
\[
    \E(X) = \sum_{i=1}^{n-k+1}i\Pr(X=i) = \frac{1}{\binom{n}{k}}\sum_{i=1}^{n-k+1}i\binom{n-i}{k-1}.
\]

Pokażemy przez indukcję po $n$, że dla dowolnego $k=1$, 2, \dots, $n$ zachodzi $E(X)=\frac{n+1}{k+1}$, co na mocy wzoru (C.8) jest równoważne z~udowodnieniem tożsamości
\[
    \binom{n+1}{k+1} = \sum_{i=1}^{n-k+1}i\binom{n-i}{k-1}.
\]

Jeśli $k=n$, to po lewej stronie powyższego wzoru mamy $\binom{n+1}{n+1}=1$, a~po prawej stronie $\sum_{i=1}^1i\binom{n-i}{n-1}=\binom{n-1}{n-1}=1$.
A~więc w~tym przypadku wzór jest prawdziwy.
Pokazaliśmy przy okazji, że spełniony jest pierwszy krok indukcji, gdy $n=1$.

W~drugim kroku zakładamy, że $n>1$ i~że dla każdego $k=1$, 2, \dots, $n$ zachodzi
\[
    \binom{n}{k} = \sum_{i=1}^{n-k+1}i\binom{n-1-i}{k-2}.
\]
Korzystając dwukrotnie z~\refExercise{C.1-7}, dla dowolnego $k=1$, 2, \dots, $n-1$ mamy
\begin{align*}
    \binom{n+1}{k+1} &= \binom{n}{k+1}+\binom{n}{k} \\
	&= \sum_{i=1}^{n-k}i\binom{n-1-i}{k-1}+\sum_{i=1}^{n-k+1}i\binom{n-1-i}{k-2} \\
	&= \sum_{i=1}^{n-k}i\biggl(\binom{n-1-i}{k-1}+\binom{n-1-i}{k-2}\biggr)+(n-k+1)\binom{k-2}{k-2} \\
	&= \sum_{i=1}^{n-k}i\binom{n-i}{k-1}+(n-k+1)\binom{k-1}{k-1} \\
	&= \sum_{i=1}^{n-k+1}i\binom{n-i}{k-1}.
\end{align*}
Wzór jest zatem prawdziwy dla wszystkich $n$ naturalnych i~wszystkich $k=1$, 2, \dots, $n$.

Pesymistyczny przypadek dla algorytmu \proc{Deterministic-Search} ma miejsce wtedy, gdy wszystkie egzemplarze $x$ zajmują w~tablicy $k$ końcowych pozycji.
Algorytm sprawdzi wówczas $n-k$ komórek tablicy, zanim odnajdzie pierwsze wystąpienie $x$.

\subproblem %5-2(g)
Przypadek średni i~pesymistyczny są równoważne przy braku $x$ w~tablicy $A$, bowiem w~obu tych przypadkach algorytm przegląda całą tablicę, co zajmuje czas $n$.

\subproblem %5-2(h)
Załóżmy, że do permutowania tablicy używany jest algorytm \proc{Randomize-In-Place}, który generuje permutację losową zgodnie z~rozkładem jednostajnym, wykonując przy tym $n$ zamian elementów.
Czas algorytmu \proc{Scramble-Search} jest wtedy sumą $n$ oraz liczby porównań wykonywanych podczas deterministycznego wyszukiwania liniowego.
Wartości te -- w~zależności od przypadku -- zostały wyznaczone w~punktach (e), (f) i~(g).

\subproblem %5-2(i)
W~przypadku gdy tablica nie zawiera szukanego elementu, czasy działania algorytmów \proc{Deterministic-Search} i~\proc{Scramble-Search} są asymptotycznie mniejsze od czasu działania \proc{Random-Search}, a~w~pozostałych przypadkach są one tego samego rzędu (o~ile traktujemy $k$ jako stałą).
Jest to wystarczający powód, aby odrzucić algorytm \proc{Random-Search} z~praktycznych zastosowań.
Spośród pozostałych dwóch \proc{Deterministic-Search} jest bardziej efektywny, ponieważ nie wprowadza narzutu w~postaci permutowania losowego tablicy i~w~efekcie działa szybciej.

Okazuje się więc, że w~problemie wyszukiwania zastosowanie randomizacji nie jest szczególnie pomocne i~zwykłe wyszukiwanie liniowe jest algorytmem optymalnym.
