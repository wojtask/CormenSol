\problem{Wyszukiwanie w~nieposortowanej tablicy} %5-2

\subproblem %5-2(a)
Oto procedura implementująca opisaną strategię:
\begin{codebox}
\Procname{$\proc{Random-Search}(A,x)$}
\li	$n\gets\attrib{A}{length}$
\li	\For $k\gets1$ \To $n$
\li		\Do $B[k]\gets\const{false}$
		\End
\li	$\id{checked}\gets0$
\li	\While $\id{checked}<n$ \label{li:random-search-while-begin}
\li		\Do $i\gets\proc{Random}(1,n)$
\li			\If $A[i]=x$
\li				\Then \Return $i$
				\End
\li			\If $B[i]=\const{false}$
\li				\Then $B[i]\gets\const{true}$
\li					$\id{checked}\gets\id{checked}+1$
				\End
		\End \label{li:random-search-while-end}
\li	\Return \const{nil}
\end{codebox}
Algorytm korzysta z~pomocniczej tablicy wartości logicznych $B[1\twodots n]$, która na pozycji $i$ przechowuje informację o~tym, czy wybrana była już $i$\nbhyphen ta pozycja tablicy $A$.
Ponadto zmienna \id{checked} przechowuje liczbę testowanych dotychczas komórek.
W~każdej iteracji pętli \kw{while} w~wierszach \ref{li:random-search-while-begin}\nbendash\ref{li:random-search-while-end} algorytm sprawdza losowo wybrany indeks tablicy $A$.
W~przypadku odnalezienia $x$ natychmiast zwracana jest jego pozycja.
Jeśli jednak element $x$ nie zostanie odnaleziony, a~bieżąca komórka tablicy $A$ nie była jeszcze wcześniej sprawdzana, to informacja ta zostaje odnotowana w~tablicy $B$, a~zmienna \id{checked} jest inkrementowana.
Jeśli elementu $x$ nie ma w~tablicy $A$, to po sprawdzeniu wszystkich indeksów co najmniej raz, algorytm zwróci specjalną wartość \const{nil}.

\subproblem %5-2(b)
Niech $X$ będzie zmienną losową oznaczającą ilość wybranych indeksów tablicy $A$ zanim odnaleziono $x$.
Szukanie $x$ realizowane przez procedurę \proc{Random-Search} jest serią prób Bernoulliego, każda z~prawdopodobieństwem sukcesu $p=1/n$.
Stosując wzór (C.31), otrzymujemy, że zostanie wybranych średnio $\E(X)=1/p=n$ indeksów tablicy $A$.

\subproblem %5-2(c)
Rozważmy ponownie zmienną losową $X$ i~analogiczną serię prób Bernoulliego do tej z~poprzedniego punktu.
Jednak w~tym przypadku sukces następuje z~prawdopodobieństwem $p=k/n$, a~zatem średnią liczbą wybranych indeksów przed odnalezieniem $x$ jest $\E(X)=1/p=n/k$.

\subproblem %5-2(d)
Ten przypadek wyszukiwania można sprowadzić do problemu kolekcjonera kuponów (patrz sekcja 5.4.2).
Pozycje tablicy $A$ reprezentują kupony, których skompletowanie (odpowiadające sprawdzeniu wszystkich pozycji tablicy) jest celem problemu.
Zgodnie z~uzasadnieniem podanym w~Podręczniku, aby uzbierać pełny zestaw $n$ kuponów pojawiających się losowo, należy zdobyć ich około $n\ln n$.

\subproblem %5-2(e)
Procedura \proc{Deterministic-Search} jest identyczna z~algorytmem wyszukiwania liniowego opisanego w~\refExercise{2.1-3}.
Z~rozwiązania \refExercise{2.2-3} wynika zatem, że czas tego algorytmu -- wyrażony jako liczba sprawdzanych indeksów tablicy -- wynosi w~średnim przypadku $(n+1)/2$, a~w~pesymistycznym $n$.

\subproblem %5-2(f)
Niech $X$ będzie zmienną losową przyjmującą liczbę sprawdzonych indeksów tablicy $A$ przed odnalezieniem $x$.
Dla $i=1$, 2, \dots, $n-k+1$, niech $X_i$ będzie zmienną losową wskaźnikową związaną ze zdarzeniem $B_i$ oznaczającym, że algorytm \proc{Deterministic-Search} sprawdzi pozycję $i$.
Zdarzenie to zachodzi, o~ile wszystkie egzemplarze $x$ tablicy $A$ znajdują się we fragmencie $A[i\twodots n]$.
Mamy zatem
\[
	\Pr(B_i) = \frac{\binom{n-i+1}{k}}{\binom{n}{k}}.
\]
Oczywiście $X=\sum_{i=1}^{n-k+1}X_i$, stąd
\[
	\E(X) = \sum_{i=1}^{n-k+1}\E(X_i) = \sum_{i=1}^{n-k+1}\Pr(B_i) = \frac{\sum_{i=1}^{n-k+1}\binom{n-i+1}{k}}{\binom{n}{k}} = \frac{\sum_{i=k}^n\binom{i}{k}}{\binom{n}{k}}.
\]
Celem uproszczenia ostatniego wyrażenia posłużymy się poniższym lematem.

\medskip
\noindent\textsf{\textbf{Lemat.}} \textit{Dla dowolnego\/ $k=1$, $2$, \dots, $n$ zachodzi}
	\[
		\sum_{i=k}^n\binom{i}{k} = \binom{n+1}{k+1}.
	\]
\begin{proof}
	Zastosujemy indukcję po $n$.
	Jeśli $n=1$, to wzór przyjmuje postać $1=1$.
	Załóżmy teraz, że $n>1$ oraz że dla każdego $k=1$, 2, \dots, $n-1$,
	\[
		\sum_{i=k}^{n-1}\binom{i}{k} = \binom{n}{k+1}.
	\]
	Wówczas
	\[
		\sum_{i=k}^n\binom{i}{k} = \sum_{i=k}^{n-1}\binom{i}{k}+\binom{n}{k} = \binom{n}{k+1}+\binom{n}{k} = \binom{n+1}{k+1},
	\]
	przy czym w~ostatniej równości wykorzystaliśmy \refExercise{C.1-7}.
	Jeśli $k=n$, to po obu stronach dowodzonej równości jest 1, zatem tożsamość zachodzi dla wszystkich $k=1$, 2, \dots, $n$.
\end{proof}

Wracając do wartości oczekiwanej zmiennej losowej $X$ i~stosując wzór (C.8), otrzymujemy
\[
	\E(X) = \frac{\binom{n+1}{k+1}}{\binom{n}{k}} = \frac{n+1}{k+1}.
\]

Pesymistyczny przypadek dla algorytmu \proc{Deterministic-Search} ma miejsce wtedy, gdy wszystkie egzemplarze $x$ zajmują w~tablicy $k$ końcowych pozycji.
Algorytm sprawdzi wówczas $n-k+1$ komórek tablicy, zanim odnajdzie pierwsze wystąpienie $x$.

\subproblem %5-2(g)
Przypadek średni i~pesymistyczny są równoważne przy braku $x$ w~tablicy $A$, bowiem w~obu tych przypadkach algorytm przegląda całą tablicę, co zajmuje czas $n$.

\subproblem %5-2(h)
Załóżmy, że do permutowania tablicy używany jest algorytm \proc{Randomize-In-Place}, który generuje permutację losową zgodnie z~rozkładem jednostajnym, wykonując przy tym $n$ zamian elementów.
Czas algorytmu \proc{Scramble-Search} jest wtedy sumą $n$ oraz liczby porównań wykonywanych podczas deterministycznego wyszukiwania liniowego.
Wartości te -- w~zależności od przypadku -- zostały wyznaczone w~punktach (e), (f) i~(g).

\subproblem %5-2(i)
W~przypadku gdy tablica nie zawiera szukanego elementu, czasy działania algorytmów \proc{Deterministic-Search} i~\proc{Scramble-Search} są asymptotycznie mniejsze od czasu działania \proc{Random-Search}, a~w~pozostałych przypadkach są one tego samego rzędu (o~ile traktujemy $k$ jako stałą).
Jest to wystarczający powód, aby odrzucić algorytm \proc{Random-Search} z~praktycznych zastosowań.
Spośród pozostałych dwóch \proc{Deterministic-Search} jest bardziej efektywny, ponieważ nie wprowadza narzutu w~postaci permutowania losowego tablicy i~w~efekcie działa szybciej.

Okazuje się więc, że w~problemie wyszukiwania zastosowanie randomizacji nie jest szczególnie pomocne i~zwykłe wyszukiwanie liniowe jest algorytmem optymalnym.
