\subchapter{Reprezentowanie struktur wskaźnikowych za pomocą tablic}

\exercise %10.3-1
\note{Oryginalna treść drugiej części zadania żąda, aby podany ciąg zilustrować jako listę dwukierunkową w~reprezentacji jednotablicowej.}

\noindent Rys.\ \ref{fig:10.3-1} przedstawia przykładowe rozmieszczenie elementów ciągu jako listy dwukierunkowej w~reprezentacji wielotablicowej i~jednotablicowej.
\begin{figure}[!ht]
	\centering \begin{tikzpicture}[
	cell/.append style = {fill=black!20, minimum size=4mm},
	med cell/.style = {column #1/.style={nodes={fill=black!40}}},
	index node/.append style = {node distance=6mm and 2mm},
	outer/.append style = {node distance=15mm and 2mm}
]

\node[outer] (pic a) {
\begin{tikzpicture}
	\matrix[
		array,
		med cell/.list = {1,3,6,8}
	] (arr) {
		& 10 & &  5 &  9 & &    & & 7 & 4 \\
		& 13 & &  8 & 19 & & 11 & & 5 & 4 \\
		&    & & 10 &  4 & &  9 & & 5 & 2 \\
	};
	\draw (arr-3-2.south west) + (1mm, 1mm) -- +(3mm, 3mm);
	\draw (arr-1-7.south west) + (1mm, 1mm) -- +(3mm, 3mm);
	\draw[very thick] (arr-3-1.south west) rectangle (arr-1-10.north east);
	\foreach \x in {2, ..., 9} {
		\draw[very thick] (arr-3-\x.south west) -- (arr-1-\x.north west);
	}
	\foreach \x in {1, ..., 10} {
		\node[index node, above=of arr-1-\x] {\x};
	}
	\node[index node, left=of arr-1-1.west] {$\id{next}$};
	\node[index node, left=of arr-2-1.west] {$\id{key}$};
	\node[index node, left=of arr-3-1.west] {$\id{prev}$};
	\node[cell, above left=3mm and 4mm of arr-1-1] (L) {2};
	\node[left=of L.west] {$L$};
	
	\draw[arrow] (L) -| (arr-1-2.110);
	\draw[arrow] (arr-1-2.70) -- +(0,5mm) -| (arr-1-10.70);
	\draw[arrow] (arr-1-10.110) -- +(0,4mm) -| (arr-1-4.110);
	\draw[arrow] (arr-1-4.70) -- +(0,3mm) -| (arr-1-5.110);
	\draw[arrow] (arr-1-5.70) -- +(0,3mm) -| (arr-1-9.70);
	\draw[arrow] (arr-1-9.110) -- +(0,2mm) -| (arr-1-7);
	
	\draw[arrow] (arr-3-7.south) -- +(0,-2mm) -| (arr-3-9.250);
	\draw[arrow] (arr-3-9.290) -- +(0,-3mm) -| (arr-3-5.290);
	\draw[arrow] (arr-3-5.250) -- +(0,-3mm) -| (arr-3-4.290);
	\draw[arrow] (arr-3-4.250) -- +(0,-4mm) -| (arr-3-10.250);
	\draw[arrow] (arr-3-10.290) -- +(0,-5mm) -| (arr-3-2);
\end{tikzpicture}
};

\node[outer, below=of pic a] (pic b) {
\begin{tikzpicture}
	\matrix[
		array,
		med cell/.list = {4,5,6,13,14,15,19,20,21,22,23,24}
	] (arr) {5 & 10 & 16 & & & & 13 & 25 & & 11 & & 1 & & & & 19 & 1 & 28 & & & & & & & 4 & 28 & 7 & 8 & 16 & 25 \\};
	\draw (arr-1-9.south west) + (1mm, 1mm) -- +(3mm, 3mm);
	\draw (arr-1-11.south west) + (1mm, 1mm) -- +(3mm, 3mm);
	\draw[very thick] (arr-1-1.south west) rectangle (arr-1-30.north east);
	\foreach \x in {4, 7, ..., 28} {
		\draw[very thick] (arr-1-\x.south west) -- (arr-1-\x.north west);
	}
	\foreach \x in {1, ..., 30} {
		\node[index node, above=of arr-1-\x] {\x};
	}
	\node[left=of arr-1-1.west] {$A$};
	\node[cell, above left=3mm and 4mm of arr-1-1] (L) {7};
	\node[left=of L.west] {$L$};
	
	\draw[arrow] (L) -| (arr-1-7);
	\draw[arrow] (arr-1-8.north) -- +(0,2mm) -| (arr-1-25);
	\draw[arrow] (arr-1-26.north) -- +(0,2mm) -| (arr-1-28);
	\draw[arrow] (arr-1-29.north) -- +(0,3mm) -| (arr-1-16);
	\draw[arrow] (arr-1-17.north) -- +(0,4mm) -| (arr-1-1);
	\draw[arrow] (arr-1-2.north) -- +(0,3mm) -| (arr-1-10);
	
	\draw[arrow] (arr-1-12.south) -- +(0,-3mm) -| (arr-1-1);
	\draw[arrow] (arr-1-3.south) -- +(0,-4mm) -| (arr-1-16);
	\draw[arrow] (arr-1-18.south) -- +(0,-4mm) -| (arr-1-28);
	\draw[arrow] (arr-1-30.south) -- +(0,-3mm) -| (arr-1-25);
	\draw[arrow] (arr-1-27.south) -- +(0,-2mm) -| (arr-1-7);

	\draw (arr-1-16.south) + (1mm, -1mm) -- +(2mm, -7mm) node[index node, anchor=50] {$\id{key}$};
	\draw (arr-1-17.south) + (0mm, -1mm) -- +(0, -10mm) node[index node, anchor=north] {$\id{next}$};
	\draw (arr-1-18.south) + (-1mm, -1mm) -- +(-2mm, -7mm) node[index node, anchor=140] {$\id{prev}$};
\end{tikzpicture}
};

\node[subpicture label, below=2mm of pic a.south] {(a)};
\node[subpicture label, below=2mm of pic b.south] {(b)};

\end{tikzpicture}

	\caption{Ciąg $\langle13,4,8,19,5,11\rangle$ w~postaci listy dwukierunkowej w~reprezentacji tablicowej.
{\sffamily\bfseries(a)} Lista reprezentowana za pomocą tablic \id{key}, \id{next} oraz \id{prev}.
{\sffamily\bfseries(b)} Ta sama lista reprezentowana w~pojedynczej tablicy $A$.} \label{fig:10.3-1}
\end{figure}

\exercise %10.3-2
Załóżmy, że właściwa lista oraz lista wolnych pozycji znajdują się w~pojedynczej tablicy $A$.
Poniższe procedury są adaptacjami operacji przydzielania i~zwalniania pamięci dla list dwukierunkowych w~reprezentacji jednotablicowej.
Wykorzystujemy tutaj fakt, że polu \id{next} odpowiada przesunięcie 1 względem początku fragmentu tablicy $A$ przechowującego dany element listy.
\begin{codebox}
\Procname{$\proc{Single-Array-Allocate-Object}()$}
\li	\If $\id{free}=\const{nil}$
\li		\Then \Error ,,brak pamięci''
		\End
\li	$x\gets\id{free}$
\li	$\id{free}\gets A[x+1]$
\li	\Return $x$
\end{codebox}

\begin{codebox}
\Procname{$\proc{Single-Array-Free-Object}(x)$}
\li	$A[x+1]\gets\id{free}$
\li	$\id{free}\gets x$
\end{codebox}

\exercise %10.3-3
Lista wolnych pozycji jest stosem implementowanym jako lista jednokierunkowa (\refExercise{10.2-2}), w~której nie są wykorzystywane pola \id{prev}.

\exercise %10.3-4
W~naszej implementacji lista wolnych pozycji będzie zajmować kolejne komórki tablic, począwszy od \id{free}, zatem wskaźniki \id{next} dla tej listy nie będą mieć znaczenia.
Korzystając ze wskazówki z~treści zadania, będziemy traktować tę listę jak stos, z~wierzchołkiem na pozycji \id{free}.
Gdy $\id{free}=\const{nil}$, stos ten jest pusty.
Przydzielanie pamięci polega na wywołaniu na tym stosie operacji \proc{Pop}, czyli zwiększeniu \id{free} o~1 i~zwróceniu poprzedniej wartości \id{free}.
Procedura \proc{Compact-List-Allocate-Object} jest więc identyczna z~oryginalną \proc{Allocate-Object}.
Z~kolei zwalnianie pamięci to wywołanie na nim operacji \proc{Push}, czyli zmniejszenie \id{free} o~1 i~wstawienie zwalnianego elementu na pozycję \id{free}.
Wpierw jednak należy zamienić zwalniany element z~tym, który znajduje się bezpośrednio na lewo od wierzchołka stosu.
Poniższy pseudokod stanowi implementację tej operacji.
\begin{codebox}
\Procname{$\proc{Compact-List-Free-Object}(x)$}
\li	\If $\id{free}=\const{nil}$
\li		\Then $y\gets\attribii{key}{length}$
\li		\Else $y\gets\id{free}-1$
		\End
\li	\If $\id{next}[x]\ne\const{nil}$
\li		\Then $\id{prev}[\id{next}[x]]\gets\id{prev}[x]$
		\End
\li	\If $\id{prev}[x]\ne\const{nil}$
\li		\Then $\id{next}[\id{prev}[x]]\gets\id{next}[x]$
		\End
\li	\If $x\ne y$
\li		\Then \If $\id{next}[y]\ne\const{nil}$
\li				\Then $\id{prev}[\id{next}[y]]\gets x$
				\End
\li			\If $\id{prev}[y]\ne\const{nil}$
\li				\Then $\id{next}[\id{prev}[y]]\gets x$
				\End
		\End
\li	skopiuj wartości pól \id{key}, \id{next} i~\id{prev} elementu $y$ do pól elementu $x$
\li	\If $L=y$
\li		\Then $L\gets x$
		\End
\li	$\proc{Free-Object}(y)$
\end{codebox}

\exercise %10.3-5
Idea procedury \proc{Compactify-List} polega na przechodzeniu po liście $L$ i~przenoszeniu kolejnych jej elementów na początkowe pozycje tablic \id{key}, \id{next} i~\id{prev}.
\begin{codebox}
\Procname{$\proc{Compactify-List}(L,F)$}
\li	$k\gets1$
\li	$l\gets L$
\li	\While $l\ne\const{nil}$ \label{li:compactify-list-while-begin}
\li		\Do zamień elementy na pozycjach $k$ i~$l$ w~tablicach \id{key}, \id{next} i~\id{prev}
\li			\If $F=k$
\li				\Then $F\gets l$
				\End
\li			$l\gets\id{next}[k]$
\li			$\id{next}[k]\gets k+1$
\li			$\id{prev}[k]\gets k-1$
\li			$k\gets k+1$
		\End \label{li:compactify-list-while-end}
\li	$\id{prev}[1]\gets\id{next}[k-1]\gets\const{nil}$
\li	$L\gets1$
\end{codebox}

W~procedurze utrzymywane są kolejny indeks $k$ tablic \id{key}, \id{next} i~\id{prev} oraz indeks $l$ aktualnego elementu listy $L$.
Przenoszenie elementów listy $L$ polega na zamianie ze sobą elementów na pozycjach $k$ i~$l$, z~jednoczesną aktualizacją wskaźników \id{prev} i~\id{next} ich poprzedników i~następników.
Ponadto, należy pamiętać o~ewentualnym przestawieniu głowy listy wolnych pozycji w~celu utrzymania jej poprawności, a~także zaktualizowaniu poprzednika i~następnika przeniesionego elementu listy $L$.
Po wyczerpaniu elementów z~listy $L$ poprzednik jej głowy i~następnik jej ogona zostają ustawione na \const{nil}, a~zmienna $L$ -- na pierwszą pozycję tablic.

Poprawność procedury można uzasadnić na podstawie niezmiennika pętli \kw{while}:
\begin{quote}
Przed każdą iteracją pętli \kw{while} w~wierszach \ref{li:compactify-list-while-begin}\nbendash\ref{li:compactify-list-while-end} pierwsze $k-1$ pozycji tablic \id{key}, \id{next} i~\id{prev} zawiera kolejne elementy listy $L$, a~zmienna $l$ zawiera indeks $k$\nbhyphen tego elementu listy $L$ (albo \const{nil} jeśli $k>m)$.
Ponadto, lista wolnych pozycji z~głową na pozycji $F$ jest poprawną listą dwukierunkową o~długości $n-m$.
\end{quote}
