\problem{Wyszukiwanie na posortowanej liście zajmującej spójny obszar pamięci (liście upakowanej)} %10-3
\note{W~pseudokodzie procedury \proc{Compact-List-Search} jest błąd -- w~linijce 4 zamiast testowania, czy\/ $\id{key}[j]<k$, powinno być sprawdzenie, czy\/ $\id{key}[j]\le k$.
Ponadto w~wywołaniach procedur \proc{Compact-List-Search} i~\proc{Compact-List-Search}$'$ na listach ich argumentów brakuje\/ $n$ jako drugiego parametru.}

\subproblem %10-3(a)
W~żadnej z~procedur nigdy nie zostanie wykonany skok na losowo wybraną pozycję bliżej głowy listy ani na pozycję zawierającą element o~kluczu większym niż $k$ -- gwarantują to warunki w~wierszach 4 obu algorytmów.
Ponadto w~każdej iteracji w~pętlach \kw{while} obu procedur następuje przesunięcie indeksu $i$ o~jedną pozycję do przodu.
Skoki te wykonywane są, dopóki indeks $i$ nie dotrze na pozycję elementu o~kluczu większym lub równym $k$, bądź nie osiągnie końca listy $L$.
A~zatem oba wywołania zwrócą ten sam wynik.

Zauważmy, że podczas ustalonej iteracji pętli \kw{while} w~procedurze \proc{Compact-List-Search} indeks $i$ nie jest bliżej głowy listy $L$ niż ten sam indeks podczas tej samej iteracji pętli \kw{for} w~procedurze \proc{Compact-List-Search}$'$.
Pętla pierwszego algorytmu nie zakończy się więc później niż pętla drugiego algorytmu, która to wykona dokładnie $t$ iteracji.
A~zatem łączna liczba iteracji pętli \kw{for} i~\kw{while} w~procedurze \proc{Compact-List-Search}$'$ wynosi co najmniej $t$.

\subproblem %10-3(b)
W~wywołaniu \proc{Compact-List-Search}$'(L,n,k,t)$ zostanie wykonanych nie więcej niż $t$ iteracji pętli \kw{for}, czyli etap ten wymaga czasu $O(t)$.
Na podstawie definicji zmiennej losowej $X_t$ otrzymujemy z~kolei, że średnia liczba wykonanych iteracji pętli \kw{while} wynosi $\E(X_t)$.
A~zatem oczekiwanym czasem działania procedury \proc{Compact-List-Search}$'(L,n,k,t)$ jest $O(t+\E(X_t))$.

\subproblem %10-3(c)
Oznaczmy przez $s$ indeks klucza $k$ na liście $L$, a~przez $j_1$, $j_2$, \dots, $j_t$ -- ciąg liczb całkowitych wyznaczonych przez $t$ wywołań $\proc{Random}(1,n)$.
Ponadto niech $\pi$ będzie funkcją przypisującą indeksowi elementu na liście $L$ jego rzeczywistą pozycję na tej liście wyznaczoną na podstawie łańcucha wskaźników \id{next}.
Po $t$ iteracjach pętli \kw{for} odległość między szukanym kluczem a~elementem o~indeksie $i$ (mierzona długością łańcucha wskaźników \id{next}) będzie większa lub równa $r$, jeśli dla każdego $q=1$, 2, \dots, $t$ spełnione będzie $\pi(j_q)\le\pi(s)-r$ lub $\pi(j_q)>\pi(s)$.
Mamy więc
\[
	\Pr(X_t\ge r) = \prod_{q=1}^t\Pr(\pi(j_q)\le\pi(s)-r\;\;\text{lub}\;\;\pi(j_q)>\pi(s)) = \prod_{q=1}^t\frac{n-r}{n} = \biggl(1-\frac{r}{n}\biggr)^t.
\]
Korzystając z~tożsamości (C.24) i~z~powyższego oszacowania, otrzymujemy
\[
	\E(X_t) = \sum_{r=1}^\infty\Pr(X_t\ge r) = \sum_{r=1}^n\Pr(X_t\ge r) = \sum_{r=1}^n\biggl(1-\frac{r}{n}\biggr)^t.
\]

\subproblem %10-3(d)
Sumę ograniczamy całką, otrzymując:
\[
	\sum_{r=0}^{n-1}r^t \le \int_0^nx^t\,dx = \biggl[\frac{x^{t+1}}{t+1}\biggr]_0^n = \frac{n^{t+1}}{t+1}.
\]

\subproblem %10-3(e)
Na podstawie rozwiązań punktów (c) i~(d) mamy:
\[
	\E(X_t) = \sum_{r=1}^n\biggl(1-\frac{r}{n}\biggr)^t = \sum_{r=1}^{n-1}\biggl(\frac{n-r}{n}\biggr)^t = \sum_{r=1}^{n-1}\biggl(\frac{r}{n}\biggr)^t = \frac{\sum_{r=0}^{n-1}r^t}{n^t} \le \frac{\frac{n^{t+1}}{t+1}}{n^t} = \frac{n}{t+1}.
\]

\subproblem %10-3(f)
Na mocy punktów (b) i~(e) dostajemy, że oczekiwany czas działania procedury \proc{Compact-List-Search}$'(L,n,k,t)$ wynosi $O(t+\E(X_t))=O(t+n/(t+1))=O(t+n/t)$.

\subproblem %10-3(g)
Oznaczmy przez $p$ pozycję listy $L$, na którą został wykonany ostatni skok w~wierszu 5 procedury \proc{Compact-List-Search}.
Zauważmy, że w~procedurze \proc{Compact-List-Search} po wykonaniu tego skoku zostanie wykonanych co najwyżej $t$ operacji $i\gets\id{next}[i]$.
Z~kolei pozycja na liście $L$, będąca celem ostatniego skoku z~wiersza 5 w~\proc{Compact-List-Search}$'$, nie może znajdować się bliżej głowy listy niż pozycja $p$.
Na tej podstawie wnioskujemy, że liczba wykonanych operacji $i\gets\id{next}[i]$ w~pętli \kw{while} algorytmu \proc{Compact-List-Search}$'$ również nie przekroczy $t$.
Oznacza to, że etap, który na podstawie poprzedniego punktu zajmuje czas $O(n/t)$, będzie w~rzeczywistości działać w~czasie $O(t)$.
Stąd $O(n/t)=O(t)$, czyli $t=O(\!\sqrt{n})$, a~więc czas działania procedury \proc{Compact-List-Search} wynosi $O(\!\sqrt{n})$.

Wynik ten jest prawdziwy także w~przypadku, gdy elementu o~kluczu $k$ nie ma na liście.
Wystarczy bowiem zdefiniować zmienną losową $X_t$ jako odległość na liście (mierzoną długością łańcucha wskaźników \id{next}) od pozycji $i$ do pozycji elementu o~kluczu większym niż $k$ albo do pozycji bezpośrednio za ogonem listy, w~przypadku, gdy taki element nie istnieje.
Analizę w~punktach \doubledash{(b)}{(g)} dla nowej definicji zmiennej $X_t$ przeprowadza się analogicznie.

\subproblem %10-3(h)
Jeśli dopuścimy, aby elementy na liście powtarzały się, to może dojść do sytuacji, w~której procedura próbuje wykonać skok na pozycję $j$ bliższą szukanemu elementowi niż pozycja $i$, ale nie wykonuje go, gdyż stwierdza w~linii 4, że $\id{key}[i]=\id{key}[j]$.
Jeśli każdy losowy skok będzie eliminowany na tej podstawie, to działanie procedury sprowadzi się do działania zwykłego algorytmu wyszukiwania na posortowanej liście.
