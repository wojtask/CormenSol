\problem{Porównanie list} %10-1
Tabela \ref{tab:10-1} zawiera pesymistyczne czasy poszczególnych operacji słownikowych dla danych czterech typów list.
Przyjmujemy, że operacje wykonywane są na listach o~rozmiarach $n$.

\begin{table}[ht]
	\centering
		\[
			\begin{array}{l|c|c|c|c}
				& \text{Nieposortowana} & \text{Posortowana} & \text{Nieposortowana} & \text{Posortowana} \\
				& \text{jedno-} & \text{jedno-} & \text{dwu-} & \text{dwu-} \\
				& \text{kierunkowa} & \text{kierunkowa} & \text{kierunkowa} & \text{kierunkowa} \\
				\hline
				\proc{Search}(L,k) & \Theta(n) & \Theta(n) & \Theta(n) & \Theta(n) \\
				\hline
				\proc{Insert}(L,x) & \Theta(1) & \Theta(n) & \Theta(1) & \Theta(n) \\
				\hline
				\proc{Delete}(L,x) & \Theta(n) & \Theta(n) & \Theta(1) & \Theta(1) \\
				\hline
				\proc{Successor}(L,x) & \Theta(n) & \Theta(1) & \Theta(n) & \Theta(1) \\
				\hline
				\proc{Predecessor}(L,x) & \Theta(n) & \Theta(n) & \Theta(n) & \Theta(1) \\
				\hline
				\proc{Minimum}(L) & \Theta(n) & \Theta(1) & \Theta(n) & \Theta(1) \\
				\hline
				\proc{Maximum}(L) & \Theta(n) & \Theta(n) & \Theta(n) & \Theta(n)
			\end{array}
		\]
	\caption{Porównanie pesymistycznych złożoności operacji słownikowych dla różnych typów list.} \label{tab:10-1}
\end{table}
Jeśli w~implementacjach list będziemy dodatkowo utrzymywać atrybut \id{tail} wskazujący na ogon listy, to operację \proc{Maximum} dla list posortowanych możemy wykonywać w~czasie stałym.
