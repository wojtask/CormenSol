\subchapter{Reprezentowanie struktur wskaźnikowych za pomocą tablic}

\exercise %10.3-1
\note{Oryginalna treść drugiej części zadania żąda, aby podany ciąg zilustrować jako listę dwukierunkową w~reprezentacji jednotablicowej.}

\noindent Rys.\ \ref{fig:10.3-1} przedstawia przykładowe rozmieszczenie elementów ciągu jako listy dwukierunkowej w~reprezentacji wielotablicowej i~jednotablicowej.
\begin{figure}[ht]
	\begin{center}
		\includegraphics{fig_10.3-1}
	\end{center}
	\caption{Ciąg $\langle13,4,8,19,5,11\rangle$ w~postaci listy dwukierunkowej w~reprezentacji tablicowej.
{\sffamily\bfseries(a)} Lista reprezentowana za pomocą tablic \id{key}, \id{next} oraz \id{prev}.
{\sffamily\bfseries(b)} Ta sama lista reprezentowana w~pojedynczej tablicy $A$.} \label{fig:10.3-1}
\end{figure}

\exercise %10.3-2
Załóżmy, że właściwa lista oraz lista wolnych pozycji znajdują się w~pojedynczej tablicy $A$.
Poniższe procedury są adaptacjami operacji przydzielania i~zwalniania pamięci dla list dwukierunkowych w~reprezentacji jednotablicowej.
Wykorzystujemy tutaj fakt, że polu \id{next} odpowiada przesunięcie 1 względem początku fragmentu tablicy $A$ przechowującego dany element listy.
\begin{codebox}
\Procname{$\proc{Single-Array-Allocate-Object}()$}
\li	\If $\id{free}=\const{nil}$
\li		\Then \Error ,,brak pamięci''
		\End
\li	$x\gets\id{free}$
\li	$\id{free}\gets A[x+1]$
\li	\Return $x$
\end{codebox}

\begin{codebox}
\Procname{$\proc{Single-Array-Free-Object}(x)$}
\li	$A[x+1]\gets\id{free}$
\li	$\id{free}\gets x$
\end{codebox}

\exercise %10.3-3
Lista wolnych pozycji jest listą jednokierunkową -- nie są więc wykorzystywane w~niej pola \id{prev}.

\exercise %10.3-4
Korzystając ze wskazówki z~treści zadania, zaimplementujemy listę wolnych pozycji jako stos.
Będzie on zajmował wszystkie pozycje na prawo od tej wskazywanej przez \id{free}, na której znajdzie się jego wierzchołek.
Stos jest pusty wtedy i~tylko wtedy, gdy $\id{free}=\const{nil}$.
Przydzielanie pamięci polega na wywołaniu na stosie wolnych pozycji operacji \proc{Pop}, a~więc procedura \proc{Compact-List-Allocate-Object} jest identyczna z~oryginalną \proc{Allocate-Object}.
Z~kolei zwalnianie pamięci to wywołanie na tym stosie operacji \proc{Push}.
Wcześniej jednak należy zamienić zwalniany element z~tym, który znajduje się bezpośrednio na lewo od wierzchołka stosu.
Poniższy pseudokod stanowi implementację tej operacji.
\begin{codebox}
\Procname{$\proc{Compact-List-Free-Object}(x)$}
\li	\If $\id{free}=\const{nil}$
\li		\Then $y\gets\attrib{key}{length}$
\li		\Else $y\gets\id{free}-1$
		\End
\li	\If $\id{next}[x]\ne\const{nil}$
\li		\Then $\id{prev}[\id{next}[x]]\gets\id{prev}[x]$
		\End
\li	\If $\id{prev}[x]\ne\const{nil}$
\li		\Then $\id{next}[\id{prev}[x]]\gets\id{next}[x]$
		\End
\li	\If $x\ne y$
\li		\Then
			\If $\id{next}[y]\ne\const{nil}$
\li				\Then $\id{prev}[\id{next}[y]]\gets x$
				\End
\li			\If $\id{prev}[y]\ne\const{nil}$
\li				\Then $\id{next}[\id{prev}[y]]\gets x$
				\End
		\End
\li	skopiuj wartości pól \id{key}, \id{next} i~\id{prev} elementu $y$ do pól elementu $x$
\li	\If $L=y$
\li		\Then $L\gets x$
		\End
\li	$\proc{Free-Object}(y)$
\end{codebox}

\exercise %10.3-5
Ogólna idea procedury \proc{Compactify-List} wygląda następująco.
Podczas przechodzenia po liście $L$ wyznaczane są te elementy, które zajmują pozycje na prawo od $m$.
Wszystkie one muszą być zamienione z~elementami listy $F$ znajdującymi się na pozycjach do $m$ włącznie.
Aby zachować czas $\Theta(m)$, procedura nie może przechodzić po liście $F$, która składa się z~$n-m$ elementów.
Zamiast tego będzie ona przeszukiwać liniowo tablice implementujące listy i~wyznaczać rekordy należące do listy wolnych pozycji.
Jednym ze sposobów rozróżniania elementów list $L$ i~$F$ jest wykorzystanie wskaźników \id{prev}.
Na początku działania procedury do pól \id{prev} wszystkich elementów listy $L$ zostanie wpisana specjalna wartość, nie będąca poprawnym indeksem tablicy, np.\ $-1$.
Tuż przed zakończeniem działania wskaźnikom \id{prev} zostaną nadane właściwe wartości, ale wpierw pola te posłużą do identyfikacji elementów listy.
\begin{codebox}
\Procname{$\proc{Compactify-List}(L,F)$}
\li	$n\gets\attrib{key}{length}$
\li	wpisz $-1$ do pól \id{prev} elementów listy $L$ i~wyznacz liczbę jej elementów $m$ \label{li:compactify-list-preprocess}
\li $x\gets L$
\li	$x'\gets\const{nil}$
\li	$y\gets1$
\li	\While $x\ne\const{nil}$ \label{li:compactify-list-while-begin}
\li		\Do
			\If $x\le m$
\li				\Then
					$x'\gets x$
\li					$x\gets\id{next}[x]$
\li				\Else
					\While $\id{prev}[y]=-1$ \label{li:compactify-list-while2-begin}
\li						\Do $y\gets y+1$
						\End \label{li:compactify-list-while2-end}
\li					zamień wartości pól \id{key}, \id{next} i~\id{prev} elementu $x$ z~polami elementu $y$ \label{li:compactify-list-swap}
\li					\If $\id{next}[x]\ne\const{nil}$
\li						\Then $\id{prev}[\id{next}[x]]\gets x$ \label{li:compactify-list-update-prev-next}
						\End
\li					\If $\id{prev}[x]\ne\const{nil}$
\li						\Then $\id{next}[\id{prev}[x]]\gets x$ \label{li:compactify-list-update-next-prev}
\li						\Else $F\gets x$ \label{li:compactify-free-list-head-update}
						\End
\li					\If $x'\ne\const{nil}$
\li						\Then $\id{next}[x']\gets y$ \label{li:compactify-list-update-predecessor-next}
\li						\Else $L\gets y$ \label{li:compactify-list-head-update}
						\End
\li					$x'\gets y$
\li					$x\gets\id{next}[y]$
				\End
		\End \label{li:compactify-list-while-end}
\li	przywróć poprawne wartości w~polach \id{prev} elementów listy $L$ \label{li:compactify-list-postprocess}
\end{codebox}

Procedura przechodzi przez listę $L$ trzy razy.
Najpierw w~wierszu \ref{li:compactify-list-preprocess} tablice przechowujące listy $L$ i~$F$ są przygotowywane do właściwego przetwarzania poprzez ustawienie pól \id{prev} wszystkich elementów listy $L$ na $-1$.
W~tym samym kroku wyznaczana jest wartość $m$ -- rozmiar listy $L$.

Kolejne przejście po liście to właściwe kompaktowanie.
Jeśli dany element $x$ listy $L$ zajmuje w~tablicach pozycję wyższą niż $m$, to zostaje zamieniony z~elementem $y$ listy $F$ znajdującym się najbardziej na lewo.
Pętla \kw{while} w~wierszach \doubledash{\ref{li:compactify-list-while2-begin}}{\ref{li:compactify-list-while2-end}} wyznacza $y$ poprzez liniowe przeglądanie tablicy \id{prev}, po czym w~wierszu \ref{li:compactify-list-swap} elementy $x$ i~$y$ zamieniają swoje pozycje.
Należy jeszcze uaktualnić pola \id{prev} i~\id{next} elementów sąsiednich na liście $F$ (wiersze \ref{li:compactify-list-update-prev-next} i~\ref{li:compactify-list-update-next-prev}), pole \id{next} poprzednika $x$ na liście $L$ (wiersz \ref{li:compactify-list-update-predecessor-next}), jak również wskaźniki $L$ i~$F$, w~przypadku gdy wśród zamienianych elementów była głowa którejś z~list (wiersze \ref{li:compactify-free-list-head-update} oraz \ref{li:compactify-list-head-update}).

Ostatnim krokiem procedury jest ponowne przejście przez listę $L$ w~wierszu \ref{li:compactify-list-postprocess} i~przywrócenie odpowiednich wartości w~polach \id{prev} wszystkich jej elementów.

Po wykonaniu procedury elementy z~listy wolnych pozycji nie zawsze będą umieszczone w~tablicach według ich kolejności na tej liście.
Z~tego powodu na liście przetworzonej procedurą \proc{Compactify-List} nie możemy korzystać z~procedur \proc{Compact-List-Allocate-Object} i~\proc{Compact-List-Free-Object} z~\refExercise{10.3-4}, które zakładają, że reprezentacja tablicowa listy wolnych pozycji stanowi stos.
