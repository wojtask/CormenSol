\problem{Listowa reprezentacja kopców złączalnych} %10-2

\subproblem %10-2(a)
Kopiec zaimplementujemy jako posortowaną listę jednokierunkową.
Operacja \proc{Make-Heap} tworzy pustą listę, co zajmuje oczywiście czas stały.
Dodanie elementu do kopca polega na dodaniu go do listy.
Aby zachować jej uporządkowanie, musimy odnaleźć miejsce, które zajmie nowy element, co w~pesymistycznym przypadku wymaga czasu $\Theta(n)$.
Dzięki uporządkowaniu listy można zaimplementować operacje \proc{Minimum} i~\proc{Extract-Min} działające w~czasie $\Theta(1)$.
Z~kolei stosując algorytm opisany w~\refExercise{6.5-8} w~wersji dla list jednokierunkowych, a~następnie przechodząc po scalonej liście w~celu usunięcia powtarzających się elementów, jesteśmy w~stanie zaimplementować operację \proc{Union} w~czasie $\Theta(n)$, gdzie $n$ jest liczbą elementów na wynikowej liście.

\subproblem %10-2(b)
W~tym przypadku także wystarczy nam lista jednokierunkowa.
Zarówno utworzenie pustego kopca, jak i~dodanie do niego nowego elementu odbywa się w~czasie stałym, ale odszukanie oraz usunięcie minimalnego elementu wiąże się z~przeszukaniem całej listy, co zajmuje czas liniowy względem rozmiaru listy.
Możemy jednak wprowadzić usprawnienie, dzięki któremu operacja \proc{Minimum} będzie działać w~czasie stałym, jednocześnie nie pogarszając czasów działania pozostałych operacji.
Wystarczy bowiem utrzymywać minimalny element listy w~jej głowie.
Wstawianie nowego elementu po takiej modyfikacji polega na umieszczeniu go w~głowie listy albo tuż za nią, w~zależności od tego, czy nowy element jest mniejszy od aktualnego minimum.
Operacja \proc{Extract-Min} po usunięciu głowy nadal jednak musi przeszukać całą listę w~celu znalezienia nowego minimum i~przeniesienia go do głowy listy.

W~operacji łączenia musimy pamiętać o~tym, że listy wejściowe nie zawsze reprezentują rozłączne zbiory, a~także o~tym, że w~głowie wynikowej listy powinien znaleźć się najmniejszy element z~obu łączonych list.
Podamy efektywną implementację operacji \proc{Union} uwzględniającą obie kwestie, w~której jako pomocniczą strukturę danych wykorzystamy tablicę z~haszowaniem.
Na początku wyznaczymy element $x$ o~minimalnym kluczu spośród elementów z~obu kopców.
Porównujemy w~tym celu elementy w~głowach obu list wejściowych i~wybieramy mniejszy z~nich.
Następnie umieścimy wszystkie klucze z~pierwszego kopca w~tablicy z~haszowaniem, wykorzystując do tego celu metodę haszowania doskonałego opisaną w~podrozdziale 11.5.
Teraz wystarczy tylko przeglądnąć drugi kopiec i~te jego elementy, których klucz nie znajduje się w~tablicy, wstawić do pierwszego kopca.
Należy jeszcze przenieść znalezione wcześniej minimum $x$ na początek wynikowej listy $L$.
Można to zrobić dzięki wywołaniu $\proc{Singly-Linked-List-Delete}(L,x)$, a~następnie $\proc{Singly-Linked-List-Insert}(L,x)$ (patrz \refExercise{10.2-1}).

Podczas działania procedury \proc{Union} dającej w~wyniku scalony kopiec o~$n$ elementach, w~tablicy z~haszowaniem umieszczany jest zbiór co najwyżej $n$ kluczy i~co najwyżej $n$ razy wykonywane jest wyszukiwanie klucza w~tej tablicy.
Na podstawie analizy z~podrozdziału 11.5 mamy, że sumaryczny pesymistyczny czas potrzebny do przeprowadzenia tych operacji na tablicy z~haszowaniem wykorzystującej haszowanie doskonałe, wynosi $\Theta(n)$.

\subproblem %10-2(c)
Przypadek jest identyczny z~poprzednim, ale tym razem nie jest wymagane wykrywanie powtórzeń podczas operacji \proc{Union}.
Łączenie list możemy więc zaimplementować jako ich konkatenację, to znaczy ustawienie głowy jednej listy jako elementu następującego po ogonie drugiej.
To, w~jakiej kolejności skleimy listy, będzie zależeć od tego, której z~nich głowa przechowuje mniejszy element, co ma na celu spełnienie usprawnienia opisanego w~poprzednim punkcie.
Operację łączenia możemy zrealizować w~czasie stałym, jeśli dla listy w~tej reprezentacji kopca będziemy pamiętać wskaźnik na jej ogon.

\bigskip
\noindent Zestawienie czasów działania poszczególnych operacji w~przypadkach pesymistycznych dla omawianych reprezentacji listowych kopców złączalnych przedstawiono w~tabeli \ref{tab:10-2}.

\begin{table}[!ht]
	\centering
		\[
			\begin{array}{l|c|c|c}
				& \text{Listy posortowane} & \text{Listy nieposortowane} & \text{Listy nieposortowane,} \\
				&  &  & \text{rozłączne zbiory w~\proc{Union}} \\
				\hline
				\proc{Make-Heap} & \Theta(1) & \Theta(1) & \Theta(1) \\
				\hline
				\proc{Insert} & \Theta(n) & \Theta(1) & \Theta(1) \\
				\hline
				\proc{Minimum} & \Theta(1) & \Theta(1) & \Theta(1) \\
				\hline
				\proc{Extract-Min} & \Theta(1) & \Theta(n) & \Theta(n) \\
				\hline
				\proc{Union} & \Theta(n) & \Theta(n) & \Theta(1)
			\end{array}
		\]
	\caption{Porównanie pesymistycznych czasów operacji słownikowych dla reprezentacji listowych kopców złączalnych.
Dla operacji \proc{Union} $n$ oznacza rozmiar zbioru po połączeniu.} \label{tab:10-2}
\end{table}
