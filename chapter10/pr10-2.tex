\problem{Listowa reprezentacja kopców złączalnych} %10-2

\subproblem %10-2(a)
Kopiec zaimplementujemy jako posortowaną listę jednokierunkową.
Operacja \proc{Make-Heap} tworzy pustą listę, co zajmuje oczywiście czas stały.
Wstawienie elementu do kopca o~$n$ elementach polega na dodaniu go do listy.
Aby zachować jej uporządkowanie, musimy odnaleźć pozycję dla nowego elementu, co w~najgorszym przypadku będzie wymagać czasu $\Theta(n)$.
Najmniejszy element posortowanej listy znajduje się w~jej głowie, więc jego odszukanie lub usunięcie (przy użyciu procedury podobnej do \proc{Singly-Linked-List-Pop} z~\refExercise{10.2-2}) odbywa się w~czasie $\Theta(1)$.
Z~kolei stosując algorytm z~\refExercise{6.5-8} w~wersji dla list jednokierunkowych, a~następnie przechodząc po scalonej liście i~usuwając powtarzające się elementy, jesteśmy w~stanie zaimplementować operację \proc{Union} w~czasie $\Theta(n)$, gdzie $n$ jest sumą elementów w~obu scalanych kopcach.

\subproblem %10-2(b)
W~tym przypadku również wystarczy nam lista jednokierunkowa.
Zarówno utworzenie pustego kopca, jak i~dodanie do niego nowego elementu odbywają się w~czasie stałym jako analogiczne operacje na liście.
Znalezienie i~ekstrakcja minimalnego elementu wiążą się jednak z~przeszukaniem całej listy, co zajmuje czas liniowy względem jej długości.
Możemy jednak wprowadzić usprawnienie, dzięki któremu operacja \proc{Minimum} będzie działać w~czasie stałym, i~jednocześnie nie pogorszymy czasów działania pozostałych operacji.
Wystarczy bowiem utrzymywać minimalny element listy w~jej głowie.
Wstawianie nowego elementu po takiej modyfikacji polega na umieszczeniu go w~głowie listy albo tuż za nią, w~zależności od tego, czy nowy element jest mniejszy od aktualnego minimum.
Operacja \proc{Extract-Min} po usunięciu głowy nadal jednak musi przeszukać całą listę w~celu znalezienia nowego minimum i~przeniesienia go do głowy listy.

W~operacji łączenia musimy pamiętać o~tym, że listy wejściowe nie zawsze reprezentują rozłączne zbiory, a~także o~tym, że w~głowie wynikowej listy powinien znaleźć się najmniejszy element z~obu łączonych list.
Zaproponujemy efektywną implementację operacji \proc{Union} uwzględniającą obie kwestie.
Listy reprezentujące kopce wejściowe byłyby sortowane, np.\ algorytmem sortowania przez scalanie dostosowanym do list jednokierunkowych, a~następnie scalane przy wykorzystaniu sposobu z~części (a).
Jeśli $n$ jest sumą elementów w~obu kopcach, to ich scalanie jesteśmy w~stanie zrealizować w~czasie $\Theta(n\lg n)$.

\subproblem %10-2(c)
Przypadek jest identyczny z~poprzednim, ale tym razem nie jest wymagane wykrywanie powtórzeń podczas operacji \proc{Union}.
Łączenie list możemy więc zaimplementować jako ich konkatenację, to znaczy uczynienie głowy jednej listy następnikiem ogona drugiej listy.
To, w~jakiej kolejności skleimy listy, będzie zależeć od tego, której z~nich głowa przechowuje mniejszy element, co ma na celu spełnienie usprawnienia opisanego w~poprzednim punkcie.
Operację łączenia możemy zrealizować w~czasie stałym, przy założeniu, że dla listy w~tej reprezentacji kopca będziemy pamiętać wskaźnik na jej ogon.

\bigskip
\noindent Zestawienie czasów działania poszczególnych operacji w~przypadkach pesymistycznych dla omawianych reprezentacji listowych kopców złączalnych przedstawiono w~tabeli \ref{tab:10-2}.

\begin{table}[!ht]
	\centering
		\[
			\begin{array}{l|c|c|c}
				& \text{Listy posortowane} & \text{Listy nieposortowane} & \text{Listy nieposortowane,} \\
				&  &  & \text{rozłączne zbiory w~\proc{Union}} \\
				\hline
				\proc{Make-Heap} & \Theta(1) & \Theta(1) & \Theta(1) \\
				\hline
				\proc{Insert} & \Theta(n) & \Theta(1) & \Theta(1) \\
				\hline
				\proc{Minimum} & \Theta(1) & \Theta(1) & \Theta(1) \\
				\hline
				\proc{Extract-Min} & \Theta(1) & \Theta(n) & \Theta(n) \\
				\hline
				\proc{Union} & \Theta(n) & \Theta(n\lg n) & \Theta(1)
			\end{array}
		\]
	\caption{Porównanie pesymistycznych czasów operacji słownikowych dla reprezentacji listowych kopców złączalnych.
W~operacji \proc{Union} $n$ oznacza sumę rozmiarów łączonych kopców.} \label{tab:10-2}
\end{table}
