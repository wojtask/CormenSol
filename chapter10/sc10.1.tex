\subchapter{Stosy i~kolejki}

\exercise %10.1-1
Ciąg operacji na stosie $S$ został przedstawiony na rys.\ \ref{fig:10.1-1}.
\begin{figure}[ht]
	\centering \begin{tikzpicture}[
	index node/.append style = {minimum size=2mm, text height=1.5ex, node distance=3.5mm and 3.5mm},
	row 1/.style = {nodes={fill=black!40, minimum size=3.5mm}},
	light cell/.style = {row 1 column #1/.style={nodes={fill=black!20}}},
	outer/.append style = {node distance=12mm and 8mm},
	arrow/.style = {<-, >=stealth},
	on grid
]

\node[outer] (pic a) {
\begin{tikzpicture}
	\matrix[array] (arr) { & & & & & \\};
	\foreach \x in {1,...,6} {
		\node[index node, above=of arr-1-\x] {\x};
	}
	\node[left=2mm of arr-1-1.west] {$S$};
	\draw[arrow] (arr-1-1.south)++(-3.5mm, 0mm) -- +(0mm, -7mm) node[below, index node] {$\attrib{S}{top}=0$};
\end{tikzpicture}
};

\node[outer, right=of pic a.east] (pic b) {
\begin{tikzpicture}
	\matrix[
		array,
		light cell/.list = {1}
	] (arr) { 4 & & & & & \\};
	\foreach \x in {1,...,6} {
		\node[index node, above=of arr-1-\x] {\x};
	}
	\node[left=2mm of arr-1-1.west] {$S$};
	\draw[arrow] (arr-1-1.south) -- +(0mm, -7mm) node[below, index node] {$\attrib{S}{top}=1$};
\end{tikzpicture}
};

\node[outer, right=of pic b.east] (pic c) {
\begin{tikzpicture}
	\matrix[
		array,
		light cell/.list = {1,2}		
	] (arr) { 4 & 1 & & & & \\};
	\foreach \x in {1,...,6} {
		\node[index node, above=of arr-1-\x] {\x};
	}
	\node[left=2mm of arr-1-1.west] {$S$};
	\draw[arrow] (arr-1-2.south) -- +(0mm, -7mm) node[below, index node] {$\attrib{S}{top}=2$};
\end{tikzpicture}
};

\node[outer, right=of pic c.east] (pic d) {
\begin{tikzpicture}
	\matrix[
		array,
		light cell/.list = {1,2,3}
	] (arr) { 4 & 1 & 3 & & & \\};
	\foreach \x in {1,...,6} {
		\node[index node, above=of arr-1-\x] {\x};
	}
	\node[left=2mm of arr-1-1.west] {$S$};
	\draw[arrow] (arr-1-3.south) -- +(0mm, -7mm) node[below, index node] {$\attrib{S}{top}=3$};
\end{tikzpicture}
};

\node[outer, below=of pic a.south] (pic e) {
\begin{tikzpicture}
	\matrix[
		array,
		light cell/.list = {1,2}
	] (arr) { 4 & 1 & 3 & & & \\};
	\foreach \x in {1,...,6} {
		\node[index node, above=of arr-1-\x] {\x};
	}
	\node[left=2mm of arr-1-1.west] {$S$};
	\draw[arrow] (arr-1-2.south) -- +(0mm, -7mm) node[below, index node] {$\attrib{S}{top}=2$};
\end{tikzpicture}
};

\node[outer, right=of pic e.east] (pic f) {
\begin{tikzpicture}
	\matrix[
		array,
		light cell/.list = {1,2,3}		
	] (arr) { 4 & 1 & 8 & & & \\};
	\foreach \x in {1,...,6} {
		\node[index node, above=of arr-1-\x] {\x};
	}
	\node[left=2mm of arr-1-1.west] {$S$};
	\draw[arrow] (arr-1-3.south) -- +(0mm, -7mm) node[below, index node] {$\attrib{S}{top}=3$};
\end{tikzpicture}
};

\node[outer, right=of pic f.east] (pic g) {
\begin{tikzpicture}
	\matrix[
		array,
		light cell/.list = {1,2}
	] (arr) { 4 & 1 & 3 & & & \\};
	\foreach \x in {1,...,6} {
		\node[index node, above=of arr-1-\x] {\x};
	}
	\node[left=2mm of arr-1-1.west] {$S$};
	\draw[arrow] (arr-1-2.south) -- +(0mm, -7mm) node[below, index node] {$\attrib{S}{top}=2$};
\end{tikzpicture}
};

\foreach \x in {a,b,c,d,e,f,g} {
	\node[subpicture label, below=2mm of pic \x.south] {(\x)};
}

\end{tikzpicture}

	\caption{Operacje wstawiania i~usuwania elementów na stosie $S$.
{\sffamily\bfseries(a)} Pusty stos $S$ reprezentowany jako tablica $S[1\twodots6]$.
{\sffamily\bfseries\doubledash{(b)}{(d)}} Stos $S$ po wykonaniu na nim kolejnych operacji \proc{Push}.
{\sffamily\bfseries(e)} Po usunięciu elementu ze stosu $S$ zmieniana jest jedynie pozycja atrybutu \attrib{S}{top}, natomiast sam element pozostaje w~tablicy.
{\sffamily\bfseries(f)} Wstawienie nowego elementu na stos $S$ nadpisuje stary element, który został usunięty w~poprzednim kroku.
{\sffamily\bfseries(g)} Stos $S$ po wykonaniu ostatniej operacji \proc{Pop}.} \label{fig:10.1-1}
\end{figure}

\exercise %10.1-2
Z~tablicą $A$ związujemy atrybuty \attrib{A}{left-top} i~\attrib{A}{right-top}, które będą wskazywać na pozycje wierzchołków, odpowiednio, pierwszego i~drugiego stosu.
Pierwszy stos będzie składał się z~elementów $A[1\twodots\attrib{A}{left-top}]$, a~drugi -- z~elementów $A[\attrib{A}{right-top}\twodots n]$.
Początkowo $\attrib{A}{left-top}=0$ i~$\attrib{A}{right-top}=n+1$.
Dodawanie i~usuwanie elementów w~pierwszym stosie działa identycznie jak dla pojedynczego stosu znajdującego się w~tablicy $A$, którego wierzchołek zajmuje pozycję \attrib{A}{left-top}.
Drugi stos zachowuje się symetrycznie do pierwszego -- podczas dodawania do niego nowego elementu atrybut \attrib{A}{right-top} będzie dekrementowany, a~podczas usuwania -- inkrementowany.
Operacje dodawania i~usuwania elementów działają oczywiście w~czasie stałym.
Jeśli $\attrib{A}{left-top}=0$, to pierwszy stos jest pusty, a~jeśli $\attrib{A}{right-top}=n+1$, to drugi stos jest pusty.
Przepełnienie ma miejsce tylko wtedy, gdy $\attrib{A}{left-top}=\attrib{A}{right-top}-1$ i~usiłujemy dodać nowy element do któregokolwiek stosu, czyli wówczas, gdy łączna liczba elementów na obu stosach wynosi $n$.

\exercise %10.1-3
Ciąg operacji na kolejce $Q$ ilustruje rys.\ \ref{fig:10.1-3}.

\begin{figure}[ht]
	\centering \begin{tikzpicture}[
	index node/.append style = {minimum size=2mm, text height=1.5ex, node distance=0.5mm},
	row 1/.style = {nodes={fill=black!40, minimum size=3.5mm}},
	row 1 column 7/.style = {nodes={draw=none, fill=none}},
	light cell/.style = {row 1 column #1/.style={nodes={fill=black!20}}},
	outer/.append style = {node distance=12mm and 5mm},
	arrow/.append style = {<-}
]

\newcommand\headarrow[2][south]{%
	\draw[arrow] (arr-1-#2.#1) -- +(0mm, -7mm) node[below, index node] {$\attrib{Q}{head}=#2$};
}
\newcommand\tailarrow[3][south]{%
	\draw[arrow] (arr-1-#2.#1) -- +(0mm, -3mm) node[below, anchor=#3, index node] {$\attrib{Q}{tail}=#2$};
}
\newcommand\arrayindexes{%
	\foreach \x in {1, ..., 6} {
		\node[index node, above=of arr-1-\x] {\x};
	}
	\node[left=2mm of arr-1-1.west] {$Q$};
}

\node[outer] (pic a) {
\begin{tikzpicture}
	\matrix[array] (arr) { & & & & & & \\};
	\arrayindexes
	\headarrow[240]{1}
	\tailarrow[300]{1}{165}
\end{tikzpicture}
};

\node[outer, right=of pic a] (pic b) {
\begin{tikzpicture}
	\matrix[
		array,
		light cell/.list = {1}
	] (arr) {4 & & & & & & \\};
	\arrayindexes
	\headarrow{1}
	\tailarrow{2}{160}
\end{tikzpicture}
};

\node[outer, right=of pic b] (pic c) {
\begin{tikzpicture}
	\matrix[
		array,
		light cell/.list = {1,2}		
	] (arr) {4 & 1 & & & & & \\};
	\arrayindexes
	\headarrow{1}
	\tailarrow{3}{120}
\end{tikzpicture}
};

\node[outer, right=of pic c] (pic d) {
\begin{tikzpicture}
	\matrix[
		array,
		light cell/.list = {1,2,3}
	] (arr) {4 & 1 & 3 & & & \\};
	\arrayindexes
	\headarrow{1}
	\tailarrow{4}{north}
\end{tikzpicture}
};

\node[outer, below=of pic a.south east, anchor=north east] (pic e) {
\begin{tikzpicture}
	\matrix[
		array,
		light cell/.list = {2,3}
	] (arr) {4 & 1 & 3 & & & & \\};
	\arrayindexes
	\headarrow{2}
	\tailarrow{4}{120}
\end{tikzpicture}
};

\node[outer, below=of pic b.south east, anchor=north east] (pic f) {
\begin{tikzpicture}
	\matrix[
		array,
		light cell/.list = {2,3,4}		
	] (arr) {4 & 1 & 3 & 8 & & & \\};
	\arrayindexes
	\headarrow{2}
	\tailarrow{5}{north}
\end{tikzpicture}
};

\node[outer, below=of pic c.south east, anchor=north east] (pic g) {
\begin{tikzpicture}
	\matrix[
		array,
		light cell/.list = {3,4}
	] (arr) {4 & 1 & 3 & 8 & & & \\};
	\arrayindexes
	\headarrow{3}
	\tailarrow{5}{120}
\end{tikzpicture}
};

\foreach \x in {d,e,f,g} {
	\node[subpicture label, below=2mm of pic \x.south] (label \x) {(\x)};
}
\foreach \x/\y in {a/e, b/f, c/g} {
	\node[subpicture label] at (label \y |- label d) {(\x)};
}

\end{tikzpicture}

	\caption{Operacje wstawiania i~usuwania elementów na kolejce $Q$.
{\sffamily\bfseries(a)} Pusta kolejka $Q$ reprezentowana jako tablica $Q[1\twodots6]$.
{\sffamily\bfseries\doubledash{(b)}{(d)}} Kolejka $Q$ po wykonaniu na niej kolejnych operacji \proc{Enqueue}.
{\sffamily\bfseries(e)} Po usunięciu elementu z~kolejki $Q$ zmieniana jest jedynie pozycja atrybutu \attrib{Q}{head}, natomiast sam element pozostaje w~tablicy.
{\sffamily\bfseries(f)} Wstawienie nowego elementu do kolejki $Q$ nadpisuje stary element, który został usunięty w~poprzednim kroku.
{\sffamily\bfseries(g)} Kolejka $Q$ po wykonaniu ostatniej operacji \proc{Dequeue}.} \label{fig:10.1-3}
\end{figure}

\exercise %10.1-4
Poniżej znajduje się pseudokod operacji analogicznej do \proc{Stack-Empty}, ale testującej pustość kolejki.
Wykorzystujemy ją podczas wykrywania błędów niedomiaru.

\begin{codebox}
\Procname{$\proc{Queue-Empty}(Q)$}
\li	\If $\attrib{Q}{head}=\attrib{Q}{tail}$
\li		\Then \Return \const{true}
\li		\Else \Return \const{false}
\End
\end{codebox}

Następujące wiersze należy dodać na początek procedury \proc{Enqueue}:
\begin{codebox}
\zi	\If $\attrib{Q}{head}=\attrib{Q}{tail}+1$
\zi		\Then \Error ,,nadmiar''
		\End
\zi	\If $\attrib{Q}{head}=1$ i~$\attrib{Q}{tail}=\attrib{Q}{length}$
\zi		\Then \Error ,,nadmiar''
		\End
\end{codebox}
Z~kolei poniższy fragment kodu umieszczamy na początku procedury \proc{Dequeue}:
\begin{codebox}
\zi	\If $\proc{Queue-Empty}(Q)$
\zi		\Then \Error ,,niedomiar''
		\End
\end{codebox}

\exercise %10.1-5
Kolejkę dwustronną implementujemy przy użyciu tablicy $D[1\twodots n]$.
Podobnie jak w~zwykłej kolejce atrybut \attrib{D}{head} wskazuje na początek kolejki, natomiast atrybut \attrib{D}{tail} wyznacza następną wolną pozycję, na którą można wstawić nowy element.
Procedury \proc{Head-Enqueue} oraz \proc{Head-Dequeue} mają na celu, odpowiednio, dodanie nowego elementu na początek kolejki i~usunięcie elementu z~początku kolejki.
Z~kolei procedury \proc{Tail-Enqueue} oraz \proc{Tail-Dequeue} implementują dodawanie i~usuwanie elementów na końcu kolejki.
Dla skrócenia zapisu pominięto sprawdzanie błędów niedomiaru i~przepełnienia.

\begin{codebox}
\Procname{$\proc{Head-Enqueue}(D,x)$}
\li	\If $\attrib{D}{head}=1$
\li		\Then $\attrib{D}{head}\gets\attrib{D}{length}$
\li		\Else $\attrib{D}{head}\gets\attrib{D}{head}-1$
		\End
\li	$D[\attrib{D}{head}]\gets x$
\end{codebox}

\begin{codebox}
\Procname{$\proc{Head-Dequeue}(D)$}
\li	$x\gets D[\attrib{D}{head}]$
\li	\If $\attrib{D}{head}=\attrib{D}{length}$
\li		\Then $\attrib{D}{head}\gets1$
\li		\Else $\attrib{D}{head}\gets\attrib{D}{head}+1$
		\End
\li	\Return $x$
\end{codebox}

\begin{codebox}
\Procname{$\proc{Tail-Enqueue}(D,x)$}
\li	$D[\attrib{D}{tail}]\gets x$
\li	\If $\attrib{D}{tail}=\attrib{D}{length}$
\li		\Then $\attrib{D}{tail}\gets1$
\li		\Else $\attrib{D}{tail}\gets\attrib{D}{tail}+1$
		\End
\end{codebox}

\begin{codebox}
\Procname{$\proc{Tail-Dequeue}(D,x)$}
\li	\If $\attrib{D}{tail}=1$
\li		\Then $\attrib{D}{tail}\gets\attrib{D}{length}$
\li		\Else $\attrib{D}{tail}\gets\attrib{D}{tail}-1$
		\End
\li	\Return $D[\attrib{D}{tail}]$
\end{codebox}

Wszystkie powyższe operacje działają w~czasie $\Theta(1)$.

\exercise %10.1-6
Wszystkie elementy kolejki będą trzymane na jednym stosie -- drugi stos będzie pełnił funkcję pomocniczą.
Dodawanie nowego elementu do kolejki oraz sprawdzanie czy kolejka jest pusta, nie różnią się od analogicznych operacji wykonywanych na pierwszym stosie.
Usuwanie elementu z~kolejki jest już operacją nieco bardziej skomplikowaną, gdyż należy pobrać element znajdujący się najgłębiej na tym stosie.
Ściągamy wpierw z~niego wszystkie elementy za pomocą operacji \proc{Pop} i~umieszczamy kolejno na drugim stosie, wywołując ciąg operacji \proc{Push}.
W~rezultacie drugi stos będzie zawierał wszystkie elementy kolejki w~odwrotnej kolejności.
Teraz pobieramy i~zapamiętujemy element ze szczytu drugiego stosu, gdyż za chwilę zwrócimy go jako wynik procedury.
Wcześniej trzeba bowiem przenieść pozostałą zawartość drugiego stosu z~powrotem do pierwszego, co przy okazji przywróci początkową kolejność elementów.

Łatwo sprawdzić, że testowanie pustości kolejki oraz dodawanie do niej nowego elementu, są wykonywane w~czasie stałym, natomiast usuwanie wymaga czasu proporcjonalnego do liczby elementów kolejki.

\exercise %10.1-7
Rozwiązanie jest analogiczne do rozwiązania poprzedniego zadania.
Sprawdzanie czy stos jest pusty, jak również dodawanie nowego elementu, to identyczne operacje wywoływane na pierwszej kolejce.
Usuwanie elementu ze stosu odbywa się poprzez pobranie wszystkich elementów z~wyjątkiem ostatniego z~pierwszej kolejki i~dodanie tych elementów do drugiej.
Ostatni z~nich także usuwamy z~kolejki, zapamiętawszy go w~celu późniejszego zwrócenia jako wyniku operacji.
Ostatnim krokiem jest przeniesienie reszty elementów z~powrotem do pierwszej kolejki.

Podobnie jak w~poprzednim zadaniu operacja odpowiedzialna za sprawdzenie czy stos jest pusty oraz dodawanie elementu działają w~czasie stałym, a~operacja usuwania -- w~czasie liniowym względem liczby elementów na stosie.
