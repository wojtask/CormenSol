\subchapter{Listy}

\exercise %10.2-1
Operację \proc{Insert}, dodającą nowy element $x$ na początek listy jednokierunkowej $L$, można zaimplementować w~czasie stałym -- wystarczy ustawić pole \attrib{x}{next} na głowę listy $L$ (albo na \const{nil}, jeśli lista $L$ jest pusta) i~uaktualnić wskaźnik \attrib{L}{head}.
\begin{codebox}
\Procname{$\proc{Singly-Linked-List-Insert}(L,x)$}
\li	$\attrib{x}{next}\gets\attrib{L}{head}$
\li	$\attrib{L}{head}\gets x$
\end{codebox}

W~operacji \proc{Delete} w~celu usunięcia elementu $x$ z~listy $L$ należy odnaleźć jego poprzednika $y$ i~zaktualizować pole \attrib{y}{next}, albo wskaźnik \attrib{L}{head}, gdy $x$ jest głową listy $L$.
Czynność ta w~najgorszym przypadku zajmuje czas proporcjonalny do liczby elementów listy $L$.
\begin{codebox}
\Procname{$\proc{Singly-Linked-List-Delete}(L,x)$}
\li	\If $x=\attrib{L}{head}$
\li		\Then $\attrib{L}{head}\gets\attribb{L}{head}{next}$
\li		\Else $y\gets\attrib{L}{head}$
\li			\While $\attrib{y}{next}\ne x$
\li				\Do $y\gets\attrib{y}{next}$
				\End
\li			$\attrib{y}{next}\gets\attrib{x}{next}$
		\End
\end{codebox}

Istnieje sposób na poprawienie średniego czasu działania operacji \proc{Delete}.
O~ile usuwany element $x$ nie jest ostatnim elementem listy, można wpierw skopiować z~\attrib{x}{next} do $x$ klucz wraz z~danymi dodatkowymi, a~następnie wskaźnik \id{next}, pozbawiając w~ten sposób listę oryginalnego następnika $x$ zamiast samego $x$.
Jednak gdy $x$ jest ogonem listy, nadal wymagane jest przejście przez całą listę, by zaktualizować poprzednika $x$.

\exercise %10.2-2
Wszystkie elementy implementowanego stosu będziemy trzymać na liście jednokierunkowej $L$ w~kolejności od szczytu w~głowie listy do dna stosu w~ogonie.
Dzięki takiemu rozwiązaniu operacje dodawania i~usuwania elementów będą działać w~czasie $\Theta(1)$.
Pseudokody operacji \proc{Push} i~\proc{Pop} dla opisanej implementacji stosu prezentujemy poniżej.
\begin{codebox}
\Procname{$\proc{Singly-Linked-List-Push}(L,k)$}
\li	$\attrib{x}{key}\gets k$
\li $\proc{Singly-Linked-List-Insert}(L,x)$
\end{codebox}

\begin{codebox}
\Procname{$\proc{Singly-Linked-List-Pop}(L)$}
\li	\If $\attrib{L}{head}=\const{nil}$
\li		\Then \Error ,,niedomiar''
		\End
\li	$x\gets\attrib{L}{head}$
\li	$\attrib{L}{head}\gets\attrib{x}{next}$
\li	\Return \attrib{x}{key}
\end{codebox}

\exercise %10.2-3
Elementy kolejki będziemy przechowywać na liście jednokierunkowej $L$ w~kolejności od głowy kolejki do jej ogona.
Aby jednak operacja dodawania była wykonalna w~czasie $\Theta(1)$, wymagane jest związanie z~listą $L$ dodatkowego atrybutu \attrib{L}{tail}, który będzie wskazywał na ogon listy $L$ albo na \const{nil}, jeżeli lista $L$ jest pusta.
Implementacje operacji \proc{Enqueue} i~\proc{Dequeue} dla listowej reprezentacji kolejki znajdują się poniżej.
\begin{codebox}
\Procname{$\proc{Singly-Linked-List-Enqueue}(L,k)$}
\li	$\attrib{x}{next}\gets\const{nil}$
\li	$\attrib{x}{key}\gets k$
\li	\If $\attrib{L}{tail}\ne\const{nil}$
\li		\Then $\attribb{L}{tail}{next}\gets x$
\li		\Else $\attrib{L}{head}\gets x$
		\End
\li	$\attrib{L}{tail}\gets x$
\end{codebox}

\begin{codebox}
\Procname{$\proc{Singly-Linked-List-Dequeue}(L)$}
\li	\If $\attrib{L}{head}=\const{nil}$
\li		\Then \Error ,,niedomiar''
		\End
\li	$x\gets\attrib{L}{head}$
\li	$\attrib{L}{head}\gets\attrib{x}{next}$
\li	\If $\attrib{L}{tail}=x$
\li		\Then $\attrib{L}{tail}\gets\const{nil}$
		\End
\li	\Return \attrib{x}{key}
\end{codebox}

\exercise %10.2-4
Zauważmy, że pole \attribb{L}{nil}{key} jest niewykorzystywane w~implementacji listy z~wartownikami.
Możemy zatem na początku działania procedury \proc{List-Search}$'$ nadać mu wartość $k$.
Pętla \kw{while} w~tej procedurze nie musi wówczas sprawdzać warunku, czy $x\ne\attrib{L}{nil}$, ponieważ jeśli na liście $L$ nie ma elementu o~kluczu $k$, to pętla zatrzyma się na wartowniku \attrib{L}{nil} i~zostanie on zwrócony jako wynik procedury.

\exercise %10.2-5
Operacja wstawiania nowego elementu na jednokierunkową listę cykliczną $L$ umieszcza go jako następnik głowy listy $L$.
Podczas operacji usuwania znajdowany jest poprzednik $y$ usuwanego elementu $x$, po czym uaktualnione zostaje \attrib{y}{next}.
Jeśli $x$ jest głową listy, to atrybut \attrib{L}{head} zostaje ustawiony na następnik $x$ albo na \const{nil}, w~przypadku gdy $x$ stanowi jedyny element listy $L$.
Wreszcie operacja wyszukiwania przechodzi całą listę $L$, począwszy od jej głowy, i~zatrzymuje się w~momencie odnalezienia elementu o~szukanym kluczu bądź w~chwili ponownego dotarcia do głowy listy $L$.

Poniższe procedury stanowią implementacje tych trzech operacji słownikowych.
\begin{codebox}
\Procname{$\proc{Circular-List-Insert}(L,x)$}
\li	\If $\attrib{L}{head}=\const{nil}$
\li		\Then $\attrib{L}{head}\gets\attrib{x}{next}\gets x$
\li		\Else $\attrib{x}{next}\gets\attribb{L}{head}{next}$
\li			$\attribb{L}{head}{next}\gets x$
		\End
\end{codebox}

\begin{codebox}
\Procname{$\proc{Circular-List-Delete}(L,x)$}
\li	$y\gets\attrib{L}{head}$
\li	\While $\attrib{y}{next}\ne x$
\li		\Do $y\gets\attrib{y}{next}$
		\End
\li	$\attrib{y}{next}\gets\attrib{x}{next}$
\li	\If $\attrib{L}{head}=x$
\li		\Then \If $\attrib{x}{next}=x$
\li				\Then $\attrib{L}{head}\gets\const{nil}$
\li				\Else $\attrib{L}{head}\gets\attrib{x}{next}$
				\End
		\End
\end{codebox}

\begin{codebox}
\Procname{$\proc{Circular-List-Search}(L,k)$}
\li	\If $\attrib{L}{head}=\const{nil}$
\li		\Then \Return \const{nil}
		\End
\li	\If $\attribb{L}{head}{key}=k$
\li		\Then \Return \attrib{L}{head}
		\End
\li	$x\gets\attribb{L}{head}{next}$
\li	\While $x\ne\attrib{L}{head}$
\li		\Do \If $\attrib{x}{key}=k$
\li				\Then \Return $x$
				\End
\li			$x\gets\attrib{x}{next}$
		\End
\li	\Return \const{nil}
\end{codebox}

Łatwo sprawdzić, że procedura implementująca operację \proc{Insert} działa w~czasie stałym, natomiast pozostałe dwie procedury dla listy o~$n$ elementach w~pesymistycznym przypadku potrzebują czasu $\Theta(n)$.

\exercise %10.2-6
Użyjemy dwukierunkowych list cyklicznych z~wartownikami do reprezentowania łączonych zbiorów.
Zbiory $S_1$ i~$S_2$ są rozłączne, więc w~reprezentacji sumy $S_1\cup S_2$ nie pojawią się powtarzające się elementy.
Operacja \proc{Union} będzie więc jedynie ,,sklejać'' listy $L_1$ i~$L_2$ implementujące zbiory $S_1$ i~$S_2$.
Podczas działania tej operacji następnikiem ogona listy $L_1$ staje się głowa listy $L_2$, a~ogon listy $L_2$ zostaje umieszczony tuż przed wartownikiem \attrib{L_1}{nil}.
Powstała lista $L_1$ reprezentuje teraz sumę $S_1\cup S_2$, zaś lista $L_2$ ulega zniszczeniu.
\begin{codebox}
\Procname{$\proc{List-Union}(L_1,L_2)$}
\li	$\attribbb{L_2}{nil}{next}{prev}\gets\attribb{L_1}{nil}{prev}$
\li	$\attribbb{L_1}{nil}{prev}{next}\gets\attribb{L_2}{nil}{next}$
\li	$\attribbb{L_2}{nil}{prev}{next}\gets\attrib{L_1}{nil}$
\li	$\attribb{L_1}{nil}{prev}\gets\attribb{L_2}{nil}{prev}$
\li	\Return $L_1$
\end{codebox}

\exercise %10.2-7
W~algorytmie wykorzystamy pomocniczą listę jednokierunkową $L'$, na którą będziemy wstawiać kolejno usuwane elementy z~głowy listy $L$.
Łatwo zauważyć, że po przeprowadzeniu tego ciągu operacji przeniesione elementy będą znajdować się w~$L'$ w~porządku odwrotnym względem początkowego ustawienia na liście $L$.
Przestawienie atrybutu \attrib{L}{head} na \attrib{L'}{head} wystarczy, aby w~jednym kroku przenieść całą zawartość listy $L'$ do listy $L$.
\begin{codebox}
\Procname{$\proc{Singly-Linked-List-Reverse}(L)$}
\li	$\attrib{L'}{head}\gets\const{nil}$
\li	\While $\attrib{L}{head}\ne\const{nil}$
\li		\Do $x\gets\attrib{L}{head}$
\li			$\proc{Singly-Linked-List-Delete}(L,\attrib{L}{head})$
\li			$\proc{Singly-Linked-List-Insert}(L',x)$
		\End
\li	$\attrib{L}{head}\gets\attrib{L'}{head}$
\end{codebox}

Procedury \proc{Singly-Linked-List-Insert} i~\proc{Singly-Linked-List-Delete} zostały zaprezentowane w~\refExercise{10.2-1}.
Ich wywołania w~powyższym algorytmie zajmują czas stały, stąd czasem działania algorytmu jest $\Theta(n)$.

\exercise %10.2-8
Zgodnie z~opisem w~treści zadania przyjmijmy, że każdy element listy zamiast wskaźników na poprzednik i~następnik posiada atrybut \id{np}, będący alternatywą wykluczającą tych dwóch wskaźników.
Załóżmy, że znamy adres elementu $x$ na liście $L$ i~elementu $y$ będącego poprzednikiem $x$ w~$L$.
Niech $z$ będzie następnikiem $x$ na tej liście.
Wówczas $\attrib{x}{np}=z\func{XOR}y$, zatem na podstawie własności operacji \func{XOR}, aby dostać się do $z$, wystarczy obliczyć wartość
\[
    z = z\func{XOR}0 = z\func{XOR}{}(y\func{XOR}y) = (z\func{XOR}y)\func{XOR}y = \attrib{x}{np}\func{XOR}y.
\]
Podobnie ma się rzecz podczas wyznaczania $y$ na podstawie $x$ i~$z$ -- wówczas $y=\attrib{x}{np}\func{XOR}z$.
Zauważmy, że jeśli element $x$ jest ogonem listy, to $\attrib{x}{np}=\const{nil}\func{XOR}y=0\func{XOR}y=y$, skąd mamy $z=\attrib{x}{np}\func{XOR}y=0=\const{nil}$ i~analogicznie dla głowy listy, z~faktu, że $\attrib{x}{np}=z\func{XOR}0=z$ wynika $y=\const{nil}$.
Podobnie jak w~zwykłych listach przyjmujemy, że atrybut \attrib{L}{head} przechowuje wskaźnik do głowy listy $L$.
Ponadto z~listą związujemy atrybut \attrib{L}{tail}, który będzie wskazywał na jej ogon -- jest on konieczny do tego, aby operacja odwracania kolejności elementów na liście działała w~czasie stałym.

Poniżej przedstawiono procedury implementujące operacje \proc{Search}, \proc{Insert} i~\proc{Delete} dla takiej listy.
\begin{codebox}
\Procname{$\proc{XOR-Linked-List-Search}(L,k)$}
\li	$x\gets\attrib{L}{head}$
\li	$y\gets\const{nil}$
\li	\While $x\ne\const{nil}$ i~$\attrib{x}{key}\ne k$
\li		\Do $z\gets\attrib{x}{np}\func{XOR}y$
\li			$y\gets x$
\li			$x\gets z$
		\End
\li	\Return $x$
\end{codebox}

\begin{codebox}
\Procname{$\proc{XOR-Linked-List-Insert}(L,x)$}
\li	$\attrib{x}{np}\gets\attrib{L}{head}$
\li	\If $\attrib{L}{head}\ne\const{nil}$
\li		\Then $\attribb{L}{head}{np}\gets\attribb{L}{head}{np}\func{XOR}x$
		\End
\li	$\attrib{L}{head}\gets x$
\li	\If $\attrib{L}{tail}=\const{nil}$
\li		\Then $\attrib{L}{tail}\gets x$
		\End
\end{codebox}

\begin{codebox}
\Procname{$\proc{XOR-Linked-List-Delete}(L,x)$}
\li	$x'\gets\attrib{L}{head}$
\li	$y\gets\const{nil}$
\li	\While $x'\ne x$
\li		\Do $z\gets\attrib{x'}{np}\func{XOR}y$
\li			$y\gets x'$
\li			$x'\gets z$
		\End
\li	$z\gets\attrib{x}{np}\func{XOR}y$
\li	\If $x=\attrib{L}{head}$
\li		\Then $\attrib{L}{head}\gets z$
\li		\Else $\attrib{y}{np}\gets\attrib{y}{np}\func{XOR}x\func{XOR}z$
		\End
\li	\If $x=\attrib{L}{tail}$
\li		\Then $\attrib{L}{tail}\gets y$
\li		\Else $\attrib{z}{np}\gets\attrib{z}{np}\func{XOR}x\func{XOR}y$
		\End
\end{codebox}

Procedury \proc{XOR-Linked-List-Search} i~\proc{XOR-Linked-List-Insert} są analogiczne do swoich odpowiedników dla zwykłej listy dwukierunkowej i~pesymistyczny czas ich działania wynosi odpowiednio $\Theta(n)$ oraz $\Theta(1)$.
W~procedurze \proc{XOR-Linked-List-Delete} należy odszukać na liście poprzednika i~następnika elementu $x$, aby uaktualnić ich pola \id{np}, co w~pesymistycznym przypadku zajmuje czas $\Theta(n)$.
Odwracanie kolejności elementów listy w~opisanej tu implementacji można zrealizować poprzez zamianę wartości jej atrybutów $head$ i~$tail$.
