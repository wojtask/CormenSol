\chapter{Wstęp}

Niniejsze opracowanie zawiera rozwiązania zadań i~problemów pochodzących z~monografii \textsl{Introduction to Algorithms} \cite{cormen} autorstwa Thomasa~H.~Cormena, Charlesa~E.~Leisersona, Ronalda~L.~Rivesta i Clifforda~Steina, na~podstawie jej drugiego wydania. Autor niniejszej pozycji korzystał z~polskiego tłumaczenia pt. \textsl{Wprowadzenie do algorytmów} \cite{cormenpl} w~wydaniu szóstym.

Opracowanie to nie jest jeszcze gotowe -- zawiera rozwiązania jedynie z~rozdziałów należących do części pierwszej (rozdz.~1\nobreakdash--5) oraz dodatków wypełniających część ósmą \textsl{Wprowadzenia}. Kolejne iteracje będą sukcesywnie uzupełniane o~rozwiązania zadań z~kolejnych części, wprowadzając jednocześnie poprawki znalezionych błędów i~doskonaląc niektóre starsze rozwiązania.

Moim celem było stworzenie kompletnego podręcznika z~rozwiązaniami, który pomaga w~przyswajaniu materiału z~\textsl{Wprowadzenia} i~służy jako wyrocznia po uprzedniej próbie rozwiązania konkretnego zadania przez Czytelnika. Oczywiście zachęcam do takich prób zamiast natychmiastowego zaglądania do rozwiązania -- z~pewnością nauczysz się więcej, a~samo zmierzenie się z~problemem może dać Ci dużo satysfakcji. Wyznaczony przeze mnie cel rzetelności przedstawianych tu wywodów zaowocował przeładowaniem treści językiem formalnym i~w~pewnych miejscach niewątpliwie trudnym, ale należy mieć na uwadze to, że dokładność i~precyzja są nieodłącznymi cechami pozycji matematycznych.

Czym różni się bieżąca pozycja od kilku podobnych, na które można natknąć się w Internecie? Udało mi się odnaleźć tylko dwie, przy czym jedna z~nich jest oficjalnym podręcznikiem lansowanym przez Autorów książki. Nie pokrywają jednak całości materiału, poza tym w~kilku miejscach zauważyłem, że przedstawione rozumowania odbiegają nieco od prawdy. Z~moich poszukiwań wynika także, że nie istnieje podobne opracowanie w~języku polskim. Ze znalezionych materiałów nieco skorzystałem, zawsze jednak miałem na uwadze własne zrozumienie rozwiązania i~przekształcenie go do postaci, moim zdaniem najbardziej odpowiedniej i~w~miarę możliwości krótkiej i~zwięzłej.

Dołożyłem wszelkich starań, aby każde rozwiązanie zostało dokładnie sprawdzone. Jeśli jednak znalazłeś błąd, merytoryczny lub typograficzny, bądź twierdzisz, że potrafisz rozwiązać pewien problem znacząco krócej lub sprytniej, powiadom mnie o~tym niezwłocznie, pisząc na adres \url{kwojtas@student.agh.edu.pl}.

Co ciekawe, w międzyczasie ukazało się już trzecie wydanie \textsl{Wprowadzenia} \cite{cormen3}, które zawiera wiele nowych treści i~problemów. Dlatego też, niedługo po ukończeniu przeze mnie opracowania dla wydania drugiego, o~ile pojawi się równie dobre polskie tłumaczenie, można spodziewać się publikacji rozwiązań zadań z~nowej edycji.

Chciałbym podziękować Autorom \textsl{Wprowadzenia do algorytmów} za masę zabawy i~satysfakcji, jakich mi dostarczyli oraz Tłumaczom polskiego wydania za dobry przekład tego bestsellerowego tytułu.

\bigskip
\noindent{\sl Kraków, 29 grudnia 2009}\hfill--- K. W.

\endinput
