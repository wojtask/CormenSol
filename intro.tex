\originalchapter{Wstęp}

Niniejsze opracowanie zawiera rozwiązania zadań i~problemów z~drugiego wydania monografii \textsl{Introduction to Algorithms} \cite{clrs2} autorstwa Thomasa H.\ Cormena, Charlesa E.\ Leisersona, Ronalda L.\ Rivesta i~Clifforda Steina.
Treści rozwiązań powstały na podstawie polskiego tłumaczenia pt.\ \textsl{Wprowadzenie do algorytmów} \cite{clrs2pl} w~wydaniu szóstym.
Obecne wydanie swoją treścią pokrywa rozdziały wchodzące w~skład części I--IV (rozdz.\ 1\nbendash17) oraz dodatki wypełniające część VIII\@.

Jako cel powziąłem sobie stworzenie rzetelnego i~kompletnego zbioru rozwiązań, służącego jako uzupełnienie w~studiowaniu materiału z~Podręcznika.
Moim zamierzeniem co do opracowywanych treści było zapewnienie przede wszystkim ich poprawności i~kompletności, a~także spójności z~treściami zadań i~materiałem w~Podręczniku oraz technicznej elegancji.
W~związku z~tym spędziłem mnóstwo czasu na weryfikację rozwiązań, nie tylko pod kątem merytorycznym, ale także językowym i~typograficznym.
Zwracam uwagę na dostarczanie optymalnych algorytmów, które następnie implementuję i~testuję, ilustruję operacje i~przykłady dokładnymi i~spójnymi z~tekstem rysunkami, wreszcie dbam o~spójność tekstów rozwiązań z~tekstem Podręcznika, także tym oryginalnym.

Niewielka część mojej pracy jest też odtwórcza.
Należy wspomnieć, że sami Autorowie Podręcznika dostarczają rozwiązań niektórych zadań i~problemów -- można je pobrać ze \href{http://mitpress.mit.edu/algorithms}{strony WWW książki}.
W~sieci można znaleźć też rozwiązania innych autorów, np.:
\begin{itemize}
    \item \href{http://sites.math.rutgers.edu/~ajl213/CLRS/CLRS.html}{rozwiązania autorstwa Michelle Bodnar i~Andrew Lohra},
    \item \href{https://ita.skanev.com}{rozwiązania autorstwa Stefana Kaneva},
    \item \href{https://donrwalsh.github.io/CLRS/}{rozwiązania autorstwa Dona R.\ Walsha},
    \item \href{https://walkccc.github.io/CLRS}{rozwiązania koordynowane przez Peng-Yu Chena, opracowywane na zasadach crowdsourcingu},
    \item \href{https://quizlet.com/explanations/textbook-solutions/introduction-to-algorithms-2nd-edition-9780262032933}{strona książki na portalu Quizlet}.
\end{itemize}
Żadne z~tych źródeł nie pokrywa wszystkich zadań z~Podręcznika albo niektóre z~tych rozwiązań nie są najwyższej jakości.
Jednak prace te stanowiły dla mnie źródło inspiracji, innych podejść i~punktów widzenia.
Korzystając z~nich, zawsze miałem na celu stworzenie na ich podstawie autorskich wersji rozwiązań, niejednokrotnie poprawionych i~ulepszonych.
W~porównaniu z~poprzednim wydaniem opracowania, obecne nie zawiera rozwiązań nowych zadań, ale ogrom wprowadzonych poprawek i~ulepszeń do istniejących, dzięki odniesieniu się do powyższych źródeł, skłonił mnie do opublikowania ich jako wersji 0.7.

Rozwiązania często powołują się na fakty przedstawione w~Podręczniku, wymagana jest zatem znajomość całego materiału przynajmniej z~bieżącego rozdziału.
W~wielu tekstach rozwiązań znajdziemy także odnośniki do innych zadań, szczególnie wówczas, gdy dane zadanie korzysta z~wyniku innego we własnym rozwiązaniu.
Na ogół wykorzystywane są rozwiązania zadań występujących w~tekście wcześniej względem danego, choć nie jest to regułą.
W~początkowych rozdziałach można zatem zaobserwować nieco większą koncentrację na szczegółach, a~w~późniejszych -- więcej odsyłaczy do zadań, w~których szczegóły te zostały już omówione.

O~każdym znalezionym błędzie lub nieścisłości w~treściach zadań zwracam uwagę w~krótkich notkach przed rozwiązaniem danego zadania.
Bazuję przede wszystkim na polskim tłumaczeniu, ale wskazuję, czy błąd występuje także w~oryginale.
Uwzględniam jednakże \href{http://www.cs.dartmouth.edu/~thc/clrs-2e-bugs/bugs.php}{erratę} do oryginału -- jeśli znajduje się w~niej wpis o~pewnej poprawce, to przyjmuję, że błąd został naprawiony i~wspominam o~jego istnieniu tylko wówczas, gdy występuje w~tłumaczeniu.

Jak podkreśliłem wcześniej, przykładam szczególną uwagę do zapewnienia poprawności przedstawianych algorytmów i~operacji na strukturach danych.
Każdy pseudokod lub opis algorytmu zamieszczony w~rozwiązaniach, jak i~w~Podręczniku przez Autorów, implementuję w~języku Python i~testuję pod kątem poprawności.
W~planach jest też przygotowanie swego rodzaju testów obciążeniowych, mających na celu (przynajmniej do pewnego stopnia) uzasadnienie złożoności czasowych implementowanych algorytmów.
Projekt z~implementacjami dostępny jest na \href{https://github.com/wojtask/CormenPy}{GitHub}.
Nie zalecam jednak wykorzystywania go jako biblioteki algorytmów do rzeczywistych zastosowań, ponieważ głównym celem projektu jest jak najwierniejsze odwzorowanie pseudokodów w~rzeczywistym języku programowania i~dokładne przetestowanie powstałego kodu.

Opracowanie w~obecnej formie powstaje od 2009 roku.
Z~powodu mojego perfekcjonistycznego podejścia w~przygotowywaniu rozwiązań i~tego, że przeznaczam na to swój wolny czas, postęp prac jest niezwykle powolny.
Od rozpoczęcia projektu, w~2009 roku ukazało się trzecie oryginalne wydanie Podręcznika \cite{clrs3}, a~w~2022 roku -- jego czwarte wydanie \cite{clrs4}.
W~związku z~tym podjąłem decyzję o~zamrożeniu obecnego projektu i~przeniesieniu uwagi w~całości na opracowanie \href{https://github.com/wojtask/clrs4e-solutions}{rozwiązań do czwartego wydania}.
Poniżej zamieszczam plan rozwoju obecnego projektu po ewentualnym odmrożeniu go w~przyszłości, jednak dokończenie go będzie zależało od tego, czy wciąż będę widział taką konieczność po zakończeniu prac nad najnowszym wydaniem:
\begin{itemize}
    \item \textbf{wersja 0.8}: dodanie rozwiązań z~części V, dokończenie problemu \refProblem{16-4},
    \item \textbf{wersja 0.9}: dodanie rozwiązań z~części VI, weryfikacja problemu \refProblem{15-5},
    \item \textbf{wersja 1.0}: dodanie rozwiązań z~części VII, ostateczne poprawki.
\end{itemize}

Dokument został przygotowany w~systemie \LaTeXe, który pozwala na precyzyjną i~estetyczną prezentację nie tylko tekstu technicznego, ale także formuł matematycznych, tabel i~pseudokodów, dbając także o~odpowiedni układ stron, sekcji, paragrafów itd.
Do składania pseudokodów użyłem pakietu \texttt{clrscode} opracowanego przez Thomasa H.\ Cormena na potrzeby pseudokodów stosowanych w~Podręczniku.
Ilustracje zostały przygotowane w~językach PGF/TikZ \cite{pgfmanual}, przy dążeniu do uzyskania spójnego stylu do tego z~Podręcznika.

Dołożyłem wszelkich starań, aby każde rozwiązanie zostało dokładnie sprawdzone.
Jeśli jednak znalazłeś błąd merytoryczny, językowy, typograficzny lub jakikolwiek inny, bądź twierdzisz, że jesteś w~stanie ulepszyć aktualne rozwiązanie, zgłoś to proszę na \href{https://github.com/wojtask/CormenSol/issues/new}{GitHubie projektu}.
Jestem otwarty na każdą sugestię i~chęć pomocy.

\bigskip
\bigskip
\noindent{\sl Kraków, grudzień 2022}\hfill--- Krzysztof Wojtas
