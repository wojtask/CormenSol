\chapter{Wstęp}

Niniejsza pozycja prezentuje rozwiązania zadań i~problemów zamieszczonych w~monografii \textsl{Introduction to Algorithms} \cite{cormen} autorstwa Thomasa~H.~Cormena, Charlesa~E.~Leisersona, Ronalda~L.~Rivesta i~Clifforda~Steina, na~podstawie jej drugiego wydania. Przy opracowywaniu rozwiązań korzystałem równolegle z~polskiego tłumaczenia pt.~\textsl{Wprowadzenie do algorytmów} \cite{cormenpl} w~wydaniu szóstym.

Opracowanie to nie jest jeszcze gotowe -- pokrywa jedynie rozdziały należące do części pierwszej (rozdz.~1\nobreakdash--5), drugiej (rozdz.~6\nobreakdash--9) oraz dodatki wypełniające część ósmą \textsl{Wprowadzenia}. Kolejne iteracje będą sukcesywnie uzupełniane o~rozwiązania zadań z~kolejnych części, wprowadzając jednocześnie poprawki znalezionych błędów i~doskonaląc niektóre starsze rozwiązania.

Moim celem było stworzenie kompletnego podręcznika z~rozwiązaniami, który pomaga w~przyswajaniu materiału z~\textsl{Wprowadzenia} i~służy jako wyrocznia po uprzedniej próbie rozwiązania konkretnego zadania przez Czytelnika. Oczywiście zachęcam do takich prób zamiast natychmiastowego zaglądania do odpowiedzi -- z~pewnością nauczysz się więcej, a~samo zmierzenie się z~problemem może dać Ci dużo satysfakcji. Wyznaczony przeze mnie cel rzetelności przedstawionych tu treści zaowocował zastosowaniem języka formalnego i~w~pewnych miejscach niewątpliwie trudnego, jednak należy mieć na uwadze to, że dokładność i~precyzja są nieodłącznymi cechami pozycji matematycznych.

Okazało się, że na pomysł rozwiązania zadań z~\textsl{Wprowadzenia} wpadło już przede mną kilka osób. Udało mi się odnaleźć w~Internecie tylko dwa takie projekty, przy czym jeden z~nich jest oficjalnym podręcznikiem lansowanym przez Autorów książki. Żaden nie pokrywa jednak całości materiału, poza tym w~kilku miejscach zauważyłem, że przedstawione rozumowania są niezbyt precyzyjne, a~kilka z~nich zostało nawet przeprowadzonych niepoprawnie. Z~moich poszukiwań wynika także, że nie istnieje podobne opracowanie w~języku polskim. Ze znalezionych materiałów nieco skorzystałem, zawsze jednak miałem na uwadze własne zrozumienie rozwiązania i~przekształcenie go do postaci, moim zdaniem najbardziej odpowiedniej i~w~miarę możliwości krótkiej i~zwięzłej.

Dołożyłem wszelkich starań, aby każde rozwiązanie zostało dokładnie sprawdzone. Jeśli jednak znalazłeś błąd merytoryczny lub typograficzny, bądź twierdzisz, że potrafisz rozwiązać pewien problem znacząco krócej lub sprytniej, powiadom mnie o~tym niezwłocznie, pisząc na adres \url{kwojtas@gmail.com}.

Co ciekawe, w~międzyczasie ukazało się już kolejne, trzecie wydanie książki \cite{cormen3}, które zawiera wiele nowych treści i~problemów. Dlatego też, zaraz po ukończeniu przeze mnie opracowania dla wydania drugiego, można spodziewać się publikacji rozwiązań zadań z~nowej edycji.

Materiał tu zaprezentowany nie jest objęty żadną licencją, możesz go używać i~kopiować do własnych celów, ale pamiętaj o~dołączeniu do kopii mojego nazwiska oraz adresu e-mail.

Chciałbym podziękować Autorom \textsl{Wprowadzenia do algorytmów} za ogrom wiedzy i~doświadczenia, jakich mi dostarczyli oraz Tłumaczom polskiego wydania za dobry przekład tego bestsellerowego tytułu.

\bigskip
\noindent{\sl Kraków, 26 lipca 2010}\hfill--- K.~W.

\endinput
