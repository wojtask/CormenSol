\chapter{\sffamily\bfseries Wstęp}

Niniejsza pozycja prezentuje rozwiązania do zadań i~problemów zawartych w~monografii \textit{Introduction to Algorithms} autorstwa Thomasa~H.~Cormena, Charlesa~E.~Leisersona, Ronalda~L.~Rivesta i Clifforda~Steina, na~podstawie jej drugiego wydania. Autor niniejszego opracowania korzystał z~polskiego tłumaczenia \textit{Wprowadzenie do algorytmów} w~wydaniu szóstym.

Opracowanie to nie jest jeszcze gotowe -- zawiera rozwiązania jedynie z~rozdziałów należących do części pierwszej (rozdz.~1\nobreakdash--5) oraz dodatków wypełniających część ósmą \textit{Wprowadzenia}. Kolejne iteracje opracowania będą sukcesywnie uzupełniane o~rozwiązania zadań z~kolejnych części, wprowadzając jednocześnie poprawki znalezionych błędów i~doskonaląc niektóre starsze rozwiązania.

Moim celem było stworzenie kompletnego podręcznika, w~który pomaga w przyswajaniu materiału z~\textit{Wprowadzenia} i~służy jako wyrocznia po uprzedniej próbie rozwiązania konkretnego zadania przez Czytelnika. Oczywiście zachęcam do takich prób zamiast natychmiastowego sięgnięcia po niniejszy podręcznik -- z~pewnością nauczysz się więcej, a~samo rozwiązanie może dać Ci dużo satysfakcji. Wyznaczony przeze mnie cel rzetelności przedstawianych tu wywodów zaowocował przeładowaniem treści językiem formalnym i~w~pewnych miejscach niewątpliwie trudnym, ale należy mieć na celu dążenie do precyzji i~perfekcji, co cechuje pozycje matematyczne.

Czym różni się bieżąca pozycja od kilku podobnych, na które można natknąć się w Internecie? Udało mi się ich odnaleźć dosłownie kilka, ale żadna nie pokrywa całości materiału, poza tym w wielu miejscach wywody dalekie są od prawdy. Z moich poszukiwań wynika też, że nie istnieje podobne opracowanie w~języku polskim. Z materiałów tych nieco skorzystałem, zawsze jednak miałem na uwadze własne zrozumienie rozwiązania i~przekształcenie go do postaci, moim zdaniem perfekcyjnej.

Perfekcja\dots\ Nigdy nie będzie taka, jaką cechują się publikacje Donalda Knutha:) Jeśli zatem znalazłeś błąd, merytoryczny lub typograficzny, bądź twierdzisz, że potrafisz rozwiązać pewne zadanie znacząco krócej lub sprytniej, powiadom mnie o~tym niezwłocznie pisząc na adres \url{kwojtas@student.agh.edu.pl}.

Chciałbym podziękować Autorom \textit{Wprowadzenia do algorytmów} za masę zabawy, jakiej mi dostarczyli oraz Tłumaczom polskiego wydania za dobre tłumaczenie tego bestsellerowego tytułu. Szczególne podziękowania kieruję do Profesora Donalda~E.~Knutha za perfekcjonizm, którego się od Niego nauczyłem i~który pragnąłem naśladować opracowując niniejsze rozwiązania.

\bigskip
\noindent{\sl Kraków, 12 grudnia 2009}\hfill--- K. W.

\endinput
