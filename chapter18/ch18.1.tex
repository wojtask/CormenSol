\subchapter{Definicja B-drzewa}

\exercise %18.1-1
Gdyby $t=1$, to każdy węzeł wewnętrzny takiego drzewa miałby co najwyżej jednego syna, czyli drzewo to byłoby zdegenerowane do listy.
Operacje na zbiorach dynamicznych zaimplementowanych na takim drzewie byłyby proporcjonalne do liczby liści tego drzewa.

\exercise %18.1-2
W~drzewie z~rys.\ 18.1 korzeń ma 1 klucz, a~każdy inny węzeł wewnętrzny ma 2 lub 3 klucze.
Taka sytuacja możliwa jest tylko dla $t=2$ lub $t=3$.

\exercise %18.1-3
\exercise %18.1-4
\exercise %18.1-5
