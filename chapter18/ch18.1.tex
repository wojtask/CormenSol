\subchapter{Definicja B-drzewa}

\exercise %18.1-1
Gdyby $t=1$, to każdy węzeł wewnętrzny takiego drzewa miałby co najwyżej jednego syna, czyli drzewo to byłoby zdegenerowane do listy.
Operacje na zbiorach dynamicznych zaimplementowanych na takim drzewie byłyby proporcjonalne do liczby liści tego drzewa.

\exercise %18.1-2
W~drzewie z~rys.\ 18.1 korzeń ma 1 klucz, a~każdy inny węzeł wewnętrzny ma 2 lub 3 klucze.
Taka sytuacja możliwa jest tylko dla $t=2$ lub $t=3$.

\exercise %18.1-3

\exercise %18.1-4
Rozważmy B-drzewo o~minimalnym stopniu $t$, w~którym liczba kluczy w~każdym węźle jest maksymalna, czyli wynosi $2t-1$.
Jest to więc pełne drzewo \singledash{$(2t)$}{arne}, w~którym na głębokości $i$ znajduje się łącznie $(2t)^i$ węzłów.
W~dowolnym B-drzewie o~wysokości $h$ sumaryczna liczba węzłów spełnia zatem nierówność
\[
	n \le \sum_{i=0}^h(2t)^i = \frac{(2t)^{h+1}-1}{2t-1}.
\]
Każdy węzeł posiada co najwyżej $2t-1$ kluczy, więc B-drzewo może pomieścić co najwyżej $(2t)^{h+1}-1$ kluczy.

\exercise %18.1-5
