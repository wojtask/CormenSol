\subchapter{Podstawowe operacje na B-drzewach}

\exercise %18.2-1
\exercise %18.2-2
\exercise %18.2-3
Operacja znajdowania minimalnego klucza w~B-drzewie jest uogólnieniem analogicznej operacji dla drzew BST.
Gdy w~danej iteracji odwiedzany jest węzeł wewnętrzny $x$, to w~kolejnej odwiedzony zostanie węzeł $\attrib{x}{c_1}$.
Po dotarciu do liścia $x$ zwrócony zostanie najmniejszy jego klucz, czyli $\attrib{x}{key_1}$.
\begin{codebox}
\Procname{$\proc{B-Tree-Minimum}(x)$}
\li	\While $\attrib{x}{leaf}=\const{false}$
\li		\Do $\proc{Disk-Read}(\attribxx{x}{c_1})$
\li     $x\gets\attribxx{x}{c_1}$
		\End
\li	\Return $\attribxx{x}{key_1}$
\end{codebox}

Podamy też pseudokod operacji znajdowania maksymalnego klucza, gdyż potrzebna nam ona będzie w~implementacji wyznaczania poprzednika.
Operacja różni się od \proc{B-Tree-Minimum} tym, że porusza się po skrajnie prawych synach węzłów wewnętrznych i~zwraca największy klucz napotkanego liścia.
\begin{codebox}
\Procname{$\proc{B-Tree-Maximum}(x)$}
\li	\While $\attrib{x}{leaf}=\const{false}$
\li		\Do $\proc{Disk-Read}(\attribxx{x}{c_{\attrib{x}{n}+1}})$
\li     $x\gets\attribxx{x}{c_{\attrib{x}{n}+1}}$
		\End
\li	\Return $\attribxx{x}{key_{\attrib{x}{n}}}$
\end{codebox}

Zastanówmy się teraz, jak wyznaczyć poprzednik klucza $k=\attribxx{x}{key_i}$, gdzie $x$ jest węzłem B-drzewa $T$.
Jeżeli $x$ jest węzłem wewnętrznym, to poprzednikiem $k$ jest największy klucz poddrzewa $\attribxx{x}{c_i}$.
W~przeciwnym przypadku szukaną wartością jest $\attribxx{x}{key_{i-1}}$ (gdy $i>1$) albo może znajdować się ona w~jednym z~przodków węzła $x$.
W~implementacji B-drzew nie posługujemy się wskaźnikami do ojców, dlatego zamiast poruszać się w~górę drzewa jak w~\proc{Tree-Predecessor} (podrozdział 12.2), będziemy symulować wyszukiwanie klucza $k$ przez schodzenie w~dół drzewa, począwszy od jego korzenia.
W~każdym kroku tego wyszukiwania zapamiętamy klucz mniejszy od szukanego, który posłużył do wskazania poddrzewa do następnego kroku.
Jeśli takiego klucza nie uda się wyznaczyć, będzie to oznaczać, że poprzednik $k$ nie istnieje.

Procedura przyjmuje na wejściu drzewo $T$, jego węzeł $x$ oraz indeks $i$ i~zwraca poprzednik klucza $k=\attribxx{x}{key_i}$ albo \const{nil}.
\begin{codebox}
\Procname{$\proc{B-Tree-Predecessor}(T,x,i)$}
\li \If $\attrib{x}{leaf}=\const{false}$
\li   \Then $\proc{Disk-Read}(\attribxx{x}{c_i})$
\li     \Return $\proc{B-Tree-Maximum}(\attribxx{x}{c_i})$
    \End
\li \If $i>1$
\li   \Then \Return \attribxx{x}{key_{i-1}}
    \End
\li $k\gets\attribxx{x}{key_i}$
\li $l\gets\const{nil}$
\li $y\gets\attrib{T}{root}$
\li \While $y\ne x$
\li   \Do $j\gets1$
\li   \While $j\le\attrib{y}{n}$ i~$k>\attribxx{y}{key_j}$
\li     \Do $j\gets j+1$
      \End
\li   \If $j>1$
\li     \Then $l\gets\attribxx{y}{key_{j-1}}$
      \End
\li   $\proc{Disk-Read}(\attribxx{y}{c_j})$
\li   $y\gets\attribxx{y}{c_j}$
    \End
\li \Return $l$
\end{codebox}

\exercise %18.2-4
\exercise %18.2-5
\exercise %18.2-6
Przeszukiwanie węzła B-drzewa wykorzystujące wyszukiwanie binarne zabiera czas $O(\lg t)$.
Całkowity czas działania tak zmodyfikowanej procedury \proc{B-Tree-Search} wynosi więc $O(\lg t\cdot\log_tn)=O(\lg n)$.
Skorzystaliśmy tu z~własności logarytmów (wzory 3.14).

\exercise %18.2-7
