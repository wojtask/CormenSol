\subchapter{Podstawowe operacje na B-drzewach}

\exercise %18.2-1
B-drzewa zostały zilustrowane na rys.\ \ref{fig:18.2-1}.
\begin{figure}[!ht]
	\centering \tikzset{
  arrow/.append style = {shorten <=-3pt},
  btree node/.style = {draw, light grayed, rectangle split, rectangle split horizontal, rectangle split parts=#1, rectangle split draw splits=false, rectangle split part align=base, inner sep=5pt}
}

\newcommand\bnodeone[2]{
  node[btree node=1]{#2} edge from parent[parent anchor=#1, child anchor=north]
}
\newcommand\bnodetwo[3]{
  node[btree node=2]{#2 \nodepart{two} #3} edge from parent[parent anchor=#1, child anchor=north]
}
\newcommand\bnodethree[4]{
  node[btree node=3]{#2 \nodepart{two} #3 \nodepart{three} #4} edge from parent[parent anchor=#1, child anchor=north]
}

\hfill
%
\begin{tikzpicture}
  \node[btree node=3] {$F$ \nodepart{two} $Q$ \nodepart{three} $S$};
  \node[subpicture label, anchor=west, left=2 mm of current bounding box] {(a)};
\end{tikzpicture}
%
\hfill
%
\begin{tikzpicture}
  \tikzstyle{level 1}=[sibling distance=15mm]
  \node[btree node=1] {$Q$} [arrow]
    child {\bnodethree{220}{$C$}{$F$}{$K$}}
    child {\bnodeone{-40}{$S$}};
  \node[subpicture label, anchor=north west] at (current bounding box.north west) {(b)};
\end{tikzpicture}
%
\hfill
%
\begin{tikzpicture}
  \tikzstyle{level 1}=[sibling distance=23mm]
  \node[btree node=2] {$F$ \nodepart{two} $Q$} [arrow]
    child {\bnodeone{200}{$C$}}
    child {\bnodethree{south}{$H$}{$K$}{$L$}}
    child {\bnodethree{-20}{$S$}{$T$}{$W$}};
  \node[subpicture label, anchor=north west] at (current bounding box.north west) {c)};
\end{tikzpicture}
%
\hfill{}
\bigskip

\hfill
%
\begin{tikzpicture}
  \tikzstyle{level 1}=[sibling distance=15mm]
  \node[btree node=3] {\nodepart{one} $F$ \nodepart{two} $Q$ \nodepart{three} $T$} [arrow]
    child {\bnodeone{195}{$C$}}
    child {\bnodethree{one split south}{$H$}{$K$}{$L$}}
    child {\bnodeone{two split south}{$S$}}
    child {\bnodetwo{-15}{$V$}{$W$}};
  \node[subpicture label, anchor=north west] at (current bounding box.north west) {(d)};
\end{tikzpicture}
%
\hfill
%
\begin{tikzpicture}
  \tikzstyle{level 1}=[sibling distance=40mm]
  \tikzstyle{level 2}=[sibling distance=15mm]
  \node[btree node=1] {$Q$} [arrow]
    child {\bnodeone{220}{$F$}
      child {\bnodeone{220}{$C$}}
      child {\bnodethree{-40}{$H$}{$K$}{$L$}}
    }
    child {\bnodeone{-40}{$T$}
      child {\bnodeone{220}{$S$}}
      child {\bnodetwo{-40}{$V$}{$W$}}
    };
  \node[subpicture label, anchor=north west] at (current bounding box.north west) {(e)};
\end{tikzpicture}
%
\hfill{}
\bigskip

\hfill
%
\begin{tikzpicture}
  \tikzstyle{level 1}=[sibling distance=40mm]
  \tikzstyle{level 2}=[sibling distance=15mm]
  \node[btree node=1] {$Q$} [arrow]
    child {\bnodetwo{210}{$F$}{$K$}
      child {\bnodeone{200}{$C$}}
      child {\bnodeone{south}{$H$}}
      child {\bnodethree{-20}{$L$}{$M$}{$N$}}
    }
    child {\bnodeone{-30}{$T$}
      child {\bnodetwo{220}{$R$}{$S$}}
      child {\bnodetwo{-40}{$V$}{$W$}}
    };
  \node[subpicture label, anchor=north west] at (current bounding box.north west) {(f)};
\end{tikzpicture}
%
\hfill
%
\begin{tikzpicture}
  \tikzstyle{level 1}=[sibling distance=30mm]
  \tikzstyle{level 2}=[sibling distance=15mm]
  \node[btree node=1] {$Q$} [arrow]
    child {\bnodethree{210}{$F$}{$K$}{$M$}
      child {\bnodeone{200}{$C$}}
      child {\bnodeone{two split south}{$H$}}
      child {\bnodeone{three split south}{$L$}}
      child {\bnodetwo{-20}{$N$}{$P$}}
    }
    child {\bnodeone{-30}{$T$}
      child {\bnodetwo{220}{$R$}{$S$}}
      child {\bnodetwo{-40}{$V$}{$W$}}
    };
  \node[subpicture label, anchor=north west] at (current bounding box.north west) {(g)};
\end{tikzpicture}
%
\bigskip

\begin{tikzpicture}
  \tikzstyle{level 1}=[sibling distance=40mm]
  \tikzstyle{level 2}=[sibling distance=15mm]
  \node[btree node=2] {$K$ \nodepart{two} $Q$} [arrow]
    child {\bnodeone{210}{$F$}
      child {\bnodethree{200}{$A$}{$B$}{$C$}}
      child {\bnodeone{-40}{$H$}}
    }
    child {\bnodeone{south}{$M$}
      child {\bnodeone{220}{$L$}}
      child {\bnodetwo{-40}{$N$}{$P$}}
    }
    child {\bnodeone{-30}{$T$}
      child {\bnodetwo{220}{$R$}{$S$}}
      child {\bnodethree{-40}{$V$}{$W$}{$X$}}
    };
  \node[subpicture label, anchor=north west] at (current bounding box.north west) {(h)};
\end{tikzpicture}
%
\bigskip

\begin{tikzpicture}
  \tikzstyle{level 1}=[sibling distance=40mm]
  \tikzstyle{level 2}=[sibling distance=15mm]
  \node[btree node=2] {$K$ \nodepart{two} $Q$} [arrow]
    child {\bnodeone{210}{$F$}
      child {\bnodethree{200}{$A$}{$B$}{$C$}}
      child {\bnodeone{-40}{$H$}}
    }
    child {\bnodeone{south}{$M$}
      child {\bnodeone{220}{$L$}}
      child {\bnodetwo{-40}{$N$}{$P$}}
    }
    child {\bnodetwo{-30}{$T$}{$W$}
      child {\bnodetwo{220}{$R$}{$S$}}
      child {\bnodeone{south}{$V$}}
      child {\bnodetwo{-40}{$X$}{$Y$}}
    };
  \node[subpicture label, anchor=north west] at (current bounding box.north west) {(i)};
\end{tikzpicture}
%
\bigskip

\begin{tikzpicture}
  \tikzstyle{level 1}=[sibling distance=40mm]
  \tikzstyle{level 2}=[sibling distance=15mm]
  \node[btree node=2] {$K$ \nodepart{two} $Q$} [arrow]
    child {\bnodetwo{south west}{$B$}{$F$}
      child {\bnodeone{south west}{$A$}}
      child {\bnodethree{two split south}{$C$}{$D$}{$E$}}
      child {\bnodeone{south east}{$H$}}
    }
    child {\bnodeone{two split south}{$M$}
      child {\bnodeone{south west}{$L$}}
      child {\bnodetwo{south east}{$N$}{$P$}}
    }
    child {\bnodetwo{south east}{$T$}{$W$}
      child {\bnodetwo{south west}{$R$}{$S$}}
      child {\bnodeone{two split south}{$V$}}
      child {\bnodethree{south east}{$X$}{$Y$}{$Z$}}
    };
  \node[subpicture label, anchor=north west] at (current bounding box.north west) {(j)};
\end{tikzpicture}

	\caption{Ilustracja wstawiania do B-drzewa.} \label{fig:18.2-1}
\end{figure}

\exercise %18.2-2
Jedyna sytuacja z~nadmiarowym wykonaniem operacji \proc{Disk-Read} może mieć miejsce, gdy klucz wstawiany jest do drzewa o~pełnym korzeniu.
W~procedurze \proc{B-Tree-Insert} w~trakcie rozbijania korzenia w~wywołaniu z~wiersza 8 tworzony jest nowy węzeł, który z~automatu trafia do pamięci wewnętrznej.
Następnie wołana jest procedura \proc{B-Tree-Insert-Nonfull} z~nowym korzeniem jako argumentem.
Nadmiarowe wczytanie do pamięci wewnętrznej może się wówczas wydarzyć w~linii 12 tejże procedury.

Z~kolei żaden zapis na dysk nie jest nadmiarowy.
Wywołanie \proc{Disk-Write} w~linii 8 procedury \proc{B-Tree-Insert-Nonfull}, jak również w~liniach 17 oraz 19 procedury \proc{B-Tree-Split-Child} poprzedzone są aktualizacjami pola $n$ zapisywanych węzłów, zaś w~linii 18 ostatniej procedury zapisywany jest nowo utworzony węzeł.

\exercise %18.2-3
Operacja znajdowania minimalnego klucza w~B-drzewie jest uogólnieniem analogicznej operacji dla drzew BST.
Gdy w~danej iteracji odwiedzany jest węzeł wewnętrzny $x$, to w~kolejnej odwiedzony zostanie węzeł $\attrib{x}{c_1}$.
Po dotarciu do liścia $x$ zwrócony zostanie najmniejszy jego klucz, czyli $\attrib{x}{key_1}$.
\begin{codebox}
\Procname{$\proc{B-Tree-Minimum}(x)$}
\li	\While $\attrib{x}{leaf}=\const{false}$
\li		\Do $\proc{Disk-Read}(\attribxx{x}{c_1})$
\li     $x\gets\attribxx{x}{c_1}$
		\End
\li	\Return $\attribxx{x}{key_1}$
\end{codebox}

Podamy też pseudokod operacji znajdowania maksymalnego klucza, gdyż potrzebna nam ona będzie w~implementacji wyznaczania poprzednika.
Operacja różni się od \proc{B-Tree-Minimum} tym, że porusza się po skrajnie prawych synach węzłów wewnętrznych i~zwraca największy klucz napotkanego liścia.
\begin{codebox}
\Procname{$\proc{B-Tree-Maximum}(x)$}
\li	\While $\attrib{x}{leaf}=\const{false}$
\li		\Do $\proc{Disk-Read}(\attribxx{x}{c_{\attrib{x}{n}+1}})$
\li     $x\gets\attribxx{x}{c_{\attrib{x}{n}+1}}$
		\End
\li	\Return $\attribxx{x}{key_{\attrib{x}{n}}}$
\end{codebox}

Zastanówmy się teraz, jak wyznaczyć poprzednik klucza $k=\attribxx{x}{key_i}$, gdzie $x$ jest węzłem B-drzewa $T$.
Jeżeli $x$ jest węzłem wewnętrznym, to poprzednikiem $k$ jest największy klucz poddrzewa $\attribxx{x}{c_i}$.
W~przeciwnym przypadku szukaną wartością jest $\attribxx{x}{key_{i-1}}$ (gdy $i>1$) albo może znajdować się ona w~jednym z~przodków węzła $x$.
W~implementacji B-drzew nie posługujemy się wskaźnikami do ojców, dlatego zamiast poruszać się w~górę drzewa jak w~\proc{Tree-Predecessor} (podrozdział 12.2), będziemy symulować wyszukiwanie klucza $k$ przez schodzenie w~dół drzewa, począwszy od korzenia.
W~każdym kroku tego wyszukiwania zapamiętamy klucz mniejszy od szukanego, który posłużył do wskazania poddrzewa do następnego kroku.
Jeśli takiego klucza nie uda się wyznaczyć, będzie to oznaczać, że poprzednik klucza $k$ nie istnieje.

Procedura przyjmuje na wejściu drzewo $T$, jego węzeł $x$ oraz indeks $i$ i~zwraca poprzednik $k'$ klucza $k=\attribxx{x}{key_i}$ albo \const{nil}.
\begin{codebox}
\Procname{$\proc{B-Tree-Predecessor}(T,x,i)$}
\li \If $\attrib{x}{leaf}=\const{false}$
\li   \Then $\proc{Disk-Read}(\attribxx{x}{c_i})$
\li     \Return $\proc{B-Tree-Maximum}(\attribxx{x}{c_i})$
    \End
\li \If $i>1$
\li   \Then \Return \attribxx{x}{key_{i-1}}
    \End
\li $k\gets\attribxx{x}{key_i}$
\li $k'\gets\const{nil}$
\li $y\gets\attrib{T}{root}$
\li \While $y\ne x$
\li   \Do $j\gets1$
\li   \While $j\le\attrib{y}{n}$ i~$k>\attribxx{y}{key_j}$
\li     \Do $j\gets j+1$
      \End
\li   \If $j>1$
\li     \Then $k'\gets\attribxx{y}{key_{j-1}}$
      \End
\li   $\proc{Disk-Read}(\attribxx{y}{c_j})$
\li   $y\gets\attribxx{y}{c_j}$
    \End
\li \Return $k'$
\end{codebox}

\exercise %18.2-4
\exercise %18.2-5
Jeśli w~liściach dopuścimy więcej niż $2t-1$ kluczy, to musimy zmodyfikować warunek rozstrzygający, czy węzeł B-drzewa jest pełny.
Teraz warunek ten będzie zależny nie tylko od wartości \attrib{x}{n}, ale też od flagi \attrib{x}{leaf}, przy czym dla węzłów wewnętrznych nie ulegnie on zmianie.
W~operacji wstawiania klucza do B-drzewa nową wersję warunku należy zastosować w~wierszu 2 procedury \proc{B-Tree-Insert} i~wierszu 13 procedury \proc{B-Tree-Insert-Nonfull}.
Zauważmy, że jeśli węzeł jest liściem, to w~trakcie operacji wstawiania nie stanie się on węzłem wewnętrznym, dlatego nie ma obawy, że liczba kluczy węzła zostanie przekroczona.

\exercise %18.2-6
Przeszukiwanie węzła B-drzewa wykorzystujące wyszukiwanie binarne zabiera czas $O(\lg t)$.
Całkowity czas działania tak zmodyfikowanej procedury \proc{B-Tree-Search} wynosi więc $O(\lg t\cdot\log_tn)=O(\lg n)$.
Skorzystaliśmy tu z~własności logarytmów (wzory 3.14).

\exercise %18.2-7
\note{W~rozwiązaniu przyjmujemy wartości z~treści oryginalnej:\/ $a=5$ milisekund i\/~$b=10$ mikrosekund.}

\noindent Założymy dla uproszczenia, że wykonanie pojedynczej instrukcji zajmuje procesorowi około 1 mikrosekundę.
Czas działania wyszukiwania w~B-drzewie zawierającym $n$ kluczy można wyrazić przez sumę czasu dostępu do stron dysku i~pozostałego czasu spędzonego na obliczeniach.
Zgodnie z~analizą z~Podręcznika, pierwsza z~tych wartości wynosi w~przybliżeniu $(1000a+bt)\log_tn$ mikrosekund, a~druga w~przybliżeniu $t\log_tn$ mikrosekund.
Ustalmy teraz $n$ i~potraktujmy tę sumę jako funkcję zmiennej $t$, gdzie $t\ge2$:
\[
	f_n(t) = (1000a+bt)\log_tn+t\log_tn = \frac{\ln n\cdot(1000a+(b+1)t)}{\ln t}.
\]

W~celu znalezienia minimum funkcji $f_n$ znajdujemy jej pochodną:
\[
	\frac{df_n}{dt}(t) = \frac{\ln n\cdot(b+1)\ln t-\frac{\ln n\cdot(1000a+(b+1)t)}{t}}{\ln^2t}.
\]
Pochodna zeruje się, gdy
\[
	\ln n\cdot(b+1)t\ln t = \ln n\cdot(1000a+(b+1)t),
\]
czyli, równoważnie,
\[
	t(\ln t-1) = \frac{1000a}{b+1}.
\]
Dla $t>e$ wyrażenie po lewej stronie znaku równości jest funkcją rosnącą (jako iloczyn funkcji rosnących i~dodatnich; patrz \refExercise{3.2-1}).
Nietrudno więc, szukając binarnie, wyznaczyć takie całkowite $t$, dla którego wyrażenie przyjmuje wartość po prawej stronie, czyli $5000/11$.
Rozwiązaniem jest $t=120$.
