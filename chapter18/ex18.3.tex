\subchapter{Usuwanie klucza z~B-drzewa}

\exercise %18.3-1
B-drzewa po usunięciu kolejnych kluczy zostały zilustrowane na rys.\ \ref{fig:18.3-1}.
\begin{figure}[!ht]
	\centering \tikzset{
  arrow/.append style = {shorten <=-3pt},
  btree node/.style = {draw, light grayed, rectangle split, rectangle split horizontal, rectangle split parts=#1, rectangle split draw splits=false, rectangle split part align=base, inner sep=5pt}
}

\newcommand\bnodeone[2]{
  node[btree node=1]{#2} edge from parent[parent anchor=#1, child anchor=north]
}
\newcommand\bnodetwo[3]{
  node[btree node=2]{#2 \nodepart{two} #3} edge from parent[parent anchor=#1, child anchor=north]
}
\newcommand\bnodethree[4]{
  node[btree node=3]{#2 \nodepart{two} #3 \nodepart{three} #4} edge from parent[parent anchor=#1, child anchor=north]
}
\newcommand\bnodefour[5]{
  node[btree node=4]{#2 \nodepart{two} #3 \nodepart{three} #4 \nodepart{four} #5} edge from parent[parent anchor=#1, child anchor=north]
}

\begin{tikzpicture}
  \tikzstyle{level 1}=[sibling distance=25mm]
  \node[btree node=4] {\nodepart{one} $L$ \nodepart{two} $P$ \nodepart{three} $T$ \nodepart{four} $X$} [arrow]
    child {\bnodefour{195}{$A$}{$E$}{$J$}{$K$}}
    child {\bnodetwo{220}{$N$}{$O$}}
    child {\bnodethree{two split south}{$Q$}{$R$}{$S$}}
    child {\bnodetwo{-40}{$U$}{$V$}}
    child {\bnodetwo{-15}{$Y$}{$Z$}};
  \node[subpicture label, anchor=north west] at (current bounding box.north west) {(a)};
\end{tikzpicture}
\bigskip

\begin{tikzpicture}
  \tikzstyle{level 1}=[sibling distance=25mm]
  \node[btree node=4] {\nodepart{one} $L$ \nodepart{two} $Q$ \nodepart{three} $T$ \nodepart{four} $X$} [arrow]
    child {\bnodefour{195}{$A$}{$E$}{$J$}{$K$}}
    child {\bnodetwo{220}{$N$}{$O$}}
    child {\bnodetwo{two split south}{$R$}{$S$}}
    child {\bnodetwo{-40}{$U$}{$V$}}
    child {\bnodetwo{-15}{$Y$}{$Z$}};
  \node[subpicture label, anchor=north west] at (current bounding box.north west) {(b)};
\end{tikzpicture}
\bigskip

\begin{tikzpicture}
  \tikzstyle{level 1}=[sibling distance=25mm]
  \node[btree node=3] {\nodepart{one} $L$ \nodepart{two} $Q$ \nodepart{three} $X$} [arrow]
    child {\bnodefour{200}{$A$}{$E$}{$J$}{$K$}}
    child {\bnodetwo{240}{$N$}{$O$}}
    child {\bnodefour{-60}{$R$}{$S$}{$T$}{$U$}}
    child {\bnodetwo{-20}{$Y$}{$Z$}};
  \node[subpicture label, anchor=north west] at (current bounding box.north west) {(c)};
\end{tikzpicture}

	\caption{Ilustracja usuwania z~B-drzewa.
    {\sffamily\bfseries(a)} Drzewo po usunięciu $C$: przypadek 3b.
    {\sffamily\bfseries(b)} Drzewo po usunięciu $P$: przypadek 2b.
    {\sffamily\bfseries(c)} Drzewo po usunięciu $V$: przypadek 3a.} \label{fig:18.3-1}
\end{figure}

\exercise %18.3-2
Z~powodu skomplikowania procedury, dla czytelności podzielimy jej pseudokod na kilka mniejszych procedur pomocniczych.
Zakładamy, że węzły B-drzewa stanowiące parametry wejściowe tych procedur zostały już wczytane do pamięci wewnętrznej.

Zacznijmy od zapisania operacji \proc{B-Tree-Left-Shift}, przyjmującej węzeł $x$ o~co najmniej $t$ kluczach oraz indeks $i$ taki, że $1\le i\le\attrib{x}{n}$.
Operacja ta, zgodnie ze swoją nazwą, przesuwa w~lewo klucze (oraz wskaźniki do synów, jeśli $x$ jest węzłem wewnętrznym), począwszy od pozycji $i+1$.
W~wyniku działania tej operacji wartość \attrib{x}{n} zostaje pomniejszona o~1, a~klucz będący początkowo na pozycji $i$ zostaje nadpisany.
Jeśli $x$ jest węzłem wewnętrznym, to nadpisywany jest też wskaźnik do syna, początkowo znajdujący się w~\attribxx{x}{c_i}, dlatego tuż przed wywołaniem tej operacji należy pamiętać o~skopiowaniu tego wskaźnika, aby nie utracić dostępu do syna.
\begin{codebox}
    \Procname{$\proc{B-Tree-Left-Shift}(x,i)$}
    \li \For $j\gets i$ \To $\attrib{x}{n}-1$
    \li     \Do $\attribxx{x}{key_j}\gets\attribxx{x}{key_{j+1}}$
            \End
    \li \If $\attrib{x}{leaf}=\const{false}$
    \li     \Then \For $j\gets i$ \To $\attrib{x}{n}$
    \li         \Do $\attribxx{x}{c_j}\gets\attribxx{x}{c_{j+1}}$
                \End
            \End
    \li $\attrib{x}{n}\gets\attrib{x}{n}-1$
    \li $\proc{Disk-Write}(x)$
\end{codebox}
Symetryczna do powyższej operacja \proc{B-Tree-Right-Shift} przesuwa w~niepełnym węźle $x$ klucze (i~ewentualnie wskaźniki do synów) w~prawo, od pozycji $i$.
Celem działania tej operacji jest przygotowanie miejsca na pozycji $i$ dla jakiegokolwiek nowego klucza, który znajdzie się w~$x$.
\begin{codebox}
    \Procname{$\proc{B-Tree-Right-Shift}(x,i)$}
    \li $\attrib{x}{n}\gets\attrib{x}{n}+1$
    \li \For $j\gets\attrib{x}{n}-1$ \Downto $i$
    \li     \Do $\attribxx{x}{key_{j+1}}\gets\attribxx{x}{key_j}$
            \End
    \li \If $\attrib{x}{leaf}=\const{false}$
    \li     \Then \For $j\gets\attrib{x}{n}$ \Downto $i$
    \li         \Do $\attribxx{x}{c_{j+1}}\gets\attribxx{x}{c_j}$
                \End
            \End
    \li $\proc{Disk-Write}(x)$
\end{codebox}

Kolejne dwie operacje zostaną użyte w~implementacji przypadku 3a z~opisu procedury w~Podręczniku.
Wejściem każdej z~nich jest węzeł wewnętrzny $x$, indeks $i$ oraz węzły $y=\attribxx{x}{c_i}$ i~$z=\attribxx{x}{c_{i+1}}$.
Pierwsza z~operacji zakłada, że węzeł $y$ zawiera $t-1$ kluczy, a~$z$ -- co najmniej $t$ kluczy -- i~wykonuje ,,rotację'' kluczy w~lewo, w~wyniku której do $y$ na pozycję $t$ trafia klucz \attribxx{x}{key_i}, a~w~jego miejsce przeniesiony zostaje \attribxx{z}{key_1}.
Dodatkowo w~przypadku, gdy $y$ i~$z$ są węzłami wewnętrznymi, w~rotacji uczestniczą także odpowiednie wskaźniki do synów.
\begin{codebox}
    \Procname{$\proc{B-Tree-Left-Rotate}(x,i,y,z)$}
    \li $\attrib{y}{n}\gets\attrib{y}{n}+1$
    \li $\attribxx{y}{key_{\attrib{y}{n}}}\gets\attribxx{x}{key_i}$
    \li $\attribxx{x}{key_i}\gets\attribxx{z}{key_1}$
    \li \If $\attrib{y}{leaf}=\const{false}$
    \li     \Then $\attribxx{y}{c_{\attrib{y}{n}+1}}\gets\attribxx{z}{c_1}$
            \End
    \li $\proc{B-Tree-Left-Shift}(z,1)$
    \li $\proc{Disk-Write}(x)$
\end{codebox}
Symetryczna ,,rotacja w~prawo'' przyjmuje, że to węzeł $y$ ma co najmniej $t$ kluczy, a~$z$ ma ich dokładnie $t-1$.
Klucze przesuwane są tutaj od $y$ przez $x$ do $z$.
\begin{codebox}
    \Procname{$\proc{B-Tree-Right-Rotate}(x,i,y,z)$}
    \li $\proc{B-Tree-Right-Shift}(z,1)$
    \li $\attribxx{z}{key_1}\gets\attribxx{x}{key_i}$
    \li $\attribxx{x}{key_i}\gets\attribxx{y}{key_{\attrib{y}{n}}}$
    \li \If $\attrib{z}{leaf}=\const{false}$
    \li     \Then $\attribxx{z}{c_1}\gets\attribxx{y}{c_{\attrib{y}{n}+1}}$
            \End
    \li $\attrib{y}{n}\gets\attrib{y}{n}-1$
    \li $\proc{Disk-Write}(x)$
    \li $\proc{Disk-Write}(y)$
\end{codebox}
Zauważmy, że operacje rotacji modyfikują niektóre węzły, ale nie zapisują ich na dysku.
Jak przekonamy się później, zapis zostanie ostatecznie wykonany w~procedurach wywołujących te operacje, dzięki czemu nie mają miejsca nadmiarowe operacje \proc{Disk-Write}.

Kolejna procedura pomocnicza stanowi odwrotność procedury \proc{B-Tree-Split-Child} z~podrozdziału 18.2 i~posłuży do zaimplementowania podkroków 2c i~3b.
Dla parametrów wejściowych -- węzła wewnętrznego $x$, indeksu $i$ oraz węzłów $y=\attribxx{x}{c_i}$, $z=\attribxx{x}{c_{i+1}}$ mających po $t-1$ kluczy -- omawiana procedura scala węzły $y$ i~$z$ poprzez przeniesienie do $y$ klucza \attribxx{x}{key_i} oraz wszystkich kluczy (i~ewentualnie wskaźników do synów) z~węzła $z$.
Pamięć przydzielona dla węzła $z$ zostaje następnie zwolniona przeciwstawną do \proc{Allocate-Node} operacją \proc{Free-Node}.
\begin{codebox}
    \Procname{$\proc{B-Tree-Merge-Children}(x,i,y,z)$}
    \li $\attrib{y}{n}\gets 2t-1$
    \li $\attribxx{y}{key_t}\gets\attribxx{x}{key_i}$
    \li \For $j\gets1$ \To $t-1$
    \li     \Do $\attribxx{y}{key_{j+t}}\gets\attribxx{z}{key_j}$
            \End
    \li \If $\attrib{y}{leaf}=\const{false}$
    \li     \Then \For $j\gets 1$ \To $t$
    \li         \Do $\attribxx{y}{c_{j+t}}\gets\attribxx{z}{c_j}$
                \End
            \End
    \li $\proc{B-Tree-Left-Shift}(x,i)$
    \li $\attribxx{x}{c_i}\gets y$
    \li $\proc{Disk-Write}(x)$
    \li $\proc{Disk-Write}(y)$
    \li $\proc{Free-Node}(z)$
\end{codebox}

Podamy teraz pseudokody procedur usuwania najmniejszego oraz największego klucza w~danym poddrzewie.
Stanowią one szczególny przypadek głównego algorytmu, ponieważ podczas wyszukiwania takich kluczy nie zachodzi nigdy przypadek 2, a~zejście rekurencyjne jest wykonywane zawsze dla najbardziej skrajnego syna.
Jednak w~odróżnieniu od ogólnej wersji procedury, te nie przyjmują na wejściu klucza do usunięcia, tylko zwracają usunięty klucz jako wynik.
\begin{codebox}
    \Procname{$\proc{B-Tree-Delete-Minimum}(x)$}
    \li \If $\attrib{x}{leaf}$
    \li     \Then $\id{min}\gets\attribxx{x}{key_1}$
    \li         $\proc{B-Tree-Left-Shift}(x,1)$
    \li         \Return \id{min}
            \End
    \li $y\gets\attribxx{x}{c_1}$
    \li $\proc{Disk-Read}(y)$
    \li \If $\attrib{y}{n}=t-1$
    \li     \Then $z\gets\attribxx{x}{c_2}$
    \li         $\proc{Disk-Read}(z)$
    \li         \If $\attrib{z}{n}\ge t$
    \li             \Then $\proc{B-Tree-Left-Rotate}(x,1,y,z)$
    \li             \Else $\proc{B-Tree-Merge-Children}(x,1,y,z)$
                    \End
            \End
    \li \Return $\proc{B-Tree-Delete-Minimum}(y)$
\end{codebox}
\begin{codebox}
    \Procname{$\proc{B-Tree-Delete-Maximum}(x)$}
    \li \If $\attrib{x}{leaf}$
    \li     \Then $\id{max}\gets\attribxx{x}{key_{\attrib{x}{n}}}$
    \li         $\attrib{x}{n}\gets\attrib{x}{n}-1$
    \li         $\proc{Disk-Write}(x)$
    \li         \Return \id{max}
            \End
    \li $y\gets\attribxx{x}{c_{\attrib{x}{n}+1}}$
    \li $\proc{Disk-Read}(y)$
    \li \If $\attrib{y}{n}=t-1$
    \li     \Then $z\gets\attribxx{x}{c_{\attrib{x}{n}}}$
    \li         $\proc{Disk-Read}(z)$
    \li         \If $\attrib{z}{n}\ge t$
    \li             \Then $\proc{B-Tree-Right-Rotate}(x,\attrib{x}{n},z,y)$
    \li             \Else $\proc{B-Tree-Merge-Children}(x,\attrib{x}{n},z,y)$
    \li                 $y\gets z$
                    \End
            \End
    \li \Return $\proc{B-Tree-Delete-Maximum}(y)$
\end{codebox}

Jesteśmy gotowi do zapisania rekurencyjnej części algorytmu opisanego w~Podręczniku.
\begin{codebox}
    \Procname{$\proc{B-Tree-Safe-Delete}(x,k)$}
    \li $i\gets1$ \label{li:b-tree-safe-delete-search-begin}
    \li \While $i\le\attrib{x}{n}$ i~$k>\attribxx{x}{key_i}$
    \li     \Do $i\gets i+1$
            \End \label{li:b-tree-safe-delete-search-end}
    \li \If $i\le\attrib{x}{n}$ i~$k=\attribxx{x}{key_i}$
    \li     \Then \If $\attrib{x}{leaf}$
    \li         \Then $\proc{B-Tree-Left-Shift}(x,i)$ \label{li:b-tree-safe-delete-remove-from-leaf}
    \li         \Else $y\gets\attribxx{x}{c_i}$ \label{li:b-tree-safe-delete-remove-predecessor-begin}
    \li             $\proc{Disk-Read}(y)$
    \li             \If $\attrib{y}{n}\ge t$
    \li                 \Then $\attribxx{x}{key_i}\gets\proc{B-Tree-Delete-Maximum}(y)$
    \li                     $\proc{Disk-Write}(x)$ \label{li:b-tree-safe-delete-remove-predecessor-end}
    \li                 \Else $z\gets\attribxx{x}{c_{i+1}}$ \label{li:b-tree-safe-delete-remove-successor-begin}
    \li                     $\proc{Disk-Read}(z)$
    \li                     \If $\attrib{z}{n}\ge t$
    \li                         \Then $\attribxx{x}{key_i}\gets\proc{B-Tree-Delete-Minimum}(z)$
    \li                             $\proc{Disk-Write}(x)$ \label{li:b-tree-safe-delete-remove-successor-end}
    \li                         \Else $\proc{B-Tree-Merge-Children}(x,i,y,z)$ \label{li:b-tree-safe-delete-remove-merge-children-1}
    \li                             $\proc{B-Tree-Safe-Delete}(y,k)$ \label{li:b-tree-safe-delete-recursive-call-1}
                                \End
                        \End
                \End
    \li     \Else $y\gets\attribxx{x}{c_i}$
    \li         $\proc{Disk-Read}(y)$
    \li         \If $\attrib{y}{n}=t-1$
    \li             \Then $z_L\gets z_R\gets\const{nil}$
    \li                 \If $i\ge2$
    \li                     \Then $z_L\gets\attribxx{x}{c_{i-1}}$
    \li                         $\proc{Disk-Read}(z_L)$
                            \End
    \li                 \If $i\le\attribxx{x}{n}$
    \li                     \Then $z_R\gets\attribxx{x}{c_{i+1}}$
    \li                         $\proc{Disk-Read}(z_R)$
                            \End
    \li                 \If $z_L\ne\const{nil}$ i~$\attrib{z_L}{n}\ge t$ \label{li:b-tree-safe-delete-rotations-begin}
    \li                     \Then $\proc{B-Tree-Right-Rotate}(x,i-1,z_L,y)$
    \li                     \ElseIf $z_R\ne\const{nil}$ i~$\attrib{z_R}{n}\ge t$
    \li                         \Then $\proc{B-Tree-Left-Rotate}(x,i,y,z_R)$ \label{li:b-tree-safe-delete-rotations-end}
    \li                     \ElseIf $z_L\ne\const{nil}$ \label{li:b-tree-safe-delete-remove-mergers-begin}
    \li                         \Then $\proc{B-Tree-Merge-Children}(x,i-1,z_L,y)$
    \li                             $y\gets z_L$
    \li                     \ElseNoIf $\proc{B-Tree-Merge-Children}(x,i,y,z_R)$
                            \End \label{li:b-tree-safe-delete-remove-mergers-end}
                    \End \label{li:b-tree-safe-delete-key-brothers-end}
    \li         $\proc{B-Tree-Safe-Delete}(y,k)$ \label{li:b-tree-safe-delete-recursive-call-2}
            \End
\end{codebox}
Procedura rozpoczyna od wyszukania w~węźle $x$ pozycji $i$ klucza $k$ (linie \ref{li:b-tree-safe-delete-search-begin}\nbendash\ref{li:b-tree-safe-delete-search-end}).
Jeśli klucz został odnaleziony, czyli $1\le i\le\attribxx{x}{n}$, a~$x$ jest liściem, to w~linii \ref{li:b-tree-safe-delete-remove-from-leaf} realizowany jest krok 1.
W~wierszach \ref{li:b-tree-safe-delete-remove-predecessor-begin}\nbendash\ref{li:b-tree-safe-delete-remove-predecessor-end} algorytm znajduje i~usuwa rekurencyjnie poprzednik klucza $k$ w~poddrzewie o~korzeniu w~$y=\attribxx{x}{c_i}$, czyli jednocześnie maksymalny klucz tego poddrzewa (podkrok 2a).
Analogicznie, w~wierszach \ref{li:b-tree-safe-delete-remove-successor-begin}\nbendash\ref{li:b-tree-safe-delete-remove-successor-end} w~poddrzewie o~korzeniu w~$z=\attribxx{x}{c_{i+1}}$ jest znajdowany i~usuwany następnik klucza $k$ (podkrok 2b).
Implementacja przypadku 2c polega na scaleniu węzłów $y$ i~$z$ dzięki wywołaniu procedury pomocniczej w~linii \ref{li:b-tree-safe-delete-remove-merge-children-1} oraz wywołaniu rekurencyjnym dla scalonego węzła $y$ w~linii \ref{li:b-tree-safe-delete-recursive-call-1}.

Pozostała część algorytmu stanowi implementację kroku 3.
Sąsiedni bracia $z_L$ i~$z_R$ węzła $y=\attribxx{x}{c_i}$ są wczytywani do pamięci wewnętrznej, przy czym należy pamiętać, że co najwyżej jeden z~tych braci może nie istnieć.
W~zależności od liczby kluczy znajdujących się w~$y$, $z_L$ i~$z_R$, algorytm wykonuje odpowiednie rotacje (podkrok 3a; wiersze \ref{li:b-tree-safe-delete-rotations-begin}\nbendash\ref{li:b-tree-safe-delete-rotations-end}) albo scala odpowiednie węzły (podkrok 3b; wiersze \ref{li:b-tree-safe-delete-remove-mergers-begin}\nbendash\ref{li:b-tree-safe-delete-remove-mergers-end}).
W~przypadku, gdy istnieje lewy brat węzła $y$, zostaje on scalony z~$y$, po czym wskaźnik $y$ ustawiany jest na wynikowy węzeł, który to zostanie następnie przekazany do wywołania rekurencyjnego.
W~razie braku lewego brata, scalany jest $y$ z~jego prawym bratem.
Wreszcie, w~linii \ref{li:b-tree-safe-delete-recursive-call-2} algorytm zostaje wywołany rekurencyjne dla węzła $y$.

Zauważmy, że opis procedury w~Podręczniku opiera się na założeniu, że klucz $k$ jest w~B-drzewie.
Jeśli dodatkowo chcielibyśmy obsłużyć sytuację jego braku, to w~przypadku 3 algorytm powinien rozpoczynać od sprawdzenia, czy $x$ jest liściem i~w~takim wypadku natychmiast zakończyć działanie.

W~trakcie działania algorytmu dostęp do pól węzła B-drzewa poprzedzony jest odczytem tego węzła z~dysku, zaś po zakończeniu działania algorytmu każdy zmodyfikowany węzeł jest zapisany na dysk.
Zapis najczęściej odbywa się w~tym samym wywołaniu, w~którym zaszła modyfikacja, ewentualnie w~następnym wywołaniu rekurencyjnym procedur \proc{B-Tree-Delete-Minimum}, \proc{B-Tree-Delete-Maximum}, \proc{B-Tree-Safe-Delete} po wykonaniu operacji rotacji.

Inicjujące wywołanie algorytmu wykonywane jest przez właściwą procedurę \proc{B-Tree-Delete}, zamieszczoną poniżej.
W~trakcie działania algorytmu korzeń drzewa mógł zostać pozbawiony swojego ostatniego klucza, dlatego wówczas procedura \proc{B-Tree-Delete} ustawia nowy korzeń i~zwalnia pamięć przydzieloną dla starego korzenia.
\begin{codebox}
    \Procname{$\proc{B-Tree-Delete}(T,k)$}
    \li $r\gets\attrib{T}{root}$
    \li $\proc{B-Tree-Safe-Delete}(r,k)$
    \li \If $\attrib{r}{n}=0$ i~$\attrib{r}{leaf}=\const{false}$
    \li     \Then $\attrib{T}{root}\gets\attribxx{r}{c_1}$
    \li         $\proc{Free-Node}(r)$
            \End
\end{codebox}
