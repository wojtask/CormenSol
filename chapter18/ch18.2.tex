\subchapter{Podstawowe operacje na B-drzewach}

\exercise %18.2-1
\exercise %18.2-2
\exercise %18.2-3
\exercise %18.2-4
\exercise %18.2-5
\exercise %18.2-6
Przeszukiwanie węzła B-drzewa wykorzystujące wyszukiwanie binarne zabiera czas $O(\lg t)$.
Całkowity czas działania tak zmodyfikowanej procedury \proc{B-Tree-Search} wynosi więc $O(\lg t\cdot\log_tn)=O(\lg n)$.
Skorzystaliśmy tu z~własności logartytmów (wzory 3.14).

\exercise %18.2-7
