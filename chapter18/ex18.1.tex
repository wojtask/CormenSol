\subchapter{Definicja B-drzewa}

\exercise %18.1-1
Gdyby $t=1$, to każdy węzeł wewnętrzny takiego drzewa miałby co najwyżej jednego syna, czyli drzewo to byłoby zdegenerowane do listy.
Operacje na zbiorach dynamicznych zaimplementowanych na takim drzewie byłyby proporcjonalne do liczby liści tego drzewa.

\exercise %18.1-2
W~drzewie z~rys.\ 18.1 korzeń ma 1 klucz, a~każdy inny węzeł wewnętrzny ma 2 lub 3 klucze.
Taka sytuacja możliwa jest tylko dla $t=2$ lub $t=3$.

\exercise %18.1-3
B-drzewa zostały zilustrowane na rys.\ \ref{fig:18.1-3}.
\begin{figure}[!ht]
	\centering \begin{tikzpicture}[
	level/.style = {level distance=15mm, sibling distance=25mm/2^#1, child anchor=north, edge from parent/.style = {arrow, ->, shorten <=-3pt}}
]

\def\valueskip{\hspace{1.5ex}}

\node[outer] (pic a) {
\begin{tikzpicture}[
	every node/.style = {draw, semithick, light grayed, minimum size=6mm, inner sep=4pt, text height=1.5ex},
	anchor = center
]
\node {2 \valueskip 4}
	child {node {1}}
	child {node {3}}
	child {node {5}};
\end{tikzpicture}
};

\node[outer, right=of pic a] (pic b) {
\begin{tikzpicture}[
	every node/.style = {draw, semithick, light grayed, minimum size=6mm, inner sep=4pt, text height=1.5ex},
	anchor = center
]
\node {2}
	child {node {1}}
	child {node {3 \valueskip 4 \valueskip 5}};
\end{tikzpicture}
};

\node[outer, right=of pic b] (pic c) {
\begin{tikzpicture}[
	every node/.style = {draw, semithick, light grayed, minimum size=6mm, inner sep=4pt, text height=1.5ex},
	anchor = center
]
\node {3}
	child {node {1 \valueskip 2}}
	child {node {4 \valueskip 5}};
\end{tikzpicture}
};

\node[outer, right=of pic c] (pic d) {
\begin{tikzpicture}[
	every node/.style = {draw, semithick, light grayed, minimum size=6mm, inner sep=4pt, text height=1.5ex},
	anchor = center
]
\node {4}
	child {node {1 \valueskip 2 \valueskip 3}}
	child {node {5}};
\end{tikzpicture}
};

\foreach \x in {a,b,c,d} {
	\node[subpicture label, below=2mm of pic \x] {(\x)};
}

\end{tikzpicture}

	\caption{Wszystkie poprawne B-drzewa o~minimalnym stopniu 2 reprezentujące zbiór $\{1,2,3,4,5\}$.} \label{fig:18.1-3}
\end{figure}

\exercise %18.1-4
Rozważmy B-drzewo o~minimalnym stopniu $t$, w~którym liczba kluczy w~każdym węźle jest maksymalna, czyli wynosi $2t-1$.
Jest to więc pełne drzewo \singledash{$(2t)$}{arne}, w~którym na głębokości $i$ znajduje się łącznie $(2t)^i$ węzłów.
W~dowolnym B-drzewie o~wysokości $h$ sumaryczna liczba węzłów spełnia zatem nierówność
\[
	n \le \sum_{i=0}^h(2t)^i = \frac{(2t)^{h+1}-1}{2t-1}.
\]
Każdy węzeł posiada co najwyżej $2t-1$ kluczy, więc B-drzewo może pomieścić co najwyżej $(2t)^{h+1}-1$ kluczy.

\exercise %18.1-5
W~opisanym procesie klucz każdego czerwonego węzła zostanie dodany do ojca tego węzła.
W~powstałym drzewie każdy węzeł będzie zatem mieć 1, 2 albo 3 klucze, w~zależności od początkowej liczby czerwonych synów.
Z~własności drzew czerwono-czarnych wynika też, że wysokością ostatecznego drzewa będzie czarna wysokość początkowego drzewa czerwono-czarnego i~każdy liść będzie znajdował się na tej samej głębokości.
Powstałą strukturę można więc traktować jak B-drzewo o~minimalnym stopniu $t=2$.
